\documentclass[handout]{ximera}

%% You can put user macros here
%% However, you cannot make new environments



\newcommand{\ffrac}[2]{\frac{\text{\footnotesize $#1$}}{\text{\footnotesize $#2$}}}
\newcommand{\vasymptote}[2][]{
    \draw [densely dashed,#1] ({rel axis cs:0,0} -| {axis cs:#2,0}) -- ({rel axis cs:0,1} -| {axis cs:#2,0});
}


%\usepackage{tcolorbox} %%Needed for Derivative Definition supposedly and product rule, natural exp log, quotient rule, inverse trig, rates of change


% \graphicspath{{./}{firstExample/}}
% \usepackage{forest}
\usepackage{amsmath}
\usepackage{amssymb}
\usepackage{array}
\usepackage[makeroom]{cancel} %% for strike outs
\usepackage{pgffor} %% required for integral for loops
\usepackage{tikz}
\usepackage{tikz-cd}
\usepackage{tkz-euclide}
\usetikzlibrary{shapes.multipart}


% \usetkzobj{all}
\tikzstyle geometryDiagrams=[ultra thick,color=blue!50!black]


\usetikzlibrary{arrows}
\tikzset{>=stealth,commutative diagrams/.cd,
  arrow style=tikz,diagrams={>=stealth}} %% cool arrow head
\tikzset{shorten <>/.style={ shorten >=#1, shorten <=#1 } } %% allows shorter vectors

\usetikzlibrary{backgrounds} %% for boxes around graphs
\usetikzlibrary{shapes,positioning}  %% Clouds and stars
\usetikzlibrary{matrix} %% for matrix
\usepgfplotslibrary{polar} %% for polar plots
\usepgfplotslibrary{fillbetween} %% to shade area between curves in TikZ



%\usepackage[width=4.375in, height=7.0in, top=1.0in, papersize={5.5in,8.5in}]{geometry}
%\usepackage[pdftex]{graphicx}
%\usepackage{tipa}
%\usepackage{txfonts}
%\usepackage{textcomp}
%\usepackage{amsthm}
%\usepackage{xy}
%\usepackage{fancyhdr}
%\usepackage{xcolor}
%\usepackage{mathtools} %% for pretty underbrace % Breaks Ximera
%\usepackage{multicol}



\newcommand{\RR}{\mathbb R}
\newcommand{\R}{\mathbb R}
\newcommand{\C}{\mathbb C}
\newcommand{\N}{\mathbb N}
\newcommand{\Z}{\mathbb Z}
\newcommand{\dis}{\displaystyle}
%\renewcommand{\d}{\,d\!}
\renewcommand{\d}{\mathop{}\!d}
\newcommand{\dd}[2][]{\frac{\d #1}{\d #2}}
\newcommand{\pp}[2][]{\frac{\partial #1}{\partial #2}}
\renewcommand{\l}{\ell}
\newcommand{\ddx}{\frac{d}{\d x}}
\newcommand{\ppx}{\frac{\partial}{\partial x}}
\newcommand{\ppy}{\frac{\partial}{\partial y}}

\newcommand{\zeroOverZero}{\ensuremath{\boldsymbol{\tfrac{0}{0}}}}
\newcommand{\inftyOverInfty}{\ensuremath{\boldsymbol{\tfrac{\infty}{\infty}}}}
\newcommand{\zeroOverInfty}{\ensuremath{\boldsymbol{\tfrac{0}{\infty}}}}
\newcommand{\zeroTimesInfty}{\ensuremath{\small\boldsymbol{0\cdot \infty}}}
\newcommand{\inftyMinusInfty}{\ensuremath{\small\boldsymbol{\infty - \infty}}}
\newcommand{\oneToInfty}{\ensuremath{\boldsymbol{1^\infty}}}
\newcommand{\zeroToZero}{\ensuremath{\boldsymbol{0^0}}}
\newcommand{\inftyToZero}{\ensuremath{\boldsymbol{\infty^0}}}


\newcommand{\numOverZero}{\ensuremath{\boldsymbol{\tfrac{\#}{0}}}}
\newcommand{\dfn}{\textbf}
%\newcommand{\unit}{\,\mathrm}
\newcommand{\unit}{\mathop{}\!\mathrm}
%\newcommand{\eval}[1]{\bigg[ #1 \bigg]}
\newcommand{\eval}[1]{ #1 \bigg|}
\newcommand{\seq}[1]{\left( #1 \right)}
\renewcommand{\epsilon}{\varepsilon}
\renewcommand{\iff}{\Leftrightarrow}

\DeclareMathOperator{\arccot}{arccot}
\DeclareMathOperator{\arcsec}{arcsec}
\DeclareMathOperator{\arccsc}{arccsc}
\DeclareMathOperator{\si}{Si}
\DeclareMathOperator{\proj}{proj}
\DeclareMathOperator{\scal}{scal}
\DeclareMathOperator{\cis}{cis}
\DeclareMathOperator{\Arg}{Arg}
%\DeclareMathOperator{\arg}{arg}
\DeclareMathOperator{\Rep}{Re}
\DeclareMathOperator{\Imp}{Im}
\DeclareMathOperator{\sech}{sech}
\DeclareMathOperator{\csch}{csch}
\DeclareMathOperator{\Log}{Log}

\newcommand{\tightoverset}[2]{% for arrow vec
  \mathop{#2}\limits^{\vbox to -.5ex{\kern-0.75ex\hbox{$#1$}\vss}}}
\newcommand{\arrowvec}{\overrightarrow}
\renewcommand{\vec}{\mathbf}
\newcommand{\veci}{{\boldsymbol{\hat{\imath}}}}
\newcommand{\vecj}{{\boldsymbol{\hat{\jmath}}}}
\newcommand{\veck}{{\boldsymbol{\hat{k}}}}
\newcommand{\vecl}{\boldsymbol{\l}}
\newcommand{\utan}{\vec{\hat{t}}}
\newcommand{\unormal}{\vec{\hat{n}}}
\newcommand{\ubinormal}{\vec{\hat{b}}}

\newcommand{\dotp}{\bullet}
\newcommand{\cross}{\boldsymbol\times}
\newcommand{\grad}{\boldsymbol\nabla}
\newcommand{\divergence}{\grad\dotp}
\newcommand{\curl}{\grad\cross}
%% Simple horiz vectors
\renewcommand{\vector}[1]{\left\langle #1\right\rangle}


\outcome{Find limits using the Squeeze Theorem}

\title{1.7 Squeeze Theorem}


\begin{document}

\begin{abstract}
Find limits using the Squeeze Theorem.
\end{abstract}

\maketitle

\section{Squeeze Theorem}
 

In this section we find limits using the Squeeze Theorem.

\begin{theorem}[The Squeeze Theorem] 
Suppose that the compound inequality
\[
g_1(x) \leq f(x) \leq g_2(x)
\]
holds for all values of $x$ in some open interval about $x=a$, except possibly for $a$ itself.
If
\[
\lim_{x\to a} g_1(x) = L \quad \text{and} \quad \lim_{x\to a} g_2(x) = L,
\]
then we can conclude that
\[
\lim_{x\to a} f(x) = L
\]
as well.

\end{theorem}


\begin{remark}
If the limits
\[
\lim_{x\to a} g_1(x)  \quad \text{and} \quad \lim_{x\to a} g_2(x)
\]
are different (or DNE), then we can make no conclusion about the limit
\[
\lim_{x\to a} f(x).
\]
\end{remark}

\begin{example}[example 1]
Suppose 
\[
 2x-1 \leq f(x) \leq x^2
\]
for all $x$ except $x=1$.
Find
\[
\lim_{x\to 1} f(x).
\]

Since
\[
\lim_{x\to 1} (2x-1) = 1
\]
and
\[
\lim_{x\to 1} x^2 = 1
\]
we can use the Squeeze Theorem to conclude that
\[
\lim_{x\to 1} f(x) = 1
\]
as well.

\end{example}

\begin{problem}(problem 1)
Suppose that 
\[
 5-x^2\leq f(x) \leq 9-4x
\]
for all $x$ except $x=2$.
Find
\[
\lim_{x\to 2} f(x).
\]
First, we find the limits of the bounds:
\[
\lim_{x \to 2} 5 - x^2 = \answer{1} \quad \text{and} \quad  \lim_{x \to 2} 9-4x = \answer{1}.
\]
Since these answers are the same, the Squeeze Theorem allows us to conclude that 
\[
\lim_{x \to 2} f(x) = \answer{1} \quad \text{as well}.
\]

\end{problem}

\begin{example}[example 2]
Find
\[
\lim_{x\to 0} x^2 \sin\Big(\frac{1}{x}\Big).
\]
Since $\frac{1}{0}$ is undefined, plugging in $x=0$
does not give a definitive answer.
Using the fact that 
\[
-1 \leq \sin \theta \leq 1
\]
for all values of $\theta$, we can create a compound inequality for
the function
\[
f(x) = x^2 \sin \frac{1}{x} 
\]
and find the limit using the Squeeze Theorem.
To begin, note that
\[
-1 \leq \sin \frac{1}{x}  \leq 1
\]
for all values of $x$ except $x=0$.
Multiplying this compound inequality by the non-negative quantity, $x^2$,
we have
\[
-x^2 \leq x^2\sin \frac{1}{x}  \leq x^2
\]
for all values of $x$ except $x=0$.
Next, note that
\[
\lim_{x\to 0} -x^2 = 0
\]
and
\[
\lim_{x\to 0} x^2 = 0.
\]
Finally, by the Squeeze Theorem, we can conclude that
\[
\lim_{x\to 0} x^2\sin \frac{1}{x}  = 0
\]
as well. The graph below also shows that the limit is zero.
Zoom in on the origin to get the full effect.

\[
\graph{x^2, -x^2, x^2 \sin(1/x)}
\]

\end{example}

\begin{problem}(problem 2a)
Find
\[
\lim_{x \to 0} x^4 \cos \frac{2}{x}
\]
using the Squeeze Theorem.\\
First, we need to find bounds. Since $-1 \leq \cos \theta \leq 1$ for all $\theta$, 
\[
\answer{-x^4} \; \leq x^4\cos \frac{2}{x} \leq \; \answer{x^4}
\]
for all $x$ except $x=0$.
Next, we need to find the limits of those bounds:
\[
\lim_{x \to 0}  - x^4 = \answer{0} \quad \text{and} \quad  \lim_{x \to 0} x^4 = \answer{0}.
\]
Since these answers are the same, the Squeeze Theorem allows us to conclude that 
\[
\lim_{x \to 0} x^4\cos \frac{2}{x} = \answer{0} \quad \text{as well}.
\]

\end{problem}

The Squeeze Theorem can also be used if $x \to \pm \infty$.

\begin{problem}(problem 2b)
Find
\[
\lim_{x \to \infty} \frac{\sin x}{x}
\]
using the Squeeze Theorem.\\
First, we need to find bounds. Since $-1 \leq \sin \theta \leq 1$ for all $\theta$, 
\[
\answer{-1/x} \; \leq \frac{\sin x}{x} \leq \;  \answer{1/x}
\]
for all $x>0$.
Next, we need to find the limits of those bounds:
\[
\lim_{x \to \infty}  -\frac{1}{x} = \answer{0} \quad \text{and} \quad \lim_{x \to \infty} \frac{1}{x} = \answer{0}.
\]
Since these answers are the same, the Squeeze Theorem allows us to conclude that 
\[
\lim_{x \to \infty} \frac{\sin x}{x} = \answer{0} \text{as well}.
\]

\end{problem}


\begin{example}[example 3]
Use the Squeeze Theorem to find a special limit: 
\[
\lim_{\theta \to 0} \frac{\sin \theta}{\theta}.
\]
Consider the figure below.  It consists of a small triangle, a sector of a circle of radius,  $r = 1$, and a large triangle.

\begin{image}
\begin{tikzpicture}
\draw[thin](2.828,0) -- (4,0) -- (4, 4) -- (2.828, 2.828);
\draw[thin]  (3.9, 0) -- (3.9, 0.1)-- (4, 0.1);
\draw[thin, blue] (0,0) -- (2.828, 0) -- (2.828, 2.828) -- cycle;
\draw[thin, blue]  (2.728, 0) -- (2.728, 0.1)-- (2.828, 0.1);
\draw[thin, red] (4,0) arc (0: 45:4);
\node[right] at (0.2,0.2) {$\theta$};
\node[below, blue] at (1.5,0) {$\cos\theta$};
\node[left, blue] at (2.8,1) {$\sin\theta$};
\node[right] at (4.1,2) {$\tan\theta$};
\node[above, blue] at (1.3,1.4) {$1$};
%\node[below] at (-.1,0) {$(0,0)$};
%\node[below] at (4,0) {$1$};
\end{tikzpicture}
\end{image}







The area of the small triangle is 
\[
A = \frac12 bh = \frac12 \cos\theta \sin\theta.
\]
The area of the sector is 
\[
A = \frac12 \theta r^2 = \frac12 \theta.
\]
The area of the large triangle is 
\[
A = \frac12 bh = \frac12 \cdot 1 \cdot \tan\theta = \frac12 \tan\theta.
\]

We can use the areas of these figures to create a compound inequality like the one found in the Squeeze Theorem.
Since the area of the small triangle is less than the area of the sector which is less than the area of the large triangle, we have:
\[
\frac12 \cos\theta \sin\theta < \frac12 \theta < \frac12 \tan\theta.
\]

Multiply through by 2:

\[
\cos\theta \sin\theta < \theta < \tan\theta.
\]

Divide through by $\sin\theta$.  Note that if $\theta$ is a small positive angle, 
then $\sin\theta >0$ so the direction of the inequality symbols remains unchanged:

\[
\cos\theta  < \frac{ \theta}{\sin\theta} < \frac{1}{\cos\theta}.
\]

Next we take reciprocals (this will change the direction of the inequality symbols):

\[
\frac{1}{\cos\theta}  > \frac{\sin\theta}{\theta} > \cos\theta,
\]
which is equivalent to 

\[
\cos\theta  < \frac{\sin\theta}{\theta} < \frac{1}{\cos\theta}.
\]

We now compute the limits of the upper and lower bounds:
\[
\lim_{\theta \to 0^+} \cos\theta = \cos 0 = 1 \quad \text{and} 
\quad \lim_{\theta \to 0^+} \frac{1}{\cos\theta} = \frac{1}{\cos 0} = 1. 
\]

Since the above limits are equal, by the Squeeze Theorem 
\[
\lim_{\theta \to 0^+} \frac{\sin\theta}{\theta} = 1 \;\; \text{  as well.}
\]

To compute the left-hand limit, we recall that $\sin(-\theta) = -\sin\theta$ for any angle $\theta$.
Therefore,
\[
\frac{\sin(-\theta)}{-\theta} = \frac{-\sin\theta}{-\theta} = \frac{\sin\theta}{\theta},
\]
which implies that the left-hand limit and the right-hand limit are equal:

\[
\lim_{\theta \to 0^-} \frac{\sin\theta}{\theta} =\lim_{\theta \to 0^+} \frac{\sin\theta}{\theta} = 1 
\]

Thus the two-sided limit exists and

\[
\lim_{\theta \to 0} \frac{\sin\theta}{\theta} =1
\]


\end{example}

\end{document}


