\documentclass[handout]{ximera}

%% You can put user macros here
%% However, you cannot make new environments



\newcommand{\ffrac}[2]{\frac{\text{\footnotesize $#1$}}{\text{\footnotesize $#2$}}}
\newcommand{\vasymptote}[2][]{
    \draw [densely dashed,#1] ({rel axis cs:0,0} -| {axis cs:#2,0}) -- ({rel axis cs:0,1} -| {axis cs:#2,0});
}


\graphicspath{{./}{firstExample/}}
\usepackage{forest}
\usepackage{amsmath}
\usepackage{amssymb}
\usepackage{array}
\usepackage[makeroom]{cancel} %% for strike outs
\usepackage{pgffor} %% required for integral for loops
\usepackage{tikz}
\usepackage{tikz-cd}
\usepackage{tkz-euclide}
\usetikzlibrary{shapes.multipart}


%\usetkzobj{all}
\tikzstyle geometryDiagrams=[ultra thick,color=blue!50!black]


\usetikzlibrary{arrows}
\tikzset{>=stealth,commutative diagrams/.cd,
  arrow style=tikz,diagrams={>=stealth}} %% cool arrow head
\tikzset{shorten <>/.style={ shorten >=#1, shorten <=#1 } } %% allows shorter vectors

\usetikzlibrary{backgrounds} %% for boxes around graphs
\usetikzlibrary{shapes,positioning}  %% Clouds and stars
\usetikzlibrary{matrix} %% for matrix
\usepgfplotslibrary{polar} %% for polar plots
\usepgfplotslibrary{fillbetween} %% to shade area between curves in TikZ



%\usepackage[width=4.375in, height=7.0in, top=1.0in, papersize={5.5in,8.5in}]{geometry}
%\usepackage[pdftex]{graphicx}
%\usepackage{tipa}
%\usepackage{txfonts}
%\usepackage{textcomp}
%\usepackage{amsthm}
%\usepackage{xy}
%\usepackage{fancyhdr}
%\usepackage{xcolor}
%\usepackage{mathtools} %% for pretty underbrace % Breaks Ximera
%\usepackage{multicol}



\newcommand{\RR}{\mathbb R}
\newcommand{\R}{\mathbb R}
\newcommand{\C}{\mathbb C}
\newcommand{\N}{\mathbb N}
\newcommand{\Z}{\mathbb Z}
\newcommand{\dis}{\displaystyle}
%\renewcommand{\d}{\,d\!}
\renewcommand{\d}{\mathop{}\!d}
\newcommand{\dd}[2][]{\frac{\d #1}{\d #2}}
\newcommand{\pp}[2][]{\frac{\partial #1}{\partial #2}}
\renewcommand{\l}{\ell}
\newcommand{\ddx}{\frac{d}{\d x}}

\newcommand{\zeroOverZero}{\ensuremath{\boldsymbol{\tfrac{0}{0}}}}
\newcommand{\inftyOverInfty}{\ensuremath{\boldsymbol{\tfrac{\infty}{\infty}}}}
\newcommand{\zeroOverInfty}{\ensuremath{\boldsymbol{\tfrac{0}{\infty}}}}
\newcommand{\zeroTimesInfty}{\ensuremath{\small\boldsymbol{0\cdot \infty}}}
\newcommand{\inftyMinusInfty}{\ensuremath{\small\boldsymbol{\infty - \infty}}}
\newcommand{\oneToInfty}{\ensuremath{\boldsymbol{1^\infty}}}
\newcommand{\zeroToZero}{\ensuremath{\boldsymbol{0^0}}}
\newcommand{\inftyToZero}{\ensuremath{\boldsymbol{\infty^0}}}


\newcommand{\numOverZero}{\ensuremath{\boldsymbol{\tfrac{\#}{0}}}}
\newcommand{\dfn}{\textbf}
%\newcommand{\unit}{\,\mathrm}
\newcommand{\unit}{\mathop{}\!\mathrm}
%\newcommand{\eval}[1]{\bigg[ #1 \bigg]}
\newcommand{\eval}[1]{ #1 \bigg|}
\newcommand{\seq}[1]{\left( #1 \right)}
\renewcommand{\epsilon}{\varepsilon}
\renewcommand{\iff}{\Leftrightarrow}

\DeclareMathOperator{\arccot}{arccot}
\DeclareMathOperator{\arcsec}{arcsec}
\DeclareMathOperator{\arccsc}{arccsc}
\DeclareMathOperator{\si}{Si}
\DeclareMathOperator{\proj}{proj}
\DeclareMathOperator{\scal}{scal}
\DeclareMathOperator{\cis}{cis}
\DeclareMathOperator{\Arg}{Arg}
%\DeclareMathOperator{\arg}{arg}
\DeclareMathOperator{\Rep}{Re}
\DeclareMathOperator{\Imp}{Im}
\DeclareMathOperator{\sech}{sech}
\DeclareMathOperator{\csch}{csch}
\DeclareMathOperator{\Log}{Log}

\newcommand{\tightoverset}[2]{% for arrow vec
  \mathop{#2}\limits^{\vbox to -.5ex{\kern-0.75ex\hbox{$#1$}\vss}}}
\newcommand{\arrowvec}{\overrightarrow}
\renewcommand{\vec}{\mathbf}
\newcommand{\veci}{{\boldsymbol{\hat{\imath}}}}
\newcommand{\vecj}{{\boldsymbol{\hat{\jmath}}}}
\newcommand{\veck}{{\boldsymbol{\hat{k}}}}
\newcommand{\vecl}{\boldsymbol{\l}}
\newcommand{\utan}{\vec{\hat{t}}}
\newcommand{\unormal}{\vec{\hat{n}}}
\newcommand{\ubinormal}{\vec{\hat{b}}}

\newcommand{\dotp}{\bullet}
\newcommand{\cross}{\boldsymbol\times}
\newcommand{\grad}{\boldsymbol\nabla}
\newcommand{\divergence}{\grad\dotp}
\newcommand{\curl}{\grad\cross}
%% Simple horiz vectors
\renewcommand{\vector}[1]{\left\langle #1\right\rangle}


\outcome{Determine when a limit is infinite}

\title{1.5 Infinite Limits}


\begin{document}

\begin{abstract}
Determine when a limit is infinite.
\end{abstract}

\maketitle

Infinite limits have a direct connection to vertical asymptotes.
\begin{theorem}
If 
\[
\lim_{x \to a^-} f(x) = \pm \infty
\]
or
\[
\lim_{x \to a^+} f(x) = \pm \infty
\]
then the line $x=a$ is a vertical asymptote for the graph of $y=f(x)$.\\
\end{theorem}

%This situation arises frequently when dealing with \textbf{rational functions}. At points where 
%the denominator is zero, but the numerator is not, a fraction of the form 
%\[
%\frac{c}{0}
%\]
%where $c\neq 0$, arises and one-sided limits in this case are either $\pm \infty$.



In the next few examples, we will investigate infinite limits of rational functions. 
These typically occur at points where the denominator of the rational function is approaching zero, 
but the numerator is not approaching zero, leading to the \textbf{undefined} fraction form
\[
\frac{c}{0}, \ \text{where} \ c \neq 0.
\]

\begin{example}[example 1]
Compute the limit: 

\[
\lim_{x \to 0^+} \frac{1}{x}.
\]

Plugging in $x=0$ yields the  expression $\frac{1}{0}$ which is undefined. 
As the denominator heads to zero, it gets smaller and smaller and so it will divide into the numerator 
more and more times.  This causes the fraction to ``blow up". In terms of the limit, it is therefore 
reasonable to expect an answer of either $\infty$ or $-\infty$.  
To determine which, we do a \textbf{sign analysis} as follows.
Since the values of $x$ are positive as $x \to 0^+$, the values of $f(x) = \frac{1}{x}$ are also positive.
%Furthermore, as the values of $x$ decrease towards zero, the values of the reciprocal, 
%$\frac{1}{x}$ increase without bound.

Hence, we can conclude that 

\[\lim_{x \to 0^+} \frac{1}{x} =  \infty. \]

The geometric significance of this result is that the line $x=0$ (the $y$-axis) 
is a vertical asymptote for the graph of the function $f(x) = \frac{1}{x}$, as shown below.


%\begin{tikzpicture}
%\begin{axis}
%\addplot[domain=.01:4, 
 %   samples=100, color=red]{1/x};
%\end{axis}
%\end{tikzpicture}
%\end{example}

\begin{center}
\begin{tikzpicture}
\begin{axis}[axis lines = center, title={The graph of $y=\dfrac{1}{x}$}]
\addplot[domain=-3:-.1, 
    samples=100, color=blue]{1/x};
\addplot[domain=.1:3, 
    samples=100, color=blue]{1/x};
\vasymptote {0}
\end{axis}
\end{tikzpicture}
\end{center}
\end{example}



\begin{problem}(problem 1)
  
	Determine the limit:
  \[
  \lim_{x \to 0^-} \frac{3}{x}.
  \]
		
		\begin{hint}
      $\frac{c}{0}$ gives $\pm \infty$
    \end{hint}
    \begin{hint}
      $x \to 0^-$ means $x<0$
    \end{hint}
    \begin{hint}
      Is $\frac{3}{x}$ positive or negative?
    \end{hint}
    
		The value of the limit  is
		(type infinity for $\infty$ or -infinity for $-\infty$)
		 $\answer{-\infty}$
		
\end{problem}
In general, when the undefined fraction form $c/0, (c\neq 0)$
arises in a one-sided limit, the answer is typically $\pm \infty$ and
a \textbf{sign analysis} of the denominator will determine which of these is correct.

\begin{example}[example 2]
Determine the limit: 
\[
\lim_{x \to 3^-} \frac{2}{x-3}.
\] 


First we observe that plugging in the value $x=3$ gives $\frac{2}{0}$ which is undefined and 
the fraction is ``blowing up" in the limit.  A sign analysis will tell us if the limit is $\pm\infty$.
Since $x \to 3^-$, we have $x<3$ and hence $x-3 <0$. 
Since the numerator is positive and the denominator is approaching $0$ through negative values,
the values of $f(x) = \frac{2}{x-3}$ in this limit are negative.  We conclude that

\[
\lim_{x \to 3^-} \frac{2}{x-3} =  -\infty. 
\]

This geometric significance of the result is that the line $x=3$ is a vertical asymptote for the 
graph of the function $f(x) = \frac{2}{x-3}$, as shown below.

%\begin{center}
%\begin{tikzpicture}
%\begin{axis}[axis lines = center, xlabel = $x$,
%    ylabel = {$f(x)$}]
%\addplot[domain=1:2.9, 
%    samples=100, color=blue]{2/(x-3)};

%\end{axis}
%\end{tikzpicture}
%\end{center}
%\end{example}


\begin{center}
\begin{tikzpicture}
\begin{axis}[axis lines = center, title={The graph of $y=\dfrac{2}{x-3}$}]
\addplot[domain=0:2.9, 
    samples=100, color=blue]{2/(x-3)};
\addplot[domain=3.1:6, 
  samples=100, color=blue]{2/(x-3)};
\vasymptote {3}
\end{axis}
\end{tikzpicture}
\end{center}
\end{example}


\begin{problem}(problem 2)
  
	Determine the limit:
  \[
  \lim_{x \to 2^+} \frac{4}{x-2}.
  \]
		
		\begin{hint}
      $\frac{c}{0}$ gives $\pm \infty$
    \end{hint}
    \begin{hint}
      $x \to 2^+$ means $x>2$
    \end{hint}
    \begin{hint}
      Is $\frac{4}{x-2}$ positive or negative?
    \end{hint}
    
		The value of the limit is
		(type infinity for $\infty$ or -infinity for $-\infty$)
		 $\answer{\infty}$
		
\end{problem}

\begin{example}[example 3]

Analyze the two-sided limit:
\[
\lim_{x\to 2} 
\frac{x + 1}{x-2}
\]

Plugging $x = 2$ into the rational function
\[f(x) = \frac{x + 1}{x-2}\]
gives the undefined expression $\frac{3}{0}$. From this information, we can conclude that the 
one-sided limits as $x$ approaches 2
will give either $\infty$ or $-\infty$, i.e., 
\[\lim_{x \to 2^-} \frac{x+1}{x-2}= \pm \infty\]
and
\[\lim_{x \to 2^+} \frac{x+1}{x-2}= \pm \infty.\]
To determine which, we will do a sign analysis on each one-sided limit. Consider the left hand limit first:
\[\lim_{x \to 2^-} \frac{x+1}{x-2}.\]
The numerator is approaching 3, which is positive. The denominator is approaching zero which is neither 
positive nor negative, but since $x \to 2^-$, we know that $x<2$ and therefore $x-2 <0$.  
Hence the denominator is negative in this limit. Since a positive divided by a 
negative is negative, we get:
\[\lim_{x \to 2^-} \frac{x+1}{x-2}  = -\infty.\]

Now, we will do a sign analysis on the right hand limit:
\[\lim_{x \to 2^+} \frac{x+1}{x-2}.\]
The numerator is approaching 3, which is positive. In the denominator, since $x \to 2^+$, 
we know that $x>2$ and therefore $x-2 >0$.  
Hence the denominator is positive in this limit. Since a positive divided by a 
positive is positive, we get:
\[\lim_{x \to 2^+} \frac{x+1}{x-2} = \infty.\]

The one-sided limits were different, so the two-sided limit does not exist:
\[\lim_{x \to 2} \frac{x+1}{x-2} \ \text{DNE}.\]
The graph of $f(x) = \frac{x+1}{x-2}$ has a vertical asymptote at $x = 2$, as shown below.

\begin{center}
\begin{tikzpicture}
\begin{axis}[axis lines = center, title={The graph of $y=\dfrac{x+1}{x-2}$}]
\addplot[domain=0:1.9, 
    samples=100, color=blue]{(x+1)/(x-2)};
\addplot[domain=2.1:4, 
    samples=100, color=blue]{(x+1)/(x-2)};
\vasymptote {2}
\end{axis}
\end{tikzpicture}
\end{center}
\end{example}



\begin{problem}(problem 3)
  
	Analyze the two-sided limit:
  \[
  \lim_{x \to 3} \frac{x-1}{x-3}.
  \]
  
    \begin{hint}
      Check the one-sided limits separately
    \end{hint}
    \begin{hint}
      Both one-sided limits involve division by zero
    \end{hint}
    \begin{hint}
      The one-sided limits are either $\pm \infty$
    \end{hint}
		\begin{hint}
		  If the one-sided limits are different, then the two-sided limit DNE
		\end{hint}	
		The value of the limit is
		(type infinity for $\infty$, -infinity for $-\infty$ or DNE)
		 $\answer{DNE}$
		
\end{problem}





\begin{example}[example 4]
Analyze the two-sided limit:
\[
\lim_{x\to -1} 
\frac{2x}{(x+1)^3}
\]

Plugging $x = -1$ into the rational function
\[f(x) = \frac{2x}{(x+1)^3}\]
gives the undefined expression $\frac{-2}{0}$. From this information, we can conclude 
that the one-sided limits
will give either $\infty$ or $-\infty$, i.e., 
\[\lim_{x \to -1^-} \frac{2x}{(x+1)^3}= \pm \infty,\]
and
\[\lim_{x \to -1^+} \frac{2x}{(x+1)^3}= \pm \infty.\]
To determine which, we will do a sign analysis on each of the one-sided limits. 
Consider the left hand limit first:
\[\lim_{x \to -1^-} \frac{2x}{(x+1)^3}.\]
The numerator is approaching $-2$, which is negative. In the denominator, since $x \to -1^-$, 
we know that $x<-1$ and therefore $x+1 <0$. Since $x+1$ is negative, so is $(x+1)^3$. Negative divided by 
negative is positive, so we get:
\[
\lim_{x \to -1^-} \frac{2x}{(x+1)^3} = \infty.
\]

Now, we will do a sign analysis on the right hand limit:
\[\lim_{x \to -1^+} \frac{2x}{(x+1)^3}.\]
The numerator is still approaching $-2$. In the denominator, since $x \to -1^+$, 
we know that $x>-1$ and therefore $x+1 >0$. Since $x+1$ is positive, so is $(x+1)^3$. Negative divided by 
positive is negative, so we get:

\[
\lim_{x \to -1^+} \frac{2x}{(x+1)^3} = -\infty.
\]

The one-sided limits were different, so the two-sided limit does not exist:
\[\lim_{x \to 2} \frac{2x}{(x+1)^3} \ \text{DNE}.\]

The graph of $f(x) = \frac{2x}{(x+1)^3}$ has a vertical asymptote at $x = -1$, as shown below.

\begin{center}
\begin{tikzpicture}
\begin{axis}[axis lines = center,  title={The graph of $y=\dfrac{2x}{(x+1)^3}$}]
\addplot[domain=-3:-1.1, 
    samples=100, color=blue]{(2*x)/(x+1)^3};
\addplot[domain=-0.9:1, 
    samples=100, color=blue]{(2*x)/(x+1)^3};
\vasymptote {-1}
\end{axis}
\end{tikzpicture}
\end{center}
\end{example}


\begin{problem}(problem 4)
  
	Analyze the limit:
  \[
  \lim_{x \to -3} \frac{x+1}{(x+3)^2}.
  \]
  
    \begin{hint}
      This is a two-sided limit. Check the one-sided limits separately
    \end{hint}
    \begin{hint}
      Both one-sided limits involve division by zero
    \end{hint}
    \begin{hint}
      The one-sided limits are either $\pm \infty$
    \end{hint}
		\begin{hint}
		  If the one-sided limits are different, then the two-sided limit DNE
		\end{hint}	
		The value of the limit is
		(type infinity for $\infty$, -infinity for $-\infty$ or DNE)
		 $\answer{-\infty}$
		
\end{problem}

The functions $\ln(x)$ and $\tan(x)$ also have infinite limits.

\begin{example}[example 5]
The graph of $f(x) = \ln(x)$ has a vertical asymptote at $x=0$
and 
\[
\lim_{x\to 0^+} \ln(x) = -\infty,
\]
as shown in the graph below.


\begin{center}
\begin{tikzpicture}
\begin{axis}[axis lines = center,  title={The graph of $y=\ln(x)$}]
\addplot[domain=0.01:5, 
    samples=100, color=blue]{ln(x)};
\end{axis}
\end{tikzpicture}
\end{center}


%\[\graph{y=\ln(x)}\]
\end{example}


\begin{problem}(problem 5a)
  
	Use the result of example 5 to determine the limit:
  \[
  \lim_{x \to {2^+}} \ln(x-2).
  \]
		\begin{hint}
		  $\ln(x) \to -\infty \quad \text{as} \quad x \to 0^+$
		\end{hint}	
		The value of the limit is
		(type infinity for $\infty$, -infinity for $-\infty$ or DNE)
		 $\answer{-\infty}$
		
\end{problem}

\begin{problem}(problem 5b)
  
	Use the result of example 5 to determine the limit:
  \[
  \lim_{x \to {1^-}} 1 - \ln(1-x).
  \]
		\begin{hint}
		  $\ln(x) \to -\infty \quad \text{as} \quad x \to 0^+$
		\end{hint}	
		The value of the limit is
		(type infinity for $\infty$, -infinity for $-\infty$ or DNE)
		 $\answer{\infty}$
		
\end{problem}

\begin{example}[example 6]
The graph of $f(\theta) = \tan(\theta)$ has a vertical asymptotes at multiples of $\pi/2$ and
%$x=-\pi/2$ and $x=\pi/2$.
we have the following limits: 
\[
\lim_{\theta\to -\pi/2^+} \tan(\theta) = -\infty,
\]
and
\[
\lim_{\theta\to \pi/2^-} \tan(\theta) = \infty,
\]

as shown in the graph below.
%\begin{center}
%\begin{tikzpicture}
%\begin{axis}[axis lines = center,  title={The graph of $y=\tan(x)$}]
%\addplot[domain=-1.55:1.55, 
%    samples=100, color=blue]{tan(deg(x))};
		
%\addplot[domain=1.6:4.7, 
%    samples=100, color=blue]{tan(deg(x))};
%\addplot[domain=-4.7:-1.6, 
%    samples=100, color=blue]{tan(deg(x))};		
%\vasymptote{-1.57}
%\vasymptote{1.57}	
			
%\end{axis}
%\end{tikzpicture}
%\end{center}
\[
\graph{tan(x)}
\]
\end{example}

\begin{problem}(problem 6a)
  
	Compute the limit:
  \[
  \lim_{\theta \to {0^+}} \cot(\theta).
  \]
		\begin{hint}
		  $\cot(\theta) = \frac{\cos(\theta)}{\sin(\theta)}$
		\end{hint}	
		\begin{hint}
		  $\cos(0) = 1$ and $\sin(0) = 0$
		\end{hint}
		
		The value of the limit is
		(type infinity for $\infty$, -infinity for $-\infty$ or DNE)
		 $\answer{\infty}$
		
\end{problem}

\begin{problem}(problem 6b)
  
	Compute the limit:
  \[
  \lim_{\theta \to {\pi^-}} \csc(\theta).
  \]
		\begin{hint}
		  $\csc(\theta) = \frac{1}{\sin(\theta)}$
		\end{hint}	
		\begin{hint}
		  $\sin(\pi) = 0$
		\end{hint}
		The value of the limit is
		(type infinity for $\infty$, -infinity for $-\infty$ or DNE)
		 $\answer{\infty}$
		
\end{problem}

\begin{problem}(problem 6c)
  
	Compute the limit:
  \[
  \lim_{\theta \to {\pi^+}} \csc(\theta).
  \]
		\begin{hint}
		  $\csc(\theta) = \frac{1}{\sin(\theta)}$
		\end{hint}	
		\begin{hint}
		  $\sin(\pi) = 0$
		\end{hint}
		The value of the limit is
		(type infinity for $\infty$, -infinity for $-\infty$ or DNE)
		 $\answer{-\infty}$
		
\end{problem}


\begin{center}
\begin{foldable}
\unfoldable{Here is a detailed, lecture style video on infinite limits:}
\youtube{AkYCqV75EOw}
\end{foldable}
\end{center}





\end{document}






