\documentclass[handout]{ximera}

%% You can put user macros here
%% However, you cannot make new environments



\newcommand{\ffrac}[2]{\frac{\text{\footnotesize $#1$}}{\text{\footnotesize $#2$}}}
\newcommand{\vasymptote}[2][]{
    \draw [densely dashed,#1] ({rel axis cs:0,0} -| {axis cs:#2,0}) -- ({rel axis cs:0,1} -| {axis cs:#2,0});
}


\graphicspath{{./}{firstExample/}}
\usepackage{forest}
\usepackage{amsmath}
\usepackage{amssymb}
\usepackage{array}
\usepackage[makeroom]{cancel} %% for strike outs
\usepackage{pgffor} %% required for integral for loops
\usepackage{tikz}
\usepackage{tikz-cd}
\usepackage{tkz-euclide}
\usetikzlibrary{shapes.multipart}


%\usetkzobj{all}
\tikzstyle geometryDiagrams=[ultra thick,color=blue!50!black]


\usetikzlibrary{arrows}
\tikzset{>=stealth,commutative diagrams/.cd,
  arrow style=tikz,diagrams={>=stealth}} %% cool arrow head
\tikzset{shorten <>/.style={ shorten >=#1, shorten <=#1 } } %% allows shorter vectors

\usetikzlibrary{backgrounds} %% for boxes around graphs
\usetikzlibrary{shapes,positioning}  %% Clouds and stars
\usetikzlibrary{matrix} %% for matrix
\usepgfplotslibrary{polar} %% for polar plots
\usepgfplotslibrary{fillbetween} %% to shade area between curves in TikZ



%\usepackage[width=4.375in, height=7.0in, top=1.0in, papersize={5.5in,8.5in}]{geometry}
%\usepackage[pdftex]{graphicx}
%\usepackage{tipa}
%\usepackage{txfonts}
%\usepackage{textcomp}
%\usepackage{amsthm}
%\usepackage{xy}
%\usepackage{fancyhdr}
%\usepackage{xcolor}
%\usepackage{mathtools} %% for pretty underbrace % Breaks Ximera
%\usepackage{multicol}



\newcommand{\RR}{\mathbb R}
\newcommand{\R}{\mathbb R}
\newcommand{\C}{\mathbb C}
\newcommand{\N}{\mathbb N}
\newcommand{\Z}{\mathbb Z}
\newcommand{\dis}{\displaystyle}
%\renewcommand{\d}{\,d\!}
\renewcommand{\d}{\mathop{}\!d}
\newcommand{\dd}[2][]{\frac{\d #1}{\d #2}}
\newcommand{\pp}[2][]{\frac{\partial #1}{\partial #2}}
\renewcommand{\l}{\ell}
\newcommand{\ddx}{\frac{d}{\d x}}

\newcommand{\zeroOverZero}{\ensuremath{\boldsymbol{\tfrac{0}{0}}}}
\newcommand{\inftyOverInfty}{\ensuremath{\boldsymbol{\tfrac{\infty}{\infty}}}}
\newcommand{\zeroOverInfty}{\ensuremath{\boldsymbol{\tfrac{0}{\infty}}}}
\newcommand{\zeroTimesInfty}{\ensuremath{\small\boldsymbol{0\cdot \infty}}}
\newcommand{\inftyMinusInfty}{\ensuremath{\small\boldsymbol{\infty - \infty}}}
\newcommand{\oneToInfty}{\ensuremath{\boldsymbol{1^\infty}}}
\newcommand{\zeroToZero}{\ensuremath{\boldsymbol{0^0}}}
\newcommand{\inftyToZero}{\ensuremath{\boldsymbol{\infty^0}}}


\newcommand{\numOverZero}{\ensuremath{\boldsymbol{\tfrac{\#}{0}}}}
\newcommand{\dfn}{\textbf}
%\newcommand{\unit}{\,\mathrm}
\newcommand{\unit}{\mathop{}\!\mathrm}
%\newcommand{\eval}[1]{\bigg[ #1 \bigg]}
\newcommand{\eval}[1]{ #1 \bigg|}
\newcommand{\seq}[1]{\left( #1 \right)}
\renewcommand{\epsilon}{\varepsilon}
\renewcommand{\iff}{\Leftrightarrow}

\DeclareMathOperator{\arccot}{arccot}
\DeclareMathOperator{\arcsec}{arcsec}
\DeclareMathOperator{\arccsc}{arccsc}
\DeclareMathOperator{\si}{Si}
\DeclareMathOperator{\proj}{proj}
\DeclareMathOperator{\scal}{scal}
\DeclareMathOperator{\cis}{cis}
\DeclareMathOperator{\Arg}{Arg}
%\DeclareMathOperator{\arg}{arg}
\DeclareMathOperator{\Rep}{Re}
\DeclareMathOperator{\Imp}{Im}
\DeclareMathOperator{\sech}{sech}
\DeclareMathOperator{\csch}{csch}
\DeclareMathOperator{\Log}{Log}

\newcommand{\tightoverset}[2]{% for arrow vec
  \mathop{#2}\limits^{\vbox to -.5ex{\kern-0.75ex\hbox{$#1$}\vss}}}
\newcommand{\arrowvec}{\overrightarrow}
\renewcommand{\vec}{\mathbf}
\newcommand{\veci}{{\boldsymbol{\hat{\imath}}}}
\newcommand{\vecj}{{\boldsymbol{\hat{\jmath}}}}
\newcommand{\veck}{{\boldsymbol{\hat{k}}}}
\newcommand{\vecl}{\boldsymbol{\l}}
\newcommand{\utan}{\vec{\hat{t}}}
\newcommand{\unormal}{\vec{\hat{n}}}
\newcommand{\ubinormal}{\vec{\hat{b}}}

\newcommand{\dotp}{\bullet}
\newcommand{\cross}{\boldsymbol\times}
\newcommand{\grad}{\boldsymbol\nabla}
\newcommand{\divergence}{\grad\dotp}
\newcommand{\curl}{\grad\cross}
%% Simple horiz vectors
\renewcommand{\vector}[1]{\left\langle #1\right\rangle}


\outcome{Simplify algebraic expressions using the rules for exponents}

\title{0.1 Exponent Rules}

\begin{document}

\begin{abstract}
We explore the rules for exponents.
\end{abstract}



\maketitle

\begin{center}
\textbf{Understanding Exponential Expressions}
\end{center}

An \textbf{exponential expression} takes the form $a^b$, where $a$ is the \textbf{base}  and 
$b$ is the \textbf{exponent}.
For instance, $2^3, x^n$ and $e^x$ are exponential expressions. When we encounter the expression $a^b$, 
where $a$ is the base and $b$ is a positive integer, it signifies the act of multiplying $a$
by itself $b$ times. Visually, this can be expressed as:
\[
a^b = \underbrace{a \cdot a \cdot \ldots \cdot a}_\text{$b$ times} 
\]


\begin{problem}(problem 1)
Compute the value of each of the following exponential expressions (without a calculator):
\[
a) \;6^2 = \answer{36} \quad b) \;7^2 = \answer{49} \quad c) \;8^2 = \answer{64} \quad d) \;9^2 = \answer{81}
\]
\end{problem}

\begin{problem}(problem 2)
Compute the value of each of the following exponential expressions (without a calculator):
\[
a) \;2^3 = \answer{8} \quad b) \;3^3 = \answer{27} \quad c) \;4^3 = \answer{64} \quad d) \;5^3 = \answer{125} 
\quad e) \;10^3 = \answer{1000}
\]
\end{problem}

\begin{problem}(problem 3)
Compute the value of each of the following exponential expressions (without a calculator):
\[
a) \;2^4 = \answer{14} \quad b) \;3^4 = \answer{81} \quad c) \;5^4 = \answer{625} \quad d) \;10^4 = \answer{10000}
\]
\end{problem}

\begin{problem}(problem 4)
Compute the value of each of the following exponential expressions (without a calculator):
\[
a) \;2^5 = \answer{32} \quad b) \;2^6 = \answer{64} \quad c) \;2^7 = \answer{128} \quad d) \;2^8 = \answer{256}
\]
\end{problem}

Note that $64 = 8^2 = 4^3 = 2^6$.\\

According to the order of operations, exponents are computed before other 
multiplications and divisions. Hence, if $f(x) = 5x^2$ then $f(3) = 5(3^2) = 5(9) = 45$.

\begin{problem}(problem 5)
Evaluate the function $f(x) = 4x^2$ as indicated (without a calculator):
\[
a) \;f(2) = \answer{16} \quad b) \;f(-2) = \answer{16} \quad c) \;f(-3) = \answer{36} \quad d) \;f(-5) = \answer{100}
\]
\end{problem}

\begin{problem}(problem 6)
Compute the value of each of the following exponential expressions (without a calculator):
\[
a) \;-2^2 = \answer{-4} \quad b) \;(-2)^2 = \answer{4} \quad c) \;-1^4 = \answer{-1} \quad d) \;(-1)^4 = \answer{1}
\]
\end{problem}

\begin{center}
\textbf{Adding Exponents}
\end{center}
When multiplying exponential expressions with the same base, we add the exponents:
\[
a^b \cdot a^c = a^{b+c}
\] 
For positive integer exponents, the rationale is shown below:
\[
a^b \cdot a^c = (\underbrace{a \cdot a \cdot a \cdot \ldots \cdot a}_\text{$b$ times}) \cdot
(\underbrace{a \cdot a \cdot a \cdot \ldots \cdot a}_\text{$c$ times})
\]

\[
= \underbrace{a \cdot a \cdot a \cdot \ldots \cdot a}_\text{$b+c$ times}
\]

\[
= a^{b+c}
\]

\begin{problem}(problem 7)
Combine each of the following exponential expressions into a single exponential expression:
\[
a) \;a^4\cdot a^5 = \answer{a^9} \quad b) \;r^2 \cdot r^3  = \answer{r^5} \quad c) \;n\cdot n^6 = \answer{n^7} 
\quad d) \;x^{14}\cdot x^{17} = \answer{x^{31}}
\]
\end{problem}


\begin{problem}(problem 8)
Distribute: $x^2(x^4 + x^7)  = \answer{x^9+x^6}$
\end{problem}


\begin{center}
\textbf{Subtracting Exponents}
\end{center}
When dividing exponential expressions with the same base, we subtract the exponents:
\[
\frac{a^b}{a^c} = a^{b-c}
\] 
For positive integer exponents $b$ and $c$ with $b > c$, the rationale is shown below:
\begin{eqnarray*}
\frac{a^b}{a^c} &=& \frac{\overbrace{a \cdot a \cdot a \cdot \ldots \cdot a}^\text{$b$ times}}
{\underbrace{ a \cdot a \cdot \ldots \cdot a}_\text{$c$ times}}\\[10pt]
&=& \underbrace{ a \cdot a \cdot \ldots \cdot a}_\text{$b-c$ times}\\[10pt]
&=& a^{b-c}
\end{eqnarray*}

\begin{problem}(problem 9)
Combine each of the following exponential expressions into a single exponential expression:
\[
a) \;\frac{a^2}{a} = \answer{a} \quad b) \;\frac{r^3}{r} = \answer{r^2} \quad c) \;\frac{n^5}{n^3} = \answer{n^2} 
\quad d) \;\frac{x^7}{x^3} = \answer{x^4}
\]
\end{problem}


\begin{problem}(problem 10)
Divide: $\frac{x^8 + x^5 + x^2}{x^2} = \answer{x^6 + x^3 + 1}$
\end{problem}


\begin{center}
\textbf{The Zero Exponent}
\end{center}
For equal bases $a$,
\[
\frac{a^b}{a^b} = 1
\] 
Also, using the exponent subtraction rule, 
\[
\frac{a^b}{a^b} = a^{b-b} = a^0
\] 
Therefore, we define $a^0 = 1$ for $a \neq 0$.

\begin{definition} 
If $a \neq 0$, then we define $a^0 \equiv 1$.
\end{definition}

\begin{problem}(problem 11)
Evaluate each of the following exponential expressions (without a calculator):
\[
a) \;-3^0 = \answer{-1} \quad b) \;e^0 = \answer{1} \quad c) \;\pi^0 = \answer{1} \quad d) \;(-1)^0 = \answer{1}
\]
\end{problem}

\begin{center}
\textbf{Negative Exponents}
\end{center}
If the exponent in the denominator is greater than the exponent in the numerator, 
then subtracting exponents yields a negative exponent. Consider the following example:

\[
\frac{a^3}{a^5} = a^{3-5} = a^{-2}
\]
On the other hand, we can rewrite this ratio of exponential expressions as follows
\[
\frac{a^3}{a^5} = \frac{a\cdot a\cdot a}{a\cdot a\cdot a \cdot a\cdot a} = \frac{1}{a\cdot a} = \frac{1}{a^2}
\]
Putting these together yields
\[
a^{-2} = \frac{1}{a^2}
\]
We generalize this idea and define negative exponents as follows:
\begin{definition} For $a \neq 0$ and $n$ a positive integer, we define $a^{-n}$ by
\[
a^{-n} = \frac{1}{a^n}
\]
\end{definition}

\begin{problem}(problem 12)
Rewrite the following expressions using negative exponents:
\[
a) \;\frac{3}{x^5} = \answer{3x^{-5}} \quad b) \;\frac{1}{3x^5} = \answer{x^{-5}/3} \quad c) \;\frac{1}{u^2} = \answer{u^{-2}}
\]
\end{problem}

\begin{problem}(problem 13)
Simply the following expression using negative exponents:
\[
\frac{x^4 + x^2 + 1}{x^5} = \answer{x^{-1} + x^{-3} + x^{-5}}
\]
\end{problem}


\begin{center}
\textbf{Multiplying Exponents}
\end{center}

When raising an exponential expression to a power, we multiply the exponents. 
In the justification presented is below, the exponents, $b$ and $c$, are positive integers:
\[
\left(a^b\right)^c = \underbrace{a^b \cdot a^b  \cdot \ldots \cdot a^b}_\text{$c$ times}
\]
Adding the exponents gives:
\[
= a^{\overbrace{b + b + \dots +b}^\text{$c$ times}} = a^{bc}
\]

\begin{problem}(problem 14)
Combine each of the following into a single exponential expression (without a calculator):
\[
a) \;\left(a^4\right)^5 = \answer{a^{20}} \quad b) \;\left(r^2\right)^3  = \answer{r^6} \quad c) \;\left(n^6\right)^6 = \answer{n^{36}} 
\quad d) \;\left(x^{12}\right)^{12} = \answer{x^{144}}
\]
\end{problem}

\begin{center}
\textbf{Rational Exponents}
\end{center}
Let $n$ be a positive integer and let $a\geq 0$ if $n$ is even. 
Then  $\sqrt[n] a$ is the number which when multiplied by itself $n$ times gives $a$.
In other words:
\[
(\sqrt[n] a)^n = a
\]
Allow for a moment the use of the exponent $\frac{1}{n}$.  Using the rule for multiplying exponents, we have
\[
\left(a^\frac{1}{n}\right)^n = a^{\left(\frac{1}{n} \cdot n\right)} = a^1 = a
\]
Thus, $\sqrt[n] a$ and $a^\frac{1}{n}$ both yield $a$ when raised to the $n^\text{th}$ power.
Hence, we define $a^{1/n} = \sqrt[n] a$.
\begin{definition}
Let $n$ be a positive integer. We define
\[
a^{\frac{1}{n}} \equiv \sqrt[n] a
\]
where $a \geq 0 $ if $n$ is even.
\end{definition}

Now consider
\[
(\sqrt[n] a)^m = \left(a^{1/n}\right)^m = a^{m/n}
\]
This leads to the following definition.
\begin{definition}
Let $m$ and $n$ be positive integers. Then we define
\[
a^{\frac{m}{n}} \equiv (\sqrt[n] a)^m = \sqrt[n]{a^m}
\]
where $a \geq 0 $ if $n$ is even.
\end{definition}


\begin{problem}(problem 15)
Rewrite the following expressions using rational (and possibly negative) exponents:
\[
a) \;\frac{1}{\sqrt a} = \answer{a^{-1/2}} \quad b) \;\sqrt[3] r = \answer{r^{1/3}} \quad c) \;\sqrt[4]{u^3} = \answer{u^{3/4}}
\]
\end{problem}


\begin{problem}(problem 16)
Simplify into a sum of powers of $x$: 
\[
\frac{x^5 + \sqrt[3]{x^2} + x^{-3}}{x\sqrt x} = x^{7/2} + x^{-5/6} + x^{-9/2}
\]
\end{problem}


\end{document}



