\documentclass[handout]{ximera}

%% You can put user macros here
%% However, you cannot make new environments



\newcommand{\ffrac}[2]{\frac{\text{\footnotesize $#1$}}{\text{\footnotesize $#2$}}}
\newcommand{\vasymptote}[2][]{
    \draw [densely dashed,#1] ({rel axis cs:0,0} -| {axis cs:#2,0}) -- ({rel axis cs:0,1} -| {axis cs:#2,0});
}


\graphicspath{{./}{firstExample/}}
\usepackage{forest}
\usepackage{amsmath}
\usepackage{amssymb}
\usepackage{array}
\usepackage[makeroom]{cancel} %% for strike outs
\usepackage{pgffor} %% required for integral for loops
\usepackage{tikz}
\usepackage{tikz-cd}
\usepackage{tkz-euclide}
\usetikzlibrary{shapes.multipart}


%\usetkzobj{all}
\tikzstyle geometryDiagrams=[ultra thick,color=blue!50!black]


\usetikzlibrary{arrows}
\tikzset{>=stealth,commutative diagrams/.cd,
  arrow style=tikz,diagrams={>=stealth}} %% cool arrow head
\tikzset{shorten <>/.style={ shorten >=#1, shorten <=#1 } } %% allows shorter vectors

\usetikzlibrary{backgrounds} %% for boxes around graphs
\usetikzlibrary{shapes,positioning}  %% Clouds and stars
\usetikzlibrary{matrix} %% for matrix
\usepgfplotslibrary{polar} %% for polar plots
\usepgfplotslibrary{fillbetween} %% to shade area between curves in TikZ



%\usepackage[width=4.375in, height=7.0in, top=1.0in, papersize={5.5in,8.5in}]{geometry}
%\usepackage[pdftex]{graphicx}
%\usepackage{tipa}
%\usepackage{txfonts}
%\usepackage{textcomp}
%\usepackage{amsthm}
%\usepackage{xy}
%\usepackage{fancyhdr}
%\usepackage{xcolor}
%\usepackage{mathtools} %% for pretty underbrace % Breaks Ximera
%\usepackage{multicol}



\newcommand{\RR}{\mathbb R}
\newcommand{\R}{\mathbb R}
\newcommand{\C}{\mathbb C}
\newcommand{\N}{\mathbb N}
\newcommand{\Z}{\mathbb Z}
\newcommand{\dis}{\displaystyle}
%\renewcommand{\d}{\,d\!}
\renewcommand{\d}{\mathop{}\!d}
\newcommand{\dd}[2][]{\frac{\d #1}{\d #2}}
\newcommand{\pp}[2][]{\frac{\partial #1}{\partial #2}}
\renewcommand{\l}{\ell}
\newcommand{\ddx}{\frac{d}{\d x}}

\newcommand{\zeroOverZero}{\ensuremath{\boldsymbol{\tfrac{0}{0}}}}
\newcommand{\inftyOverInfty}{\ensuremath{\boldsymbol{\tfrac{\infty}{\infty}}}}
\newcommand{\zeroOverInfty}{\ensuremath{\boldsymbol{\tfrac{0}{\infty}}}}
\newcommand{\zeroTimesInfty}{\ensuremath{\small\boldsymbol{0\cdot \infty}}}
\newcommand{\inftyMinusInfty}{\ensuremath{\small\boldsymbol{\infty - \infty}}}
\newcommand{\oneToInfty}{\ensuremath{\boldsymbol{1^\infty}}}
\newcommand{\zeroToZero}{\ensuremath{\boldsymbol{0^0}}}
\newcommand{\inftyToZero}{\ensuremath{\boldsymbol{\infty^0}}}


\newcommand{\numOverZero}{\ensuremath{\boldsymbol{\tfrac{\#}{0}}}}
\newcommand{\dfn}{\textbf}
%\newcommand{\unit}{\,\mathrm}
\newcommand{\unit}{\mathop{}\!\mathrm}
%\newcommand{\eval}[1]{\bigg[ #1 \bigg]}
\newcommand{\eval}[1]{ #1 \bigg|}
\newcommand{\seq}[1]{\left( #1 \right)}
\renewcommand{\epsilon}{\varepsilon}
\renewcommand{\iff}{\Leftrightarrow}

\DeclareMathOperator{\arccot}{arccot}
\DeclareMathOperator{\arcsec}{arcsec}
\DeclareMathOperator{\arccsc}{arccsc}
\DeclareMathOperator{\si}{Si}
\DeclareMathOperator{\proj}{proj}
\DeclareMathOperator{\scal}{scal}
\DeclareMathOperator{\cis}{cis}
\DeclareMathOperator{\Arg}{Arg}
%\DeclareMathOperator{\arg}{arg}
\DeclareMathOperator{\Rep}{Re}
\DeclareMathOperator{\Imp}{Im}
\DeclareMathOperator{\sech}{sech}
\DeclareMathOperator{\csch}{csch}
\DeclareMathOperator{\Log}{Log}

\newcommand{\tightoverset}[2]{% for arrow vec
  \mathop{#2}\limits^{\vbox to -.5ex{\kern-0.75ex\hbox{$#1$}\vss}}}
\newcommand{\arrowvec}{\overrightarrow}
\renewcommand{\vec}{\mathbf}
\newcommand{\veci}{{\boldsymbol{\hat{\imath}}}}
\newcommand{\vecj}{{\boldsymbol{\hat{\jmath}}}}
\newcommand{\veck}{{\boldsymbol{\hat{k}}}}
\newcommand{\vecl}{\boldsymbol{\l}}
\newcommand{\utan}{\vec{\hat{t}}}
\newcommand{\unormal}{\vec{\hat{n}}}
\newcommand{\ubinormal}{\vec{\hat{b}}}

\newcommand{\dotp}{\bullet}
\newcommand{\cross}{\boldsymbol\times}
\newcommand{\grad}{\boldsymbol\nabla}
\newcommand{\divergence}{\grad\dotp}
\newcommand{\curl}{\grad\cross}
%% Simple horiz vectors
\renewcommand{\vector}[1]{\left\langle #1\right\rangle}


\outcome{In this section we define the cross product and we use it to create orthogonal vectors.}

\title{1.5 The Cross Product}



\begin{document}

\begin{abstract}
In this section we define the cross product and we use it to create orthogonal vectors.
\end{abstract}
 
\maketitle
The cross product is a special operation that helps us to create a vector that is orthogonal to two given two vectors in $\R^3$.
\begin{definition}[Cross Product in $\R^3$]
If $\avec{v_1}$ and $\avec{v_2}$ are vectors in $\R^3$ given by
\[
\avec{v_1} = \vector{x_1, y_1, z_1} \text{  and   } \;\avec{v_2} = \vector{x_2, y_2, z_2}
\]
then the cross product $\avec{v_1} \cross \avec{v_2}$ is defined by
\[
\avec{v_1} \cross \avec{v_2} =  (y_1z_2 - z_1y_2) \avec{i} + (z_1x_2 - x_1z_2) \avec{j} + (x_1y_2 - y_1x_2) \avec{k} 
\]
\end{definition}

The definition of the cross product of two vectors is easier to understand in the context of matrix determinants.

\begin{definition}[Determinant of $2 \times 2$ Matrix]
The determinant of the $2 \times 2$ matrix 
\[
\begin{bmatrix}
a & b\\
c & d
\end{bmatrix}
\]
is the number given by
\[
\begin{vmatrix}
a & b\\
c & d
\end{vmatrix}
= ad-bc
\]

\end{definition}

\begin{definition}[Determinant of $3 \times 3$ Matrix]
The determinant of the $3 \times 3$ matrix 
\[
\begin{bmatrix}
a & b & c\\
d & e & f\\
g & h & i
\end{bmatrix}
\]
is the number given by
\[
\begin{vmatrix}
a & b & c\\
d & e & f\\
g & h & i
\end{vmatrix}
= a(ei-fh) - b(di-fg) + c(dh-eg)
\]
Note that the quantites in the parentheses in the above definition are the determinants of the $2 \times 2$ matrices
obtained by removing the first row and a column (the column of the coefficient of the quantity in the parentheses) from the original $3 \times 3$ matrix.
\end{definition}
 Now to create the cross product of two vectors, we use the determinant of a $3 \times 3$ matrix with the vectors $\avec{i}, \avec{j}$ and $\avec{k}$ in the first row.
 So, if $\avec{v} = \vector{x_1, y_1, z_1}$ and $\avec{w} = \vector{x_2, y_2, z_2}$ then the cross product is given by
 \[
\avec{v} \cross \avec{w} = 
 \begin{vmatrix}
\avec{i} & \avec{j} & \avec{k}\\
x_1 & y_1 & z_1\\
x_2 & y_2 & z_2
\end{vmatrix} = (y_1z_2 - z_1y_2)\avec{i} - (x_1z_2 - z_1x_2)\avec{j} + (x_1y_2-y_1x_2)\avec{k}
\]
which agrees with the definition above after multiplying through by $-1$ in the middle term.

The vector $\avec{v}\cross \avec{w}$ is orthogonal to both $\avec{v}$ and $\avec{w}$. To see this we compute the dot product:
\begin{align*}
\left(\avec{v} \cross \avec{w} \right) \dotp \avec{v} &= \vector{y_1z_2 - z_1y_2, z_1x_2 - x_1z_2, x_1y_2 - y_1x_2} \dotp \vector{x_1, y_1, z_1}\\
                                                &= x_1(y_1z_2 - z_1y_2) + y_1(z_1x_2 - x_1z_2) + z_1(x_1y_2 - y_1x_2)\\
                                                &= x_1y_1z_2 - x_1y_2z_1 + x_2y_1z_1 - x_1y_1z_2 + x_1y_2z_1 - x_2y_1z_1\\
                                                &= x_1y_1z_2 - x_1y_1z_2 + x_1y_2z_1 - x_1y_2z_1 + x_2y_1z_1 - x_2y_1z_1\\
                                                &= 0
\end{align*}
Hence, $\avec{v} \cross \avec{w}$ and $\avec{v}$ are orthogonal.  The reader should perfrom a similar computation to verify that
$\avec{v} \cross \avec{w}$ and $\avec{w}$ are also orthogonal.\\

It should be noted that the cross product operation is not commutative.  In fact, it is anti-commutative:
\[
\avec{v} \cross \avec{w} = -(\avec{w} \cross \avec{v})
\]

\end{document}



