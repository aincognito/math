\documentclass[handout]{ximera}

%% You can put user macros here
%% However, you cannot make new environments



\newcommand{\ffrac}[2]{\frac{\text{\footnotesize $#1$}}{\text{\footnotesize $#2$}}}
\newcommand{\vasymptote}[2][]{
    \draw [densely dashed,#1] ({rel axis cs:0,0} -| {axis cs:#2,0}) -- ({rel axis cs:0,1} -| {axis cs:#2,0});
}


\graphicspath{{./}{firstExample/}}
\usepackage{forest}
\usepackage{amsmath}
\usepackage{amssymb}
\usepackage{array}
\usepackage[makeroom]{cancel} %% for strike outs
\usepackage{pgffor} %% required for integral for loops
\usepackage{tikz}
\usepackage{tikz-cd}
\usepackage{tkz-euclide}
\usetikzlibrary{shapes.multipart}


%\usetkzobj{all}
\tikzstyle geometryDiagrams=[ultra thick,color=blue!50!black]


\usetikzlibrary{arrows}
\tikzset{>=stealth,commutative diagrams/.cd,
  arrow style=tikz,diagrams={>=stealth}} %% cool arrow head
\tikzset{shorten <>/.style={ shorten >=#1, shorten <=#1 } } %% allows shorter vectors

\usetikzlibrary{backgrounds} %% for boxes around graphs
\usetikzlibrary{shapes,positioning}  %% Clouds and stars
\usetikzlibrary{matrix} %% for matrix
\usepgfplotslibrary{polar} %% for polar plots
\usepgfplotslibrary{fillbetween} %% to shade area between curves in TikZ



%\usepackage[width=4.375in, height=7.0in, top=1.0in, papersize={5.5in,8.5in}]{geometry}
%\usepackage[pdftex]{graphicx}
%\usepackage{tipa}
%\usepackage{txfonts}
%\usepackage{textcomp}
%\usepackage{amsthm}
%\usepackage{xy}
%\usepackage{fancyhdr}
%\usepackage{xcolor}
%\usepackage{mathtools} %% for pretty underbrace % Breaks Ximera
%\usepackage{multicol}



\newcommand{\RR}{\mathbb R}
\newcommand{\R}{\mathbb R}
\newcommand{\C}{\mathbb C}
\newcommand{\N}{\mathbb N}
\newcommand{\Z}{\mathbb Z}
\newcommand{\dis}{\displaystyle}
%\renewcommand{\d}{\,d\!}
\renewcommand{\d}{\mathop{}\!d}
\newcommand{\dd}[2][]{\frac{\d #1}{\d #2}}
\newcommand{\pp}[2][]{\frac{\partial #1}{\partial #2}}
\renewcommand{\l}{\ell}
\newcommand{\ddx}{\frac{d}{\d x}}

\newcommand{\zeroOverZero}{\ensuremath{\boldsymbol{\tfrac{0}{0}}}}
\newcommand{\inftyOverInfty}{\ensuremath{\boldsymbol{\tfrac{\infty}{\infty}}}}
\newcommand{\zeroOverInfty}{\ensuremath{\boldsymbol{\tfrac{0}{\infty}}}}
\newcommand{\zeroTimesInfty}{\ensuremath{\small\boldsymbol{0\cdot \infty}}}
\newcommand{\inftyMinusInfty}{\ensuremath{\small\boldsymbol{\infty - \infty}}}
\newcommand{\oneToInfty}{\ensuremath{\boldsymbol{1^\infty}}}
\newcommand{\zeroToZero}{\ensuremath{\boldsymbol{0^0}}}
\newcommand{\inftyToZero}{\ensuremath{\boldsymbol{\infty^0}}}


\newcommand{\numOverZero}{\ensuremath{\boldsymbol{\tfrac{\#}{0}}}}
\newcommand{\dfn}{\textbf}
%\newcommand{\unit}{\,\mathrm}
\newcommand{\unit}{\mathop{}\!\mathrm}
%\newcommand{\eval}[1]{\bigg[ #1 \bigg]}
\newcommand{\eval}[1]{ #1 \bigg|}
\newcommand{\seq}[1]{\left( #1 \right)}
\renewcommand{\epsilon}{\varepsilon}
\renewcommand{\iff}{\Leftrightarrow}

\DeclareMathOperator{\arccot}{arccot}
\DeclareMathOperator{\arcsec}{arcsec}
\DeclareMathOperator{\arccsc}{arccsc}
\DeclareMathOperator{\si}{Si}
\DeclareMathOperator{\proj}{proj}
\DeclareMathOperator{\scal}{scal}
\DeclareMathOperator{\cis}{cis}
\DeclareMathOperator{\Arg}{Arg}
%\DeclareMathOperator{\arg}{arg}
\DeclareMathOperator{\Rep}{Re}
\DeclareMathOperator{\Imp}{Im}
\DeclareMathOperator{\sech}{sech}
\DeclareMathOperator{\csch}{csch}
\DeclareMathOperator{\Log}{Log}

\newcommand{\tightoverset}[2]{% for arrow vec
  \mathop{#2}\limits^{\vbox to -.5ex{\kern-0.75ex\hbox{$#1$}\vss}}}
\newcommand{\arrowvec}{\overrightarrow}
\renewcommand{\vec}{\mathbf}
\newcommand{\veci}{{\boldsymbol{\hat{\imath}}}}
\newcommand{\vecj}{{\boldsymbol{\hat{\jmath}}}}
\newcommand{\veck}{{\boldsymbol{\hat{k}}}}
\newcommand{\vecl}{\boldsymbol{\l}}
\newcommand{\utan}{\vec{\hat{t}}}
\newcommand{\unormal}{\vec{\hat{n}}}
\newcommand{\ubinormal}{\vec{\hat{b}}}

\newcommand{\dotp}{\bullet}
\newcommand{\cross}{\boldsymbol\times}
\newcommand{\grad}{\boldsymbol\nabla}
\newcommand{\divergence}{\grad\dotp}
\newcommand{\curl}{\grad\cross}
%% Simple horiz vectors
\renewcommand{\vector}[1]{\left\langle #1\right\rangle}


\outcome{In this section we define the cross product and we use it to create orthogonal vectors.}

\title{1.5 The Cross Product}



\begin{document}

\begin{abstract}
In this section we define the cross product and we use it to create orthogonal vectors.
\end{abstract}
 
\maketitle
The cross product is a special operation that helps us to create a vector that is orthogonal to two given vectors in $\R^3$.
\begin{definition}[Cross Product in $\R^3$]
Let $\vec{v}_1$ and $\vec{v}_2$ be vectors in $\R^3$ with components:
\[
\vec{v}_1 = \vector{x_1, y_1, z_1} \text{  and   } \;\vec{v}_2 = \vector{x_2, y_2, z_2}
\]
The \textbf{cross product} is defined by
\[
\vec{v}_1 \cross \vec{v}_2 =  (y_1z_2 - z_1y_2) \vec{i} + (z_1x_2 - x_1z_2) \vec{j} + (x_1y_2 - y_1x_2) \vec{k} 
\]
\end{definition}

The definition of the cross product of two vectors is easier to understand in the context of \textbf{matrix determinants}.

\begin{definition}[Determinant of a $2 \times 2$ Matrix]
The determinant of the $2 \times 2$ matrix 
\[
\begin{bmatrix}
a & b\\
c & d
\end{bmatrix}
\]
is  given by
\[
\begin{vmatrix}
a & b\\
c & d
\end{vmatrix}
= ad-bc
\]
Note that the determinant of a matrix is a number.
\end{definition}

\begin{definition}[Determinant of $3 \times 3$ Matrix]
The determinant of a $3 \times 3$ matrix is given in terms of determinants of $2 \times 2$ sub-matrices:
\[
\begin{vmatrix}
a & b & c\\
d & e & f\\
g & h & i
\end{vmatrix}
= a \cdot
\begin{vmatrix}
 e & f\\
 h & i
\end{vmatrix}
-b\cdot
\begin{vmatrix}
d  & f\\
g  & i
\end{vmatrix}
+c \cdot
\begin{vmatrix}
d & e \\
g & h 
\end{vmatrix}
\]
\[
= a(ei-fh) - b(di-fg) + c(dh-eg)
\]
Note that the determinant of a matrix is a number. 
Also note that each sub-matrix is obtained from the original matrix by removing the row and column of its coefficient, $a, b$ or $c$. 
\end{definition}

 We can use the definition of the determinant of a $3 \times 3$ matrix to compute the cross product of two 
 vectors by replacing the numbers in the top row with the standard basis vectors.
 
 \begin{proposition}[Cross Product] 
 Let  $\vec{v_1} = \vector{x_1, y_1, z_1}$ and $\vec{v_2} = \vector{x_2, y_2, z_2}$ be vectors in $\R^3$. 
 Then the cross product is given by
 \[
\vec{v_1} \cross \vec{v_2} = 
 \begin{vmatrix}
\vec{i} & \vec{j} & \vec{k}\\
x_1 & y_1 & z_1\\
x_2 & y_2 & z_2
\end{vmatrix} 
\]

\end{proposition}


\begin{proof}
\[
 \begin{vmatrix}
\vec{i} & \vec{j} & \vec{k}\\
x_1 & y_1 & z_1\\
x_2 & y_2 & z_2
\end{vmatrix} 
= \begin{vmatrix}
 y_1 & z_1  \\
 y_2 & z_2 \\
 \end{vmatrix} \cdot \vec i
 - \begin{vmatrix}
 x_1 & z_1  \\
 x_2 & z_2 \\
 \end{vmatrix} \cdot \vec j
 + \begin{vmatrix}
 x_1 & y_1  \\
 x_2 & y_2 \\
 \end{vmatrix} \cdot \vec k
 \]
 \begin{align*}
 &= (y_1z_2 - z_1y_2) \vec{i} - (x_1z_2-z_1x_2) \vec{j} + (x_1y_2 - y_1x_2) \vec{k}\\
 &=(y_1z_2 - z_1y_2) \vec{i} + (z_1x_2 - x_1z_2) \vec{j} + (x_1y_2 - y_1x_2) \vec{k}
 \end{align*}
\[
 = \vec{v_1} \cross \vec{v_2}
\]
\end{proof}

\begin{example}[Example 1]
Let $\vec{v_1} = \vector{1,2,3}$ and $\vec{v_2} = \vector{4, 5, 6}$.  
Compute the cross product: $\vec{v_1} \cross \vec{v_2}$\\
\begin{align*}
\vec{v_1} \cross \vec{v_2} &= \begin{vmatrix}
\vec{i} & \vec{j} & \vec{k}\\
1 & 2 & 3\\
4 & 5 & 6
\end{vmatrix}\\ 
&= \begin{vmatrix}
 2 & 3  \\
 5 & 6 \\
 \end{vmatrix} \cdot \vec i
 - \begin{vmatrix}
 1 & 3  \\
 4 & 6 \\
 \end{vmatrix} \cdot \vec j
 + \begin{vmatrix}
 1 & 2  \\
 4 & 5 \\
 \end{vmatrix} \cdot \vec k\\
 &= (12-15)\vec i - (6 - 12) \vec j + (5-8) \vec k\\
 &= -3\vec i + 6 \vec j -3 \vec k\\
 &= \vector{-3, 6, -3}
 \end{align*}
 \end{example}

\begin{problem}(Problem 1)
Let $\vec{v_1} = \vector{1,2,3}$ and $\vec{v_2} = \vector{4, 5, 6}$.\\  
Compute the cross product: 
\[
\vec{v_2} \cross \vec{v_1} = \vector{\answer{3}, \answer{-6}, \answer{3}}
\]
\end{problem}

\begin{proposition}[Orthogonality] 
Let  $\vec{v_1} = \vector{x_1, y_1, z_1}$ and $\vec{v_2} = \vector{x_2, y_2, z_2}$ be vectors in $\R^3$. 
The vector $\vec{v_1}\cross \vec{v_2}$ is orthogonal to both $\vec{v_1}$ and $\vec{v_2}$. 
\end{proposition}
\begin{proof} We will show that 
\[
 \vec{v_1} \dotp \left(\vec{v_1} \cross \vec{v_2} \right) = 0
\]
and leave it to the reader to show that
\[
\vec{v_2} \dotp \left(\vec{v_1} \cross \vec{v_2} \right) = 0
\]
We have:
\begin{align*}
\vec{v_1} \dotp \left(\vec{v_1} \cross \vec{v_2} \right)  
                            &= \vector{x_1, y_1, z_1}\dotp \vector{y_1z_2 - z_1y_2, z_1x_2 - x_1z_2, x_1y_2 - y_1x_2} \\
                                                &= x_1(y_1z_2 - z_1y_2) + y_1(z_1x_2 - x_1z_2) + z_1(x_1y_2 - y_1x_2)\\
                                                &= x_1y_1z_2 - x_1y_2z_1 + x_2y_1z_1 - x_1y_1z_2 + x_1y_2z_1 - x_2y_1z_1\\
                                                &= x_1y_1z_2 - x_1y_1z_2 + x_1y_2z_1 - x_1y_2z_1 + x_2y_1z_1 - x_2y_1z_1\\
                                                &= 0
\end{align*}
Hence, $\vec{v_1} \cross \vec{v_2}$ and $\vec{v_1}$ are orthogonal.  
The reader should perform a similar computation to verify that the vectors
$\vec{v_1} \cross \vec{v_2}$ and $\vec{v_2}$ are also orthogonal.
\end{proof}
In example 1, we found that
\[
\vector{1, 2, 3} \cross \vector{4, 5, 6} = \vector{-3, 6, -3}
\]
According to the above proposition, the vector $\vector{-3, 6, -3}$ should be orthogonal to both $\vector{1, 2, 3}$ and $\vector{4, 5, 6}$.
We can verify this using the dot product:
\[
\vector{1, 2, 3} \dotp \vector{-3, 6, -3} = (1)(-3) +(2)(6)+(3)(-3) = -3+12-9 = 0
\]
and
\[
\vector{4, 5, 6} \dotp \vector{-3, 6, -3} = (4)(-3) +(5)(6)+(6)(-3) = -12 + 30 -18 = 0
\]
Hence, the vector $\vector{-3, 6, -3}$ is orthogonal to each of the vectors $\vector{1, 2, 3}$ and $\vector{4, 5, 6}$ as claimed in the proposition.
Now we turn to a fact about parallel vectors.
Recall that parallel vectors are scalar multiples of each other.
\begin{proposition}[Parallel Vectors]
Let  $\vec{v_1} = \vector{x_1, y_1, z_1}$ and $\vec{v_2} = \vector{x_2, y_2, z_2}$ be vectors in $\R^3$.
If $\vec{v_1}$ and $\vec{v_2}$ are parallel then 
\[
\vec{v_1} \cross \vec{v_2} = \vec 0
\]
\end{proposition}
\begin{proof}
Since $\vec{v_1}$ and $\vec{v_2}$ are parallel, we can assume that there exists a scalar $`c'$ such that 
\[
\vec{v_2} = c\vec{v_1}
\]
Computing the cross product gives
\begin{align*}
\vec{v_1} \cross \vec{v_2} &= \vec{v_1} \cross c\vec{v_1}
                         =  \begin{vmatrix}
                                  \vec{i} & \vec{j} & \vec{k}\\
                                      x_1 & y_1 & z_1\\
                                       cx_1 & cy_1 & cz_1
                            \end{vmatrix} \\
&= \begin{vmatrix}
 y_1 & z_1  \\
 cy_1 & cz_1 \\
 \end{vmatrix} \cdot \vec i
 - \begin{vmatrix}
 x_1 & z_1  \\
 cx_1 & cz_1 \\
 \end{vmatrix} \cdot \vec j
 + \begin{vmatrix}
 x_1 & y_1  \\
 cx_1 & cy_1 \\
 \end{vmatrix} \cdot \vec k\\
 &= (y_1cz_1 - z_1cy_1) \vec{i} - (x_1cz_1-z_1cx_1) \vec{j} + (x_1cy_1 - y_1cx_1) \vec{k}\\
 &=0 \vec{i} + 0 \vec{j} + 0 \vec{k}\\
 &= \vec 0
 \end{align*}
 \end{proof}
Since any vector is parallel to itself, we have the following corollary.
\begin{corollary}
\[
\vec v \cross \vec v = \vec 0
\]
\end{corollary}

We now list some basic properties of the cross product.\\

\begin{enumerate}
\item The cross product is not commutative.  In fact, it is anti-commutative:
\[
\vec{v} \cross \vec{w} = -(\vec{w} \cross \vec{v})
\]
\item Scalars can factor out of a cross product:
\[
(c\vec{v}) \cross \vec{w} = c(\vec{v} \cross \vec{w})
\]
and
\[
\vec{v} \cross (c\vec{w}) = c(\vec{v} \cross \vec{w})
\]
\item The cross product distributes over vector addition 
(note that it is important to maintain the order of the vectors in the cross product due to its anticommutativity):
\[
\vec{u} \cross (\vec{v} + \vec{w}) = (\vec{u} \cross \vec{v}) + (\vec{u} \cross \vec{w})\quad \text{Left Distributive Property}
\]
and
\[
(\vec{u} + \vec{v}) \cross \vec{w} = (\vec{u} \cross \vec{w}) + (\vec{v} \cross \vec{w})\quad \text{Right Distributive Property}
\]
\end{enumerate}

Cross products of the standard basis vectors are useful:
\[
\vec{i} \cross \vec{j}=
 \begin{vmatrix}
\vec{i} & \vec{j} & \vec{k}\\
1 & 0 & 0\\
0 & 1 & 0
\end{vmatrix}
= (0)\vec{i} - (0)\vec{j} + (1)\vec{k} = \vec{k}
\]
\[
\vec{j} \cross \vec{k}=
 \begin{vmatrix}
\vec{i} & \vec{j} & \vec{k}\\
0 & 1 & 0\\
0 & 0 & 1
\end{vmatrix}
= (1)\vec{i} - (0)\vec{j} + (0)\vec{k} = \vec{i}
\]
and
\[
\vec{k} \cross \vec{i}=
 \begin{vmatrix}
\vec{i} & \vec{j} & \vec{k}\\
0 & 0 & 1\\
1 & 0 & 0
\end{vmatrix}
= (0)\vec{i} - (-1)\vec{j} + (1)\vec{k} = \vec{j}
\]
These results can be obtained with the aide of the following figure.
\begin{image}
\begin{tikzpicture}
%\draw (0,0) circle (3);
\node at (0,3.5){$\vec{i}$};
\node at (3.2,-1.2){$\vec{j}$};
\node at (-3.2,-1.2){$\vec{k}$};
\draw[-<] (3,0) arc (0:30:3);
\draw[-<] (3,0) arc (0:150:3);
\draw[-<] (3,0) arc (0:270:3);
\draw (3,0) arc (0:359.9:3);
\node at (0,1){$\vec{i} \cross \vec{j} = \vec{k}$};
\node at (0,0){$\vec{j} \cross \vec{k} = \vec{i}$};
\node at (0,-1){$\vec{k} \cross \vec{i} = \vec{j}$};
\node at (0, -3.5) {Follow the circle clockwise to compute the cross products};
\end{tikzpicture}
\end{image}

\begin{remark}
Since the cross product is anticommutative, the figure above yields the following additional results:
\begin{align*}
\vec{i} \cross \vec{k} &= \vec{-j}\\
\vec{k} \cross \vec{j} &= \vec{-i}\\
\vec{j} \cross \vec{i} &= \vec{-k}
\end{align*}
\end{remark}

\begin{example}[Example 2]
Compute the cross product: $(2\vec{i} - 3\vec{k}) \cross (\vec{j} + 4\vec{k})$.\\
Using the properties listed above and the results of the cross products involving the standard basis vectors, we have:
\begin{align*}
(2\vec{i} - 3\vec{k}) \cross (\vec{j} + 4\vec{k}) & = [(2\vec{i} - 3\vec{k}) \cross \vec{j}] + [(2\vec{i} - 3\vec{k}) \cross   4\vec{k}]\\
                                  &= (2\vec{i} \cross \vec{j}) + [(-3\vec{k}) \cross \vec{j})] +(2\vec{i} \cross   4\vec{k}) + [ (- 3\vec{k}) \cross   4\vec{k}]\\
                                  &= 2\vec{k} + 3 \vec{i} -8\vec{j} + \vec{0}\\
                                  &= 3\vec{i} - 8\vec{j} +2\vec{k}\\
                                  &= \vector{3, -8, 2}
\end{align*}
\end{example}

\begin{problem}(Problem 2)
Use the properties of the cross product to compute: 
\[
(\vec{i} + 2\vec{j}) \cross (3\vec{i} - \vec{k}) = \vector{\answer{-2},\answer{1},\answer{-6}}
\]
\end{problem}



The magnitude of a cross product is of great interest in applications.
\begin{proposition}
Let $\vec{v}$ and $\vec{w}$ be vectors in $\R^3$ (or $\R^2$). Then the magnitude of their cross product is given by
\[
|\vec{v} \cross \vec{w}| = |\vec{v}| \cdot |\vec{w}| \sin \theta
\]
where $\theta$ is the angle between $\vec{v}$ and $\vec{w}$.
\end{proposition}

Two vectors $\vec{v}$ and $\vec{w}$ generate a parallelogram and the area of this parallelogram is given by then magnitude of their cross product:
\[
A = |\vec{v} \cross \vec{w}|
\]

\begin{image}
\begin{tikzpicture}
\draw[blue, ->, thick] (0,0) -- (6, 0) node[below right]{$\vec{v}$};
\draw[red, ->, thick] (0,0) -- (4, 2) node[above left]{$\vec{w}$};
\draw[red, dashed, thin] (6,0) -- (10,2) ;
\draw[blue, dashed, thin] (4,2) -- (10,2) ;
\node at (5.5, -1){Parallelogram generated by $\vec{v}$ and $\vec{w}$ with area $A = |\vec{v} \cross \vec{w}|$};
\end{tikzpicture}
\end{image}

\begin{example}[Example 3]
Find the area of the parallelogram determined by the vectors $\vec v = \vector{2, -1, -3}$ and $\vec w = \vector{4, 5, -1}$\\
To find the area of this parallelogram we first compute the cross product of $\vec v$ and $\vec w$:
\begin{align*}
\vec v \cross \vec w &= \vector{2, -1, -3} \cross \vector{4, 5, -1}\\
                    &=  \begin{vmatrix}
\vec{i} & \vec{j} & \vec{k}\\
2 & -1 & 3\\
4 & 5 & -1
\end{vmatrix}\\
&= (-14)\vec{i} - (-14)\vec{j} + (14)\vec{k}\\
&= 14 \vector{-1, 1, 1}
\end{align*}
Now, the area of the parallelogram is the magnitude of this cross product:
\[
\text{Area} \; = |14\vector{-1, 1, 1}| = 14\sqrt 3
\]
\end{example}

\begin{problem}(Problem 3a) The points $P(1,0,2), Q(3,3,3), R(7, 5, 8)$ and $S(5, 2, 7)$ form a parallelogram in $\R^3$.
This parallelogram is generated by either the vector pair $\avec{PQ}, \avec{PR}$ or the vector pair $\avec{PQ}, \avec{PS}$.
Calculate the areas of the parallelograms created by these vector pairs.\\
\begin{hint}
The vector from the point $(x_1, y_1, z_1)$ to the point $(x_2, y_2, z_2)$ is given by $\vector{x_2-x_1, y_2-y_1, z_2-z_1}$
\end{hint}
Area of parallelogram determined by $\avec{PQ}$ and $\avec{PR}$  $=\answer{\sqrt{269}}$\\
Area of parallelogram determined by $\avec{PQ}$ and $\avec{PS}$ $=\answer{\sqrt{269}}$\\
Remarkably, these answers are equal!  This is due to \link[Cavalieri's Principle]{https://en.wikipedia.org/wiki/Cavalieri's_principle}.
\end{problem}



We can also use the cross product to find the area of a triangle with sides given by the vectors $\vec{v}, \vec{w}$ and $\vec{v}-\vec{w}$. 
This is because the area of this triangle is half of the area of the parallelogram generated by $\vec{v}$ and $\vec{w}$.

\begin{image}
\begin{tikzpicture}
\draw[blue, ->, thick] (0,0) -- (6, 0) node[midway, below]{$\vec{v}$};
\draw[red, ->, thick] (0,0) -- (4, 2) node[midway, above left]{$\vec{w}$};
\draw[brown!50!black, <-, thick] (6,0) -- (4,2) node[midway, above right]{$\vec{v}-\vec{w}$};
\node at (3, -1){Triangle with sides $\vec{v}, \vec{w}$ and $\vec{v} - \vec{w}$ with area $A = \frac12|\vec{v} \cross \vec{w}|$};
\end{tikzpicture}
\end{image}

\begin{problem}(Problem 3b)
Find the area of the triangle in $\R^3$ with vertices $P(0,0,-3), Q(4,2,0)$ and $R(3,3,1)$.\\
Area of triangle  $=\answer{1/2 \sqrt{86}}$
\end{problem}

Three vectors in $\R^3$ can generate a three dimensional analogue of a parallelogram. 
A parallelepiped is a three dimensional figure with 6 sides, each of which is a parallelogram.
The sides of these parallelograms can be determined using three vectors.
Given $\vec{u}, \vec{v}$ and $\vec{w}$
in $\R^3$ which generate a parallelepiped, the volume of this parallelepiped is given by
\[
V = |\vec{u} \dotp \left(\vec{v} \cross \vec{w}\right)|
\]


\begin{image}
\begin{tikzpicture}
\draw[blue, ->, thick] (0,0) -- (6, 0) node[midway, below]{$\vec{v}$};
\draw[brown!30!black, ->, thick] (0,0) -- (4, 2) node[midway, above left]{$\vec{w}$};
\draw[red, ->, thick] (0,0) -- (1, 5) node[midway, above left]{$\vec{u}$};
\draw[red, dashed, thin] (6,0) -- (7,5) (4,2)--(5,7) (10,2) -- (11,7);
\draw[blue, dashed, thin] (4,2) -- (10,2) (1,5) -- (7,5) (5,7) -- (11,7);
\draw[brown!30!black, dashed, thin] (6,0)--(10,2) (1,5)--(5,7) (7,5)--(11,7);
\node at (5.5, -1){Parallelepiped generated by $\vec{u}, \vec{v}$ and $\vec{w}$ with volume $V = |\vec{u} \dotp \left(\vec{v} \cross \vec{w}\right)|$};
\end{tikzpicture}
\end{image}

\begin{problem}(Problem 3c)
Find the volume of the parallelepiped generated by the vectors $\vec u = \vector{2,-4,-5}, \vec v = \vector{3,0,-1}$ and $\vec w = \vector{-6,2,3}$\\
Volume of parallelepiped $= \answer{14}$
\end{problem}

\end{document}

