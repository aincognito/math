\documentclass[handout]{ximera}

%% You can put user macros here
%% However, you cannot make new environments



\newcommand{\ffrac}[2]{\frac{\text{\footnotesize $#1$}}{\text{\footnotesize $#2$}}}
\newcommand{\vasymptote}[2][]{
    \draw [densely dashed,#1] ({rel axis cs:0,0} -| {axis cs:#2,0}) -- ({rel axis cs:0,1} -| {axis cs:#2,0});
}


\graphicspath{{./}{firstExample/}}
\usepackage{forest}
\usepackage{amsmath}
\usepackage{amssymb}
\usepackage{array}
\usepackage[makeroom]{cancel} %% for strike outs
\usepackage{pgffor} %% required for integral for loops
\usepackage{tikz}
\usepackage{tikz-cd}
\usepackage{tkz-euclide}
\usetikzlibrary{shapes.multipart}


%\usetkzobj{all}
\tikzstyle geometryDiagrams=[ultra thick,color=blue!50!black]


\usetikzlibrary{arrows}
\tikzset{>=stealth,commutative diagrams/.cd,
  arrow style=tikz,diagrams={>=stealth}} %% cool arrow head
\tikzset{shorten <>/.style={ shorten >=#1, shorten <=#1 } } %% allows shorter vectors

\usetikzlibrary{backgrounds} %% for boxes around graphs
\usetikzlibrary{shapes,positioning}  %% Clouds and stars
\usetikzlibrary{matrix} %% for matrix
\usepgfplotslibrary{polar} %% for polar plots
\usepgfplotslibrary{fillbetween} %% to shade area between curves in TikZ



%\usepackage[width=4.375in, height=7.0in, top=1.0in, papersize={5.5in,8.5in}]{geometry}
%\usepackage[pdftex]{graphicx}
%\usepackage{tipa}
%\usepackage{txfonts}
%\usepackage{textcomp}
%\usepackage{amsthm}
%\usepackage{xy}
%\usepackage{fancyhdr}
%\usepackage{xcolor}
%\usepackage{mathtools} %% for pretty underbrace % Breaks Ximera
%\usepackage{multicol}



\newcommand{\RR}{\mathbb R}
\newcommand{\R}{\mathbb R}
\newcommand{\C}{\mathbb C}
\newcommand{\N}{\mathbb N}
\newcommand{\Z}{\mathbb Z}
\newcommand{\dis}{\displaystyle}
%\renewcommand{\d}{\,d\!}
\renewcommand{\d}{\mathop{}\!d}
\newcommand{\dd}[2][]{\frac{\d #1}{\d #2}}
\newcommand{\pp}[2][]{\frac{\partial #1}{\partial #2}}
\renewcommand{\l}{\ell}
\newcommand{\ddx}{\frac{d}{\d x}}

\newcommand{\zeroOverZero}{\ensuremath{\boldsymbol{\tfrac{0}{0}}}}
\newcommand{\inftyOverInfty}{\ensuremath{\boldsymbol{\tfrac{\infty}{\infty}}}}
\newcommand{\zeroOverInfty}{\ensuremath{\boldsymbol{\tfrac{0}{\infty}}}}
\newcommand{\zeroTimesInfty}{\ensuremath{\small\boldsymbol{0\cdot \infty}}}
\newcommand{\inftyMinusInfty}{\ensuremath{\small\boldsymbol{\infty - \infty}}}
\newcommand{\oneToInfty}{\ensuremath{\boldsymbol{1^\infty}}}
\newcommand{\zeroToZero}{\ensuremath{\boldsymbol{0^0}}}
\newcommand{\inftyToZero}{\ensuremath{\boldsymbol{\infty^0}}}


\newcommand{\numOverZero}{\ensuremath{\boldsymbol{\tfrac{\#}{0}}}}
\newcommand{\dfn}{\textbf}
%\newcommand{\unit}{\,\mathrm}
\newcommand{\unit}{\mathop{}\!\mathrm}
%\newcommand{\eval}[1]{\bigg[ #1 \bigg]}
\newcommand{\eval}[1]{ #1 \bigg|}
\newcommand{\seq}[1]{\left( #1 \right)}
\renewcommand{\epsilon}{\varepsilon}
\renewcommand{\iff}{\Leftrightarrow}

\DeclareMathOperator{\arccot}{arccot}
\DeclareMathOperator{\arcsec}{arcsec}
\DeclareMathOperator{\arccsc}{arccsc}
\DeclareMathOperator{\si}{Si}
\DeclareMathOperator{\proj}{proj}
\DeclareMathOperator{\scal}{scal}
\DeclareMathOperator{\cis}{cis}
\DeclareMathOperator{\Arg}{Arg}
%\DeclareMathOperator{\arg}{arg}
\DeclareMathOperator{\Rep}{Re}
\DeclareMathOperator{\Imp}{Im}
\DeclareMathOperator{\sech}{sech}
\DeclareMathOperator{\csch}{csch}
\DeclareMathOperator{\Log}{Log}

\newcommand{\tightoverset}[2]{% for arrow vec
  \mathop{#2}\limits^{\vbox to -.5ex{\kern-0.75ex\hbox{$#1$}\vss}}}
\newcommand{\arrowvec}{\overrightarrow}
\renewcommand{\vec}{\mathbf}
\newcommand{\veci}{{\boldsymbol{\hat{\imath}}}}
\newcommand{\vecj}{{\boldsymbol{\hat{\jmath}}}}
\newcommand{\veck}{{\boldsymbol{\hat{k}}}}
\newcommand{\vecl}{\boldsymbol{\l}}
\newcommand{\utan}{\vec{\hat{t}}}
\newcommand{\unormal}{\vec{\hat{n}}}
\newcommand{\ubinormal}{\vec{\hat{b}}}

\newcommand{\dotp}{\bullet}
\newcommand{\cross}{\boldsymbol\times}
\newcommand{\grad}{\boldsymbol\nabla}
\newcommand{\divergence}{\grad\dotp}
\newcommand{\curl}{\grad\cross}
%% Simple horiz vectors
\renewcommand{\vector}[1]{\left\langle #1\right\rangle}


\outcome{In this section we define the cross product and we use it to create orthogonal vectors.}

\title{1.5 The Cross Product}



\begin{document}

\begin{abstract}
In this section we define the cross product and we use it to create orthogonal vectors.
\end{abstract}
 
\maketitle
The cross product is a special operation that helps us to create a vector that is orthogonal to two given two vectors in $\R^3$.
\begin{definition}[Cross Product in $\R^3$]
If $\vec{v}_1$ and $\vec{v}_2$ are vectors in $\R^3$ given by
\[
\vec{v}_1 = \vector{x_1, y_1, z_1} \text{  and   } \;\vec{v}_2 = \vector{x_2, y_2, z_2}
\]
then the cross product $\vec{v}_1 \cross \vec{v}_2$ is defined by
\[
\vec{v}_1 \cross \vec{v}_2 =  (y_1z_2 - z_1y_2) \vec{i} + (z_1x_2 - x_1z_2) \vec{j} + (x_1y_2 - y_1x_2) \vec{k} 
\]
\end{definition}

The definition of the cross product of two vectors is easier to understand in the context of matrix determinants.

\begin{definition}[Determinant of $2 \times 2$ Matrix]
The determinant of the $2 \times 2$ matrix 
\[
\begin{bmatrix}
a & b\\
c & d
\end{bmatrix}
\]
is the number given by
\[
\begin{vmatrix}
a & b\\
c & d
\end{vmatrix}
= ad-bc
\]

\end{definition}

\begin{definition}[Determinant of $3 \times 3$ Matrix]
The determinant of the $3 \times 3$ matrix 
\[
\begin{bmatrix}
a & b & c\\
d & e & f\\
g & h & i
\end{bmatrix}
\]
is the number given by
\[
\begin{vmatrix}
a & b & c\\
d & e & f\\
g & h & i
\end{vmatrix}
= a(ei-fh) - b(di-fg) + c(dh-eg)
\]
Note that the quantities in the parentheses in the above definition are the determinants of the $2 \times 2$ matrices
obtained by removing the first row and a column (the column of the coefficient of the quantity in the parentheses) from the original $3 \times 3$ matrix.
\end{definition}
 Now to create the cross product of two vectors, we use the determinant of a $3 \times 3$ matrix with the vectors $\vec{i}, \vec{j}$ and $\vec{k}$ in the first row.
 So, if $\vec{v} = \vector{x_1, y_1, z_1}$ and $\vec{w} = \vector{x_2, y_2, z_2}$ then the cross product is given by
 \[
\vec{v} \cross \vec{w} = 
 \begin{vmatrix}
\vec{i} & \vec{j} & \vec{k}\\
x_1 & y_1 & z_1\\
x_2 & y_2 & z_2
\end{vmatrix} = (y_1z_2 - z_1y_2)\vec{i} - (x_1z_2 - z_1x_2)\vec{j} + (x_1y_2-y_1x_2)\vec{k}
\]
which agrees with the definition above after multiplying through by $-1$ in the middle term.

The vector $\vec{v}\cross \vec{w}$ is orthogonal to both $\vec{v}$ and $\vec{w}$. To see this we compute the dot product:
\begin{align*}
\left(\vec{v} \cross \vec{w} \right) \dotp \vec{v} &= \vector{y_1z_2 - z_1y_2, z_1x_2 - x_1z_2, x_1y_2 - y_1x_2} \dotp \vector{x_1, y_1, z_1}\\
                                                &= x_1(y_1z_2 - z_1y_2) + y_1(z_1x_2 - x_1z_2) + z_1(x_1y_2 - y_1x_2)\\
                                                &= x_1y_1z_2 - x_1y_2z_1 + x_2y_1z_1 - x_1y_1z_2 + x_1y_2z_1 - x_2y_1z_1\\
                                                &= x_1y_1z_2 - x_1y_1z_2 + x_1y_2z_1 - x_1y_2z_1 + x_2y_1z_1 - x_2y_1z_1\\
                                                &= 0
\end{align*}
Hence, $\vec{v} \cross \vec{w}$ and $\vec{v}$ are orthogonal.  The reader should perform a similar computation to verify that
$\vec{v} \cross \vec{w}$ and $\vec{w}$ are also orthogonal.\\
The cross product has a variety of interesting properties which we now briefly explore.\\
The cross product of a vector with itself gives the zero vector:
\[
\vec{v} \cross \vec{v} = \vec{0}
\]

The cross product operation is not commutative.  In fact, it is anti-commutative:
\[
\vec{v} \cross \vec{w} = -(\vec{w} \cross \vec{v})
\]
Scalars can factor out of a cross product:
\[
(c\vec{v}) \cross \vec{w} = c(\vec{v} \cross \vec{w})
\]
and
\[
\vec{v} \cross (c\vec{w}) = c(\vec{v} \cross \vec{w})
\]
The cross product distributes over vector addition 
(note that it is important to maintain the order of the vectors in the cross product due to its anticommutativity):
\[
\vec{u} \cross (\vec{v} + \vec{w}) = (\vec{u} \cross \vec{v}) + (\vec{u} \cross \vec{w})\quad \text{Left Distributive Property}
\]
and
\[
(\vec{u} + \vec{v}) \cross \vec{w} = (\vec{u} \cross \vec{w}) + (\vec{v} \cross \vec{w})\quad \text{Right Distributive Property}
\]

Cross products involving the special basis vectors are useful and easy enough to compute from the definition:
\[
\vec{i} \cross \vec{j}=
 \begin{vmatrix}
\vec{i} & \vec{j} & \vec{k}\\
1 & 0 & 0\\
0 & 1 & 0
\end{vmatrix}
= (0)\vec{i} - (0)\vec{j} + (1)\vec{k} = \vec{k}
\]
\[
\vec{j} \cross \vec{k}=
 \begin{vmatrix}
\vec{i} & \vec{j} & \vec{k}\\
0 & 1 & 0\\
0 & 0 & 1
\end{vmatrix}
= (1)\vec{i} - (0)\vec{j} + (0)\vec{k} = \vec{i}
\]
and
\[
\vec{k} \cross \vec{i}=
 \begin{vmatrix}
\vec{i} & \vec{j} & \vec{k}\\
0 & 0 & 1\\
1 & 0 & 0
\end{vmatrix}
= (0)\vec{i} - (-1)\vec{j} + (1)\vec{k} = \vec{j}
\]
These results can be obtained with the aide of the following figure.
\begin{image}
\begin{tikzpicture}
%\draw (0,0) circle (3);
\node at (0,3.5){$\vec{i}$};
\node at (3.2,-1.2){$\vec{j}$};
\node at (-3.2,-1.2){$\vec{k}$};
\draw[-<] (3,0) arc (0:30:3);
\draw[-<] (3,0) arc (0:150:3);
\draw[-<] (3,0) arc (0:270:3);
\draw (3,0) arc (0:359.9:3);
\node at (0,1){$\vec{i} \cross \vec{j} = \vec{k}$};
\node at (0,0){$\vec{j} \cross \vec{k} = \vec{i}$};
\node at (0,-1){$\vec{k} \cross \vec{i} = \vec{j}$};
\node at (0, -3.5) {Follow the circle clockwise to compute the cross products};
\end{tikzpicture}
\end{image}

\begin{remark}
Since the cross product operation is anticommutative, we this example immediately yields the following additional results:
\begin{align*}
\vec{i} \cross \vec{k} &= \vec{-j}\\
\vec{k} \cross \vec{j} &= \vec{-i}\\
\vec{j} \cross \vec{i} &= \vec{-k}
\end{align*}
\end{remark}

\begin{example}
Compute the cross product: $(2\vec{i} - 3\vec{k}) \cross (\vec{j} + 4\vec{k})$.\\
Using the properties listed above and the results of the cross products involving the special basis vectors, we have:
\begin{align*}
(2\vec{i} - 3\vec{k}) \cross (\vec{j} + 4\vec{k}) & = [(2\vec{i} - 3\vec{k}) \cross \vec{j}] + [(2\vec{i} - 3\vec{k}) \cross   4\vec{k}]\\
                                  &= (2\vec{i} \cross \vec{j}) + [(-3\vec{k}) \cross \vec{j})] +(2\vec{i} \cross   4\vec{k}) + [ (- 3\vec{k}) \cross   4\vec{k}]\\
                                  &= 2\vec{k} + 3 \vec{i} -8\vec{j} + \vec{0}\\
                                  &= 3\vec{i} - 8\vec{j} +2\vec{k}\\
                                  &= \vector{3, -8, 2}
\end{align*}
\end{example}

\begin{problem}
Use the properties of the cross product and the cross products of the special basis vectors to compute: 
\[
(\vec{i} + 2\vec{j}) \cross (3\vec{i} - \vec{k}) = \answer{\vector{-2,1,-6}}
\]
\end{problem}



The magnitude of a cross product is of great interest in applications.
\begin{proposition}
Let $\vec{v}$ and $\vec{w}$ be vectors in $\R^3$ (or $\R^2$). Then the magnitude of their cross product is given by
\[
|\vec{v} \cross \vec{w}| = |\vec{v}| \cdot |\vec{w}| \sin \theta
\]
where $\theta$ is the angle between $\vec{v}$ and $\vec{w}$.
\end{proposition}

Two vectors $\vec{v}$ and $\vec{w}$ generate a parallelogram and the area of this parallelogram is given by then magnitude of their cross product:
\[
A = |\vec{v} \cross \vec{w}|
\]

\begin{image}
\begin{tikzpicture}
\draw[blue, ->, thick] (0,0) -- (6, 0) node[below right]{$\vec{v}$};
\draw[red, ->, thick] (0,0) -- (4, 2) node[above left]{$\vec{w}$};
\draw[red, dashed, thin] (6,0) -- (10,2) ;
\draw[blue, dashed, thin] (4,2) -- (10,2) ;
\node at (5.5, -1){Parallelogram generated by $\vec{v}$ and $\vec{w}$ with area $A = |\vec{v} \cross \vec{w}|$};
\end{tikzpicture}
\end{image}

This can be used to find the area of a triangle with sides $\vec{v}, \vec{w}$ and $\vec{v}-\vec{w}$ since the area of this 
triangle is simple half of the area of the parallelogram generated by $\vec{v}$ and $\vec{w}$.

\begin{image}
\begin{tikzpicture}
\draw[blue, ->, thick] (0,0) -- (6, 0) node[midway, below]{$\vec{v}$};
\draw[red, ->, thick] (0,0) -- (4, 2) node[midway, above left]{$\vec{w}$};
\draw[brown!50!black, <-, thick] (6,0) -- (4,2) node[midway, above right]{$\vec{v}-\vec{w}$};
\node at (3, -1){Triangle with sides $\vec{v}, \vec{w}$ and $\vec{v} - \vec{w}$ with area $A = \frac12|\vec{v} \cross \vec{w}|$};
\end{tikzpicture}
\end{image}


Moreover, three vectors in $\R^3$ can generate a three dimensional analogue of a parallelogram. 
A parallelepiped is a three dimensional figure with 6 sides, each of which is a parallelogram.
The sides of these parallelograms can be determined using three vectors.
Given $\vec{u}, \vec{v}$ and $\vec{w}$
in $\R^3$ which generate a parallelepiped, the volume of this parallelepiped is given by
\[
V = |\vec{u} \dotp \left(\vec{v} \cross \vec{w}\right)|
\]


\begin{image}
\begin{tikzpicture}
\draw[blue, ->, thick] (0,0) -- (6, 0) node[midway, below]{$\vec{v}$};
\draw[brown!30!black, ->, thick] (0,0) -- (4, 2) node[midway, above left]{$\vec{w}$};
\draw[red, ->, thick] (0,0) -- (1, 5) node[midway, above left]{$\vec{u}$};
\draw[red, dashed, thin] (6,0) -- (7,5) (4,2)--(5,7) (10,2) -- (11,7);
\draw[blue, dashed, thin] (4,2) -- (10,2) (1,5) -- (7,5) (5,7) -- (11,7);
\draw[brown!30!black, dashed, thin] (6,0)--(10,2) (1,5)--(5,7) (7,5)--(11,7);
\node at (5.5, -1){Parallelepiped generated by $\vec{u}, \vec{v}$ and $\vec{w}$ with volume $V = |\vec{u} \dotp \left(\vec{v} \cross \vec{w}\right)|$};
\end{tikzpicture}
\end{image}



\end{document}


The tangent and Inverse tangent graphs

\begin{image}
\begin{tikzpicture}
\draw (-3, 0) -- (3, 0);
\draw (0, -4) -- (0,4);
\draw[blue, thick] (-0.15, -0.05) -- (0.15, 0.05);
\draw[dashed, thin, blue] (-2, -3.5) -- (-2, 3.5) (2, -3.5) -- (2, 3.5);
\draw[blue, thick, <-] (-1.95, -3.5) .. controls (-1.6,-2.3) and (-0.7,-0.3) ..  (-0.15, -0.05); %add control
\draw[blue, thick, <-] (1.95, 3.5) .. controls (1.6,2.3) and (0.7,0.3) .. (0.15, 0.05); %add control
\node[blue] at (-1, 1) {$y = \tan x$};
\end{tikzpicture}
\end{image}

\begin{image}
\begin{tikzpicture}
\draw (-4, 0) -- (4, 0);
\draw (0, -3) -- (0,3);
\draw[red, thick] (-0.05, -0.15) -- (0.05, 0.15);
\draw[dashed, thin, red] (-3.5, -2) -- (3.5,-2) (-3.5, 2) -- (3.5, 2);
\draw[red, thick, <-] (-3.5, -1.95) .. controls (-2.3, -1.6) and (-0.3, -0.7) ..  (-0.05, -0.15); %add control
\draw[red, thick, <-] (3.5, 1.95) .. controls (2.3, 1.6) and (0.3, 0.7) .. (0.05, 0.15); %add control
\node[red] at (1, -1) {$y = \tan^{-1}x$};
\end{tikzpicture}
\end{image}

\begin{image}
\begin{tikzpicture}
\draw (-4, 0) -- (4, 0);
\draw (0, -4) -- (0,4);
\draw[blue, thick] (-0.15, -0.05) -- (0.15, 0.05);
\draw[dashed, thin, blue] (-2, -4) -- (-2, -2.2) (2, 2.2) -- (2, 4);
\draw[blue, thick, <-] (-1.95, -3.5) .. controls (-1.6,-2.3) and (-0.7,-0.3) ..  (-0.15, -0.05); %add control
\draw[blue, thick, <-] (1.95, 3.5) .. controls (1.6,2.3) and (0.7,0.3) .. (0.15, 0.05); %add control
\node[blue] at (1, 3) {$y = \tan x$};

\draw[red, thick] (-0.05, -0.15) -- (0.05, 0.15);
\draw[dashed, thin, red] (-4, -2) -- (-2.2,-2) (2.2, 2) -- (4, 2);
\draw[red, thick, <-] (-3.5, -1.95) .. controls (-2.3, -1.6) and (-0.3, -0.7) ..  (-0.05, -0.15); %add control
\draw[red, thick, <-] (3.5, 1.95) .. controls (2.3, 1.6) and (0.3, 0.7) .. (0.05, 0.15); %add control
\node[red] at (3, 1.3) {$y = \tan^{-1}x$};

%add hash marks to axes
\end{tikzpicture}
\end{image}

\begin{tikzpicture} 
\draw[help lines] (0,0) grid (8,3); \draw    (1,0)      .. controls (4,2) ..    (7,0);   
\end{tikzpicture}  

\begin{tikzpicture} 
\draw[help lines] (0,0) grid (8,3); \draw    (1,0)      .. controls (3,2) and (5,2) ..    (7,0);   
\end{tikzpicture} 
 
\begin{tikzpicture} \draw[help lines] (0,0) grid (8,3); \draw[overlay]    (1,0)      .. controls (10,2) and (-2,2) ..    (7,0);   
\end{tikzpicture}  
\begin{tikzpicture} \draw[help lines] (0,-1) grid (8,1); \draw    (1,0)      .. controls (3,2) and (5,-2) ..    (7,0);   
\end{tikzpicture} 





\begin{tikzpicture}
\draw (0,0) % [show curve controls] (0, 0) 
  .. controls ++(165:-1) and ++(240: 1) .. ( 3, 2)
  .. controls ++(240:-1) and ++(165:-1) .. ( 2, 4)
  .. controls ++(165: 1) and ++(175:-2) .. (-1, 2)
  .. controls ++(175: 2) and ++(165: 1) .. ( 0, 0);
\end{tikzpicture}

\begin{tikzpicture}
\draw (0,0) % [show curve controls] (0, 0) 
  .. controls ++(180:-1) and ++(270: 1) .. ( 2, 2)
  .. controls ++(90:1) and ++(180:-1) .. ( 0, 4)
  .. controls ++(180: 1) and ++(270:-1) .. (-2, 2)
  .. controls ++(270: 1) and ++(180: 1) .. ( 0, 0);
\end{tikzpicture}


\tikzset{
  show curve controls/.style={
    decoration={
      show path construction,
      curveto code={
        \draw[#1!50]
        (\tikzinputsegmentfirst)
        -- (\tikzinputsegmentsupporta)
        -- (\tikzinputsegmentsupportb)
        -- (\tikzinputsegmentlast)
        ;
        \fill[#1!50] (\tikzinputsegmentsupporta) circle(1pt);
        \fill[#1!50] (\tikzinputsegmentsupportb) circle(1pt);
        \draw[#1,line width=1pt]
        (\tikzinputsegmentfirst)
        .. controls (\tikzinputsegmentsupporta)
                and (\tikzinputsegmentsupportb) ..
        (\tikzinputsegmentlast);
      }
    },decorate
  }
}


\foreach \p in {0,10,...,360} {
  \begin{tikzpicture}
    \begin{scope}
      \path (-4,-2) rectangle (4,2.1);
      \coordinate (a) at (-2,0);
      \coordinate (b) at (2,0);
      \path (a) ++(1,0) ++(\p:0 and 2) coordinate (a1);
      \path (b) ++(-1,0) ++({180-\p}:0 and 2) coordinate (b1);
      \draw[show curve controls={red}] (a) .. controls (a1) and (b1) .. (b);
    \end{scope}
    \begin{scope}[yshift=-4.5cm]
      \path (-4,-1) rectangle (4,4);
      \coordinate (a) at (-2,0);
      \coordinate (b) at (2,0);
      \path (a) ++(45:3) ++(\p:3 and 0) coordinate (a1);
      \path (b) ++(90+45:3) ++(180-\p:3 and 0) coordinate (b1);
      \draw[show curve controls={blue}] (a) .. controls (a1) and (b1) .. (b);
    \end{scope}
    \begin{scope}[yshift=-6cm]
      \path (-4,-3) rectangle (4,4);
      \coordinate (a) at (-2,0);
      \coordinate (b) at (2,0);
      \path (a) ++(1,0) [rotate=45] ++(\p:0 and 2) coordinate (a1);
      \path (b) ++(-1,0) [rotate=45] ++({180+\p}:0 and 2) coordinate (b1);
      \draw[show curve controls={green!50!black}]
        (a) .. controls (a1) and (b1) .. (b);
    \end{scope}
    \begin{pgfonlayer}{background}
      \fill[white] (current bounding box.south west)
         rectangle (current bounding box.north east);
    \end{pgfonlayer}
  \end{tikzpicture}
}





