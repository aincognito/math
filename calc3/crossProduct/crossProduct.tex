\documentclass[handout]{ximera}

%% You can put user macros here
%% However, you cannot make new environments



\newcommand{\ffrac}[2]{\frac{\text{\footnotesize $#1$}}{\text{\footnotesize $#2$}}}
\newcommand{\vasymptote}[2][]{
    \draw [densely dashed,#1] ({rel axis cs:0,0} -| {axis cs:#2,0}) -- ({rel axis cs:0,1} -| {axis cs:#2,0});
}


\graphicspath{{./}{firstExample/}}
\usepackage{forest}
\usepackage{amsmath}
\usepackage{amssymb}
\usepackage{array}
\usepackage[makeroom]{cancel} %% for strike outs
\usepackage{pgffor} %% required for integral for loops
\usepackage{tikz}
\usepackage{tikz-cd}
\usepackage{tkz-euclide}
\usetikzlibrary{shapes.multipart}


%\usetkzobj{all}
\tikzstyle geometryDiagrams=[ultra thick,color=blue!50!black]


\usetikzlibrary{arrows}
\tikzset{>=stealth,commutative diagrams/.cd,
  arrow style=tikz,diagrams={>=stealth}} %% cool arrow head
\tikzset{shorten <>/.style={ shorten >=#1, shorten <=#1 } } %% allows shorter vectors

\usetikzlibrary{backgrounds} %% for boxes around graphs
\usetikzlibrary{shapes,positioning}  %% Clouds and stars
\usetikzlibrary{matrix} %% for matrix
\usepgfplotslibrary{polar} %% for polar plots
\usepgfplotslibrary{fillbetween} %% to shade area between curves in TikZ



%\usepackage[width=4.375in, height=7.0in, top=1.0in, papersize={5.5in,8.5in}]{geometry}
%\usepackage[pdftex]{graphicx}
%\usepackage{tipa}
%\usepackage{txfonts}
%\usepackage{textcomp}
%\usepackage{amsthm}
%\usepackage{xy}
%\usepackage{fancyhdr}
%\usepackage{xcolor}
%\usepackage{mathtools} %% for pretty underbrace % Breaks Ximera
%\usepackage{multicol}



\newcommand{\RR}{\mathbb R}
\newcommand{\R}{\mathbb R}
\newcommand{\C}{\mathbb C}
\newcommand{\N}{\mathbb N}
\newcommand{\Z}{\mathbb Z}
\newcommand{\dis}{\displaystyle}
%\renewcommand{\d}{\,d\!}
\renewcommand{\d}{\mathop{}\!d}
\newcommand{\dd}[2][]{\frac{\d #1}{\d #2}}
\newcommand{\pp}[2][]{\frac{\partial #1}{\partial #2}}
\renewcommand{\l}{\ell}
\newcommand{\ddx}{\frac{d}{\d x}}

\newcommand{\zeroOverZero}{\ensuremath{\boldsymbol{\tfrac{0}{0}}}}
\newcommand{\inftyOverInfty}{\ensuremath{\boldsymbol{\tfrac{\infty}{\infty}}}}
\newcommand{\zeroOverInfty}{\ensuremath{\boldsymbol{\tfrac{0}{\infty}}}}
\newcommand{\zeroTimesInfty}{\ensuremath{\small\boldsymbol{0\cdot \infty}}}
\newcommand{\inftyMinusInfty}{\ensuremath{\small\boldsymbol{\infty - \infty}}}
\newcommand{\oneToInfty}{\ensuremath{\boldsymbol{1^\infty}}}
\newcommand{\zeroToZero}{\ensuremath{\boldsymbol{0^0}}}
\newcommand{\inftyToZero}{\ensuremath{\boldsymbol{\infty^0}}}


\newcommand{\numOverZero}{\ensuremath{\boldsymbol{\tfrac{\#}{0}}}}
\newcommand{\dfn}{\textbf}
%\newcommand{\unit}{\,\mathrm}
\newcommand{\unit}{\mathop{}\!\mathrm}
%\newcommand{\eval}[1]{\bigg[ #1 \bigg]}
\newcommand{\eval}[1]{ #1 \bigg|}
\newcommand{\seq}[1]{\left( #1 \right)}
\renewcommand{\epsilon}{\varepsilon}
\renewcommand{\iff}{\Leftrightarrow}

\DeclareMathOperator{\arccot}{arccot}
\DeclareMathOperator{\arcsec}{arcsec}
\DeclareMathOperator{\arccsc}{arccsc}
\DeclareMathOperator{\si}{Si}
\DeclareMathOperator{\proj}{proj}
\DeclareMathOperator{\scal}{scal}
\DeclareMathOperator{\cis}{cis}
\DeclareMathOperator{\Arg}{Arg}
%\DeclareMathOperator{\arg}{arg}
\DeclareMathOperator{\Rep}{Re}
\DeclareMathOperator{\Imp}{Im}
\DeclareMathOperator{\sech}{sech}
\DeclareMathOperator{\csch}{csch}
\DeclareMathOperator{\Log}{Log}

\newcommand{\tightoverset}[2]{% for arrow vec
  \mathop{#2}\limits^{\vbox to -.5ex{\kern-0.75ex\hbox{$#1$}\vss}}}
\newcommand{\arrowvec}{\overrightarrow}
\renewcommand{\vec}{\mathbf}
\newcommand{\veci}{{\boldsymbol{\hat{\imath}}}}
\newcommand{\vecj}{{\boldsymbol{\hat{\jmath}}}}
\newcommand{\veck}{{\boldsymbol{\hat{k}}}}
\newcommand{\vecl}{\boldsymbol{\l}}
\newcommand{\utan}{\vec{\hat{t}}}
\newcommand{\unormal}{\vec{\hat{n}}}
\newcommand{\ubinormal}{\vec{\hat{b}}}

\newcommand{\dotp}{\bullet}
\newcommand{\cross}{\boldsymbol\times}
\newcommand{\grad}{\boldsymbol\nabla}
\newcommand{\divergence}{\grad\dotp}
\newcommand{\curl}{\grad\cross}
%% Simple horiz vectors
\renewcommand{\vector}[1]{\left\langle #1\right\rangle}


\outcome{In this section we define the cross product and we use it to create orthogonal vectors.}

\title{1.5 The Cross Product}



\begin{document}

\begin{abstract}
In this section we define the cross product and we use it to create orthogonal vectors.
\end{abstract}
 
\maketitle
The cross product is a special operation that helps us to create a vector that is orthogonal to two given two vectors in $\R^3$.
\begin{definition}[Cross Product in $\R^3$]
If $\avec{v_1}$ and $\avec{v_2}$ are vectors in $\R^3$ given by
\[
\avec{v_1} = \vector{x_1, y_1, z_1} \text{  and   } \;\avec{v_2} = \vector{x_2, y_2, z_2}
\]
then the cross product $\avec{v_1} \cross \avec{v_2}$ is defined by
\[
\avec{v_1} \cross \avec{v_2} =  (y_1z_2 - z_1y_2) \avec{i} + (z_1x_2 - x_1z_2) \avec{j} + (x_1y_2 - y_1x_2) \avec{k} 
\]
\end{definition}

\end{document}


\begin{example}
Find the indicated dot product: $\left(3\avec{i} -2\avec{j}\right) \dotp \left(2\avec{i} +5\avec{j}\right)$.\\
According to the definition, we have
\[
\left(3\avec{i} -2\avec{j}\right) \dotp \left(2\avec{i} +5\avec{j}\right) = (3)(2)+(-2)(5) = 6-10 = -4
\]
\end{example}

Note that the dot product of two vectors is a real number.  For this reason, the dot product is sometimes called the scalar product.

\begin{problem}
Find the indicated dot product: $\left(4\avec{i} + \avec{j}\right) \dotp \left(5\avec{i} -7\avec{j}\right) =\answer{13}$.\\
\end{problem}

It is interesting to compute the dot product of a vector with itself.  Let $\avec{v} = \vector{x, y}$. Then
\[
\avec{v} \dotp \avec{v} = x^2 + y^2 = |\avec{v}|^2
\]
We see that the dot product of a vector with itself gives the square of the magnitude of the vector.

It is also interesting to compute the dot product between two unit vectors in polar form.
If $\avec{u_1} = \vector{\cos \alpha, \sin \alpha}$ and $\avec{u_2} = \vector{\cos \beta, \sin \beta}$ then
\[
\avec{u_1} \dotp \avec{u_2} = \cos \alpha \cos \beta + \sin \alpha \sin \beta = \cos(\alpha - \beta)
\]
We see here that the dot product of unit vectors gives the cosine of the angle between the vectors.

\begin{image}
\begin{tikzpicture}
\draw[<->] (-3, 0)--(3,0);
\draw[<->] (0,-3)--(0,3);
\draw[thin] (0,0) circle (2);
\draw[->, thick, blue] (0,0) -- (-1.732, 1) node[above left]{$\avec{u_1}$};
\draw[->, thick, red] (0,0) -- (1.414, 1.414) node[above right]{$\avec{u_2}$};
\draw[thin] (0.5 ,0.5) arc (45:150: 0.7) node[midway, above]{$\alpha - \beta$};
\draw[blue, thin] (0.3, 0) arc (0:150: 0.3) node[midway, above]{$\alpha$};
\draw[red, thin] (0.5, 0) arc (0:45: 0.5) node[midway, right]{$\beta$};
\node at(0, -3.5) {$\avec{u_1} = \vector{\cos \alpha, \sin \alpha}$ and $\avec{u_2} = \vector{\cos \beta, \sin \beta}$};
\draw[thin] (2, 0.2) -- (2, -0.2) node[below right]{$1$};
\draw[thin] (0.2, 2) -- (-0.2, 2) node[above left]{$1$};
\end{tikzpicture}
\end{image}

In general, the dot product of two vectors is related to both the angle between the vectors and their magnitudes.

\begin{proposition}
Given two non-zero vectors in $\R^2$, $\avec{v_1} = \vector{x_1, y_1}$ and $\avec{v_2} = \vector{x_2, y_2}$, we have
\[
\avec{v_1} \dotp \avec{v_2} = |\avec{v_1}| \cdot |\avec{v_2}| \cos \theta
\]
where $\theta$ is the angle between the two vectors.
\begin{proof}
Write the vectors in polar form: $\avec{v_1} = |\avec{v_1}| \vector{\cos \alpha, \sin \alpha}$ and 
$\avec{v_2} = |\avec{v_2}| \vector{\cos \beta, \sin \beta}$. To calculate the dot product, first multiply by the magnitudes:
\begin{align*}
\avec{v_1} \cdot \avec{v_2} & = \left(|\avec{v_1}| \vector{\cos \alpha, \sin \alpha}\right) \dotp \left( |\avec{v_2}| \vector{\cos \beta, \sin \beta} \right)\\
                            &= \vector{|\avec{v_1}|\cos \alpha, |\avec{v_1}|\sin \alpha} \dotp   \vector{|\avec{v_2}|\cos \beta, |\avec{v_2}|\sin \beta} \\
                            &= |\avec{v_1}| \cos \alpha |\avec{v_2}|\cos \beta + |\avec{v_1}|\sin \alpha |\avec{v_2}|\sin \beta\\
                            &= |\avec{v_1}|\cdot |\avec{v_2}| \left(\cos \alpha \cos \beta + \sin \alpha \sin \beta\right)\\
                            &= |\avec{v_1}| \cdot|\avec{v_2}| \cos(\alpha - \beta)\\
                            &= |\avec{v_1}| \cdot |\avec{v_2}| \cos \theta
\end{align*}
where $\theta = \alpha - \beta$ is the angle between $\avec{v_1}$ and $\avec{v_2}$.
\end{proof}
\end{proposition}


In three dimensions, a vector does not have a polar form, but we can still use the dot product to find the angle between two vectors.
\begin{definition}[Dot Product in $\R^3$]
If $\avec{v_1}$ and $\avec{v_2}$ are vectors in $\R^3$ given by
\[
\avec{v_1} = \vector{x_1, y_1, z_1} \text{  and   } \;\avec{v_2} = \vector{x_2, y_2, z_2}
\]
then the dot product $\avec{v_1} \dotp \avec{v_2}$ is defined by
\[
\avec{v_1} \dotp \avec{v_2} = x_1x_2 + y_1y_2 + z_1z_2
\]
\end{definition}

\begin{example}
Compute the indicated dot products:\\
a) $\vector{1, 2, 3} \dotp \vector{5, -2, 4} = (1)(5)+ (2)(-2) + (3)(4) = 13$\\
b) $\left(3\avec{i} - 2\avec{k}\right) \dotp \left(4\avec{j} + 5\avec{k}\right) = (3)(0) + (0)(4) + (-2)(5) = -10$
\end{example}

\begin{problem}
Compute the indicated dot products:\\
a) $\vector{6, -1, 4} \dotp \vector{-3, 7, 2} = \answer{-17}$\\
b) $\left(2\avec{i} + 4\avec{j} - 3\avec{k}\right) \dotp \left(\avec{i} -5\avec{j}\right) = \answer{-18}$
\end{problem}

\begin{proposition}[Properties of the Dot Product]
Let $\avec{v} = \vector{x, y, z}, \avec{v_1} = \vector{x_1, y_1, z_1}, \avec{v_2} = \vector{x_2, y_2, z_2}$ and $c$ be any scalar. 
Then the following properties hold for the dot product:\\
i) $\avec{v_1} \dotp \avec{v_2} = \avec{v_2} \dotp \avec{v_1}$\\
ii) $\left(c\avec{v_1}\right) \dotp \avec{v_2} = \avec{v_1} \dotp \left(c\avec{v_2}\right) = c\left(\avec{v_1} \dotp \avec{v_2}\right)$\\
iii)  $\avec{v} \dotp \left(\avec{v_1} + \avec{v_2}\right) = \avec{v} \dotp \avec{v_1} + \avec{v} \dotp \avec{v_2}$\\
iv) $\avec{v} \dotp \avec{0} = 0$\\
v) $\avec{v} \dotp \avec{v} = |\avec{v}|^2$
\end{proposition}

Suppose that $\avec{v_1}=\vector{x_1, y_1, z_1}$ and $\avec{v_2}=\vector{x_2, y_2, z_2}$  are vectors in $\R^3$ placed in standard position.  
Then the initial point of both vectors is the origin, $(0,0,0)$ and the final points are
$(x_1, y_1, z_1)$ and $(x_2, y_2, z_2)$.  Together, these three points form a triangle in $\R^3$.  
The angle between the vectors $\avec{v_1}$ and $\avec{v_2}$ is the angle in this triangle with vertex at the origin.

\begin{proposition}
Let $\avec{v_1} = \vector{x_1, y_1, z_1}$ and $\avec{v_2} = \vector{x_2, y_2, z_2}$
be vectors in $\R^3$ and let $\theta$ denote the angle between these vectors.  Then, just as in $\R^2$, the dot product can be expressed in terms of the 
magnitudes of the vectors $\avec{v_1}$ and $\avec{v_2}$ and the angle $\theta$ between them as follows:
\[
\avec{v_1} \dotp \avec{v_2} = |\avec{v_1}|\cdot |\avec{v_2}| \cos\theta
\]
\begin{proof}
According to the Law of Cosines, 
\[
|\avec{v_1} - \avec{v_2}|^2 = |\avec{v_1}|^2 + |\avec{v_2}|^2 - 2|\avec{v_1}|\cdot| \avec{v_2}|\cos \theta
\]
See the figure below. By property $(v)$ of the preceding proposition, 
\[
|\avec{v_1} - \avec{v_2}|^2 = \left(\avec{v_1} - \avec{v_2}\right)\dotp \left(\avec{v_1} - \avec{v_2}\right)
\]
which then by properties $(ii)$ and $(iii)$of that same proposition then equals
\[
\avec{v_1} \dotp \avec{v_1} + \avec{v_2} \dotp \avec{v_2} - 2\left(\avec{v_1} \dotp \avec{v_2}\right)
\]
Using property $(v)$ again, this can be rewritten using magnitudes as follows
\[
|\avec{v_1}|^2 + |\avec{v_2}|^2 - 2\left(\avec{v_1} \dotp \avec{v_2}\right)
\]
Comparing this with right hand side of the Law of Cosines we can see that
\[
- 2\left(\avec{v_1} \dotp \avec{v_2}\right) = - 2|\avec{v_1}|\cdot| \avec{v_2}|\cos \theta
\]
Diving through by $-2$ gives the final result.

\begin{image}
\begin{tikzpicture}
\draw[blue, ->, thick] (0,0) -- (3, 1) node[midway, below]{$\avec{v_1}$};
\draw[blue, ->, thick] (0,0) -- (1, 4) node[midway, left]{$\avec{v_2}$};
\draw[red, ->, thick] (1,4) -- (3, 1) node[midway,right]{$\avec{v_1} - \avec{v_2}$};
\draw[blue] (0.8,.28) ++(2:0) arc (20:70:1);
\path[blue] (0,0.33) ++(10:0.4cm) node{$\theta$};
%\draw[blue] (0.5, 0.166) arc (20:70:.6) node[midway, above right]{$\theta$};
\node at (1.5, -1){Law of Cosines:};
\node at (1.5, -1.5){ $|\avec{v_1} - \avec{v_2}|^2 = |\avec{v_1}|^2 + |\avec{v_2}|^2 - 2|\avec{v_1}|\cdot| \avec{v_2}|\cos \theta$};
%\draw[white] (-3, -1.5) -- (7, -1.5);
\end{tikzpicture}
\end{image}

\end{proof}

\end{proposition}

From the above proposition, we see that the dot product of two vectors is zero under special conditions, i.e.,  if 
\[
\avec{v_1} \dotp \avec{v_2} = 0
\]
then either
\[
|\avec{v_1}| = 0, |\avec{v_2}| = 0 \; \text{or } \; \cos \theta = 0
\]
Noting that $\cos \theta = 0$ precisely when $\theta = 90^\circ$, we can conclude the following
\begin{corollary} 
Two non-zero vectors $\avec{v_1}$ and $\avec{v_2}$ are perpendicular if and only if $\avec{v_1} \dotp \avec{v_2} = 0$.
\end{corollary}
Perpendicular vectors are commonly called {\bf orthogonal}.\\


The shortest distance between two points in $\R^3$ is a straight line.  This basic fact is illustrated in the following proposition.

\begin{proposition}[Triangle Inequality]
Let $\avec{v}$ and $\avec{w}$ be vectors in $\R^3$. Then the following inequality holds:
\[
|\avec{v} + \avec{w}| \leq |\avec{v}| +|\avec{w}| 
\]



\begin{proof}
Using the properties of the dot product in $\R^3$, we can write the following
\begin{align*}
|\avec{v} + \avec{w}|^2 &= \left(\avec{v} + \avec{w}\right) \dotp \left(\avec{v} + \avec{w}\right)\\
                        &= \avec{v} \dotp \avec{v} + \avec{w}\dotp \avec{w} + 2\avec{v} \dotp \avec{w}\\
                        &= |\avec{v}|^2 +|\avec{w}|^2 + 2|\avec{v}|\cdot |\avec{w}| \cos \theta\\
                        &\leq |\avec{v}|^2 +|\avec{w}|^2 + 2|\avec{v}|\cdot |\avec{w}|\\
                        &= \left(|\avec{v}| + |\avec{w}|\right)^2
\end{align*}
where $\theta$ represents the angle between the vectors $\avec{v}$ and $\avec{w}$.  The result follows immediately by taking the square root.
          
\begin{image}
\begin{tikzpicture}
\draw[blue, ->, thick] (0,0) -- (3, 1) node[midway, below]{$\avec{v}$};
\draw[red, ->, thick] (0,0) -- (1, 4) node[midway, left]{$\avec{v} +\avec{w}$};
\draw[blue, <-, thick] (1,4) -- (3, 1) node[midway,right]{$\avec{w}$};
%\draw[blue, ->, thick] (0,0) -- (2, 1) node[midway, below]{$\avec{v}$};
%\draw[blue, ->, thick] (2, 1) -- (3, 3) node[midway, right]{$\avec{w}$};
%\draw[red, ->, thick] (0,0) -- (3,3) node[midway, above left]{$\avec{v} + \avec{w}$};
\node at (1.5, -1){Triangle Inequality: $|\avec{v} + \avec{w}| \leq |\avec{v}| +|\avec{w}| $};
\end{tikzpicture}
\end{image}

              
\end{proof}

\end{proposition}

Note that the proof gives us a little bit more information about the inequality. If $\cos \theta = 1$, then the inequality becomes equality.  This happens precisely 
when the vectors $\avec{v}$ and $\avec{w}$ are parallel, i.e., when $\theta = 0$. The result holds for vectors in $\R^2$ as well using the same proof.

\end{document}


\begin{tikzpicture}[->]
\draw (0,0) -- (xyz cs:x=1);
\draw (0,0) -- (xyz cs:y=1);
\draw (0,0) -- (xyz cs:z=1);
\end{tikzpicture}

\begin{tikzpicture}
\draw (-1,0) -- +(3.5,0);
\draw (1,0) ++(210:2cm) -- +(30:4cm);
\draw (1,0) +(0:1cm) arc (0:30:1cm);
\draw (1,0) +(180:1cm) arc (180:210:1cm);
\path (1,0) ++(15:.75cm) node{$\alpha$};
\path (1,0) ++(15:-.75cm) node{$\beta$};
\end{tikzpicture}



