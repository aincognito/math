\documentclass[handout]{ximera}

%% You can put user macros here
%% However, you cannot make new environments



\newcommand{\ffrac}[2]{\frac{\text{\footnotesize $#1$}}{\text{\footnotesize $#2$}}}
\newcommand{\vasymptote}[2][]{
    \draw [densely dashed,#1] ({rel axis cs:0,0} -| {axis cs:#2,0}) -- ({rel axis cs:0,1} -| {axis cs:#2,0});
}


\graphicspath{{./}{firstExample/}}
\usepackage{forest}
\usepackage{amsmath}
\usepackage{amssymb}
\usepackage{array}
\usepackage[makeroom]{cancel} %% for strike outs
\usepackage{pgffor} %% required for integral for loops
\usepackage{tikz}
\usepackage{tikz-cd}
\usepackage{tkz-euclide}
\usetikzlibrary{shapes.multipart}


%\usetkzobj{all}
\tikzstyle geometryDiagrams=[ultra thick,color=blue!50!black]


\usetikzlibrary{arrows}
\tikzset{>=stealth,commutative diagrams/.cd,
  arrow style=tikz,diagrams={>=stealth}} %% cool arrow head
\tikzset{shorten <>/.style={ shorten >=#1, shorten <=#1 } } %% allows shorter vectors

\usetikzlibrary{backgrounds} %% for boxes around graphs
\usetikzlibrary{shapes,positioning}  %% Clouds and stars
\usetikzlibrary{matrix} %% for matrix
\usepgfplotslibrary{polar} %% for polar plots
\usepgfplotslibrary{fillbetween} %% to shade area between curves in TikZ



%\usepackage[width=4.375in, height=7.0in, top=1.0in, papersize={5.5in,8.5in}]{geometry}
%\usepackage[pdftex]{graphicx}
%\usepackage{tipa}
%\usepackage{txfonts}
%\usepackage{textcomp}
%\usepackage{amsthm}
%\usepackage{xy}
%\usepackage{fancyhdr}
%\usepackage{xcolor}
%\usepackage{mathtools} %% for pretty underbrace % Breaks Ximera
%\usepackage{multicol}



\newcommand{\RR}{\mathbb R}
\newcommand{\R}{\mathbb R}
\newcommand{\C}{\mathbb C}
\newcommand{\N}{\mathbb N}
\newcommand{\Z}{\mathbb Z}
\newcommand{\dis}{\displaystyle}
%\renewcommand{\d}{\,d\!}
\renewcommand{\d}{\mathop{}\!d}
\newcommand{\dd}[2][]{\frac{\d #1}{\d #2}}
\newcommand{\pp}[2][]{\frac{\partial #1}{\partial #2}}
\renewcommand{\l}{\ell}
\newcommand{\ddx}{\frac{d}{\d x}}

\newcommand{\zeroOverZero}{\ensuremath{\boldsymbol{\tfrac{0}{0}}}}
\newcommand{\inftyOverInfty}{\ensuremath{\boldsymbol{\tfrac{\infty}{\infty}}}}
\newcommand{\zeroOverInfty}{\ensuremath{\boldsymbol{\tfrac{0}{\infty}}}}
\newcommand{\zeroTimesInfty}{\ensuremath{\small\boldsymbol{0\cdot \infty}}}
\newcommand{\inftyMinusInfty}{\ensuremath{\small\boldsymbol{\infty - \infty}}}
\newcommand{\oneToInfty}{\ensuremath{\boldsymbol{1^\infty}}}
\newcommand{\zeroToZero}{\ensuremath{\boldsymbol{0^0}}}
\newcommand{\inftyToZero}{\ensuremath{\boldsymbol{\infty^0}}}


\newcommand{\numOverZero}{\ensuremath{\boldsymbol{\tfrac{\#}{0}}}}
\newcommand{\dfn}{\textbf}
%\newcommand{\unit}{\,\mathrm}
\newcommand{\unit}{\mathop{}\!\mathrm}
%\newcommand{\eval}[1]{\bigg[ #1 \bigg]}
\newcommand{\eval}[1]{ #1 \bigg|}
\newcommand{\seq}[1]{\left( #1 \right)}
\renewcommand{\epsilon}{\varepsilon}
\renewcommand{\iff}{\Leftrightarrow}

\DeclareMathOperator{\arccot}{arccot}
\DeclareMathOperator{\arcsec}{arcsec}
\DeclareMathOperator{\arccsc}{arccsc}
\DeclareMathOperator{\si}{Si}
\DeclareMathOperator{\proj}{proj}
\DeclareMathOperator{\scal}{scal}
\DeclareMathOperator{\cis}{cis}
\DeclareMathOperator{\Arg}{Arg}
%\DeclareMathOperator{\arg}{arg}
\DeclareMathOperator{\Rep}{Re}
\DeclareMathOperator{\Imp}{Im}
\DeclareMathOperator{\sech}{sech}
\DeclareMathOperator{\csch}{csch}
\DeclareMathOperator{\Log}{Log}

\newcommand{\tightoverset}[2]{% for arrow vec
  \mathop{#2}\limits^{\vbox to -.5ex{\kern-0.75ex\hbox{$#1$}\vss}}}
\newcommand{\arrowvec}{\overrightarrow}
\renewcommand{\vec}{\mathbf}
\newcommand{\veci}{{\boldsymbol{\hat{\imath}}}}
\newcommand{\vecj}{{\boldsymbol{\hat{\jmath}}}}
\newcommand{\veck}{{\boldsymbol{\hat{k}}}}
\newcommand{\vecl}{\boldsymbol{\l}}
\newcommand{\utan}{\vec{\hat{t}}}
\newcommand{\unormal}{\vec{\hat{n}}}
\newcommand{\ubinormal}{\vec{\hat{b}}}

\newcommand{\dotp}{\bullet}
\newcommand{\cross}{\boldsymbol\times}
\newcommand{\grad}{\boldsymbol\nabla}
\newcommand{\divergence}{\grad\dotp}
\newcommand{\curl}{\grad\cross}
%% Simple horiz vectors
\renewcommand{\vector}[1]{\left\langle #1\right\rangle}


\outcome{Determine tangent planes to surfaces.}

\title{3.5 Tangent Planes}



\begin{document}

\begin{abstract}
In this section we determine tangent planes to surfaces.
\end{abstract}

\maketitle

\begin{proposition}
Let $f(x,y)$ be a function of two variables and let $(x_0, y_0)$ be a point on the (smooth) level curve $f(x,y) = k$, i.e. $f(x_0, y_0) = k$.
Then, the gradient vector at this point, $\grad f(x_0, y_0)$ is orthogonal to the level curve.
\end{proposition}
\begin{proof}
Let $\vec r(t)$ be a parameterization of the (smooth) level curve $f(x,y) = k$ and let $\vec u \neq \vec 0$ be the direction 
vector for $\vec r(t)$ at the point $(x_0, y_0)$, i.e., 
$\vec u = \vec T(t)$. Then the rate of change of $f(x,y)$ in the direction of $\vec u$ at $(x_0, y_0)$ is zero since $f(x,y)$ is 
constant on the level curve $f(x,y) = k$ which implies that
\[
D_{\vec u} f(x_0, y_0) = 0
\]
But,
\[
0 = D_{\vec u} f(x_0, y_0)  = \grad f(x_0, y_0) \dotp \vec u
\]
from which we can conclude that the gradient is orthogonal to the level curve at the point $(x_0, y_0)$.
\end{proof}


\begin{example}[Example 1]
Verify the proposition by showing directly that the gradient vector is orthogonal to the level curve of the paraboloid $z = f(x,y) = x^2 + y^2$ at any point $(x,y)$.\\
The level curves have the form 
\[
x^2 + y^2 = k
\]
which (for $k >0$) is a circle centered at the origin with radius $\sqrt k$. 
A basic fact from geometry is that at any point on a circle, the tangent line is perpendicular to the radius. 
Now, note that the gradient vector is in the same direction as the radius vector at any point $(x, y)$:
\[
\grad f(x,y) = \vector{2x, 2y} = 2\vector{x,y}
\]
For the case $k = 0$, the level curve is just the origin and at the origin, the gradient is the zero vector 
which is considered to be orthogonal to any vector.
\end{example}

\begin{problem}(Problem 1a)
Verify the proposition by showing directly that the gradient is orthogonal to the level curve  of the plane $z = f(x,y) = ax + by$ at any point $(x,y)$.
\begin{hint}
To find the direction vector of the line $ax + by = k$ rewrite the line as $\vec r(x) = \vector{x, \frac{k-ax}{b}}$
\end{hint}
\end{problem}


\begin{problem}(Problem 1b)
Verify the proposition by showing directly that the gradient is orthogonal to the level curve  of the hemisphere $z = f(x,y) = \sqrt{1 - x^2 - y^2}$ at any point $(x,y)$.
\begin{hint}
At any point on a circle, the tangent line is perpendicular to the radius.
\end{hint}
\end{problem}

To create the normal vector to a surface at a point, we need the analogue of the previous proposition for level surfaces.

\begin{proposition}
Let $f(x,y,z)$ be a function of three variables and let let $(x_0, y_0, z_0)$ be a point on the level surface $f(x,y, z) = k$.
Then, the gradient vector at this point, $\grad f(x_0, y_0, z_0)$ is orthogonal to the level surface.
\end{proposition}
\begin{proof}
Let $\vec r(t)$ be any smooth curve on the level surface $f(x,y, z) = k$ passing through the point $(x_0, y_0, z_0)$ with direction vector $\vec u$ at this point.
Since $f$ is constant on $\vec r(t)$, the directional derivative $D_{\vec u} f(x_0, y_0, z_0)$ is zero, and, as in the previous proposition, 
$\grad f(x_0, y_0, z_0) \perp \vec u$.  
Since the curve $\vec r(t)$ is an arbitrary curve on the surface passing through $(x_0, y_0, z_0)$, the gradient vector must be 
orthogonal to the surface itself at this point.
\end{proof}

\begin{example}[Example 2]
Verify the proposition by showing that the gradient vector is orthogonal to the level surfaces of $f(x, y, z) = x^2 + y^2 + z^2$ at any point.\\
The level surfaces have the form
\[
x^2 + y^2 + z^2 = k
\]
which are spheres centered at the origin with radius $\sqrt k$ (for $k>0$).
At any point $(x, y, z)$, the gradient vector is
\[
\grad f(x, y, z) = \vector{2x, 2y, 2z} = 2\vector{x, y, z}
\]
which is in the direction of the radius of the sphere at that point. Since the radius is orthogonal the sphere at any point,
the gradient, $\grad f(x,y,z)$ is orthogonal to the level surface $f(x, y, z,) = k$ at any point, as claimed by the proposition.
\end{example}

\begin{problem}(Problem 2)
Verify the preceding proposition by showing directly that the gradient is orthogonal to the level curve  of $f(x,y, z) = ax + by + cz$ at any point $(x,y, z)$.
\begin{hint}
The level curves are planes with normal vector $\vector{a, b, c}$
\end{hint}
\end{problem}


\begin{theorem}[Tangent Plane]
The tangent plane to the surface $z = f(x,y)$ at the point $(x_0, y_0, z_0)$ is given by
\[
z = z_0 + f_x(x_0, y_0)(x - x_0) + f_y(x_0, y_0)(y - y_0)
\]
\end{theorem}
\begin{proof}
The surface $z = f(x,y)$ is the level surface $g(x,y,z) = 0$ where $g(x,y,z) = z - f(x,y)$.
According to the previous proposition, the vector $\grad g(x_0, y_0, z_0)$ is orthogonal to the surface at the point $(x_0, y_0, z_0)$.
Hence, this vector is the normal to the plane and the equation of the plane is
\[
\grad g(x_0, y_0, z_0) \dotp \vector{x- x_0, y-y_0, z-z_0} = 0
\]
Since $\grad g(x_0, y_0, z_0) = \vector{-f_x(x_0, y_0), -f_y(x_0, y_0), 1}$, 
the equation of the plane can be rewritten as
\[
-f_x(x_0, y_0)(x-x_0) - f_y(x_0, y_0)(y-y_0) + (z-z_0) = 0
\]
which in turn can be written as
\[
z = z_0 + f_x(x_0, y_0)(x-x_0) + f_y(x_0, y_0)(y-y_0)
\]
as claimed.
\end{proof}

\begin{corollary}
The equation of the tangent plane to a surface of the form $g(x, y, x) = 0$ at the point $(x_0, y_0, z_0)$ is given by
\[
\grad g(x_0, y_0, z_0) \dotp \vector{x - x_0, y - y_0, z - z_0} = 0
\]
\end{corollary}

\begin{example}[Example 3]
Find the equation of the tangent plane to the paraboloid $z = x^2 + y^2$ at the point $(3, -4, 25)$.\\
Write $g(x, y, z) = z - f(x,y) = 0$ and compute the gradient of $g$:
\[
\grad g(3, -4, 25) = \vector{-f_x(3, -4), -f_y(3, -4), 1} = \vector{-6, 8, 1}
\]
Then the equation of the tangent plane is
\[
\vector{-6, 8, 1} \dotp \vector{x-3, y+4, z-25} = 0
\]
which yields
\[
z = 25 + 6(x-3) - 8(y+4)
\]
or
\[
z = 6x - 8y - 25
\]
\end{example}

\begin{problem}(Problem 3a)
Find the equation of the tangent plane to the paraboloid $z = x^2 + y^2$ at the point $(-2, 1, 5)$.\\
\[
z = \answer{-5-4x+2y}
\]

\end{problem}


\begin{problem}(Problem 3b)
Find the equation of the tangent plane to the elliptic paraboloid $z = x^2 + 4y^2$ at the point $(4, -1, 17)$.\\
\[
z = \answer{-23 + 8x -8y}
\]
\end{problem}


\begin{example}[Example 4]
Find the equation of the tangent plane to the surface $z = \ln(x^2 + y^2)$ at the point $(-6, 8, 2\ln 10)$.\\
Write $g(x, y, z) = z - f(x,y) = 0$ and compute the gradient of $g$:
\[
\grad g(-6, 8, 2\ln 10) = \vector{-f_x(-6, 8), -f_y(-6, 8), 1} = \vector{-\frac{3}{25}, \frac{4}{25}, 1}
\]
Then the equation of the tangent plane is
\[
\vector{-\frac{3}{25}, \frac{4}{25}, 1} \dotp \vector{x+6, y-8, z-2\ln 10} = 0
\]
which yields
\[
z = 2\ln 10 - \frac{3}{25}(x+6) +\frac{4}{25}(y-8)
\]
\end{example}

\begin{problem}(Problem 4a)
Find the equation of the tangent plane to the hemisphere $z = \sqrt{50 -x^2 - y^2}$ at the point $(3, 4, 5)$.\\
\[
z = \answer{10-\frac35 x - \frac45 y}
\]

\end{problem}


\begin{problem}(Problem 4b)
Find the equation of the tangent plane to the surface $z =e^{xy}$ at the point $(2, 1, e^2)$.\\
\[
z = \answer{-3e^2 + e^2 x + 2e^2 y}
\]
\end{problem}

\begin{example}[Example 5]
Find the equation of the tangent plane to the hyperboloid of 1 sheet given by 
\[
\frac{x^2}{4} + \frac{y^2}{9} - \frac{z^2}{16} = 1
\]
at the point $(2, 3, 4)$.\\
Rewrite the surface as
\[
g(x,y,z) = \frac{x^2}{4} + \frac{y^2}{9} - \frac{z^2}{16} - 1 = 0
\]
Compute the gradient of $g$:
\[
\grad g(x,y,z) = \vector{\frac{x}{2} , \frac{2y}{9}, - \frac{z}{8}}
\]
and evaluate it at the given point:
\[
\grad g(2,3,4) = \vector{1, \frac{2}{3}, -\frac12}
\]
The equation of the tangent plane is given by 
\[
\grad g(2, 3, 4) \dotp \vector{x-2, y-3, z-4} = 0
\]
which gives
\[
(x-2) + \frac23 (y-3) - \frac12 (z-4) = 0
\]
or
\[
6x + 4y - 3z = 12
\]
\end{example}

\begin{problem}(Problem 5a)
Find the equation of the tangent plane to the ellipsoid given by 
\[
\frac{x^2}{12} + \frac{y^2}{6} + \frac{z^2}{8} = 1
\]
at the point $(2, -1, -2)$.\\
\[
\answer{2x-2y-3z} = 12
\]
\end{problem}

\begin{problem}(Problem 5b)
Find the equation of the tangent plane to the hyperboloid of 2 sheets given by 
\[
\frac{x^2}{8} + \frac{y^2}{32} - \frac{z^2}{8} = -1
\]
at the point $(2, 4, -4)$.\\
\[
\answer{2x+y + 4z} = -8
\]
\end{problem}




\section{Linear Approximation}
We can use the equation of the tangent plane to approximate the values of a function $f(x,y)$ near the point of tangency.

Suppose the value $f(x_0, y_0)$ is known. The linear approximation to the function $f$ at a point $(x,y)$ near the point $(x_0, y_0)$
is given by
\[
f(x,y) \approx f(x_0, y_0) + f_x(x_0, y_0)(x-x_0) + f_y(x_0, y_0)(y-y_0)
\]

\begin{example}[Example 6]
The temperature $T$ (in Celsius) at a point $(x,y)$ on a metal plate is given by 
\[
T = x^2 - xy + 3y^2
\]
Use a linear approximation to estimate the temperature at the point $(2.95, -0.96)$.\\
The point $(2.95, -0.96)$ is near the point $(3,-1)$ where the temperature and its gradient are easily computed.
We have
\[
\grad T(x,y) = \vector{2x - y, 6y - x}
\]
and at the point $(3, -1)$ the gradient is
\[
\grad T(3, -1) = \vector{7, -9}
\]
Since $T(3, -1) = 9 + 3 + 3 = 15$, the linear approximation is given by
\begin{align*}
T = f(x,y) &\approx T(3, -1) + \grad T(3, -1) \dotp \vector{x-3, y+1}
           &= 15 + 7(x-3) - 9(y+1)\\
\end{align*}
Hence, the temperature at $(2.95, -0.96)$ is approximately
\[
T \approx 15 + 7(2.95 - 3) - 9(-0.96 + 1) = 14.29^\circ \text{C}
\]
\end{example}


\begin{problem}(Problem 6)
The body mass index $I$ of a person weighing $W$ kilograms that has a height $H$ meters is given by
\[
I = \frac{W}{H^2}
\]
Estimate the BMI of a person weighing $101.6$ kg with a height of $1.98$ meters.\\
The BMI of a person weighing $100$ kg with a height of $2$ meters is $I = \answer{25}$
\[
\grad I(W,H) = \vector{\answer{1/H^2}, \answer{-2W/H^3}}
\]
\[
\grad I(100, 2) = \vector{\answer{1/4}, \answer{-25}}
\]
The BMI of a person with weight $101.6$ kg and height $1.98$ meters is approximately (one decimal place)
\[
I \approx \answer{25.9}
\]
\end{problem}



\end{document}
