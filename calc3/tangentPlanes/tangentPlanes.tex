\documentclass[handout]{ximera}

%% You can put user macros here
%% However, you cannot make new environments



\newcommand{\ffrac}[2]{\frac{\text{\footnotesize $#1$}}{\text{\footnotesize $#2$}}}
\newcommand{\vasymptote}[2][]{
    \draw [densely dashed,#1] ({rel axis cs:0,0} -| {axis cs:#2,0}) -- ({rel axis cs:0,1} -| {axis cs:#2,0});
}


%\usepackage{tcolorbox} %%Needed for Derivative Definition supposedly and product rule, natural exp log, quotient rule, inverse trig, rates of change


% \graphicspath{{./}{firstExample/}}
% \usepackage{forest}
\usepackage{amsmath}
\usepackage{amssymb}
\usepackage{array}
\usepackage[makeroom]{cancel} %% for strike outs
\usepackage{pgffor} %% required for integral for loops
\usepackage{tikz}
\usepackage{tikz-cd}
\usepackage{tkz-euclide}
\usetikzlibrary{shapes.multipart}


% \usetkzobj{all}
\tikzstyle geometryDiagrams=[ultra thick,color=blue!50!black]


\usetikzlibrary{arrows}
\tikzset{>=stealth,commutative diagrams/.cd,
  arrow style=tikz,diagrams={>=stealth}} %% cool arrow head
\tikzset{shorten <>/.style={ shorten >=#1, shorten <=#1 } } %% allows shorter vectors

\usetikzlibrary{backgrounds} %% for boxes around graphs
\usetikzlibrary{shapes,positioning}  %% Clouds and stars
\usetikzlibrary{matrix} %% for matrix
\usepgfplotslibrary{polar} %% for polar plots
\usepgfplotslibrary{fillbetween} %% to shade area between curves in TikZ



%\usepackage[width=4.375in, height=7.0in, top=1.0in, papersize={5.5in,8.5in}]{geometry}
%\usepackage[pdftex]{graphicx}
%\usepackage{tipa}
%\usepackage{txfonts}
%\usepackage{textcomp}
%\usepackage{amsthm}
%\usepackage{xy}
%\usepackage{fancyhdr}
%\usepackage{xcolor}
%\usepackage{mathtools} %% for pretty underbrace % Breaks Ximera
%\usepackage{multicol}



\newcommand{\RR}{\mathbb R}
\newcommand{\R}{\mathbb R}
\newcommand{\C}{\mathbb C}
\newcommand{\N}{\mathbb N}
\newcommand{\Z}{\mathbb Z}
\newcommand{\dis}{\displaystyle}
%\renewcommand{\d}{\,d\!}
\renewcommand{\d}{\mathop{}\!d}
\newcommand{\dd}[2][]{\frac{\d #1}{\d #2}}
\newcommand{\pp}[2][]{\frac{\partial #1}{\partial #2}}
\renewcommand{\l}{\ell}
\newcommand{\ddx}{\frac{d}{\d x}}
\newcommand{\ppx}{\frac{\partial}{\partial x}}
\newcommand{\ppy}{\frac{\partial}{\partial y}}

\newcommand{\zeroOverZero}{\ensuremath{\boldsymbol{\tfrac{0}{0}}}}
\newcommand{\inftyOverInfty}{\ensuremath{\boldsymbol{\tfrac{\infty}{\infty}}}}
\newcommand{\zeroOverInfty}{\ensuremath{\boldsymbol{\tfrac{0}{\infty}}}}
\newcommand{\zeroTimesInfty}{\ensuremath{\small\boldsymbol{0\cdot \infty}}}
\newcommand{\inftyMinusInfty}{\ensuremath{\small\boldsymbol{\infty - \infty}}}
\newcommand{\oneToInfty}{\ensuremath{\boldsymbol{1^\infty}}}
\newcommand{\zeroToZero}{\ensuremath{\boldsymbol{0^0}}}
\newcommand{\inftyToZero}{\ensuremath{\boldsymbol{\infty^0}}}


\newcommand{\numOverZero}{\ensuremath{\boldsymbol{\tfrac{\#}{0}}}}
\newcommand{\dfn}{\textbf}
%\newcommand{\unit}{\,\mathrm}
\newcommand{\unit}{\mathop{}\!\mathrm}
%\newcommand{\eval}[1]{\bigg[ #1 \bigg]}
\newcommand{\eval}[1]{ #1 \bigg|}
\newcommand{\seq}[1]{\left( #1 \right)}
\renewcommand{\epsilon}{\varepsilon}
\renewcommand{\iff}{\Leftrightarrow}

\DeclareMathOperator{\arccot}{arccot}
\DeclareMathOperator{\arcsec}{arcsec}
\DeclareMathOperator{\arccsc}{arccsc}
\DeclareMathOperator{\si}{Si}
\DeclareMathOperator{\proj}{proj}
\DeclareMathOperator{\scal}{scal}
\DeclareMathOperator{\cis}{cis}
\DeclareMathOperator{\Arg}{Arg}
%\DeclareMathOperator{\arg}{arg}
\DeclareMathOperator{\Rep}{Re}
\DeclareMathOperator{\Imp}{Im}
\DeclareMathOperator{\sech}{sech}
\DeclareMathOperator{\csch}{csch}
\DeclareMathOperator{\Log}{Log}

\newcommand{\tightoverset}[2]{% for arrow vec
  \mathop{#2}\limits^{\vbox to -.5ex{\kern-0.75ex\hbox{$#1$}\vss}}}
\newcommand{\arrowvec}{\overrightarrow}
\renewcommand{\vec}{\mathbf}
\newcommand{\veci}{{\boldsymbol{\hat{\imath}}}}
\newcommand{\vecj}{{\boldsymbol{\hat{\jmath}}}}
\newcommand{\veck}{{\boldsymbol{\hat{k}}}}
\newcommand{\vecl}{\boldsymbol{\l}}
\newcommand{\utan}{\vec{\hat{t}}}
\newcommand{\unormal}{\vec{\hat{n}}}
\newcommand{\ubinormal}{\vec{\hat{b}}}

\newcommand{\dotp}{\bullet}
\newcommand{\cross}{\boldsymbol\times}
\newcommand{\grad}{\boldsymbol\nabla}
\newcommand{\divergence}{\grad\dotp}
\newcommand{\curl}{\grad\cross}
%% Simple horiz vectors
\renewcommand{\vector}[1]{\left\langle #1\right\rangle}


\outcome{Determine tangent planes to surfaces.}

\title{3.5 Tangent Planes}



\begin{document}

\begin{abstract}
In this section we determine tangent planes to surfaces.
\end{abstract}

\maketitle

\begin{proposition}
Let $f(x,y)$ be a function of two variables and let $(x_0, y_0)$ be a point on the (smooth) level curve $f(x,y) = k$, i.e. $f(x_0, y_0) = k$.
Then, at this point, the gradient vector is perpendicular to the level curve.
\end{proposition}
\begin{proof}
Let $\vec r(t)$ be a parameterization of the (smooth) level curve $f(x,y) = k$ and let $\vec u \neq \vec 0$ be the direction 
vector for $\vec r(t)$ at the point $(x_0, y_0)$, i.e., 
$\vec u = \vec T(t)$. Then the rate of change of $f(x,y)$ is zero, since $f(x,y)$ is constant on the level curve $f(x,y) = k$. Thus, the directional derivative
$D_{\vec u} f(x_0, y_0)$ is equal to zero. Hence,
\[
0 = D_{\vec u} f(x_0, y_0)  = \grad f(x_0, y_0) \dotp \vec u
\]
which implies that the gradient is orthogonal to the level curve at the point $(x_0, y_0)$ as claimed.
\end{proof}

\begin{proposition}
Let $f(x,y,z)$ be a function of three variables and let let $(x_0, y_0, z_0)$ be a point on the level surface $f(x,y, z) = k$, i.e. $f(x_0, y_0, z_0) = k$.
Then, at this point, the gradient vector is perpendicular to the level surface.
\end{proposition}
\begin{proof}
Let $\vec r(t)$ be any smooth curve on the level surface $f(x,y, z) = k$ passing through the point $(x_0, y_0, z_0)$ with direction vector $\vec u$ at this point.
Since $f$ is constant on $\vec r(t)$, the directional derivative $D_{\vec u} f(x_0, y_0, z_0)$ is zero, and as in the proposition above, 
$\grad f(x_0, y_0, z_0) \perp \vec u$.  
Since the curve $\vec r(t)$ is an arbitrary curve on the surface passing through $(x_0, y_0, z_0)$, the gradient vector must be 
orthogonal to the surface itself at this point.
\end{proof}

\begin{theorem}[Tangent Plane]
The tangent plane to the surface $z = f(x,y)$ at the point $(x_0, y_0, z_0)$ is given by
\[
z = z_0 + f_x(x_0, y_0)(x - x_0) + f_y(x_0, y_0)(y - y_0)
\]
\end{theorem}
\begin{proof}
The surface $z = f(x,y)$ is the level surface $g(x,y,z) = 0$ where $g(x,y,z) = z - f(x,y)$.
According to the previous proposition, the vector $\grad g(x_0, y_0, z_0)$ is orthogonal to the surface at the point $(x_0, y_0, z_0)$.
Hence, this vector is the normal to the plane and the equation of the plane is
\[
g_x(x - x_0) + g_y(y-y_0) + g_z(z -z_0) = 0
\]
Since $\grad g(x_0, y_0, z_0) = \vector{g_x, g_y, g_z} = \vector{-f_x, -f_y, 1}$, the equation of the plane can be rewritten as
\[
-f_x(x-x_0) - f_y(y-y_0) + (z-z_0) = 0
\]
which in turn can be written as
\[
z = z_0 + f_x(x-x_0) + f_y(y-y_0)
\]
as claimed.
\end{proof}


\section{Linear Approximation}
We can use the equation of the tangent plane to approximate the values of a function $f(x,y)$ near the point of tangnecy.

Suppose the value $f(x_0, y_0)$ is known. The linear approximation to the function $f$ at a point $(x,y)$ near the point $(x_0, y_0)$
is given by
\[
f(x,y) \approx f(x_0, y_0) + f_x(x_0, y_0)(x-x_0) + f_y(x_0, y_0)(y-y_0)
\]




\end{document}
