\documentclass[handout]{ximera}

%% You can put user macros here
%% However, you cannot make new environments



\newcommand{\ffrac}[2]{\frac{\text{\footnotesize $#1$}}{\text{\footnotesize $#2$}}}
\newcommand{\vasymptote}[2][]{
    \draw [densely dashed,#1] ({rel axis cs:0,0} -| {axis cs:#2,0}) -- ({rel axis cs:0,1} -| {axis cs:#2,0});
}


\graphicspath{{./}{firstExample/}}
\usepackage{forest}
\usepackage{amsmath}
\usepackage{amssymb}
\usepackage{array}
\usepackage[makeroom]{cancel} %% for strike outs
\usepackage{pgffor} %% required for integral for loops
\usepackage{tikz}
\usepackage{tikz-cd}
\usepackage{tkz-euclide}
\usetikzlibrary{shapes.multipart}


%\usetkzobj{all}
\tikzstyle geometryDiagrams=[ultra thick,color=blue!50!black]


\usetikzlibrary{arrows}
\tikzset{>=stealth,commutative diagrams/.cd,
  arrow style=tikz,diagrams={>=stealth}} %% cool arrow head
\tikzset{shorten <>/.style={ shorten >=#1, shorten <=#1 } } %% allows shorter vectors

\usetikzlibrary{backgrounds} %% for boxes around graphs
\usetikzlibrary{shapes,positioning}  %% Clouds and stars
\usetikzlibrary{matrix} %% for matrix
\usepgfplotslibrary{polar} %% for polar plots
\usepgfplotslibrary{fillbetween} %% to shade area between curves in TikZ



%\usepackage[width=4.375in, height=7.0in, top=1.0in, papersize={5.5in,8.5in}]{geometry}
%\usepackage[pdftex]{graphicx}
%\usepackage{tipa}
%\usepackage{txfonts}
%\usepackage{textcomp}
%\usepackage{amsthm}
%\usepackage{xy}
%\usepackage{fancyhdr}
%\usepackage{xcolor}
%\usepackage{mathtools} %% for pretty underbrace % Breaks Ximera
%\usepackage{multicol}



\newcommand{\RR}{\mathbb R}
\newcommand{\R}{\mathbb R}
\newcommand{\C}{\mathbb C}
\newcommand{\N}{\mathbb N}
\newcommand{\Z}{\mathbb Z}
\newcommand{\dis}{\displaystyle}
%\renewcommand{\d}{\,d\!}
\renewcommand{\d}{\mathop{}\!d}
\newcommand{\dd}[2][]{\frac{\d #1}{\d #2}}
\newcommand{\pp}[2][]{\frac{\partial #1}{\partial #2}}
\renewcommand{\l}{\ell}
\newcommand{\ddx}{\frac{d}{\d x}}

\newcommand{\zeroOverZero}{\ensuremath{\boldsymbol{\tfrac{0}{0}}}}
\newcommand{\inftyOverInfty}{\ensuremath{\boldsymbol{\tfrac{\infty}{\infty}}}}
\newcommand{\zeroOverInfty}{\ensuremath{\boldsymbol{\tfrac{0}{\infty}}}}
\newcommand{\zeroTimesInfty}{\ensuremath{\small\boldsymbol{0\cdot \infty}}}
\newcommand{\inftyMinusInfty}{\ensuremath{\small\boldsymbol{\infty - \infty}}}
\newcommand{\oneToInfty}{\ensuremath{\boldsymbol{1^\infty}}}
\newcommand{\zeroToZero}{\ensuremath{\boldsymbol{0^0}}}
\newcommand{\inftyToZero}{\ensuremath{\boldsymbol{\infty^0}}}


\newcommand{\numOverZero}{\ensuremath{\boldsymbol{\tfrac{\#}{0}}}}
\newcommand{\dfn}{\textbf}
%\newcommand{\unit}{\,\mathrm}
\newcommand{\unit}{\mathop{}\!\mathrm}
%\newcommand{\eval}[1]{\bigg[ #1 \bigg]}
\newcommand{\eval}[1]{ #1 \bigg|}
\newcommand{\seq}[1]{\left( #1 \right)}
\renewcommand{\epsilon}{\varepsilon}
\renewcommand{\iff}{\Leftrightarrow}

\DeclareMathOperator{\arccot}{arccot}
\DeclareMathOperator{\arcsec}{arcsec}
\DeclareMathOperator{\arccsc}{arccsc}
\DeclareMathOperator{\si}{Si}
\DeclareMathOperator{\proj}{proj}
\DeclareMathOperator{\scal}{scal}
\DeclareMathOperator{\cis}{cis}
\DeclareMathOperator{\Arg}{Arg}
%\DeclareMathOperator{\arg}{arg}
\DeclareMathOperator{\Rep}{Re}
\DeclareMathOperator{\Imp}{Im}
\DeclareMathOperator{\sech}{sech}
\DeclareMathOperator{\csch}{csch}
\DeclareMathOperator{\Log}{Log}

\newcommand{\tightoverset}[2]{% for arrow vec
  \mathop{#2}\limits^{\vbox to -.5ex{\kern-0.75ex\hbox{$#1$}\vss}}}
\newcommand{\arrowvec}{\overrightarrow}
\renewcommand{\vec}{\mathbf}
\newcommand{\veci}{{\boldsymbol{\hat{\imath}}}}
\newcommand{\vecj}{{\boldsymbol{\hat{\jmath}}}}
\newcommand{\veck}{{\boldsymbol{\hat{k}}}}
\newcommand{\vecl}{\boldsymbol{\l}}
\newcommand{\utan}{\vec{\hat{t}}}
\newcommand{\unormal}{\vec{\hat{n}}}
\newcommand{\ubinormal}{\vec{\hat{b}}}

\newcommand{\dotp}{\bullet}
\newcommand{\cross}{\boldsymbol\times}
\newcommand{\grad}{\boldsymbol\nabla}
\newcommand{\divergence}{\grad\dotp}
\newcommand{\curl}{\grad\cross}
%% Simple horiz vectors
\renewcommand{\vector}[1]{\left\langle #1\right\rangle}


\outcome{Determine maxima and minima of a surface.}

\title{3.7 Maxima and Minima}



\begin{document}

\begin{abstract}
In this section we determine local maxima and minima of a surface.
\end{abstract}

\maketitle

\begin{definition}[Critical Point]
A critical point for the function $f(x,y)$ is a point $(x_0, y_0)$ in the domain of $f$ for which either
\[
\grad f(x_0, y_0) = \vec 0
\]
or one of $f_x$ or $f_y$ (or both) is undefined at $(x_0, y_0)$.
\end{definition}

\begin{example}[Example 1]
Find the critical points of the function $f(x,y) = x^2 + y^2 - 6y$.\\
We set the gradient equal to the zero vector and we solve for $x$ and $y$:
\[
\grad f(x,y) = \vector{2x, 2y -6} = \vector{0,0}
\]
which gives $x = 0$ and $y = 3$.  So the only critical point of $f$ is $(0,3)$.
\end{example}

\begin{problem}(Problem 1)
Find the critical points of the function $f(x,y) = x+ y - xy$.
\end{problem}




\begin{example}[Example 2]
Find the critical points of the function $f(x,y) = 2x^2 - y^2 +2xy + 3x + y+ 5$.\\
We set the gradient equal to the zero vector and we solve for $x$ and $y$:
\[
\grad f(x,y) = \vector{4x+2y + 3, 2x - 2y + 1} = \vector{0,0}
\]
which gives the $2 \times 2$ linear system:
\begin{align*}
4x+2y + 3 &= 0\\
2x -2y + 1 & = 0
\end{align*}
Adding the equations gives 
\[
6x + 4 = 0
\]
so $x = -2/3$. Plugging this into either of the original two equations gives $y = -1/6$.
Hence, $f$ has one critical point: $(-2/3, -1/6)$.
\end{example}

\begin{problem}(Problem 2)
Find the critical points of the function $f(x,y) = xy - 2x-2y-x^2 - y^2$.
\end{problem}


\begin{example}[Example 3]
Find the critical points of the function $f(x,y) = x^2y +xy^2 +3x + 4$.\\
We set the gradient equal to the zero vector and we solve for $x$ and $y$:
\[
\grad f(x,y) = \vector{2xy + y^2 + 3, x^2 + 2xy} = \vector{0,0}
\]
which gives the $2 \times 2$ system:
\begin{align*}
2xy + y^2 +3  &= 0\\
x^2 + 2xy & = 0
\end{align*}
Solving the bottom equation for $y$ gives $y = -\frac{x}{2}$ and substituting this into the top equation give
\[
2x\left(-\frac{x}{2}\right) + \left(-\frac{x}{2}\right)^2 +3 = 0
\]
which simplifies to
\[
-x^2 + \frac14 x^2 + 3 =0
\]
\[
\frac34 x^2 = 3
\]
\[
x^2 = 4
\]
and finally, we have $x=\pm 2$. Since $y = -x/2$, when $x=2$ we have $y=-1$ and when $x = -2$ we have $y = 1$.
Thus the two critical points are
\[
(2,-1) \quad \text{and} \quad (-2, 1)
\]
\end{example}

\begin{problem}(Problem 3)
Find the critical points of the function $f(x,y) = 12xy - x^3 - y^3$.
\end{problem}



\begin{example}[Example 4]
Find the critical points of the function $f(x,y) = x^4 + y^3 + 2x^2y^2 -27y + 1$.\\
We set the gradient equal to the zero vector and we solve for $x$ and $y$:
\[
\grad f(x,y) = \vector{4x^3 + 4xy^2, 3y^2 + 4x^2y -27} = \vector{0,0}
\]
which gives the $2 \times 2$ system:
\begin{align*}
4x^3 + 4xy^2 &= 0\\
3y^2 + 4x^2y -27 &= 0
\end{align*}
The first equation can be written as $4x(x^2 + y^2) = 0$ from which we can deduce that $x = 0$ or $(x, y) = (0,0)$. 
Letting $x = 0$ in the bottom equation gives
\[
3y^2 - 27 = 0
\]
from which we can conclude that $y = \pm 3$. We also see that $(0,0)$ is not a critical point.
Hence the critical points are $(3,0)$ and $(-3,0)$.
\end{example}

\begin{problem}(Problem 4)
Find the critical points of the function $f(x,y) = 2-x^4 + 2x^2 - y^2$.
\end{problem}

\begin{theorem}[Fermat's Theorem]
If $f(x,y)$ has a local extreme at the point $(x_0, y_0)$, then $(x_0, y_0)$ is a critical point for $f(x,y)$.
\end{theorem}

To determine the nature of $f(x,y)$ at a critical point, we use the second derivatives of $f$.

\begin{theorem}[Second Derivatives Test]
Let $(x_0, y_0)$ be a critical point for the function $f(x,y)$, and consider the discriminant,
\[
D(x,y) = f_{xx}(x,y) \cdot f_{yy}(x,y) - \left(f_{xy}(x,y)\right)^2
\]
1) If $D(x_0, y_0) > 0$ and $f_{xx}(x_0, y_0) > 0$, then $f(x,y)$ has a local minimum at $(x_0, y_0)$.\\
2) If $D(x_0, y_0) > 0$ and $f_{xx}(x_0, y_0)< 0$, then $f(x,y)$ has a local maximum at $(x_0, y_0)$.\\
3) If $D(x_0, y_0) < 0$, then $(x_0, y_0)$t is a saddle point.\\
4) If $D(x_0, y_0) = 0$, then there is no conclusion about $f(x,y)$ at $(x_0, y_0)$.\\
\end{theorem}

\begin{remark}
In parts 1 and 2 of the Second Derivatives Test, the condition $f_{xx}(x_0, y_0) > 0$ can be 
replaced with $f_{yy}(x_0, y_0) > 0$ since they both have the same sign when the discriminant is positive.
\end{remark}

\begin{example}[Example 5]
Determine the local extrema of  $f(x,y) = x^2 + y^2 - 6y$.\\
In Example 1, we found that the only critical point of $f$ is $(0,3)$.
The second derivatives are
\[
f_{xx} = 2,\; f_{yy} = 2 \quad \text{and} \quad f_{xy} = 0
\]
The discriminant is
\[
D(x,y) = f_{xx} f_{yy} - f_{xy}^2 = 4
\]
and at the critical point, it is
\[
D(0,3) = 4
\]
Since $D(0, 3) > 0$ and $f_{xx}(0,3) = 2 > 0$, the Second Derivatives Test tells us that $f(x,y) = x^2 + y^2 - 6y$
has a local minimum at the critical point $(0,3)$.
\end{example}

\begin{problem}(Problem 5)
Determine the local extrema of $f(x,y) = x+ y - xy$ (from Problem 1).\\
\end{problem}

\begin{example}[Example 6]
Determine the local extrema of $f(x,y) = 2x^2 - y^2 +2xy + 3x + y+ 5$.\\
In Example 2, we found that the only critical point of $f$ is $(-2/3,-1/6)$.
The second derivatives are
\[
f_{xx} = 4,\; f_{yy} = -2 \quad \text{and} \quad f_{xy} = 2
\]
The discriminant is
\[
D(x,y) = f_{xx} f_{yy} - f_{xy}^2 = -8 -4 = -12
\]
and at the critical point, it is
\[
D(-2/3,-1/6) = -12
\]
Since $D(-2/3,-1/6) < 0$, the Second Derivatives Test tells us that $f(x,y) = 2x^2 - y^2 +2xy + 3x + y+ 5$
has a saddle point at the critical point $(-2/3,-1/6)$.
\end{example}

\begin{problem}(Problem 6)
Determine the local extrema of $f(x,y) = xy - 2x-2y-x^2 - y^2$ (from Problem 2).\\
\end{problem}


\begin{example}[Example 7]
Determine the local extrema of  $f(x,y) = x^2y +xy^2 +3x + 4$.\\
In Example 3, we found that the critical points of $f$ are $(2,-1)$ and $(-2, 1)$.
The second derivatives are
\[
f_{xx} = 2y,\; f_{yy} = 2x \quad \text{and} \quad f_{xy} = 2x + 2y
\]
The discriminant is
\[
D(x,y) = f_{xx} f_{yy} - f_{xy}^2 = 4xy - (2x+2y)^2 = 4xy - 4x^2 - 8xy - 4y^2 = -4(x+y)^2
\]
and at the critical points, it is
\[
D(2, -1) = -4 \quad \text \quad D(-2, 1) = -4
\]
Since $D < 0$, at both critical points, the Second Derivatives Test tells us that $f(x,y) = x^2y +xy^2 +3x + 4$
has saddle points at both of the critical points $(2,-1)$ and $(-2, 1)$.
\end{example}


\begin{problem}(Problem 7)
Determine the local extrema of $f(x,y) = 12xy - x^3 - y^3$ (from Problem 3).\\
\end{problem}



\begin{example}[Example 8]
Determine the local extrema of $f(x,y) = x^4 + y^3 + 2x^2y^2 -27y + 1$.\\
In Example 4, we found that the critical points of $f$ are $(3,0)$ and $(-3,0)$.
The second derivatives are
\[
f_{xx} = 12x^2 + 8y^2,\; f_{yy} = 6y + 8x^2 \quad \text{and} \quad f_{xy} = 8xy
\]
The discriminant is
\[
D(x,y) = f_{xx} f_{yy} - f_{xy}^2 = (12x^2 + 8y^2)(6y + 8x^2) - 64x^2y^2
\]
and at the critical points, it is
\[
D(3, 0) = D(-3,0) = (108)(72) - (0)^2 = 7776
\]
\end{example}

\begin{problem}(Problem 8)
Determine the local extrema of $f(x,y) = 2-x^4 + 2x^2 - y^2$ (from Problem 4).\\
\end{problem}
\end{document}
