\documentclass[handout]{ximera}

%% You can put user macros here
%% However, you cannot make new environments



\newcommand{\ffrac}[2]{\frac{\text{\footnotesize $#1$}}{\text{\footnotesize $#2$}}}
\newcommand{\vasymptote}[2][]{
    \draw [densely dashed,#1] ({rel axis cs:0,0} -| {axis cs:#2,0}) -- ({rel axis cs:0,1} -| {axis cs:#2,0});
}


\graphicspath{{./}{firstExample/}}
\usepackage{forest}
\usepackage{amsmath}
\usepackage{amssymb}
\usepackage{array}
\usepackage[makeroom]{cancel} %% for strike outs
\usepackage{pgffor} %% required for integral for loops
\usepackage{tikz}
\usepackage{tikz-cd}
\usepackage{tkz-euclide}
\usetikzlibrary{shapes.multipart}


%\usetkzobj{all}
\tikzstyle geometryDiagrams=[ultra thick,color=blue!50!black]


\usetikzlibrary{arrows}
\tikzset{>=stealth,commutative diagrams/.cd,
  arrow style=tikz,diagrams={>=stealth}} %% cool arrow head
\tikzset{shorten <>/.style={ shorten >=#1, shorten <=#1 } } %% allows shorter vectors

\usetikzlibrary{backgrounds} %% for boxes around graphs
\usetikzlibrary{shapes,positioning}  %% Clouds and stars
\usetikzlibrary{matrix} %% for matrix
\usepgfplotslibrary{polar} %% for polar plots
\usepgfplotslibrary{fillbetween} %% to shade area between curves in TikZ



%\usepackage[width=4.375in, height=7.0in, top=1.0in, papersize={5.5in,8.5in}]{geometry}
%\usepackage[pdftex]{graphicx}
%\usepackage{tipa}
%\usepackage{txfonts}
%\usepackage{textcomp}
%\usepackage{amsthm}
%\usepackage{xy}
%\usepackage{fancyhdr}
%\usepackage{xcolor}
%\usepackage{mathtools} %% for pretty underbrace % Breaks Ximera
%\usepackage{multicol}



\newcommand{\RR}{\mathbb R}
\newcommand{\R}{\mathbb R}
\newcommand{\C}{\mathbb C}
\newcommand{\N}{\mathbb N}
\newcommand{\Z}{\mathbb Z}
\newcommand{\dis}{\displaystyle}
%\renewcommand{\d}{\,d\!}
\renewcommand{\d}{\mathop{}\!d}
\newcommand{\dd}[2][]{\frac{\d #1}{\d #2}}
\newcommand{\pp}[2][]{\frac{\partial #1}{\partial #2}}
\renewcommand{\l}{\ell}
\newcommand{\ddx}{\frac{d}{\d x}}

\newcommand{\zeroOverZero}{\ensuremath{\boldsymbol{\tfrac{0}{0}}}}
\newcommand{\inftyOverInfty}{\ensuremath{\boldsymbol{\tfrac{\infty}{\infty}}}}
\newcommand{\zeroOverInfty}{\ensuremath{\boldsymbol{\tfrac{0}{\infty}}}}
\newcommand{\zeroTimesInfty}{\ensuremath{\small\boldsymbol{0\cdot \infty}}}
\newcommand{\inftyMinusInfty}{\ensuremath{\small\boldsymbol{\infty - \infty}}}
\newcommand{\oneToInfty}{\ensuremath{\boldsymbol{1^\infty}}}
\newcommand{\zeroToZero}{\ensuremath{\boldsymbol{0^0}}}
\newcommand{\inftyToZero}{\ensuremath{\boldsymbol{\infty^0}}}


\newcommand{\numOverZero}{\ensuremath{\boldsymbol{\tfrac{\#}{0}}}}
\newcommand{\dfn}{\textbf}
%\newcommand{\unit}{\,\mathrm}
\newcommand{\unit}{\mathop{}\!\mathrm}
%\newcommand{\eval}[1]{\bigg[ #1 \bigg]}
\newcommand{\eval}[1]{ #1 \bigg|}
\newcommand{\seq}[1]{\left( #1 \right)}
\renewcommand{\epsilon}{\varepsilon}
\renewcommand{\iff}{\Leftrightarrow}

\DeclareMathOperator{\arccot}{arccot}
\DeclareMathOperator{\arcsec}{arcsec}
\DeclareMathOperator{\arccsc}{arccsc}
\DeclareMathOperator{\si}{Si}
\DeclareMathOperator{\proj}{proj}
\DeclareMathOperator{\scal}{scal}
\DeclareMathOperator{\cis}{cis}
\DeclareMathOperator{\Arg}{Arg}
%\DeclareMathOperator{\arg}{arg}
\DeclareMathOperator{\Rep}{Re}
\DeclareMathOperator{\Imp}{Im}
\DeclareMathOperator{\sech}{sech}
\DeclareMathOperator{\csch}{csch}
\DeclareMathOperator{\Log}{Log}

\newcommand{\tightoverset}[2]{% for arrow vec
  \mathop{#2}\limits^{\vbox to -.5ex{\kern-0.75ex\hbox{$#1$}\vss}}}
\newcommand{\arrowvec}{\overrightarrow}
\renewcommand{\vec}{\mathbf}
\newcommand{\veci}{{\boldsymbol{\hat{\imath}}}}
\newcommand{\vecj}{{\boldsymbol{\hat{\jmath}}}}
\newcommand{\veck}{{\boldsymbol{\hat{k}}}}
\newcommand{\vecl}{\boldsymbol{\l}}
\newcommand{\utan}{\vec{\hat{t}}}
\newcommand{\unormal}{\vec{\hat{n}}}
\newcommand{\ubinormal}{\vec{\hat{b}}}

\newcommand{\dotp}{\bullet}
\newcommand{\cross}{\boldsymbol\times}
\newcommand{\grad}{\boldsymbol\nabla}
\newcommand{\divergence}{\grad\dotp}
\newcommand{\curl}{\grad\cross}
%% Simple horiz vectors
\renewcommand{\vector}[1]{\left\langle #1\right\rangle}


\outcome{In this section we describe planes in space analytically.}

\title{1.7 Planes in Space}
%Vectors are represented graphically by arrows.
%and in three dimensions we write $\vec{v} = \vector{x,y, z}$.
%The length of the arrow represents the magnitude of the vector and the arrow points in the direction of the vector.
\begin{document}

\begin{abstract}
In this section we describe planes in space analytically.
\end{abstract}
 
\maketitle

A plane in $\R^3$ is determined by three non collinear points. 
However, this fundamental description is not the best starting point for creating the equation of a plane.
Better is to recognize that a plane, $P$, in $\R^3$ is determined by a single point $Q(x_0, y_0, z_0)$ and a 
vector $\vec{n} = \vector{a, b, c}$ which is perpendicular to the plane. 
The vector $\vec{n}$ is called the \textbf{normal vector} to the plane.
In addition to orthogonal, normal is yet another word for perpendicular.
See the figure below.

\begin{image}
\begin{tikzpicture}[rotate = 10]
\filldraw[fill=blue!20!white, draw=blue, ultra thin] (0, 0) -- (5, 0) -- (8, 2) -- (3, 2) --(0,0);
\draw[thick, red!50!white, fill] (4, 1) circle (0.05) node[right, red!80!white]{$Q$};
\draw[thick, red!70!white, ->] (4,1) -- (4, 4) node[right]{$\vec{n}$};
\draw[thick] (3.6, 1) -- (3.6, 1.4) -- (4, 1.4);
\node[blue] at (8, 1.5){$P$};
\node at (3.5, -1.6) {The vector $\vec{n}$ is normal to the plane $P$ at the point $Q$};
\end{tikzpicture}
\end{image}

Conceptualizing a plane using a point and a normal vector makes it remarkably easy to construct the equation of the plane.
Consider a point $R(x, y, z)$ in $\R^3$.
This point is in the plane $P$ passing through the point $Q(x_0, y_0, z_0)$ with normal vector $\vec{n} =\vector{a, b, c}$
if and only if the vector $\avec{QR}$ is orthogonal to the normal vector $\vec{n}$.
This condition can be written in terms of the dot product.
The point $R(x, y, z)$ must satisfy the equation
\[
\avec{QR} \dotp \vec{n} = 0
\]
This amounts to 
\[
\vector{x - x_0, y-y_0, z-z_0} \dotp \vector{a, b, c} = 0
\]
From the definition of the dot product we can write
\[
a(x-x_0) + b(y-y_0) + c(z-z_0) = 0
\]
Hence we have the equation of a plane, and we can see how nicely the ingredients $\vec{n} = \vector{a,b,c}$
and $Q(x_0, y_0, z_0)$ fit into it.
Sometimes we prefer to distribute the $a, b$ and $c$ and write the constant on the other side of the equation:
\[
ax + by + cz = ax_0 + by_0 + cz_0
\]
Finally, we replace the clunky constant with $d$ to write the elegant
\[
ax + by + cz = d
\]
Thus a linear equation in three variables gives us the equation of a plane in $\R^3$.
We summarize the results of the preceding computations in the following proposition.

\begin{proposition}[Equation of a Plane in $\R^3$]
Suppose that the vector $\vec{n} = \vector{a,b,c}$ is normal to the plane $P$ passing through the point $Q(x_0, y_0, z_0)$.
Then the equation of the plane can be written in the following ways:
\[
\vec{n} \dotp \vector{x-x_0, y-y_0, z-z_0} = 0 %\quad \text{(Canonical vector form)}
\]
\[
a(x-x_0) + b(y-y_0) + c(z-z_0) = 0 %\quad \text{(Canonical linear form)}
\]
Using $d = ax_0 + by_0 + cz_0$,  we can write the equations of the plane more succinctly:
\[
\vec{n} \dotp \vector{x, y, z} = d %\quad \text{(Vector form)}
\]
\[
ax+by+cz = d 
\]
The equation $ax+by+cz =d$ is called the \textbf{standard form} for the equation of a plane.
\end{proposition}

\begin{example}[Example 1] 
Find the equation of the plane through the point $(3, 5, -7)$ with normal vector $\vector{1, -1, 2}$.\\
A point $(x, y, z)$ is on the plane if and only if 
\[
\vector{1, -1, 2} \dotp \vector{x - 3, y-5, z+7} =0
\]
Hence, the equation of the plane is
\[
(x-3) - (y-5) + 2(z+7) = 0
\]
or, in standard form,
\[
x - y + 2z = -16
\]
\end{example}

\begin{problem}(Problem 1a)
Find the equation of the plane through the point $(4, -2, 6)$ with normal vector $\vector{2, -3, -4}$\\
The equation of the plane in standard form is $\answer{2x-3y-4z} = \answer{-10}$
\end{problem}

\begin{problem}(Problem 1b)
Find the equation of the plane through the point $(8, 0, -3)$ which is parallel to the plane $3x - y + 6z = 14$\\
\begin{hint}
Parallel planes have the same normal vector
\end{hint}
The equation of the plane in standard form is $\answer{3x-y+6z} = \answer{6}$
\end{problem}

\begin{problem}(Problem 1c)
Find the equation of the plane through the points $P(1, 4, 9), Q(4, -1, -7)$ and $R(5, 0, -3)$\\
\begin{hint}
The vectors $\avec{PQ}$ and $\avec{PR}$ lie in the plane
\end{hint}
\begin{hint}
Use the cross product to find the normal vector
\end{hint}
The equation of the plane in standard form is $\answer{x+7y-2z} = 11$
\begin{hint}
Simplify the coefficients in your answer
\end{hint}
\end{problem}


To sketch a plane in $\R^3$, plot its $x, y$ and $z$-intercepts.
% it is helpful to find the intersection of the plane with each of the coordinate planes.  
%These are called the {\bf traces} of the plane.


\begin{example}[Example 2]
Sketch the plane $P$ given by $2x + y + 3z = 6$ by finding its intercepts with each of the coordinate axes.\\
%sketching its traces.\\
To find the $x$-intercept, set $y$ and $z$ equal to $0$:
\[
2x = 6 \quad \Rightarrow \quad x = 3
\]
Similarly the $y$ and $z$-intercepts are $y=6$ and $z = 2$ respectively.
The graph of the plane contains the triangle formed by the intercepts as well as the intercepts themselves.
See the figure below.

%The figure below shows the three intercepts along with the triangle formed by them.

%To find the trace of $P$ in the $xy$-plane, we set $z = 0$ and we get
%\[
%2x +y = 6
%\]
%which is the equation of a line in the $xy$-plane.
%To find the trace of $P$ in the $xz$-plane, we set $y = 0$ and we get
%\[
%2x +3z = 6
%\]
%which is the equation of a line in the $xz$-plane
%To find the trace of $P$ in the $yz$-plane, we set $x = 0$ and we get
%\[
% y + 3z = 6
%\]
%which is also the equation of a line in the $yz$-plane.
%Sketching these three lines gives a triangle which determines the plane.
%See the figure below.

\begin{image}
\begin{tikzpicture}
\filldraw[blue!20!white] (-4, -4)--  (8.2,-0.8) -- (3.7, 3.1) -- (-5.5, 1.2) -- (-4,-4);
\filldraw[blue!30!white] (5.7, 0) -- (0, 2) -- (-2.4, -1.8) -- (5.7,0);
\draw[thick, ->] (0,0) -- (6.3,0) node[right]{$y$};
\draw[thick, ->] (0,0) -- (0,4.4) node[above]{$z$};
\draw[thick, ->] (0,0) -- (-3.6,-2.7) node[below, left]{$x$};
\node at (2.5, -4){The plane $2x+y+3z =6$} ;

\draw[blue] (5.7, 0) -- (0, 2) -- (-2.4, -1.8) -- (5.7,0);

\draw[blue, fill] (5.7, 0) circle (0.05) node[below]{$6$};
\draw[blue, fill] (0,2) circle (0.05) node[left]{$2$};
\draw[blue, fill] (-2.4, -1.8) circle (0.05) node[below right]{$3$};
\end{tikzpicture}
\end{image}

\end{example}

\begin{problem}(Problem 2)
Sketch the plane $x - 2y - 4z = 4$ by plotting its $x, y$ and $z$-intercepts.
\end{problem}

Two planes are either parallel or they intersect in a line.
Parallel planes are easy to detect, since they will have the same normal vector and hence their standard forms
will have the same left hand side, i.e., the planes
\[
ax+by+cz = d_1 \quad \text{and} \quad ax+by+cz = d_2
\]
are parallel.


\begin{example}[Example 3]
Find the intersection of the planes 
\[
2x + y + 3z = 6 \quad \text{and} \quad x - y + z = 2
\]
If we can find two points in the intersection, then we can write the vector form of the equation of the line of intersection.
We will choose a $z$-value for each point and then determine the corresponding $x$ and $y$-coordinates in the 
intersection by solving a $2 \times 2$ linear system.\\
For the first point, we can choose $z = 0$. Substituting this into the equations of the planes yields the
following $2 \times 2$ linear system:
\begin{align*}
2x+y &= 6\\
x -y &= 2
\end{align*}
Observe that adding these equations will eliminate the variable $y$,  giving
\[
3x = 8
\]
and hence $x = \frac83$.  Plugging this into the bottom equation (or the top equation), we can find $y$:
\[
 \frac83 -y = 2 \quad \Rightarrow \quad y = \frac23
\]
Hence, one point of intersection of the two planes is $P\left(\frac83, \frac23, 0\right)$.\\
For the second point, we can choose $z = 1$. Substituting this into the equations of the planes yields the 
following $2 \times 2$ linear system:
\begin{align*}
2x + y &= 3\\
x - y &= 1
\end{align*}
Adding these equations gives
\[
3x = 4 \quad \Rightarrow \quad x = \frac43
\]
Using the bottom equation (since it is simpler), we find $y$:
\[
\frac43 - y = 1 \quad \Rightarrow \quad y = \frac13
\]
Hence, the second point of intersection is $Q\left(\frac43, \frac13, 1\right)$.
The direction vector of the line of intersection of the two planes is the vector
\[
\avec{PQ} = \vector{\frac43 - \frac83, \frac13 - \frac23, 1-0} = \vector{-\frac43, -\frac13, 1}
\]
Using the point $P$ and the direction vector $\avec{PQ}$, we can write the equation of the line of intersection:
\[
\vector{x, y, z} = \vector{\frac83, \frac23, 0} + t \vector{-\frac43, -\frac13, 1}
\]

\end{example}

\begin{problem}(Problem 3)
Find the intersection of the planes $3x - y -4z = 3$ and $x - 4y + z = 2$.\\
\end{problem}


Next, we find the distance between parallel planes.

\begin{example}[Example 4]
Find the distance between the planes 
\[
P_1: x +2y - z = 1 \quad \text{and} \quad P_2: x+2y-z = 4
\]
The planes are parallel since they have the same normal vector, $\vec{n} = \vector{1, 2, -1}$. 
We need one point in each plane and the vector between them.
On $P_1$ we can take the point $Q_1(1, 0, 0)$ and on $P_2$, we can take the point $Q_2(4, 0, 0)$. 
The vector between them is 
\[
\vec{v} = \avec{Q_1Q_2} = \vector{3,0,0} = 3\vec{i}
\]
The distance between the planes is the magnitude of the projection of the vector $\vec{v} = \avec{Q_1Q_2}$ onto the 
normal vector, $\vec{n}$. We have,
\begin{align*}
\text{distance} &= | \proj_{\vec{n}}\vec{v}| \\
               &= \left| \frac{\vec{v}\dotp \vec{n}}{\vec{n}\dotp \vec{n}} \vec{n}\right|\\
               &= \left|\frac{3}{6} \vector{1, 2, -1}\right|\\
               &= \left| \vector{\frac12, 1, -\frac12}\right|\\
               &= \sqrt{\left( \frac12\right)^2 + 1^2 + \left( -\frac12\right)^2}\\
               & = \sqrt{\frac32}
\end{align*}
See the figure below.

\begin{image}
\begin{tikzpicture}
\filldraw[blue!40!white] (-1,0) -- (6,0) -- (10, 2) node[blue!80!white, midway, below]{$P_1$} -- (3, 2) -- (-1,0);
\filldraw[blue!40!white] (0,4) -- (7,4) -- (11, 6) node[blue!80!white, midway, below]{$P_2$} -- (4, 6) -- (0,4);
\draw[fill] (4,1) circle (0.05) node[black, left]{$Q_1$};
\draw[fill] (7,5.5) circle (0.05) node[right]{$Q_2$};
\draw[fill] (4,5.5) circle (0.05) ;
\draw[dashed] (4, 5.5)--(7, 5.5);
\draw[thick, ->] (4, 1) -- (7, 5.5) node[midway,below right]{$\vec{v} = \avec{Q_1Q_2}$};
\draw[orange!80!white, ->, thick] (4, 1)  -- (4, 7) node[left]{$\vec{n}$};
\draw[thick, ->, red!80!white] (4, 1) -- (4, 5.5) node[midway, left]{$\proj_{\vec{n}}\vec{v}$};
\end{tikzpicture}
\end{image}

\end{example}

\begin{problem}(Problem 4a)
Find the distance between the planes 
\[
2x -5y + 3z = 6 \quad \text{and} \quad 2x-5y+3z = -2
\]
\end{problem}

\begin{problem}(Problem 4b)
Find the distance between the line $\vector{1, 2, 3} + t\vector{-4, 5, 6}$ and the plane $2x-2y+3z = -2$.\\
\begin{hint}
Verify that the line and the plane do not intersect by substituting the parametric equations 
of the line into the equation of the plane
\end{hint}
\begin{hint}
Find any point on the line and compute its distance to the plane
\end{hint}
\end{problem}

\begin{problem}(Problem 4c)
Find the distance between the skew lines 
\[
\vector{1, 2, 3} + t\vector{4, 5, 6}\quad \text{and} \quad \vector{3, 2, 1} + t\vector{6, 5, -4}
\]
\begin{hint}
Skew lines are contained in parallel planes
\end{hint}
\begin{hint}
Use the cross product of the direction vectors to find the normal vector
\end{hint}
\begin{hint}
Write the equations of the parallel planes using one point from each plane and the normal vector
\end{hint}
\end{problem}

\begin{problem}(Problem 4d)
Find the angle between the planes
\[
4x + 2y - z = 11 \quad \text{and} \quad 9x - 3y + 4z = 1
\]
\begin{hint}
The angle between two planes is defined as the angle between their normal vectors
\end{hint}
\end{problem}

\end{document}





























The pic (below) compiles but not sure if it bakes in xake
\begin{tikzpicture}
\begin{scope}[decoration={markings,
    mark=at position 0.25 with {\arrow[blue,very thick]{>}},
    mark=at position 0.75 with {\arrowreversed[blue,very thick]{>}}}
              ]
\draw[postaction={decorate},thick,->] (0, -3) -- (0,3) node [black,above] {$y$};
\end{scope}
\begin{scope}[decoration={markings,
    mark=at position 0.25 with {\arrowreversed[blue,very thick]{>}},
    mark=at position 0.75 with {\arrow[blue,very thick]{>}}}
              ]
\draw[postaction={decorate},thick,->] (-3, 0) -- (3,0) node [black,right] {$x$};
\end{scope}
\begin{scope}[decoration={markings,
    mark=at position 22mm with {\arrow[red,very thick]{>}}}
              ]
\draw[postaction={decorate},thick] plot[domain=0.33:3]      (\x,{1/\x});
\draw[postaction={decorate},thick] plot[domain=-0.33:-3]    ({\x},{1/\x});
\draw[postaction={decorate},thick] plot[domain=0.33:3]      (\x,{-1/\x});
\draw[postaction={decorate},thick] plot[domain=-0.33:-3]    (\x,{-1/\x});
\end{scope}
    \end{tikzpicture}
    
    



The next few pics all bake in xake!

\begin{tikzpicture}[
tangent/.style={
decoration={
markings,% switch on markings
mark=
at position #1
with
{
\coordinate (tangent point-\pgfkeysvalueof{/pgf/decoration/mark info/sequence number}) at (0pt,0pt);
\coordinate (tangent unit vector-\pgfkeysvalueof{/pgf/decoration/mark info/sequence number}) at (1,0pt);
\coordinate (tangent orthogonal unit vector-\pgfkeysvalueof{/pgf/decoration/mark info/sequence number}) at (0pt,1);
}
},
postaction=decorate
},
use tangent/.style={
shift=(tangent point-#1),
x=(tangent unit vector-#1),
y=(tangent orthogonal unit vector-#1)
},
use tangent/.default=1
]

\draw[->] (-.5,0) -- (6.5,0) node(xline)[right] {$x$};
\draw[->] (0,-.5) -- (0,5) node(yline)[right] {$y$};


\draw (-.5,1.2) to[out=0, in=180] coordinate[pos=.5] (A) (5,2.8);
\draw[dashed] (A) -- (A |- xline) node[below] {$a$};

\end{tikzpicture}


\begin{tikzpicture}[
tangent/.style={
decoration={
markings,% switch on markings
mark=
at position #1
with
{
\coordinate (tangent point-\pgfkeysvalueof{/pgf/decoration/mark info/sequence number}) at (0pt,0pt);
\coordinate (tangent unit vector-\pgfkeysvalueof{/pgf/decoration/mark info/sequence number}) at (1,0pt);
\coordinate (tangent orthogonal unit vector-\pgfkeysvalueof{/pgf/decoration/mark info/sequence number}) at (0pt,1);
}
},
postaction=decorate
},
use tangent/.style={
shift=(tangent point-#1),
x=(tangent unit vector-#1),
y=(tangent orthogonal unit vector-#1)
},
use tangent/.default=1
]

\draw[->] (-.5,0) -- (6.5,0) node(xline)[right] {$x$};
\draw[->] (0,-.5) -- (0,5) node(yline)[right] {$y$};


\draw (.5,4) to[out=-90, in=90] coordinate[pos=.5] (B) (4,-.5);
\draw[dashed] (B) -- (B |- xline) node[below] {$b$};

\end{tikzpicture}

\begin{tikzpicture}[
tangent/.style={
decoration={
markings,% switch on markings
mark=
at position #1
with
{
\coordinate (tangent point-\pgfkeysvalueof{/pgf/decoration/mark info/sequence number}) at (0pt,0pt);
\coordinate (tangent unit vector-\pgfkeysvalueof{/pgf/decoration/mark info/sequence number}) at (1,0pt);
\coordinate (tangent orthogonal unit vector-\pgfkeysvalueof{/pgf/decoration/mark info/sequence number}) at (0pt,1);
}
},
postaction=decorate
},
use tangent/.style={
shift=(tangent point-#1),
x=(tangent unit vector-#1),
y=(tangent orthogonal unit vector-#1)
},
use tangent/.default=1
]

\draw[tangent=0.4] (1,1) to[out=70, in=200] (4,4);
\filldraw[use tangent] (0,0) circle (2pt);
\draw[use tangent] (-1,0) -- (1,0);
\end{tikzpicture}

\begin{tikzpicture}[
tangent/.style={
decoration={
markings,% switch on markings
mark=
at position #1
with
{
\coordinate (tangent point-\pgfkeysvalueof{/pgf/decoration/mark info/sequence number}) at (0pt,0pt);
\coordinate (tangent unit vector-\pgfkeysvalueof{/pgf/decoration/mark info/sequence number}) at (1,0pt);
\coordinate (tangent orthogonal unit vector-\pgfkeysvalueof{/pgf/decoration/mark info/sequence number}) at (0pt,1);
}
},
postaction=decorate
},
use tangent/.style={
shift=(tangent point-#1),
x=(tangent unit vector-#1),
y=(tangent orthogonal unit vector-#1)
},
use tangent/.default=1
]

\draw[tangent=0.4] (.5,2)
to[out=-60,in=170] (1.5,1)
to[out=10,in=-120] (2.5,2);
\filldraw[use tangent] (0,0) circle (2pt);
\draw[use tangent] (-1,0) -- (1,0);

\end{tikzpicture}


\begin{tikzpicture}[
tangent/.style={
decoration={
markings,% switch on markings
mark=
at position #1
with
{
\coordinate (tangent point-\pgfkeysvalueof{/pgf/decoration/mark info/sequence number}) at (0pt,0pt);
\coordinate (tangent unit vector-\pgfkeysvalueof{/pgf/decoration/mark info/sequence number}) at (1,0pt);
\coordinate (tangent orthogonal unit vector-\pgfkeysvalueof{/pgf/decoration/mark info/sequence number}) at (0pt,1);
}
},
postaction=decorate
},
use tangent/.style={
shift=(tangent point-#1),
x=(tangent unit vector-#1),
y=(tangent orthogonal unit vector-#1)
},
use tangent/.default=1
]
\draw[tangent=0.0] (0,0) sin (1,1) cos (2,0);
\draw (0,0) sin (-1,-1) cos (-2,0);
\draw[use tangent] (-2,0) -- (2,0);
\end{tikzpicture}






\begin{tikzpicture}[
tangent/.style={
decoration={
markings,% switch on markings
mark=
at position #1
with
{
\coordinate (tangent point-\pgfkeysvalueof{/pgf/decoration/mark info/sequence number}) at (0pt,0pt);
\coordinate (tangent unit vector-\pgfkeysvalueof{/pgf/decoration/mark info/sequence number}) at (1,0pt);
\coordinate (tangent orthogonal unit vector-\pgfkeysvalueof{/pgf/decoration/mark info/sequence number}) at (0pt,1);
}
},
postaction=decorate
},
use tangent/.style={
shift=(tangent point-#1),
x=(tangent unit vector-#1),
y=(tangent orthogonal unit vector-#1)
},
use tangent/.default=1
]


\draw[->] (-.5,0) -- (6,0) node(xline)[right] {$x$};
\draw[->] (0,-.5) -- (0,6) node(yline)[right] {$y$};
\draw[tangent=0.2] (1,1) to[out=-30,in=270]
coordinate[pos=0.2] (A)
coordinate[pos=0.7] (B)
(5,5) node[right] {$y=f(x)$};
\draw[use tangent, name path=tan] (-2,0) -- (3.5,0) node[right] {$y=L(x)$};
\draw[dashed] (A) -- (A |- xline) node(E)[below] {$a$};
\draw[dashed, name path=horiz, shorten <= -.2cm]
(A -| yline) node[left, xshift=-.2cm] {$f(a)$} -- (A -| B);
\draw[dashed, name path=vert] (B) -- (B |- xline) node(F)[below] {$x$};
\fill[name intersections={of=horiz and vert}] (intersection-1) circle (.2pt) coordinate(D);
\fill[name intersections={of=tan and vert}] (intersection-1) circle (.2pt) coordinate(C);
\draw[
thick,
decoration={brace, mirror, raise=.2cm},
decorate
] (E) -- node[below, yshift=-.3cm] {$x-a$} (F);
\draw[
thick,
decoration={brace, raise=.2cm},
decorate
] (B) -- node[right, xshift=.2cm] {$E(x)$} (C);

\draw[
thick,
decoration={brace, raise=.2cm},
decorate
] (C) -- node[right, xshift=.2cm] {$f’(a)(x-a)$} (D);

\end{tikzpicture}



\begin{tikzpicture}[
tangent/.style={
decoration={
markings,% switch on markings
mark=
at position #1
with
{
\coordinate (tangent point-\pgfkeysvalueof{/pgf/decoration/mark info/sequence number}) at (0pt,0pt);
\coordinate (tangent unit vector-\pgfkeysvalueof{/pgf/decoration/mark info/sequence number}) at (1,0pt);
\coordinate (tangent orthogonal unit vector-\pgfkeysvalueof{/pgf/decoration/mark info/sequence number}) at (0pt,1);
}
},
postaction=decorate
},
use tangent/.style={
shift=(tangent point-#1),
x=(tangent unit vector-#1),
y=(tangent orthogonal unit vector-#1)
},
use tangent/.default=1
]


\draw[->] (-.5,0) -- (7,0) node(xline)[right] {$x$};
\draw[->] (0,-.5) -- (0,5) node(yline)[right] {$y$};
\draw[thick, blue, smooth, tangent=0.255, tangent=0.745] (1,2)
to[out=-60, in=170] coordinate[pos=0.5] (A) (2,1)
to[out=10, in=190] (5,4)
to[out=-10, in=120] coordinate[pos=0.5] (B) (6,3) node[right] {$y=f(x)$};
\fill (A) circle (2pt);
\fill (B) circle (2pt);
\draw[shorten <= -.5cm, shorten >= -.5cm] (A) -- (B);
\draw[thick, use tangent] (-1,0) -- coordinate (C) (1,0);
\draw[thick, use tangent=2] (-1,0) -- coordinate (D) (1,0);
\fill (C) circle (2pt);
\fill (D) circle (2pt);
\draw[dashed] (A) -- (A |- xline) node[below] {$a$};
\draw[dashed] (B) -- (B |- xline) node[below] {$b$};
\draw[dashed] (C) -- (C |- xline) node[below] {$c_1$};
\draw[dashed] (D) -- (D |- xline) node[below] {$c_2$};

\end{tikzpicture}











\begin{image}
\begin{tikzpicture}
\draw (0,0) sin (1,1);
\draw (2, 0) sin (3, -1);
\draw  (4,0) cos (5, 1);
\draw  (6,0) cos (7, -1);

\draw (0, 3) sin (1, 4) cos (2, 3) sin (3, 2) cos (4, 3);
\end{tikzpicture}
\end{image}

















\begin{align*}
\vec{n} \dotp \vector{x-x_0, y-y_0, z-z_0} &= 0 \\
\vec{n} \dotp \vector{x, y, z} &= d \\
a(x-x_0) + b(y-y_0) + c(z-z_0) &= 0\\
ax+by+cz &= d
\end{align*}
\[
\vec{n} \dotp \vector{x-x_0, y-y_0, z-z_0} = 0 \\
\]
\[
\vec{n} \dotp \vector{x, y, z} = d \\
\]
\[
a(x-x_0) + b(y-y_0) + c(z-z_0) = 0\\
\]
\[
ax+by+cz = d
\]
\begin{align*}
&\vec{n} \dotp \vector{x-x_0, y-y_0, z-z_0} = 0 \\
&\vec{n} \dotp \vector{x, y, z} = d \\
&a(x-x_0) + b(y-y_0) + c(z-z_0) = 0\\
&ax+by+cz = d
\end{align*}
\[
\vec{n} \dotp \vector{x-x_0, y-y_0, z-z_0} = 0 \\
\]
\[
a(x-x_0) + b(y-y_0) + c(z-z_0) = 0\\
\]
\[
\vec{n} \dotp \vector{x, y, z} = d \\
\]
\[
ax+by+cz = d
\]




\outcome{In this section we describe curves in space.}

\title{2.1 Space Curves}



\begin{abstract}
In this section we describe curves in space.
\end{abstract}

\maketitle





A curve in $\R^3$ is given by a vector valued function,
\[
\vec r(t) = \vector{f(t), g(t), h(t)}
\]
Note that this ia a vector equation.
 
 
 
 
 






