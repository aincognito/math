\documentclass[handout]{ximera}

%% You can put user macros here
%% However, you cannot make new environments



\newcommand{\ffrac}[2]{\frac{\text{\footnotesize $#1$}}{\text{\footnotesize $#2$}}}
\newcommand{\vasymptote}[2][]{
    \draw [densely dashed,#1] ({rel axis cs:0,0} -| {axis cs:#2,0}) -- ({rel axis cs:0,1} -| {axis cs:#2,0});
}


\graphicspath{{./}{firstExample/}}
\usepackage{forest}
\usepackage{amsmath}
\usepackage{amssymb}
\usepackage{array}
\usepackage[makeroom]{cancel} %% for strike outs
\usepackage{pgffor} %% required for integral for loops
\usepackage{tikz}
\usepackage{tikz-cd}
\usepackage{tkz-euclide}
\usetikzlibrary{shapes.multipart}


%\usetkzobj{all}
\tikzstyle geometryDiagrams=[ultra thick,color=blue!50!black]


\usetikzlibrary{arrows}
\tikzset{>=stealth,commutative diagrams/.cd,
  arrow style=tikz,diagrams={>=stealth}} %% cool arrow head
\tikzset{shorten <>/.style={ shorten >=#1, shorten <=#1 } } %% allows shorter vectors

\usetikzlibrary{backgrounds} %% for boxes around graphs
\usetikzlibrary{shapes,positioning}  %% Clouds and stars
\usetikzlibrary{matrix} %% for matrix
\usepgfplotslibrary{polar} %% for polar plots
\usepgfplotslibrary{fillbetween} %% to shade area between curves in TikZ



%\usepackage[width=4.375in, height=7.0in, top=1.0in, papersize={5.5in,8.5in}]{geometry}
%\usepackage[pdftex]{graphicx}
%\usepackage{tipa}
%\usepackage{txfonts}
%\usepackage{textcomp}
%\usepackage{amsthm}
%\usepackage{xy}
%\usepackage{fancyhdr}
%\usepackage{xcolor}
%\usepackage{mathtools} %% for pretty underbrace % Breaks Ximera
%\usepackage{multicol}



\newcommand{\RR}{\mathbb R}
\newcommand{\R}{\mathbb R}
\newcommand{\C}{\mathbb C}
\newcommand{\N}{\mathbb N}
\newcommand{\Z}{\mathbb Z}
\newcommand{\dis}{\displaystyle}
%\renewcommand{\d}{\,d\!}
\renewcommand{\d}{\mathop{}\!d}
\newcommand{\dd}[2][]{\frac{\d #1}{\d #2}}
\newcommand{\pp}[2][]{\frac{\partial #1}{\partial #2}}
\renewcommand{\l}{\ell}
\newcommand{\ddx}{\frac{d}{\d x}}

\newcommand{\zeroOverZero}{\ensuremath{\boldsymbol{\tfrac{0}{0}}}}
\newcommand{\inftyOverInfty}{\ensuremath{\boldsymbol{\tfrac{\infty}{\infty}}}}
\newcommand{\zeroOverInfty}{\ensuremath{\boldsymbol{\tfrac{0}{\infty}}}}
\newcommand{\zeroTimesInfty}{\ensuremath{\small\boldsymbol{0\cdot \infty}}}
\newcommand{\inftyMinusInfty}{\ensuremath{\small\boldsymbol{\infty - \infty}}}
\newcommand{\oneToInfty}{\ensuremath{\boldsymbol{1^\infty}}}
\newcommand{\zeroToZero}{\ensuremath{\boldsymbol{0^0}}}
\newcommand{\inftyToZero}{\ensuremath{\boldsymbol{\infty^0}}}


\newcommand{\numOverZero}{\ensuremath{\boldsymbol{\tfrac{\#}{0}}}}
\newcommand{\dfn}{\textbf}
%\newcommand{\unit}{\,\mathrm}
\newcommand{\unit}{\mathop{}\!\mathrm}
%\newcommand{\eval}[1]{\bigg[ #1 \bigg]}
\newcommand{\eval}[1]{ #1 \bigg|}
\newcommand{\seq}[1]{\left( #1 \right)}
\renewcommand{\epsilon}{\varepsilon}
\renewcommand{\iff}{\Leftrightarrow}

\DeclareMathOperator{\arccot}{arccot}
\DeclareMathOperator{\arcsec}{arcsec}
\DeclareMathOperator{\arccsc}{arccsc}
\DeclareMathOperator{\si}{Si}
\DeclareMathOperator{\proj}{proj}
\DeclareMathOperator{\scal}{scal}
\DeclareMathOperator{\cis}{cis}
\DeclareMathOperator{\Arg}{Arg}
%\DeclareMathOperator{\arg}{arg}
\DeclareMathOperator{\Rep}{Re}
\DeclareMathOperator{\Imp}{Im}
\DeclareMathOperator{\sech}{sech}
\DeclareMathOperator{\csch}{csch}
\DeclareMathOperator{\Log}{Log}

\newcommand{\tightoverset}[2]{% for arrow vec
  \mathop{#2}\limits^{\vbox to -.5ex{\kern-0.75ex\hbox{$#1$}\vss}}}
\newcommand{\arrowvec}{\overrightarrow}
\renewcommand{\vec}{\mathbf}
\newcommand{\veci}{{\boldsymbol{\hat{\imath}}}}
\newcommand{\vecj}{{\boldsymbol{\hat{\jmath}}}}
\newcommand{\veck}{{\boldsymbol{\hat{k}}}}
\newcommand{\vecl}{\boldsymbol{\l}}
\newcommand{\utan}{\vec{\hat{t}}}
\newcommand{\unormal}{\vec{\hat{n}}}
\newcommand{\ubinormal}{\vec{\hat{b}}}

\newcommand{\dotp}{\bullet}
\newcommand{\cross}{\boldsymbol\times}
\newcommand{\grad}{\boldsymbol\nabla}
\newcommand{\divergence}{\grad\dotp}
\newcommand{\curl}{\grad\cross}
%% Simple horiz vectors
\renewcommand{\vector}[1]{\left\langle #1\right\rangle}


\outcome{In this section we describe planes in space analytically.}

\title{1.7 Planes in Space}
%Vectors are represented graphically by arrows.
%and in three dimensions we write $\vec{v} = \vector{x,y, z}$.
%The length of the arrow represents the magnitude of the vector and the arrow points in the direction of the vector.
\begin{document}

\begin{abstract}
In this section we describe planes in space analytically.
\end{abstract}
 
\maketitle

A plane in $\R^3$ is determined by three non collinear points. 
However, this fundamental description is not the best starting point for creating the equation of a plane.
Better is to recognize that a plane, $P$, in $\R^3$ is determined by a single point $Q(x_0, y_0, z_0)$ and a 
vector $\vec{n} = \vector{a, b, c}$ which is perpendicular to the plane. 
The vector $\vec{n}$ is called the {\bf normal} vector to the plane.
In addition to orthogonal, normal is yet another word for perpendicular.
See the figure below.

\begin{image}
\begin{tikzpicture}
\filldraw[fill=blue!50!white, draw=blue] (0, 0) -- (5, 0) -- (8, 2) -- (3, 2) --(0,0);
\draw[red, fill] (4, 1) circle (0.05) node[right]{$Q$};
\draw[thick, red, ->] (4,1) -- (4, 4) node[right]{$\vec{n}$};
\draw (3.7, 1) -- (3.7, 1.3) -- (4, 1.3);
\node[blue] at (8, 1.5){$P$};
\node at (4, -1) {The vector $\vec{n}$ is normal to the plane $P$ at the point $Q$};
\end{tikzpicture}
\end{image}

Conceptualizing a plane using a point and a normal vector makes it remarkably easy to construct the equation of the plane.
Consider a point $R(x, y, z)$ in $\R^3$.
This point is in the plane $P$ passing through the point $Q(x_0, y_0, z_0)$ with normal vector $\vec{n} =\vector{a, b, c}$
if and only if the vector $\vector{QR}$ is orthogonal to the normal vector $\vec{n}$.
This condition can be written in terms of the dot product.
The point $R(x, y, z)$ must satisfy the equation
\[
\vec{QR} \dotp \vec{n} = 0
\]
This amounts to 
\[
\vector{x - x_0, y-y_0, z-z_0} \dotp \vector{a, b, c} = 0
\]
From the definition of the dot product we can write
\[
a(x-x_0) + b(y-y_0) + c(z-z_0) = 0
\]
Hence we have the equation of a plane, and we can see how nicely the ingredients $\vec{n} = \vector{a,b,c}$
and $Q(x_0, y_0, z_0)$ fit into it.
Sometimes we prefer to distribute the $a, b$ and $c$ and write the constant on the other side of the equation:
\[
ax + by + cz = ax_0 + by_0 + cz_0
\]
Finally, we replace the clunky constant with $d$ to write the elegant
\[
ax + by + cz = d
\]
Thus a linear equation in three variables gives us the equation of a plane in $\R^3$.
We summarize the results of the preceding computations in the following proposition.

\begin{proposition}[Equation of a Plane in $\R^3$]
Suppose that the vector $\vec{n} = \vector{a,b,c}$ is normal to the plane $P$ passing through the point $Q(x_0, y_0, z_0)$.
Then the equation of the plane can be written as
\[
\vec{n} \dotp \vector{x-x_0, y-y_0, z-z_0} = 0 \quad \text{(Canonical vector form)}
\]
\[
a(x-x_0) + b(y-y_0) + c(z-z_0) = 0 \quad \text{(Canonical linear form)}
\]
or using $d = ax_0 + by_0 + cz_0$,  we can write the more succinct
\[
\vec{n} \dotp \vector{x, y, z} = d \quad \text{(Standard vector form)}
\]
\[
ax+by+cz = d \quad \text{(Standard linear form)}
\]
\end{proposition}

To sketch a plane in $\R^3$ it is helpful to find the intersection of the plane with each of the coordinate planes.  
These are called the {\bf traces} of the plane.

\begin{example}
Sketch the plane $P$ given by $2x + y + 3z = 6$ by sketching its traces.\\
To find the trace of $P$ in the $xy$-plane, we set $z = 0$ and we get
\[
2x +y = 6
\]
which is the equation of a line in the $xy$-plane.
To find the trace of $P$ in the $xz$-plane, we set $y = 0$ and we get
\[
2x +3z = 6
\]
which is the equation of a line in the $xz$-plane
To find the trace of $P$ in the $yz$-plane, we set $x = 0$ and we get
\[
 y + 3z = 6
\]
which is also the equation of a line in the $yz$-plane.
Sketching these three lines gives a triangle which determines the plane.
See the figure below.

\begin{image}
\begin{tikzpicture}
\filldraw[blue!20!white] (-4, -4)--  (8.2,-0.8) -- (3.7, 3.1) -- (-5.5, 1.2) -- (-4,-4);
\filldraw[blue!30!white] (5.7, 0) -- (0, 2) -- (-2.4, -1.8) -- (5.7,0);
\draw[thick, ->] (0,0) -- (6.3,0) node[right]{$y$};
\draw[thick, ->] (0,0) -- (0,5.4) node[above]{$z$};
\draw[thick, ->] (0,0) -- (-3.6,-2.7) node[below, left]{$x$};
\node at (2.5, -4){The plane $2x+y+3z =6$ along with its traces} ;

\draw[blue] (5.7, 0) -- (0, 2) -- (-2.4, -1.8) -- (5.7,0);

\draw[blue, fill] (5.7, 0) circle (0.05) node[below]{$6$};
\draw[blue, fill] (0,2) circle (0.05) node[left]{$2$};
\draw[blue, fill] (-2.4, -1.8) circle (0.05) node[below right]{$3$};
\end{tikzpicture}
\end{image}

\end{example}

Two planes are either parallel or they intersect in a line.
Parallel planes are easy to detect, since they will have the same normal vector and hence their standard forms
will have the same left hand side
\[
ax+by+cz = d_1 \quad \text{and} \quad ax+by+cz = d_2
\]
are parallel planes if $d_1 \neq d_2$.

\begin{example}
Find the intersection of the planes $2x + y + 3z = 6$ and $x - y + z = 2$.\\
If we can find two points in the intersection, then we can write the vector form of the equation of the line of intersection.
We will choose two different $z$ values and determine the corresponding $x$ and $y$ coordinates.
If $z = 0$ the equations of the planes becomes:
\begin{align*}
2x+y &= 6\\
x -y &= 2
\end{align*}
Adding these equations gives
\[
3x = 8
\]
and hence $x = \frac83$.  Plugging this into the bottom equation, we can find $y$:
\[
x - y = 2 \Rightarrow \frac83 -y = 2 \Rightarrow y = \frac23
\]
Hence, one point of intersection of the two planes is $\left(\frac83, \frac23, 0\right)$.
Now we let $z = 1$ (for example) in both of the plane equations we get the system
\begin{align*}
2x + y &= 3\\
x - y &= 1
\end{align*}
Adding these gives
\[
3x = 4 \Rightarrow x = \frac43
\]
Again, using the bottom equation, we find $y$:
\[
x - y = 1 \Rightarrow \frac43 - y = 1 \Rightarrow y = \frac13
\]
Hence the second point of intersection is $\left(\frac43, \frac13, 1\right)$.
The direction vector of the line of intersection of the two planes is 
\[
\vec{v} = \vector{\frac43 - \frac83, \frac13 - \frac23, 1-0} = \vector{-\frac43, -\frac13, 1}
\]
Choosing the first point, we can write the equation of the line of intersection of the two planes in vector form as
\[
\vector{x, y, z} = \vector{\frac83, \frac23, 0} + t \vector{-\frac43, -\frac13, 1}
\]

\end{example}

Next, we find the distance between parallel planes.

\begin{example}
Find the distance between the planes $P_1: x +2y - z = 1$ and $P_2: x+2y-z = 4$.\\
The normal vector to both planes is $\vec{n} = \vector{1, 2, -1}$. We need one point in each plane and the vector between them.
On $P_1$ we can take the point $Q_1(1, 0, 0)$ and on $P_2$, we can take the point $Q_2(4, 0, 0)$. The vector between them is 
\[
\vec{v} = \vec{Q_1Q_2} = \vector{3,0,0} = 3\vec{i}
\]
The distance between the planes is the magnitude of the projection of the vector $\vec{v} = \vec{Q_1Q_2}$ onto the normal vector, $\vec{n}$. We have,
\begin{align*}
\text{distance} &= | \proj_{\vec{n}}\vec{v}| \\
               &= \left| \frac{\vec{v}\dotp \vec{n}}{\vec{n}\dotp \vec{n}} \vec{n}\right|\\
               &= \left|\frac{3}{9} \vector{1, 2, -1}\right|\\
               &= \left| \vector{\frac13, \frac23, -\frac13}\right|\\
               &= \sqrt{\left( \frac13\right)^2 + \left( \frac23\right)^2 + \left( \frac13\right)^2}\\
               & = \frac{\sqrt{6}}{3}
\end{align*}
\end{example}

\begin{image}
\begin{tikzpicture}
\filldraw[blue!40!white] (0,0) -- (6,0) -- (10, 2) node[blue!80!white, midway, below]{$P_1$} -- (4, 2) -- (0,0);
\filldraw[blue!40!white] (2,4) -- (8,4) -- (12, 6) node[blue!80!white, midway, below]{$P_2$} -- (6, 6) -- (2,4);
\draw[fill] (4,1.5) circle (0.05) node[black, left]{$Q_1$};
\draw[fill] (8,5.5) circle (0.05) node[right]{$Q_2$};
\draw[thick, ->] (4, 1.5) -- (8, 5.5) node[midway,below right]{$\vec{v} = \vec{Q_1Q_2}$};
\draw[thick, ->, orange!80!white] (4, 1.5) -- (4, 5.5) node[above]{$\proj_{\vec{n}}\vec{v}$};
\draw[red!80!white, ->, thick] (4, 1.5)  -- (4, 3) node[left]{$\vec{n}$};
\end{tikzpicture}
\end{image}

\end{document}


Distance between skew lines






\begin{align*}
\vec{n} \dotp \vector{x-x_0, y-y_0, z-z_0} &= 0 \\
\vec{n} \dotp \vector{x, y, z} &= d \\
a(x-x_0) + b(y-y_0) + c(z-z_0) &= 0\\
ax+by+cz &= d
\end{align*}
\[
\vec{n} \dotp \vector{x-x_0, y-y_0, z-z_0} = 0 \\
\]
\[
\vec{n} \dotp \vector{x, y, z} = d \\
\]
\[
a(x-x_0) + b(y-y_0) + c(z-z_0) = 0\\
\]
\[
ax+by+cz = d
\]
\begin{align*}
&\vec{n} \dotp \vector{x-x_0, y-y_0, z-z_0} = 0 \\
&\vec{n} \dotp \vector{x, y, z} = d \\
&a(x-x_0) + b(y-y_0) + c(z-z_0) = 0\\
&ax+by+cz = d
\end{align*}
\[
\vec{n} \dotp \vector{x-x_0, y-y_0, z-z_0} = 0 \\
\]
\[
a(x-x_0) + b(y-y_0) + c(z-z_0) = 0\\
\]
\[
\vec{n} \dotp \vector{x, y, z} = d \\
\]
\[
ax+by+cz = d
\]



\end{document}


To describe the line analytically, i.e., to find the equation (or equations) of the line,
we will need the vector associated with these two points and we will also need to take advantage of the end to end method for adding vectors.
Suppose the line $L$ in $\R^3$ goes through the points $P(x_1, y_1, z_1)$ and $Q(x_2, y_2, z_2)$. The vector from $P$ to $Q$ is given by
\[
\vec{v} = \vec{PQ} = \vector{x_2-x_1,y_2-y_1,z_2-z_1}
\]
This vector is called the direction vector of the line. 
A point on the line can be seen as the final point of and vector emanating from $P$ and parallel to $\vec{v}$.
To describe such points, we rely on the end to end method of vector addition. A point $(x, y, z)$ lies on the line $L$ if
it is the final point of a vector of the form:
\[
\vector{x_1, y_1, z_1} + t\vector{v}
\]
where $t$ is any scalar. See the figure below.

\begin{image}
\begin{tikzpicture}
\draw[ ->] (0,0) -- (1, 2) ;
\draw[red, ->] (1,2) -- (5, 3);
\draw[blue, fill] (3, 2.5) circle (0.05) node[above]{$Q$};
\draw[blue, fill] (1, 2) circle (0.05)node[above]{$P$};
\draw[blue, fill] (0,0) circle (0.05) node[left]{$O$};
\node at (2.5, -0.5) {The vector $\vector{x_1, y_1, z_1} + t\vec{v}$ (in red)};
\end{tikzpicture}
\end{image}
 
If we associate the vector $\vector{x, y, z}$ with the point $(x, y, z)$, then by the end to end method of adding vectors, we see that
\[
\vector{x, y, z} = \vector{x_1, y_1, z_1} + t\vec{v}
\]
To simplify the notation somewhat, if the vector $\vec{v}$ has the form $\vector{a, b, c}$, then this equation becomes
\[
\vector{x, y, z} = \vector{x_1, y_1, z_1} + t\vector{a, b, c}
\]
This is called the {\bf vector form} of the equation of the line $L$ in $\R^3$ which passes through the 
point $(x_1, y_1, z_1)$ and has direction vector $\vec{v} = \vector{a, b, c}$.\\
Isolating the variables $x, y$ and $z$ by equating the components of the vector form, we obtain the {\bf parametric form} $L$:
\begin{align*}
x &= x_1 + at\\
y &= y_1 + bt\\
z &= z_1 + ct
\end{align*}
Solving these equations for $t$ (assuming $a, b$ and $c$ are all non-zero) we obtain the {\bf symmetric form} of $L$:
\[
\frac{x-x_1}{a} = \frac{y-y_1}{b} = \frac{z-z_1}{c}
\]

\begin{example}
Find the vector form, parametric form and symmetric form for the line in $\R^3$ passing through the points $(2, 1, -4)$ and $(-3, 2, 5)$.\\
The direction vector for the line is 
\[
\vec{v} = \vector{-3-2, 2-1, 5-(-4)} = \vector{-5, 1, 9}
\]
Note that any non-zero multiple of this vector would also serve as a suitable direction vector for $L$. Using the point $(2, 1, -4)$ we have
\[
\vector{x, y, z} = \vector{2, 1, 4} + t\vector{-5, 1, 9} \quad \text{(Vector form)}
\]
\begin{align*}
x &= 2 -5t\\
y &= 1 + t \quad\text{(Parametric form)}\\
z &= 4 + 9t\\
\end{align*}
and
\[
\frac{x-2}{-5} = y-1 = \frac{z-4}{9} \quad \text{(Symmetric form)}
\]
\end{example}

\begin{example}
Find the vector form, parametric form and symmetric form of the line with direction vector $\vec{v} = \vector{2, -7, 4}$ that goes through 
the point $P = (-1, -3, 5)$. \\
The vector form is 
\[
\vector{x, y, z} = \vector{-1, -3, 5} + t\vector{2, -7, 4}
\]
The parametric form is
\begin{align*}
x &= -1 +2t\\
y &= -3 -7t \quad\text{(Parametric form)}\\
z &= 5 + 4t\\
\end{align*}
and the symmetric form
\[
\frac{x+1}{2} = \frac{y+3}{-7} = \frac{z-5}{4} 
\]
\end{example}

\begin{example}
Find the point of intersection of the lines
\[
L_1: \vector{x, y, z} = \vector{4, 6, -2} + t\vector{1, -2, 2}
\]
and 
\[
L_2: \vector{x, y, z} = \vector{-3, 5, -1} + t\vector{2, 1, -1}
\]
To find the point of intersection, it will be helpful to distinguish between the parameters in the two lines.
In $L_1$, replace $t$ with $t_1$ and likewise, replace $t$ by $t_2$ in $L_2$. At the point of intersection, the $x$-coordinates are equal.
This leads to the following equation:
\begin{align*}
x-\text{coordinate of} L_1 &= x-\text{coordinate of} L_2\\
4+t_1 &= -3 + 2t_2
\end{align*}
We wish to solve for $t_1$ and $t_2$, so we need another equation. Equating the $y$-coordinates, we get:
\[
6 -2t_1 = 5 + t_2
\]
Now, we solve the linear system
\begin{align*}
4+t_1 &= -3 + 2t_2\\
6 -2t_1 &= 5 + t_2
\end{align*}
Solving the top equation for $t_1$ gives $t_1 = -7 + 2t_2$.  substituting this into the bottom equation gives
\begin{align*}
6 - 2(-7+2t_2) &= 5 + t_2\\
6 + 14 - 4t_2 & = 5 + t_2\\
15 &= 5t_2\\
t_2 &=3
\end{align*}
Back substituting gives $t_1 = -7 + 2t_2 = -7 + 2(3) = -1$.
Hence, if the lines intersect, it will be at the point where $t_1 = -1$ on $L_1$ and $t_2 = 3$ on $L_2$.
Note that it is possible that the lines do not intersect- in which case, plugging these values in would give different $z$-coordinates.
The vector associated with the point of intersection is
\begin{align*}
L_1: \vector{x, y, z} &= \vector{4, 6, -2} + (-1)\vector{1, -2, 2} = \vector{3, 8, -4}\\
L_2: \vector{x, y, z} &= \vector{-3, 5, -1} + 3\vector{2, 1, -1} = \vector{3, 8, -4}
\end{align*}
Since these agree, we can conclude definitively that the lines do indeed intersect and furthermore, the point of intersection is $(3, 8, -4)$.
\end{example}
Two lines in $\R^3$ which are not parallel but also do not intersect are called skew lines.  
A basic problem is to find the distance between such skew lines 
(finding the distance between parallel lines, while also interesting, turns out to be considerably easier).
Of great importance in solving this problem is the cross product of the direction vectors of the two skew lines.
Since the full solution to the problem involves planes, we will tackle it in the next section, which covers planes in $\R^3$.

\end{document}
 
 
 
 
 
 
 

















