\documentclass[handout]{ximera}

%% You can put user macros here
%% However, you cannot make new environments



\newcommand{\ffrac}[2]{\frac{\text{\footnotesize $#1$}}{\text{\footnotesize $#2$}}}
\newcommand{\vasymptote}[2][]{
    \draw [densely dashed,#1] ({rel axis cs:0,0} -| {axis cs:#2,0}) -- ({rel axis cs:0,1} -| {axis cs:#2,0});
}


\graphicspath{{./}{firstExample/}}
\usepackage{forest}
\usepackage{amsmath}
\usepackage{amssymb}
\usepackage{array}
\usepackage[makeroom]{cancel} %% for strike outs
\usepackage{pgffor} %% required for integral for loops
\usepackage{tikz}
\usepackage{tikz-cd}
\usepackage{tkz-euclide}
\usetikzlibrary{shapes.multipart}


%\usetkzobj{all}
\tikzstyle geometryDiagrams=[ultra thick,color=blue!50!black]


\usetikzlibrary{arrows}
\tikzset{>=stealth,commutative diagrams/.cd,
  arrow style=tikz,diagrams={>=stealth}} %% cool arrow head
\tikzset{shorten <>/.style={ shorten >=#1, shorten <=#1 } } %% allows shorter vectors

\usetikzlibrary{backgrounds} %% for boxes around graphs
\usetikzlibrary{shapes,positioning}  %% Clouds and stars
\usetikzlibrary{matrix} %% for matrix
\usepgfplotslibrary{polar} %% for polar plots
\usepgfplotslibrary{fillbetween} %% to shade area between curves in TikZ



%\usepackage[width=4.375in, height=7.0in, top=1.0in, papersize={5.5in,8.5in}]{geometry}
%\usepackage[pdftex]{graphicx}
%\usepackage{tipa}
%\usepackage{txfonts}
%\usepackage{textcomp}
%\usepackage{amsthm}
%\usepackage{xy}
%\usepackage{fancyhdr}
%\usepackage{xcolor}
%\usepackage{mathtools} %% for pretty underbrace % Breaks Ximera
%\usepackage{multicol}



\newcommand{\RR}{\mathbb R}
\newcommand{\R}{\mathbb R}
\newcommand{\C}{\mathbb C}
\newcommand{\N}{\mathbb N}
\newcommand{\Z}{\mathbb Z}
\newcommand{\dis}{\displaystyle}
%\renewcommand{\d}{\,d\!}
\renewcommand{\d}{\mathop{}\!d}
\newcommand{\dd}[2][]{\frac{\d #1}{\d #2}}
\newcommand{\pp}[2][]{\frac{\partial #1}{\partial #2}}
\renewcommand{\l}{\ell}
\newcommand{\ddx}{\frac{d}{\d x}}

\newcommand{\zeroOverZero}{\ensuremath{\boldsymbol{\tfrac{0}{0}}}}
\newcommand{\inftyOverInfty}{\ensuremath{\boldsymbol{\tfrac{\infty}{\infty}}}}
\newcommand{\zeroOverInfty}{\ensuremath{\boldsymbol{\tfrac{0}{\infty}}}}
\newcommand{\zeroTimesInfty}{\ensuremath{\small\boldsymbol{0\cdot \infty}}}
\newcommand{\inftyMinusInfty}{\ensuremath{\small\boldsymbol{\infty - \infty}}}
\newcommand{\oneToInfty}{\ensuremath{\boldsymbol{1^\infty}}}
\newcommand{\zeroToZero}{\ensuremath{\boldsymbol{0^0}}}
\newcommand{\inftyToZero}{\ensuremath{\boldsymbol{\infty^0}}}


\newcommand{\numOverZero}{\ensuremath{\boldsymbol{\tfrac{\#}{0}}}}
\newcommand{\dfn}{\textbf}
%\newcommand{\unit}{\,\mathrm}
\newcommand{\unit}{\mathop{}\!\mathrm}
%\newcommand{\eval}[1]{\bigg[ #1 \bigg]}
\newcommand{\eval}[1]{ #1 \bigg|}
\newcommand{\seq}[1]{\left( #1 \right)}
\renewcommand{\epsilon}{\varepsilon}
\renewcommand{\iff}{\Leftrightarrow}

\DeclareMathOperator{\arccot}{arccot}
\DeclareMathOperator{\arcsec}{arcsec}
\DeclareMathOperator{\arccsc}{arccsc}
\DeclareMathOperator{\si}{Si}
\DeclareMathOperator{\proj}{proj}
\DeclareMathOperator{\scal}{scal}
\DeclareMathOperator{\cis}{cis}
\DeclareMathOperator{\Arg}{Arg}
%\DeclareMathOperator{\arg}{arg}
\DeclareMathOperator{\Rep}{Re}
\DeclareMathOperator{\Imp}{Im}
\DeclareMathOperator{\sech}{sech}
\DeclareMathOperator{\csch}{csch}
\DeclareMathOperator{\Log}{Log}

\newcommand{\tightoverset}[2]{% for arrow vec
  \mathop{#2}\limits^{\vbox to -.5ex{\kern-0.75ex\hbox{$#1$}\vss}}}
\newcommand{\arrowvec}{\overrightarrow}
\renewcommand{\vec}{\mathbf}
\newcommand{\veci}{{\boldsymbol{\hat{\imath}}}}
\newcommand{\vecj}{{\boldsymbol{\hat{\jmath}}}}
\newcommand{\veck}{{\boldsymbol{\hat{k}}}}
\newcommand{\vecl}{\boldsymbol{\l}}
\newcommand{\utan}{\vec{\hat{t}}}
\newcommand{\unormal}{\vec{\hat{n}}}
\newcommand{\ubinormal}{\vec{\hat{b}}}

\newcommand{\dotp}{\bullet}
\newcommand{\cross}{\boldsymbol\times}
\newcommand{\grad}{\boldsymbol\nabla}
\newcommand{\divergence}{\grad\dotp}
\newcommand{\curl}{\grad\cross}
%% Simple horiz vectors
\renewcommand{\vector}[1]{\left\langle #1\right\rangle}


\outcome{Derive differentiation rules for vector-valued functions.}

\title{2.3 Differentiation Rules}



\begin{document}

\begin{abstract}
In this section we will derive differentiation rules for vector-valued functions.
\end{abstract}

\maketitle

\section{Sum and Difference Rules}


The derivative of the sum or difference of two vector-valued functions is the sum or difference of their derivatives.

\begin{proposition}[Sum and Difference Rules]
Let $\vec r_1(t)$ and $\vec r_2(t)$ be vector-valued functions. Then
\[
\frac{d}{dt}\left[\vec r_1(t) \pm \vec r_2(t) \right] = \vec r_1\,'(t) \pm \vec r_2\,'(t)
\]
\end{proposition}
The reader should verify that the previous proposition is a direct consequence of the 
componentwise sum of vectors and the analogous differentiation rules for ordinary functions.

\section{Product Rules}
There are three types of multiplication involving vectors: multiplication by a scalar, the dot product, and the cross product.
We will use the product rule for ordinary functions to derive a product rule for all three of these operations.
Recall the product rule for functions $f(x)$ and $g(x)$:
\[
\frac{d}{dx}\left[f(x)g(x)\right] = f(x)g'(x) + f'(x)g(x)
\]
We begin with scalar multiplication.

\begin{proposition}[Product Rule for Scalar Multiplication]
Let $\vec r(t) = \vector{x(t), y(t),z(t)}$ be a vector-valued function with differentiable components, and let $f(t)$ be an ordinary, differentiable real-valued function.
Then
\[
\frac{d}{dt}\left[f(t)\vec r(t)\right] = f(t)\vec r\,'(t)) + f'(t)\vec r(t)
\]
\end{proposition}

\begin{proof}
Multiplying $f(t)$ into each component of $\vec r(t)$ and then differentiating yields
\begin{align*}
\frac{d}{dt}\left[f(t)\vec r(t)\right] &= \frac{d}{dt} \left[ \vector{f(t)x(t), f(t)y(t),f(t)z(t)}\right]\\
&=  \vector{\frac{d}{dt} \left[f(t)x(t)\right], \frac{d}{dt} \left[f(t)y(t)\right],\frac{d}{dt} \left[f(t)z(t)\right]}\\
&= \vector{f(t)x'(t) + f'(t)x(t), f(t)y'(t) + f'(t)y(t), f(t)z'(t) + f'(t)z(t)}\\
&=\vector{f(t)x'(t) , f(t)y'(t), f(t)z'(t)}+ \vector{f'(t)x(t), f'(t)y(t), f'(t)z(t)}\\
&= f(t)\vector{x'(t), y'(t), z'(t)} + f'(t)\vector{x(t), y(t), z(t)}\\
&= f(t)\vec r\,'(t)) + f'(t)\vec r(t)
\end{align*}
as claimed.
\end{proof}

\begin{proposition}[Product Rule for the Dot Product]
Let $\vec r_1(t) = \vector{x_1(t), y_1(t),z_1(t)}$ and $\vec r_2(t) = \vector{x_2(t), y_2(t),z_2(t)}$ be vector-valued functions with differentiable components.
Then
\[
\frac{d}{dt}\left[\vec r_1(t) \dotp \vec r_2(t)\right] = \vec r_1(t)\dotp \vec r_2\,'(t) + \vec r_1\,'(t)\dotp\vec r_2(t)
\]
\end{proposition}
\begin{proof}
Computing the dot product and then differentiating yields
\begin{align*}
\frac{d}{dt}\left[\vec r_1(t) \dotp \vec r_2(t)\right] &= \frac{d}{dt} \left[ \vector{x_1(t)x_2(t), y_1(t)y_2(t), z_1(t)z_2(t)}\right]\\
&=  \vector{\frac{d}{dt} \left[x_1(t)x_2(t)\right], \frac{d}{dt} \left[y_1(t)y_2(t)\right],\frac{d}{dt} \left[z_1(t)z_2(t)\right]}\\
&= \vector{x_1(t)x_2'(t) +x_1'(t)x_2(t), y_1(t)y_2'(t) +y_1'(t)y_2(t), z_1(t)z_2'(t) +z_1'(t)z_2(t)}\\
&=\vector{x_1(t)x_2'(t) , y_1(t)y_2'(t), z_1(t)z_2'(t)}+ \vector{x_1'(t)x_2(t), y_1'(t)y_2(t), z_1'(t)z_2(t)}\\
&= \vector{x_1(t), y_1(t), z_1(t)}\dotp \vector{x_2'(t), y_2'(t), z_2'(t)} + \vector{x_1'(t), y_1'(t), z_1'(t)}\dotp \vector{x_2(t), y_2(t), z_2(t)}\\
&= \vec r_1(t) \dotp \vec r_2\,'(t) + \vec r_1\,'(t) \dotp \vec r_2(t)
\end{align*}
as claimed.
\end{proof}

\begin{proposition}
Let $\vec r(t)$ be a vector-valued function and suppose $|\vec r(t)|$ is constant. Then $\vec r(t) \dotp \vec r\,'(t) = 0$.\\
\end{proposition}
\begin{proof}
Since $|\vec r(t)|^2 = c$, a constant, we have
\[
\frac{d}{dt} |\vec r(t)|^2 = 0
\]
However, $|\vec r(t)|^2$ can be expressed using the dot product and computing this derivative using the product rule for the dot product gives,
\begin{align*}
0 &= \frac{d}{dt} |\vec r(t)|^2 \\
&= \frac{d}{dt} \left[\vec r(t) \dotp \vec r(t)\right] \\
  &= \vec r(t) \dotp \vec r\,'(t) + \vec r\,'(t) \dotp \vec r(t) \\
  &= 2 \left[\vec r(t) \dotp \vec r\,'(t)\right]
\end{align*}
The last equation is due to the fact that the dot product is commutative. Hence $\vec r(t) \dotp \vec r\,'(t) = 0$
\end{proof}

\begin{remark}
The geometric interpretation of the above proposition is that if the graph of space curve $\vec r(t)$ is contained 
on the surface of a sphere with center at the origin,
then its tangent vectors will be orthogonal to its position vectors.  
This is due to the fact that for such a sphere, the position vector of any point on the sphere is 
orthogonal to the plane tangent to the sphere at that point.
\end{remark}

\begin{problem}(Problem 1)
Let $\vec r(t)$ be a vector-valued function. Show that if $\vec r(t) \neq \vec 0$ then 
\[
\frac{d}{dt} |\vec r(t)| = \frac{1}{|\vec r(t)|} \vec r(t) \dotp \vec r\,'(t)
\]
\end{problem}

\begin{definition}[Unit Tangent Vector]
Let $\vec r(t)$ be a vector-valued function. Then the \bf{unit tangent vector}, $\vec t(t)$ is defined as
\[
\vec T(t) = \frac{\vec r\,'(t)}{|\vec r\,'(t)|}
\]
Note that $\vec T(t)$ is in fact a unit vector.
\end{definition}

\begin{corollary}
Let $\vec r(t)$ be a vector valued function and let $\vec T(t)$ be the unit tangent vector.
Then 
\[
\vec T(t) \dotp \vec T\,'(t) = 0
\]
\end{corollary}
\begin{proof}
The magnitude of the unit tangent vector, $\vec T(t)$, is 1 and hence constant. The result then follows from the proposition.
\end{proof}


\begin{proposition}[Product Rule for the Cross Product]
Let $\vec r_1(t) = \vector{x_1(t), y_1(t),z_1(t)}$ and $\vec r_2(t) = \vector{x_2(t), y_2(t),z_2(t)}$ be vector-valued functions with differentiable components.
Then
\[
\frac{d}{dt}\left[\vec r_1(t) \cross \vec r_2(t)\right] = \vec r_1(t) \cross \vec r_2\,'(t) + \vec r_1\,'(t) \cross \vec r_2(t)
\]
\end{proposition}
\begin{proof}
To simply the notation, we will suppress the variable and write $x_1$ and $x_1'$ in place of $x_1(t)$ and $x_1'(t)$ and similarly for $y_1$ and $z_1$.
Recall the definition of the cross product:
\[
\vec{r}_1 \cross \vec{r}_2 =  \vector{y_1z_2 - z_1y_2, z_1x_2 - x_1z_2, x_1y_2 - y_1x_2}
\]
Computing the cross product and then differentiating yields:
\begin{align*}
\frac{d}{dt}\left[\vec r_1(t) \cross \vec r_2(t)\right] &= \frac{d}{dt} \left[ \vector{y_1z_2 - z_1y_2, z_1x_2 - x_1z_2, x_1y_2 - y_1x_2}\right]\\
&=  \vector{\frac{d}{dt} \left[y_1z_2 - z_1y_2\right], \frac{d}{dt} \left[z_1x_2 - x_1z_2\right],\frac{d}{dt} \left[x_1y_2 - y_1x_2\right]}\\
&= \vector{y_1z_2' + y_1'z_2- z_1y_2'  - z_1'y_2,  z_1x_2' + z_1'x_2- x_1z_2'  - x_1'z_2, x_1y_2' +x_1'y_2- y_1x_2'  - y_1'x_2      }\\
&=\vector{y_1z_2' - z_1y_2'  ,  z_1x_2' - x_1z_2', x_1y_2' - y_1x_2'}+ \vector{ y_1'z_2  - z_1'y_2, z_1'x_2 - x_1'z_2, x_1'y_2 - y_1'x_2}\\
&= \vector{x_1, y_1, z_1}\cross \vector{x_2', y_2', z_2'} + \vector{x_1', y_1', z_1'}\cross \vector{x_2, y_2, z_2}\\
&= \vec r_1(t) \cross \vec r_2\,'(t) + \vec r_1\,'(t) \cross \vec r_2(t)
\end{align*}
as claimed.


\end{proof}

\begin{problem}(Problem 2)
Let $\vec r(t)$ be a vector-valued function whose components are twice differentiable. Use the product rule for the cross product to show that
\[
\frac{d}{dt}\left[\vec r(t) \cross \vec r\,'(t) \right] = \vec r\,'(t) \cross \vec r\,''(t)
\]
\begin{hint}
The cross product of parallel vectors is the zero vector
\end{hint}
\end{problem}

\begin{problem}(Problem 3)
Let $\vec r(t)$ be a vector-valued function whose components are three times differentiable. Use the product rules for both the dot and cross products to show that
\[
\frac{d}{dt}\left[\vec r(t) \dotp \left(\vec r\,'(t) \cross \vec r\,''(t)\right)\right] = \vec r(t) \dotp \left(\vec r\,'(t) \cross \vec r\,'''(t)\right)
\]
\begin{hint}
The cross product of two vectors is orthogonal to both of them
\end{hint}
\begin{hint}
The cross product of parallel vectors is the zero vector
\end{hint}
\end{problem}

\begin{problem}(Problem 4)
Let $\vec u(t), \vec v(t)$ and $\vec w(t)$ be vector-valued functions. Find an expression for
\[
\frac{d}{dt}\left[ \vec u(t) \dotp \left(\vec v(t) \cross \vec w(t)\right)\right]
\]
Compare this to the derivative of a product of three ordinary functions.
\begin{hint}
For ordinary functions $,f, g$ and $h$
\[
\frac{d}{dx}\left[f(x)g(x)h(x)\right] = f(x)g(x)h'(x) + f(x)g'(x)h(x) + f'(x)g(x)h(x)
\]
\end{hint}
\end{problem}

\section{Chain rule}
Recall the chain rule for the derivative of a composition of two differentiable functions $f$ and $g$:
\[
\frac{d}{dx} f(g(x)) = f'(g(x)) \cdot g'(x)
\]

\begin{proposition}[Chain Rule for Vector-Valued Functions]
Let $\vec r(t) = \vector{x(t), y(t),z(t)}$ be a vector-valued function with differentiable components, and let $g(t)$ be an ordinary, differentiable real-valued function.
Then
\[
\frac{d}{dt}\vec r(g(t)) = g'(t)\,\vec r\,'(g(t))
\]
\end{proposition}
\begin{proof}
We write the composition in component form and then use the chain rule on each component:
\begin{align*}
\frac{d}{dt} \vec r(g(t)) &= \frac{d}{dt} \vector{x(g(t)), y(g(t)),z(g(t))}\\
&= \vector{\frac{d}{dt} x(g(t)), \frac{d}{dt} y(g(t)), \frac{d}{dt} z(g(t))}\\
&= \vector{x'(g(t)) g'(t), y'(g(t)) g'(t), z'(g(t)) g'(t)}\\
&= g'(t)\vector{x'(g(t)), y'(g(t)), z'(g(t))}\\
&= g'(t) \vec r\,'(g(t))
\end{align*}
as claimed.
\end{proof}

\begin{problem}(Problem 5)
Let $g(t) = \sqrt t$ and let $\vec r(t) = \vector{t^2, \sec(t), \tan(t)}$. Find
\[
\frac{d}{dt} \vec r(g(t))
\]
\end{problem}
 

\end{document}
