\documentclass[handout]{ximera}

%% You can put user macros here
%% However, you cannot make new environments



\newcommand{\ffrac}[2]{\frac{\text{\footnotesize $#1$}}{\text{\footnotesize $#2$}}}
\newcommand{\vasymptote}[2][]{
    \draw [densely dashed,#1] ({rel axis cs:0,0} -| {axis cs:#2,0}) -- ({rel axis cs:0,1} -| {axis cs:#2,0});
}


\graphicspath{{./}{firstExample/}}
\usepackage{forest}
\usepackage{amsmath}
\usepackage{amssymb}
\usepackage{array}
\usepackage[makeroom]{cancel} %% for strike outs
\usepackage{pgffor} %% required for integral for loops
\usepackage{tikz}
\usepackage{tikz-cd}
\usepackage{tkz-euclide}
\usetikzlibrary{shapes.multipart}


%\usetkzobj{all}
\tikzstyle geometryDiagrams=[ultra thick,color=blue!50!black]


\usetikzlibrary{arrows}
\tikzset{>=stealth,commutative diagrams/.cd,
  arrow style=tikz,diagrams={>=stealth}} %% cool arrow head
\tikzset{shorten <>/.style={ shorten >=#1, shorten <=#1 } } %% allows shorter vectors

\usetikzlibrary{backgrounds} %% for boxes around graphs
\usetikzlibrary{shapes,positioning}  %% Clouds and stars
\usetikzlibrary{matrix} %% for matrix
\usepgfplotslibrary{polar} %% for polar plots
\usepgfplotslibrary{fillbetween} %% to shade area between curves in TikZ



%\usepackage[width=4.375in, height=7.0in, top=1.0in, papersize={5.5in,8.5in}]{geometry}
%\usepackage[pdftex]{graphicx}
%\usepackage{tipa}
%\usepackage{txfonts}
%\usepackage{textcomp}
%\usepackage{amsthm}
%\usepackage{xy}
%\usepackage{fancyhdr}
%\usepackage{xcolor}
%\usepackage{mathtools} %% for pretty underbrace % Breaks Ximera
%\usepackage{multicol}



\newcommand{\RR}{\mathbb R}
\newcommand{\R}{\mathbb R}
\newcommand{\C}{\mathbb C}
\newcommand{\N}{\mathbb N}
\newcommand{\Z}{\mathbb Z}
\newcommand{\dis}{\displaystyle}
%\renewcommand{\d}{\,d\!}
\renewcommand{\d}{\mathop{}\!d}
\newcommand{\dd}[2][]{\frac{\d #1}{\d #2}}
\newcommand{\pp}[2][]{\frac{\partial #1}{\partial #2}}
\renewcommand{\l}{\ell}
\newcommand{\ddx}{\frac{d}{\d x}}

\newcommand{\zeroOverZero}{\ensuremath{\boldsymbol{\tfrac{0}{0}}}}
\newcommand{\inftyOverInfty}{\ensuremath{\boldsymbol{\tfrac{\infty}{\infty}}}}
\newcommand{\zeroOverInfty}{\ensuremath{\boldsymbol{\tfrac{0}{\infty}}}}
\newcommand{\zeroTimesInfty}{\ensuremath{\small\boldsymbol{0\cdot \infty}}}
\newcommand{\inftyMinusInfty}{\ensuremath{\small\boldsymbol{\infty - \infty}}}
\newcommand{\oneToInfty}{\ensuremath{\boldsymbol{1^\infty}}}
\newcommand{\zeroToZero}{\ensuremath{\boldsymbol{0^0}}}
\newcommand{\inftyToZero}{\ensuremath{\boldsymbol{\infty^0}}}


\newcommand{\numOverZero}{\ensuremath{\boldsymbol{\tfrac{\#}{0}}}}
\newcommand{\dfn}{\textbf}
%\newcommand{\unit}{\,\mathrm}
\newcommand{\unit}{\mathop{}\!\mathrm}
%\newcommand{\eval}[1]{\bigg[ #1 \bigg]}
\newcommand{\eval}[1]{ #1 \bigg|}
\newcommand{\seq}[1]{\left( #1 \right)}
\renewcommand{\epsilon}{\varepsilon}
\renewcommand{\iff}{\Leftrightarrow}

\DeclareMathOperator{\arccot}{arccot}
\DeclareMathOperator{\arcsec}{arcsec}
\DeclareMathOperator{\arccsc}{arccsc}
\DeclareMathOperator{\si}{Si}
\DeclareMathOperator{\proj}{proj}
\DeclareMathOperator{\scal}{scal}
\DeclareMathOperator{\cis}{cis}
\DeclareMathOperator{\Arg}{Arg}
%\DeclareMathOperator{\arg}{arg}
\DeclareMathOperator{\Rep}{Re}
\DeclareMathOperator{\Imp}{Im}
\DeclareMathOperator{\sech}{sech}
\DeclareMathOperator{\csch}{csch}
\DeclareMathOperator{\Log}{Log}

\newcommand{\tightoverset}[2]{% for arrow vec
  \mathop{#2}\limits^{\vbox to -.5ex{\kern-0.75ex\hbox{$#1$}\vss}}}
\newcommand{\arrowvec}{\overrightarrow}
\renewcommand{\vec}{\mathbf}
\newcommand{\veci}{{\boldsymbol{\hat{\imath}}}}
\newcommand{\vecj}{{\boldsymbol{\hat{\jmath}}}}
\newcommand{\veck}{{\boldsymbol{\hat{k}}}}
\newcommand{\vecl}{\boldsymbol{\l}}
\newcommand{\utan}{\vec{\hat{t}}}
\newcommand{\unormal}{\vec{\hat{n}}}
\newcommand{\ubinormal}{\vec{\hat{b}}}

\newcommand{\dotp}{\bullet}
\newcommand{\cross}{\boldsymbol\times}
\newcommand{\grad}{\boldsymbol\nabla}
\newcommand{\divergence}{\grad\dotp}
\newcommand{\curl}{\grad\cross}
%% Simple horiz vectors
\renewcommand{\vector}[1]{\left\langle #1\right\rangle}


\outcome{Compute the gradient vector and directional derivatives.}

\title{3.4 The Gradient Vector}



\begin{document}

\begin{abstract}
In this section we compute the gradient vector and directional derivatives.
\end{abstract}

\maketitle

\begin{definition}[Gradient Vector]
For a function of two variables, $f(x,y)$, the gradient vector is defined by
\[
\grad f(x,y) = \vector{f_x(x,y), f_y(x,y)} \quad \text{or just} \quad \vector{f_x, f_y} \quad \text{for short}
\]
Similarly, for a function of three variables, $f(x,y, z)$, the gradient vector is defined by
\[
\grad f(x,y, z) = \vector{f_x, f_y, f_z}
\]
\end{definition}

\begin{example}[Example 1]
Find the gradient vector for 
\[
\text{a)} \;f(x,y) = x^2 + y^2\quad \text{and} \quad  \text{b)} \;g(x, y, z) = xyz
\]
a) $\grad f(x,y) = \vector{f_x, f_y} = \vector{2x, 2y}$ \\
b) $\grad g(x,y, z) = \vector{g_x, g_y, g_z}= \vector{yz, xz, yz}$
\end{example}

\begin{problem}(Problem 1)
Find the gradient of each of the following:\\
a) $\;f(x,y) = xy - y^3; \quad \quad \grad f(x,y) = \vector{\answer{y}, \answer{x - 3y^2}}$\\
b) $\;g(x,y) = \sqrt{x^2 + y^2}; \quad \quad  \grad g(x,y) = \vector{\answer{\frac{x}{\sqrt{x^2 + y^2}}}, \answer{\frac{y}{\sqrt{x^2 + y^2}}}}$\\
c) $\;h(x,y, z) = x\sin(yz); \quad \quad \grad h(x,y, z) = \vector{\answer{\sin(yz)}, \answer{xz\cos(yz)} , \answer{xy\cos(yz)} }$\\
\end{problem}


\begin{example}[Example 2]
Let $f(x,y) = x^2 + 3xy$.  Find $\grad f(2, -3)$.\\
First we compute the gradient and then we plug in the point $(2, -3)$.  We have
\[
\grad f(x,y) = \vector{f_x(x,y), f_y(x,y)} = \vector{2x + 3y, 3x}
\]
At the given point, we have
\[
\grad f(2, -3) = \vector{2(2) + 3(-3), 3(2)} = \vector{-5, 6}
\]
\end{example}

\begin{problem}(Problem 2)
Let $f(x,y) = 2xy - y^3$.  Find $\grad f(-1, 2)$.\\
\[
\grad f(-1, 2) = \vector{\answer{4}, \answer{-14}}
\]
\end{problem}
The gradient is used to find the slope of a surface in a particular direction.

\begin{definition}[Directional Derivative]
Let $z = f(x,y)$ be a function of two variables whose graph is a surface in $R^3$ and let $\vec u = \vector{a,b}$ be a unit vector in $\R^2$.
The \textbf{derivative of} $\mathbf{f(x,y)}$ \textbf{in the direction of} $\mathbf{\vec u}$ is given by
\[
D_{\vec u} f(x,y) = \lim_{h \to 0} \frac{f(x + ah, y + bh) -f(x,y)}{h}
\]
This derivative gives the \textbf{rate of change of} $\mathbf{f(x,y)}$ \textbf{in the direction of} $\mathbf{\vec u}$.
\end{definition}

\begin{example}[Example 3]
Show that the derivative of $f(x,y)$ in the direction of the vector $\vec i = \vector{1,0}$ is the 
partial derivative with respect to $x$, $f_x(x,y)$.\\
Noting that $\vec i$ is a unit vector, we apply the definition of directional 
derivative using $\vec u = \vec i = \vector{1,0}$ to obtain:
\begin{align*}
D_{\vec i} f(x,y) &= \lim_{h \to 0} \frac{f(x + 1h, y + 0h) -f(x,y)}{h}\\
                  &= \lim_{h \to 0} \frac{f(x + h, y) -f(x,y)}{h}\\
                  &= f_x(x,y)
\end{align*}
The last equation follows from the definition of the partial derivative of $f$ with respect to $x$.
\end{example}

\begin{problem}(Problem 3)
Use the definition  to show that the derivative of $f(x,y)$ in the direction of 
the vector $\vec j = \vector{0,1}$ is the partial derivative with respect to $y$, $f_y(x,y)$.
\end{problem}


\begin{proposition}
Let $f(x,y)$ be a function of two variables and let $\vec u = \vector{a,b}$ be a unit vector in $\R^2$. Then
\[
D_{-\vec u} f(x,y) = -D_{\vec u} f(x,y)
\]
where the vector $-\vec u = \vector{-a, -b}$ is the negative of the vector $\vec u$.
\end{proposition}

\begin{proof}
The proof follows from the definition of directional derivative:
\begin{align*}
D_{-\vec u} f(x,y) &= \lim_{h \to 0} \frac{f(x - ah, y - bh) -f(x,y)}{h} \\
                   &= -\lim_{h \to 0} \frac{f(x - ah, y - bh) -f(x,y)}{-h} \\
                   &= -\lim_{k \to 0} \frac{f(x + ak, y  + bk) -f(x,y)}{k} \\
                   &= -D_{\vec u} f(x,y)
\end{align*}
Note the change of the increment from $h$ to $k$ using $k = -h$.
\end{proof}

The following theorem gives the relationship between the gradient vector and directional derivatives.

\begin{theorem}
Let $f(x,y)$ be a function of two variables with gradient vector $\grad f = \vector{f_x, f_y}$, 
and let $\vec u = \vector{a,b}$ be a unit vector.
Then
\[
D_{\vec u} f(x,y) = \grad f(x,y) \dotp \vec u = af_x(x,y) + bf_y(x,y)
\]
\end{theorem}

\begin{proof}
A rigorous proof requires the chain rule which will be covered in a later section. 
However, based solely on the definition, we present here a convincing argument for the conclusion of the theorem. 
The definition of directional derivative gives:
\begin{align*}
D_{\vec u} f(x,y) &= \lim_{h \to 0} \frac{f(x + ah, y + bh) -f(x,y)}{h} \\
                  &= \lim_{h \to 0} \frac{f(x + ah, y + bh) - f(x , y+ bh) + f(x, y+bh) -f(x,y)}{h} \\
                  &= \lim_{h \to 0} \frac{f(x + ah, y + bh) - f(x , y+ bh)}{h} + \lim_{h \to 0} \frac{f(x , y+ bh) -f(x,y)}{h} \\
                  &= a\cdot \lim_{h \to 0} \frac{f(x + ah, y + bh) - f(x , y+ bh)}{ah} + b\cdot \lim_{h \to 0} \frac{f(x , y+ bh) -f(x,y)}{bh} \\
                  &= a\cdot \lim_{k \to 0} \frac{f(x + k, y + ck) - f(x , y+ ck)}{k} + b\cdot \lim_{l \to 0} \frac{f(x , y+ l) -f(x,y)}{l} \\
                  &= (?) \; a f_x + bf_y
\end{align*}
In the second to the last line, the constant $c$ is replacing the fraction $b/a$.  
The question mark in the last line is due to the fact that the first limit is not
identical to the definition of $f_x$ since the $y$ coordinates are ``floating".  
However, the $y$ coordinates are equal in the first limit and as $k \to 0$, they are approaching $y$, 
so perhaps this first limit is simply $af_x(x,y)$. 
To strengthen our case for the result, we next present a similar computation, 
changing only the term which was added and subtracted in the numerator:
\begin{align*}
D_{\vec u} f(x,y) &= \lim_{h \to 0} \frac{f(x + ah, y + bh) -f(x,y)}{h} \\
                  &= \lim_{h \to 0} \frac{f(x + ah, y + bh) - f(x+ah , y) + f(x+ah, y) -f(x,y)}{h} \\
                  &= \lim_{h \to 0} \frac{f(x + ah, y + bh) - f(x+ah, y)}{h} + \lim_{h \to 0} \frac{f(x+ah , y) -f(x,y)}{h} \\
                  &= b\cdot \lim_{h \to 0} \frac{f(x + ah, y + bh) - f(x+ah , y)}{bh} + a\cdot \lim_{h \to 0} \frac{f(x+ah , y) -f(x,y)}{ah} \\
                  &= b\cdot \lim_{k \to 0} \frac{f(x + k, y + ck) - f(x+k , y)}{k} + a\cdot \lim_{l \to 0} \frac{f(x+ cl , y+ l) -f(x,y)}{l} \\
                  &= (?)\;  b f_y + af_x
\end{align*}
Here, in the second to the last line, the constant $c$ is replacing the fraction $a/b$.  
The question mark in the last line is due to the fact that the first limit is not
identical to the definition of $f_y$ since the $x$ coordinates are ``floating".  
However, the $x$ coordinates are equal in the first limit and as $k \to 0$, they are approaching $x$, 
so perhaps this first limit is simply $bf_y(x,y)$.
Each of the above computations gives us one of the partial derivatives exactly and the other one in a questionable format. 
This is compelling evidence for the validity of the claim in the theorem.
\end{proof}


\begin{example}[Example 4]
Find the derivative of $f(x,y) = 2x + 5y$ in the direction of the vector $\vector{3,4}$ at the point $(-1, 2)$.\\
Note that the surface $z = f(x,y) = 2x + 5y$ is a plane and at the point $(-1, 2)$ the height of the plane is $z = 2(-1) + 5(2) = 8$.
Now we compute the directional derivative by taking the dot product of the gradient vector (at the given point) with a unit 
vector in the direction of the given vector.
The unit vector in the direction of $\vector{3,4}$ is
\[
\vec u = \frac{\vector{3,4}}{|\vector{3,4}|} = \vector{\frac35, \frac45}
\]
The gradient of $f$ at the point $(-1, 2)$ is
\[
\grad f(-1, 2) = \vector{f_x(-1,2), f_y(-1,2)} = \vector{2,5}
\]
The directional derivative is thus
\[
D_{\vec u} f (-1, 2) = \grad f(-1, 2) \dotp \vec u = 2 \cdot \frac35 + 5 \cdot \frac45 = \frac{26}{5}
\]
Conceptually, this means that the rate of change of $f$ in the direction of $\vec u$ at the point $(-1, 2)$ is $\frac{26}{5}$.
Geometrically, this means that the slope of the plane $z = f(x,y) = 3x + 4y$ at the 
point $(-1, 2, 8)$ in the direction of $\vec u$ is $\frac{26}{5}$.
\end{example}

\begin{remark}
Note that the directional derivative in the last example did not depend on the point $(-1, 2)$.  
This is because our surface was a plane and the slope or rate of change of a plane in a particular 
direction is the same at every point in the plane.
\end{remark}

\begin{problem}(Problem 4)
Find the derivative of $f(x,y) = 4x - 2y$ in the direction of the vector $\vector{4, -3}$ at the point $(0, 6)$.\\
A unit vector in the direction of $\vector{4, -3}$ is $\vec u = \vector{\answer{4/5}, \answer{-3/5}}$\\
The gradient vector at the point $(0,6)$ is $\grad f(0,6) = \vector{\answer{4}, \answer{-2}}$\\
The directional derivative is $D_{\vec u} f(0,6) = \answer{22/5}$
\end{problem}


\begin{example}[Example 5]
Find the derivative of $f(x,y) = x+y$ at the point $(0,0)$ in the direction of the following vectors:
\[
\vec{i}, \vec{j}, \vector{1,1}, \vector{1, -1}, \vector{5, -12} \;\text{and} \; \vector{-5, 12}
\]
We begin by computing the gradient vector at the point $(0,0)$:
\[
\grad f(0,0) = \vector{f_x(0,0), f_y(0,0)} = \vector{1,1}
\]
Note that the gradient vector did not depend on the point.\\
The derivative in the direction of $\vec i$ is
\[
D_{\vec i} f(0,0) = \grad f(0,0) \dotp \vec{i} = \vector{1,1} \dotp \vector{1,0} = 1
\]
The derivative in the direction of $\vec j$ is
\[
D_{\vec j} f(0,0) = \grad f(0,0) \dotp \vec{j} = \vector{1,1} \dotp \vector{0,1} = 1
\]
Next, the vector $\vector{1,1}$ is not a unit vector, so we divide it by its magnitude to get
\[
\vec u = \vector{\frac{1}{\sqrt 2}, \frac{1}{\sqrt 2}}
\]
The derivative in the direction of $\vec u$ is
\[
D_{\vec u} f(0,0) = \grad f(0,0) \dotp \vec{u} = \vector{1,1} \dotp \vector{\frac{1}{\sqrt 2}, \frac{1}{\sqrt 2}} = \sqrt 2
\]
Note that the vector $\vec u$ above is in the same direction as the gradient vector.

Next, the vector $\vector{1,-1}$ is not a unit vector, so we divide it by its magnitude to get
\[
\vec v = \vector{\frac{1}{\sqrt 2}, -\frac{1}{\sqrt 2}}
\]
The derivative in the direction of $\vec v$ is
\[
D_{\vec v} f(0,0) = \grad f(0,0) \dotp \vec{v} = \vector{1,1} \dotp \vector{\frac{1}{\sqrt 2}, -\frac{1}{\sqrt 2}} = 0
\]

Next, the vector $\vector{5,-12}$ is not a unit vector, so we divide it by its magnitude to get
\[
\vec w_1 = \vector{\frac{5}{13}, -\frac{12}{13}}
\]
The derivative in the direction of $\vec{w_1}$ is
\[
D_{\vec{w_1}} f(0,0) = \grad f(0,0) \dotp \vec{w_1} = \vector{1,1} \dotp \vector{\frac{5}{13}, -\frac{12}{13}} = -\frac{7}{13}
\]

Finally, the vector $\vector{-5,12}$ is not a unit vector, so we divide it by its magnitude to get
\[
\vec w_2 = \vector{-\frac{5}{13}, \frac{12}{13}}
\]
The derivative in the direction of $\vec w_2$ is
\[
D_{\vec w_2} f(0,0) = \grad f(0,0) \dotp \vec{w_2} = \vector{1,1} \dotp \vector{-\frac{5}{13}, \frac{12}{13}} = \frac{7}{13}
\]

Note that the last two vectors, $\vec w_1$ and $\vec w_2$, are negatives of one another, and that the 
derivatives in their respective directions are negatives of each other.\\
Below, we show several level curves of the plane $z = f(x,y) = x+y$ along with the given (unitized) direction vectors.

\begin{image}
\begin{tikzpicture}
\draw[thick, <->] (-5, 0) -- (5,0);
\draw[thick, <->] (0,-5) -- (0,5);
\draw[thick, blue!70!white] (-4, 4) -- (4, -4) node[below]{$k = 0$};
\draw[thick, blue!70!white] (-3.5, 4.5) -- (4.5, -3.5) node[below]{$k = 1/2$};
\draw[thick, blue!70!white] (-3, 5) -- (5, -3) node[below]{$k = 1$};
\draw[thick, blue!70!white] (-2.5, 5.5) -- (5.5, -2.5) node[below]{$k = 3/2$};
\draw[thick, blue!70!white] (-4.5, 3.5) -- (3.5, -4.5) node[below]{$k = -1/2$};
\draw[thick, blue!70!white] (-5, 3) -- (3, -5) node[below]{$k = -1$};
\draw[thick, blue!70!white] (-5.5, 2.5) -- (2.5, -5.5) node[below]{$k = -3/2$};

\draw[red, ->] (0,0) -- (2,0) node[above,right]{$\vec i$};
\draw[red, ->] (0,0) -- (0,2) node[above, right]{$\vec j$};
\draw[red, ->] (0,0) -- (1.414, 1.414) node[above, right]{$\vec u$};
\draw[red, ->] (0,0) -- (1.414, -1.414) node[below, right]{$\vec v$};
\draw[red, ->] (0,0) -- (0.7692, -1.8461) node[below, right]{$\vec w_1$};
\draw[red, ->] (0,0) -- (-0.7692, 1.8461) node[above, left]{$\vec w_2$};
\node at (0, -6.5) {Note that the derivative in the direction of the level curves is 0};
\node at (0, -7) {and that the gradient vector is perpendicular to the level curves};
\end{tikzpicture}
\end{image}


\end{example}

\begin{problem}[Problem 5]
Find the derivative of $f(x,y) = x-y$ at the point $(0,0)$ in the direction of the following vectors:
\[
\vec{i}, \vec{j}, \vector{1,-1}, \vector{1, 1}, \vector{8, -15} \;\text{and} \; \vector{-8, 15}
\]
\[
\grad f(0,0) = \vector{\answer{1}, \answer{-1}}
\]
\[
D_{\vec i} f(0,0) = \answer{1}
\]
\[
D_{\vec j} f(0,0) = \answer{-1}
\]
Let $\vec u$ represent the unit vector in the direction of $\vector{1,-1}$, then 
\[
D_{\vec u} f(0,0) = \answer{\sqrt 2}
\]
Let $\vec v$ represent the unit vector in the direction of $\vector{1,1}$, then 
\[
D_{\vec v} f(0,0) = \answer{0}
\]
Let $\vec w_1$ represent the unit vector in the direction of $\vector{8,-15}$, then 
\[
D_{\vec w_1} f(0,0) = \answer{23/17}
\]
Let $\vec w_2$ represent the unit vector in the direction of $\vector{-8,15}$, then 
\[
D_{\vec w_2} f(0,0) = \answer{-23/17}
\]

\end{problem}


\begin{example}[Example 6]
Find the derivative of $f(x,y) = \ln(x^2 + y^2)$ in the direction of the vector $\vector{2, 3}$ at the point $(1, -2)$.\\
First, the gradient vector is
\[
\grad f(x,y) = \vector{\frac{2x}{x^2 +y^2}, \frac{2y}{x^2 +y^2}}
\]
and at the given point this vector is
\[
\grad f(1,-2) = \vector{\frac{2}{5}, -\frac{4}{5}}
\]
Next, the vector $\vector{2, 3}$ is not a unit vector, so we divide it by its magnitude:
\[
\vec u = \frac{\vector{2, 3}}{|\vector{2, 3}|} = \vector{\frac{2}{\sqrt{13}}, \frac{3}{\sqrt{13}}}
\]
Finally, we can determine the directional derivative:
\[
D_{\vec u} f(1, -2) = \grad f(1,-2) \dotp \vec u = \vector{\frac{2}{5}, -\frac{4}{5}} \dotp \vector{\frac{2}{\sqrt{13}}, \frac{3}{\sqrt{13}}} = -\frac{8}{5\sqrt{13}}
\]
\end{example}

\begin{problem}(Problem 6a)
Find the derivative of $f(x,y) = \ln(x^2 + y^2)$ in the direction of the vector $\vector{1, -4}$ at the point $(-2, 3)$.\\
\[
\grad f(-2, 3) = \vector{\answer{-\frac{4}{13}}, \answer{\frac{6}{13}}}
\]
The unit vector in the direction of $\vector{1, -4}$ is
\[
\vec u =  \vector{\answer{\frac{1}{\sqrt{17}}}, \answer{-\frac{4}{\sqrt{17}}}}
\]
The directional derivative is
\[
D_{\vec u} f(-2, 3) = \answer{-\frac{28}{13\sqrt{17}}}
\]
\end{problem}

\begin{problem}(Problem 6b)
Find the derivative of $f(x,y) = x^2 + 4y^2$ at the point $P(3, 5)$ in the direction of the point $Q(2,4)$.\\
\[
\grad f(3, 5) = \vector{\answer{6}, \answer{40}}
\]
The vector from the point $P$ to the point $Q$ is
\[
\vec PQ = \vector{\answer{-1}, \answer{-1}}
\]
The unit vector in the direction of $\vec PQ$ is
\[
\vec u = \vector{\answer{-1/\sqrt 2}, \answer{-1/\sqrt 2}}
\]
The directional derivative is
\[
D_{\vec u} f(3, 5) = \answer{-46/\sqrt 2}
\]

\end{problem}

\begin{problem}(Problem 6c)
Find the derivative of $f(x,y) = x^3 y^4$ at the point $(1,-1)$ in the same direction as the 
unit vector that makes an angle of $\pi/3$ with the positive $x$-axis.\\
\[
\grad f(1, -1) = \vector{\answer{3}, \answer{-4}}
\]
The unit vector in the given direction is
\[
\vec u = \vector{1/2, \sqrt{3} /2}
\]
The directional derivative is
\[
D_{\vec u} f(1, -1) = \answer{\frac{3-4\sqrt 3}{2}}
\]
\end{problem}

\begin{example}[Example 7]
The altitude $A$ of a point on a mountain at coordinates $(x, y)$ is given by
\[
A = 3000 - 0.02x^2 - 0.05y^2
\]
where $x$ and $y$ are measured in feet.\\
Find the rate of change of altitude at the point $(100, 200)$ in the northeast direction. 
(Assume that the positive $y$-axis is north and the positive $x$-axis is east.)\\
The rate of change of $A$ at the given point in the direction of a unit vector $\vec u$ is given by the directional derivative
\[
D_{\vec u} A(100, 200) = \grad A(100,200) \dotp \vec u.
\]
The gradient is
\[
\grad A(x,y) = \vector{-0.04x, -0.1y}
\]
and at the given point it is
\[
\grad A(100,200) = \vector{-4, -20}
\]
The vector that points northeast is $\vec i  + \vec j$ and the unit vector in this direction is
\[
\vec u = \vector{\frac{1}{\sqrt 2}, \frac{1}{\sqrt 2}}
\]
Hence, the rate of change of altitude is given by
\[
D_{\vec u} A(100, 200) = \vector{-4, -20} \dotp \vector{\frac{1}{\sqrt 2}, \frac{1}{\sqrt 2}} = -\frac{24}{\sqrt 2} \approx -16.8
\]
Thus, moving northeast from the point $(100, 200)$ the altitude is decreasing at a rate of 
$16.8$ feet for every foot moved in that direction.
\end{example}

\begin{problem}(Problem 7)
The temperature $T$ in degrees Fahrenheit of a point $(x,y)$ on a metal plate is given by 
\[
T = 350 + 3x - 2y + 2x^2 - 4y^2
\]
where $x$ and $y$ are measured in inches.  Find the rate of change in the temperature at the origin in the direction of the vector $\vector{2,-1}$.\\
The gradient of $T$ at the origin is given by
\[
\grad T(0,0) = \vector{\answer{3}, \answer{-2}}
\]
The unit vector in the direction of $\vector{2, -1}$ is
\[
\vec u = \vector{\answer{2/\sqrt 5}, \answer{-1/\sqrt 5}}
\]
The rate of change of temperature at the origin in the direction of $\vec u$ is
\[
D_{\vec u} T(0,0) = \answer{8/ \sqrt 5} \; ^\circ F / in.
\]
\end{problem}



\begin{proposition}
Let $f(x,y)$ be a function of two variables and suppose $\grad f \neq \vec 0$. 
Then, the rate of change of $f$ is greatest in the direction of the gradient vector.
Moreover, the maximum rate of change of $f(x,y)$ is the magnitude of the gradient, $|\grad f|$.

\end{proposition}

\begin{proof}
The direction of greatest increase is the vector $\vec u$ for which $D_{\vec u}f$ is the largest. 
Recall that the dot product of two vectors is the product of their magnitudes times the cosine of the angle between them.  
Hence a dot product is maximized when the cosine is one, i.e., the angle between the vectors is zero.  
In our context, this leads to the direction of greatest
increase in $f$ as the direction of the gradient as follows:
\begin{align*}
D_{\vec u} f &= \grad f \dotp \vec u = |\grad f| \cdot |\vec u| \cos \theta\\
             &= |\grad f| \cos \theta \leq |\grad f|
\end{align*}
where the last inequality is equality if $\cos \theta = 1$, i.e., if the angle between $\vec u$ and $\grad f$ is zero.
But, this angle is zero if and only if $\vec u$ is in the same direction as $\grad f$.  Hence, the directional derivative
$D_{\vec u}$ is maximized when the vector $\vec u$ is a unit vector in the same direction as the gradient vector, $\grad f$.
This also shows that the maximum rate of change of the function is equal to the magnitude of the gradient.
\end{proof}

\begin{corollary}
The direction of greatest decrease in $f$ is in the opposite direction of the gradient vector, $\grad f$ and the rate of greatest decrease is given by
the negative of the magnitude of the gradient.
\end{corollary}
\begin{proof}
The result follows from the previous proposition and the fact that $D_{-\vec u} f = -D_{\vec u}$.
\end{proof}

\begin{example}[Example 8]
Find the maximum rate of change of $f(x,y) = \sqrt{100 - x^2 - y^2}$ at the point $(4, 6)$.\\
The gradient is given by
\[
\grad f(x,y) = \vector{-\frac{x}{\sqrt{100 - x^2 - y^2}}, -\frac{y}{\sqrt{100 - x^2 - y^2}}}
\]
and at the given point, it is
\[
\grad f(4,6) = \vector{-\frac{4}{\sqrt{48}}, -\frac{6}{\sqrt{48}}} = \vector{-\frac{\sqrt 3}{3}, -\frac{\sqrt 3}{2}}
\]
The maximum rate of change of $f(x,y)$ occurs in this direction and it is given by the magnitude of the gradient:
\[
|\grad f(4,6)| = \sqrt{\frac13 + \frac34} = \frac{\sqrt{33}}{6}
\]
\end{example}

\begin{problem}(Problem 8a)
Find the maximum rate of increase in the function $f(x,y) = e^{xy}$ at the point $(1, 1)$.\\
The maximum rate of increase is 
\[
|\grad f(1, 1)| = \answer{e\sqrt 2}
\]
\end{problem}

\begin{problem}(Problem 8b)
Find the maximum rate of decrease of the function $f(x,y) = y^2/x$ at the point $(3, 6)$.\\
The maximum rate of decrease is 
\[
-|\grad f(3, 6)| = \answer{-4\sqrt 2}
\]
\end{problem}


\end{document}
