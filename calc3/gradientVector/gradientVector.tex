\documentclass[handout]{ximera}

%% You can put user macros here
%% However, you cannot make new environments



\newcommand{\ffrac}[2]{\frac{\text{\footnotesize $#1$}}{\text{\footnotesize $#2$}}}
\newcommand{\vasymptote}[2][]{
    \draw [densely dashed,#1] ({rel axis cs:0,0} -| {axis cs:#2,0}) -- ({rel axis cs:0,1} -| {axis cs:#2,0});
}


\graphicspath{{./}{firstExample/}}
\usepackage{forest}
\usepackage{amsmath}
\usepackage{amssymb}
\usepackage{array}
\usepackage[makeroom]{cancel} %% for strike outs
\usepackage{pgffor} %% required for integral for loops
\usepackage{tikz}
\usepackage{tikz-cd}
\usepackage{tkz-euclide}
\usetikzlibrary{shapes.multipart}


%\usetkzobj{all}
\tikzstyle geometryDiagrams=[ultra thick,color=blue!50!black]


\usetikzlibrary{arrows}
\tikzset{>=stealth,commutative diagrams/.cd,
  arrow style=tikz,diagrams={>=stealth}} %% cool arrow head
\tikzset{shorten <>/.style={ shorten >=#1, shorten <=#1 } } %% allows shorter vectors

\usetikzlibrary{backgrounds} %% for boxes around graphs
\usetikzlibrary{shapes,positioning}  %% Clouds and stars
\usetikzlibrary{matrix} %% for matrix
\usepgfplotslibrary{polar} %% for polar plots
\usepgfplotslibrary{fillbetween} %% to shade area between curves in TikZ



%\usepackage[width=4.375in, height=7.0in, top=1.0in, papersize={5.5in,8.5in}]{geometry}
%\usepackage[pdftex]{graphicx}
%\usepackage{tipa}
%\usepackage{txfonts}
%\usepackage{textcomp}
%\usepackage{amsthm}
%\usepackage{xy}
%\usepackage{fancyhdr}
%\usepackage{xcolor}
%\usepackage{mathtools} %% for pretty underbrace % Breaks Ximera
%\usepackage{multicol}



\newcommand{\RR}{\mathbb R}
\newcommand{\R}{\mathbb R}
\newcommand{\C}{\mathbb C}
\newcommand{\N}{\mathbb N}
\newcommand{\Z}{\mathbb Z}
\newcommand{\dis}{\displaystyle}
%\renewcommand{\d}{\,d\!}
\renewcommand{\d}{\mathop{}\!d}
\newcommand{\dd}[2][]{\frac{\d #1}{\d #2}}
\newcommand{\pp}[2][]{\frac{\partial #1}{\partial #2}}
\renewcommand{\l}{\ell}
\newcommand{\ddx}{\frac{d}{\d x}}

\newcommand{\zeroOverZero}{\ensuremath{\boldsymbol{\tfrac{0}{0}}}}
\newcommand{\inftyOverInfty}{\ensuremath{\boldsymbol{\tfrac{\infty}{\infty}}}}
\newcommand{\zeroOverInfty}{\ensuremath{\boldsymbol{\tfrac{0}{\infty}}}}
\newcommand{\zeroTimesInfty}{\ensuremath{\small\boldsymbol{0\cdot \infty}}}
\newcommand{\inftyMinusInfty}{\ensuremath{\small\boldsymbol{\infty - \infty}}}
\newcommand{\oneToInfty}{\ensuremath{\boldsymbol{1^\infty}}}
\newcommand{\zeroToZero}{\ensuremath{\boldsymbol{0^0}}}
\newcommand{\inftyToZero}{\ensuremath{\boldsymbol{\infty^0}}}


\newcommand{\numOverZero}{\ensuremath{\boldsymbol{\tfrac{\#}{0}}}}
\newcommand{\dfn}{\textbf}
%\newcommand{\unit}{\,\mathrm}
\newcommand{\unit}{\mathop{}\!\mathrm}
%\newcommand{\eval}[1]{\bigg[ #1 \bigg]}
\newcommand{\eval}[1]{ #1 \bigg|}
\newcommand{\seq}[1]{\left( #1 \right)}
\renewcommand{\epsilon}{\varepsilon}
\renewcommand{\iff}{\Leftrightarrow}

\DeclareMathOperator{\arccot}{arccot}
\DeclareMathOperator{\arcsec}{arcsec}
\DeclareMathOperator{\arccsc}{arccsc}
\DeclareMathOperator{\si}{Si}
\DeclareMathOperator{\proj}{proj}
\DeclareMathOperator{\scal}{scal}
\DeclareMathOperator{\cis}{cis}
\DeclareMathOperator{\Arg}{Arg}
%\DeclareMathOperator{\arg}{arg}
\DeclareMathOperator{\Rep}{Re}
\DeclareMathOperator{\Imp}{Im}
\DeclareMathOperator{\sech}{sech}
\DeclareMathOperator{\csch}{csch}
\DeclareMathOperator{\Log}{Log}

\newcommand{\tightoverset}[2]{% for arrow vec
  \mathop{#2}\limits^{\vbox to -.5ex{\kern-0.75ex\hbox{$#1$}\vss}}}
\newcommand{\arrowvec}{\overrightarrow}
\renewcommand{\vec}{\mathbf}
\newcommand{\veci}{{\boldsymbol{\hat{\imath}}}}
\newcommand{\vecj}{{\boldsymbol{\hat{\jmath}}}}
\newcommand{\veck}{{\boldsymbol{\hat{k}}}}
\newcommand{\vecl}{\boldsymbol{\l}}
\newcommand{\utan}{\vec{\hat{t}}}
\newcommand{\unormal}{\vec{\hat{n}}}
\newcommand{\ubinormal}{\vec{\hat{b}}}

\newcommand{\dotp}{\bullet}
\newcommand{\cross}{\boldsymbol\times}
\newcommand{\grad}{\boldsymbol\nabla}
\newcommand{\divergence}{\grad\dotp}
\newcommand{\curl}{\grad\cross}
%% Simple horiz vectors
\renewcommand{\vector}[1]{\left\langle #1\right\rangle}


\outcome{Compute the gradient vector and directional derivatives.}

\title{3.4 The Gradient Vector}



\begin{document}

\begin{abstract}
In this section we compute the gradient vector and directional derivatives.
\end{abstract}

\maketitle

\begin{definition}[Gradient Vector]
For a function of two variables, $f(x,y)$, the gradient vector is defined by
\[
\grad f(x,y) = \vector{f_x(x,y), f_y(x,y)} \quad \text{or just} \quad \vector{f_x, f_y} \quad \text{for short}
\]
Similarly, for a function of three variables, $f(x,y, z)$, the gradient vector is defined by
\[
\grad f(x,y, z) = \vector{f_x, f_y, f_z}
\]
\end{definition}

\begin{example}[Example 1]
Find the gradient vector for 
\[
\text{a)} \;f(x,y) = x^2 + y^2\quad \text{and} \quad  \text{b)} \;g(x, y, z) = xyz
\]
a) $\grad f(x,y) = \vector{f_x, f_y} = \vector{2x, 2y}$ \\
b) $\grad g(x,y, z) = \vector{g_x, g_y, g_z}= \vector{yz, xz, yz}$
\end{example}

\begin{problem}(Problem 1)
Find the gradient of each of the following:\\
a) $\;f(x,y) = xy - y^2; \quad \quad \grad f(x,y) = \vector{\answer{y}, \answer{x - 3y^2}}$\\
b) $\;g(x,y) = \sqrt{x^2 + y^2}; \quad \quad  \grad g(x,y) = \vector{\answer{\frac{x}{\sqrt{x^2 + y^2}}}, \answer{\frac{y}{\sqrt{x^2 + y^2}}}}$\\
a) $\;h(x,y, z) = x\sin(yz); \quad \quad \grad h(x,y, z) = \vector{\answer{\sin(yz)}, \answer{xz\cos(yz)} , \answer{xy\cos(yz)} }$\\
\end{problem}


\begin{example}[Example 2]
Let $f(x,y) = x^2 + 3xy$.  Find $\grad f(2, -3)$.\\
First we compute the gradient and then we plug in the point $(2, -3)$.  We have
\[
\grad f(x,y) = \vector{f_x(x,y), f_y(x,y)} = \vector{2x + 3y, 3x}
\]
At the given point, we have
\[
\grad f(2, -3) = \vector{2(2) + 3(-3), 3(2)} = \vector{-5, 6}
\]
\end{example}

\begin{problem}(Problem 2)
Let $f(x,y) = 2xy - y^3$.  Find $\grad f(-1, 2)$.\\
\[
\grad f(-1, 2) = \vector{\answer{4}, \answer{-14}}
\]
\end{problem}
The gradient is used to find the slope of a surface in a particular direction.

\begin{definition}[Directional Derivative]
Let $f(x,y)$ be a function of two variables whose graph is a surface in $R^3$ and let $\vec u = \vector{a,b}$ be a unit vector in $\R^2$.
The rate of change of $f$ in the direction of $\vec u$ is given by
\[
D_{\vec u} f(x,y) = \lim_{h \to 0} \frac{f(x + ah, y + bh) -f(x,y)}{h}
\]
\end{definition}


\begin{remark}
The partial derivative $f_x$ is the directional derivative in the direction of the unit vector $\vec i = \vector{1,0}$ and 
the partial derivative $f_y$ is the directional derivative in the direction of the unit vector $\vec j = \vector{0,1}$
\end{remark}

\begin{proposition}
Let $f(x,y)$ be a function of two variables and let $\vec u = \vector{a,b}$ be a unit vector in $\R^2$. Then
\[
D_{-\vec u} f(x,y) = -D_{\vec u} f(x,y)
\]
where the vector $-\vec u = \vector{-a, -b}$ is the negative of the vector $\vec u$.
\end{proposition}

\begin{proof}
The proof follows from the definition of directional derivative:
\begin{align*}
D_{-\vec u} f(x,y) &= \lim_{h \to 0} \frac{f(x - ah, y - bh) -f(x,y)}{h} \\
                   &= -\lim_{h \to 0} \frac{f(x - ah, y - bh) -f(x,y)}{-h} \\
                   &= -\lim_{k \to 0} \frac{f(x + ak, y  + bk) -f(x,y)}{k} \\
                   &= -D_{\vec u} f(x,y)
\end{align*}
\end{proof}

The following theorem gives the relationship between the gradient vector and directional derivatives.

\begin{theorem}
Let $f(x,y)$ be a function of two variables with gradient vector $\grad f = \vector{f_x, f_y}$, and let $\vec u = \vector{a,b}$ be a unit vector.
Then
\[
D_{\vec u} f(x,y) = \grad f \dotp \vec u = \vector{af_x, bf_y}
\]
\end{theorem}

\begin{proof}
A rigorous proof requires the chain rule which will be covered in a later section. 
However, based solely on the definition, we present here a convincing argument for the conclusion of the theorem. 
The definition of directional derivative gives:
\begin{align*}
D_{\vec u} f(x,y) &= \lim_{h \to 0} \frac{f(x + ah, y + bh) -f(x,y)}{h} \\
                  &= \lim_{h \to 0} \frac{f(x + ah, y + bh) - f(x , y+ bh) + f(x, y+bh) -f(x,y)}{h} \\
                  &= \lim_{h \to 0} \frac{f(x + ah, y + bh) - f(x , y+ bh)}{h} + \lim_{h \to 0} \frac{f(x , y+ bh) -f(x,y)}{h} \\
                  &= a\cdot \lim_{h \to 0} \frac{f(x + ah, y + bh) - f(x , y+ bh)}{ah} + b\cdot \lim_{h \to 0} \frac{f(x , y+ bh) -f(x,y)}{bh} \\
                  &= a\cdot \lim_{k \to 0} \frac{f(x + k, y + ck) - f(x , y+ ck)}{k} + b\cdot \lim_{l \to 0} \frac{f(x , y+ l) -f(x,y)}{l} \\
                  &= (?) \; a f_x + bf_y
\end{align*}
In the second to the last line, the constant $c$ is replacing the fraction $b/a$.  
The question mark in the last line is due to the fact that the first limit is not
identical to the definition of $f_x$ since the $y$ coordinates are ``floating".  
However, the $y$ coordinates are equal in the first limit and as $k \to 0$, they are approaching $y$, so perhaps this first limit is simply $af_x(x,y)$.
We next present a similar computation, changing only the term which was added and subtracted in the numerator:
\begin{align*}
D_{\vec u} f(x,y) &= \lim_{h \to 0} \frac{f(x + ah, y + bh) -f(x,y)}{h} \\
                  &= \lim_{h \to 0} \frac{f(x + ah, y + bh) - f(x+ah , y) + f(x+ah, y) -f(x,y)}{h} \\
                  &= \lim_{h \to 0} \frac{f(x + ah, y + bh) - f(x+ah, y)}{h} + \lim_{h \to 0} \frac{f(x+ah , y) -f(x,y)}{h} \\
                  &= b\cdot \lim_{h \to 0} \frac{f(x + ah, y + bh) - f(x+ah , y)}{bh} + a\cdot \lim_{h \to 0} \frac{f(x+ah , y) -f(x,y)}{ah} \\
                  &= b\cdot \lim_{k \to 0} \frac{f(x + k, y + ck) - f(x+k , y)}{k} + a\cdot \lim_{l \to 0} \frac{f(x+ cl , y+ l) -f(x,y)}{l} \\
                  &= (?)\;  b f_y + af_x
\end{align*}
Here, in the second to the last line, the constant $c$ is replacing the fraction $a/b$.  
The question mark in the last line is due to the fact that the first limit is not
identical to the definition of $f_y$ since the $x$ coordinates are ``floating".  
However, the $x$ coordinates are equal in the first limit and as $k \to 0$, they are approaching $x$, so perhaps this first limit is simply $bf_y(x,y)$.
Each of the above computations gives us one of the partial derivatives exactly and the other one in a questionable format. 
This is compelling evidence for the validity of the claim in the theorem.
\end{proof}


\begin{example}[Example 3]
Find the derivative of $f(x,y) = 2x + 5y$ in the direction of the vector $\vector{3,4}$ at the point $(-1, 2)$.\\
Note that the surface $z = f(x,y) = 2x + 5y$ is a plane and at the point $(-1, 2)$ the height of the plane is $z = 2(-1) + 5(2) = 8$.
Now we compute the directional derivative by taking the dot product of the gradient vector (at the given point) with a unit 
vector in the direction of the given vector.
The unit vector in the direction of $\vector{3,4}$ is
\[
\vec u = \frac{\vector{3,4}}{|\vector{3,4}|} = \vector{\frac35, \frac45}
\]
The gradient of $f$ at the point $(-1, 2)$ is
\[
\grad f(-1, 2) = \vector{f_x(-1,2), f_y(-1,2)} = \vector{2,5}
\]
The directional derivative is thus
\[
D_{\vec u} f (-1, 2) = \grad f(-1, 2) \dotp \vec u = 2 \cdot \frac35 + 5 \cdot \frac45 = \frac{26}{5}
\]
Conceptually, this means that the rate of change of $f$ in the direction of $\vec u$ at the point $(-1, 2)$ is $\frac{26}{5}$.
Geometrically, this means that the slope of the plane $z = f(x,y) = 3x + 4y$ at the point $(-1, 2, 8)$ in the direction of $\vec u$ is $\frac{26}{5}$.
\end{example}

\begin{remark}
Note that the directional derivative in the last example did not depend on the point $(-1, 2)$.  
This is because our surface was a plane and the slope or rate of change of a plane in a particular direction is the same at every point in the plane.
\end{remark}

\begin{problem}(Problem 3)
Find the derivative of $f(x,y) = 4x - 2y$ in the direction of the vector $\vector{4, -3}$ at the point $(0, 6)$.\\
A unit vector in the direction of $\vector{4, -3}$ is $\vec u = \vector{\answer{4/5}, \answer{-3/5}}$\\
The gradient vector at the point $(0,6)$ is $\grad f(0,6) = \vector{\answer{4}, \answer{-2}}$\\
The directional derivative is $D_{\vec u} f(0,6) = \answer{2}$
\end{problem}

\begin{proposition}
Let $f(x,y)$ be a function of two variables and suppose $\grad f \neq \vec 0$. Then, the rate of change of $f$ is greatest in the direction of the gradient vector.

\end{proposition}

\begin{proof}
The direction of greatest increase is the vector $\vec u$ for which $D_{\vec u}f$ is the largest. 
Recall that the dot product of two vectors is the product of their magnitudes times the cosine of the angle between them.  
Hence a dot product is maximized when the cosine is one, i.e., the angle between the vectors is zero.  In our context, this leads to the direction of greatest
increase in $f$ as the direction of the gradient as follows:
\begin{align*}
D_{\vec u} f &= \grad f \dotp \vec u = |\grad f| \dot |\vec u| \cos \theta\\
             &= |\grad f| \cos \theta \leq |\grad f|
\end{align*}
where the last inequality is equality if $\cos \theta = 1$, i.e., if the angle between $\vec u$ and $\grad f$ is zero.
But, this angle is zero if and only if $\vec u$ is in the same direction as $\grad f$.  Hence, the directional derivative
$D_{\vec u}$ is maximized when the vector $\vec u$ is a unit vector in the same direction as the gradient vector, $\grad f$.
\end{proof}

\begin{corollary}
The direction of greatest decrease in $f$ is in the opposite direction of the gradient vector, $\grad f$.
\end{corollary}
\begin{proof}
The result follows from the previous proposition and the fact that $D_{-\vec u} f = -D_{\vec u}$.
\end{proof}

\end{document}
