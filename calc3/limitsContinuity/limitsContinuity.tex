\documentclass[handout]{ximera}

%% You can put user macros here
%% However, you cannot make new environments



\newcommand{\ffrac}[2]{\frac{\text{\footnotesize $#1$}}{\text{\footnotesize $#2$}}}
\newcommand{\vasymptote}[2][]{
    \draw [densely dashed,#1] ({rel axis cs:0,0} -| {axis cs:#2,0}) -- ({rel axis cs:0,1} -| {axis cs:#2,0});
}


\graphicspath{{./}{firstExample/}}
\usepackage{forest}
\usepackage{amsmath}
\usepackage{amssymb}
\usepackage{array}
\usepackage[makeroom]{cancel} %% for strike outs
\usepackage{pgffor} %% required for integral for loops
\usepackage{tikz}
\usepackage{tikz-cd}
\usepackage{tkz-euclide}
\usetikzlibrary{shapes.multipart}


%\usetkzobj{all}
\tikzstyle geometryDiagrams=[ultra thick,color=blue!50!black]


\usetikzlibrary{arrows}
\tikzset{>=stealth,commutative diagrams/.cd,
  arrow style=tikz,diagrams={>=stealth}} %% cool arrow head
\tikzset{shorten <>/.style={ shorten >=#1, shorten <=#1 } } %% allows shorter vectors

\usetikzlibrary{backgrounds} %% for boxes around graphs
\usetikzlibrary{shapes,positioning}  %% Clouds and stars
\usetikzlibrary{matrix} %% for matrix
\usepgfplotslibrary{polar} %% for polar plots
\usepgfplotslibrary{fillbetween} %% to shade area between curves in TikZ



%\usepackage[width=4.375in, height=7.0in, top=1.0in, papersize={5.5in,8.5in}]{geometry}
%\usepackage[pdftex]{graphicx}
%\usepackage{tipa}
%\usepackage{txfonts}
%\usepackage{textcomp}
%\usepackage{amsthm}
%\usepackage{xy}
%\usepackage{fancyhdr}
%\usepackage{xcolor}
%\usepackage{mathtools} %% for pretty underbrace % Breaks Ximera
%\usepackage{multicol}



\newcommand{\RR}{\mathbb R}
\newcommand{\R}{\mathbb R}
\newcommand{\C}{\mathbb C}
\newcommand{\N}{\mathbb N}
\newcommand{\Z}{\mathbb Z}
\newcommand{\dis}{\displaystyle}
%\renewcommand{\d}{\,d\!}
\renewcommand{\d}{\mathop{}\!d}
\newcommand{\dd}[2][]{\frac{\d #1}{\d #2}}
\newcommand{\pp}[2][]{\frac{\partial #1}{\partial #2}}
\renewcommand{\l}{\ell}
\newcommand{\ddx}{\frac{d}{\d x}}

\newcommand{\zeroOverZero}{\ensuremath{\boldsymbol{\tfrac{0}{0}}}}
\newcommand{\inftyOverInfty}{\ensuremath{\boldsymbol{\tfrac{\infty}{\infty}}}}
\newcommand{\zeroOverInfty}{\ensuremath{\boldsymbol{\tfrac{0}{\infty}}}}
\newcommand{\zeroTimesInfty}{\ensuremath{\small\boldsymbol{0\cdot \infty}}}
\newcommand{\inftyMinusInfty}{\ensuremath{\small\boldsymbol{\infty - \infty}}}
\newcommand{\oneToInfty}{\ensuremath{\boldsymbol{1^\infty}}}
\newcommand{\zeroToZero}{\ensuremath{\boldsymbol{0^0}}}
\newcommand{\inftyToZero}{\ensuremath{\boldsymbol{\infty^0}}}


\newcommand{\numOverZero}{\ensuremath{\boldsymbol{\tfrac{\#}{0}}}}
\newcommand{\dfn}{\textbf}
%\newcommand{\unit}{\,\mathrm}
\newcommand{\unit}{\mathop{}\!\mathrm}
%\newcommand{\eval}[1]{\bigg[ #1 \bigg]}
\newcommand{\eval}[1]{ #1 \bigg|}
\newcommand{\seq}[1]{\left( #1 \right)}
\renewcommand{\epsilon}{\varepsilon}
\renewcommand{\iff}{\Leftrightarrow}

\DeclareMathOperator{\arccot}{arccot}
\DeclareMathOperator{\arcsec}{arcsec}
\DeclareMathOperator{\arccsc}{arccsc}
\DeclareMathOperator{\si}{Si}
\DeclareMathOperator{\proj}{proj}
\DeclareMathOperator{\scal}{scal}
\DeclareMathOperator{\cis}{cis}
\DeclareMathOperator{\Arg}{Arg}
%\DeclareMathOperator{\arg}{arg}
\DeclareMathOperator{\Rep}{Re}
\DeclareMathOperator{\Imp}{Im}
\DeclareMathOperator{\sech}{sech}
\DeclareMathOperator{\csch}{csch}
\DeclareMathOperator{\Log}{Log}

\newcommand{\tightoverset}[2]{% for arrow vec
  \mathop{#2}\limits^{\vbox to -.5ex{\kern-0.75ex\hbox{$#1$}\vss}}}
\newcommand{\arrowvec}{\overrightarrow}
\renewcommand{\vec}{\mathbf}
\newcommand{\veci}{{\boldsymbol{\hat{\imath}}}}
\newcommand{\vecj}{{\boldsymbol{\hat{\jmath}}}}
\newcommand{\veck}{{\boldsymbol{\hat{k}}}}
\newcommand{\vecl}{\boldsymbol{\l}}
\newcommand{\utan}{\vec{\hat{t}}}
\newcommand{\unormal}{\vec{\hat{n}}}
\newcommand{\ubinormal}{\vec{\hat{b}}}

\newcommand{\dotp}{\bullet}
\newcommand{\cross}{\boldsymbol\times}
\newcommand{\grad}{\boldsymbol\nabla}
\newcommand{\divergence}{\grad\dotp}
\newcommand{\curl}{\grad\cross}
%% Simple horiz vectors
\renewcommand{\vector}[1]{\left\langle #1\right\rangle}


\outcome{Compute limits and define continuity.}

\title{3.2 Limits and Continuity}



\begin{document}

\begin{abstract}
In this section we compute limits and define continuity.
\end{abstract}

\maketitle

On the real number line, $\R$, there are two directions for a variable $x$ to approach a value $a$- from 
the left, denoted $x \to a^-$, and from the right, denoted $x \to a^+$.
As a result, we say that the limit as $x \to a$ exists for a function of one variable, $f(x)$, if
\[
\lim_{x \to a^-} f(x) = \lim_{x \to a^+}
\]
For a function of two variables, the situation is much more complicated. Consider a variable coordinate pair $(x, y)$ in $\R^2$
approaching a fixed coordinate pair $(a,b)$.  There are infinitely many paths the variable pair can take.

\begin{image}
\begin{tikzpicture}

\filldraw (0,0) circle (0.1);
\node at (0, -0.7) {$(a,b)$};
\draw[blue, ->] (-2.5, 2.5) -- (-0.3, 0.3);
\draw[blue, ->] (0, 3) -- (0, 0.4);
\draw[blue, ->] (1.5, 2.5) to [out = 270, in = 0] (0.4, 0);
\draw[blue, ->] (-3, 0) to [out = 45, in = 180] (-2.4, 0.5) to [out = 0, in = 135] (-1.8, 0) to [out = 315, in =180] (-1.1, -0.5) to [out = 0, in =200] (-0.4, -0.1);
\draw[blue, ->] (-2, -1) to [out = 0, in =270] (1, -0.8) to [out = 90, in = 315] (0.3, -0.3);
\node at (0, -2) {Some of the paths to the point $(a, b)$};
\end{tikzpicture}
\end{image} 

For a function of two variables $f(x,y)$ we say that the limit of $f$ as $(x, y) \to (a,b)$ exists written
\[
\lim_{(x,y) \to (a,b)} f(x,y)
\]
if all possible paths yield the same limiting value. We begin our examples with limits that do not exist. 
To show that a limit does not exist, it suffices to demonstrate two different paths which yield different limiting values.
It should be noted that in next three examples, plugging in the terminal values of $x$ and $y$ will give the $\frac00$ indeterminate form.

\begin{example}[Example 1]
Show that the following limit does not exist:
\[
\lim_{(x,y) \to (0,0)} \frac{x}{x^2 + y^2}
\]
First consider a path where $x = 0$ (but $y \neq 0$). Along this path, the function reduces to
\[
\frac{x}{x^2 + y^2} = \frac{0}{y^2} = 0
\]
and so the limit is zero.\\
Now consider a path in which $y = 0$ (but $x \neq 0$). Along this path, the function simplifies to
\[
\frac{x}{x^2 + y^2} = \frac{x}{x^2} = \frac{1}{x}
\]
and the limit as $x \to 0$ is infinite.\\
Since the limits along the two paths were different, we must conclude that
\[
\lim_{(x,y) \to (0,0)} \frac{x}{x^2 + y^2} \quad \text{DNE}
\]
\end{example}

\begin{problem}(Problem 1)
Show that the following limit does not exist:
\[
\lim_{(x,y) \to (0,0)} \frac{y}{x^2 + y^2}
\]
\begin{hint}
Consider different paths
\end{hint}
\begin{hint}
Try letting $x$ and $y$ equal zero separately
\end{hint}
\end{problem}

\begin{example}[Example 2]
Show that the following limit does not exist:
\[
\lim_{(x,y) \to (0,0)} \frac{x\sin(y)}{x^2 + y^2}
\]
If we let $x = 0$ then the function reduces to just $0$ and so the limit will be $0$. If we let $y = 0$, then since $sin (0) = 0$, we will again get that the limit is $0$.
Since these two paths yield the same limit, we cannot yet conclude that the limit does not exist, and we must try another path.
If we allow the point $(x,y)$ to approach $(0,0)$ along the line $y = x$, the fraction becomes:
\[
\frac{x\sin(y)}{x^2 + y^2} = \frac{x\sin(x)}{2x^2} = \frac{\sin(x)}{2x}
\]
To find this limit, we can apply L'Hopital's rule since both the numerator and the denominator are approaching zero:
\[
\lim_{x \to 0} \frac{\sin(x)}{2x} = \lim_{x \to 0} \frac{\cos(x)}{2} = \frac12
\]
We have now obtained two different paths that yield different limits and we can conclude that
\[
\lim_{(x,y) \to (0,0)} \frac{y}{x^2 + y^2} \quad \text{DNE}
\]
\end{example}

\begin{problem}(Problem 2)
Show that the following limit does not exist:
\[
\lim_{(x,y) \to (0,0)} \frac{xy}{x^2 + y^2}
\]
\end{problem}

\begin{example}[Example 3]
Show that the following limit does not exist:
\[
\lim_{(x,y) \to (0,0)} \frac{x^2y}{x^4 + y^2}
\]
Along the axes, $f(x,y) = 0$ giving a limit of zero. Along the line $y = x$ we obtain 
\[
\lim_{x\to 0} \frac{x^3}{x^4 + x^2} = \lim_{x\to 0} \frac{x}{x^2 + 1} = 0
\]
Along lines $ y = mx$, we have
\[
\lim_{x\to 0} \frac{mx^3}{x^4 + m^2x^2} = \lim_{x\to 0} \frac{mx}{x^2 + m^2} = 0
\]
All of the paths so far yield the same limit.  Now, however, consider a parabolic path toward the origin in which $y = x^2$.
Along this path the limit becomes
\[
\lim_{(x,y) \to (0,0)} \frac{x^2y}{x^4 + y^2} =  \lim_{x \to 0} \frac{x^4}{2x^4} = \frac12
\]
Finally, we get a different limit and only now we can conclude that
\[
\lim_{(x,y) \to (0,0)} \frac{x^2y}{x^4 + y^2} \quad \text{DNE}
\]

\end{example}

\begin{problem}(Problem 3)
Show that the following limit does not exist:
\[
\lim_{(x,y) \to (0,0)} \frac{xy^2}{x^2 + y^4}
\]
\end{problem}


We now look at a limit that does exist, and we will use the squeeze theorem to determine its value.

\begin{example}[Example 4]
Compute the following limit:
\[
\lim_{(x,y) \to (0,0)} \frac{xy^2}{x^2 + y^2}
\]
We begin by using the fact that for any real number $x$, we have
\[
-|x| \leq x \leq |x| 
\]
combined with the fact that if $a\leq b$ and $c\geq 0$, then $ac \leq bc$ to obtain
\[
-\frac{|x|y^2}{x^2 + y^2} \leq \frac{xy^2}{x^2 + y^2} \leq \frac{|x|y^2}{x^2 + y^2}
\]
We now use the fact that
\[
\quad y^2 \leq x^2 + y^2
\]
together with the fact that if $a \leq b$ then $-a \geq -b$ to obtain
\[
-\frac{|x|(x^2 + y^2)}{x^2 + y^2} \leq \frac{xy^2}{x^2 + y^2} \leq \frac{|x|(x^2 + y^2)}{x^2 + y^2}
\]
Simplifying the left and right members of the compound inequality, we have
\[
-|x| \leq \frac{xy^2}{x^2 + y^2} \leq |x|
\]
Clearly, 
\[
\lim_{(x,y) \to (0,0)} (-|x|) = \lim_{(x,y) \to (0,0)} |x| = 0
\]
and so by the squeeze theorem, 
\[
\lim_{(x,y) \to (0,0)} \frac{xy^2}{x^2 + y^2} =0
\]
as well.
\end{example}

\begin{problem}(Problem 4)
Compute the following limit:
\[
\lim_{(x,y) \to (0,0)} \frac{x^4y}{x^4 + y^4}
\]
\begin{hint}
Use $-|y| \leq y \leq |y|$ to create a compound inequality
\end{hint}
\begin{hint}
Use the squeeze theorem
\end{hint}
\end{problem}

\begin{example}[Example 5]
Compute the following limit:
\[
\lim_{(x,y) \to (0,0)} \frac{x^3y^2}{x^4 + y^4}
\]
We will use the squeeze theorem again.  The key inequality we need is
\[
x^2y^2 \leq x^4 + y^4 
\]
To understand this inequality, consider
\[
0 \leq (x^2 - y^2)^2 = x^4 -2x^2y^2 + y^4
\]
which gives
\[
2x^2y^2 \leq x^4 + y^4
\]
and the key inequality follows since $x^2y^2 \leq 2x^2y^2$.
As in the previous example, we now use $-|x| \leq x \leq |x|$ combined with the key inequality to obtain
\[
\frac{x^3y^2}{x^4 + y^4} \leq \frac{|x|x^2y^2}{x^4 + y^4} \leq \frac{|x|(x^4 + y^4)}{x^4 + y^4} = |x|
\]
and 
\[
-|x| = -\frac{|x|(x^4 + y^4)}{x^4 + y^4} \leq -\frac{|x|x^2y^2}{x^4 + y^4} \leq \frac{x^3y^2}{x^4 + y^4}
\]
Hence,
\[
-|x| \leq \frac{x^3y^2}{x^4 + y^4} \leq |x|
\]
and as in example 4, the squeeze theorem implies
\[
\lim_{(x,y) \to (0,0)} \frac{x^3y^2}{x^4 + y^4} = 0
\]
\end{example}


\begin{problem}(Problem 5)
Compute the following limit:
\[
\lim_{(x,y) \to (0,0)} \frac{x^5y^4}{x^8 + y^8}
\]
\begin{hint}
First show that $x^4y^4 \leq x^8 + y^8$
\end{hint}
\begin{hint}
Use $-|x| \leq x \leq |x|$ to create a compound inequality
\end{hint}
\begin{hint}
Use the squeeze theorem
\end{hint}
\end{problem}
The preceding examples and problems require your intuition about limits.  That is,
\[
\lim_{(x,y) \to (a,b)} f(x,y) = L
\]
means that if the point $(x,y)$ is ``near" the point $(a,b)$, then the value of the function $f(x, y)$ is ``near" $L$.
The next definition makes this precise.


\begin{definition}[Limit]
Let $\epsilon > 0$. We say that
\[
\lim_{(x,y) \to (a,b)} f(x,y) = L
\]
if there exists a number $\delta >0$ such that:
\[
\text{if} \quad 0< \left|(x,y) - (a,b)\right| < \delta \quad \text{then} \quad \left|f(x,y) - L\right| < \epsilon
\]
Here, the notation $|(x,y) - (a,b)|$ represents the distance in $\R^2$ between the points $(x,y)$ and $(a,b)$, i.e.,
\[
|(x,y) - (a,b)| = \sqrt{(x-a)^2 + (y-b)^2}
\]
\end{definition}

\begin{remark}
The inequality $0< |(x,y) - (a,b)|$ in the definition of a limit reflects the fact that the point $(x,y)$ is never equal to $(a,b)$
as it approaches the point $(a,b)$.
\end{remark}

\begin{example}[Example 6]
Use the definition of limit to show that 
\[
\lim_{(x,y) \to (a,b)} x = a
\]
Let $\epsilon >0$.  We must find a number $\delta >0$ such that
\[
\text{if} \quad \sqrt{(x-a)^2 + (y-b)^2} < \delta \quad \text{then} \quad |x-a| < \epsilon
\]
If we let $\delta = \epsilon$ we have
\[
|x-a| = \sqrt{(x-a)^2} \leq \sqrt{(x-a)^2 + (y-b)^2} < \delta = \epsilon
\]
Hence, we can conclude from the definition of limit that
\[
\lim_{(x,y) \to (a,b)} x = a
\]
\end{example}

\begin{problem}(Problem 6a)
Use the definition of limit to show that 
\[
\lim_{(x,y) \to (a,b)} y = b
\]
\end{problem}

\begin{problem}(Problem 6c)
Use the definition of limit to show that the limit of a constant is the constant:
\[
\lim_{(x,y) \to (a,b)} c = c
\]
\begin{hint}
Let $\epsilon >0$ and choose any positive value for $\delta$
\end{hint}
\end{problem}


\section{Continuity}

\begin{definition}[Continuity]
We say that the function $f(x,y)$ is continuous at a point $(a,b)$ in the domain of $f$ if
\[
\lim_{(x,y) \to (a,b)} f(x,y) = f(a,b)
\]
\end{definition}

Example 6, problem 6a and problem 6b show that the functions $f(x,y) = x, f(x,y) = y$ and $f(x,y) = c$ are continuous at every point in $\R^2$.

We can use the following limit laws to help us construct other continuous functions.

\subsection{Limit Laws}
For readability, we abbreviate $\lim_{(x,y) \to (a,b)} f(x,y)$ with $\lim f(x,y)$.\\

Suppose that $\lim f(x,y)$ and $\lim g(x,y)$ exist, then we can conclude each of the following:\\

\textbf{Sum Law} $\quad \displaystyle  \lim [f(x,y) + g(x,y)] = \lim f(x,y) +\lim g(x,y) $\\

\textbf{Constant Multiple Law} $\quad \displaystyle \lim[cf(x,y) ] = c\lim f(x,y) $\\

\textbf{Product Law} $\quad \displaystyle \lim[f(x,y) \cdot g(x,y)] = \lim f(x,y) \cdot \lim g(x,y)$\\ 
 
\textbf{Quotient Law} $\quad \displaystyle \lim \frac{f(x,y)}{g(x,y)} = \frac{\lim f(x,y)}{\lim g(x,y)} $\\

Note that the quotient law is valid provided that the limit in the denominator is not zero.


\begin{example}[Example 7]
Use the limit laws to show that 
\[
\lim_{(x,y) \to (2,-1)} 5x^2y^3 = -20
\]
We have
\begin{align*}
\lim_{(x,y) \to (2,-1)} 5x^2y^3 &= 5\lim x^2y^3 \quad \text{(Constant Multiple Law)}\\
  &= 5 \lim x^2 \cdot \lim y^3  \quad \text{(Product Law)}\\
  &= 5 \left(\lim x\right)^2 \cdot \left(\lim y\right)^3  \quad \text{(Product Law (3 times))}\\
  &= 5(2)^2(-1)^3 \quad \text{(Example 6 and Problem 6a)}\\
  &= -20
\end{align*}
where we have used ``$\lim$" as an abbreviation for the given limit.
Note that this calculation shows that the function $f(x,y) = 5x^2y^3$ is continuous at the point $(2, -1)$ since $f(2, -1) = -20$.
\end{example}


\begin{problem}(Problem 7)
Use the limit laws to show that 
\[
\lim_{(x,y) \to (3,2)} (2x - y) = 4
\]
\begin{hint}
Use the constant multiple law to change the subtraction into addition
\end{hint}
\end{problem}

\subsection{Continuous Functions}
A polynomial in the variables $x$ and $y$ is a function of the form
\[
\sum ax^ny^m
\]
where $n$ and $m$ are non-negative integers and the number of terms in the sum is finite.
For example, the function
\[
f(x,y) = 3x + 2y - 5xy
\]
is a polynomial in $x$ and $y$. The next proposition follows from the limit laws.

\begin{proposition}
Polynomials in $x$ and $y$ are continuous at every point in $\R^2$.
\end{proposition}

A rational function in the variables $x$ and $y$ is a ratio of polynomials in the variables $x$ and $y$.
For example, the function
\[
f(x,y) = \frac{2x^2y^3 - 5}{x^4 + 3y^2}
\]
is a rational function in the variables $x$ and $y$.
The next proposition follows from the limit laws.

\begin{proposition}
Rational functions in $x$ and $y$ are continuous at every point in $\R^2$ where the denominator is nonzero.
\end{proposition}

The next proposition involves a composition of a function of two variables with a function of a single variable.

\begin{proposition}
Suppose $f(x,y)$ is continuous at the point $(a,b)$ and $g(u)$ is continuous at the point $ c = f(a,b)$.
The the function $g(f(x,y))$ is continuous at the point $(a,b)$.
\end{proposition}

The functions $\sin(x), \cos(x)$ and $e^x$ are continuous for all $x$.  Hence, the functions
\[
\sin(f(x,y)),\; \cos(f(x,y)) \quad \text{and} \quad e^{f(x,y)}
\]
are continuous at any point $(a,b)$ where the function $f$ is continuous. 
The function $\ln(f(x,y))$ is continuous at any point $(a,b)$ where $f$ is continuous and positive.
Where is the function $\sqrt{f(x,y)}$ continuous?


\end{document}
