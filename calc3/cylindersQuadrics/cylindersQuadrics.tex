\documentclass[handout]{ximera}

%% You can put user macros here
%% However, you cannot make new environments



\newcommand{\ffrac}[2]{\frac{\text{\footnotesize $#1$}}{\text{\footnotesize $#2$}}}
\newcommand{\vasymptote}[2][]{
    \draw [densely dashed,#1] ({rel axis cs:0,0} -| {axis cs:#2,0}) -- ({rel axis cs:0,1} -| {axis cs:#2,0});
}


\graphicspath{{./}{firstExample/}}
\usepackage{forest}
\usepackage{amsmath}
\usepackage{amssymb}
\usepackage{array}
\usepackage[makeroom]{cancel} %% for strike outs
\usepackage{pgffor} %% required for integral for loops
\usepackage{tikz}
\usepackage{tikz-cd}
\usepackage{tkz-euclide}
\usetikzlibrary{shapes.multipart}


%\usetkzobj{all}
\tikzstyle geometryDiagrams=[ultra thick,color=blue!50!black]


\usetikzlibrary{arrows}
\tikzset{>=stealth,commutative diagrams/.cd,
  arrow style=tikz,diagrams={>=stealth}} %% cool arrow head
\tikzset{shorten <>/.style={ shorten >=#1, shorten <=#1 } } %% allows shorter vectors

\usetikzlibrary{backgrounds} %% for boxes around graphs
\usetikzlibrary{shapes,positioning}  %% Clouds and stars
\usetikzlibrary{matrix} %% for matrix
\usepgfplotslibrary{polar} %% for polar plots
\usepgfplotslibrary{fillbetween} %% to shade area between curves in TikZ



%\usepackage[width=4.375in, height=7.0in, top=1.0in, papersize={5.5in,8.5in}]{geometry}
%\usepackage[pdftex]{graphicx}
%\usepackage{tipa}
%\usepackage{txfonts}
%\usepackage{textcomp}
%\usepackage{amsthm}
%\usepackage{xy}
%\usepackage{fancyhdr}
%\usepackage{xcolor}
%\usepackage{mathtools} %% for pretty underbrace % Breaks Ximera
%\usepackage{multicol}



\newcommand{\RR}{\mathbb R}
\newcommand{\R}{\mathbb R}
\newcommand{\C}{\mathbb C}
\newcommand{\N}{\mathbb N}
\newcommand{\Z}{\mathbb Z}
\newcommand{\dis}{\displaystyle}
%\renewcommand{\d}{\,d\!}
\renewcommand{\d}{\mathop{}\!d}
\newcommand{\dd}[2][]{\frac{\d #1}{\d #2}}
\newcommand{\pp}[2][]{\frac{\partial #1}{\partial #2}}
\renewcommand{\l}{\ell}
\newcommand{\ddx}{\frac{d}{\d x}}

\newcommand{\zeroOverZero}{\ensuremath{\boldsymbol{\tfrac{0}{0}}}}
\newcommand{\inftyOverInfty}{\ensuremath{\boldsymbol{\tfrac{\infty}{\infty}}}}
\newcommand{\zeroOverInfty}{\ensuremath{\boldsymbol{\tfrac{0}{\infty}}}}
\newcommand{\zeroTimesInfty}{\ensuremath{\small\boldsymbol{0\cdot \infty}}}
\newcommand{\inftyMinusInfty}{\ensuremath{\small\boldsymbol{\infty - \infty}}}
\newcommand{\oneToInfty}{\ensuremath{\boldsymbol{1^\infty}}}
\newcommand{\zeroToZero}{\ensuremath{\boldsymbol{0^0}}}
\newcommand{\inftyToZero}{\ensuremath{\boldsymbol{\infty^0}}}


\newcommand{\numOverZero}{\ensuremath{\boldsymbol{\tfrac{\#}{0}}}}
\newcommand{\dfn}{\textbf}
%\newcommand{\unit}{\,\mathrm}
\newcommand{\unit}{\mathop{}\!\mathrm}
%\newcommand{\eval}[1]{\bigg[ #1 \bigg]}
\newcommand{\eval}[1]{ #1 \bigg|}
\newcommand{\seq}[1]{\left( #1 \right)}
\renewcommand{\epsilon}{\varepsilon}
\renewcommand{\iff}{\Leftrightarrow}

\DeclareMathOperator{\arccot}{arccot}
\DeclareMathOperator{\arcsec}{arcsec}
\DeclareMathOperator{\arccsc}{arccsc}
\DeclareMathOperator{\si}{Si}
\DeclareMathOperator{\proj}{proj}
\DeclareMathOperator{\scal}{scal}
\DeclareMathOperator{\cis}{cis}
\DeclareMathOperator{\Arg}{Arg}
%\DeclareMathOperator{\arg}{arg}
\DeclareMathOperator{\Rep}{Re}
\DeclareMathOperator{\Imp}{Im}
\DeclareMathOperator{\sech}{sech}
\DeclareMathOperator{\csch}{csch}
\DeclareMathOperator{\Log}{Log}

\newcommand{\tightoverset}[2]{% for arrow vec
  \mathop{#2}\limits^{\vbox to -.5ex{\kern-0.75ex\hbox{$#1$}\vss}}}
\newcommand{\arrowvec}{\overrightarrow}
\renewcommand{\vec}{\mathbf}
\newcommand{\veci}{{\boldsymbol{\hat{\imath}}}}
\newcommand{\vecj}{{\boldsymbol{\hat{\jmath}}}}
\newcommand{\veck}{{\boldsymbol{\hat{k}}}}
\newcommand{\vecl}{\boldsymbol{\l}}
\newcommand{\utan}{\vec{\hat{t}}}
\newcommand{\unormal}{\vec{\hat{n}}}
\newcommand{\ubinormal}{\vec{\hat{b}}}

\newcommand{\dotp}{\bullet}
\newcommand{\cross}{\boldsymbol\times}
\newcommand{\grad}{\boldsymbol\nabla}
\newcommand{\divergence}{\grad\dotp}
\newcommand{\curl}{\grad\cross}
%% Simple horiz vectors
\renewcommand{\vector}[1]{\left\langle #1\right\rangle}


\outcome{In this section discover cylinders and quadric surfaces.}

\title{1.8 Cylinders and Quadric Surfaces}
%Vectors are represented graphically by arrows.
%and in three dimensions we write $\vec{v} = \vector{x,y, z}$.
%The length of the arrow represents the magnitude of the vector and the arrow points in the direction of the vector.

\begin{document}

\begin{abstract}
In this section we discover cylinders and quadric surfaces in $\R^3$.
\end{abstract}
 
\maketitle

\section{Cylinders}
We are all familiar with the right, circular cylinders like the one shown below.


\begin{image}
\begin{tikzpicture}
\node (A) [cylinder, shape border rotate=90, draw,minimum height=1.5cm,minimum width=1cm, color = blue!70!white, fill = blue!30!white]{};

\draw[dashed, blue!70!white] (0,-0.535) ellipse (0.5 and 0.115);
\end{tikzpicture}
\end{image}

While we frequently refer to this as just a``cylinder", the right, circular cylinder is just one of a much larger class of surfaces known as cylinders.

A {\bf cylinder} in $\R^3$ is a surface that is defined in terms of a plane curve and parallel lines passing through the curve.
In the case of the right circular cylinder, the curve is a circle and the lines pass through the circle perpendicular to the plane containing the circle.
\begin{image}
\begin{tikzpicture}
\node (A) [cylinder, shape border rotate=90, draw,minimum height=3cm,minimum width=2cm, color = blue!30!white, fill = blue!30!white]{};
\draw[blue!70!white, thick] (0,0.1) ellipse (1 and 0.075);
\draw[blue!50!white, thick] (0,1.475) ellipse (1 and 0.11);
\draw[blue!50!white, thick] (0,-1.3) ellipse (1 and 0.11);
\draw[blue!50!white, thick] (-1, -1.3) -- (-1, 1.475);
\draw[blue!70!white, <->, thick] (1, -1.3) -- (1, 1.475);
\filldraw[blue!70!white] (1, 0.1) circle (0.05);
\draw[thin] (0.8, 0.1) -- (0.8, 0.3) -- (1, 0.3);
\node[blue] at (0, -2) {\scriptsize Generating circle and one of the ruling lines};
\node[blue] at (0, -2.3) {\scriptsize for a right, circular cylinder};
\end{tikzpicture}
\end{image}

In general, a \textbf{cylinder} in $\R^3$ is a surface created from a plane curve by passing parallel lines 
through each point on the the curve. The plane curve is called the \textbf{generating curve} and the parallel 
lines are called the \textbf{rulings}. See the figure below.
\begin{image}
\begin{tikzpicture}
\draw[thick, ->] (0.5,-0.5) -- (2.5,-0.5) node[right]{$y$};
\draw[thick, ->] (0.5,-0.5) -- (0.5,1.3) node[above]{$z$};
\draw[thick, ->] (0.5,-0.50) -- (-0.7,-1.4) node[below, left]{$x$};

\foreach \x in {0,2, 4}
\draw[blue!70!white] (-2.5+\x, -0.7+0.5*\x) parabola (-3+\x, 1.3+0.5*\x)  
        (-2.5+\x, -0.7+0.5*\x) parabola (-2+\x, 0.7+0.5*\x); 
        
\draw[blue!70!white, thick] (-0.5, 0.3) parabola (-1, 2.3)  
        (-0.5, 0.3) parabola (0, 1.7); 
        
\draw[blue!70!white, thick, <->] (1.5, 1.3) -- (-2.5, -0.7);  
\draw[blue!70!white, thick, <->] (2, 2.7) -- (-2, 0.7);  
\draw[blue!70!white, thick, <->] (1, 3.3) -- (-3, 1.3);
\node[blue] at (-1, -2) {Cylinder in $\R^3$ generated by a parabola};  
\end{tikzpicture}
\end{image}

%\node at (0.1, -0.2){$O$} ;\draw[blue!40!white] (4,4) ellipse (0.7 and 0.2);
%\draw[blue!40!white] (4,1.26) ellipse (0.7 and 0.2);

%\draw[blue!40!white] (4,2.63) ellipse (0.7 and 0.2);

%\draw[thick, blue!40!white, <->] (3.3,1.26) -- (3.3, 4);
%\draw[thick,blue!40!white] (4.7,1.26) -- (4.7, 4);

For simplicity, we will consider cylinders in which the curve lies in one of the 
three coordinate planes and the {\bf rulings} are perpendicular to this plane.
In this case, the equation of the cylinder will have an easily recognized form: 
one of the variables, $x, y$ or $z$ will be missing from the equation.
For example, the equation $x^2+y^2 = 1$ makes the unit circle in the $xy$-plane.  
As an equation in $\R^3$, the variable $z$ is missing.  
This means that the surface corresponding to this equation is created by lines 
parallel to the $z$-axis which pass through the circle.
The resulting surface is, of course, a right, circular cylinder).
\begin{example}[Example 1]
Describe the surface in $\R^3$ given by the equation
\[
y= x^2
\]
The graph of $y = x^2$ in the $xy$-plane is a parabola opening upward with vertex at the origin.
Since the variable $z$ is missing from the equation, the corresponding surface in $\R^3$
is a cylinder with rulings parallel to the $z$-axis.
\end{example}

\begin{problem}(Problem 1a)
Describe the surface in $\R^3$ given by the equation
\[
z= x^2
\]
\end{problem}

\begin{problem}(Problem 1b)
Describe the surface in $\R^3$ given by the equation
\[
z= e^y
\]
\end{problem}

\begin{problem}(Problem 1c)
Describe the surface in $\R^3$ given by the equation
\[
y= \sin{x}
\]
\end{problem}

\begin{problem}(Problem 1d)
Describe the surface in $\R^3$ given by the equation
\[
\frac{x^2}{4} + \frac{y^2}{9} = 1
\]
\end{problem}



\section{Quadric Surfaces}
A quadric surface is a three dimensional analogue of a conic section.
The primary conic sections are the circle, ellipse, parabola and hyperbola. 
These can be drawn as curves in $\R^2$.
The corresponding surfaces in $\R^3$ are the sphere, ellipsoid, paraboloid and hyperboloid. 
\begin{image}
\begin{tikzpicture}
\begin{axis}[title = A hyperbolic paraboloid]
\addplot3 [surf,shader=flat,draw=black] {x^2-y^2} ;
\addlegendentry{$z = x^2 -y^2$};
\end{axis}


\end{tikzpicture}
\end{image}

\begin{image}
\begin{tikzpicture}
\begin{axis}[title = An elliptic paraboloid]
\addplot3 [domain = -4:4, domain y = -4:4, surf, shader=flat,draw=black] {x^2+y^2};
\addlegendentry{$z = x^2 +y^2$};
\end{axis}
\end{tikzpicture}
\end{image}


\begin{image}
\begin{tikzpicture}
\begin{axis}[title = An ellipsoid (top half only)]
\addplot3 [domain = -4:4, domain y = -4:4, surf, shader=flat,draw=black] {(32 - x^2 - y^2)^(1/2)};
\addlegendentry{$z = \sqrt{32 - x^2  - y^2}$};
\end{axis}
\end{tikzpicture}
\end{image}

\begin{image}
\begin{tikzpicture}
\begin{axis}[title = A hyperboloid of one sheet (top half only)]
\addplot3 [domain = 2:4, domain y = -4:4, surf, shader=flat,draw=black] {(x^2 + y^2 - 4)^(1/2)};
\addlegendentry{$z = \sqrt{ x^2  + y^2 - 4}$};
\end{axis}
\end{tikzpicture}
\end{image}

\begin{image}
\begin{tikzpicture}
\begin{axis}[title = A hyperboloid of two sheets (top half only)]
\addplot3 [domain = -4:4, domain y = -4:4, surf, shader=flat,draw=black] {(x^2 + y^2 + 1)^(1/2)};
\addlegendentry{$z = \sqrt{ 1+ x^2  + y^2 }$};
\end{axis}
\end{tikzpicture}
\end{image}

\begin{image}
\begin{tikzpicture}
\begin{axis}[title = A hyperboloid of two sheets (bottom half only)]
\addplot3 [domain = -4:4, domain y = -4:4, surf, shader=flat,draw=black] {-(x^2 + y^2 + 1)^(1/2)};
\addlegendentry{$z = -\sqrt{ 1+ x^2  + y^2 }$};
\end{axis}
\end{tikzpicture}
\end{image}


\link[Online 3D surface plotter]{https://academo.org/search/?q=surface}
\link[Another 3D grapher]{https://math.libretexts.org/Learning_Objects/CalcPlot3D_Interactive_Figures/CalcPlot3D}

The general equation of a quadric surface is 
\[
Ax^2 + By^2 + Cz^2 + Dxy + Exz + Fyz + Gx + Hy + Iz + K = 0
\]
This equation can be simplified a bit if we "center" the surface at the origin and 
align it with the $z$-axis where applicable. 
Below we see a list of quadric surfaces and the simplest equation that can be used to describe them.\\

The equation of an \textbf{ellipsoid} centered at the origin with axes parallel to the coordinate axes, is given by 
\[
\frac{x^2}{a^2} + \frac{y^2}{b^2}+ \frac{z^2}{c^2} = 1
\]
Note that a sphere is a special ellipsoid with $ a=b=c$.
The traces in planes parallel to the coordinate planes are ellipses.
\begin{image}
\includegraphics{Ellipsoid.png}
\end{image}


\begin{image}
\begin{tikzpicture}

  \pgfmathsetmacro{\p}{1}
  \pgfmathsetmacro{\q}{1.5}
  \begin{axis}[
    %xlabel = {$x$},
    %ylabel = {$y$},
    %zlabel = {$z$},
    view = {60}{30},
    domain = 0 : pi,
    y domain = 0 : 2 * pi,
    z buffer = sort,
    unit vector ratio = 1 1,
    hide axis,
    colormap/violet,
    declare function = {
      xp(\x, \y) = sin(deg(\x)) * cos(deg(\y));
      yp(\x, \y) = \p * sin(deg(\x)) * sin(deg(\y));
      zp(\x, \y) = \q * cos(deg(\x));
    }, ]
    \addplot3[patch]({xp(x, y)}, {yp(x, y)}, {zp(x, y)});
    %\draw[->] (1, 0, 0) -- (2, 0, 0) node[right]{$x$};
    %\draw[->] (0, \p, 0) -- (0, 2, 0) node[right]{$y$};
    %\draw[->] (0, 0, \q) -- (0, 0, 2) node[above]{$z$};
  \end{axis}
\end{tikzpicture}
\begin{tikzpicture}

  \pgfmathsetmacro{\p}{1.0}
  \pgfmathsetmacro{\q}{0.5}
  \begin{axis}[
    xlabel = {$x$},
    ylabel = {$y$},
    zlabel = {$z$},
    view = {60}{30},
    domain = 0 : pi,
    y domain = 0 : 2 * pi,
    z buffer = sort,
    unit vector ratio = 1 1,
    hide axis,
    colormap/violet,
    declare function = {
      xp(\x, \y) = sin(deg(\x)) * cos(deg(\y));
      yp(\x, \y) = \p * sin(deg(\x)) * sin(deg(\y));
      zp(\x, \y) = \q * cos(deg(\x));
    }, ]
    \addplot3[patch]({xp(x, y)}, {yp(x, y)}, {zp(x, y)});
    %\draw[->] (1, 0, 0) -- (2, 0, 0) node[right]{$x$};
    %\draw[->] (0, \p, 0) -- (0, 2, 0) node[right]{$y$};
    %\draw[->] (0, 0, \q) -- (0, 0, 2) node[above]{$z$};
  \end{axis}
\end{tikzpicture}
\end{image}


\begin{example}[Example 2]
Describe the trace of the ellipsoid in the plane $z = 2$:
\[
\frac{x^2}{4} + \frac{y^2}{9}+ \frac{z^2}{16} = 1
\]
Setting $z = 2$ in the equation of the ellipsoid yields
\[
\frac{x^2}{4} + \frac{y^2}{9} = \frac34
\]
This can be put in the form
\[
\frac{x^2}{a^2} + \frac{y^2}{b^2} = 1
\]
which is the equation of an ellipse centered at the origin. The precise values of $a$ and $b$ are
\[
a = \frac{\sqrt 3}{3} \quad \text{and} \quad b = \frac{3\sqrt 3}{2}
\]
The graph of this ellipse in the $xy$-plane is shown below.
The actual trace is identical to this ellipse, except that it is in the plane $z = 2$, which is parallel to the $xy$-plane.

\begin{image}
\begin{tikzpicture}
\draw[<->] (-1.4, 0) -- (1.4, 0) node[right]{$x$};
\draw[<->] (0, -1.8) -- (0, 1.8) node[above]{$y$};
\node[ellipse, draw = blue!70!white,
    minimum width = 1.2cm, 
    minimum height = 2cm] (e) at (0,0) {};
\draw (0.15,1) -- (-0.15, 1) node[ left]{$b$};
\draw (0.15,-1) -- (-0.15, -1) node[ left]{$-b$};
\draw (0.6, 0.15) -- (0.6, -0.15) node[ right]{$a$};
\draw (-0.6, 0.15) -- (-0.6, -0.15) node[left]{$-a$};
\end{tikzpicture}
\end{image}

\end{example}

\begin{problem}(Problem 2a)
Describe the trace of the ellipsoid in the plane $y = 2$:
\[
\frac{x^2}{36} + \frac{y^2}{8}+ \frac{z^2}{25} = 1
\]
The trace is \wordChoice{\choice{ a parabola}\choice[correct]{an ellipse} \choice{a tiger}}
\end{problem}

The equation of an \textbf{elliptic paraboloid} is given by 
\[
\frac{z}{c} = \frac{x^2}{a^2} + \frac{y^2}{b^2} 
\]
Traces of the elliptic paraboloid inplanes of the form $z = k$ are ellipses (so long as $k/c > 0$).
On the other hand, traces in planes parallel to the $xz$ and $yz$ planes are parabolas.

\begin{problem}[Problem 2b]
Find the trace of elliptical paraboloid in the plane $y = 3$:
\[
z = 2x^2 + \frac{y^2}{36}
\]
\end{problem}

The equation of a \textbf{hyperbolic paraboloid} is given by 
\[
\frac{z}{c} = \frac{x^2}{a^2} - \frac{y^2}{b^2} 
\]

\begin{image}
\begin{tikzpicture}
\begin{axis}[
grid=major,
axis lines=middle,
inner axis line style={=>},
ticks=none
]
\addplot3 [surf,shader=flat,fill opacity=.4,draw=black] {y^2-x^2};
\end{axis}
\end{tikzpicture}
\end{image}


The equation of a \textbf{hyperboloid of one sheet} is given by 
\[
\frac{x^2}{a^2} + \frac{y^2}{b^2}- \frac{z^2}{c^2} = 1
\]

\begin{problem}[Problem 2c]
Find and describe the trace of hyperboloid of one sheet in the plane $x = -1$:
\[
\frac{x^2}{4} + \frac{y^2}{9} - 4z^2 = 1
\]
\begin{hint}
The graph of the equation
\[
\frac{x^2}{a^2} - \frac{y^2}{b^2} = 1
\]
is a hyperbola in the $xy$-plane.
\end{hint}
\end{problem}

The equation of a \textbf{hyperboloid of two sheets} is given by 
\[
-\frac{x^2}{a^2} -\frac{y^2}{b^2} + \frac{z^2}{c^2} = 1
\]

The equation of a \textbf{cone} is given by 
\[
\frac{z^2}{c^2} = \frac{x^2}{a^2} + \frac{y^2}{b^2}
\]

Note: the orientation of the above surfaces can be changed by interchanging the variables.

\end{document}








\[
\vec{v} = \vector{x, y, z}
\]
The magnitude of a vector in $\R^3$ comes from the distance formula:
\[
|\vec{v}| = |\vector{x, y, z}| = \sqrt{x^2 + y^2 + z^2}
\]
The special basis vectors in $\R^3$ are
\begin{align*}
\vec{i} &= \vector{1, 0, 0}\\
\vec{j} &= \vector{0, 1, 0}\\
\vec{k} &= \vector{0, 0, 1}\\
\end{align*}
These are unit vectors in the direction of the positive $x, y$ and $z$-axes respectively.
A vector in $\R^3$ can be expressed in terms of these vectors:
\[
\vec{v} = \vector{x, y, z} = x\vec{i} + y\vec{j} + z\vec{k}
\]

The zero vector in $\R^3$ is given by
\[
\vec{0} = \vector{0,0,0}
\]
has a magnitude of $0$ and is not assigned a direction.
As in $\R^2$, a vector in $\R^3$ has an initial point and a final point.  The vector is in standard position if its initial point is the origin.
Also, as in $\R^2$, a the vector with initial point $P = (x_1, y_1, z_1)$ and final point $Q = (x_2, y_2, z_2)$ is given by the difference of the coordinates:
\[
\avec{PQ} = \vector{x_2 - x_1, y_2-y_1, z_2-z_1}
\]
The magnitude of this vector is the distance between the points $P$ and $Q$
\[
|\avec{PQ}| = \sqrt{(x_2-x_1)^2 + (y_2-y_1)^2 +(z_2-z_1)^2} = d(P, Q)
\]
The operations of scalar multiplication and addition are performed analogously to those in $\R^2$.
If $\vec{v} = \vector{x, y, z}$ and if $c$ is a scalar in $\R$, then
\[
c\vec{v} = c\vector{x, y, z} = \vector{cx, cy, cz}
\]
and if $\vec{v}_1 = \vector{x_1, y_1, z_1}$ and $\vec{v}_2 = \vector{x_2, y_2, z_2}$ are vectors in $\R^3$ then
\[
\vec{v}_1 + \vec{v}_2 = \vector{x_1+x_2, y_1+y_2, z_1+z_2}
\]
The effect on magnitude of multiplication by a scalar is the same in $\R^3$ as it was in $\R^2$:
\[
|c\vec{v}| = |c| |\vec{v}|
\]
Because of this, a unit vector in the same direction as a non-zero vector $\vec{v}$ in $\R^3$ is given by
\[
\vec{u} = \frac{1}{|\vec{v}|} \vec{v}
\]
just as in $\R^2$.

Due to their component-wise computation, the vector operations of scalar multiplication and addition have some familiar properties:
\begin{align*}
&\text{Distributive Property:} & &c(\vec{v}_1 + \vec{v}_2) = c\vec{v}_1 + c\vec{v}_2   \\
& \text{Commutative Property:}& &\vec{v}_1 + \vec{v}_2 = \vec{v}_2 + \vec{v}_1 \\
& \text{Associative Property:}&  &\vec{v}_1 + (\vec{v}_2+ \vec{v}_3) =  (\vec{v}_1 + \vec{v}_2) + \vec{v}_3 \\
&\text{Identity Property:} & &\vec{v} + \vec{0} = \vec{v} \\
& \text{Additive Inverse Property:} & &\vec{v} + (-\vec{v}) = \vec{0} 
\end{align*}

\begin{example}
Find a unit vector in the same direction as $\vec{v} = 2 \vec{i} - 3\vec{j} + 4\vec{k}$.\\
The magnitude of $\vec{v}$ is
\[
|\vec{v}| = |\vector{2, -3, 4}| = \sqrt{4+9+16} = \sqrt{29}
\]
Hence, a unit vector in the direction of $\vec{v}$ is
\[
\vec{u} = \frac{1}{\sqrt{29}}\vec{v} = \frac{2}{\sqrt{29}}\vec{i} - \frac{3}{\sqrt{29}}\vec{j} + \frac{4}{\sqrt{29}}\vec{k}
\]
\end{example}

\begin{problem}
Find each of the following:\\
a) $3\vector{2, 4, -1} - 4\vector{5, -3, 2} = \vector{\answer{-14}, \answer{24}, \answer{-11}}$\\
b) $|3\vec{i} - 5\vec{k}| = \answer{\sqrt{34}}$\\
c) a unit vector in the direction of $\vector{1, -2, 2} = \vector{\answer{1/3}, \answer{-2/3}, \answer{2/3}}$
\end{problem}


\end{document}
