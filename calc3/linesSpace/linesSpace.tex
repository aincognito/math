\documentclass[handout]{ximera}

%% You can put user macros here
%% However, you cannot make new environments



\newcommand{\ffrac}[2]{\frac{\text{\footnotesize $#1$}}{\text{\footnotesize $#2$}}}
\newcommand{\vasymptote}[2][]{
    \draw [densely dashed,#1] ({rel axis cs:0,0} -| {axis cs:#2,0}) -- ({rel axis cs:0,1} -| {axis cs:#2,0});
}


%\usepackage{tcolorbox} %%Needed for Derivative Definition supposedly and product rule, natural exp log, quotient rule, inverse trig, rates of change


% \graphicspath{{./}{firstExample/}}
% \usepackage{forest}
\usepackage{amsmath}
\usepackage{amssymb}
\usepackage{array}
\usepackage[makeroom]{cancel} %% for strike outs
\usepackage{pgffor} %% required for integral for loops
\usepackage{tikz}
\usepackage{tikz-cd}
\usepackage{tkz-euclide}
\usetikzlibrary{shapes.multipart}


% \usetkzobj{all}
\tikzstyle geometryDiagrams=[ultra thick,color=blue!50!black]


\usetikzlibrary{arrows}
\tikzset{>=stealth,commutative diagrams/.cd,
  arrow style=tikz,diagrams={>=stealth}} %% cool arrow head
\tikzset{shorten <>/.style={ shorten >=#1, shorten <=#1 } } %% allows shorter vectors

\usetikzlibrary{backgrounds} %% for boxes around graphs
\usetikzlibrary{shapes,positioning}  %% Clouds and stars
\usetikzlibrary{matrix} %% for matrix
\usepgfplotslibrary{polar} %% for polar plots
\usepgfplotslibrary{fillbetween} %% to shade area between curves in TikZ



%\usepackage[width=4.375in, height=7.0in, top=1.0in, papersize={5.5in,8.5in}]{geometry}
%\usepackage[pdftex]{graphicx}
%\usepackage{tipa}
%\usepackage{txfonts}
%\usepackage{textcomp}
%\usepackage{amsthm}
%\usepackage{xy}
%\usepackage{fancyhdr}
%\usepackage{xcolor}
%\usepackage{mathtools} %% for pretty underbrace % Breaks Ximera
%\usepackage{multicol}



\newcommand{\RR}{\mathbb R}
\newcommand{\R}{\mathbb R}
\newcommand{\C}{\mathbb C}
\newcommand{\N}{\mathbb N}
\newcommand{\Z}{\mathbb Z}
\newcommand{\dis}{\displaystyle}
%\renewcommand{\d}{\,d\!}
\renewcommand{\d}{\mathop{}\!d}
\newcommand{\dd}[2][]{\frac{\d #1}{\d #2}}
\newcommand{\pp}[2][]{\frac{\partial #1}{\partial #2}}
\renewcommand{\l}{\ell}
\newcommand{\ddx}{\frac{d}{\d x}}
\newcommand{\ppx}{\frac{\partial}{\partial x}}
\newcommand{\ppy}{\frac{\partial}{\partial y}}

\newcommand{\zeroOverZero}{\ensuremath{\boldsymbol{\tfrac{0}{0}}}}
\newcommand{\inftyOverInfty}{\ensuremath{\boldsymbol{\tfrac{\infty}{\infty}}}}
\newcommand{\zeroOverInfty}{\ensuremath{\boldsymbol{\tfrac{0}{\infty}}}}
\newcommand{\zeroTimesInfty}{\ensuremath{\small\boldsymbol{0\cdot \infty}}}
\newcommand{\inftyMinusInfty}{\ensuremath{\small\boldsymbol{\infty - \infty}}}
\newcommand{\oneToInfty}{\ensuremath{\boldsymbol{1^\infty}}}
\newcommand{\zeroToZero}{\ensuremath{\boldsymbol{0^0}}}
\newcommand{\inftyToZero}{\ensuremath{\boldsymbol{\infty^0}}}


\newcommand{\numOverZero}{\ensuremath{\boldsymbol{\tfrac{\#}{0}}}}
\newcommand{\dfn}{\textbf}
%\newcommand{\unit}{\,\mathrm}
\newcommand{\unit}{\mathop{}\!\mathrm}
%\newcommand{\eval}[1]{\bigg[ #1 \bigg]}
\newcommand{\eval}[1]{ #1 \bigg|}
\newcommand{\seq}[1]{\left( #1 \right)}
\renewcommand{\epsilon}{\varepsilon}
\renewcommand{\iff}{\Leftrightarrow}

\DeclareMathOperator{\arccot}{arccot}
\DeclareMathOperator{\arcsec}{arcsec}
\DeclareMathOperator{\arccsc}{arccsc}
\DeclareMathOperator{\si}{Si}
\DeclareMathOperator{\proj}{proj}
\DeclareMathOperator{\scal}{scal}
\DeclareMathOperator{\cis}{cis}
\DeclareMathOperator{\Arg}{Arg}
%\DeclareMathOperator{\arg}{arg}
\DeclareMathOperator{\Rep}{Re}
\DeclareMathOperator{\Imp}{Im}
\DeclareMathOperator{\sech}{sech}
\DeclareMathOperator{\csch}{csch}
\DeclareMathOperator{\Log}{Log}

\newcommand{\tightoverset}[2]{% for arrow vec
  \mathop{#2}\limits^{\vbox to -.5ex{\kern-0.75ex\hbox{$#1$}\vss}}}
\newcommand{\arrowvec}{\overrightarrow}
\renewcommand{\vec}{\mathbf}
\newcommand{\veci}{{\boldsymbol{\hat{\imath}}}}
\newcommand{\vecj}{{\boldsymbol{\hat{\jmath}}}}
\newcommand{\veck}{{\boldsymbol{\hat{k}}}}
\newcommand{\vecl}{\boldsymbol{\l}}
\newcommand{\utan}{\vec{\hat{t}}}
\newcommand{\unormal}{\vec{\hat{n}}}
\newcommand{\ubinormal}{\vec{\hat{b}}}

\newcommand{\dotp}{\bullet}
\newcommand{\cross}{\boldsymbol\times}
\newcommand{\grad}{\boldsymbol\nabla}
\newcommand{\divergence}{\grad\dotp}
\newcommand{\curl}{\grad\cross}
%% Simple horiz vectors
\renewcommand{\vector}[1]{\left\langle #1\right\rangle}


\outcome{In this section we describe lines in space analytically.}

\title{1.6 Lines in Space}
%Vectors are represented graphically by arrows.
%and in three dimensions we write $\vec{v} = \vector{x,y, z}$.
%The length of the arrow represents the magnitude of the vector and the arrow points in the direction of the vector.
\begin{document}

\begin{abstract}
In this section we describe lines in space analytically.
\end{abstract}
 
\maketitle

A line in $\R^3$ is determined by two points. To describe the line analytically, i.e., to find the equation (or equations) of the line,
we will need the vector associated with these two points and we will also need to take advantage of the end to end method for adding vectors.
Suppose the line $L$ in $\R^3$ goes through the points $P(x_1, y_1, z_1)$ and $Q(x_2, y_2, z_2)$. The vector from $P$ to $Q$ is given by
\[
\vec{v} = \vec{PQ} = \vector{x_2-x_1,y_2-y_1,z_2-z_1}
\]
This vector is called the direction vector of the line. 
A point on the line can be seen as the final point of and vector emanating from $P$ and parallel to $\vec{v}$.
To describe such points, we rely on the end to end method of vector addition. A point $(x, y, z)$ lies on the line $L$ if
it is the final point of a vector of the form:
\[
\vector{x_1, y_1, z_1} + t\vector{v}
\]
where $t$ is any scalar. See the figure below.

\begin{image}
\begin{tikzpicture}
\draw[ ->] (0,0) -- (1, 2) ;
\draw[red, ->] (1,2) -- (5, 3);
\draw[blue, fill] (3, 2.5) circle (0.05) node[above]{$Q$};
\draw[blue, fill] (1, 2) circle (0.05)node[above]{$P$};
\draw[blue, fill] (0,0) circle (0.05) node[left]{$O$};
\node at (2.5, -0.5) {The vector $\vector{x_1, y_1, z_1} + t\vec{v}$ (in red)};
\end{tikzpicture}
\end{image}
 
If we associate the vector $\vector{x, y, z}$ with the point $(x, y, z)$, then by the end to end method of adding vectors, we see that
\[
\vector{x, y, z} = \vector{x_1, y_1, z_1} + t\vec{v}
\]
To simplify the notation somewhat, if the vector $\vec{v}$ has the form $\vector{a, b, c}$, then this equation becomes
\[
\vector{x, y, z} = \vector{x_1, y_1, z_1} + t\vector{a, b, c}
\]
This is called the {\bf vector form} of the equation of the line $L$ in $\R^3$ which passes through the 
point $(x_1, y_1, z_1)$ and has direction vector $\vec{v} = \vector{a, b, c}$.\\
Isolating the variables $x, y$ and $z$ by equating the components of the vector form, we obtain the {\bf parametric form} $L$:
\begin{align*}
x &= x_1 + at\\
y &= y_1 + bt\\
z &= z_1 + ct
\end{align*}
Solving these equations for $t$ (assuming $a, b$ and $c$ are all non-zero) we obtain the {\bf symmetric form} of $L$:
\[
\frac{x-x_1}{a} = \frac{y-y_1}{b} = \frac{z-z_1}{c}
\]

\begin{example}
Find the vector form, parametric form and symmetric form for the line in $\R^3$ passing through the points $(2, 1, -4)$ and $(-3, 2, 5)$.\\
The direction vector for the line is 
\[
\vec{v} = \vector{-3-2, 2-1, 5-(-4)} = \vector{-5, 1, 9}
\]
Note that any non-zero multiple of this vector would also serve as a suitable direction vector for $L$. Using the point $(2, 1, -4)$ we have
\[
\vector{x, y, z} = \vector{2, 1, 4} + t\vector{-5, 1, 9} \quad \text{(Vector form)}
\]
\begin{align*}
x &= 2 -5t\\
y &= 1 + t \quad\text{(Parametric form)}\\
z &= 4 + 9t\\
\end{align*}
and
\[
\frac{x-2}{-5} = y-1 = \frac{z-4}{9} \quad \text{(Symmetric form)}
\]
\end{example}
\end{document}
 

