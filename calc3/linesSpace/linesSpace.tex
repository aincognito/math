\documentclass[handout]{ximera}

%% You can put user macros here
%% However, you cannot make new environments



\newcommand{\ffrac}[2]{\frac{\text{\footnotesize $#1$}}{\text{\footnotesize $#2$}}}
\newcommand{\vasymptote}[2][]{
    \draw [densely dashed,#1] ({rel axis cs:0,0} -| {axis cs:#2,0}) -- ({rel axis cs:0,1} -| {axis cs:#2,0});
}


\graphicspath{{./}{firstExample/}}
\usepackage{forest}
\usepackage{amsmath}
\usepackage{amssymb}
\usepackage{array}
\usepackage[makeroom]{cancel} %% for strike outs
\usepackage{pgffor} %% required for integral for loops
\usepackage{tikz}
\usepackage{tikz-cd}
\usepackage{tkz-euclide}
\usetikzlibrary{shapes.multipart}


%\usetkzobj{all}
\tikzstyle geometryDiagrams=[ultra thick,color=blue!50!black]


\usetikzlibrary{arrows}
\tikzset{>=stealth,commutative diagrams/.cd,
  arrow style=tikz,diagrams={>=stealth}} %% cool arrow head
\tikzset{shorten <>/.style={ shorten >=#1, shorten <=#1 } } %% allows shorter vectors

\usetikzlibrary{backgrounds} %% for boxes around graphs
\usetikzlibrary{shapes,positioning}  %% Clouds and stars
\usetikzlibrary{matrix} %% for matrix
\usepgfplotslibrary{polar} %% for polar plots
\usepgfplotslibrary{fillbetween} %% to shade area between curves in TikZ



%\usepackage[width=4.375in, height=7.0in, top=1.0in, papersize={5.5in,8.5in}]{geometry}
%\usepackage[pdftex]{graphicx}
%\usepackage{tipa}
%\usepackage{txfonts}
%\usepackage{textcomp}
%\usepackage{amsthm}
%\usepackage{xy}
%\usepackage{fancyhdr}
%\usepackage{xcolor}
%\usepackage{mathtools} %% for pretty underbrace % Breaks Ximera
%\usepackage{multicol}



\newcommand{\RR}{\mathbb R}
\newcommand{\R}{\mathbb R}
\newcommand{\C}{\mathbb C}
\newcommand{\N}{\mathbb N}
\newcommand{\Z}{\mathbb Z}
\newcommand{\dis}{\displaystyle}
%\renewcommand{\d}{\,d\!}
\renewcommand{\d}{\mathop{}\!d}
\newcommand{\dd}[2][]{\frac{\d #1}{\d #2}}
\newcommand{\pp}[2][]{\frac{\partial #1}{\partial #2}}
\renewcommand{\l}{\ell}
\newcommand{\ddx}{\frac{d}{\d x}}

\newcommand{\zeroOverZero}{\ensuremath{\boldsymbol{\tfrac{0}{0}}}}
\newcommand{\inftyOverInfty}{\ensuremath{\boldsymbol{\tfrac{\infty}{\infty}}}}
\newcommand{\zeroOverInfty}{\ensuremath{\boldsymbol{\tfrac{0}{\infty}}}}
\newcommand{\zeroTimesInfty}{\ensuremath{\small\boldsymbol{0\cdot \infty}}}
\newcommand{\inftyMinusInfty}{\ensuremath{\small\boldsymbol{\infty - \infty}}}
\newcommand{\oneToInfty}{\ensuremath{\boldsymbol{1^\infty}}}
\newcommand{\zeroToZero}{\ensuremath{\boldsymbol{0^0}}}
\newcommand{\inftyToZero}{\ensuremath{\boldsymbol{\infty^0}}}


\newcommand{\numOverZero}{\ensuremath{\boldsymbol{\tfrac{\#}{0}}}}
\newcommand{\dfn}{\textbf}
%\newcommand{\unit}{\,\mathrm}
\newcommand{\unit}{\mathop{}\!\mathrm}
%\newcommand{\eval}[1]{\bigg[ #1 \bigg]}
\newcommand{\eval}[1]{ #1 \bigg|}
\newcommand{\seq}[1]{\left( #1 \right)}
\renewcommand{\epsilon}{\varepsilon}
\renewcommand{\iff}{\Leftrightarrow}

\DeclareMathOperator{\arccot}{arccot}
\DeclareMathOperator{\arcsec}{arcsec}
\DeclareMathOperator{\arccsc}{arccsc}
\DeclareMathOperator{\si}{Si}
\DeclareMathOperator{\proj}{proj}
\DeclareMathOperator{\scal}{scal}
\DeclareMathOperator{\cis}{cis}
\DeclareMathOperator{\Arg}{Arg}
%\DeclareMathOperator{\arg}{arg}
\DeclareMathOperator{\Rep}{Re}
\DeclareMathOperator{\Imp}{Im}
\DeclareMathOperator{\sech}{sech}
\DeclareMathOperator{\csch}{csch}
\DeclareMathOperator{\Log}{Log}

\newcommand{\tightoverset}[2]{% for arrow vec
  \mathop{#2}\limits^{\vbox to -.5ex{\kern-0.75ex\hbox{$#1$}\vss}}}
\newcommand{\arrowvec}{\overrightarrow}
\renewcommand{\vec}{\mathbf}
\newcommand{\veci}{{\boldsymbol{\hat{\imath}}}}
\newcommand{\vecj}{{\boldsymbol{\hat{\jmath}}}}
\newcommand{\veck}{{\boldsymbol{\hat{k}}}}
\newcommand{\vecl}{\boldsymbol{\l}}
\newcommand{\utan}{\vec{\hat{t}}}
\newcommand{\unormal}{\vec{\hat{n}}}
\newcommand{\ubinormal}{\vec{\hat{b}}}

\newcommand{\dotp}{\bullet}
\newcommand{\cross}{\boldsymbol\times}
\newcommand{\grad}{\boldsymbol\nabla}
\newcommand{\divergence}{\grad\dotp}
\newcommand{\curl}{\grad\cross}
%% Simple horiz vectors
\renewcommand{\vector}[1]{\left\langle #1\right\rangle}


\outcome{In this section we describe lines in space analyically.}

\title{1.6 Lines in Space}
%Vectors are represented graphically by arrows.
%and in three dimensions we write $\vec{v} = \vector{x,y, z}$.
%The length of the arrow represents the magnitude of the vector and the arrow points in the direction of the vector.
\begin{document}

\begin{abstract}
In this section we describe lines in space analytically.
\end{abstract}
 
\maketitle

A line in $\R^3$ is determined by two points. To describe the line analytically, i.e., to find the equation (or equations) of the line,
we will need the vector associated with these two points and we will also need to take advantage of the end to end method for adding vecotrs.
Suppose the line $L$ in $\R^3$ goes through the points $P(x_1, y_1, z_1)$ and $Q(x_2, y_2, z_2)$. The vector from $P$ to $Q$ is given by
\[
\vec{v} = \vec{PQ} = \vector{x_2-x_1,y_2-y_1,z_2-z_1}
\]
This vector is called the direction vector of the line. 
A point on the line can be seen as the final point of and vector eminating from $P$ and parallel to $\vec{v}$.
To describe such points, we rely on the end to end method of vector addition. A point $(x, y, z)$ lies on the line $L$ if
it is the final point of a vector of the form:
\[
\vector{x_1, y_1, z_1} + t\vector{v}
\]
where $t$ is any scalar. See the figure below.

\begin{image}
\begin{tikzpicture}
\draw[ ->] (0,0) -- (1, 2) ;
\draw[red, ->] (1,2) -- (5, 3);
\draw[blue, fill] (3, 2.5) circle (0.05) node[above]{$Q$};
\draw[blue, fill] (1, 2) circle (0.05)node[above]{$P$};
\draw[blue, fill] (0,0) circle (0.05) node[left]{$O$};
\node at (2.5, -0.5) {The vector $\vector{x_1, y_1, z_1} + t\vec{v}$ (in red)};
\end{tikzpicture}
\end{image}
 
If we associate the vector $\vector{x, y, z}$ with the point $(x, y, z)$, then by the end to end method of adding vectors, we see that
\[
\vector{x, y, z} = \vector{x_1, y_1, z_1} + t\vec{v}
\]
To simplify the notation somewhat, if the vector $\vec{v}$ has the form $\vector{a, b, c}$, then this equation becomes
\[
\vector{x, y, z} = \vector{x_1, y_1, z_1} + t\vector{a, b, c}
\]
This is called the {\bf vector form} of the equation of the line $L$ in $\R^3$ which passes through the 
point $(x_1, y_1, z_1)$ and has direction vector $\vec{v} = \vector{a, b, c}$.\\
Isolating the variables $x, y$ and $z$ by equating the components of the vector form, we obtain the {\bf parametric form} $L$:
\begin{align*}
x &= x_1 + at\\
y &= y_1 + bt\\
z &= z_1 + ct
\end{align*}
Solving these equations for $t$ (assuming $a, b$ and $c$ are all non-zero) we obtain the {\bf symmetric form} of $L$:
\[
\frac{x-x_1}{a} = \frac{y-y_1}{b} = \frac{z-z_1}{c}
\]

\begin{example}
Find the vector form, parametric form and symmetric form for the line in $\R^3$ passing through the points $(2, 1, -4)$ and $(-3, 2, 5)$.\\
The direction vector for the line is 
\[
\vec{v} = \vector{-3-2, 2-1, 5-(-4)} = \vector{-5, 1, 9}
\]
Note that any non-zero multiple of this vector would also serve as a suitable direction vector for $L$. Using the point $(2, 1, -4)$ we have
\[
\vector{x, y, z} = \vector{2, 1, 4} + t\vector{-5, 1, 9} \quad \text{(Vector form)}
\]
\begin{align*}
x &= 2 -5t\\
y &= 1 + t \quad\text{(Parametric form)}\\
z &= 4 + 9t\\
\end{align*}
and
\[
\frac{x-2}{-5} = y-1 = \frac{z-4}{9} \quad \text{(Symmetric form)}
\]
\end{example}
\end{document}
 

Just as in $\R^2$, a vector in $\R^3$ is a quantity that has both magnitude and direction.
In $\R^3$, vectors have three components rather than two:
\[
\vec{v} = \vector{x, y, z}
\]
The magnitude of a vector in $\R^3$ comes from the distance formula:
\[
|\vec{v}| = |\vector{x, y, z}| = \sqrt{x^2 + y^2 + z^2}
\]
The special basis vectors in $\R^3$ are
\begin{align*}
\vec{i} &= \vector{1, 0, 0}\\
\vec{j} &= \vector{0, 1, 0}\\
\vec{k} &= \vector{0, 0, 1}\\
\end{align*}
These are unit vectors in the direction of the positive $x, y$ and $z$-axes respectively.
A vector in $\R^3$ can be expressed in terms of these vectors:
\[
\vec{v} = \vector{x, y, z} = x\vec{i} + y\vec{j} + z\vec{k}
\]

The zero vector in $\R^3$ is given by
\[
\vec{0} = \vector{0,0,0}
\]
has a magnitude of $0$ and is not assigned a direction.
As in $\R^2$, a vector in $\R^3$ has an initial point and a final point.  The vector is in standard position if its initial point is the origin.
Also, as in $\R^2$, a the vector with initial point $P = (x_1, y_1, z_1)$ and final point $Q = (x_2, y_2, z_2)$ is given by the difference of the coordinates:
\[
\vec{PQ} = \vector{x_2 - x_1, y_2-y_1, z_2-z_1}
\]
The magnitude of this vector is the distance between the points $P$ and $Q$
\[
|\vec{PQ}| = \sqrt{(x_2-x_1)^2 + (y_2-y_1)^2 +(z_2-z_1)^2} = d(P, Q)
\]
The operations of scalar multiplication and addition are performed analogously to those in $\R^2$.
If $\vec{v} = \vector{x, y, z}$ and if $c$ is a scalar in $\R$, then
\[
c\vec{v} = c\vector{x, y, z} = \vector{cx, cy, cz}
\]
and if $\vec{v}_1 = \vector{x_1, y_1, z_1}$ and $\vec{v}_2 = \vector{x_2, y_2, z_2}$ are vectors in $\R^3$ then
\[
\vec{v}_1 + \vec{v}_2 = \vector{x_1+x_2, y_1+y_2, z_1+z_2}
\]
The effect on magnitude of multiplication by a scalar is the same in $\R^3$ as it was in $\R^2$:
\[
|c\vec{v}| = |c| |\vec{v}|
\]
Because of this, a unit vector in the same direction as a non-zero vector $\vec{v}$ in $\R^3$ is given by
\[
\vec{u} = \frac{1}{|\vec{v}|} \vec{v}
\]
just as in $\R^2$.

Due to their component-wise computation, the vector operations of scalar multiplication and addition have some familiar properties:
\begin{align*}
&\text{Distributive Property:} & &c(\vec{v}_1 + \vec{v}_2) = c\vec{v}_1 + c\vec{v}_2   \\
& \text{Commutative Property:}& &\vec{v}_1 + \vec{v}_2 = \vec{v}_2 + \vec{v}_1 \\
& \text{Associative Property:}&  &\vec{v}_1 + (\vec{v}_2+ \vec{v}_3) =  (\vec{v}_1 + \vec{v}_2) + \vec{v}_3 \\
&\text{Identity Property:} & &\vec{v} + \vec{0} = \vec{v} \\
& \text{Additive Inverse Property:} & &\vec{v} + (-\vec{v}) = \vec{0} 
\end{align*}

\begin{example}
Find a unit vector in the same direction as $\vec{v} = 2 \vec{i} - 3\vec{j} + 4\vec{k}$.\\
The magnitude of $\vec{v}$ is
\[
|\vec{v}| = |\vector{2, -3, 4}| = \sqrt{4+9+16} = \sqrt{29}
\]
Hence, a unit vector in the direction of $\vec{v}$ is
\[
\vec{u} = \frac{1}{\sqrt{29}}\vec{v} = \frac{2}{\sqrt{29}}\vec{i} - \frac{3}{\sqrt{29}}\vec{j} + \frac{4}{\sqrt{29}}\vec{k}
\]
\end{example}

\begin{problem}
Find each of the following:\\
a) $3\vector{2, 4, -1} - 4\vector{5, -3, 2} = \vector{\answer{-14}, \answer{24}, \answer{-11}}$\\
b) $|3\vec{i} - 5\vec{k}| = \answer{\sqrt{34}}$\\
c) a unit vector in the direction of $\vector{1, -2, 2} = \vector{\answer{1/3}, \answer{-2/3}, \answer{2/3}}$
\end{problem}


\end{document}
