\documentclass[handout]{ximera}

%% You can put user macros here
%% However, you cannot make new environments



\newcommand{\ffrac}[2]{\frac{\text{\footnotesize $#1$}}{\text{\footnotesize $#2$}}}
\newcommand{\vasymptote}[2][]{
    \draw [densely dashed,#1] ({rel axis cs:0,0} -| {axis cs:#2,0}) -- ({rel axis cs:0,1} -| {axis cs:#2,0});
}


%\usepackage{tcolorbox} %%Needed for Derivative Definition supposedly and product rule, natural exp log, quotient rule, inverse trig, rates of change


% \graphicspath{{./}{firstExample/}}
% \usepackage{forest}
\usepackage{amsmath}
\usepackage{amssymb}
\usepackage{array}
\usepackage[makeroom]{cancel} %% for strike outs
\usepackage{pgffor} %% required for integral for loops
\usepackage{tikz}
\usepackage{tikz-cd}
\usepackage{tkz-euclide}
\usetikzlibrary{shapes.multipart}


% \usetkzobj{all}
\tikzstyle geometryDiagrams=[ultra thick,color=blue!50!black]


\usetikzlibrary{arrows}
\tikzset{>=stealth,commutative diagrams/.cd,
  arrow style=tikz,diagrams={>=stealth}} %% cool arrow head
\tikzset{shorten <>/.style={ shorten >=#1, shorten <=#1 } } %% allows shorter vectors

\usetikzlibrary{backgrounds} %% for boxes around graphs
\usetikzlibrary{shapes,positioning}  %% Clouds and stars
\usetikzlibrary{matrix} %% for matrix
\usepgfplotslibrary{polar} %% for polar plots
\usepgfplotslibrary{fillbetween} %% to shade area between curves in TikZ



%\usepackage[width=4.375in, height=7.0in, top=1.0in, papersize={5.5in,8.5in}]{geometry}
%\usepackage[pdftex]{graphicx}
%\usepackage{tipa}
%\usepackage{txfonts}
%\usepackage{textcomp}
%\usepackage{amsthm}
%\usepackage{xy}
%\usepackage{fancyhdr}
%\usepackage{xcolor}
%\usepackage{mathtools} %% for pretty underbrace % Breaks Ximera
%\usepackage{multicol}



\newcommand{\RR}{\mathbb R}
\newcommand{\R}{\mathbb R}
\newcommand{\C}{\mathbb C}
\newcommand{\N}{\mathbb N}
\newcommand{\Z}{\mathbb Z}
\newcommand{\dis}{\displaystyle}
%\renewcommand{\d}{\,d\!}
\renewcommand{\d}{\mathop{}\!d}
\newcommand{\dd}[2][]{\frac{\d #1}{\d #2}}
\newcommand{\pp}[2][]{\frac{\partial #1}{\partial #2}}
\renewcommand{\l}{\ell}
\newcommand{\ddx}{\frac{d}{\d x}}
\newcommand{\ppx}{\frac{\partial}{\partial x}}
\newcommand{\ppy}{\frac{\partial}{\partial y}}

\newcommand{\zeroOverZero}{\ensuremath{\boldsymbol{\tfrac{0}{0}}}}
\newcommand{\inftyOverInfty}{\ensuremath{\boldsymbol{\tfrac{\infty}{\infty}}}}
\newcommand{\zeroOverInfty}{\ensuremath{\boldsymbol{\tfrac{0}{\infty}}}}
\newcommand{\zeroTimesInfty}{\ensuremath{\small\boldsymbol{0\cdot \infty}}}
\newcommand{\inftyMinusInfty}{\ensuremath{\small\boldsymbol{\infty - \infty}}}
\newcommand{\oneToInfty}{\ensuremath{\boldsymbol{1^\infty}}}
\newcommand{\zeroToZero}{\ensuremath{\boldsymbol{0^0}}}
\newcommand{\inftyToZero}{\ensuremath{\boldsymbol{\infty^0}}}


\newcommand{\numOverZero}{\ensuremath{\boldsymbol{\tfrac{\#}{0}}}}
\newcommand{\dfn}{\textbf}
%\newcommand{\unit}{\,\mathrm}
\newcommand{\unit}{\mathop{}\!\mathrm}
%\newcommand{\eval}[1]{\bigg[ #1 \bigg]}
\newcommand{\eval}[1]{ #1 \bigg|}
\newcommand{\seq}[1]{\left( #1 \right)}
\renewcommand{\epsilon}{\varepsilon}
\renewcommand{\iff}{\Leftrightarrow}

\DeclareMathOperator{\arccot}{arccot}
\DeclareMathOperator{\arcsec}{arcsec}
\DeclareMathOperator{\arccsc}{arccsc}
\DeclareMathOperator{\si}{Si}
\DeclareMathOperator{\proj}{proj}
\DeclareMathOperator{\scal}{scal}
\DeclareMathOperator{\cis}{cis}
\DeclareMathOperator{\Arg}{Arg}
%\DeclareMathOperator{\arg}{arg}
\DeclareMathOperator{\Rep}{Re}
\DeclareMathOperator{\Imp}{Im}
\DeclareMathOperator{\sech}{sech}
\DeclareMathOperator{\csch}{csch}
\DeclareMathOperator{\Log}{Log}

\newcommand{\tightoverset}[2]{% for arrow vec
  \mathop{#2}\limits^{\vbox to -.5ex{\kern-0.75ex\hbox{$#1$}\vss}}}
\newcommand{\arrowvec}{\overrightarrow}
\renewcommand{\vec}{\mathbf}
\newcommand{\veci}{{\boldsymbol{\hat{\imath}}}}
\newcommand{\vecj}{{\boldsymbol{\hat{\jmath}}}}
\newcommand{\veck}{{\boldsymbol{\hat{k}}}}
\newcommand{\vecl}{\boldsymbol{\l}}
\newcommand{\utan}{\vec{\hat{t}}}
\newcommand{\unormal}{\vec{\hat{n}}}
\newcommand{\ubinormal}{\vec{\hat{b}}}

\newcommand{\dotp}{\bullet}
\newcommand{\cross}{\boldsymbol\times}
\newcommand{\grad}{\boldsymbol\nabla}
\newcommand{\divergence}{\grad\dotp}
\newcommand{\curl}{\grad\cross}
%% Simple horiz vectors
\renewcommand{\vector}[1]{\left\langle #1\right\rangle}


\outcome{In this section we define the double integral over a rectangle.}

\title{4.1 Double Integrals}



\begin{document}

\begin{abstract}
In this section we define the double integral over a rectangle.
\end{abstract}
 
\maketitle


Recall that for a function of one variable $f(x)$ defined and continuous on an interval $[a,b]$, we have the definite integral
\[
\int_a^b f(x) \, dx = \lim_{n \to \infty} \sum_{i = 1}^n f(x_i^*) \Delta x
\]
In other words, the definite integral is defined as a limit of Riemann Sums.  The points $f(x_i^*)$ are called sample points and they are 
distributed fairly evenly throughout the interval $[a,b]$.

For double integrals, we modify this approach by considering a rectangle in the domain of continuity of a function of two variables, $f(x,y)$.
Let $R$ be the rectangle defined by
\[
R = \{(x,y) | a\leq x \leq b, c\leq y \leq d\}
\]

\begin{definition}[Double Integral]
Let $f(x,y)$ be a continuous function defined on the rectangle $R$ described above. The double integral of $f$ over $R$ is defined by
\[
\iint_R f(x,y) \, dA = \lim_{m, n \to \infty} \sum_{j = 1}^m\sum_{i = 1}^n f(x_{ij}^*, y_{ij}^*) \Delta x \Delta y
\]
where the points $(x_{ij}^*, y_{ij}^*)$ are sample points roughly evenly distributed in the rectangle $R$.
\end{definition}

The double integral is thus defined as the limit of a double Riemann sum.

\begin{remark}
The double integral of a positive function $f(x,y)$ gives the volume of the solid situated above the rectangle $R$ and beneath the surface $z = f(x,y)$.
\end{remark}


\begin{example}[Example 1]
Compute $\iint_R 5 \, dA$ where $R$ is the rectangle with $-1 \leq x \leq 3$ and $-2 \leq y \leq 1$.\\
The function $f(x,y) = 5$ is continuous and positive on the rectangle $R$.  Since the function is constant and $R$ is a rectangle, the solid contained above $R$
and beneath the graph of $f$ is a rectangular solid whose volume is length $\times$ width $\times$ height.
Hence the value of the double integral is
\[
\iint_R 5 \, dA = 5[3-(-1)] \cdot [1-(-2)] = 5 \cdot 4 \cdot 3 = 60
\]
\end{example}

\begin{problem}(Problem 1)
Compute $\iint_R 40 \, dA$ where $R$ is the rectangle with $-20 \leq x \leq 20$ and $-5 \leq y \leq 15$.\\
\[
\iint_R 40 \, dA = \answer{32000}
\]
\end{problem}

\begin{example}[Example 2]
Compute $\iint_R \sqrt{4 - x^2} \, dA$ where $R$ is the square with $-2 \leq x \leq 2$ and $-2 \leq y \leq 2$.\\
The function $f(x,y) = \sqrt{4-x^2}$ is continuous and positive on the rectangle $R$.  To determine the nature of the 
surface $z = \sqrt{4-x^2}$, it is helpful to square both sides and rewrite as
\[
x^2 + z^2 = 4
\]
This is a circle of radius $2$ centered at the origin of the $xz$-plane.  In $\R^3$ this is a surface type called a cylinder (as discussed in section 1.8).
In fact, this particular surface is a right circular cylinder of radius $2$ whose central axis is the $y$-axis. The surface intersects the $xy$-plane in the 
band $-2 \leq x \leq 2$ which exactly matches the $x$-dimension of the square $R$.
Thus the double integral in question is asking for the volume of half (since $z \geq 0$) of this cylinder from $y = -2$ to $ y = 2$. Thus
\[
\iint_R \sqrt{4 - x^2} \, dA = \frac12 \pi r^2 h = \frac12 \pi \cdot 2^2 \cdot 4 = 8\pi
\]
\end{example}


\begin{problem}(Problem 2)
Compute $\iint_R \sqrt{9 - y^2} \, dA$ where $R$ is the rectangle with $-1 \leq x \leq 5$ and $-3 \leq y \leq 3$.\\
\[
\iint_R 40 \, dA = \answer{27\pi}
\]
\end{problem}

\begin{remark} If the function $f(x,y) \leq 0$ over the region $R$, then the double integral
\[
\iint_{R} f(x,y) dA \leq 0
\]
and its value represents the negative of the volume of the associated region.
\end{remark}

\begin{example}[Example 3]
Compute $\iint_R (-2) \, dA$ where $R$ is the rectangle with $-1 \leq x \leq 3$ and $-2 \leq y \leq 1$.\\
The function $f(x,y) = -2$ is continuous and negative on the rectangle $R$.  Since the function is constant and $R$ is a rectangle, the solid contained below $R$
and above the graph of $f$ is a rectangular solid whose volume is length $\times$ width $\times$ height.
Hence the value of the double integral is the negative of this volume
\[
\iint_R (-2) \, dA = (-2)[3-(-1)] \cdot [1-(-2)] = -2 \cdot 4 \cdot 3 = -24
\]
\end{example}

\begin{problem}(Problem 3)
Compute $\iint_R (-7) \, dA$ where $R$ is the rectangle with $-6 \leq x \leq -3$ and $-1 \leq y \leq 5$.\\
\[
\iint_R (-7) \, dA = \answer{-126}
\]
\end{problem}

We now consider double integrals that we are not able to compute from purely geometric considerations.

\subsection{Iterated Integrals}

\begin{theorem}[Fubini's Theorem]
If $f(x,y)$ is continuous on the rectangle $R$ given by $a \leq x \leq b$ and $c \leq y \leq d$
then
\[
\iint_R f(x,y) \, dA = \int_a^b \int_c^d f(x,y) \, dy\, dx = \int_c^d \int_a^b f(x,y) \, dx \, dy
\]
\end{theorem}

\begin{remark} The second and third integrals in Fubini's Theorem are called \textbf{iterated integrals}
and they are computed in a manner similar to second order mixed partial derivatives. We compute the inner integral first, treating the 
variable of the outer integral as a constant.  Then we compute the outer integral.
\end{remark}

\begin{example}[Example 4]
Compute $\iint_R (x^2 + y^2) \, dA$ where $R$ is the rectangle with $0 \leq x \leq 2$ and $0 \leq y \leq 3$.\\
The function $f(x,y) = x^2 + y^2$ is continuous on the rectangle $R$, so we can use Fubini's Theorem to compute the double integral.
We have
\begin{align*}
\iint_R (x^2 + y^2) \, dA &= \int_0^2 \int_0^3 (x^2 + y^2) \, dy\, dx\\
                          &= \int_0^2 \left(x^2 y + \frac13 y^3 \bigg|_0^3 \right) \, dx\\
                          &= \int_0^2 \left[(3x^2 + 9) - (0+0) \right] \, dx\\
                          &= \int_0^2 (3x^2 + 9) \, dx\\
                          &= x^3 + 9x \bigg|_0^2\\
                          &= 26
\end{align*}
We can also compute the iterated integral in the other order:
\begin{align*}
\iint_R (x^2 + y^2) \, dA &= \int_0^3 \int_0^2 (x^2 + y^2) \, dx\, dy\\
                          &= \int_0^3 \left( \frac13 x^3 + xy^2 \bigg|_0^2 \right) \, dy\\
                          &= \int_0^3 \left(\frac83 + 2y^2 \right) \, dy\\
                          &= \frac83 y + \frac23 y^3 \bigg|_0^3\\
                          &= 8 + 18 = 26
\end{align*}
\end{example}

\begin{problem}(Problem 4)
Compute $\iint_R (x^4 + y^4) \, dA$ where $R$ is the rectangle with $0 \leq x \leq 5$ and $0 \leq y \leq 1$.\\
\[
\iint_R (x^4 + y^4) \, dA = \answer{626}
\]
\end{problem}

\begin{example}[Example 5]
Compute $\iint_R (x^3y^2 - 4xy) \, dA$ where $R$ is the unit square with $0 \leq x \leq 1$ and $0 \leq y \leq 1$.\\
The function $f(x,y) = x^3y^2 - 4xy$ is continuous on the unit square, so we can use Fubini's Theorem to compute the double integral.
We have
\begin{align*}
\iint_R (x^3y^2 - 4xy) \, dA &= \int_0^1 \int_0^1 (x^3y^2 - 4xy) \, dy\, dx\\
                          &= \int_0^1 \left(\frac13 x^3 y^3  - 2xy^2 \bigg|_0^1 \right) \, dx\\
                          &= \int_0^1 \left(\frac13 x^3 - 2x \right) \, dx\\
                          &= \frac{1}{12} x^4 - x^2 \bigg|_0^1\\
                          &= -\frac{11}{12}
\end{align*}
\end{example}

\begin{remark}
In general the value of a double integral gives the volume of the region above the $xy$-plane minus
the volume of the region below the $xy$-plane.
\end{remark}

\begin{problem}(Problem 5)
Compute $\iint_R (x^2y - 2xy^2 \, dA$ where $R$ is the rectangle with $0 \leq x \leq 3$ and $0 \leq y \leq 4$.\\
\[
\iint_R (x^2 y -2xy^2) \, dA = \answer{-120}
\]
\end{problem}


\begin{example}[Example 6]
Compute $\iint_R xy\cos(x^2y) \, dA$ where $R$ is the unit square with $0 \leq x \leq 1$ and $0 \leq y \leq 1$.\\
The function $f(x,y) = xy\cos(x^2y)$ is continuous on the unit square, so we can use Fubini's Theorem to compute the double integral.
Thus the double integral can be written as either
\[
\int_0^1 \int_0^1 xy\cos(x^2y) \, dy\, dx \quad \text{or} \quad \int_0^1 \int_0^1 xy\cos(x^2y) \, dx\, dy
\]
and for this integral it makes a difference.
Since the integral
\[
\int x \cos(x^2) \, dx
\]
can be computed using $u$-substitution and the integral
\[
\int y \cos (y) \, dy
\]
requires integration by parts, the first option is to be preferred.
With the $u$-substitution $u = x^2y$, we have
\begin{align*}
\iint_R xy\cos(x^2y) \, dA &= \int_0^1 \int_0^1 xy\cos(x^2y) \, dx\, dy\\
                          &= \frac12 \int_0^1 \int_0^y \cos(u) \, du \, dy\\
                          &= \frac12 \int_0^1 \sin(y) \, dy\\
                          &= -\frac12 \cos(y) \bigg|_0^1\\
                          &= \frac12 \left(1 - \cos(1)\right)
\end{align*}
\end{example}


\begin{problem}(Problem 6)
Compute $\iint_R 2xy e^{xy^2} \, dA$ where $R$ is the unit square, $0 \leq x \leq 1$ and $0 \leq y \leq 1$.\\
\[
\iint_R 2xy e^{xy^2} \, dA = \answer{e-2}
\]
\end{problem}

\end{document}
