\documentclass[handout]{ximera}

%% You can put user macros here
%% However, you cannot make new environments



\newcommand{\ffrac}[2]{\frac{\text{\footnotesize $#1$}}{\text{\footnotesize $#2$}}}
\newcommand{\vasymptote}[2][]{
    \draw [densely dashed,#1] ({rel axis cs:0,0} -| {axis cs:#2,0}) -- ({rel axis cs:0,1} -| {axis cs:#2,0});
}


\graphicspath{{./}{firstExample/}}
\usepackage{forest}
\usepackage{amsmath}
\usepackage{amssymb}
\usepackage{array}
\usepackage[makeroom]{cancel} %% for strike outs
\usepackage{pgffor} %% required for integral for loops
\usepackage{tikz}
\usepackage{tikz-cd}
\usepackage{tkz-euclide}
\usetikzlibrary{shapes.multipart}


%\usetkzobj{all}
\tikzstyle geometryDiagrams=[ultra thick,color=blue!50!black]


\usetikzlibrary{arrows}
\tikzset{>=stealth,commutative diagrams/.cd,
  arrow style=tikz,diagrams={>=stealth}} %% cool arrow head
\tikzset{shorten <>/.style={ shorten >=#1, shorten <=#1 } } %% allows shorter vectors

\usetikzlibrary{backgrounds} %% for boxes around graphs
\usetikzlibrary{shapes,positioning}  %% Clouds and stars
\usetikzlibrary{matrix} %% for matrix
\usepgfplotslibrary{polar} %% for polar plots
\usepgfplotslibrary{fillbetween} %% to shade area between curves in TikZ



%\usepackage[width=4.375in, height=7.0in, top=1.0in, papersize={5.5in,8.5in}]{geometry}
%\usepackage[pdftex]{graphicx}
%\usepackage{tipa}
%\usepackage{txfonts}
%\usepackage{textcomp}
%\usepackage{amsthm}
%\usepackage{xy}
%\usepackage{fancyhdr}
%\usepackage{xcolor}
%\usepackage{mathtools} %% for pretty underbrace % Breaks Ximera
%\usepackage{multicol}



\newcommand{\RR}{\mathbb R}
\newcommand{\R}{\mathbb R}
\newcommand{\C}{\mathbb C}
\newcommand{\N}{\mathbb N}
\newcommand{\Z}{\mathbb Z}
\newcommand{\dis}{\displaystyle}
%\renewcommand{\d}{\,d\!}
\renewcommand{\d}{\mathop{}\!d}
\newcommand{\dd}[2][]{\frac{\d #1}{\d #2}}
\newcommand{\pp}[2][]{\frac{\partial #1}{\partial #2}}
\renewcommand{\l}{\ell}
\newcommand{\ddx}{\frac{d}{\d x}}

\newcommand{\zeroOverZero}{\ensuremath{\boldsymbol{\tfrac{0}{0}}}}
\newcommand{\inftyOverInfty}{\ensuremath{\boldsymbol{\tfrac{\infty}{\infty}}}}
\newcommand{\zeroOverInfty}{\ensuremath{\boldsymbol{\tfrac{0}{\infty}}}}
\newcommand{\zeroTimesInfty}{\ensuremath{\small\boldsymbol{0\cdot \infty}}}
\newcommand{\inftyMinusInfty}{\ensuremath{\small\boldsymbol{\infty - \infty}}}
\newcommand{\oneToInfty}{\ensuremath{\boldsymbol{1^\infty}}}
\newcommand{\zeroToZero}{\ensuremath{\boldsymbol{0^0}}}
\newcommand{\inftyToZero}{\ensuremath{\boldsymbol{\infty^0}}}


\newcommand{\numOverZero}{\ensuremath{\boldsymbol{\tfrac{\#}{0}}}}
\newcommand{\dfn}{\textbf}
%\newcommand{\unit}{\,\mathrm}
\newcommand{\unit}{\mathop{}\!\mathrm}
%\newcommand{\eval}[1]{\bigg[ #1 \bigg]}
\newcommand{\eval}[1]{ #1 \bigg|}
\newcommand{\seq}[1]{\left( #1 \right)}
\renewcommand{\epsilon}{\varepsilon}
\renewcommand{\iff}{\Leftrightarrow}

\DeclareMathOperator{\arccot}{arccot}
\DeclareMathOperator{\arcsec}{arcsec}
\DeclareMathOperator{\arccsc}{arccsc}
\DeclareMathOperator{\si}{Si}
\DeclareMathOperator{\proj}{proj}
\DeclareMathOperator{\scal}{scal}
\DeclareMathOperator{\cis}{cis}
\DeclareMathOperator{\Arg}{Arg}
%\DeclareMathOperator{\arg}{arg}
\DeclareMathOperator{\Rep}{Re}
\DeclareMathOperator{\Imp}{Im}
\DeclareMathOperator{\sech}{sech}
\DeclareMathOperator{\csch}{csch}
\DeclareMathOperator{\Log}{Log}

\newcommand{\tightoverset}[2]{% for arrow vec
  \mathop{#2}\limits^{\vbox to -.5ex{\kern-0.75ex\hbox{$#1$}\vss}}}
\newcommand{\arrowvec}{\overrightarrow}
\renewcommand{\vec}{\mathbf}
\newcommand{\veci}{{\boldsymbol{\hat{\imath}}}}
\newcommand{\vecj}{{\boldsymbol{\hat{\jmath}}}}
\newcommand{\veck}{{\boldsymbol{\hat{k}}}}
\newcommand{\vecl}{\boldsymbol{\l}}
\newcommand{\utan}{\vec{\hat{t}}}
\newcommand{\unormal}{\vec{\hat{n}}}
\newcommand{\ubinormal}{\vec{\hat{b}}}

\newcommand{\dotp}{\bullet}
\newcommand{\cross}{\boldsymbol\times}
\newcommand{\grad}{\boldsymbol\nabla}
\newcommand{\divergence}{\grad\dotp}
\newcommand{\curl}{\grad\cross}
%% Simple horiz vectors
\renewcommand{\vector}[1]{\left\langle #1\right\rangle}


\outcome{In this section we compute triple integrals over various regions.}

\title{4.4 Triple Integrals}



\begin{document}

\begin{abstract}
In this section we compute triple integrals over various regions.
\end{abstract}
 
\maketitle

A triple integral is computed in a manner similar to a double integral.  Instead of integrating over a region in the $xy$-plane,
we integrate over regions in $xyz$-space.  Thus, triple integrals will become three iterated integrals rather than two.
For simplicity, we will assume that our regions include boundaries for the variable $z$ of the form 
\[
g_1(x,y) \leq z \leq g_2(x,y)
\]
and we will integrate either $dz\, dy\, dx$ or $dz\, dx\, dy$.


\begin{example}[Example 1]
Compute $\iiint_R xyz \, dV$ where $R$ is the rectangular region bounded by the 
planes $x = 0, x = 3, y = 1, y = 2, z = -2$ and $z = 3$.\\
The region $R$ is rectangular with constant boundaries for each of the three variables.  Furthermore, the integrand
is a separable function, i.e., $f(x,y,z) = xyz = f_1(x) \cdot f_2(y) \cdot f_3(z)$. Hence, our iterated integrals can 
be written as a product of three ordinary definite integrals. We have:

\begin{align*}
\iiint_R xyz \, dV & = \int_0^3\int_1^2 \int_{-2}^3 xyz \,dz\, dy \, dx\\
                 & =\left(\int_0^3 x \, dx \right) \cdot \left(\int_1^2 y\, dy\right) \cdot \left(\int_{-2}^3z \,dz \right)\\
                 & = \frac92 \cdot \left(\frac{4}{2} -\frac12\right) \cdot \left(\frac92 - \frac42\right)\\
                 &=  \frac{135}{8}
\end{align*}      

\end{example}


\begin{problem}(Problem 1a)
Compute $\iiint_R xyz \, dV$ where $R$ is the rectangular region bounded by the 
planes $x = 1, x = 2, y = -1, y = 4, z = 0$ and $z = 3$.\\
\[
\iiint_R xyz \, dV = \answer{405/8}
\]
\end{problem}

\begin{problem}(Problem 1b)
Compute $\iiint_R \sin(2x)e^{-3y}\sqrt{1+ 2z} \, dV$ where $R$ is the rectangular region bounded by the 
planes $x = 0, x = \pi/2, y = 0, y = 1, z = 0$ and $z = 4$.\\
\[
\iiint_R \sin(2x)e^{-3y}\sqrt{1+ 2z} \, dV = \answer{26/9 (1 - 1/e^3)}
\]
\end{problem}

\begin{example}[Example 2]
Compute $\iiint_R x \, dV$ where $R$ is the tetrahedron in the first octant with 
vertices $(0,0,0), P(1, 0, 0), Q(0, 2, 0)$ and $R(0,0,3)$.


The plane determined by the points $P, Q$ and $R$ has normal vector $\vec{PQ} \cross \vec{PR} = \vector{6, 3, 2}$
and has equation 
\[
6x + 3y + 2z = 6
\]
Hence, the boundaries for $z$ come from
\[
0 \leq z \leq \frac{6 - 6x - 3y}{2}
\]

The base of the tetrahedron is the triangle in the $xy$-plane with vertices $(0,0), (1,0)$ and $(2,0)$. 
The line connecting the points $(1,0)$ and $(0,2)$ has equation $y = -2x + 2$.
Thus, the base of the tetrahedron can be described using the inequalities
\[
\quad 0 \leq y \leq -2x+2, \quad 0 \leq x \leq 1
\]
The triple integral over the tetrahedron can now be written as iterated integrals and computed:
\begin{align*}
\iiint_R x \, dV & = \int_0^1 \int_0^{-2x+2} \int_0^{3 - 3x - 3y/2} x \, dz\, dy \, dx\\
                 & = \int_0^1 \int_0^{-2x+2}  xz\bigg|_0^{3 - 3x - 3y/2} \, dy \, dx\\
                 & = \int_0^1 \int_0^{-2x+2}  (3x - 3x^2 - 3xy/2) \, dy \, dx\\
                 &= \int_0^1   (3xy - 3x^2y - 3xy^2/4)\bigg|_0^{-2x+2} \, dx\\
                 &= \int_0^1 3x(-2x+2) -3x^2(-2x+2) -3x(-2x+2)^2/4 \, dx\\
                 &= \int_0^1   (3x^3 - 6x^2  +  3x)   \, dx\\
                 &= \frac34 - 2 + \frac32 = \frac14
\end{align*}  

\end{example}

\begin{problem}(Problem 2)
Compute $\iiint_R y \, dV$ where $R$ is the tetrahedron in the first octant with 
vertices $(0,0,0), P(3, 0, 0), Q(0, 2, 0)$ and $R(0,0,1)$.\\
\begin{hint}
Integrate $dz \, dx \, dy$
\end{hint}
\[
\iiint_R y \, dV = \answer{1/2}
\]
\end{problem}

\begin{example}[Example 3]
Compute $\iiint_R  e^{z/y} \, dV$ where $R$ is the region in the first octant below the 
surface $z = xy$ and above the triangle in the $xy$-plane with vertices $(0,0), (1,0)$ and $(1,1)$.\\


The boundaries for $z$ are $0 \leq z \leq xy$. The boundaries for $x$ and $y$ come from the triangle bounded by the lines
$y = 0, x = 1$ and $y = x$.  We choose to integrate with respect to $x$ and then with respect to $y$ because the
integrand is (slightly) easier to deal with.  This detail is not evident until after we integrate with respect to $z$.
We have:

\begin{align*}
\iiint_R e^{z/y} \, dV & = \int_0^1 \int_y^1 \int_0^{xy} e^{z/y} \, dz \, dx \, dy\\
                 & = \int_0^1 \int_y^1 ye^{z/y}\bigg|_{0}^{xy} \,dx \, dy\\
                 & = \int_0^1 \int_y^1 \left(ye^x - y\right) \,dx \, dy\\
                 &= \int_0^1 \left(ye^x - xy\right) \bigg|_y^1 \, dy\\
                 &= \int_0^1 \left(ye - y - ye^y + y^2\right) \, dy\\
                 &= \frac{y^2 e}{2} - \frac{y^2}{2} - (y-1)e^y + \frac{y^3}{3} \bigg|_0^1\\
                 &= \frac{e}{2} - \frac12 + \frac13 - (-1)\\
                 &= \frac{e-1}{2} - \frac23
\end{align*} 

\end{example}

\begin{problem}(Problem 3)
Compute $\iiint_R  cos(z/x) \, dV$ where $R$ is the region in the first octant below the 
surface $z = xy$ and above the triangle in the $xy$-plane with vertices $(0,0), (\pi, 0)$ and $(\pi, \pi)$.\\
\begin{hint}
\[
\int f(ax) \, dx = \frac{1}{a} F(ax) + C
\]
\end{hint}
\[
\iiint_R  \cos(z/x) \, dV = \answer{2 +\frac{\pi^2}{2}}
\]

\end{problem}

\end{document}
