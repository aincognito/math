\documentclass[handout]{ximera}

%% You can put user macros here
%% However, you cannot make new environments



\newcommand{\ffrac}[2]{\frac{\text{\footnotesize $#1$}}{\text{\footnotesize $#2$}}}
\newcommand{\vasymptote}[2][]{
    \draw [densely dashed,#1] ({rel axis cs:0,0} -| {axis cs:#2,0}) -- ({rel axis cs:0,1} -| {axis cs:#2,0});
}


\graphicspath{{./}{firstExample/}}
\usepackage{forest}
\usepackage{amsmath}
\usepackage{amssymb}
\usepackage{array}
\usepackage[makeroom]{cancel} %% for strike outs
\usepackage{pgffor} %% required for integral for loops
\usepackage{tikz}
\usepackage{tikz-cd}
\usepackage{tkz-euclide}
\usetikzlibrary{shapes.multipart}


%\usetkzobj{all}
\tikzstyle geometryDiagrams=[ultra thick,color=blue!50!black]


\usetikzlibrary{arrows}
\tikzset{>=stealth,commutative diagrams/.cd,
  arrow style=tikz,diagrams={>=stealth}} %% cool arrow head
\tikzset{shorten <>/.style={ shorten >=#1, shorten <=#1 } } %% allows shorter vectors

\usetikzlibrary{backgrounds} %% for boxes around graphs
\usetikzlibrary{shapes,positioning}  %% Clouds and stars
\usetikzlibrary{matrix} %% for matrix
\usepgfplotslibrary{polar} %% for polar plots
\usepgfplotslibrary{fillbetween} %% to shade area between curves in TikZ



%\usepackage[width=4.375in, height=7.0in, top=1.0in, papersize={5.5in,8.5in}]{geometry}
%\usepackage[pdftex]{graphicx}
%\usepackage{tipa}
%\usepackage{txfonts}
%\usepackage{textcomp}
%\usepackage{amsthm}
%\usepackage{xy}
%\usepackage{fancyhdr}
%\usepackage{xcolor}
%\usepackage{mathtools} %% for pretty underbrace % Breaks Ximera
%\usepackage{multicol}



\newcommand{\RR}{\mathbb R}
\newcommand{\R}{\mathbb R}
\newcommand{\C}{\mathbb C}
\newcommand{\N}{\mathbb N}
\newcommand{\Z}{\mathbb Z}
\newcommand{\dis}{\displaystyle}
%\renewcommand{\d}{\,d\!}
\renewcommand{\d}{\mathop{}\!d}
\newcommand{\dd}[2][]{\frac{\d #1}{\d #2}}
\newcommand{\pp}[2][]{\frac{\partial #1}{\partial #2}}
\renewcommand{\l}{\ell}
\newcommand{\ddx}{\frac{d}{\d x}}

\newcommand{\zeroOverZero}{\ensuremath{\boldsymbol{\tfrac{0}{0}}}}
\newcommand{\inftyOverInfty}{\ensuremath{\boldsymbol{\tfrac{\infty}{\infty}}}}
\newcommand{\zeroOverInfty}{\ensuremath{\boldsymbol{\tfrac{0}{\infty}}}}
\newcommand{\zeroTimesInfty}{\ensuremath{\small\boldsymbol{0\cdot \infty}}}
\newcommand{\inftyMinusInfty}{\ensuremath{\small\boldsymbol{\infty - \infty}}}
\newcommand{\oneToInfty}{\ensuremath{\boldsymbol{1^\infty}}}
\newcommand{\zeroToZero}{\ensuremath{\boldsymbol{0^0}}}
\newcommand{\inftyToZero}{\ensuremath{\boldsymbol{\infty^0}}}


\newcommand{\numOverZero}{\ensuremath{\boldsymbol{\tfrac{\#}{0}}}}
\newcommand{\dfn}{\textbf}
%\newcommand{\unit}{\,\mathrm}
\newcommand{\unit}{\mathop{}\!\mathrm}
%\newcommand{\eval}[1]{\bigg[ #1 \bigg]}
\newcommand{\eval}[1]{ #1 \bigg|}
\newcommand{\seq}[1]{\left( #1 \right)}
\renewcommand{\epsilon}{\varepsilon}
\renewcommand{\iff}{\Leftrightarrow}

\DeclareMathOperator{\arccot}{arccot}
\DeclareMathOperator{\arcsec}{arcsec}
\DeclareMathOperator{\arccsc}{arccsc}
\DeclareMathOperator{\si}{Si}
\DeclareMathOperator{\proj}{proj}
\DeclareMathOperator{\scal}{scal}
\DeclareMathOperator{\cis}{cis}
\DeclareMathOperator{\Arg}{Arg}
%\DeclareMathOperator{\arg}{arg}
\DeclareMathOperator{\Rep}{Re}
\DeclareMathOperator{\Imp}{Im}
\DeclareMathOperator{\sech}{sech}
\DeclareMathOperator{\csch}{csch}
\DeclareMathOperator{\Log}{Log}

\newcommand{\tightoverset}[2]{% for arrow vec
  \mathop{#2}\limits^{\vbox to -.5ex{\kern-0.75ex\hbox{$#1$}\vss}}}
\newcommand{\arrowvec}{\overrightarrow}
\renewcommand{\vec}{\mathbf}
\newcommand{\veci}{{\boldsymbol{\hat{\imath}}}}
\newcommand{\vecj}{{\boldsymbol{\hat{\jmath}}}}
\newcommand{\veck}{{\boldsymbol{\hat{k}}}}
\newcommand{\vecl}{\boldsymbol{\l}}
\newcommand{\utan}{\vec{\hat{t}}}
\newcommand{\unormal}{\vec{\hat{n}}}
\newcommand{\ubinormal}{\vec{\hat{b}}}

\newcommand{\dotp}{\bullet}
\newcommand{\cross}{\boldsymbol\times}
\newcommand{\grad}{\boldsymbol\nabla}
\newcommand{\divergence}{\grad\dotp}
\newcommand{\curl}{\grad\cross}
%% Simple horiz vectors
\renewcommand{\vector}[1]{\left\langle #1\right\rangle}


\outcome{Compute partial derivatives using the chain rule.}

\title{3.6 Chain Rule}



\begin{document}

\begin{abstract}
In this section we compute partial derivatives using the chain rule.
\end{abstract}

\maketitle

For differentiable functions of one variable, the chain rule states that if $y = f(g(x))$, then 
\[
y' = \frac{d}{dx} f(g(x)) = f'(g(x)) \cdot g'(x)
\]
To extend this to the multi-variable setting, we will need to express this rule in Leibniz notation.
If instead of writing $y = f(g(x))$ we write that $y = f(u)$ where $ u = g(x)$.
Then the chain rule becomes
\[
\frac{dy}{dx} = \frac{dy}{du} \cdot \frac{du}{dx}
\]

Now, we move to the multi-variable case. If $z = f(x,y)$ where $x$ and $y$ are functions of $t$, then ultimately, $z$ is a function of $t$.
A change in $t$ causes a change in both $x$ and $y$ which in turn causes a change in $z$ leading to the following version of the chain rule:
\[
\frac{dz}{dt} = \frac{\partial z}{\partial x} \cdot \frac{dx}{dt} + \frac{\partial z}{\partial y} \cdot \frac{dy}{dt}
\]

\begin{example}[Example 1]
Let $z = x^2 + y^2 + 3xy$ and let $x = 4\cos 5t$ and $y = 4\sin 5t$. Find $\frac{dz}{dt}$.\\
From the chain rule, we have
\begin{align*}
\frac{dz}{dt} &= \frac{\partial z}{\partial x} \cdot \frac{dx}{dt} + \frac{\partial z}{\partial y} \cdot \frac{dy}{dt}\\
              &= \left(2x + 3y\right) \cdot \left(-20 \sin 5t\right) + \left(2y + 3x\right) \cdot \left(20 \cos 5t\right)\\
              &= -20(8 \cos 5t + 12 \sin 5t) \cdot \sin 5t  + 20 (8 \sin 5t + 12 \cos 5t) \cdot \cos 5t \\
              &= -160 \cos 5t \sin 5t -240 \sin^2 5t + 160 \sin 5t \cos 5t + 240 \cos^2 5t\\
              &= 240 (\cos^2 5t - \sin^2 5t)\\
              &= 240 \cos 10t
\end{align*}
Note, that it may have been easier to find $\frac{dz}{dt}$ by writing $z$ as a function of $t$ at the outset:
\begin{align*}
z &= x^2 + y^2 + 3xy \\
   &= 16 \cos^2 5t + 16 \sin^2 5t + 48 \cos 5t \sin 5t\\
   & = 16 + 24 \sin 10t
\end{align*}
Now, the ordinary (single variable) chain rule gives
\[
\frac{dz}{dt} = 240 \cos 10t
\]
which agrees with our previous answer.
\end{example}

\begin{problem}(Problem 1a)
Let $z = x^2 + y^2 + 4xy$ and let $x = \cos t$ and $y = \sin t$. Use the chain rule to find $\frac{dz}{dt}$
\end{problem}


\begin{problem}(Problem 1b)
Let $z = x^3 y^3 + 3x - 3y$ and let $x = 2\cos \pi t$ and $y = 2\sin \pi t$. Find $\frac{dz}{dt}\bigg|_{t = 0}$
\end{problem}





We next look an an extension of the previous chain rule to the case  where $x$ and $y$ are themselves functions of two variables.
Let $z$ be a function of $x$ and $y$ and let $x$ and $y$ both be functions of $s$ and $t$.  Then $z$ is ultimately a function of $s$ and $t$ and the chain rule says
\[
\frac{\partial z}{\partial s} = \frac{\partial z}{\partial x} \cdot \frac{\partial x}{\partial s} + \frac{\partial z}{\partial y} \cdot \frac{\partial y}{\partial s}
\]
\begin{center} 
and 
\end{center}
\[
\frac{\partial z}{\partial t} = \frac{\partial z}{\partial x} \cdot \frac{\partial x}{\partial t} + \frac{\partial z}{\partial y} \cdot \frac{\partial y}{\partial t} 
\]

\begin{example}[Example 2]
Let $z = x^3 + xy^4, x = se^t$ and $y = \ln(st)$. Find $\frac{\partial z}{\partial s}\bigg|_{s = 1, t = 2}$ and $\frac{\partial z}{\partial t}\bigg|_{s = 1, t = 2}$.\\



We have:
\begin{align*}
\frac{\partial z}{\partial s} &= \frac{\partial z}{\partial x} \cdot \frac{\partial x}{\partial s} + \frac{\partial z}{\partial y} \cdot \frac{\partial y}{\partial s}\\
                              &= (3x^2 + y^4)e^t + 4xy^3\cdot\frac{1}{st} \cdot t\\
                              &= (3x^2 + y^4)e^t + \frac{4xy^3}{s}
\end{align*}
When $s = 1$ and $t = 2$, we have $x = e^2$ and $y = \ln(2)$, hence
\[
\frac{\partial z}{\partial s}\bigg|_{s = 1, t = 2} = (3e^4 + \ln^4(2))e^2 + 4e^2 \ln^3(2)
\]

As for the other partial derivative, we have:
\begin{align*}
\frac{\partial z}{\partial t} &= \frac{\partial z}{\partial x} \cdot \frac{\partial x}{\partial t} + \frac{\partial z}{\partial y} \cdot \frac{\partial y}{\partial t}\\
                              &= (3x^2 + y^4)(se^t) + 4xy^3\cdot\frac{1}{st} \cdot s\\
                              &= se^t(3x^2 + y^4) + \frac{4xy^3}{t}
\end{align*}
and
\[
\frac{\partial z}{\partial t}\bigg|_{s = 1, t = 2} = e^2(3e^4 + \ln^4(2)) + 2e^2 \ln^3(2)
\]

\end{example}


\begin{problem}(Problem 2)
Let $z = e^x \sin y$ where $x = s^2 - t^2 - s - t$ and $y = \pi st$. 
Find $\frac{\partial z}{\partial s}\bigg|_{s = 2, t = 1}$ and $\frac{\partial z}{\partial t}\bigg|_{s = 2, t = 1}$
\end{problem}

\subsection{Polar Coordinates}
The polar coordinates $(r, \theta)$ of a point in the $xy$-plane give the distance from the origin and the angle from the positive $x$-axis. 
The relationship to standard rectangular coordinates $(x,y)$ is given by
\[
x = r \cos \theta \quad \text{and} \quad y = r \sin \theta
\]

If $z = f(x,y)$, then we can rewrite $z$ using polar coordinates as $z = f(r\cos \theta, r \sin \theta)$ and we can use the chain rule to 
compute the partial derivatives of $z$ with respect to $r$ and $\theta$:
\begin{align*}
\frac{\partial z}{\partial r} &= \frac{\partial z}{\partial x} \cdot \frac{\partial x}{\partial r} + \frac{\partial z}{\partial y} \cdot \frac{\partial y}{\partial r}\\
                              &= \cos \theta \frac{\partial z}{\partial x} + \sin \theta \frac{\partial z}{\partial y}
\end{align*}
and
\begin{align*}
\frac{\partial z}{\partial \theta} &= \frac{\partial z}{\partial x} \cdot \frac{\partial x}{\partial \theta} 
                                      + \frac{\partial z}{\partial y} \cdot \frac{\partial y}{\partial \theta}\\
                              &= -r\sin \theta \frac{\partial z}{\partial x} + r \cos \theta \frac{\partial z}{\partial y}
\end{align*}


\begin{example}[Example 3]
Let $z = \ln(x^2 + y^2)$ where $x = r\cos \theta$ and $y = r\sin \theta$.  Find $\frac{\partial z}{\partial r}$ and $\frac{\partial z}{\partial \theta}$.\\
Working directly with the chain rule, we have
\[
\frac{\partial z}{\partial r} = \frac{2x}{x^2+y^2} \cos \theta + \frac{2y}{x^2+y^2} \sin \theta
\]
and
\[
\frac{\partial z}{\partial \theta} = \frac{2x}{x^2+y^2} (-r\sin \theta) + \frac{2y}{x^2+y^2} (r\cos \theta)
\]
Using the relations $x = r\cos \theta$ and $y = r\sin \theta$, we can simplify each of these derivatives.
However, we can also use these relations before applying the chain rule to write $z$ in terms of $r$ and $\theta$.
Noting that
\[
x^2 + y^2 = r^2 \cos^2 \theta + r^2 \sin^2 \theta = r^2
\]
we have $z = \ln(r^2) = 2\ln r$ and hence
\[
\frac{\partial z}{\partial r} = \frac{2}{r} \quad \text{and} \quad \frac{\partial z}{\partial \theta} = 0
\]
\end{example}


\begin{problem}(Problem 3a)
Let $z = e^-(x^2 + y^2)$ where $x = r\cos \theta$ and $y = r\sin \theta$.  
Find $\frac{\partial z}{\partial r}$ and $\frac{\partial z}{\partial \theta}$ using the chain rule, 
and then again, by first writing $z$ as a function of $r$ and $\theta$ directly. Compare your answers.
\end{problem}

\begin{problem}(Problem 3b)
Let $z = \tan^{-1}\left(\frac{y}{x}\right)$ where $x = r\cos \theta$ and $y = r\sin \theta$.  
Find $\frac{\partial z}{\partial r}$ and $\frac{\partial z}{\partial \theta}$ using the chain rule, and then again, 
by first writing $z$ as a function of $r$ and $\theta$ directly.
Compare your answers.
\end{problem}


\begin{example}[Example 4]
If $z = f(x,y)$ where $x = r\cos \theta$ and $y = r \sin \theta$, find $z_{rr}$.\\
We have already established that
\[
z_r = \cos \theta z_x + \sin \theta z_y
\]
where $z_x$ and $z_y$ are each functions of both $x$ and $y$. Hence,
\begin{align*}
z_{rr} &= \cos \theta \frac{\partial}{\partial r} z_x + \sin \theta \frac{\partial}{\partial r} z_y\\
       &= \cos \theta \left[z_{xx}\cos \theta + z_{xy} \sin \theta \right] + 
       \sin \theta \left[ z_{yx} \cos \theta + z_{yy} \sin \theta  \right]\\
       &= \cos^2 \theta z_{xx} + 2\cos \theta \sin \theta z_{xy} + \sin^2 \theta z_{yy}
\end{align*}
\end{example}

\begin{problem}(Problem 4) 
If $z = f(x,y)$ where $x = r\cos \theta$ and $y = r \sin \theta$, find $z_{\theta \theta}$.
\end{problem}
       
\subsection{General Chain Rule}
Suppose $z = f(x_1, x_2, ..., x_n)$ and each of the variables $x_i$ for $i = 1, 2, ..., n$ is a function of the variables
$t_1, t_2,..., t_m$. Then for $1 \leq j \leq m$,
\begin{align*}
\frac{\partial z}{\partial t_j} &= \frac{\partial z}{\partial x_1}\frac{\partial x_1}{\partial t_j} 
                     + \frac{\partial z}{\partial x_2}\frac{\partial x_2}{\partial t_j} + \cdots + \frac{\partial z}{\partial x_n}\frac{\partial x_n}{\partial t_j}\\
                               &= \sum_{i = 1}^n \frac{\partial z}{\partial x_i}\frac{\partial x_i}{\partial t_j}
\end{align*}

\begin{example}[Example 5]
Suppose $w = f(x, y, z)$ where $x, y$ and $z$ are functions of $r, s$ and $t$. Find an expression for $\frac{\partial w}{\partial t}$.\\
Using the general form of the chain rule, we have
\[
\frac{\partial w}{\partial t}= \frac{\partial w}{\partial x}\frac{\partial x}{\partial t}+\frac{\partial w}{\partial y}\frac{\partial y}{\partial t}
+\frac{\partial w}{\partial z}\frac{\partial z}{\partial t}
\]
\end{example}

\begin{problem}(Problem 5a)
Suppose $w = f(x, y, z)$ where $x, y$ and $z$ are functions of $r, s$ and $t$. Find an expression for $\frac{\partial w}{\partial s}$.
\end{problem}

\begin{problem}(Problem 5b)
Suppose $v = f(w, x, y, z)$ where $x, y$ and $z$ are functions of $s$ and $t$. Find an expression for $\frac{\partial v}{\partial t}$.
\end{problem}


\end{document}
