\documentclass[handout]{ximera}

%% You can put user macros here
%% However, you cannot make new environments



\newcommand{\ffrac}[2]{\frac{\text{\footnotesize $#1$}}{\text{\footnotesize $#2$}}}
\newcommand{\vasymptote}[2][]{
    \draw [densely dashed,#1] ({rel axis cs:0,0} -| {axis cs:#2,0}) -- ({rel axis cs:0,1} -| {axis cs:#2,0});
}


%\usepackage{tcolorbox} %%Needed for Derivative Definition supposedly and product rule, natural exp log, quotient rule, inverse trig, rates of change


% \graphicspath{{./}{firstExample/}}
% \usepackage{forest}
\usepackage{amsmath}
\usepackage{amssymb}
\usepackage{array}
\usepackage[makeroom]{cancel} %% for strike outs
\usepackage{pgffor} %% required for integral for loops
\usepackage{tikz}
\usepackage{tikz-cd}
\usepackage{tkz-euclide}
\usetikzlibrary{shapes.multipart}


% \usetkzobj{all}
\tikzstyle geometryDiagrams=[ultra thick,color=blue!50!black]


\usetikzlibrary{arrows}
\tikzset{>=stealth,commutative diagrams/.cd,
  arrow style=tikz,diagrams={>=stealth}} %% cool arrow head
\tikzset{shorten <>/.style={ shorten >=#1, shorten <=#1 } } %% allows shorter vectors

\usetikzlibrary{backgrounds} %% for boxes around graphs
\usetikzlibrary{shapes,positioning}  %% Clouds and stars
\usetikzlibrary{matrix} %% for matrix
\usepgfplotslibrary{polar} %% for polar plots
\usepgfplotslibrary{fillbetween} %% to shade area between curves in TikZ



%\usepackage[width=4.375in, height=7.0in, top=1.0in, papersize={5.5in,8.5in}]{geometry}
%\usepackage[pdftex]{graphicx}
%\usepackage{tipa}
%\usepackage{txfonts}
%\usepackage{textcomp}
%\usepackage{amsthm}
%\usepackage{xy}
%\usepackage{fancyhdr}
%\usepackage{xcolor}
%\usepackage{mathtools} %% for pretty underbrace % Breaks Ximera
%\usepackage{multicol}



\newcommand{\RR}{\mathbb R}
\newcommand{\R}{\mathbb R}
\newcommand{\C}{\mathbb C}
\newcommand{\N}{\mathbb N}
\newcommand{\Z}{\mathbb Z}
\newcommand{\dis}{\displaystyle}
%\renewcommand{\d}{\,d\!}
\renewcommand{\d}{\mathop{}\!d}
\newcommand{\dd}[2][]{\frac{\d #1}{\d #2}}
\newcommand{\pp}[2][]{\frac{\partial #1}{\partial #2}}
\renewcommand{\l}{\ell}
\newcommand{\ddx}{\frac{d}{\d x}}
\newcommand{\ppx}{\frac{\partial}{\partial x}}
\newcommand{\ppy}{\frac{\partial}{\partial y}}

\newcommand{\zeroOverZero}{\ensuremath{\boldsymbol{\tfrac{0}{0}}}}
\newcommand{\inftyOverInfty}{\ensuremath{\boldsymbol{\tfrac{\infty}{\infty}}}}
\newcommand{\zeroOverInfty}{\ensuremath{\boldsymbol{\tfrac{0}{\infty}}}}
\newcommand{\zeroTimesInfty}{\ensuremath{\small\boldsymbol{0\cdot \infty}}}
\newcommand{\inftyMinusInfty}{\ensuremath{\small\boldsymbol{\infty - \infty}}}
\newcommand{\oneToInfty}{\ensuremath{\boldsymbol{1^\infty}}}
\newcommand{\zeroToZero}{\ensuremath{\boldsymbol{0^0}}}
\newcommand{\inftyToZero}{\ensuremath{\boldsymbol{\infty^0}}}


\newcommand{\numOverZero}{\ensuremath{\boldsymbol{\tfrac{\#}{0}}}}
\newcommand{\dfn}{\textbf}
%\newcommand{\unit}{\,\mathrm}
\newcommand{\unit}{\mathop{}\!\mathrm}
%\newcommand{\eval}[1]{\bigg[ #1 \bigg]}
\newcommand{\eval}[1]{ #1 \bigg|}
\newcommand{\seq}[1]{\left( #1 \right)}
\renewcommand{\epsilon}{\varepsilon}
\renewcommand{\iff}{\Leftrightarrow}

\DeclareMathOperator{\arccot}{arccot}
\DeclareMathOperator{\arcsec}{arcsec}
\DeclareMathOperator{\arccsc}{arccsc}
\DeclareMathOperator{\si}{Si}
\DeclareMathOperator{\proj}{proj}
\DeclareMathOperator{\scal}{scal}
\DeclareMathOperator{\cis}{cis}
\DeclareMathOperator{\Arg}{Arg}
%\DeclareMathOperator{\arg}{arg}
\DeclareMathOperator{\Rep}{Re}
\DeclareMathOperator{\Imp}{Im}
\DeclareMathOperator{\sech}{sech}
\DeclareMathOperator{\csch}{csch}
\DeclareMathOperator{\Log}{Log}

\newcommand{\tightoverset}[2]{% for arrow vec
  \mathop{#2}\limits^{\vbox to -.5ex{\kern-0.75ex\hbox{$#1$}\vss}}}
\newcommand{\arrowvec}{\overrightarrow}
\renewcommand{\vec}{\mathbf}
\newcommand{\veci}{{\boldsymbol{\hat{\imath}}}}
\newcommand{\vecj}{{\boldsymbol{\hat{\jmath}}}}
\newcommand{\veck}{{\boldsymbol{\hat{k}}}}
\newcommand{\vecl}{\boldsymbol{\l}}
\newcommand{\utan}{\vec{\hat{t}}}
\newcommand{\unormal}{\vec{\hat{n}}}
\newcommand{\ubinormal}{\vec{\hat{b}}}

\newcommand{\dotp}{\bullet}
\newcommand{\cross}{\boldsymbol\times}
\newcommand{\grad}{\boldsymbol\nabla}
\newcommand{\divergence}{\grad\dotp}
\newcommand{\curl}{\grad\cross}
%% Simple horiz vectors
\renewcommand{\vector}[1]{\left\langle #1\right\rangle}


\outcome{Compute curvature.}

\title{2.5 Curvature}



\begin{document}

\begin{abstract}
In this section we compute the curvature of a space curve.
\end{abstract}

\maketitle

Curvature is a scalar quantity that tells us how sharply a curve is bending.
To measure curvature, we need to take into account the change in the tangent vector as we move along the curve.
The tangent vector will change in two regards. First, we can expect its magnitude to change, 
we would not expect this to be related to curvature.
We can also expect a change in the direction of the tangent vector and it is this 
that should account for how sharply the curve is bending.
Recall that the tangent vector to a space curve $\vec r(t)$ is given by $\vec\,r'(t)$.
When measuring curvature, we are interested in the rate of change of the tangent vector, but we do not want to take into account any change in magnitude.
To eliminate change in magnitude from the situation, we consider the arc length parameterization of the curve.
When a curve is parameterized using the arc length parameter, the tangent vector is always a unit vector.
Now suppose $\vec r(s)$ represents the arc length parameterization of a space curve. Then its tangent vector is given by $\vec r\,'(s) = \vec T(s)$.
Curvature is determined by the rate of change of this vector, which can be expressed as a derivative: $\frac{d}{ds} \vec T(s)$.  
However, we wish curvature to be a scalar quantity, so we define it to be the magnitude of this vector.
\begin{definition}[Curvature]
Let $\vec r(s) = \vector{x(s), y(s), z(s)}$ where $s$ is the arc length parameter. Then the curvature of $\vec r(s)$ is given by 
\[
\kappa = \left|\frac{d\vec T}{ds}\right|
\]
where $\vec T(s)$ is the unit tangent vector, $\vec T(s) = \vec r\,'(s)$ 
\end{definition}


\begin{example}[example 1]
Find the curvature of the line $\vec r(s) = \vector{2 + \frac{s}{\sqrt 2}, 3 + \frac{s}{\sqrt 3}, 3+ \frac{s}{\sqrt 6}}$\\
Note that since $\vec r\,'(s) = \vector{\frac{1}{\sqrt 2}, \frac{1}{\sqrt 3}, \frac{1}{\sqrt 6}}$ is a unit vector, $s$ is indeed an arc length parameter.
The curvature is then
\[
\kappa = \frac{d}{ds} \vec T(s) = \frac{d}{ds} \vec r\,'(s) = \frac{d}{ds}\vector{\frac{1}{\sqrt 2}, \frac{1}{\sqrt 3}, \frac{1}{\sqrt 6}} = 0
\]
It should come as no surprise that the curvature of a line is zero, since the tangent vector does not change direction as we move along the line.
Note that it is possible to parameterize a line in such a way that the magnitude of the tangent vector does change! 
\end{example}

\begin{problem}(problem 1a)
Let $\vector{a, b, c}$ be a unit vector.  Show that the line $\vec r(s) = \vector{x_0 + as, y_0 + bs, z_0 + cs}$ has zero curvature.
\begin{hint}
First show that $s$ is actually an arc length parameter
\end{hint}
\begin{hint}
Compute $\kappa$ as in the example above
\end{hint}
\end{problem}

\begin{problem}(Problem 1b)
Show that the space curve $\vec r(t) = \vector{t^3, t^3, t^3}$ is a \textbf{line} whose tangent vectors are not of constant magnitude.\\
\begin{hint}
Compute the magnitude and note that it is a non-constant function of the parameter $t$
\end{hint}
\begin{hint}
Show that this is the line $\vec r_1(\tau) \vector{\tau, \tau, \tau}$ by changing the parameter to $\tau$ by letting $\tau = t^3$
\end{hint}
\end{problem}

\end{document}
