\documentclass[handout]{ximera}

%% You can put user macros here
%% However, you cannot make new environments



\newcommand{\ffrac}[2]{\frac{\text{\footnotesize $#1$}}{\text{\footnotesize $#2$}}}
\newcommand{\vasymptote}[2][]{
    \draw [densely dashed,#1] ({rel axis cs:0,0} -| {axis cs:#2,0}) -- ({rel axis cs:0,1} -| {axis cs:#2,0});
}


\graphicspath{{./}{firstExample/}}
\usepackage{forest}
\usepackage{amsmath}
\usepackage{amssymb}
\usepackage{array}
\usepackage[makeroom]{cancel} %% for strike outs
\usepackage{pgffor} %% required for integral for loops
\usepackage{tikz}
\usepackage{tikz-cd}
\usepackage{tkz-euclide}
\usetikzlibrary{shapes.multipart}


%\usetkzobj{all}
\tikzstyle geometryDiagrams=[ultra thick,color=blue!50!black]


\usetikzlibrary{arrows}
\tikzset{>=stealth,commutative diagrams/.cd,
  arrow style=tikz,diagrams={>=stealth}} %% cool arrow head
\tikzset{shorten <>/.style={ shorten >=#1, shorten <=#1 } } %% allows shorter vectors

\usetikzlibrary{backgrounds} %% for boxes around graphs
\usetikzlibrary{shapes,positioning}  %% Clouds and stars
\usetikzlibrary{matrix} %% for matrix
\usepgfplotslibrary{polar} %% for polar plots
\usepgfplotslibrary{fillbetween} %% to shade area between curves in TikZ



%\usepackage[width=4.375in, height=7.0in, top=1.0in, papersize={5.5in,8.5in}]{geometry}
%\usepackage[pdftex]{graphicx}
%\usepackage{tipa}
%\usepackage{txfonts}
%\usepackage{textcomp}
%\usepackage{amsthm}
%\usepackage{xy}
%\usepackage{fancyhdr}
%\usepackage{xcolor}
%\usepackage{mathtools} %% for pretty underbrace % Breaks Ximera
%\usepackage{multicol}



\newcommand{\RR}{\mathbb R}
\newcommand{\R}{\mathbb R}
\newcommand{\C}{\mathbb C}
\newcommand{\N}{\mathbb N}
\newcommand{\Z}{\mathbb Z}
\newcommand{\dis}{\displaystyle}
%\renewcommand{\d}{\,d\!}
\renewcommand{\d}{\mathop{}\!d}
\newcommand{\dd}[2][]{\frac{\d #1}{\d #2}}
\newcommand{\pp}[2][]{\frac{\partial #1}{\partial #2}}
\renewcommand{\l}{\ell}
\newcommand{\ddx}{\frac{d}{\d x}}

\newcommand{\zeroOverZero}{\ensuremath{\boldsymbol{\tfrac{0}{0}}}}
\newcommand{\inftyOverInfty}{\ensuremath{\boldsymbol{\tfrac{\infty}{\infty}}}}
\newcommand{\zeroOverInfty}{\ensuremath{\boldsymbol{\tfrac{0}{\infty}}}}
\newcommand{\zeroTimesInfty}{\ensuremath{\small\boldsymbol{0\cdot \infty}}}
\newcommand{\inftyMinusInfty}{\ensuremath{\small\boldsymbol{\infty - \infty}}}
\newcommand{\oneToInfty}{\ensuremath{\boldsymbol{1^\infty}}}
\newcommand{\zeroToZero}{\ensuremath{\boldsymbol{0^0}}}
\newcommand{\inftyToZero}{\ensuremath{\boldsymbol{\infty^0}}}


\newcommand{\numOverZero}{\ensuremath{\boldsymbol{\tfrac{\#}{0}}}}
\newcommand{\dfn}{\textbf}
%\newcommand{\unit}{\,\mathrm}
\newcommand{\unit}{\mathop{}\!\mathrm}
%\newcommand{\eval}[1]{\bigg[ #1 \bigg]}
\newcommand{\eval}[1]{ #1 \bigg|}
\newcommand{\seq}[1]{\left( #1 \right)}
\renewcommand{\epsilon}{\varepsilon}
\renewcommand{\iff}{\Leftrightarrow}

\DeclareMathOperator{\arccot}{arccot}
\DeclareMathOperator{\arcsec}{arcsec}
\DeclareMathOperator{\arccsc}{arccsc}
\DeclareMathOperator{\si}{Si}
\DeclareMathOperator{\proj}{proj}
\DeclareMathOperator{\scal}{scal}
\DeclareMathOperator{\cis}{cis}
\DeclareMathOperator{\Arg}{Arg}
%\DeclareMathOperator{\arg}{arg}
\DeclareMathOperator{\Rep}{Re}
\DeclareMathOperator{\Imp}{Im}
\DeclareMathOperator{\sech}{sech}
\DeclareMathOperator{\csch}{csch}
\DeclareMathOperator{\Log}{Log}

\newcommand{\tightoverset}[2]{% for arrow vec
  \mathop{#2}\limits^{\vbox to -.5ex{\kern-0.75ex\hbox{$#1$}\vss}}}
\newcommand{\arrowvec}{\overrightarrow}
\renewcommand{\vec}{\mathbf}
\newcommand{\veci}{{\boldsymbol{\hat{\imath}}}}
\newcommand{\vecj}{{\boldsymbol{\hat{\jmath}}}}
\newcommand{\veck}{{\boldsymbol{\hat{k}}}}
\newcommand{\vecl}{\boldsymbol{\l}}
\newcommand{\utan}{\vec{\hat{t}}}
\newcommand{\unormal}{\vec{\hat{n}}}
\newcommand{\ubinormal}{\vec{\hat{b}}}

\newcommand{\dotp}{\bullet}
\newcommand{\cross}{\boldsymbol\times}
\newcommand{\grad}{\boldsymbol\nabla}
\newcommand{\divergence}{\grad\dotp}
\newcommand{\curl}{\grad\cross}
%% Simple horiz vectors
\renewcommand{\vector}[1]{\left\langle #1\right\rangle}


\outcome{Compute curvature.}

\title{2.5 Curvature}



\begin{document}

\begin{abstract}
In this section we compute the curvature of a space curve.
\end{abstract}

\maketitle

Curvature is a scalar quantity that tells us how sharply a curve is bending.
To measure curvature, we need to take into account the change in the tangent vector as we move along the curve.
The tangent vector will change in two regards. First, we can expect its magnitude to change, but
we would not expect this to be related to curvature. Moreover, curvature should be independent of the magnitude of the tangent vector.
We can also expect a change in the direction of the tangent vector and it is this 
that should account for how sharply the curve is bending.
Recall that the tangent vector to a space curve $\vec r(t)$ is given by $\vec\,r\,'(t)$.
When measuring curvature, we are interested in the rate of change of the tangent vector, but we do not want to take into account any change in magnitude.
To eliminate change in magnitude from the situation, we could consider the unit tangent vector, 
\[
T(t) = \frac{\vec r\,'(t) }{|\vec r\,'(t)|}
\]
However the rate of change of this vector still depends on the magnitude of $\vec r\,'$ as we can see here:

\begin{align*}
\frac{d}{dt} T(t) &= \frac{d}{dt} \left(\frac{\vec r\,'(t) }{|\vec r\,'(t)|}\right) \\
&= \frac{\vec r\,''(t) }{|\vec r\,'(t)|} + \frac{d}{dt} \left( \frac{1}{|\vec r\,'(t)|}\right)\vec r\,'(t)
\end{align*}

The way to resolve the issue is to consider the arc length parameterization of the curve.
When a curve is parameterized using the arc length parameter, the tangent vector is always a unit vector, so that
\[
T(s) = \vec r\,'(s)
\]
%Now suppose $\vec r(s)$ represents the arc length parameterization of a space curve. Then its tangent vector is given by $\vec r\,'(s) = \vec T(s)$.
Curvature is thus determined by the rate of change of this vector, which can be expressed as a derivative: $\frac{d}{ds} \vec T(s)$.  
However, we wish curvature to be a scalar quantity, so we define it to be the magnitude of this vector.
\begin{definition}[Curvature]
Let $\vec r(s) = \vector{x(s), y(s), z(s)}$ where $s$ is the arc length parameter. Then the curvature of $\vec r(s)$ is given by 
\[
\kappa = \left|\frac{d\vec T}{ds}\right|
\]
where $\vec T(s)$ is the unit tangent vector, $\vec T(s) = \vec r\,'(s)$ 
\end{definition}


\begin{example}[example 1]
Find the curvature of the line $\vec r(s) = \vector{2 + \frac{s}{\sqrt 2}, 3 + \frac{s}{\sqrt 3}, 3+ \frac{s}{\sqrt 6}}$\\
Note that since $\vec r\,'(s) = \vector{\frac{1}{\sqrt 2}, \frac{1}{\sqrt 3}, \frac{1}{\sqrt 6}}$ is a unit vector, $s$ is indeed an arc length parameter.
The curvature is then
\[
\kappa = \frac{d}{ds} \vec T(s) = \frac{d}{ds} \vec r\,'(s) = \frac{d}{ds}\vector{\frac{1}{\sqrt 2}, \frac{1}{\sqrt 3}, \frac{1}{\sqrt 6}} = 0
\]
It should come as no surprise that the curvature of a line is zero, since the tangent vector does not change direction as we move along the line.
Note that it is possible to parameterize a line in such a way that the magnitude of the tangent vector does change! 
\end{example}

\begin{problem}(problem 1a)
Let $\vector{a, b, c}$ be a unit vector.  Show that the line $\vec r(s) = \vector{x_0 + as, y_0 + bs, z_0 + cs}$ has zero curvature.
\begin{hint}
First show that $s$ is actually an arc length parameter
\end{hint}
\begin{hint}
Compute $\kappa$ as in the example above
\end{hint}
\end{problem}

\begin{problem}(Problem 1b)
Show that the space curve $\vec r(t) = \vector{t^3, t^3, t^3}$ is a \textbf{line} whose tangent vectors are not of constant magnitude.\\
\begin{hint}
Compute the magnitude and note that it is a non-constant function of the parameter $t$
\end{hint}
\begin{hint}
Show that this is the line $\vec r_1(\tau) \vector{\tau, \tau, \tau}$ by changing the parameter to $\tau$ by letting $\tau = t^3$
\end{hint}
\end{problem}

\subsection{Curvature of a Circle}
A circle of radius $R$ centered at the origin in $\R^2$ can be parameterized as
\[
\vec r(t) = \vector{R\cos t, R\sin t}, \; 0\leq t \leq 2\pi
\]
We can embed the circle in $\R^3$ by letting $z = 0$ and write instead,
\[
\vec r(t) = \vector{R\cos t, R\sin t, 0}, \; 0\leq t \leq 2\pi
\]
We now reparameterizing the circle using the arc length parameter,
\[
s = \int_0^t |\vec r\,'(t)| \, dt = \int_0^t \sqrt{ (-R\sin t)^2 + (R\cos t)^2 + 0^2} \, dt = \int_0^t R \, dt = Rt
\]
Hence 
\[
t = f^{-1}(s) = \frac{s}{R}
\]
and the circle becomes
\[
\vec r(f^{-1}(s)) = \vector{R \cos\left(\frac{s}{R}\right), R \sin\left(\frac{s}{R}\right), 0}, \; 0 \leq s \leq \frac{2\pi}{R}
\]
\begin{example}[Example 2]
Find the curvature of the circle of radius $R$ given by the arc length parameterization
\[
\vec r(s) = \vector{R \cos\left(\frac{s}{R}\right), R \sin\left(\frac{s}{R}\right), 0}
\]
Since $s$ is an arc length parameter, 
\[
T(s) = \vec r\,'(s) = \vector{-\sin\left(\frac{s}{R}\right), \cos\left(\frac{s}{R}\right), 0}
\]
and so the curvature is given by
\[
\kappa = \left|\frac{dT}{ds}\right| = \left|\vector{-\frac{1}{R}\cos\left(\frac{s}{R}\right), -\frac{1}{R}\sin\left(\frac{s}{R}\right), 0}\right| = \frac{1}{R}
\]
In conclusion, the curvature of a circle of radius R is 
\[
\kappa = \frac{1}{R}
\]
\end{example}

\begin{problem}(Problem 2)
Compute the curvature of a circle of radius 3 using the \textbf{method} of example 2.
\end{problem}

\begin{example}[Example 3]
Compute the curvature of the spiral helix $\vec r(t) = \vector{2\cos(3t), 2 \sin(3t), 5t}$.\\
The arc length parameter is given by
\[
s = \int_0^t \sqrt{\left(-6\sin(3t) \right)^2+\left(6\cos(3t) \right)^2 + 25} \, du = \int_0^t \sqrt{61} \, du = t\sqrt 61
\]
Reparameterizing the helix by arc length, we have
\[
\vec r(s) = \vector{2\cos\left(\frac{3s}{\sqrt{61}}\right), 2 \sin\left(\frac{3s}{\sqrt{61}}\right), \frac{5s}{\sqrt{61}}}
\]
Then
\[
T(s) = \vec r\,'(s) = \vector{-\frac{6}{\sqrt{61}}\sin\left(\frac{3s}{\sqrt{61}}\right), \frac{6}{\sqrt{61}}\cos\left(\frac{3s}{\sqrt{61}}\right), \frac{5}{\sqrt{61}}}
\]
The curvature of the spiral helix is given by
\[
\kappa = \left| \frac{dT}{ds} \right| = \left|\vector{-\frac{18}{61}\cos\left(\frac{3s}{\sqrt{61}}\right), -\frac{18}{61}\sin\left(\frac{3s}{\sqrt{61}}\right), 0}\right| = \frac{18}{61}
\]
\end{example}


\begin{problem}(Problem 3)
Compute the curvature of spiral helix $\vec r(t) = \vector{4t, 5 \sin(2t),5\cos(2t)}$ using the method of example 3.\\
The arc length parameter is given by $s = \answer{t\sqrt{116}}$\\
The unit tangent vector is given by $T(s) = \vector{\answer{4/\sqrt{116}}, \answer{5/\sqrt{29} \cos(s/\sqrt{29})}, \answer{-5/\sqrt{29} \sin(s/\sqrt{29})}}$\\
The curvature is $\kappa = \answer{5/29}$
\end{problem}

Recall that for a space curve $\vec r(t)$, the tangent vector is given by $\vec r\,'(t)$ and the unit tangent vector is given by
\[
\vec T(t) = \frac{\vec r\,'(t)}{|\vec r\,'(t)|}
\]

\begin{proposition}[Second Form of Curvature]
Let $\vec r(t)$ be a smooth space curve, i.e., $\vec r\,'(t) \neq \vec 0$. Then curvature of $\vec r(t)$ can be computed as follows:
\[
\kappa = \frac{|\vec T\,'(t)|}{|\vec r\,'(t)|}
\]
\end{proposition}
\begin{proof}
Let $s = f(t)$ be the arc length parameter for $\vec r(t)$. By the chain rule
\[
\kappa = \left|\frac{dT}{ds}\right| = \left|\frac{dT}{dt} \cdot \frac{dt}{ds}\right| = \left|T\,'(t) \frac{1}{ds/dt}\right| = \frac{|T\,'(t)|}{|\vec r\,'(t)|}
\]
Here, we have used the fact that since
\[
s = f(t) = \int_0^t |\vec r\,'(u)| \, du
\]
we have
\[
\frac{ds}{dt} = |\vec r\,'(t)|
\]
\end{proof}


\begin{example}[Example 4]
Use the second form of curvature to compute $\kappa$ for the space curve $\vec r(t) = \vector{e^t, t\sqrt 2 , e^{-t}}$.\\
We begin with
\[
|\vec r\,'(t)| = \left|\vector{e^t, \sqrt 2 , -e^{-t}}\right| = \sqrt{e^{2t} + 2 + e^{-2t}} = e^t + e^{-t}
\]
Next, we compute the unit tangent vector:
\begin{align*}
\vec T(t) &= \frac{\vec r\,'(t)}{|\vec r\,'(t)|}\\
             &= \vector{\frac{e^t}{e^t + e^{-t}}, \frac{\sqrt 2}{e^t + e^{-t}} , \frac{e^{-t}}{e^t + e^{-t}}}\\
             &= \vector{\frac{e^{2t}}{e^{2t} + 1}, \frac{e^t\sqrt 2}{e^{2t} + 1} , \frac{1}{e^{2t} + 1}}\\
\end{align*}
We use the quotient rule on each component to calculate the derivative of the unit tangent vector. The reader should verify that
\[
\vec T\,'(t) = \frac{1}{\left(e^{2t} +1\right)^2} \vector{2e^{2t}, e^t\sqrt 2 (1 - e^{2t}), -2e^{2t}} = \frac{e^t}{\left(e^{2t} +1\right)^2} \vector{2e^{t}, \sqrt 2 (1 - e^{2t}), -2e^{t}}
\]
Next, the magnitude of the derivative of the unit tangent vector is
\begin{align*}
\left|\vec T\,'(t)\right| &= \frac{e^t}{\left(e^{2t} +1\right)^2} \sqrt{4e^{2t} + 2 (1 - e^{2t})^2 + 4e^{2t}}\\
          &= \frac{e^t}{\left(e^{2t} +1\right)^2} \sqrt{8e^{2t} + 2 (1 -2e^{2t}+ e^{4t})}\\
        &= \frac{e^t \sqrt 2}{\left(e^{2t} +1\right)^2} \sqrt{4e^{2t} +  (1 -2e^{2t}+ e^{4t})}\\
          &= \frac{e^t \sqrt 2}{\left(e^{2t} +1\right)^2} \sqrt{1 +2e^{2t}+ e^{4t}}\\
          &= \frac{e^t \sqrt 2}{\left(e^{2t} +1\right)^2} \sqrt{(1 +e^{2t})^2}\\
            &= \frac{e^t \sqrt 2}{e^{2t} +1}
\end{align*}
Finally, we obtain the curvature $\kappa$ by dividing the magnitude of the the derivative of the unit tangent vector by the magnitude of the tangent vector:
\begin{align*}
\kappa &= \frac{|\vec T\,'(t)|}{|\vec r\,'(t)|}\\
       &= \frac{e^t \sqrt 2}{e^{2t} +1} \cdot \frac{1}{e^t + e^{-t}}\\
      &= \frac{e^t \sqrt 2}{e^{2t} +1} \cdot \frac{e^t}{e^{2t} + 1}\\
     &= \frac{e^{2t} \sqrt 2}{\left(e^{2t} +1\right)^2}
\end{align*}
\end{example}

\begin{problem}(Problem 4)
Use the second form of curvature to compute $\kappa$ for the space curve $\vec r(t) = \vector{\frac32 t^2, t\sqrt 6 , \ln(2t)}$.


\begin{hint}
\[
\left|\vec r\,'(t) \right| = 3t + \frac{1}{t}  \quad \text{and} \quad \frac{1}{\left|\vec r\,'(t) \right|} = \frac{t}{3t^2 + 1}
\]
\end{hint}

\begin{hint}
\[
\vec T(t) = \vector{\frac{3t^2}{3t^2 +1},\frac{t\sqrt 6}{3t^2 +1} , \frac{1}{3t^2 +1}}
\]
\end{hint}

\begin{hint}
\[
\vec T\,'(t) = \frac{1}{\left(3t^2 +1\right)^2} \vector{6t, \sqrt 6 (1 - 3t^2), -6t}
\]
\end{hint}

\begin{hint}
\[
\kappa = \frac{t\sqrt 6}{\left(3t^2 +1\right)^2}
\]
\end{hint}

\end{problem}


\begin{proposition}[Third Form of Curvature]
Let $\vec r(t)$ be a smooth space curve, i.e., $\vec r\,'(t) \neq \vec 0$. Then curvature of $\vec r(t)$ can be computed as follows:
\[
\kappa = \frac{|\vec r\,'(t) \cross \vec r\,''(t)|}{|\vec r\,'(t)|^3}
\]
\end{proposition}


\begin{proof}
Since
\[
\vec T = \frac{\vec r\,'}{|\vec r\, '|} = \frac{\vec r\,'}{|ds/dt|}
\]
we have
\[
\vec r\,' = \left(\frac{ds}{dt}\right) \vec T
\]
Now, by the product rule,
\[
 \vec r\,''(t) = \left(\frac{d^2s}{dt^2}\right) \vec T + \left(\frac{ds}{dt}\right) \vec T\,'
\]
Noting that the cross product of a vector with itself is the zero vector, we have
\[
\vec r\,' \cross \vec r\,'' = \left(\frac{ds}{dt}\right)^2 \vec T \cross \vec T\,'
\]
We now wish to compute the magnitude of both sides of this equation.
Recall that $\vec T$ and $\vec T\,'$ are orthogonal since $|\vec T| = 1$ so it is constant. Also recall that the 
magnitude of the cross product is given by the product of the magnitudes times the sine of the angle between them. 
Using these facts gives
\[
|\vec r\,' \cross \vec r\,''| = \left(\frac{ds}{dt}\right)^2 | \vec T|\cdot | \vec T\,'|\sin \theta =\left(\frac{ds}{dt}\right)^2 | \vec T\,'|
\]
which means that
\[
| \vec T\,'| = \frac{|\vec r\,' \cross \vec r\,''|}{|\vec r\,'|^2}
\]
since $\frac{ds}{dt} = |\vec r\,'|$.
Finally,
\[
\kappa = \frac{| \vec T\,'|}{|\vec r\,'|} = \frac{|\vec r\,' \cross \vec r\,''|}{|\vec r\,'|^3}
\]
\end{proof}


\begin{example}[Example 5]
Use the third form of curvature to compute the curvature of the twisted cubic $\vec r(t) = \vector{t, t^2, t^3}$.\\
The derivatives are
\[
\vec r\,'(t) = \vector{1, 2t, 3t^2} \quad \text{and} \quad \vec r\,''(t) = \vector{0, 2, 6t}
\]
Now the cross product is
 \begin{align*}
 \vec  r\,'(t) \cross \vec r\,''(t) &= 
 \begin{vmatrix}
\vec{i} & \vec{j} & \vec{k}\\
1 & 2t & 3t^2\\
0 & 2 & 6t
\end{vmatrix}\\
&= (12t^2 - 6t^2) \vec{i} - (6t - 0) \vec{j} + (2-0) \vec{k}\\
 &=6t^2 \vec{i} -6t \vec{j} + 2 \vec{k}\\
 &=\vector{6t^2 - 6t +2}
 \end{align*}
Finally, according to the third form of curvature, we have
\begin{align*}
\kappa &=  \frac{|\vec r\,'(t) \cross \vec r\,''(t)|}{|\vec r\,'(t)|^3}\\
       &=  \frac{|\vector{6t^2,  -6t, 2}|}{|\vector{1, 2t, 3t^2}|^3}\\
       &= \frac{\sqrt{36t^4 + 36t^2 + 4}}{(1 + 4t^2 + 9t^4)^{3/2}}\\
       &= \frac{2\sqrt{1 + 9t^2 + 9t^4}}{(1 + 4t^2 + 9t^4)^{3/2}}
\end{align*}
\end{example}

\begin{problem}(Problem 5a)
Use the third form of curvature to find $\kappa$ if 
\[
\vec r(t) = \vector{t, 2t, 3+t^2}
\]

\begin{multipleChoice}
\choice{$\displaystyle \frac{\sqrt 5}{\sqrt{4t^2 + 6}}$}
\choice{$\displaystyle \frac{\sqrt {10}}{\sqrt{4t^2 + 6}}$}
\choice{$\displaystyle \frac{\sqrt 5}{\sqrt{2t^2 + 3}}$}
\choice[correct]{$\displaystyle \frac{\sqrt {10}}{\sqrt{2t^2 + 3}}$}
\end{multipleChoice}

\end{problem}

\begin{problem}(Problem 5b)
A cycloid is a plane curve that traces the path of a point on a rolling circle.
Use the third form of curvature to find $\kappa$ for the cycloid 
\[
\vec r(t) = \vector{t - \sin t, 1 - \cos t, 0}
\]

\begin{multipleChoice}
\choice{$\displaystyle \frac{\sqrt 2}{2 \sqrt{1 - \cos t}}$}
\choice{$\displaystyle \frac{1}{4 \sqrt{1 - \cos t}}$}
\choice{$\displaystyle \frac{\sqrt 2}{\sqrt{1 - \cos t}}$}
\choice[correct]{$\displaystyle \frac{\sqrt 2}{4 \sqrt{1 - \cos t}}$}
\end{multipleChoice}

\end{problem}


\begin{proposition}[Fourth Form of Curvature]
Suppose $f(x)$ is a twice differentiable function. Then the curvature of the graph of $y = f(x)$ is given by
\[
\kappa = \frac{|f\,''(x)|}{\left[1+ f\,'(x)^2\right]^{3/2}}
\]
\end{proposition}

\begin{proof}
The graph of $y = f(x)$ can be expressed as a space curve with parameter $x$ as
\[
\vec r(x) = \vector{x, f(x), 0}
\]
We calculate $\kappa$ using the third form of curvature. We have
\[
\vec r\,'(x) = \vector{1, f\,'(x), 0} \quad \text{and} \quad \vec r\,''(x) = \vector{0, f\,''(x), 0} 
\]
Verify that $\vec r\,'(x) \cross \vec r\,''(x) = \vector{0,0,f\,''(x)}$.
Finally, the third form of curvature gives
\[
\kappa = \frac{|\vec r\,'(t) \cross \vec r\,''(t)|}{|\vec r\,'(t)|^3} = \frac{|f\,''(x)|}{\left[1+ f\,'(x)^2\right]^{3/2}}
\]
\end{proof}

\begin{example}[Example 6]
Use the fourth form of curvature to find the formula for the curvature of the graph of $y = e^x$.  At what point is the curvature a maximum?\\
Plugging $f\,'(x) = e^x$ and $f\,''(x) = e^x$ into the fourth form of curvature gives
\[
\kappa = \frac{e^x}{(1+e^{2x})^{3/2}}
\]
Noting that $\kappa$ is a function of $x$, we can find the point of maximum curvature by solving the equation $\kappa\,' = 0$ for $x$
We have
\[
\kappa ' = \frac{e^x(1+e^{2x})^{3/2} - 3e^x(1+e^{2x})^{1/2}}{(1+e^{2x})^3} = \frac{e^x\left(e^{2x} - 2\right)}{(1+e^{2x})^{5/2}} = 0
\]
which gives
\[
 e^{2x} = 2
\]
whose solution is $ x = \frac12 \ln(2)$. We can verify that this gives a maximum by noting that $\kappa \,' (0) < 0$ and $\kappa \,' (\ln 2) > 0$. \\
Thus the point of maximum curvature is $\left(\frac12 \ln(2), \sqrt 2\right)$.
\end{example}

\begin{problem}(Problem 6a)
Use the fourth form of curvature to find the formula for the curvature of the parabola $y = x^2$.  At what point is the curvature a maximum?\\
\end{problem}

\begin{problem}(Problem 6b)
Use the fourth form of curvature to find the formula for the curvature of the curve $y = \sin(x)$.  
At what points on the interval $(0, 2\pi)$ is the curvature a maximum? a minimum?\\
\end{problem}


\begin{problem}(Problem 6c)
Let $f(x)$ be a function with continuous second derivative. Explain why the curvature of the graph of $y = f(x)$ is zero at an inflection point.\\
\begin{hint}
An inflection point is a point on the graph where the concavity changes.
\end{hint}
\end{problem}
\end{document}
