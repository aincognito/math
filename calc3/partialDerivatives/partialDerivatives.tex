\documentclass[handout]{ximera}

%% You can put user macros here
%% However, you cannot make new environments



\newcommand{\ffrac}[2]{\frac{\text{\footnotesize $#1$}}{\text{\footnotesize $#2$}}}
\newcommand{\vasymptote}[2][]{
    \draw [densely dashed,#1] ({rel axis cs:0,0} -| {axis cs:#2,0}) -- ({rel axis cs:0,1} -| {axis cs:#2,0});
}


\graphicspath{{./}{firstExample/}}
\usepackage{forest}
\usepackage{amsmath}
\usepackage{amssymb}
\usepackage{array}
\usepackage[makeroom]{cancel} %% for strike outs
\usepackage{pgffor} %% required for integral for loops
\usepackage{tikz}
\usepackage{tikz-cd}
\usepackage{tkz-euclide}
\usetikzlibrary{shapes.multipart}


%\usetkzobj{all}
\tikzstyle geometryDiagrams=[ultra thick,color=blue!50!black]


\usetikzlibrary{arrows}
\tikzset{>=stealth,commutative diagrams/.cd,
  arrow style=tikz,diagrams={>=stealth}} %% cool arrow head
\tikzset{shorten <>/.style={ shorten >=#1, shorten <=#1 } } %% allows shorter vectors

\usetikzlibrary{backgrounds} %% for boxes around graphs
\usetikzlibrary{shapes,positioning}  %% Clouds and stars
\usetikzlibrary{matrix} %% for matrix
\usepgfplotslibrary{polar} %% for polar plots
\usepgfplotslibrary{fillbetween} %% to shade area between curves in TikZ



%\usepackage[width=4.375in, height=7.0in, top=1.0in, papersize={5.5in,8.5in}]{geometry}
%\usepackage[pdftex]{graphicx}
%\usepackage{tipa}
%\usepackage{txfonts}
%\usepackage{textcomp}
%\usepackage{amsthm}
%\usepackage{xy}
%\usepackage{fancyhdr}
%\usepackage{xcolor}
%\usepackage{mathtools} %% for pretty underbrace % Breaks Ximera
%\usepackage{multicol}



\newcommand{\RR}{\mathbb R}
\newcommand{\R}{\mathbb R}
\newcommand{\C}{\mathbb C}
\newcommand{\N}{\mathbb N}
\newcommand{\Z}{\mathbb Z}
\newcommand{\dis}{\displaystyle}
%\renewcommand{\d}{\,d\!}
\renewcommand{\d}{\mathop{}\!d}
\newcommand{\dd}[2][]{\frac{\d #1}{\d #2}}
\newcommand{\pp}[2][]{\frac{\partial #1}{\partial #2}}
\renewcommand{\l}{\ell}
\newcommand{\ddx}{\frac{d}{\d x}}

\newcommand{\zeroOverZero}{\ensuremath{\boldsymbol{\tfrac{0}{0}}}}
\newcommand{\inftyOverInfty}{\ensuremath{\boldsymbol{\tfrac{\infty}{\infty}}}}
\newcommand{\zeroOverInfty}{\ensuremath{\boldsymbol{\tfrac{0}{\infty}}}}
\newcommand{\zeroTimesInfty}{\ensuremath{\small\boldsymbol{0\cdot \infty}}}
\newcommand{\inftyMinusInfty}{\ensuremath{\small\boldsymbol{\infty - \infty}}}
\newcommand{\oneToInfty}{\ensuremath{\boldsymbol{1^\infty}}}
\newcommand{\zeroToZero}{\ensuremath{\boldsymbol{0^0}}}
\newcommand{\inftyToZero}{\ensuremath{\boldsymbol{\infty^0}}}


\newcommand{\numOverZero}{\ensuremath{\boldsymbol{\tfrac{\#}{0}}}}
\newcommand{\dfn}{\textbf}
%\newcommand{\unit}{\,\mathrm}
\newcommand{\unit}{\mathop{}\!\mathrm}
%\newcommand{\eval}[1]{\bigg[ #1 \bigg]}
\newcommand{\eval}[1]{ #1 \bigg|}
\newcommand{\seq}[1]{\left( #1 \right)}
\renewcommand{\epsilon}{\varepsilon}
\renewcommand{\iff}{\Leftrightarrow}

\DeclareMathOperator{\arccot}{arccot}
\DeclareMathOperator{\arcsec}{arcsec}
\DeclareMathOperator{\arccsc}{arccsc}
\DeclareMathOperator{\si}{Si}
\DeclareMathOperator{\proj}{proj}
\DeclareMathOperator{\scal}{scal}
\DeclareMathOperator{\cis}{cis}
\DeclareMathOperator{\Arg}{Arg}
%\DeclareMathOperator{\arg}{arg}
\DeclareMathOperator{\Rep}{Re}
\DeclareMathOperator{\Imp}{Im}
\DeclareMathOperator{\sech}{sech}
\DeclareMathOperator{\csch}{csch}
\DeclareMathOperator{\Log}{Log}

\newcommand{\tightoverset}[2]{% for arrow vec
  \mathop{#2}\limits^{\vbox to -.5ex{\kern-0.75ex\hbox{$#1$}\vss}}}
\newcommand{\arrowvec}{\overrightarrow}
\renewcommand{\vec}{\mathbf}
\newcommand{\veci}{{\boldsymbol{\hat{\imath}}}}
\newcommand{\vecj}{{\boldsymbol{\hat{\jmath}}}}
\newcommand{\veck}{{\boldsymbol{\hat{k}}}}
\newcommand{\vecl}{\boldsymbol{\l}}
\newcommand{\utan}{\vec{\hat{t}}}
\newcommand{\unormal}{\vec{\hat{n}}}
\newcommand{\ubinormal}{\vec{\hat{b}}}

\newcommand{\dotp}{\bullet}
\newcommand{\cross}{\boldsymbol\times}
\newcommand{\grad}{\boldsymbol\nabla}
\newcommand{\divergence}{\grad\dotp}
\newcommand{\curl}{\grad\cross}
%% Simple horiz vectors
\renewcommand{\vector}[1]{\left\langle #1\right\rangle}


\outcome{Define and compute partial derivatives.}

\title{3.3 Partial Derivatives}



\begin{document}

\begin{abstract}
In this section we define and compute partial derivatives.
\end{abstract}

\maketitle

The derivative of a function $f(x)$ is defined by:

\[
f\,'(x) =  \frac{d}{dx} f(x) = \lim_{h \to 0} \frac{f(x+h) -f(x)}{h}
\]

The symbol $\frac{d}{dx}$ is called the derivative operator. If we state that $y = f(x)$, then the derivative can also be written as $\frac{dy}{dx}$.
The expression in the limit is called the difference quotient and it gives the slope of a secant line.
%The situation with a function of two variables is more complicated because at a point on the graph of $z = f(x,y)$ there are 
%infinitely many directions to move rather than just two. 
The partial derivatives of a function of two variables are defined analogously to the single variable case.

\begin{definition}[Partial Derivatives]
The partial derivative of $f(x,y)$ with respect to $x$, denoted $f_x(x,y)$ or $\frac{\partial}{\partial x} f(x,y)$, is defined by
\[
f_x(x,y)= \lim_{h \to 0} \frac{f(x+h, y) -f(x, y)}{h}
\]

The partial derivative with respect to $y$, denoted $f_y(x,y)$ or $\frac{\partial}{\partial y} f(x,y)$, is defined by


\[
f_y(x,y) = \lim_{h \to 0} \frac{f(x, y+h) -f(x, y)}{h}
\]

The symbols $\frac{\partial}{\partial x}$ and $\frac{\partial}{\partial y}$ are called partial derivative operators.

\end{definition}



\begin{example}[Example 1]
Use the definition of partial derivative to compute $f_x(x,y)$ for the function
\[
f(x,y) = 2xy^2 + 4xy - 2x -1
\]
We compute $f(x+h, y)$ as an aside:
\begin{align*}
f(x+h, y) &= 2(x+h)y^2 + 4(x+h)y - 2(x+h)  -1\\
          &= 2xy^2 + 2hy^2 + 4xy + 4hy - 2x - 2h - 1
\end{align*}
Note that among the 7 terms in this expression, 4 of them are exactly the terms in the original function.
Hence, the numerator of the difference quotient will simplify to:
\[
f(x+h, y) - f(x,y) =  2hy^2 + 4hy - 2h
\]
Observe that each of these terms contains a factor of $h$. Thus the difference quotient simplifies to
\[
\frac{f(x+h, y) -f(x, y)}{h} = 2y^2 + 4y - 2
\]
Finally, we take the limit as $h \to 0$ to obtain the partial derivative with respect to $x$:
\[
f_x(x,y) = \lim_{h \to 0} \left(2y^2 + 4y - 2\right) = 2y^2 + 4y - 2
\]

\end{example}

\begin{remark}
Note that this result can be obtained by treating the variable $y$ as a constant and computing an ordinary derivative with respect to $x$.
\end{remark}

\begin{problem}(Problem 1)
Use the definition of partial derivative to compute $f_x(x,y)$ for the function
\[
f(x,y) = 3x^2y + 5xy - 2y - 4
\]
Start by computing $f(x+h, y)$:
\[
f(x+h, y) = \answer{3(x+h)^2y + 5(x+h)y - 2y - 4}
\]

The numerator of the difference quotient is:
\[
f(x+h,y) - f(x,y) = \answer{6hxy + 3h^2y + 5hy}
\]

The difference quotient simplifies to:
\[
\frac{f(x+h,y) - f(x,y)}{h} = \answer{6xy + 3hy + 5y}
\]


The partial derivative with respect to $x$ is
\[
\frac{\partial}{\partial x} f(x,y) = \answer{6xy + 5y}
\]

\end{problem}

\begin{example}[Example 2]
Use the definition of partial derivative to compute $f_y(x,y)$ for the function
\[
f(x,y) = 2xy^2 + 4xy - 2x -1
\]
We compute $f(x, y+h)$ as an aside:
\begin{align*}
f(x, y+h) &= 2x(y+h)^2 + 4x(y+h) - 2x  -1\\
          &= 2x(y^2 + 2hy + h^2) + 4xy + 4hx - 2x - 1\\
          &= 2xy^2 + 4hxy + 2h^2x + 4xy + 4hx - 2x - 1
\end{align*}
Note that among the 7 terms in this expression, 4 of them are exactly the terms in the original function.
Hence, the numerator of the difference quotient will simplify to:
\[
f(x+h, y) - f(x,y) =  4hxy + 2h^2x  + 4hx
\]
Observe that each of these terms contains a factor of $h$. Thus the difference quotient simplifies to
\[
\frac{f(x+h, y) -f(x, y)}{h} = 4xy + 2hx +4x
\]
Finally, we take the limit as $h \to 0$ to obtain the partial derivative with respect to $x$:
\[
f_x(x,y) = \lim_{h \to 0} \left(4xy + 2hx +4x\right) = 4xy + 4x
\]

\end{example}

\begin{remark}
Note that this result can be obtained by treating the variable $x$ as a constant and computing an ordinary derivative with respect to $y$.
\end{remark}

\begin{problem}(Problem 2)
Use the definition of partial derivative to compute $f_y(x,y)$ for the function
\[
f(x,y) = 3x^2y + 5xy - 2y - 4
\]

Start by computing $f(x, y+h)$:
\[
f(x+h, y) = \answer{3x^2(y+h) + 5x(y+h) - 2(y+h) - 4}
\]

The numerator of the difference quotient is:
\[
f(x,y+h) - f(x,y) = \answer{3hx^2 + 5hx - 2h}
\]

The difference quotient simplifies to:
\[
\frac{f(x,y+h) - f(x,y)}{h} = \answer{3x^2 + 5x -2}
\]

The partial derivative with respect to $y$ is
\[
\frac{\partial}{\partial y} f(x,y) = \answer{3x^2 + 5x -2}
\]
\end{problem}

As was noted in the remarks following each of the two previous examples, partial derivatives can be computed by considering one of the variables to be constant.

\begin{example}[Example 3]
Find $f_x(x,y)$ for the function 
\[
f(x,y) = x^4y^3 + 2xy + y\sin(x) - e^y
\]

Since we are trying to find the partial derivative with respect to $x$, we treat the variable $y$ as a constant.
Differentiating term by term in this fashion gives:
\[
f_x(x,y) = 4x^3y^3 + 2y + y\cos(x)
\]
\end{example}

\begin{problem}(Problem 3a)
Find $f_x(x,y)$ for the function 
\[
f(x,y) = x^5y^2 - 3x\sqrt y + y\ln(x) + 3y
\]

\[
f_x(x,y) = \answer{5x^4y^2 - 3\sqrt y + y/x}
\]
\end{problem}

\begin{problem}(Problem 3b)
Find $f_x(x,y)$ for the function $f(x,y) = \frac{2y}{x} + y^2\tan(x) + 4\sec(y)$.\\
\[
f_x(x,y) = \answer{-\frac{2y}{x^2} + y\sec^2(x)}
\]
\end{problem}


\begin{example}[Example 4]
Find $f_y(x,y)$ for the function 
\[
f(x,y) = x^4y^3 + 2x\sqrt{y} + y^2\cos(x) - e^y
\]

Since we are trying to find the partial derivative with respect to $y$, we treat the variable $x$ as a constant.
Differentiating term by term in this fashion gives:
\[
f_y(x,y) = 3x^4y^2 + \frac{x}{\sqrt{y}} + 2y\cos(x) - e^y
\]
\end{example}

\begin{problem}(Problem 4a)
Find $f_y(x,y)$ for the function 
\[
f(x,y) = x^5y^2 - \frac{3x}{y} + y\ln(x) + 3y
\]

\[
f_y(x,y) = \answer{2x^5 y + \frac{3x}{y^2} + \ln(x) + 3}
\]
\end{problem}

\begin{problem}(Problem 4b)
Find $f_y(x,y)$ for the function 
\[
f(x,y) = \frac{2y}{x} + y^2\tan(x) + 4\sec(y)
\]

\[
f_y(x,y) = \answer{\frac{2}{x} + 2y\tan(x) + 4\sec(y)\tan(y)}
\]
\end{problem}

The next example uses the chain rule.  Recall that $\frac{d}{dx} \sqrt x = \frac{1}{2\sqrt x}$.

\begin{example}[Example 5]
Compute $f_x(x,t)$ and $f_t(x,t)$ for the function 
\[
f(x,t) = \sqrt{3x + 4t}
\]
First, note that the independent variables in the function $f$ are $x$ and $t$.
The function $f$ can be written as a composition of two functions:
\[
f(x,t) = \sqrt {u}, \quad \text{where} \quad u= u(x,t) = 3x + 4t
\]
From the chain rule,
\begin{align*}
f_x(x,t) &= \frac{d}{du}\left( \sqrt u\right) \cdot u_x\\
  & = \frac{1}{2\sqrt u} \cdot 3\\
  &= \frac{3}{2\sqrt{3x + 4t}}
\end{align*}
Similarly:
\begin{align*}
f_t(x,t) &= \frac{d}{du} \sqrt u \cdot u_t\\
  & = \frac{1}{2\sqrt u} \cdot 4\\
  &= \frac{4}{2\sqrt{3x + 4t}}\\
 &= \frac{2}{\sqrt{3x + 4t}}
\end{align*}
  
\end{example}


In the next example we use the product rule in addition to the chain rule.

\begin{example}[Example 6]
Compute $f_x(x,y)$ and $f_y(x,y)$ for the function 
\[
f(x,y) = x\sin(xy)
\]
Viewed as a function of $x$, $f$ is a product first and then a composition. We will need to use the product rule and then
the chain rule to compute $f_x$.
We have
\begin{align*}
f_x(x,y) &= \sin(x,y) + x \frac{\partial}{\partial x} \sin(xy)\\
         & = \sin(x,y) + x \cos(xy) \cdot y\\
         & = \sin(x,y) + xy \cos(xy)
\end{align*}
As a function of $y, f$ is a composition of the sine function and the function $xy$, so
\[
f_y(x,y) x \cos(xy) \cdot x = x^2 \cos(xy)
\]

\end{example}


\section{Higher Order Derivatives}

\begin{theorem}[Clairault's Theorem]
If the mixed second order partial derivatives $f_{xy}(x,y)$ and $f_{yx}(x,y)$ are continuous in the neightbohood of a point, 
then they are equal at that point.
\end{theorem}


\begin{remark}
In practice, the functions we will be differentiating are continuous on their domains, 
so we will freely interchange the order of differentiation.
\end{remark}


\end{document}
