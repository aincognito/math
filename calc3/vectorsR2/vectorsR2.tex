\documentclass[handout]{ximera}

%% You can put user macros here
%% However, you cannot make new environments



\newcommand{\ffrac}[2]{\frac{\text{\footnotesize $#1$}}{\text{\footnotesize $#2$}}}
\newcommand{\vasymptote}[2][]{
    \draw [densely dashed,#1] ({rel axis cs:0,0} -| {axis cs:#2,0}) -- ({rel axis cs:0,1} -| {axis cs:#2,0});
}


\graphicspath{{./}{firstExample/}}
\usepackage{forest}
\usepackage{amsmath}
\usepackage{amssymb}
\usepackage{array}
\usepackage[makeroom]{cancel} %% for strike outs
\usepackage{pgffor} %% required for integral for loops
\usepackage{tikz}
\usepackage{tikz-cd}
\usepackage{tkz-euclide}
\usetikzlibrary{shapes.multipart}


%\usetkzobj{all}
\tikzstyle geometryDiagrams=[ultra thick,color=blue!50!black]


\usetikzlibrary{arrows}
\tikzset{>=stealth,commutative diagrams/.cd,
  arrow style=tikz,diagrams={>=stealth}} %% cool arrow head
\tikzset{shorten <>/.style={ shorten >=#1, shorten <=#1 } } %% allows shorter vectors

\usetikzlibrary{backgrounds} %% for boxes around graphs
\usetikzlibrary{shapes,positioning}  %% Clouds and stars
\usetikzlibrary{matrix} %% for matrix
\usepgfplotslibrary{polar} %% for polar plots
\usepgfplotslibrary{fillbetween} %% to shade area between curves in TikZ



%\usepackage[width=4.375in, height=7.0in, top=1.0in, papersize={5.5in,8.5in}]{geometry}
%\usepackage[pdftex]{graphicx}
%\usepackage{tipa}
%\usepackage{txfonts}
%\usepackage{textcomp}
%\usepackage{amsthm}
%\usepackage{xy}
%\usepackage{fancyhdr}
%\usepackage{xcolor}
%\usepackage{mathtools} %% for pretty underbrace % Breaks Ximera
%\usepackage{multicol}



\newcommand{\RR}{\mathbb R}
\newcommand{\R}{\mathbb R}
\newcommand{\C}{\mathbb C}
\newcommand{\N}{\mathbb N}
\newcommand{\Z}{\mathbb Z}
\newcommand{\dis}{\displaystyle}
%\renewcommand{\d}{\,d\!}
\renewcommand{\d}{\mathop{}\!d}
\newcommand{\dd}[2][]{\frac{\d #1}{\d #2}}
\newcommand{\pp}[2][]{\frac{\partial #1}{\partial #2}}
\renewcommand{\l}{\ell}
\newcommand{\ddx}{\frac{d}{\d x}}

\newcommand{\zeroOverZero}{\ensuremath{\boldsymbol{\tfrac{0}{0}}}}
\newcommand{\inftyOverInfty}{\ensuremath{\boldsymbol{\tfrac{\infty}{\infty}}}}
\newcommand{\zeroOverInfty}{\ensuremath{\boldsymbol{\tfrac{0}{\infty}}}}
\newcommand{\zeroTimesInfty}{\ensuremath{\small\boldsymbol{0\cdot \infty}}}
\newcommand{\inftyMinusInfty}{\ensuremath{\small\boldsymbol{\infty - \infty}}}
\newcommand{\oneToInfty}{\ensuremath{\boldsymbol{1^\infty}}}
\newcommand{\zeroToZero}{\ensuremath{\boldsymbol{0^0}}}
\newcommand{\inftyToZero}{\ensuremath{\boldsymbol{\infty^0}}}


\newcommand{\numOverZero}{\ensuremath{\boldsymbol{\tfrac{\#}{0}}}}
\newcommand{\dfn}{\textbf}
%\newcommand{\unit}{\,\mathrm}
\newcommand{\unit}{\mathop{}\!\mathrm}
%\newcommand{\eval}[1]{\bigg[ #1 \bigg]}
\newcommand{\eval}[1]{ #1 \bigg|}
\newcommand{\seq}[1]{\left( #1 \right)}
\renewcommand{\epsilon}{\varepsilon}
\renewcommand{\iff}{\Leftrightarrow}

\DeclareMathOperator{\arccot}{arccot}
\DeclareMathOperator{\arcsec}{arcsec}
\DeclareMathOperator{\arccsc}{arccsc}
\DeclareMathOperator{\si}{Si}
\DeclareMathOperator{\proj}{proj}
\DeclareMathOperator{\scal}{scal}
\DeclareMathOperator{\cis}{cis}
\DeclareMathOperator{\Arg}{Arg}
%\DeclareMathOperator{\arg}{arg}
\DeclareMathOperator{\Rep}{Re}
\DeclareMathOperator{\Imp}{Im}
\DeclareMathOperator{\sech}{sech}
\DeclareMathOperator{\csch}{csch}
\DeclareMathOperator{\Log}{Log}

\newcommand{\tightoverset}[2]{% for arrow vec
  \mathop{#2}\limits^{\vbox to -.5ex{\kern-0.75ex\hbox{$#1$}\vss}}}
\newcommand{\arrowvec}{\overrightarrow}
\renewcommand{\vec}{\mathbf}
\newcommand{\veci}{{\boldsymbol{\hat{\imath}}}}
\newcommand{\vecj}{{\boldsymbol{\hat{\jmath}}}}
\newcommand{\veck}{{\boldsymbol{\hat{k}}}}
\newcommand{\vecl}{\boldsymbol{\l}}
\newcommand{\utan}{\vec{\hat{t}}}
\newcommand{\unormal}{\vec{\hat{n}}}
\newcommand{\ubinormal}{\vec{\hat{b}}}

\newcommand{\dotp}{\bullet}
\newcommand{\cross}{\boldsymbol\times}
\newcommand{\grad}{\boldsymbol\nabla}
\newcommand{\divergence}{\grad\dotp}
\newcommand{\curl}{\grad\cross}
%% Simple horiz vectors
\renewcommand{\vector}[1]{\left\langle #1\right\rangle}


\outcome{In this section we define vectors in two dimensions and study their algebraic and geometric properties.}

\title{1.2 Vectors in $\R^2$}
%Vectors are represented graphically by arrows.
%and in three dimensions we write $\avec{v} = \vector{x,y, z}$.
%The length of the arrow represents the magnitude of the vector and the arrow points in the direction of the vector.
\begin{document}

\begin{abstract}
In this section we define vectors in two dimensions and study their algebraic and geometric properties.
\end{abstract}
 
\maketitle

A {\bf vector} is a quantity that has both magnitude and direction. Examples include a force acting on an object and the velocity of an object in motion.
Algebraically, vectors are represented in a manner very similar to points.  In two dimensions, a vector is written $\avec{v} = \vector{x,y}$  
where $x$ and $y$ are the {\bf components} of $\vec{v}$.


\section{Graphing Vectors in $\R^2$}
The graphical representation of the vector $\avec{v} = \vector{x,y}$ in standard position
is an arrow emanating from the origin (initial point) and terminating at the point $(x,y)$ (the final point).
Below is a graphical representation of the vector $\vector{1,1}$ in standard position.

\begin{image}
\begin{tikzpicture}
\draw[ ->] (0,0) -- (3.6, 0) node[right]{$x$};
\draw[ ->] (0,0) -- (0, 3.6) node[above]{$y$};
\draw[->, blue] (0,0) -- (2, 2) node[right]{$\vector{1,1}$};
\draw[blue, fill] (0,0) circle (0.03);
\draw[thin] (2, 0.15) -- (2, -0.15) node[below]{$1$};
\draw[thin] (0.15,2) -- (-0.15,2) node[left]{$1$};
\end{tikzpicture}
\end{image}

It is not necessary to place a vector in standard position.  
In practical applications of vectors, it is typical to place the initial point of a vector in a position of interest.  
For example, if the vector represents the velocity of an object, then one would likely choose the initial point to be the location of the object.
Below is a graphical representation of the vector, $\vector{1,1}$ placed in several different positions.
Since each of the arrows below are the same length and point in the same direction, they represent the same vector.
Notice that in each copy of the vector, the final point is one unit to the right and one unit up from the initial point. 
\begin{image}
\begin{tikzpicture}
\draw[step=1.0,black,thin] (-2.5,-2.5) grid (2.5,2.5);
\draw[thick, <->] (-2.5,0) -- (2.5, 0) node[right]{$x$};
\draw[thick, <->] (0,-2.5) -- (0, 2.5) node[above]{$y$};
\draw[->, blue] (0,0) -- (1, 1) ;
\draw[->, blue] (-2,0) -- (-1, 1) ;
\draw[->, blue] (0,1) -- (1, 2) ;
\draw[->, blue] (-2,1) -- (-1, 2) ;
\draw[->, blue] (1,-2) -- (2, -1) ;
\draw[->, blue] (0,-1) -- (1, 0) ;
\draw[->, blue] (-2,-2) -- (-1, -1) ;
\draw[->, blue] (1,-1) -- (2, 0) ;
\node at (2.2, -0.3){$2$};
\node at (-1.8, -0.3){$-2$};
\node at (-0.3,2){$2$};
\node at (-0.3, -2){$-2$};
\node at (0, -2.8) {Multiple copies of the vector $\vector{1,1}$};
\end{tikzpicture}
\end{image}

Two special vectors in $\R^2$ are $\avec{i} = \vector{1,0}$ and $\avec{j} = \vector{0,1}$.
These vectors both have a magnitude of $1$, making them {\bf unit vectors}. 
Vertical bars are used to indicate magnitude, so $|\avec{i}| = 1$ and $|\avec{j}| = 1$.
The vectors $\avec{i}$ and $\avec{j}$ are shown below.

\begin{image}
\begin{tikzpicture}
\draw[ <->] (-1.5,0) -- (1.8, 0) node[right]{$x$};
\draw[<->] (0,-1.5) -- (0,1.8) node[above]{$y$};
\draw[thick,->, blue] (0,0) -- (1, 0) node[above ]{$\avec{i}$};
\draw[thick,->, blue] (0,0) -- (0,1) node[right]{$\avec{j}$};
\draw[blue, fill] (0,0) circle (0.01);
\draw[thin] (1, 0.1) -- (1, -0.1) node[below]{$1$};
\draw[thin] (0.1,1) -- (-0.1,1) node[left]{$1$};
\end{tikzpicture}
\end{image}

Another special vector is the zero vector, $\avec{0} = \vector{0,0}$.  
This vector has magnitude zero and no direction.  
It is represented graphically by a single point.

\begin{image}
\begin{tikzpicture}
\draw[ <->] (-2,0) -- (2, 0) node[right]{$x$};
\draw[<->] (0,-2) -- (0, 2) node[above]{$y$};
\draw[blue, fill] (0,0) circle (0.03) node[below right]{$\avec{0}$};
%\draw[thin] (1, 0.15) -- (1, -0.15) node[below]{$1$};
%\draw[thin] (0.15,1) -- (-0.15,1) node[left]{$1$};
\end{tikzpicture}
\end{image}

The vector with initial point $P(x_1, y_1)$ and final point $Q(x_2, y_2)$ has components $\avec{PQ} = \vector{x_2 - x_1, y_2-y_1}$


\begin{image}
\begin{tikzpicture}
\draw[ <->] (-2,0) -- (2, 0) node[right]{$x$};
\draw[ <->] (0,-1) node[below]{The vector $\avec{PQ} = \vector{x_2 - x_1, y_2-y_1}$} -- (0, 2) node[above]{$y$};
\draw[->, blue] (-1,0.5) node[left]{$P(x_1,y_1)$} -- (1.5, 1.5) node[right]{$Q(x_2,y_2)$};
\draw[blue, fill] (-1, 0.5) circle (0.03);
%\draw[thin] (2, 0.15) -- (2, -0.15) node[below]{$1$};
%\draw[thin] (0.15,2) -- (-0.15,2) node[left]{$1$};
\end{tikzpicture}
\end{image}

\begin{example}
Find the components of the vector from $P(-2, 3)$ to the point $Q(1, 1)$ and draw two copies of this vector- the first from the point $P$ to the point $Q$
and the second in standard position.\\
The components of $\avec{PQ}$ are $\vector{1-(-2), 1-3} = \vector{3, -2}$.
The two copies of the vector are shown below.

\begin{image}
\begin{tikzpicture}
\draw[step=1.0,black,thin] (-3.5,-3.5) grid (3.5,3.5);
\draw[thick, <->] (-3.5,0) -- (3.5, 0) node[right]{$x$};
\draw[thick, <->] (0,-3.5) -- (0, 3.5) node[above]{$y$};
\draw[->, blue, thick] (0,0) -- (3, -2) ;
\draw[blue, fill] (0, 0) circle (0.03);
\draw[->, blue, thick] (-2,3) node[above left]{$P$} -- (1, 1) node[right]{$Q$} ;
\draw[blue, fill] (-2, 3) circle (0.03);
\node at (2.1, -0.3){$2$};
\node at (-2, -0.3){$-2$};
\node at (-0.3,2){$2$};
\node at (-0.3, -2){$-2$};
\node at (0, -3.8) {Two copies of the vector $\avec{PQ} = \vector{3,-2}$};
\end{tikzpicture}
\end{image}

\end{example}

\begin{problem}
Find the components of the vector from $P(-3, 2)$ to the point $Q(2, -1)$ and draw two copies of this vector- the first from the point $P$ to the point $Q$
and the second in standard position.\\
\end{problem}


\subsection{Magnitude of a Vector}

The magnitude of a vector $\avec{v} = \vector{x,y}$ in $\R^2$ is given by the distance formula:
\[
|\avec{v}| = |\vector{x,y}| = \sqrt{x^2 +y^2}
\]

\begin{example}
Find the magnitude of the vector $\vector{1,1}$.\\
Its magnitude is
\[
|\vector{1,1}| = \sqrt{1^2 +1^2} = \sqrt 2
\]
\end{example}

\begin{problem}
Find the magnitudes of the following vectors:\\
a) $|\vector{2,3}| = \answer{\sqrt{13}}$\\
b) $|\vector{-4,2}| = \answer{2\sqrt{5}}$\\
c) $|\vector{12,-5}| = \answer{13}$
\end{problem}

The magnitude of the zero vector is $0$ and it is the only vector with magnitude zero
\[
|\avec{0}| = |\vector{0,0}| = \sqrt{0^2 +0^2} = 0
\]

\section{Vector Operations}
In this section, we learn about two fundamental operations involving vectors: scalar multiplication and addition.
The more advanced dot product and cross product operations will be handled in future sections.

\subsection{Scalar Multiplication}
Given a vector $\avec{v} = \vector{x,y}$ in $\R^2$ and a scalar $c$ in $\R$, we can multiply the scalar and the vector in the following
natural way
\[
c\avec{v} = c\vector{x,y} = \vector{cx,cy}
\]
In words, we multiply a vector by a scalar by multiplying each component by the scalar.
The first thing to note is that if the scalar is 0, then the resulting vector will be the zero vector:
\[
0\avec{v} = 0\vector{x,y} = \vector{0,0} = \avec{0}
\]

The effect of multiplying a vector by a scalar on its magnitude is given in the following proposition.
\begin{proposition}
Given a vector $\avec{v} = \vector{x,y}$ in $\R^2$ and a scalar $c$ in $\R$,
\[
|c\avec{v}| = |c| |\avec{v}|
\]
\begin{proof}
Write $\avec{v}$ in component form as $\avec{v} = \vector{x,y}$, then
\begin{align*}
|c\avec{v}| &= |c\vector{x,y}|\\
            &= |\vector{cx,cy}|\\
            &= \sqrt{(cx)^2 + (cy)^2}\\
            &= \sqrt{c^2x^2 + c^2y^2}\\
            &= \sqrt{c^2(x^2 + y^2)}\\
            &= \sqrt{c^2} \sqrt{x^2 + y^2}\\
            &= |c| |\avec{v}|
\end{align*}
\end{proof}
\end{proposition}

When $c>0$, the vector $c\avec{v}$ has the same direction as $\avec{v}$, and when $c<0$, the
the vector $c\avec{v}$ has the opposite direction as $\avec{v}$ as illustrated below.



\begin{image}
\begin{tikzpicture}
\draw[blue, thick, ->] (-3,0) -- (-2,0) node[right]{$\avec{v}$};
\draw[red, thick, <-] (-3,-1) node[left ]{$-\avec{v}$} -- (-2,-1) ;
\draw[blue, thick, ->] (0,0) -- (2,0) node[right]{$2\avec{v}$};
\draw[red, thick, <-] (0,-1) node[left]{$-2\avec{v}$} -- (2,-1);
\draw[blue, thick, ->] (4,0) -- (4.5,0) node[right]{$\tfrac12\avec{v}$};
\draw[red, thick, <-] (4,-1)node[left]{$-\tfrac12\avec{v}$} -- (4.5,-1) ;
\node at (0.75, -2) {Multiplying $\avec{v}$ by positive and negative scalars};
%\draw[white] (0,-4) -- (6, -4);
\end{tikzpicture}
\end{image}


\subsection{Vector Addition}
Vector addition is done componentwise.  If $\avec{v_1} = \vector{x_1, y_1}$ and $\avec{v_2} = \vector{x_2, y_2}$
then their sum is given by
\[
\avec{v_1 }+ \avec{ v_2}= \vector{x_1+x_2,  y_1+y_2}
\]
Vector subtraction is addition of the negative which results in componentwise subtraction:
\[
\avec{v_1 }- \avec{ v_2}= \avec{v_1 }+ \left(- \avec{ v_2}\right)  = \vector{x_1,y_1} + \vector{-x_2, -y_2} = \vector{x_1-x_2,  y_1-y_2}
\]

\begin{example}
Let $\avec{u} = \vector{-4, 2}$ and $\avec{v} = \vector{5, -7}$.  Compute each of the following:\\
a) $\avec{u}+\avec{v} = \vector{1, -5}$\\
b) $\avec{u}-\avec{v} = \vector{-9, 9}$\\
c) $\avec{v}-\avec{u} = \vector{9, -9}$\\
d) $2\avec{u}+3\avec{v} = \vector{7, -17}$\\
\end{example}

\begin{problem}
Let $\avec{u} = \vector{2, -3}$ and $\avec{v} = \vector{-1, 6}$.  Compute each of the following:\\
a) $\avec{u}+\avec{v} = \vector{\answer{1}, \answer{3}}$\\
b) $\avec{u}-\avec{v} = \vector{\answer{3}, \answer{-9}}$\\
c) $\avec{v}-\avec{u} = \vector{\answer{-3}, \answer{9}}$\\
d) $3\avec{u}+5\avec{v} = \vector{\answer{1}, \answer{21}}$\\
\end{problem}

Graphically, vector addition can be performed using either the paralleogram method or the end to end method as pictured below.
\begin{image}
\begin{tikzpicture}
\draw[blue, ->, thick] (0,0) -- (3, 1) node[midway, below]{$\avec{u}$};
\draw[blue, ->, thick] (0,0) -- (1, 4) node[midway, left]{$\avec{v}$};
\draw[blue, thin, dashed] (3, 1) -- (4, 5);
\draw[blue, thin, dashed] (1, 4) -- (4, 5);
\draw[red, ->, thick] (0,0) -- (4, 5) node[above]{$\avec{u} + \avec{v}$};
\node at (1.5, -1){Parallelogram Method};
\draw[blue, ->, thick] (6,0) -- (9, 1) node[midway, below]{$\avec{u}$};
\draw[blue, ->, thick] (9,1) -- (10, 5) node[midway, right]{$\avec{v}$};
\draw[red, ->, thick] (6,0) -- (10, 5) node[above]{$\avec{u} + \avec{v}$};
\node at (7.5, -1){End to End Method};
\end{tikzpicture}
\end{image}

The difference $\avec{u} - \avec{v}$ is the vector from $\avec{v}$ to $\avec{u}$ as shown below.
\begin{image}
\begin{tikzpicture}
\draw[blue, ->, thick] (0,0) -- (3, 1) node[midway, below]{$\avec{u}$};
\draw[blue, ->, thick] (0,0) -- (1, 4) node[midway, left]{$\avec{v}$};
\draw[red, ->, thick] (1,4) -- (3, 1) node[midway,right]{$\avec{u} - \avec{v}$};
\node at (1.5, -1){The vector $\avec{u} - \avec{v}$};
\draw[white] (-3, -1.5) -- (7, -1.5);
\end{tikzpicture}
\end{image}
Note that in the figure above, if we add the vectors $\avec{v}$ and $\avec{u} -\avec{v}$ using the end to end method, 
we get the vector $\avec{u}$, i.e.,
\[
\avec{v} + \left(\avec{u} -\avec{v}\right) = \avec{u}
\]

Vectors in $\R^2$ can be written using the {\bf standard basis} vectors $\avec{i} = \vector{1,0}$ and $\avec{j} = \vector{0,1}$.  
Given a vector $\avec{v} = \vector{x,y}$, we can use the addition and scalar multiplication operations to rewrite it as
\begin{align*}
\avec{v} &= \vector{x,y}\\
         &= \vector{x,0} + \vector{0,y}\\
         &= x\vector{1,0} + y \vector{0,1}\\
         &= x\avec{i} + y\avec{j}
\end{align*}

A vector with magnitude 1 is called a {\bf unit vector}. Given a vector $\avec{v}$, we will want to find a unit vector in the direction of $\avec{v}$.
This unit vector will have the form $c\avec{v}$ for some positive scalar $c$.  
\begin{proposition}
The unit vector $\avec{u}$ in the same direction as the vector $\avec{v} \neq \avec{0}$ is given by 
\[
\avec{u} = \frac{1}{|\avec{v}|} \avec{v}
\]
\begin{proof}
Since $|\avec{v}| \neq 0$, the scalar $\displaystyle \frac{1}{|\avec{v}|}$ is a positive real number.  Hence, the vector $\avec{u}$, being a positive multiple of 
the vector $\avec{v}$ points in the same direction as $\avec{v}$. We must now show that $|\avec{u}| = 1$.  By the previous proposition, we have
\begin{align*}
|\avec{u}| &= \left| \frac{1}{|\avec{v}|} \avec{v} \right|\\
           &= \left| \frac{1}{|\avec{v}|} \right| |\avec{v}|\\
           &= \frac{1}{|\avec{v}|} |\avec{v}|\\
           &= 1
\end{align*}
\end{proof}
\end{proposition}

\begin{example}
Find a unit vector in the same direction as the vector $\avec{v} = \vector{3, -4}$.\\
First we find the magnitude of $\avec{v}$:
\[
|\avec{v}| = |\vector{3, -4}| = \sqrt{3^2 + (-4)^2} = \sqrt{25} = 5
\]
By the proposition above, we have
\[
\avec{u} = \frac{1}{|\avec{v}|} \avec{v} = \frac15 \vector{3, -4} = \vector{\frac35, -\frac45}
\]
where $\avec{u}$ is a unit vector in the same direction as $\avec{v}$.
\end{example}

\begin{problem}
Find a unit vector in the same direction as the vector $\avec{v} = \vector{-8, 15}$.\\
$\avec{u} = \vector{\answer{-8/17}, \answer{15/17}}$
\end{problem}

\begin{problem}
Find a unit vector in the opposite direction of the vector $\avec{v} = \vector{9, -12}$.\\
$\avec{u} = \vector{\answer{-3/5}, \answer{4/5}}$
\end{problem}

\end{document}












\begin{image}
\begin{tikzpicture}
\draw[blue, thick, ->] (-3,0) -- (-2,0) node[right]{$\avec{v}$};
\draw[blue, thick, <-] (-3,-0.5) node[left ]{$-\avec{v}$} -- (-2,-0.5) ;
\draw[red, thick, ->] (0,-1) -- (2,-1) node[right]{$2\avec{v}$};
\draw[red, thick, <-] (0,-1.5) node[left]{$-2\avec{v}$} -- (2,-1.5);
\draw[green, thick, ->] (0,-2) -- (0.5,-2) node[right]{$\tfrac12\avec{v}$};
\draw[green, thick, <-] (0,-2.5)node[left]{$-\tfrac12\avec{v}$} -- (0.5,-2.5) ;

\end{tikzpicture}
\end{image}



\begin{image}
\begin{tikzpicture}
\draw[thick, ->] (0,0) -- (2,0) node[right]{$y$};
\draw[thick, ->] (0,0) -- (0,1.8) node[above]{$z$};
\draw[thick, ->] (0,0) -- (-1.2,-0.9) node[below, left]{$x$};
\draw[blue] (0.3,0.2) -- (1.7, 0.5) -- (1.7, 1.7) -- cycle;
\draw[thin, blue]  (1.6, 0.47) -- (1.6, .59) -- (1.7, .62);
\draw[fill, blue] (0.3,0.2) circle [radius=0.03] node[left, blue]{$(x_1, y_1, z_1)$};
\draw[fill, blue] (1.7,1.7) circle [radius=0.03] node[right, blue]{$(x_2, y_2, z_2)$};
\draw[fill, blue] (1.7,0.5) circle [radius=0.037] node[right, blue]{$(x_2, y_2, z_1)$};
\node[below, blue] at (1.3,0.4) {$d_1$}; %adjacent
\node[right, blue] at (1.9,1) {$d_2$}; %opposite 
%\node[right] at (4.1,2) {$\tan(\theta)$};
\node[above, blue] at (1,1.4) {$d$}; %hypotenuse
%\node[below] at (1.75,-0.6) {$d = \sqrt{(x_2 - x_1)^2 + |y_2 - y_1|^2}$};
%\node[above] at (1.75,3.6) {A right triangle with $\cos(\theta) = x$};
\end{tikzpicture}
\end{image}
