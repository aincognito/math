\documentclass[handout]{ximera}

%% You can put user macros here
%% However, you cannot make new environments



\newcommand{\ffrac}[2]{\frac{\text{\footnotesize $#1$}}{\text{\footnotesize $#2$}}}
\newcommand{\vasymptote}[2][]{
    \draw [densely dashed,#1] ({rel axis cs:0,0} -| {axis cs:#2,0}) -- ({rel axis cs:0,1} -| {axis cs:#2,0});
}


\graphicspath{{./}{firstExample/}}
\usepackage{forest}
\usepackage{amsmath}
\usepackage{amssymb}
\usepackage{array}
\usepackage[makeroom]{cancel} %% for strike outs
\usepackage{pgffor} %% required for integral for loops
\usepackage{tikz}
\usepackage{tikz-cd}
\usepackage{tkz-euclide}
\usetikzlibrary{shapes.multipart}


%\usetkzobj{all}
\tikzstyle geometryDiagrams=[ultra thick,color=blue!50!black]


\usetikzlibrary{arrows}
\tikzset{>=stealth,commutative diagrams/.cd,
  arrow style=tikz,diagrams={>=stealth}} %% cool arrow head
\tikzset{shorten <>/.style={ shorten >=#1, shorten <=#1 } } %% allows shorter vectors

\usetikzlibrary{backgrounds} %% for boxes around graphs
\usetikzlibrary{shapes,positioning}  %% Clouds and stars
\usetikzlibrary{matrix} %% for matrix
\usepgfplotslibrary{polar} %% for polar plots
\usepgfplotslibrary{fillbetween} %% to shade area between curves in TikZ



%\usepackage[width=4.375in, height=7.0in, top=1.0in, papersize={5.5in,8.5in}]{geometry}
%\usepackage[pdftex]{graphicx}
%\usepackage{tipa}
%\usepackage{txfonts}
%\usepackage{textcomp}
%\usepackage{amsthm}
%\usepackage{xy}
%\usepackage{fancyhdr}
%\usepackage{xcolor}
%\usepackage{mathtools} %% for pretty underbrace % Breaks Ximera
%\usepackage{multicol}



\newcommand{\RR}{\mathbb R}
\newcommand{\R}{\mathbb R}
\newcommand{\C}{\mathbb C}
\newcommand{\N}{\mathbb N}
\newcommand{\Z}{\mathbb Z}
\newcommand{\dis}{\displaystyle}
%\renewcommand{\d}{\,d\!}
\renewcommand{\d}{\mathop{}\!d}
\newcommand{\dd}[2][]{\frac{\d #1}{\d #2}}
\newcommand{\pp}[2][]{\frac{\partial #1}{\partial #2}}
\renewcommand{\l}{\ell}
\newcommand{\ddx}{\frac{d}{\d x}}

\newcommand{\zeroOverZero}{\ensuremath{\boldsymbol{\tfrac{0}{0}}}}
\newcommand{\inftyOverInfty}{\ensuremath{\boldsymbol{\tfrac{\infty}{\infty}}}}
\newcommand{\zeroOverInfty}{\ensuremath{\boldsymbol{\tfrac{0}{\infty}}}}
\newcommand{\zeroTimesInfty}{\ensuremath{\small\boldsymbol{0\cdot \infty}}}
\newcommand{\inftyMinusInfty}{\ensuremath{\small\boldsymbol{\infty - \infty}}}
\newcommand{\oneToInfty}{\ensuremath{\boldsymbol{1^\infty}}}
\newcommand{\zeroToZero}{\ensuremath{\boldsymbol{0^0}}}
\newcommand{\inftyToZero}{\ensuremath{\boldsymbol{\infty^0}}}


\newcommand{\numOverZero}{\ensuremath{\boldsymbol{\tfrac{\#}{0}}}}
\newcommand{\dfn}{\textbf}
%\newcommand{\unit}{\,\mathrm}
\newcommand{\unit}{\mathop{}\!\mathrm}
%\newcommand{\eval}[1]{\bigg[ #1 \bigg]}
\newcommand{\eval}[1]{ #1 \bigg|}
\newcommand{\seq}[1]{\left( #1 \right)}
\renewcommand{\epsilon}{\varepsilon}
\renewcommand{\iff}{\Leftrightarrow}

\DeclareMathOperator{\arccot}{arccot}
\DeclareMathOperator{\arcsec}{arcsec}
\DeclareMathOperator{\arccsc}{arccsc}
\DeclareMathOperator{\si}{Si}
\DeclareMathOperator{\proj}{proj}
\DeclareMathOperator{\scal}{scal}
\DeclareMathOperator{\cis}{cis}
\DeclareMathOperator{\Arg}{Arg}
%\DeclareMathOperator{\arg}{arg}
\DeclareMathOperator{\Rep}{Re}
\DeclareMathOperator{\Imp}{Im}
\DeclareMathOperator{\sech}{sech}
\DeclareMathOperator{\csch}{csch}
\DeclareMathOperator{\Log}{Log}

\newcommand{\tightoverset}[2]{% for arrow vec
  \mathop{#2}\limits^{\vbox to -.5ex{\kern-0.75ex\hbox{$#1$}\vss}}}
\newcommand{\arrowvec}{\overrightarrow}
\renewcommand{\vec}{\mathbf}
\newcommand{\veci}{{\boldsymbol{\hat{\imath}}}}
\newcommand{\vecj}{{\boldsymbol{\hat{\jmath}}}}
\newcommand{\veck}{{\boldsymbol{\hat{k}}}}
\newcommand{\vecl}{\boldsymbol{\l}}
\newcommand{\utan}{\vec{\hat{t}}}
\newcommand{\unormal}{\vec{\hat{n}}}
\newcommand{\ubinormal}{\vec{\hat{b}}}

\newcommand{\dotp}{\bullet}
\newcommand{\cross}{\boldsymbol\times}
\newcommand{\grad}{\boldsymbol\nabla}
\newcommand{\divergence}{\grad\dotp}
\newcommand{\curl}{\grad\cross}
%% Simple horiz vectors
\renewcommand{\vector}[1]{\left\langle #1\right\rangle}


\outcome{Compute arc length.}

\title{2.4 Arc Length}



\begin{document}

\begin{abstract}
In this section we compute arc length and we define the arc length parameter.
\end{abstract}

\maketitle


\begin{definition}
The arc length of the space curve
\[
\vec r(t) = \vector{x(t), y(t), z(t)}, \; a \leq t \leq b
\]
is given by
\begin{align*}
L &= \int_a^b |\vec r\,'(t)|\, dt = \int_a^b \sqrt{x'(t)^2 + y'(t)^2 +z'(t)^2}\, dt \\
&= \int_a^b \sqrt{\left(\frac{dx}{dt}\right)^2 + \left(\frac{dy}{dt}\right)^2 +\left(\frac{dz}{dt}\right)^2}\, dt\\
&= \int_a^b |\vec r\,'(t)| \, dt
\end{align*}

\end{definition}


Consider a space curve $\vec r(t), \; a \leq t \leq b$.  Subdivide the curve into $n$ pieces, by subdividing the interval $[a, b]$ into $n$ equal 
sub-intervals each of length $\Delta t = \frac{b-a}{n}$. Denote the endpoints of these $n$ sub-intervals by $a = t_0, t_1, t_2, ..., t_n = b$.
The length of the line segment connecting the points $\vec r(t_i)$ and $\vec r(t_{i-1})$ is given by
\[
\Delta \vec r_i = \sqrt{(\Delta x_i) ^2 + (\Delta y_i)^2 + (\Delta z_i)^2}
\]
and the length of the curve is approximately
\[
\sum_{i = 1}^n \sqrt{(\Delta x_i) ^2 + (\Delta y_i)^2 + (\Delta z_i)^2}
\]
\[
= \sum_{i = 1}^n \sqrt{\left(\frac{\Delta x_i}{\Delta t}\right)^2 + \left(\frac{\Delta y_i}{\Delta t}\right)^2 + \left(\frac{\Delta z_i}{\Delta t}\right)^2}\, \Delta t
\]
applying the Mean Value Theorem to each ratio in the radicand gives
\[
= \sum_{i = 1}^n \sqrt{x'(t_i^*)^2 + y'(t_i^{**})^2 + z'(t_i^{***})^2} \,\Delta t
\]
where $t_i^*, t_i^{**}, t_i^{***} \in [t_{i-1}, t_i]$.\\
Taking the limit as $n \to \infty$ gives the arc length formula in the definition:
\[
\lim_{n \to \infty} \sum_{i = 1}^n \sqrt{x'(t_i^*)^2 + y'(t_i^{**})^2 + z'(t_i^{***})^2} \,\Delta t  = \int_a^b \sqrt{x'(t)^2 + y'(t)^2 +z'(t)^2}\, dt  = \int_a^b |\vec r\,'(t)|\, dt
\]

\begin{image}
\begin{tikzpicture}
\draw[thick] (-2, 2) to [out = 45, in = 210] (-1, 4) ;
\draw[thick, blue!70!white, ->] (0,0) node[below left]{$O$} -- (-2, 2) node[above left]{$\vec r(t_0)$};
\draw[thick, blue!70!white, ->] (0,0) -- (-1, 4) node[above]{$\vec r(t_1)$};
\draw[thick] (-1,4) to [out = 20, in = 110] (0, 3);
\draw[thick, blue!70!white, ->] (0,0) -- (0, 3) node[above right]{$\vec r(t_{i-1})$};
\draw[thick] (0,3) to [out = 290, in= 180] (1,1);
\draw[thick, blue!70!white, ->] (0,0) -- (1,1) node[above right]{$\vec r(t_i)$};
\draw[thick] (1,1) to [out = 0, in= 100] (2,0) ;
\draw[thick, blue!70!white, ->] (0,0) -- (2,0) node[right]{$\vec r(t_n)$};
\draw[thick,red!70!black] (-2, 2) -- (-1, 4) -- (0, 3) -- (1, 1) node[midway, right]{$\Delta \vec r_i$}--(2, 0);
\node at (0, -1) {Approximating arc length using line segments};
\filldraw[red!70!black] (0,3) circle (0.04);
\filldraw[red!70!black] (1,1) circle (0.04);
\filldraw[red!70!black] (-2, 2) circle (0.04);
\filldraw[red!70!black] (-1, 4) circle (0.04);
\filldraw[red!70!black] (2, 0) circle (0.04);
\end{tikzpicture}
\end{image}

\begin{remark}
The form
\[
L = \int_a^b |\vec r\,'(t)| \, dt
\]
has a nice physical interpretation.  If $\vec r(t)$ represents the path of a particle, then $|\vec r\,'(t)|$ represents the speed of the particle and
the integrand then reflects the basic formula ``distance equals rate times time".
\end{remark}

\begin{example}[Example 1]
Find the arc length of the space curve given by $\vec r(t) = \vector{e^t, t\sqrt 2 , e^{-t}}$ from $t = 0$ to $t = 1$.\\
According to the formula for arc length, we have
\begin{align*}
L &= \int_0^1 \sqrt{\left(\frac{dx}{dt}\right)^2 + \left(\frac{dy}{dt}\right)^2 + \left(\frac{dz}{dt}\right)^2} \, dt\\
&= \int_0^1 \sqrt{ \left(e^t\right)^2 + \left( \sqrt 2\right)^2 + \left(-e^{-t}\right)^2} \, dt\\
&= \int_0^1 \sqrt{ e^{2t} + 2 + e^{-2t} }\, dt\\
&= \int_0^1 \sqrt{\left(e^t + e^{-t}\right)^2}\, dt\\
&= \int_0^1 \left(e^{t} + e^{-t} \right) \, dt\\
&=  e^{t} - e^{-t} \bigg|_0^1\\
&=   e - \frac{1}{e}
\end{align*}

\end{example}

\begin{problem}(Problem 1a)
Find the arc length of the spiral helix 
\[
\vec r(t) = \vector{\cos(t), \sin(t), t}
\]
from $ t= 0$ to $t = 1$.
\[
L = \answer{\sqrt 2}
\]
\end{problem}


\begin{problem}(Problem 1b)
Find the arc length of the space curve given by
\[
\vec r(t) = \vector{t\cos(t), t\sin(t), \frac{2\sqrt 2}{3} t^{3/2}}
\]
from $ t= 0$ to $t = 4$.
\[
L = \answer{12}
\]
\end{problem}


\begin{problem}(Problem 1c)
Find the arc length of the space curve given by
\[
\vec r(t) = \vector{\frac92 t^2, t\sqrt 6, \ln(2t)}
\]
from $ t= 1$ to $t = e$.
\[
L = \answer{\frac{3e^2 -1}{2}}
\]
\end{problem}




\subsection{Arc Length Parameter}

Let $\vec r(t) $ be a space curve and let 
\[
s = \int_0^t |\vec r\,'(u)| \, du
\]
Then $s$ is the arc length of the space curve $\vec r(t)$ over the interval $[0, t]$.
We refer to $s$ as the \textbf{arc length parameter}.
%If we parameterize the space curve using $s$ instead of $t$, then the arc length of $\vec r(s)$ will be exactly $s$.
%In other words, using the arc length parameter $s$, the curve will have a constant speed of $1$ unit.

The above expression for $s$ shows that $s$ is a function of $t$, i.e., $s = f(t)$.
If $|\vec r\,'(u)| \neq 0$ for all $u$, then the function $s = f(t)$ is strictly increasing and hence has an inverse (which is also strictly increasing).
Thus we can write $t = f^{-1}(s)$.

\begin{example}[Example 2]
Parameterize the spiral helix $\vec r(t) = \vector{\cos t, \sin t, t}$ using the arc length parameter.\\
First, we compute $|\vec r\,'(t)|$:
\[
|\vec r\,'(t)| = |\vector{-\sin t, \cos t, 1}| = \sqrt{\sin^2 t + \cos^2 t + 1} = \sqrt 2
\]
Computing the arc length parameter, $s$, gives
\[
s = f(t) = \int_0^t |\vec r\,'(u)| \, du = \int_0^t \sqrt 2 \, du = t \sqrt 2
\]
Hence
\[
t = f^{-1}(s) = \frac{s}{\sqrt 2}
\]
and 
\[
\vec r(t) = \vec r(f^{-1}(s)) = \vector{\cos \left(\frac{s}{\sqrt 2}\right), \sin \left(\frac{s}{\sqrt 2}\right),\frac{s}{\sqrt 2}}
\]
\end{example}


\begin{problem}[Problem 2]
Parameterize the line $\vec r(t) =\vector{1+2t, 3+4t, 5+6t}$ using the arc length parameter.
\[
\vec r(t) = \vec r(f^{-1}(s)) = \vector{\answer{1 + s/\sqrt{14}}, \answer{3 + 2s/\sqrt{14}}, \answer{5 + 3s/\sqrt{14}}}
\]
\end{problem}

\begin{proposition}
Let  $\vec r(t) = \vector{x(t), y(t), z(t)} $ be a space curve for which $\vec r\,'(t) \neq 0$ for all $t$, i.e., $\vec r(t)$ is smooth.
Define
\[
s = f(t) = \int_0^t |\vec r\,'(u)|\, du
\]
If we reparameterize the curve using the \textbf{arc length parametrization}:
\[
\vec r(f^{-1}(s)) = \vector{x(f^{-1}(s)), y(f^{-1}(s)), z(f^{-1}(s))},
\]
then
\[
\int_a^b \left|\frac{d}{du} r(f^{-1} (u))\right| \, du = b-a
\]
\end{proposition}
\begin{proof}
Let $v = f^{-1}(u)$ then $dv = \frac{d}{du}f^{-1}(u) \, du = \left|\frac{d}{du}f^{-1}(u)\right| \, du$ since the function $f^{-1}$ is increasing.
Now
\begin{align*}
\int_a^b \left|\frac{d}{du} r(f^{-1} (u))\right| \, du &= \int_a^b \left| r'(f^{-1} (u))\frac{d}{du}f^{-1} (u)\right| \, du\\
&= \int_a^b | r'(f^{-1} (u))| \cdot \left|\frac{d}{du}f^{-1} (u)\right| \, du\\
&= \int_{f^{-1}(a)}^{f^{-1}(b)} | r'(v)| \, dv\\
&= \int_0^{f^{-1}(b)} | r'(v)| \, dv - \int_0^{f^{-1}(a)} | r'(v)| \, dv\\
&= f(f^{-1}(b)) - f(f^{-1}(a))\\
&= b -a
\end{align*}
as claimed.
\end{proof}

\begin{remark}
The above proposition shows that a change in the arc length parameter $s$ corresponds to the exact same change in the length of the curve $\vec r = \vec r(f^{-1}(s))$.
\end{remark}





\begin{example}[Example 3]
Verify the proposition above for the arc length parameterization of the spiral helix, $\vec r(t) = \vector{\cos t, \sin t, t}$.\\ 
Reparameterize the curve using the arc length parameter from Example 2:
\[
\vec r(f^{-1}(s)) = \vector{\cos \left(\frac{s}{\sqrt 2}\right), \sin \left(\frac{s}{\sqrt 2}\right),\frac{s}{\sqrt 2}}
\]
Computing the definite integral in the proposition gives
\begin{align*}
\int_a^b \left| \frac{d}{du} \vec r(f^{-1}(u))\right| \, du &= \int_a^b \left| \frac{d}{du} \vector{\cos \left(\frac{u}{\sqrt 2}\right), \sin \left(\frac{u}{\sqrt 2}\right),\frac{u}{\sqrt 2}}\right|\, du\\
&= \int_a^b \left| \vector{ -\frac{1}{\sqrt 2} \sin\left(\frac{u}{\sqrt 2}\right), \frac{1}{\sqrt 2} \cos\left(\frac{u}{\sqrt 2}\right), \frac{1}{\sqrt 2}}\right|\, du\\
&= \int_a^b \frac{1}{\sqrt 2} \left|\vector{-\sin\left(\frac{u}{\sqrt 2}\right),\cos\left(\frac{u}{\sqrt 2}\right),1} \right|\, du\\
&= \int_a^b \frac{1}{\sqrt 2} \left(\sqrt 2\right) \, du\\
&= \int_a^b 1 \, du\\
&= b - a
\end{align*}
which verifies the proposition.
\end{example}

\begin{problem}(Problem 3)
Verify the proposition above for the arc length paramterization of the line, 
\[
\vec r(t) = \vector{1 + 2t, 3+4t, 5+6t}
\]
\end{problem}


\begin{proposition}
Let $\vec r(t)$ be a vector valued function and let $\vec r(f^{-1}(s))$ be its arc length parameterization.
Then $\left| \frac{d}{ds} \vec r(f^{-1}(s))\right| = 1$.\\
\end{proposition}
\begin{proof}
Since 
\[
s = f(t) = \int_0^t |\vec r \,'(u)|\, du
\]
it follows from the Fundamental Theorem of Calculus, part 2, that
\[
\frac{ds}{dt} = |\vec r \,'(t)|
\]
Now, by the chain rule,
\[
 \left| \frac{d}{ds} \vec r(f^{-1}(s)) \right| = \left|\vec r\,'(f^{-1}(s)) \frac{d}{ds} f^{-1}(s)\right|
 \]
 Rewriting, using $t = f^{-1}(s)$ gives
 \begin{align*}
 &= \left|\vec r\,'(t) \frac{dt}{ds} \right|\\
 &= \left|\vec r\,'(t) \frac{1}{ds/dt} \right|\\
 &=\left|\vec r\,'(t) \frac{1}{\vec r\,'(t)} \right| \\
 &= 1
\end{align*}
as claimed.
\end{proof}

\begin{problem}(Problem 4a)
Verify the conclusion of the above proposition for the spiral helix $\vec r(t) = \vector{\cos t, \sin t, t}$.
\end{problem}

\begin{problem}(Problem 4b)
Verify the conclusion of the above proposition for the line $\vec r(t) = \vector{1+2t, 3+4t, 5+6t}$.
\end{problem}


\end{document}
