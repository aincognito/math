\documentclass[handout]{ximera}

%% You can put user macros here
%% However, you cannot make new environments



\newcommand{\ffrac}[2]{\frac{\text{\footnotesize $#1$}}{\text{\footnotesize $#2$}}}
\newcommand{\vasymptote}[2][]{
    \draw [densely dashed,#1] ({rel axis cs:0,0} -| {axis cs:#2,0}) -- ({rel axis cs:0,1} -| {axis cs:#2,0});
}


\graphicspath{{./}{firstExample/}}
\usepackage{forest}
\usepackage{amsmath}
\usepackage{amssymb}
\usepackage{array}
\usepackage[makeroom]{cancel} %% for strike outs
\usepackage{pgffor} %% required for integral for loops
\usepackage{tikz}
\usepackage{tikz-cd}
\usepackage{tkz-euclide}
\usetikzlibrary{shapes.multipart}


%\usetkzobj{all}
\tikzstyle geometryDiagrams=[ultra thick,color=blue!50!black]


\usetikzlibrary{arrows}
\tikzset{>=stealth,commutative diagrams/.cd,
  arrow style=tikz,diagrams={>=stealth}} %% cool arrow head
\tikzset{shorten <>/.style={ shorten >=#1, shorten <=#1 } } %% allows shorter vectors

\usetikzlibrary{backgrounds} %% for boxes around graphs
\usetikzlibrary{shapes,positioning}  %% Clouds and stars
\usetikzlibrary{matrix} %% for matrix
\usepgfplotslibrary{polar} %% for polar plots
\usepgfplotslibrary{fillbetween} %% to shade area between curves in TikZ



%\usepackage[width=4.375in, height=7.0in, top=1.0in, papersize={5.5in,8.5in}]{geometry}
%\usepackage[pdftex]{graphicx}
%\usepackage{tipa}
%\usepackage{txfonts}
%\usepackage{textcomp}
%\usepackage{amsthm}
%\usepackage{xy}
%\usepackage{fancyhdr}
%\usepackage{xcolor}
%\usepackage{mathtools} %% for pretty underbrace % Breaks Ximera
%\usepackage{multicol}



\newcommand{\RR}{\mathbb R}
\newcommand{\R}{\mathbb R}
\newcommand{\C}{\mathbb C}
\newcommand{\N}{\mathbb N}
\newcommand{\Z}{\mathbb Z}
\newcommand{\dis}{\displaystyle}
%\renewcommand{\d}{\,d\!}
\renewcommand{\d}{\mathop{}\!d}
\newcommand{\dd}[2][]{\frac{\d #1}{\d #2}}
\newcommand{\pp}[2][]{\frac{\partial #1}{\partial #2}}
\renewcommand{\l}{\ell}
\newcommand{\ddx}{\frac{d}{\d x}}

\newcommand{\zeroOverZero}{\ensuremath{\boldsymbol{\tfrac{0}{0}}}}
\newcommand{\inftyOverInfty}{\ensuremath{\boldsymbol{\tfrac{\infty}{\infty}}}}
\newcommand{\zeroOverInfty}{\ensuremath{\boldsymbol{\tfrac{0}{\infty}}}}
\newcommand{\zeroTimesInfty}{\ensuremath{\small\boldsymbol{0\cdot \infty}}}
\newcommand{\inftyMinusInfty}{\ensuremath{\small\boldsymbol{\infty - \infty}}}
\newcommand{\oneToInfty}{\ensuremath{\boldsymbol{1^\infty}}}
\newcommand{\zeroToZero}{\ensuremath{\boldsymbol{0^0}}}
\newcommand{\inftyToZero}{\ensuremath{\boldsymbol{\infty^0}}}


\newcommand{\numOverZero}{\ensuremath{\boldsymbol{\tfrac{\#}{0}}}}
\newcommand{\dfn}{\textbf}
%\newcommand{\unit}{\,\mathrm}
\newcommand{\unit}{\mathop{}\!\mathrm}
%\newcommand{\eval}[1]{\bigg[ #1 \bigg]}
\newcommand{\eval}[1]{ #1 \bigg|}
\newcommand{\seq}[1]{\left( #1 \right)}
\renewcommand{\epsilon}{\varepsilon}
\renewcommand{\iff}{\Leftrightarrow}

\DeclareMathOperator{\arccot}{arccot}
\DeclareMathOperator{\arcsec}{arcsec}
\DeclareMathOperator{\arccsc}{arccsc}
\DeclareMathOperator{\si}{Si}
\DeclareMathOperator{\proj}{proj}
\DeclareMathOperator{\scal}{scal}
\DeclareMathOperator{\cis}{cis}
\DeclareMathOperator{\Arg}{Arg}
%\DeclareMathOperator{\arg}{arg}
\DeclareMathOperator{\Rep}{Re}
\DeclareMathOperator{\Imp}{Im}
\DeclareMathOperator{\sech}{sech}
\DeclareMathOperator{\csch}{csch}
\DeclareMathOperator{\Log}{Log}

\newcommand{\tightoverset}[2]{% for arrow vec
  \mathop{#2}\limits^{\vbox to -.5ex{\kern-0.75ex\hbox{$#1$}\vss}}}
\newcommand{\arrowvec}{\overrightarrow}
\renewcommand{\vec}{\mathbf}
\newcommand{\veci}{{\boldsymbol{\hat{\imath}}}}
\newcommand{\vecj}{{\boldsymbol{\hat{\jmath}}}}
\newcommand{\veck}{{\boldsymbol{\hat{k}}}}
\newcommand{\vecl}{\boldsymbol{\l}}
\newcommand{\utan}{\vec{\hat{t}}}
\newcommand{\unormal}{\vec{\hat{n}}}
\newcommand{\ubinormal}{\vec{\hat{b}}}

\newcommand{\dotp}{\bullet}
\newcommand{\cross}{\boldsymbol\times}
\newcommand{\grad}{\boldsymbol\nabla}
\newcommand{\divergence}{\grad\dotp}
\newcommand{\curl}{\grad\cross}
%% Simple horiz vectors
\renewcommand{\vector}[1]{\left\langle #1\right\rangle}


\outcome{In this section we define vectors in three dimensions and study their algebraic and geometric properties.}

\title{1.3 Vectors in Space}
%Vectors are represented graphically by arrows.
%and in three dimensions we write $\vec{v} = \vector{x,y, z}$.
%The length of the arrow represents the magnitude of the vector and the arrow points in the direction of the vector.

\begin{document}

\begin{abstract}
In this section we define vectors in three dimensions and study their algebraic and geometric properties.
\end{abstract}
 
\maketitle

Just as in $\R^2$, a vector in $\R^3$ is a quantity that has both magnitude and direction.
In $\R^3$, vectors have three components rather than two:
\[
\vec{v} = \vector{x, y, z}
\]
The magnitude of a vector in $\R^3$ comes from the distance formula:
\[
|\vec{v}| = |\vector{x, y, z}| = \sqrt{x^2 + y^2 + z^2}
\]
The standard basis vectors in $\R^3$ are
\begin{align*}
\vec{i} &= \vector{1, 0, 0}\\
\vec{j} &= \vector{0, 1, 0}\\
\vec{k} &= \vector{0, 0, 1}\\
\end{align*}
These are unit vectors in the direction of the positive $x, y$ and $z$-axes respectively.
A vector in $\R^3$ can be expressed in terms of these vectors:
\[
\vec{v} = \vector{x, y, z} = x\vec{i} + y\vec{j} + z\vec{k}
\]

The zero vector in $\R^3$ is given by
\[
\vec{0} = \vector{0,0,0}
\]
has a magnitude of $0$ and is not assigned a direction.
As in $\R^2$, a vector in $\R^3$ has an initial point and a final point.  The vector is in standard position if its initial point is the origin.
Also, as in $\R^2$, a the vector with initial point $P = (x_1, y_1, z_1)$ and final point $Q = (x_2, y_2, z_2)$ is given by the difference of the coordinates:
\[
\avec{PQ} = \vector{x_2 - x_1, y_2-y_1, z_2-z_1}
\]
The magnitude of this vector is the distance between the points $P$ and $Q$
\[
|\avec{PQ}| = \sqrt{(x_2-x_1)^2 + (y_2-y_1)^2 +(z_2-z_1)^2} = d(P, Q)
\]
The operations of scalar multiplication and addition are performed analogously to those in $\R^2$.
If $\vec{v} = \vector{x, y, z}$ and if $c$ is a scalar in $\R$, then
\[
c\vec{v} = c\vector{x, y, z} = \vector{cx, cy, cz}
\]
and if $\vec{v}_1 = \vector{x_1, y_1, z_1}$ and $\vec{v}_2 = \vector{x_2, y_2, z_2}$ are vectors in $\R^3$ then
\[
\vec{v}_1 + \vec{v}_2 = \vector{x_1+x_2, y_1+y_2, z_1+z_2}
\]
The effect on magnitude of multiplication by a scalar is the same in $\R^3$ as it was in $\R^2$:
\[
|c\vec{v}| = |c| |\vec{v}|
\]
Because of this, a unit vector in the same direction as a non-zero vector $\vec{v}$ in $\R^3$ is given by
\[
\vec{u} = \frac{1}{|\vec{v}|} \vec{v}
\]
just as in $\R^2$.

Due to their component-wise computation, the vector operations of scalar multiplication and addition have some familiar properties:
\begin{align*}
&\text{Distributive Property:} & &c(\vec{v}_1 + \vec{v}_2) = c\vec{v}_1 + c\vec{v}_2   \\
& \text{Commutative Property:}& &\vec{v}_1 + \vec{v}_2 = \vec{v}_2 + \vec{v}_1 \\
& \text{Associative Property:}&  &\vec{v}_1 + (\vec{v}_2+ \vec{v}_3) =  (\vec{v}_1 + \vec{v}_2) + \vec{v}_3 \\
&\text{Identity Property:} & &\vec{v} + \vec{0} = \vec{v} \\
& \text{Additive Inverse Property:} & &\vec{v} + (-\vec{v}) = \vec{0} 
\end{align*}

\begin{example}[Example 1]
Find a unit vector in the same direction as $\vec{v} = 2 \vec{i} - 3\vec{j} + 4\vec{k}$.\\
The magnitude of $\vec{v}$ is
\[
|\vec{v}| = |\vector{2, -3, 4}| = \sqrt{4+9+16} = \sqrt{29}
\]
Hence, a unit vector in the direction of $\vec{v}$ is
\[
\vec{u} = \frac{1}{\sqrt{29}}\vec{v} = \frac{2}{\sqrt{29}}\vec{i} - \frac{3}{\sqrt{29}}\vec{j} + \frac{4}{\sqrt{29}}\vec{k}
\]
\end{example}

\begin{problem}(Problem 1)
Find each of the following:\\
a) $3\vector{2, 4, -1} - 4\vector{5, -3, 2} = \vector{\answer{-14}, \answer{24}, \answer{-11}}$\\
b) $|3\vec{i} + 4\vec{j}- 5\vec{k}| = \answer{5\sqrt{2}}$\\
c) a unit vector in the direction of $\vector{1, -2, 2} = \vector{\answer{1/3}, \answer{-2/3}, \answer{2/3}}$
\end{problem}


\end{document}


