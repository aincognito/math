\documentclass[handout]{ximera}

%% You can put user macros here
%% However, you cannot make new environments



\newcommand{\ffrac}[2]{\frac{\text{\footnotesize $#1$}}{\text{\footnotesize $#2$}}}
\newcommand{\vasymptote}[2][]{
    \draw [densely dashed,#1] ({rel axis cs:0,0} -| {axis cs:#2,0}) -- ({rel axis cs:0,1} -| {axis cs:#2,0});
}


\graphicspath{{./}{firstExample/}}
\usepackage{forest}
\usepackage{amsmath}
\usepackage{amssymb}
\usepackage{array}
\usepackage[makeroom]{cancel} %% for strike outs
\usepackage{pgffor} %% required for integral for loops
\usepackage{tikz}
\usepackage{tikz-cd}
\usepackage{tkz-euclide}
\usetikzlibrary{shapes.multipart}


%\usetkzobj{all}
\tikzstyle geometryDiagrams=[ultra thick,color=blue!50!black]


\usetikzlibrary{arrows}
\tikzset{>=stealth,commutative diagrams/.cd,
  arrow style=tikz,diagrams={>=stealth}} %% cool arrow head
\tikzset{shorten <>/.style={ shorten >=#1, shorten <=#1 } } %% allows shorter vectors

\usetikzlibrary{backgrounds} %% for boxes around graphs
\usetikzlibrary{shapes,positioning}  %% Clouds and stars
\usetikzlibrary{matrix} %% for matrix
\usepgfplotslibrary{polar} %% for polar plots
\usepgfplotslibrary{fillbetween} %% to shade area between curves in TikZ



%\usepackage[width=4.375in, height=7.0in, top=1.0in, papersize={5.5in,8.5in}]{geometry}
%\usepackage[pdftex]{graphicx}
%\usepackage{tipa}
%\usepackage{txfonts}
%\usepackage{textcomp}
%\usepackage{amsthm}
%\usepackage{xy}
%\usepackage{fancyhdr}
%\usepackage{xcolor}
%\usepackage{mathtools} %% for pretty underbrace % Breaks Ximera
%\usepackage{multicol}



\newcommand{\RR}{\mathbb R}
\newcommand{\R}{\mathbb R}
\newcommand{\C}{\mathbb C}
\newcommand{\N}{\mathbb N}
\newcommand{\Z}{\mathbb Z}
\newcommand{\dis}{\displaystyle}
%\renewcommand{\d}{\,d\!}
\renewcommand{\d}{\mathop{}\!d}
\newcommand{\dd}[2][]{\frac{\d #1}{\d #2}}
\newcommand{\pp}[2][]{\frac{\partial #1}{\partial #2}}
\renewcommand{\l}{\ell}
\newcommand{\ddx}{\frac{d}{\d x}}

\newcommand{\zeroOverZero}{\ensuremath{\boldsymbol{\tfrac{0}{0}}}}
\newcommand{\inftyOverInfty}{\ensuremath{\boldsymbol{\tfrac{\infty}{\infty}}}}
\newcommand{\zeroOverInfty}{\ensuremath{\boldsymbol{\tfrac{0}{\infty}}}}
\newcommand{\zeroTimesInfty}{\ensuremath{\small\boldsymbol{0\cdot \infty}}}
\newcommand{\inftyMinusInfty}{\ensuremath{\small\boldsymbol{\infty - \infty}}}
\newcommand{\oneToInfty}{\ensuremath{\boldsymbol{1^\infty}}}
\newcommand{\zeroToZero}{\ensuremath{\boldsymbol{0^0}}}
\newcommand{\inftyToZero}{\ensuremath{\boldsymbol{\infty^0}}}


\newcommand{\numOverZero}{\ensuremath{\boldsymbol{\tfrac{\#}{0}}}}
\newcommand{\dfn}{\textbf}
%\newcommand{\unit}{\,\mathrm}
\newcommand{\unit}{\mathop{}\!\mathrm}
%\newcommand{\eval}[1]{\bigg[ #1 \bigg]}
\newcommand{\eval}[1]{ #1 \bigg|}
\newcommand{\seq}[1]{\left( #1 \right)}
\renewcommand{\epsilon}{\varepsilon}
\renewcommand{\iff}{\Leftrightarrow}

\DeclareMathOperator{\arccot}{arccot}
\DeclareMathOperator{\arcsec}{arcsec}
\DeclareMathOperator{\arccsc}{arccsc}
\DeclareMathOperator{\si}{Si}
\DeclareMathOperator{\proj}{proj}
\DeclareMathOperator{\scal}{scal}
\DeclareMathOperator{\cis}{cis}
\DeclareMathOperator{\Arg}{Arg}
%\DeclareMathOperator{\arg}{arg}
\DeclareMathOperator{\Rep}{Re}
\DeclareMathOperator{\Imp}{Im}
\DeclareMathOperator{\sech}{sech}
\DeclareMathOperator{\csch}{csch}
\DeclareMathOperator{\Log}{Log}

\newcommand{\tightoverset}[2]{% for arrow vec
  \mathop{#2}\limits^{\vbox to -.5ex{\kern-0.75ex\hbox{$#1$}\vss}}}
\newcommand{\arrowvec}{\overrightarrow}
\renewcommand{\vec}{\mathbf}
\newcommand{\veci}{{\boldsymbol{\hat{\imath}}}}
\newcommand{\vecj}{{\boldsymbol{\hat{\jmath}}}}
\newcommand{\veck}{{\boldsymbol{\hat{k}}}}
\newcommand{\vecl}{\boldsymbol{\l}}
\newcommand{\utan}{\vec{\hat{t}}}
\newcommand{\unormal}{\vec{\hat{n}}}
\newcommand{\ubinormal}{\vec{\hat{b}}}

\newcommand{\dotp}{\bullet}
\newcommand{\cross}{\boldsymbol\times}
\newcommand{\grad}{\boldsymbol\nabla}
\newcommand{\divergence}{\grad\dotp}
\newcommand{\curl}{\grad\cross}
%% Simple horiz vectors
\renewcommand{\vector}[1]{\left\langle #1\right\rangle}


\outcome{In this section we compute double integral over various regions.}

\title{4.2 More Double Integrals}



\begin{document}

\begin{abstract}
In this section we compute double integrals over various regions.
\end{abstract}
 
\maketitle




\begin{example}[Example 1]
Compute $\iint_R xy \, dA$ where $R$ is the triangular region bounded by the lines $x = 0, y = 0$ and $y = 4-2x$\\
\begin{image}
\begin{tikzpicture}
\filldraw[blue!30!white] (0,0) -- (2,0) -- (0,4)-- (0,0);\draw[thick, <->] (-0.75, 0) -- (3, 0) node[right]{$x$};
\draw[thick, <->] (0,-.75) -- (0, 5) node[above]{$y$};
\draw[thick] (0,4) node[left]{$4$} -- (2,0) node[below]{$2$};
\node at (2, 2.3) {$y = 4-2x$};
\node at (1.1, -1.25) {The triangular region, $R$};

\end{tikzpicture}
\end{image}

The function $f(x,y) = xy$ is continuous the region $R$, so we can use Fubini's theorem and write the double integral as iterated integrals.
The issue is how the triangular region affects the endpoints of integration. We will examine the situation for both orders of integration.
We will first integrate $dy\, dx$. When computing the integral $dy$, we treat the variable $x$ as constant. 
The possible $x$-values in the triangular region are $0 \leq x \leq 2$.  
For each $x$ between $0$ and $2$, the values of $y$ will range from $0$ up to $4 - 2x$ which is the value of $y$ on the given line.
Hence, the double integral can be computed as
\begin{align*}
\iint_R xy \, dA & = \int_0^2 \int_0^{4 - 2x} xy \, dy \, dx\\
                 & = \int_0^2 \left(\frac12 xy^2 \right) \bigg|_{0}^{4 - 2x} \, dx\\
                 & = \int_0^2 \frac12 x (4-2x)^2 \, dx\\
                 &=  \int_0^2 \left(8x - 8x^2 + 2x^3\right) \, dx\\
                 &= \left(4x^2 - \frac83 x^3 + \frac12 x^4 \right) \bigg|_0^2\\
                 &= 16 - \frac{64}{3} + 8 = \frac83
\end{align*}      
\begin{image}
\begin{tikzpicture}
\filldraw[blue!30!white] (0,0) -- (2,0) -- (0,4)-- (0,0);\draw[thick, <->] (-.75, 0) -- (3, 0) node[right]{$x$};
\draw[thick, <->] (0,-.75) -- (0, 5) node[above]{$y$};
\draw[thick] (0,4) node[left]{$4$} -- (2,0) node[below]{$2$};
\draw[thick] (1, 0) node[below]{$x$} -- (1, 2) node[above, right]{$y = 4 - 2x$};
\node at (1.1, -1.25) {For each $x$ between $0$ and $2$,};
\node at (1.1, -1.75) {$y$ varies from $0$ to $4 - 2x$};
\end{tikzpicture}
\end{image}


We will now integrate in the other order, $dx \, dy$. To do so, it is helpful to rewrite the equation of the line $y = 4 - 2x$ in terms of $x$, which gives
$x = 4 - \frac{y}{2}$. 
We will compute the integral $dx$ first, treating the variable $y$ as constant. 
The possible $y$-values in the triangular region are $0 \leq y \leq 4$.  
For each $y$ between $0$ and $4$, the values of $x$ will range from $0$ up to $2 - \frac{y}{2}$ which is the value of $x$ on the given line.
Hence, the double integral can be computed as
\begin{align*}
\iint_R xy \, dA & = \int_0^4 \int_0^{2 - \frac{y}{2}} xy \, dx \, dy\\
                 & = \int_0^4 \left(\frac12 x^2y \right) \bigg|_{0}^{2 - \frac{y}{2}} \, dy\\
                 & = \int_0^4 \frac12 y \left(2 - \frac{y}{2}\right)^2 \, dy\\
                 &= \int_0^4 \left(2y - y^2 + \frac18 y^3\right) \, dy\\
                 &= \left(y^2 - \frac13 y^3 + \frac{1}{32} y^4 \right) \bigg|_0^4\\
                 &= 16 - \frac{64}{3} + 8 = \frac83
\end{align*}      

\begin{image}
\begin{tikzpicture}
\filldraw[blue!30!white] (0,0) -- (2,0) -- (0,4)-- (0,0);
\draw[thick, <->] (-0.75, 0) -- (3, 0) node[right]{$x$};
\draw[thick, <->] (0,-.75) -- (0, 5) node[above]{$y$};
\draw[thick] (0,4) node[left]{$4$} -- (2,0) node[below]{$2$};
\draw[thick] (0, 2) node[left]{$y$} -- (1, 2) node[above, right]{$x = 2 - \frac{y}{2}$};
\node at (1.1, -1.25) {For each $y$ between $0$ and $4$,};
\node at (1.1, -1.75) {$x$ varies from $0$ to $2 - \frac{y}{2}$};
\end{tikzpicture}
\end{image}

\end{example}


\begin{problem}(Problem 1)
Compute $\iint_R xy \, dA$ where $R$ is the triangular region bounded by the lines $x = 0, y = 0$ and $y = 3-x$\\
\[
\iint_R xy \, dA = \answer{27/8}
\]
\end{problem}

\begin{example}[Example 2]
Compute $\iint_R x \, dA$ where $R$ is the quarter circle in the first quadrant bounded by $x=0, y=0$ and $x^2 + y^2 = 4$

\begin{image}
\begin{tikzpicture}
\filldraw[blue!30!white] (0,0) -- (3,0) arc (0: 90: 3) -- cycle;
\draw[thick] (3,0) node[below]{$2$} arc (0: 90: 3) node[left]{$2$};
\draw[thick, <->] (-0.75, 0) -- (4, 0) node[right]{$x$};
\draw[thick, <->] (0,-.75) -- (0, 4) node[above]{$y$};
\draw[thick] (1.8,0) node[below]{$x$} -- (1.8, 2.4) node[above right]{$y = \sqrt{4 - x^2}$};
\node at (1.7, -1.25) {For each $x$ between $0$ and $2$,};
\node at (1.7, -1.75) {$y$ varies from $0$ to $\sqrt{4 - x^2}$};
\end{tikzpicture}
\end{image}
We first integrate $dy\, dx$.  From the figure above, we can determine the limits of integration on the iterated integrals.
The computation requires a $u$-substitution, with $u = 4-x^2, \, du = -2x \, dx$
\begin{align*}
\iint_R x \, dA & = \int_0^2 \int_0^{\sqrt{4 - x^2}} x \, dy \, dx\\
                 & = \int_0^2  xy \bigg|_{0}^{\sqrt{4 - x^2}} \, dx\\
                 & = \int_0^2  x \sqrt{4-x^2}\, dx\\
                 &= -\frac12 \int_4^0  \sqrt u \, du\\
                 &= \frac12 \int_0^4 \sqrt u \, du\\
                 &= \frac13 u^{3/2}  \bigg|_0^4\\
                 &= \frac13 \cdot 8 = \frac{8}{3}
\end{align*}  


\begin{image}
\begin{tikzpicture}
\filldraw[blue!30!white] (0,0) -- (3,0) arc (0: 90: 3) -- cycle;
\draw[thick] (3,0) node[below]{$2$} arc (0: 90: 3) node[left]{$2$};
\draw[thick, <->] (-0.75, 0) -- (4, 0) node[right]{$x$};
\draw[thick, <->] (0,-.75) -- (0, 4) node[above]{$y$};
\draw[thick] (0, 1.8) node[left]{$y$} -- (2.4, 1.8) node[above right]{$x = \sqrt{4 - y^2}$};
\node at (1.7, -1.25) {For each $y$ between $0$ and $2$,};
\node at (1.7, -1.75) {$x$ varies from $0$ to $\sqrt{4 - y^2}$};
\end{tikzpicture}
\end{image}
We next integrate $dx\, dy$.  From the figure above, we can determine the limits of integration on the iterated integrals.

\begin{align*}
\iint_R x \, dA & = \int_0^2 \int_0^{\sqrt{4 - y^2}} x \, dx \, dy\\
                 & = \int_0^2  \frac{x^2}{2} \bigg|_{0}^{\sqrt{4 - y^2}} \, dy\\
                 & = \int_0^2  \frac12(4-y^2) \, dy\\
                 &= \frac12 \left(4y - \frac{y^3}{3}\right)  \bigg|_0^2\\
                 &= \frac12 \left(8 - \frac83\right) = \frac{8}{3}
\end{align*} 


\end{example}

\begin{problem}(Problem 2)
Compute $\iint_R y \, dA$ where $R$ is the semicircular region in the upper-half plane bounded by $y=0$ and $x^2 + y^2 = 9$
\[
\iint_R y \, dA = \answer{18}
\]
\end{problem}


\begin{example}[Example 3]
Compute $\iint_R y^2 e^{xy} \, dA$ where $R$ is the triangle bounded by $y = x, y=4$ and $x=0$.\\

\begin{image}
\begin{tikzpicture}
\filldraw[blue!30!white] (0,0) -- (0,4) node[black,left]{$4$} -- (4,4) -- cycle;
\draw[thick] (0,0) -- (4,4) -- (0,4);
\draw[thick, <->] (-0.75, 0) -- (4, 0) node[right]{$x$};
\draw[thick, <->] (0,-.75) -- (0, 5) node[above]{$y$};
\draw[thick] (0, 2.25) node[left]{$y$} -- (2.25, 2.25) node[below right]{$x = y$};
\node at (1.7, -1.25) {For each $y$ between $0$ and $4$,};
\node at (1.7, -1.75) {$x$ varies from $0$ to $y$};
\end{tikzpicture}
\end{image}

To avoid using integration by parts (twice), we integrate $dx\, dy$.  
The figure above is used to determine the limits of integration on the iterated integrals.
\begin{align*}
\iint_R y^2 e^{xy} \, dA & = \int_0^4 \int_0^{y} y^2 e^{xy} \, dx \, dy\\
                 & = \int_0^4 y e^{xy} \bigg|_{0}^{y} \, dy\\
                 & = \int_0^4  \left(ye^{y^2} - y \right)\, dy\\
                 &= \frac12 \left(e^{y^2} - y^2\right) \bigg|_0^4\\
                 &= \frac12 \left(e^{16} - 17\right)
\end{align*} 

\end{example}

\begin{problem}(Problem 3)
Compute $\iint_R y^3 \cos(xy) \, dA$ where $R$ is the region bounded by $y = \sqrt x, y = 1$ and $x = 0$.\\
\[
\iint_R y^3 \cos(xy) \, dA = \answer{1/3 - 1/3 \cos(1)}
\]
\end{problem}

\begin{example}[Example 4]
Compute $\iint_R x \, dA$ where $R$ is region bounded by $y = x^2 -x$ and $y = x+ 3$.\\

\begin{image}
\begin{tikzpicture}
\draw[color = blue!30!white, fill = blue!30!white] (.5, -0.25) parabola (3,6) -- (0.5, 3.5) --cycle;
\draw[color = blue!30!white, fill = blue!30!white] (.5, -0.25) parabola (-1,2) -- (0.5, 3.5) --cycle;
\draw[thick, <->] (-1.75, 0) -- (4, 0) node[right]{$x$};
\draw[thick, <->] (0,-.75) -- (0, 6) node[above]{$y$};
\draw[thick] (0.5, -0.25) parabola (3, 6);
%\shade[blue!30!white] (0.5, -0.25) parabola (-1,2);
%\shade[blue!30!white] (0.5, -0.25) parabola (3,6);
\draw[thick] (0.5, -0.25) parabola (-1, 2);
\draw[thick] (-1.5, 1.5) -- (3.5,6.5) ;
\node at (1, 5){$y = x+3$};
\node at (2.5, 0.75){$y = x^2 - x$};
\draw[thick] (1.5, 4.5) -- (1.5, 0.75);
\draw[thin] (1.5, 0.15) -- (1.5, -0.15) node[below] {$x$};
\draw[thin] (-1, 0.15) -- (-1, -0.15) node[below] {$-1$};
\draw[thin] (3, 0.15) -- (3, -0.15) node[below] {$3$};
\node at (1.25, -1.25) {For each $x$ between $-1$ and $3$,};
\node at (1.25, -1.75) {$y$ varies from $x^2 - x$ to $x+ 3$};
\end{tikzpicture}
\end{image}

The nature of this region requires us to integrate $dy\, dx$.  
From the figure above, we can determine the limits of integration on the iterated integrals.
The computation requires a $u$-substitution, with $u = 4-x^2, \, du = -2x \, dx$
\begin{align*}
\iint_R x \, dA & = \int_{-1}^3 \int_{x^2 - x}^{x+3} x \, dy \, dx\\
                 & = \int_{-1}^3  xy \bigg|_{x^2 - x}^{x+3} \, dx\\
                 & = \int_{-1}^3  x\left[(x+3)-(x^2-x)\right]\, dx\\
                 &= \int_{-1}^3 \left(3x+ 2x^2  -x^3\right) \, dx\\
                 &= \left(\frac{3x^2}{2} + \frac{2x^3}{3} - \frac{x^4}{4}\right) \bigg|_{-1}^3\\
                 &= \left(\frac{27}{2} + 18 - \frac{81}{4} \right) - \left( \frac32 - \frac23 - \frac14 \right)\\
                 &= \frac{32}{3}
\end{align*}  

\end{example}


\begin{problem}(Problem 4)
Compute $\iint_R x \, dA$ where $R$ is region bounded by $y = x^2 -4x$ and $y = 2x$.\\
\[
\iint_R x \, dA = \answer{108}
\]
\end{problem}


\begin{example}[Example 5]
Compute $\iint_R y \, dA$ where $R$ is the triangle with vertices $(0,0), (1,1)$ and $(4,0)$.\\

\begin{image}
\begin{tikzpicture}
\draw[black, fill = blue!30!white, thick] (0,0) -- (4,0) -- (1,1) -- cycle;
\draw[thick, <->] (-0.75, 0) -- (5, 0) node[right]{$x$};
\draw[thick, <->] (0,-.5) -- (0, 1.5) node[above]{$y$};
\draw[thin] (0.15, 0.4) -- (-0.15, 0.4) node[left]{$y$};
\draw[thin] (0.15, 1) -- (-0.15, 1) node[left]{$1$};
\draw[thin] (4, 0.15) -- (4, -0.15) node[below]{$4$};
\node at (1, 1.5) {$x = y$};
\draw[thick] (0.4, 0.4) -- (2.8, 0.4) node[above right]{$x = 4-3y$};
\draw[thin, ->] (0.8, 1.4) ..controls (0.5, 1.1)..  (0.65, 0.75);
\node at (1.7, -1.25) {For each $y$ between $0$ and $1$,};
\node at (1.7, -1.75) {$x$ varies from $y$ to $4-3y$};
\end{tikzpicture}
\end{image}

\begin{align*}
\iint_R x \, dA & = \int_0^1 \int_{y}^{4-3y} y \, dx \, dy\\
                 & = \int_0^1   xy \bigg|_{y}^{4-3y} \, dy\\
                 & = \int_0^1   y\left[(4-3y) -y\right] \, dy\\
                 &= \int_0^1 \left(4y - 4y^2\right) \, dy\\
                 &= \left(2y^2 - \frac{4y^3}{3}\right) \bigg|_0^1\\
                 &= 2- \frac43 = \frac23
\end{align*} 

\end{example}

\begin{problem}(Problem 5)
Compute $\iint_R y \, dA$ where $R$ is triangle with vertices $(0,0), (0, 3)$ and $(2,2)$.\\
\[
\iint_R x \, dA = \answer{5}
\]
\end{problem}

\begin{example}[Example 6]
Compute $\int_0^2 \int_{y/2}^1 y\cos(x^3 -1) \,dx \, dy$ by changing the order of integration.\\

\begin{image}
\begin{tikzpicture}
\draw[black, fill = blue!30!white, thick] (0,0) -- (2,4) -- (2,0) -- cycle;
\draw[thick, <->] (-0.75, 0) -- (3, 0) node[right]{$x$};
\draw[thick, <->] (0,-.5) -- (0, 4.5) node[above]{$y$};
\draw[thin] (0.15, 1.5) -- (-0.15, 1.5) node[left]{$y$};
\draw[thin] (0.15, 4) -- (-0.15, 4) node[left]{$2$};
\draw[thin] (2, 0.15) -- (2, -0.15) node[below]{$1$};
\draw[thick] (0.75, 1.5)  -- (2, 1.5);
\node at (1, 3.5) {$x = y/2$};
\draw[thin, ->] (0.8, 3.4) ..controls (0.5, 2.5)..  (0.75, 1.6);
\node at (1.4, -1.25) {For each $y$ between $0$ and $2$,};
\node at (1.4, -1.75) {$x$ varies from $y/2$ to $1$};


\draw[black, fill = blue!30!white, thick] (5,0) -- (7,4) -- (7,0) -- cycle;
\draw[thick, <->] (4.25, 0) -- (8, 0) node[right]{$x$};
\draw[thick, <->] (5,-.5) -- (5, 4.5) node[above]{$y$};
\draw[thin] (5.15, 4) -- (4.85, 4) node[left]{$2$};
\draw[thin] (7, 0.15) -- (7, -0.15) node[below]{$1$};
\draw[thick] (6.25, 0) node[below]{$x$} -- (6.25, 2.5) ;
\node at (6, 3.25) {$y = 2x$};
\node at (6.4, -1.25) {For each $x$ between $0$ and $1$,};
\node at (6.4, -1.75) {$y$ varies from $0$ to $2x$};




\end{tikzpicture}
\end{image}

\begin{align*}
\iint_R x \, dA & = \int_0^1 \int_0^{2x} y \cos(x^3-1) \, dy \, dx\\
                 & = \int_0^1  \frac{y^2}{2} \cos (x^3 - 1) \bigg|_0^{2x} \, dx\\
                 & = \int_0^1  2x^2 \cos(x^3 -1) \, dx\\
                 &= \frac23 \int_{-1}^0 \cos(u) \, du\\
                 &= \frac23 \sin(u) \bigg|_{-1}^0\\
                 &= \frac{2}{3}\sin(1)
\end{align*}  

\end{example}

\begin{problem}(Problem 6)
Compute $\displaystyle \int_0^1 \int_{x^2}^1 \sqrt y \sin(y) \,dy \, dx$ by changing the order of integration.\\
\begin{hint}
The second integral requires integration by parts: $\int u\, dv = uv - \int v \, du$
\end{hint}
\[
\int_0^1 \int_{x^2}^1 \sqrt y \sin(y) \,dy \, dx = \answer{\sin(1) - \cos(1)}
\]
\end{problem}

\end{document}
