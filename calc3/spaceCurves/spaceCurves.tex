\documentclass[handout]{ximera}

%% You can put user macros here
%% However, you cannot make new environments



\newcommand{\ffrac}[2]{\frac{\text{\footnotesize $#1$}}{\text{\footnotesize $#2$}}}
\newcommand{\vasymptote}[2][]{
    \draw [densely dashed,#1] ({rel axis cs:0,0} -| {axis cs:#2,0}) -- ({rel axis cs:0,1} -| {axis cs:#2,0});
}


\graphicspath{{./}{firstExample/}}
\usepackage{forest}
\usepackage{amsmath}
\usepackage{amssymb}
\usepackage{array}
\usepackage[makeroom]{cancel} %% for strike outs
\usepackage{pgffor} %% required for integral for loops
\usepackage{tikz}
\usepackage{tikz-cd}
\usepackage{tkz-euclide}
\usetikzlibrary{shapes.multipart}


%\usetkzobj{all}
\tikzstyle geometryDiagrams=[ultra thick,color=blue!50!black]


\usetikzlibrary{arrows}
\tikzset{>=stealth,commutative diagrams/.cd,
  arrow style=tikz,diagrams={>=stealth}} %% cool arrow head
\tikzset{shorten <>/.style={ shorten >=#1, shorten <=#1 } } %% allows shorter vectors

\usetikzlibrary{backgrounds} %% for boxes around graphs
\usetikzlibrary{shapes,positioning}  %% Clouds and stars
\usetikzlibrary{matrix} %% for matrix
\usepgfplotslibrary{polar} %% for polar plots
\usepgfplotslibrary{fillbetween} %% to shade area between curves in TikZ



%\usepackage[width=4.375in, height=7.0in, top=1.0in, papersize={5.5in,8.5in}]{geometry}
%\usepackage[pdftex]{graphicx}
%\usepackage{tipa}
%\usepackage{txfonts}
%\usepackage{textcomp}
%\usepackage{amsthm}
%\usepackage{xy}
%\usepackage{fancyhdr}
%\usepackage{xcolor}
%\usepackage{mathtools} %% for pretty underbrace % Breaks Ximera
%\usepackage{multicol}



\newcommand{\RR}{\mathbb R}
\newcommand{\R}{\mathbb R}
\newcommand{\C}{\mathbb C}
\newcommand{\N}{\mathbb N}
\newcommand{\Z}{\mathbb Z}
\newcommand{\dis}{\displaystyle}
%\renewcommand{\d}{\,d\!}
\renewcommand{\d}{\mathop{}\!d}
\newcommand{\dd}[2][]{\frac{\d #1}{\d #2}}
\newcommand{\pp}[2][]{\frac{\partial #1}{\partial #2}}
\renewcommand{\l}{\ell}
\newcommand{\ddx}{\frac{d}{\d x}}

\newcommand{\zeroOverZero}{\ensuremath{\boldsymbol{\tfrac{0}{0}}}}
\newcommand{\inftyOverInfty}{\ensuremath{\boldsymbol{\tfrac{\infty}{\infty}}}}
\newcommand{\zeroOverInfty}{\ensuremath{\boldsymbol{\tfrac{0}{\infty}}}}
\newcommand{\zeroTimesInfty}{\ensuremath{\small\boldsymbol{0\cdot \infty}}}
\newcommand{\inftyMinusInfty}{\ensuremath{\small\boldsymbol{\infty - \infty}}}
\newcommand{\oneToInfty}{\ensuremath{\boldsymbol{1^\infty}}}
\newcommand{\zeroToZero}{\ensuremath{\boldsymbol{0^0}}}
\newcommand{\inftyToZero}{\ensuremath{\boldsymbol{\infty^0}}}


\newcommand{\numOverZero}{\ensuremath{\boldsymbol{\tfrac{\#}{0}}}}
\newcommand{\dfn}{\textbf}
%\newcommand{\unit}{\,\mathrm}
\newcommand{\unit}{\mathop{}\!\mathrm}
%\newcommand{\eval}[1]{\bigg[ #1 \bigg]}
\newcommand{\eval}[1]{ #1 \bigg|}
\newcommand{\seq}[1]{\left( #1 \right)}
\renewcommand{\epsilon}{\varepsilon}
\renewcommand{\iff}{\Leftrightarrow}

\DeclareMathOperator{\arccot}{arccot}
\DeclareMathOperator{\arcsec}{arcsec}
\DeclareMathOperator{\arccsc}{arccsc}
\DeclareMathOperator{\si}{Si}
\DeclareMathOperator{\proj}{proj}
\DeclareMathOperator{\scal}{scal}
\DeclareMathOperator{\cis}{cis}
\DeclareMathOperator{\Arg}{Arg}
%\DeclareMathOperator{\arg}{arg}
\DeclareMathOperator{\Rep}{Re}
\DeclareMathOperator{\Imp}{Im}
\DeclareMathOperator{\sech}{sech}
\DeclareMathOperator{\csch}{csch}
\DeclareMathOperator{\Log}{Log}

\newcommand{\tightoverset}[2]{% for arrow vec
  \mathop{#2}\limits^{\vbox to -.5ex{\kern-0.75ex\hbox{$#1$}\vss}}}
\newcommand{\arrowvec}{\overrightarrow}
\renewcommand{\vec}{\mathbf}
\newcommand{\veci}{{\boldsymbol{\hat{\imath}}}}
\newcommand{\vecj}{{\boldsymbol{\hat{\jmath}}}}
\newcommand{\veck}{{\boldsymbol{\hat{k}}}}
\newcommand{\vecl}{\boldsymbol{\l}}
\newcommand{\utan}{\vec{\hat{t}}}
\newcommand{\unormal}{\vec{\hat{n}}}
\newcommand{\ubinormal}{\vec{\hat{b}}}

\newcommand{\dotp}{\bullet}
\newcommand{\cross}{\boldsymbol\times}
\newcommand{\grad}{\boldsymbol\nabla}
\newcommand{\divergence}{\grad\dotp}
\newcommand{\curl}{\grad\cross}
%% Simple horiz vectors
\renewcommand{\vector}[1]{\left\langle #1\right\rangle}


\outcome{Describe a curve in space.}

\title{2.1 Space Curves}



\begin{document}

\begin{abstract}
In this section we create the describe curves in space.
\end{abstract}

\maketitle


A curve in space can be described using a \textbf{vector-valued function}:
\[
\vec r(t) = \vector{f(t), g(t), h(t)}
\]

\begin{example}[Example 1]
Express the line through the point $(3, -4, 1)$ with direction vector $\vector{2, 5, -8}$ using a vector-valued function.\\
The vector equation of the line is 
\[
\vector{x, y, z} = \vector{3, -4, 1} + t\vector{2, 5, -8}
\]
Distributing the parameter $t$ and adding the vectors on the right hand side, we have
\[
\vector{x, y, z} = \vector{3+2t,-4+ 5t, 1-8t}
\]
Renaming the vector on the left hand side as $\vec r(t)$ we have
\[
\vec r(t) = \vector{3+2t,-4+ 5t, 1-8t}
\]
which is a vector-valued function.

\end{example}

\begin{remark}

In general, a line with vector equation
\[
\vector{x, y, z} = \vector{x_0, y_0, z_0} + t\vector{a, b, c}
\]
can be expressed as a vector-valued function as follows:
\[
\vec r(t) = \vector{x_0 + at, y_0 + bt, z_0 + ct}
\]
\end{remark}

\begin{problem}(Problem 1a)
Express the line through the point $(6, 7, 2)$ with direction vector $\vector{9, -3, 5}$ as a vector-valued function.
\end{problem}

Here is a video solution of problem 1a:\\
\begin{foldable}
\youtube{h1bi7oLXd80}
\end{foldable}

\begin{problem}(Problem 1b)
Identify the space curve given by the vector-valued function:
\[
\vec r(t) = \vector{3 -t, -2 + 2t, 7t}
\]
\end{problem}  
Here is a video solution of problem 1b:\\
\begin{foldable}
\youtube{N8lwvCmYtDw}
\end{foldable}

The domain of a space curve is the largest subset of $\R$ for which each of the component functions are defined.

\begin{example}[Example 2]
Find the domain of the space curve
\[
\vec r(t) = \vector{\frac{1}{t}, \sqrt t, \ln(1-t)}
\]
 The domain of $f(t) = 1/t$ is $t \neq 0$; the domain of $g(t) = \sqrt t$ is $t \geq 0$ and the domain
 of $h(t) = \ln(t)$ is $1-t > 0$ or $t<1$.\\
 The domain of $\vec r(t)$ must satisfy all three conditions:
 \[
 t\neq 0, \quad t \geq 0, \quad \text{and} \quad t <1
 \]
 Thus the domain of $\vec r(t)$ is the interval $(0, 1)$.
 \end{example}
 
\begin{problem}(Problem 2a)
Find the domain of the space curve
\[
\vec r(t) = \vector{\frac{1}{t^2 - 4}, \sqrt{2- t}, \ln (t)}
\]
\end{problem}

\begin{problem}(Problem 2b)
Find the domain of the space curve
\[
\vec r(t) = \vector{\tan(t), \cot(t), \frac{1}{1+t^2}}
\]
\end{problem}


The \textbf{orientation} of a space curve is in the direction of increasing values of the parameter $t$.



A line segment from the point $P(x_1, y_1, z_1)$ to the point $Q(x_2, y_2, z_2)$ is a space curve that can be written as
\begin{align*}
\vec r(t) &= \vec P + t \arrowvec{PQ}, (\vec P \text{is the position vector of the point} \;P)\\
          &=\vector{x_1, y_1, z_1} + t\vector{x_2-x_1, y_2-y_1, z_2-z_1} \\
          &= (1-t)\vector{x_1, y_1, z_1} + t\vector{x_2, y_2, z_2}, 
\end{align*}
where $\; 0 \leq t \leq 1$.\\
If we write the vector from the origin to $P$ as $\vec P$ and similarly for $\vec Q$, then the line segment has the elegant form 
\[
\vec r(t) = (1-t) \vec P + t \vec Q, \quad 0 \leq t \leq 1
\]
\begin{image}
\begin{tikzpicture}
\draw[thick, blue!70!white] (0,0) -- (8,4);
\draw[thick, Circle->, blue!70!white] (0,0) node[left]{$P$} -- (2, 1);
\draw[thick, ->, blue!70!white] (2,1) -- (4, 2) ;
\draw[thick, ->, blue!70!white] (4, 2) node[below, rotate = 26.6, black]{orientation arrows point in the direction of increasing $t$} -- (6, 3);
\draw[thick, -Circle, blue!70!white] (6,3) -- (8,4) node[right]{$Q$};
\node at (4, -1) {The line segment from $P$ to $Q$};
\node[rotate = 26.6] at (4, 2.7) {$\vec r(t) = (1-t)\vec P + t \vec Q, 0 \leq t \leq 1$};
\end{tikzpicture}
\end{image}

\begin{example}[Example 3]
Express the line segment from the point $P(1, 2, -1)$ to the point $Q(-2, -2, 1)$ as a vector-valued function.  
Include the domain. Sketch the line segment with orientation arrows.\\
The vector form of the line segment is
\begin{align*}
    \vec r(t) &= (1-t)\vector{1, 2, -1} + t \vector{-2, -2, 1}\\
              &= \vector{1-t, 2-2t, -1+t} + \vector{-2t, -2t, t}\\
              &= \vector{1-3t, 2-4t, -1+2t}
\end{align*}
and its domain is $0 \leq t \leq 1$. The sketch is below.




\begin{image}
\begin{tikzpicture}
\draw[thick, ->] (0,0) -- (2.2,0) node[right]{$y$};
\draw[thick, ->] (0,0) -- (0,2) node[above]{$z$};
\draw[thick, ->] (0,0) -- (-1.6,-1.2) node[below, left]{$x$};
%\node at (0.1, -0.2){$O$} ;
\draw[thick, blue!70!white, Circle-Circle] (2, -1.5) node[right]{$P$} -- (-1, 2) node[left]{$Q$};
\draw[->, blue!70!white] (2, -1.5) -- (0.5, 0.25);
\draw[->, blue!70!white] (2, -1.5) -- (1.25, -0.625);
\draw[->, blue!70!white] (2, -1.5) -- (-.25, 1.125);
\end{tikzpicture}
\end{image}


\end{example}



\begin{problem}(Problem 3)
Express the line segment from the point $P(-1, -1, 2)$ to the point $Q(2, 3, -1)$ as a vector-valued function.  
Include the domain. Sketch the line segment with orientation arrows.
\end{problem}

Here is a video solution of problem 3:\\
\begin{foldable}
\youtube{l2aKZEo8-k0}
\end{foldable}

Other interesting space curves include the \textbf{helix}
\[
\vec r(t) = \vector{\cos t, \sin t, t}
\]
and the \textbf{twisted cubic}
\[
\vec r(t) = \vector{t, t^2, t^3}
\]

Plot both of these space curves using this online 
\link[graph plotter]{https://math.libretexts.org/Learning_Objects/CalcPlot3D_Interactive_Figures/CalcPlot3D}

\end{document}

