\documentclass[handout]{ximera}

%% You can put user macros here
%% However, you cannot make new environments



\newcommand{\ffrac}[2]{\frac{\text{\footnotesize $#1$}}{\text{\footnotesize $#2$}}}
\newcommand{\vasymptote}[2][]{
    \draw [densely dashed,#1] ({rel axis cs:0,0} -| {axis cs:#2,0}) -- ({rel axis cs:0,1} -| {axis cs:#2,0});
}


\graphicspath{{./}{firstExample/}}
\usepackage{forest}
\usepackage{amsmath}
\usepackage{amssymb}
\usepackage{array}
\usepackage[makeroom]{cancel} %% for strike outs
\usepackage{pgffor} %% required for integral for loops
\usepackage{tikz}
\usepackage{tikz-cd}
\usepackage{tkz-euclide}
\usetikzlibrary{shapes.multipart}


%\usetkzobj{all}
\tikzstyle geometryDiagrams=[ultra thick,color=blue!50!black]


\usetikzlibrary{arrows}
\tikzset{>=stealth,commutative diagrams/.cd,
  arrow style=tikz,diagrams={>=stealth}} %% cool arrow head
\tikzset{shorten <>/.style={ shorten >=#1, shorten <=#1 } } %% allows shorter vectors

\usetikzlibrary{backgrounds} %% for boxes around graphs
\usetikzlibrary{shapes,positioning}  %% Clouds and stars
\usetikzlibrary{matrix} %% for matrix
\usepgfplotslibrary{polar} %% for polar plots
\usepgfplotslibrary{fillbetween} %% to shade area between curves in TikZ



%\usepackage[width=4.375in, height=7.0in, top=1.0in, papersize={5.5in,8.5in}]{geometry}
%\usepackage[pdftex]{graphicx}
%\usepackage{tipa}
%\usepackage{txfonts}
%\usepackage{textcomp}
%\usepackage{amsthm}
%\usepackage{xy}
%\usepackage{fancyhdr}
%\usepackage{xcolor}
%\usepackage{mathtools} %% for pretty underbrace % Breaks Ximera
%\usepackage{multicol}



\newcommand{\RR}{\mathbb R}
\newcommand{\R}{\mathbb R}
\newcommand{\C}{\mathbb C}
\newcommand{\N}{\mathbb N}
\newcommand{\Z}{\mathbb Z}
\newcommand{\dis}{\displaystyle}
%\renewcommand{\d}{\,d\!}
\renewcommand{\d}{\mathop{}\!d}
\newcommand{\dd}[2][]{\frac{\d #1}{\d #2}}
\newcommand{\pp}[2][]{\frac{\partial #1}{\partial #2}}
\renewcommand{\l}{\ell}
\newcommand{\ddx}{\frac{d}{\d x}}

\newcommand{\zeroOverZero}{\ensuremath{\boldsymbol{\tfrac{0}{0}}}}
\newcommand{\inftyOverInfty}{\ensuremath{\boldsymbol{\tfrac{\infty}{\infty}}}}
\newcommand{\zeroOverInfty}{\ensuremath{\boldsymbol{\tfrac{0}{\infty}}}}
\newcommand{\zeroTimesInfty}{\ensuremath{\small\boldsymbol{0\cdot \infty}}}
\newcommand{\inftyMinusInfty}{\ensuremath{\small\boldsymbol{\infty - \infty}}}
\newcommand{\oneToInfty}{\ensuremath{\boldsymbol{1^\infty}}}
\newcommand{\zeroToZero}{\ensuremath{\boldsymbol{0^0}}}
\newcommand{\inftyToZero}{\ensuremath{\boldsymbol{\infty^0}}}


\newcommand{\numOverZero}{\ensuremath{\boldsymbol{\tfrac{\#}{0}}}}
\newcommand{\dfn}{\textbf}
%\newcommand{\unit}{\,\mathrm}
\newcommand{\unit}{\mathop{}\!\mathrm}
%\newcommand{\eval}[1]{\bigg[ #1 \bigg]}
\newcommand{\eval}[1]{ #1 \bigg|}
\newcommand{\seq}[1]{\left( #1 \right)}
\renewcommand{\epsilon}{\varepsilon}
\renewcommand{\iff}{\Leftrightarrow}

\DeclareMathOperator{\arccot}{arccot}
\DeclareMathOperator{\arcsec}{arcsec}
\DeclareMathOperator{\arccsc}{arccsc}
\DeclareMathOperator{\si}{Si}
\DeclareMathOperator{\proj}{proj}
\DeclareMathOperator{\scal}{scal}
\DeclareMathOperator{\cis}{cis}
\DeclareMathOperator{\Arg}{Arg}
%\DeclareMathOperator{\arg}{arg}
\DeclareMathOperator{\Rep}{Re}
\DeclareMathOperator{\Imp}{Im}
\DeclareMathOperator{\sech}{sech}
\DeclareMathOperator{\csch}{csch}
\DeclareMathOperator{\Log}{Log}

\newcommand{\tightoverset}[2]{% for arrow vec
  \mathop{#2}\limits^{\vbox to -.5ex{\kern-0.75ex\hbox{$#1$}\vss}}}
\newcommand{\arrowvec}{\overrightarrow}
\renewcommand{\vec}{\mathbf}
\newcommand{\veci}{{\boldsymbol{\hat{\imath}}}}
\newcommand{\vecj}{{\boldsymbol{\hat{\jmath}}}}
\newcommand{\veck}{{\boldsymbol{\hat{k}}}}
\newcommand{\vecl}{\boldsymbol{\l}}
\newcommand{\utan}{\vec{\hat{t}}}
\newcommand{\unormal}{\vec{\hat{n}}}
\newcommand{\ubinormal}{\vec{\hat{b}}}

\newcommand{\dotp}{\bullet}
\newcommand{\cross}{\boldsymbol\times}
\newcommand{\grad}{\boldsymbol\nabla}
\newcommand{\divergence}{\grad\dotp}
\newcommand{\curl}{\grad\cross}
%% Simple horiz vectors
\renewcommand{\vector}[1]{\left\langle #1\right\rangle}


\outcome{In this section we describe curves in space.}

\title{2.1 Space Curves}

\begin{document}

\begin{abstract}
In this section we describe curves in space.
\end{abstract}
\end{document} 
\maketitle





A curve in  is given by a vector valued function,

$\R^3$
\[
\vec r(t) = \vector{f(t), g(t), h(t)}
\]




we will need the vector associated with these two points and we will also need to take advantage of the end to end method for adding vectors.
Suppose the line $L$ in $\R^3$ goes through the points $P(x_0, y_0, z_0)$ and $Q(x_1, y_1, z_1)$. The vector from $P$ to $Q$ is given by
\[
\vec{v} = \avec{PQ} = \vector{x_1-x_0,y_1-y_0,z_1-z_0} = \vector{a,b,c}
\]
This vector $\vec v = \vector{a,b,c}$ is called the \textbf{direction vector} of the line. 
A point $(x, y, z)$ on the line $L$ can be seen as the final point of a vector which is the sum of two vectors.
The first vector is a vector from the origin to the point $P(x_0, y_0, z_0)$.
The second vector is a multiple of the direction vector of the line, $\vec v = \vector{a, b, c}$.
To describe such points, we rely on the end to end method of vector addition. A point $(x, y, z)$ lies on the line $L$ if
it is the final point of a vector of the form:
\[
\vector{x_0, y_0, z_0} + t\vector{a, b, c}
\]
where $t$ is any scalar. See the figure below.

\begin{image}
\begin{tikzpicture}
\draw[ ->] (0,0) -- (1, 2) ;
\draw[red, ->] (1,2) -- (5, 3);
\draw[blue, ->] (0,0) -- (5,3);
\draw[blue, fill] (3, 2.5) circle (0.05) node[above]{$Q$};
\draw[blue, fill] (1, 2) circle (0.05)node[above]{$P$};
\draw[blue, fill] (0,0) circle (0.05) node[left]{$O$};
\draw[fill] (5, 3) circle (0.05) node[above]{$(x, y, z)$};
\node at (2.5, -0.5) {The vector $\vector{x_0, y_0, z_0} + t\vector{a, b, c}$ (in blue)};%
\end{tikzpicture}
\end{image}
 
If we associate the vector $\vector{x, y, z}$ with the point $(x, y, z)$, then we obtain the vector form of the equation of a line in $\R^3$.

\begin{definition}[Vector Form of a Line in Space]
A point $(x, y, z)$ is on the line through the point $P(x_0, y_0, z_0)$ with direction vector $\vec v = \vector{a, b, c}$
if
\[
\vector{x, y, z} = \vector{x_0, y_0, z_0} + t\vector{a, b, c}
\]
for some value of the parameter $t$.
This is called the {\bf vector form} of the equation of a line in $\R^3$.
\end{definition}

If we equate the components of the vectors in the above definition, we obtain another way of describing the line.

\begin{definition}[Parametric Form of the Equation of a Line in Space]
The \textbf{parametric form} of the equation of a line in $\R^3$ which passes through the point $(x_0, y_0, z_0)$ with 
direction vector $\vector{a, b, c}$ is given by
\begin{align*}
x &= x_0 + at\\
y &= y_0 + bt\\
z &= z_0 + ct
\end{align*}
\end{definition}
Note that the parametric form of the equation of a line in $\R^3$ consists of three equations.
There is yet another way to describe a line in space. If we solve each of the three equations in the parametric form for the parameter $t$, 
we obtain the following.
\begin{definition}[Symmetric Form of the Equation of a Line in Space]
The \textbf{symmetric form} of the equation of a line in $\R^3$ which passes through the point $(x_0, y_0, z_0)$ with 
direction vector $\vector{a, b, c}$ is given by
\[
\frac{x-x_0}{a} = \frac{y-y_0}{b} = \frac{z-z_0}{c}
\]
\end{definition}
\begin{remark}
It is only possible to create the symmetric form of the equation of a line in space if \textbf{each} of the components of 
the direction vector $\vector{a, b, c}$ is non-zero.
\end{remark}


\begin{example}[Example 1]
Find the vector, parametric and symmetric forms of the line with direction 
vector $\vec{v} = \vector{2, -7, 4}$ that goes through the point $P(-1, -3, 5)$. \\
We will substitute $(-1, -3, 5)$ for $(x_0, y_0, z_0)$ and $\vector{2, -7, 4}$ for $\vector{a,b,c}$ to  
obtain the three forms directly from their definitions:
\[
\vector{x, y, z} = \vector{-1, -3, 5} + t\vector{2, -7, 4} \quad \text{(Vector form)}
\]
\begin{align*}
x &= -1 +2t\\
y &= -3 -7t \quad\quad \text{(Parametric form)}\\
z &= 5 + 4t\\
\end{align*}
\[
\frac{x+1}{2} = \frac{y+3}{-7} = \frac{z-5}{4} \quad \text{(Symmetric form)}
\]
\end{example}

\begin{problem}(Problem 1)
Find the vector, parametric and symmetric forms of the line with direction 
vector $\vec{v} = \vector{-6, 12, 5}$ that goes through the point $P(13, 2, -7)$. \\
\end{problem}

\begin{example}[Example 2]
Find the vector, parametric and symmetric forms for the line in $\R^3$ passing 
through the points $(2, 1, -4)$ and $(-3, 2, 5)$.\\
The direction vector for the line is 
\[
\vec{v} = \vector{-3-2, 2-1, 5-(-4)} = \vector{-5, 1, 9}
\]
Note that any non-zero multiple of this vector would also serve as a suitable direction vector. 
Using the point $(2, 1, -4)$ we have
\[
\vector{x, y, z} = \vector{2, 1, -4} + t\vector{-5, 1, 9} \quad \text{(Vector form)}
\]
\begin{align*}
x &= 2 -5t\\
y &= 1 + t \quad\quad\text{(Parametric form)}\\
z &= -4 + 9t\\
\end{align*}
\[
\frac{x-2}{-5} = y-1 = \frac{z+4}{9} \quad \text{(Symmetric form)}
\]
\end{example}

\begin{problem}(Problem 2a)
Find the vector form, parametric form and symmetric forms for the line in $\R^3$ passing 
through the points $(8, -11, 17)$ and $(4, -9, 16)$.\\
\end{problem}

\begin{problem}(Problem 2b)
Find the vector, parametric and symmetric forms for the line in $\R^3$ passing through the origin 
and parallel to the line passing though the points $(5, 6, 7)$ and $(2, 4, 6)$.\\
\end{problem}

\begin{problem}(Problem 2c)
Show that the vector form of the equation of the line passing through the 
points $P(x_1, y_1, z_1)$ and $Q(x_2, y_2, z_2)$
can be written as
\[
\vector{x,y,z} = (1-t)\vector{x_1, y_1, z_1} + t\vector{x_2, y_2, z_2}
\]
\end{problem}

\begin{problem}(Problem 2d)
Which of the following lines are parallel? Are any of them identical?
\begin{align*}
L_1:&\quad \vector{x, y, z} = \vector{3,-1,4} + t\vector{-2,7,-2}\\
L_2:& \quad x = 5-4t,\; y = -8+ 14t,\; z = 6-4t\\
L_3:& \quad \frac{x - 4 }{2} = \frac{3 -y}{7}= \frac{z - 8}{2}\\
L_4:&\quad  \text{The line passing through the points} \;\; P(1, 6, 2)\;\; \text{and} \;\; Q(5, -8, 6)
\end{align*}
\begin{hint}
Parallel lines have parallel direction vectors
\end{hint}
\begin{hint}
The next hint contains the answer!
\end{hint}
\begin{hint}
Lines $L_1, L_2$ and $L_4$ are identical and $L_3$ is parallel to it/them
\end{hint}
\end{problem}


\begin{example}[Example 3]
Find the point of intersection of the lines
\[
L_1: \vector{x, y, z} = \vector{4, 6, -2} + t\vector{1, -2, 2}
\]
\[
L_2: \vector{x, y, z} = \vector{-3, 5, -1} + t\vector{2, 1, -1}
\]
To find the point of intersection, it will be helpful to distinguish between the parameters in the two lines.
In $L_1$, replace $t$ with $t_1$ and likewise, replace $t$ by $t_2$ in $L_2$. 
At the point of intersection, the $x$-coordinates are equal:
\[
4+t_1 = -3 + 2t_2
\]
We wish to solve for $t_1$ and $t_2$, so we need another equation. Equating the $y$-coordinates gives:
\[
6 -2t_1 = 5 + t_2
\]
Now, we solve the $2 \times 2$ linear system
\begin{align*}
4+t_1 &= -3 + 2t_2\\
6 -2t_1 &= 5 + t_2
\end{align*}
Solving the top equation for $t_1$ gives $t_1 = -7 + 2t_2$.  Substituting this into the bottom equation gives
\begin{align*}
6 - 2(-7+2t_2) &= 5 + t_2\\
6 + 14 - 4t_2 & = 5 + t_2\\
15 &= 5t_2\\
t_2 &=3
\end{align*}
Back substituting gives $t_1 = -7 + 2t_2 = -7 + 2(3) = -1$.
Hence, if the lines intersect, it will be at the point where $t_1 = -1$ on $L_1$ and $t_2 = 3$ on $L_2$.
Note that it is possible that the lines do not intersect- in which case, plugging these values in would 
give different $z$-coordinates.
The vector associated with the point of intersection is
\begin{align*}
L_1: \vector{x, y, z} &= \vector{4, 6, -2} + (-1)\vector{1, -2, 2} = \vector{3, 8, -4}\\
L_2: \vector{x, y, z} &= \vector{-3, 5, -1} + 3\vector{2, 1, -1} = \vector{3, 8, -4}
\end{align*}
Since the $z$ components agree, we can conclude definitively that the lines do indeed intersect and furthermore, 
the point of intersection is $(3, 8, -4)$.
\end{example}

\begin{problem}(Problem 3)
Find the point of intersection of the lines
\[
L_1: \vector{x, y, z} = \vector{-2, 1, 5} + t\vector{3, 1, -2}
\]
and 
\[
L_2: \vector{x, y, z} = \vector{3, -6, -3} + s\vector{1, -4, -3}
\]
The point of intersection is $\;\answer{(1, 2, 3)}$
\end{problem}


Next, we solve the classic problem of finding the distance between a point and a line.

\begin{example}[Example 4]
Find the distance between the line $\vector{x,y,z} = \vector{-2,3,1} + t \vector{4,-1,5}$ and the point $Q(1, 2, 3)$.\\
First, we verify that the given point is not actually on the line. 
To this end, let's examine the parametric equation for the $x$-coordinate:
\[
x = -2 + 4t
\]
The $x$=coordinate of the given point is $1$ and setting these equal gives a value of $t$:
\[
1 = -2 + 4t
\]
\[
t = \frac34
\]
Now we will compare the $y$-coordinates (and $z$ if necessary) when $t= 3/4$:
\[
y = 3 - t = 3 -\frac34 = \frac94 \neq 2
\]
Hence the point is not on the line.
From the given vector form of the line, we can see that the point $P(-2, 3, 1)$ is on the line.  
Consider the vector from this point to the point $Q$:
\[
\vec{w} = \avec{PQ} = \vector{3, -1, 2}
\]
Project this vector onto the direction vector of the line, $\vec{v} = \vector{4, -1, 5}$ :
\[
\proj_{\vec{v}}\vec{w} = \frac{\vec{v} \dotp \vec{w}}{\vec{v} \dotp \vec{v}}\vec{v} = \frac{23}{14}\vector{4,-1,5}
\]
The distance between the point $Q(1, 2, 3)$ and the line is the magnitude of the 
vector $\vec{n} = \vec{w} - \proj_{\vec{v}}\vec{w}$ (see the figure below)

\begin{image}
\begin{tikzpicture}
\draw[orange!80!black, ->, thick] (0,0) -- (3, 0) node[right]{$\vec{v}$};
\draw[blue!50!black, ->, thick] (0,0) -- (2, 4) node[midway, left]{$\vec{w}$};
\draw[red!90!black, ->, thin] (0,0) --(2,0) node[midway, below]{$\proj_{\vec{v}} \vec{w}$};
\draw[green!30!black, thick, <-] (2, 4) -- (2, 0) node[midway, right]{$\vec{n}= \vec{w} -  \proj_{\vec{v}} \vec{w}$};
\draw[blue, thin] (1.7, 0) -- ++(0, 0.3) -- + (0.3, 0);
\node at (1.5, -1){The distance is $|\vec{n}| $};
\draw[fill, blue!50!black] (0,0) circle (0.05) node[left]{$P$};
\draw[fill, blue!50!black] (2,4) circle (0.05) node[left]{$Q$};
\end{tikzpicture}
\end{image}

\begin{image}
\begin{tikzpicture}
\draw[blue!60!white, ->, thick] (0,0) -- (3, 0) node[right]{$\vec{v}$};
\draw[blue!85!white, ->, thick] (0,0) --(2,0) node[midway, below]{$\proj_{\vec{v}} \vec{w}$};
\draw[blue!70!white, ->, thick] (0,0) -- (2, 4) node[midway, left]{$\vec{w}$};
\draw[blue!90!white, thick, <-] (2, 4) -- (2, 0) node[midway, right]{$\vec{n}= \vec{w} -  \proj_{\vec{v}} \vec{w}$};
\draw[blue, thin] (1.7, 0) -- ++(0, 0.3) -- + (0.3, 0);
\node at (1.5, -1){The distance is $|\vec{n}| $};
\draw[fill, blue!70!white] (0,0) circle (0.05) node[left]{$P$};
\draw[fill, blue!70!white] (2,4) circle (0.05) node[left]{$Q$};
\end{tikzpicture}
\end{image}

\begin{image}
\begin{tikzpicture}
\draw[orange!80!black, ->, thick] (0,0) -- (3, 0) node[right]{$\vec{v}$};
\draw[blue!50!black, ->, thick] (0,0) -- (2, 4) node[midway, left]{$\vec{w}$};
\draw[red!90!black, ->, thin] (0,0) --(2,0) node[midway, below]{$\proj_{\vec{v}} \vec{w}$};
\draw[red!90!black, thick, <-] (2, 4) -- (2, 0) node[midway, right]{$\vec{n}= \vec{w} -  \proj_{\vec{v}} \vec{w}$};
\draw[blue, thin] (1.7, 0) -- ++(0, 0.3) -- + (0.3, 0);
\node at (1.5, -1){The distance is $|\vec{n}| $};
\draw[fill, blue!50!black] (0,0) circle (0.05) node[left]{$P$};
\draw[fill, blue!50!black] (2,4) circle (0.05) node[left]{$Q$};
\end{tikzpicture}
\end{image}

The vector $\vec{n}$ is given by
\begin{align*}
\vec{n} &= \vec{w} -  \proj_{\vec{v}} \vec{w}\\
&=\vector{3, -1, 2} - \frac{23}{14}\vector{4,-1,5} \\
&= \vector{-\frac{50}{14},\frac{9}{14},-\frac{87}{14}}
\end{align*}
and the distance we seek is
\[
|\vec{n}| = \sqrt{ \left(\frac{50}{14}\right)^2 +  \left(\frac{9}{14}\right)^2 + \left(\frac{87}{14}\right)^2} = \frac{\sqrt{10150}}{14}\approx 7.2
\]

\end{example}

\begin{problem}(Problem 4)
Find the distance between the point $Q(1, 2, 3)$ the line 
\[
L: \; \frac{x- 2}{3} = \frac{1-y}{4} = \frac{z}{2}
\]
\begin{hint}
The point $P(2, 1, 0)$ is on the line $L$
\end{hint}
\begin{hint}
The direction vector of $L$ is $\vec v = \vector{3, -4, 2}$
\end{hint}
\begin{hint}
Project the vector $\avec{PQ}$ onto the direction vector $\vec v$
\end{hint}
\begin{hint}
The distance is the magnitude of difference of $\avec{PQ}$ and the projection vector
\end{hint}
The distance is $\; \answer{\frac{\sqrt{9222}}{29}}$
\end{problem}

Another classic problem is to find the distance between two lines.
Two lines in $\R^3$ which do not intersect are either parallel or {\bf skew}.  
The distance between parallel lines can be found using the method of the previous example by 
taking any point from one of the lines,
and finding its distance to the other line.
The situation for skew lines is more complicated, involving cross products and planes.
In the next section, we will study planes in $\R^3$ and at the end of that section we will get back to the 
problem of finding the distance between two skew lines.\\
We conclude this section with line segments and in particular, we will verify the 
midpoint formula for $\R^3$ of section 1.1.  In problem 10 of that section, we were asked to use the distance formula to 
show that the point 
\[
\text{M} = \left(\frac{x_1 + x_2}{2}, \frac{y_1 + y_2}{2}, \frac{z_1 + z_2}{2}\right)
\]
is equidistant to the points $P(x_1, y_1, z_1)$ and $Q(x_2, y_2, z_2)$. 
Having done that, it remains to show that the point M is on the line segment between P and Q.

\begin{definition}
The parametric form of the line segment between the points $P(x_1, y_1, z_1)$ and $Q(x_2, y_2, z_2)$
is given by
\begin{align*}
x &= x_1 + t(x_2 - x_1)\\
y &= y_1 + t(y_2 - y_1)\\
z &= z_1 + t(z_2 - z_1)\\
\end{align*}
where $t \in [0,1]$, i.e., $0 \leq t \leq 1$.
%the parameter $t$ is between 0 and 1, 
\end{definition}

The reader should verify that this corresponds exactly to the equation of the line containing the points P and Q.
In other words, the parametric equations of the line segment are identical to those of the line. 
The only difference is the restriction of the parameter $t$ to the interval $[0,1]$. 

\begin{problem}(Problem 5)
Use the parametric equations of the line segment between the points  $P(x_1, y_1, z_1)$ and $Q(x_2, y_2, z_2)$
to determine which points correspond the the $t$-values $0, 1/2, $ and $1$. 
Is one of these the midpoint of the line segment?
\end{problem}
 

\end{document}
 
 
 
 
 
 
 
 
 
 
 
 
 
 
 
 
 
 
 
 
\begin{image}
\begin{tikzpicture}
\draw[blue!10!white, thick] (0,10) -- (10, 10) ;
\draw[blue!20!white, thick] (0,9) -- (10, 9) ;
\draw[blue!30!white, thick] (0,8) -- (10, 8) ;
\draw[blue!40!white, thick] (0,7) -- (10, 7) ;
\draw[blue!50!white, thick] (0,6) -- (10, 6) ;
\draw[blue!60!white, thick] (0,5) -- (10, 5) ;
\draw[blue!70!white, thick] (0,4) -- (10, 4) ;
\draw[blue!80!white, thick] (0,3) -- (10, 3) ;
\draw[blue!90!white, thick] (0,2) -- (10, 2) ;
\draw[blue, thick] (0,1) -- (10,1) node[right]{blue};

\draw[blue!90!black, thick] (0,0) -- (10, 0) ;
\draw[blue!80!black, thick] (0,-1) -- (10, -1) ;
\draw[blue!70!black, thick] (0,-2) -- (10, -2) ;
\draw[blue!60!black, thick] (0,-3) -- (10, -3) ;
\draw[blue!50!black, thick] (0,-4) -- (10, -4) ;
\draw[blue!40!black, thick] (0,-5) -- (10, -5) ;
\draw[blue!30!black, thick] (0,-6) -- (10, -6) ;
\draw[blue!20!black, thick] (0,-7) -- (10, -7) ;
\draw[blue!10!black, thick] (0,-8) -- (10, -8) ;
\end{tikzpicture}
\end{image}

\begin{image}
\begin{tikzpicture}
\draw[red!10!white, thick] (0,10) -- (10, 10) ;
\draw[red!20!white, thick] (0,9) -- (10, 9) ;
\draw[red!30!white, thick] (0,8) -- (10, 8) ;
\draw[red!40!white, thick] (0,7) -- (10, 7) ;
\draw[red!50!white, thick] (0,6) -- (10, 6) ;
\draw[red!60!white, thick] (0,5) -- (10, 5) ;
\draw[red!70!white, thick] (0,4) -- (10, 4) ;
\draw[red!80!white, thick] (0,3) -- (10, 3) ;
\draw[red!90!white, thick] (0,2) -- (10, 2) ;
\draw[red, thick] (0,1) -- (10,1) node[right]{red};

\draw[red!90!black, thick] (0,0) -- (10, 0) ;
\draw[red!80!black, thick] (0,-1) -- (10, -1) ;
\draw[red!70!black, thick] (0,-2) -- (10, -2) ;
\draw[red!60!black, thick] (0,-3) -- (10, -3) ;
\draw[red!50!black, thick] (0,-4) -- (10, -4) ;
\draw[red!40!black, thick] (0,-5) -- (10, -5) ;
\draw[red!30!black, thick] (0,-6) -- (10, -6) ;
\draw[red!20!black, thick] (0,-7) -- (10, -7) ;
\draw[red!10!black, thick] (0,-8) -- (10, -8) ;
\end{tikzpicture}
\end{image}


\begin{image}
\begin{tikzpicture}
\draw[green!10!white, thick] (0,10) -- (10, 10) ;
\draw[green!20!white, thick] (0,9) -- (10, 9) ;
\draw[green!30!white, thick] (0,8) -- (10, 8) ;
\draw[green!40!white, thick] (0,7) -- (10, 7) ;
\draw[green!50!white, thick] (0,6) -- (10, 6) ;
\draw[green!60!white, thick] (0,5) -- (10, 5) ;
\draw[green!70!white, thick] (0,4) -- (10, 4) ;
\draw[green!80!white, thick] (0,3) -- (10, 3) ;
\draw[green!90!white, thick] (0,2) -- (10, 2) ;
\draw[green, thick] (0,1) -- (10,1) node[right]{green};

\draw[green!90!black, thick] (0,0) -- (10, 0) ;
\draw[green!80!black, thick] (0,-1) -- (10, -1) ;
\draw[green!70!black, thick] (0,-2) -- (10, -2) ;
\draw[green!60!black, thick] (0,-3) -- (10, -3) ;
\draw[green!50!black, thick] (0,-4) -- (10, -4) ;
\draw[green!40!black, thick] (0,-5) -- (10, -5) ;
\draw[green!30!black, thick] (0,-6) -- (10, -6) ;
\draw[green!20!black, thick] (0,-7) -- (10, -7) ;
\draw[green!10!black, thick] (0,-8) -- (10, -8) ;
\end{tikzpicture}
\end{image}



\begin{image}
\begin{tikzpicture}
\draw[green!50!blue, thick] (0,9) -- (10, 9)  node[right]{green!50!blue};
\draw[red!50!blue, thick] (0,8) -- (10, 8)  node[right]{red!50!blue};
\draw[brown!50!blue, thick] (0,7) -- (10, 7)  node[right]{brown blue};
\draw[orange!50!blue, thick] (0,6) -- (10, 6)  node[right]{orange blue};
\draw[cyan!50!blue, thick] (0,5) -- (10, 5)  node[right]{cyan blue};
\draw[purple!50!blue, thick] (0,4) -- (10, 4)  node[right]{purple blue};
\draw[violet!50!blue, thick] (0,3) -- (10, 3)  node[right]{violet blue};
\draw[yellow!50!blue, thick] (0,2) -- (10, 2)  node[right]{yellow blue};
\draw[blue, thick] (0,1) -- (10,1) node[right]{blue};

\draw[olive!50!blue, thick] (0,0) -- (10, 0)  node[right]{olive blue};
\draw[pink!50!blue, thick] (0,-1) -- (10, -1)  node[right]{pink blue};
\draw[teal!50!blue, thick] (0,-2) -- (10, -2)  node[right]{teal blue};
\draw[magenta!50!blue, thick] (0,-3) -- (10, -3)  node[right]{magenta blue};
\draw[lime!50!blue, thick] (0,-4) -- (10, -4)  node[right]{lime blue};
\draw[black!50!blue, thick] (0,-5) -- (10, -5)  node[right]{black blue};
\draw[gray!50!blue, thick] (0,-6) -- (10, -6)  node[right]{gray blue};
\draw[white!50!blue, thick] (0,-7) -- (10, -7)  node[right]{white blue};
\draw[darkgray!50!blue, thick] (0,-8) -- (10, -8)  node[right]{darkgray blue};
\draw[lightgray!50!blue, thick] (0,-9) -- (10, -9)  node[right]{lightgray blue};
\end{tikzpicture}
\end{image}




 
 
 
 
 
 









