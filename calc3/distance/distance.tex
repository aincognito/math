\documentclass[handout]{ximera}

%% You can put user macros here
%% However, you cannot make new environments



\newcommand{\ffrac}[2]{\frac{\text{\footnotesize $#1$}}{\text{\footnotesize $#2$}}}
\newcommand{\vasymptote}[2][]{
    \draw [densely dashed,#1] ({rel axis cs:0,0} -| {axis cs:#2,0}) -- ({rel axis cs:0,1} -| {axis cs:#2,0});
}


\graphicspath{{./}{firstExample/}}
\usepackage{forest}
\usepackage{amsmath}
\usepackage{amssymb}
\usepackage{array}
\usepackage[makeroom]{cancel} %% for strike outs
\usepackage{pgffor} %% required for integral for loops
\usepackage{tikz}
\usepackage{tikz-cd}
\usepackage{tkz-euclide}
\usetikzlibrary{shapes.multipart}


%\usetkzobj{all}
\tikzstyle geometryDiagrams=[ultra thick,color=blue!50!black]


\usetikzlibrary{arrows}
\tikzset{>=stealth,commutative diagrams/.cd,
  arrow style=tikz,diagrams={>=stealth}} %% cool arrow head
\tikzset{shorten <>/.style={ shorten >=#1, shorten <=#1 } } %% allows shorter vectors

\usetikzlibrary{backgrounds} %% for boxes around graphs
\usetikzlibrary{shapes,positioning}  %% Clouds and stars
\usetikzlibrary{matrix} %% for matrix
\usepgfplotslibrary{polar} %% for polar plots
\usepgfplotslibrary{fillbetween} %% to shade area between curves in TikZ



%\usepackage[width=4.375in, height=7.0in, top=1.0in, papersize={5.5in,8.5in}]{geometry}
%\usepackage[pdftex]{graphicx}
%\usepackage{tipa}
%\usepackage{txfonts}
%\usepackage{textcomp}
%\usepackage{amsthm}
%\usepackage{xy}
%\usepackage{fancyhdr}
%\usepackage{xcolor}
%\usepackage{mathtools} %% for pretty underbrace % Breaks Ximera
%\usepackage{multicol}



\newcommand{\RR}{\mathbb R}
\newcommand{\R}{\mathbb R}
\newcommand{\C}{\mathbb C}
\newcommand{\N}{\mathbb N}
\newcommand{\Z}{\mathbb Z}
\newcommand{\dis}{\displaystyle}
%\renewcommand{\d}{\,d\!}
\renewcommand{\d}{\mathop{}\!d}
\newcommand{\dd}[2][]{\frac{\d #1}{\d #2}}
\newcommand{\pp}[2][]{\frac{\partial #1}{\partial #2}}
\renewcommand{\l}{\ell}
\newcommand{\ddx}{\frac{d}{\d x}}

\newcommand{\zeroOverZero}{\ensuremath{\boldsymbol{\tfrac{0}{0}}}}
\newcommand{\inftyOverInfty}{\ensuremath{\boldsymbol{\tfrac{\infty}{\infty}}}}
\newcommand{\zeroOverInfty}{\ensuremath{\boldsymbol{\tfrac{0}{\infty}}}}
\newcommand{\zeroTimesInfty}{\ensuremath{\small\boldsymbol{0\cdot \infty}}}
\newcommand{\inftyMinusInfty}{\ensuremath{\small\boldsymbol{\infty - \infty}}}
\newcommand{\oneToInfty}{\ensuremath{\boldsymbol{1^\infty}}}
\newcommand{\zeroToZero}{\ensuremath{\boldsymbol{0^0}}}
\newcommand{\inftyToZero}{\ensuremath{\boldsymbol{\infty^0}}}


\newcommand{\numOverZero}{\ensuremath{\boldsymbol{\tfrac{\#}{0}}}}
\newcommand{\dfn}{\textbf}
%\newcommand{\unit}{\,\mathrm}
\newcommand{\unit}{\mathop{}\!\mathrm}
%\newcommand{\eval}[1]{\bigg[ #1 \bigg]}
\newcommand{\eval}[1]{ #1 \bigg|}
\newcommand{\seq}[1]{\left( #1 \right)}
\renewcommand{\epsilon}{\varepsilon}
\renewcommand{\iff}{\Leftrightarrow}

\DeclareMathOperator{\arccot}{arccot}
\DeclareMathOperator{\arcsec}{arcsec}
\DeclareMathOperator{\arccsc}{arccsc}
\DeclareMathOperator{\si}{Si}
\DeclareMathOperator{\proj}{proj}
\DeclareMathOperator{\scal}{scal}
\DeclareMathOperator{\cis}{cis}
\DeclareMathOperator{\Arg}{Arg}
%\DeclareMathOperator{\arg}{arg}
\DeclareMathOperator{\Rep}{Re}
\DeclareMathOperator{\Imp}{Im}
\DeclareMathOperator{\sech}{sech}
\DeclareMathOperator{\csch}{csch}
\DeclareMathOperator{\Log}{Log}

\newcommand{\tightoverset}[2]{% for arrow vec
  \mathop{#2}\limits^{\vbox to -.5ex{\kern-0.75ex\hbox{$#1$}\vss}}}
\newcommand{\arrowvec}{\overrightarrow}
\renewcommand{\vec}{\mathbf}
\newcommand{\veci}{{\boldsymbol{\hat{\imath}}}}
\newcommand{\vecj}{{\boldsymbol{\hat{\jmath}}}}
\newcommand{\veck}{{\boldsymbol{\hat{k}}}}
\newcommand{\vecl}{\boldsymbol{\l}}
\newcommand{\utan}{\vec{\hat{t}}}
\newcommand{\unormal}{\vec{\hat{n}}}
\newcommand{\ubinormal}{\vec{\hat{b}}}

\newcommand{\dotp}{\bullet}
\newcommand{\cross}{\boldsymbol\times}
\newcommand{\grad}{\boldsymbol\nabla}
\newcommand{\divergence}{\grad\dotp}
\newcommand{\curl}{\grad\cross}
%% Simple horiz vectors
\renewcommand{\vector}[1]{\left\langle #1\right\rangle}


\outcome{Create and apply the distance formula in three dimensions.}

\title{1.1 Distance in Space}



\begin{document}

\begin{abstract}
In this section we create the distance formula in $\R^3$ and apply it to spheres.
\end{abstract}

\maketitle


\section{One Dimension}

The real number line, denoted $\R$ (or $\R^1$ for emphasis) consists of all real numbers.  In interval notation, we can write $\R = (-\infty, \infty)$.
The distance between two real numbers $x_1$ and $x_2$ is given by $d = |x_2 - x_1|$. Note that $d = 0$ if and only if $x_1 = x_2$.


\begin{center}
\begin{tikzpicture}
\draw[<->] (-3,0) -- (3,0) node[right]{$\R$};
\draw (-2,-.2) -- (-2,.2); %tick mark
\draw (2,-.2) -- (2,.2);
\draw[fill] (-2,0) circle [radius=0.07];
\draw[fill] (2,0) circle [radius=0.07];
\node[below] at (-2.1,-.3) {$x_2$};
\node[below] at (2,-.3) {$x_1$};
\node[below] at (0,-.8) {$d = |x_2 - x_1|$};
\end{tikzpicture}
\end{center}

\begin{example}[Example 1]
Show that the points $-3$ and $6$ in $\R^1$ are equidistant to the point $1.5$.\\
Let $d_1$ be the distance from $-3$ to $1.5$ and let $d_2$ be the distance from $6$ to $1.5$. Then
\[
d_1 = |-3 - 1.5| = |-4.5| = 4.5
\]
and
\[
d_2 = |6 - 1.5| = |4.5| = 4.5
\]
Since $d_1 = d_2$, the points $-3$ and $6$ are equidistant to $1.5$ in $\R^1$.
\end{example}

\begin{problem}(Problem 1)
Use the distance formula in $\R^1$ to verify that the midpoint between the real numbers $x_1$ and $x_2$ is 
\[]
\text{midpoint} = \frac{x_1 +x_2}{2}
\]
\begin{hint}
You may assume that $x_1 < x_2$
\end{hint}

\begin{hint}
Compute the distances from $x_1$ and $x_2$ from the supposed midpoint
\end{hint}
\end{problem}

\section{Two Dimensions}
Two dimensional Euclidean space, denoted $\R^2$, is typically represented geometrically by the $xy$-plane. 
In formal terms $\R^2$ is the set of all coordinate pars of real numbers:
 \[
\R^2 = \{(x,y) \,|\, x, y \in \R\}
\]

The distance formula in $\R^2$ comes from the Pythagorean Theorem. If $(x_1, y_1$ and $(x_2, y_2)$ are points in $\R^2$, then from the triangle below,
we can see that the distance between them is given by 
\[
d = \sqrt{(x_2 - x_1)^2 + (y_2 - y_1)^2}
\]



\begin{image}
\begin{tikzpicture}
\draw[blue] (0,0) -- (3, 0) -- (3, 3) -- cycle;
\draw[thin, blue]  (2.8, 0) -- (2.8, 0.2)-- (3, 0.2);
\draw[fill, blue] (0,0) circle [radius=0.07] node[left, blue]{$(x_1, y_1)$};
\draw[fill, blue] (3,3) circle [radius=0.07] node[right, blue]{$(x_2, y_2)$};
\draw[fill, blue] (3,0) circle [radius=0.07] node[right, blue]{$(x_2, y_1)$};
\node[below, blue] at (1.5,-0.1) {$|x_2 - x_1|$}; %adjacent
\node[right, blue] at (3,1.4) {$|y_2 - y_1|$}; %opposite 
%\node[right] at (4.1,2) {$\tan(\theta)$};
\node[above, blue] at (1.3,1.4) {$d$}; %hypotenuse
%\node[below] at (1.75,-0.6) {$d = \sqrt{(x_2 - x_1)^2 + |y_2 - y_1|^2}$};
%\node[above] at (1.75,3.6) {A right triangle with $\cos(\theta) = x$};
\end{tikzpicture}
\end{image}
Note that the distance between the points $(x_1, y_1)$ and $(x_2, y_1)$ is given by 
$d = \sqrt{(x_2-x_1)^2} = |x_2 -x_1|$, since, in general $\sqrt{x^2} = |x|$.

\begin{example}[Example 2]
Find the distance between the points $(3,-4)$ and $(9,8)$ in $\R^2$.\\
From the distance formula in $\R^2$, we have
\begin{align*}
d &= \sqrt{(3 - 9)^2 + [(-4) - 8]^2}\\
  &= \sqrt{(-6)^2 + (-12)^2}\\
  & = \sqrt{36 +144}\\
  &= \sqrt{180}\\
  &= 6\sqrt 5
\end{align*}
\end{example}

\begin{problem}(Problem 2)
Find the distance between the points $(5,-8)$ and $(10,7)$ in $\R^2$.\\
\[
d = \answer{5\sqrt{10}}
\]
\end{problem}

The distance formula in $R^2$ can be used to construct the equation of a circle. Consider a circle with center at $(h,k)$ and radius, $r$.
A point $(x,y)$ is on the circle if and only if the distance between $(x,y)$ and $(h,k)$ is $r$. 
Squaring both sides of the distance formula, we get the following equation of the circle: 
\[
(x-h)^2 + (y-k)^2 = r^2
\]

\begin{example}[Example 3]
Find the equation of the circle with center at $(2, -3)$ and radius $4$.\\
Substituting $h = 2, k = -3$ and $r =4$ into the formula
\[
(x-h)^2 + (y-k)^2 = r^2
\]
we obtain the equation of the circle:
\[
(x-2)^2 + (y+3)^2 = 16
\]
\end{example}

\begin{problem}(Problem 3)
Find the equation of the circle with center at $(-1, 5)$ and radius $3$.
\begin{multipleChoice}
\choice{$(x-1)^2 + (y+5)^2 = 9$}
\choice{$(x+1)^2 + (y-5)^2 = 3$}
\choice[correct]{$(x+1)^2 + (y-5)^2 = 9$}
\end{multipleChoice}
\end{problem}

\begin{example}[Example 4]
Find the center and radius of the circle $x^2 + y^2 -11 = 2x - 4y$\\
At first, it is not even clear that this equation represents a circle.  Rewrite the equation as
\[
x^2 - 2x + y^2 + 4y = 11
\]
and {\bf complete the square} in both variables. We get
\[
x^2 - 2x + 1 + y^2 + 4y + 4 = 11 + 1 + 4
\]
which can be written as 
\[
(x-1)^2 +(y+2)^2 = 16
\]
Comparing this to the standard form of the equation of a circle,
\[
(x-h)^2 + (y-k)^2 = r^2
\]
we see that this is the equation of a circle with 
\[
\text{center: } (1, -2) \quad  \text{and} \quad \text{radius} = 4
\]
\end{example}

\begin{problem}(Problem 4)
Find the center and radius of the circle $6y - 8x= 24 -x^2 - y^2$\\
The center is $\answer{(4, -3)}$ and the radius is $\answer{7}$\\
\begin{hint}
Enter the center in the form $(a,b)$
\end{hint}
\end{problem}

The distance formula can also be used to verify the fact that the midpoint of the line segment between the points 
$(x_1,y_1)$ and $(x_2, y_2)$ is given by
\[
\text{midpoint} = \left(\frac{x_1+x_2}{2}, \frac{y_1+y_2}{2}\right)
\]

\begin{example}[Example 5]
Find the midpoint of the line segment connecting the points $(,)$ and $(,)$.\\
Following the formula above, the $x$-coordinate of the midpoint is
\[
x = \frac{ + }{2} = 
\]
and the $y$-coordinate is
\[
y = \frac{ + }{2} = 
\]
Hence the midpoint is
\[
(x,y) = (,)
\]
\end{example}

In the following problem, you are asked to verify that the midpoint is indeed equidistant to each of the endpoints of the segment.

\begin{problem}(Problem 5a)
Show that the midpoint given above is equidistant to the points $(x_1, y_1)$ and $(x_2, y_2)$.
\end{problem}
In the next problem, you will verify that the midpoint is actually on the line segment by comparing slopes.
\begin{problem}(Problem 5b)
Show that the slopes of the segments from $(x_1, y_1)$ to the midpoint and from the midpoint to $(x_2, y_2)$ are equal.
\end{problem}

\section{Three Dimensions}
Three dimensional Euclidean space, denoted $\R^3$, is the set of all triples of real numbers:
\[
\R^3 = \{(x,y,z)\,|\, x,y,z \in \R \}
\]
Geometrically, $\R^3$ is represented using three coordinate axes, $x, y$ and $z$.
The figure below shows the positive direction of each axis and the origin, $O = (0,0,0)$.
\begin{image}
\begin{tikzpicture}
\draw[thick, ->] (0,0) -- (2,0) node[right]{$y$};
\draw[thick, ->] (0,0) -- (0,1.8) node[above]{$z$};
\draw[thick, ->] (0,0) -- (-1.2,-0.9) node[below, left]{$x$};
\node at (0.1, -0.2){$O$} ;
\end{tikzpicture}
\end{image}
In addition to the coordinate axes, there are also three coordinate planes.  
The $xy$-plane is the plane containing the $x$ and $y$ axes.  
The $z$-coordinate of each point in the $xy$-plane is zero, hence its equation is
\[
z = 0 \;\; \text{(the $xy$-plane)}
\]
Similarly, the $xz$-plane has equation $y = 0$ and the $yz$-plane has equation $x=0$.
The three coordinate planes are shown below.
\begin{image}
\begin{tikzpicture}

\fill[blue!20] (-1, -0.8) -- (0,0) -- (1.7,0) -- (0.7, -0.8) --cycle;
\fill[red!20]  (0,0) -- (1.7,0) -- (1.7, 1.7) -- (0,1.7) --cycle;
\fill[green!20] (-1, -0.7) -- (0,0) -- (0,1.7) -- (-1, 0.9) --cycle;
\draw[thick, ->] (0,0) -- (2,0) node[right]{$y$};
\draw[thick, ->] (0,0) -- (0,1.8) node[above]{$z$};
\draw[thick, ->] (0,0) -- (-1.2,-0.9) node[below, left]{$x$};
\node at (0.5, -0.4){$z = 0$} ;
\node at (1, 1){$x = 0$} ;
\node at (-0.5, 0.7){$y = 0$} ;
\end{tikzpicture}
\end{image}

To find the distance formula in $\R^3$, we need to use the Pythagorean Theorem twice.
We wish to find the distance between the points $(x_1, y_1, z_1)$ and $(x_2, y_2, z_2)$ in $\R^3$.
First, we find the distance between the points $(x_1, y_1, z_1)$ and $(x_2, y_2, z_1)$.  These two points have the same $z$-coordinate, and hence they lie
in a plane parallel to the $xy$-plane.  The distance between these two points is the same as the distance between the points $(x_1, y_1$ and $(x_2, y_2)$
in $\R^2$:
\[
d_1 = \sqrt{(x_2 -x_1)^2 + (y_2 - y_1)^2}
\]
(This was the first application of the Pythagorean Theorem.) Now, consider the line segment connecting the points $(x_2, y_2, z_1)$  and $(x_2, y_2, z_2)$.
These points have the same $x$ and $y$ coordinates and hence the segment between them is perpendicular to the $xy$-plane. 
Thus the distance between them is the same as the distance between the points $z_1$ and $z_2$ on the $z$-axis, which can be found using our distance formula in $R^1$:
\[
d_2 = |z_2 -z_1|
\]
Now observe that the segments from $(x_1, y_1, z_1)$ to $(x_2, y_2, z_1)$ and $(x_2, y_2, z_1)$ to $(x_2, y_2, z_2)$ are perpendicular, 
so that the triangle with vertices $(x_1, y_1, z_1)$, $(x_2, y_2, z_1)$ and $(x_2, y_2, z_2)$ is a right triangle (see the figure below).





\begin{image}
\begin{tikzpicture}
\draw[blue] (0,0) -- (3, 0) -- (3, 3) -- cycle;
\draw[thin, blue]  (2.8, 0) -- (2.8, 0.2)-- (3, 0.2);
\draw[fill, blue] (0,0) circle [radius=0.07] node[left, blue]{$(x_1, y_1, z_1)$};
\draw[fill, blue] (3,3) circle [radius=0.07] node[right, blue]{$(x_2, y_2, z_2)$};
\draw[fill, blue] (3,0) circle [radius=0.07] node[right, blue]{$(x_2, y_2, z_1)$};
\node[below, blue] at (1.5,-0.1) {$d_1$}; %adjacent
\node[right, blue] at (3,1.4) {$d_2$}; %opposite 
%\node[right] at (4.1,2) {$\tan(\theta)$};
\node[above, blue] at (1.3,1.4) {$d$}; %hypotenuse
\node[below] at (1.75,-0.6) {$d_1 = \sqrt{(x_2 - x_1)^2 + (y_2 - y_1)^2}$ and $d_2 = |z_2 -z_1|$};
%\node[above] at (1.75,3.6) {A right triangle with $\cos(\theta) = x$};
\end{tikzpicture}
\end{image}

Finally, the Pythagorean Theorem gives us the distance between the points $(x_1, y_1, z_1)$ and $(x_2, y_2, z_2)$:
\[
d = \sqrt{d_1^2 +d_2^2} = \sqrt{(x_2 -x_1)^2+(y_2 -y_1)^2+(z_2 -z_1)^2}
\]

%\sqrt{ \sqrt{(x_2 -x_1)^2+(y_2 -y_1)^2}^2 + |z_2 -z_1|^2} 

\begin{example}[Example 6]
Compute the distance between the points $(4, 6, -7)$ and $(1,10, 3)$.\\
Using the distance formula in $\R^3$, with $(x_1, y_1, z_1) = (4,6,-7)$ and $(x_2, y_2, z_2) = (1,10,3)$ we have
\begin{align*}
d &= \sqrt{(x_2 -x_1)^2+(y_2 -y_1)^2+(z_2 -z_1)^2}\\
  &= \sqrt{(1-4)^2+(10-6 )^2+[3-(-7)]^2}\\
  &= \sqrt{(-3)^2 + 4^2 +10^2}\\
  &= \sqrt{125}\\
  &= 5\sqrt 5
\end{align*}

\end{example}
\begin{problem}(Problem 6) 
Compute the distance between the points $(1, -2, 3)$ and $(-2, 5, 1)$.\\
$d = \answer{\sqrt{62}}$
\end{problem}

\subsection{Spheres}
A sphere in $\R^3$ is the locus of points in $\R^3$ that are equidistant to a given point (the center).
The distance formula gives us the equation of a sphere in $\R^3$.  A point $(x, y, z)$ is on the sphere of radius $r$ with center $(h, k, l)$
if and only if
\[
(x-h)^2 + (y-k)^2 + (z -l)^2 = r^2
\]
giving us the equation of a sphere.

\begin{example}[Example 7]
Identify the center and radius of the sphere $x^2 + (y-2)^2 + (z+3)^2 = 5$.\\
The center is $(0, 2, -3)$ and the radius is $\sqrt 5$.
\end{example}

\begin{problem}(Problem 7)
Identify the center and radius of the sphere $(x+1)^2 + y^2 + (z-4)^2 = 144$.\\
The center is $\answer{(-1, 0, 4)}$ and the radius is $\answer{12}$.
\end{problem}


\begin{example}[Example 8]
Determine if the following equation is that of a sphere, and if it is, determine its center and radius:
\[
x^2 + 4x + y^2 - 6x + z^2 + 10z + 29 = 0
\]
We begin by completing the square in each variable:
\[
x^2 + 4x + 4 + y^2 - 6x + 9 + z^2 + 10z + 25 =-29 + 4 + 9 + 25
\]
which yields
\[
(x+2)^2 +(y-3)^2 + (z+5)^2 = 9
\]
This is the equation of a sphere with center $(-2, 3, -5)$ and radius $3$. 
\end{example}

\begin{problem}(Problem 8)
Determine if the following equation is that of a sphere, and if it is, determine its center and radius:
\[
x^2 -8x + y^2 + 12x + z^2 + 2z + 20 = 0
\]
This is/is not the equation of a sphere.\\
The center is $\answer{(4,-6,-1)}$ and the radius is $\answer{6}$
\begin{hint}
Enter the center in the form $(a, b, c)$
\end{hint}
\end{problem}

\begin{example}[Example 9]
Describe the intersection of the sphere in the previous examples with each of the coordinate planes.\\
To find the intersection with the $xy$-plane, we set the $z$-coordinate equal to zero:
\[
(x+2)^2 +(y-3)^2 + 5^2 = 9
\]
Subtracting 25 from both sides, we see that the right hand side is negative.  
Since the sum of squares cannot be negative, this equation has no solutions.
Geometrically, this means that the sphere does not intersect the $xy$-plane. 
(This makes sense considering that the $z$-coordinate of the center is 5 units 
below the $xy$-plane and the radius of the sphere is only 3.)\\
To find the intersection with the $xz$-plane, we set $y = 0$:
\[
(x+2)^2 + (-3)^2 + (z+5)^2 = 9
\]
Here we see that the sum of two squares is 0.  Thus each square must be zero, so $x = -2$ and $z = -5$ and 
the intersection of the sphere with the $xz$-plane is the single point $(-2, 0, -5)$.\\
To find the intersection with the $yz$-plane, we set $x = 0$:
\[
2^2 + (y-3)^2 + (z+5)^2 = 9
\]
which becomes
\[
(y-3)^2 + (x+5)^2 = 5
\]
This is the equation of a circle in the $yz$-plane with center $(0, 3, -5)$ and radius $\sqrt 5$.\\
In this example, we saw each of the three possibilities for the intersection of a 
plane and a sphere, namely a circle, a point and no intersection at all.
\end{example}


\begin{problem}(Problem 9)
Determine the center and radius of the sphere whose equation is given below.  
Then describe its intersection with each of the coordinate planes.
\[
x^2 - 2x + y^2 + 12y + z^2 - 8z +37 = 0
\]
The center is $\answer{(1, -6, 4)}$ and the radius is $\answer{4}$\\
The intersection of the sphere with the $xy$-plane is a 
\begin{multipleChoice}
\choice{circle}
\choice[correct]{point}
\choice{no intersection}
\end{multipleChoice}

The intersection of the sphere with the $xz$-plane is a 
\begin{multipleChoice}
\choice{circle}
\choice{point}
\choice[correct]{no intersection}
\end{multipleChoice}

The intersection of the sphere with the $yz$-plane is a 
\begin{multipleChoice}
\choice[correct]{circle}
\choice{point}
\choice{no intersection}
\end{multipleChoice}

\end{problem}


\subsection{Midpoint}
In $\R^3$, the midpoint of the segment between the points $(x_1, y_1, z_1)$ and $(x_1, y_1, z_1)$ is given by 
\[
\text{midpoint} = \left(\frac{x_1 + x_2}{2}, \frac{y_1 + y_2}{2}, \frac{z_1 + z_2}{2}\right)
\]

\begin{example}[Example 10]
Find the midpooint of the line segment connecting the points $(0,3,-5)$ and $(5,-11,9)$ in $\R^3$.\\
Letting $(x_1, y_1, z_1) = (0,3,-5)$ and $(x_2, y_2, z_2) = (5,-11,9)$ the midpoint formula gives
\[
\text{midpoint} = \left(\frac{0+5}{2}, \frac{3+(-11)}{2}, \frac{-5+9}{2}\right) = \left(\frac{5}{2}, -4,2\right)
\]
\end{example}

\begin{problem}(Problem 10)
Use the distance formula to show that the midpoint is, in fact, 
equidistant to the two points $(x_1, y_1, z_1)$ and $(x_1, y_1, z_1)$.
\end{problem}

In two dimensions, we used slopes to verify that the midpoint was actually on the line segment connecting the endpoints.  
To accomplish this in three dimensions, we need to use vectors.
\end{document}







\begin{image}
\begin{tikzpicture}
\draw[thick, ->] (0,0) -- (2,0) node[right]{$y$};
\draw[thick, ->] (0,0) -- (0,1.8) node[above]{$z$};
\draw[thick, ->] (0,0) -- (-1.2,-0.9) node[below, left]{$x$};
\draw[blue] (0.3,0.2) -- (1.7, 0.5) -- (1.7, 1.7) -- cycle;
\draw[thin, blue]  (1.6, 0.47) -- (1.6, .59) -- (1.7, .62);
\draw[fill, blue] (0.3,0.2) circle [radius=0.03] node[left, blue]{$(x_1, y_1, z_1)$};
\draw[fill, blue] (1.7,1.7) circle [radius=0.03] node[right, blue]{$(x_2, y_2, z_2)$};
\draw[fill, blue] (1.7,0.5) circle [radius=0.037] node[right, blue]{$(x_2, y_2, z_1)$};
\node[below, blue] at (1.3,0.4) {$d_1$}; %adjacent
\node[right, blue] at (1.9,1) {$d_2$}; %opposite 
%\node[right] at (4.1,2) {$\tan(\theta)$};
\node[above, blue] at (1,1.4) {$d$}; %hypotenuse
%\node[below] at (1.75,-0.6) {$d = \sqrt{(x_2 - x_1)^2 + |y_2 - y_1|^2}$};
%\node[above] at (1.75,3.6) {A right triangle with $\cos(\theta) = x$};
\end{tikzpicture}
\end{image}
