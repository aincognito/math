\documentclass[handout]{ximera}

%% You can put user macros here
%% However, you cannot make new environments



\newcommand{\ffrac}[2]{\frac{\text{\footnotesize $#1$}}{\text{\footnotesize $#2$}}}
\newcommand{\vasymptote}[2][]{
    \draw [densely dashed,#1] ({rel axis cs:0,0} -| {axis cs:#2,0}) -- ({rel axis cs:0,1} -| {axis cs:#2,0});
}


\graphicspath{{./}{firstExample/}}
\usepackage{forest}
\usepackage{amsmath}
\usepackage{amssymb}
\usepackage{array}
\usepackage[makeroom]{cancel} %% for strike outs
\usepackage{pgffor} %% required for integral for loops
\usepackage{tikz}
\usepackage{tikz-cd}
\usepackage{tkz-euclide}
\usetikzlibrary{shapes.multipart}


%\usetkzobj{all}
\tikzstyle geometryDiagrams=[ultra thick,color=blue!50!black]


\usetikzlibrary{arrows}
\tikzset{>=stealth,commutative diagrams/.cd,
  arrow style=tikz,diagrams={>=stealth}} %% cool arrow head
\tikzset{shorten <>/.style={ shorten >=#1, shorten <=#1 } } %% allows shorter vectors

\usetikzlibrary{backgrounds} %% for boxes around graphs
\usetikzlibrary{shapes,positioning}  %% Clouds and stars
\usetikzlibrary{matrix} %% for matrix
\usepgfplotslibrary{polar} %% for polar plots
\usepgfplotslibrary{fillbetween} %% to shade area between curves in TikZ



%\usepackage[width=4.375in, height=7.0in, top=1.0in, papersize={5.5in,8.5in}]{geometry}
%\usepackage[pdftex]{graphicx}
%\usepackage{tipa}
%\usepackage{txfonts}
%\usepackage{textcomp}
%\usepackage{amsthm}
%\usepackage{xy}
%\usepackage{fancyhdr}
%\usepackage{xcolor}
%\usepackage{mathtools} %% for pretty underbrace % Breaks Ximera
%\usepackage{multicol}



\newcommand{\RR}{\mathbb R}
\newcommand{\R}{\mathbb R}
\newcommand{\C}{\mathbb C}
\newcommand{\N}{\mathbb N}
\newcommand{\Z}{\mathbb Z}
\newcommand{\dis}{\displaystyle}
%\renewcommand{\d}{\,d\!}
\renewcommand{\d}{\mathop{}\!d}
\newcommand{\dd}[2][]{\frac{\d #1}{\d #2}}
\newcommand{\pp}[2][]{\frac{\partial #1}{\partial #2}}
\renewcommand{\l}{\ell}
\newcommand{\ddx}{\frac{d}{\d x}}

\newcommand{\zeroOverZero}{\ensuremath{\boldsymbol{\tfrac{0}{0}}}}
\newcommand{\inftyOverInfty}{\ensuremath{\boldsymbol{\tfrac{\infty}{\infty}}}}
\newcommand{\zeroOverInfty}{\ensuremath{\boldsymbol{\tfrac{0}{\infty}}}}
\newcommand{\zeroTimesInfty}{\ensuremath{\small\boldsymbol{0\cdot \infty}}}
\newcommand{\inftyMinusInfty}{\ensuremath{\small\boldsymbol{\infty - \infty}}}
\newcommand{\oneToInfty}{\ensuremath{\boldsymbol{1^\infty}}}
\newcommand{\zeroToZero}{\ensuremath{\boldsymbol{0^0}}}
\newcommand{\inftyToZero}{\ensuremath{\boldsymbol{\infty^0}}}


\newcommand{\numOverZero}{\ensuremath{\boldsymbol{\tfrac{\#}{0}}}}
\newcommand{\dfn}{\textbf}
%\newcommand{\unit}{\,\mathrm}
\newcommand{\unit}{\mathop{}\!\mathrm}
%\newcommand{\eval}[1]{\bigg[ #1 \bigg]}
\newcommand{\eval}[1]{ #1 \bigg|}
\newcommand{\seq}[1]{\left( #1 \right)}
\renewcommand{\epsilon}{\varepsilon}
\renewcommand{\iff}{\Leftrightarrow}

\DeclareMathOperator{\arccot}{arccot}
\DeclareMathOperator{\arcsec}{arcsec}
\DeclareMathOperator{\arccsc}{arccsc}
\DeclareMathOperator{\si}{Si}
\DeclareMathOperator{\proj}{proj}
\DeclareMathOperator{\scal}{scal}
\DeclareMathOperator{\cis}{cis}
\DeclareMathOperator{\Arg}{Arg}
%\DeclareMathOperator{\arg}{arg}
\DeclareMathOperator{\Rep}{Re}
\DeclareMathOperator{\Imp}{Im}
\DeclareMathOperator{\sech}{sech}
\DeclareMathOperator{\csch}{csch}
\DeclareMathOperator{\Log}{Log}

\newcommand{\tightoverset}[2]{% for arrow vec
  \mathop{#2}\limits^{\vbox to -.5ex{\kern-0.75ex\hbox{$#1$}\vss}}}
\newcommand{\arrowvec}{\overrightarrow}
\renewcommand{\vec}{\mathbf}
\newcommand{\veci}{{\boldsymbol{\hat{\imath}}}}
\newcommand{\vecj}{{\boldsymbol{\hat{\jmath}}}}
\newcommand{\veck}{{\boldsymbol{\hat{k}}}}
\newcommand{\vecl}{\boldsymbol{\l}}
\newcommand{\utan}{\vec{\hat{t}}}
\newcommand{\unormal}{\vec{\hat{n}}}
\newcommand{\ubinormal}{\vec{\hat{b}}}

\newcommand{\dotp}{\bullet}
\newcommand{\cross}{\boldsymbol\times}
\newcommand{\grad}{\boldsymbol\nabla}
\newcommand{\divergence}{\grad\dotp}
\newcommand{\curl}{\grad\cross}
%% Simple horiz vectors
\renewcommand{\vector}[1]{\left\langle #1\right\rangle}


\outcome{In this section we define the dot product and we use it to find the angle between vectors.}

\title{1.4 The Dot Product}



\begin{document}

\begin{abstract}
In this section we define the dot product and we use it to find the angle between vectors.
\end{abstract}
 
\maketitle
The dot product is a special operation that helps us to find the angle between two vectors.
\begin{definition}[Dot Product in $\R^2$]
If $\vec{v}_1$ and $\vec{v}_2$ are vectors in $\R^2$ given by
\[
\vec{v}_1 = \vector{x_1, y_1} \text{  and   } \;\vec{v}_2 = \vector{x_2, y_2}
\]
then the {\bf dot product} $\vec{v}_1 \dotp \vec{v}_2$ is defined by
\[
\vec{v}_1 \dotp \vec{v}_2 = x_1x_2 + y_1y_2
\]
\end{definition}

\begin{example}[Example 1]
Find the indicated dot product: $\left(3\vec{i} -2\vec{j}\right) \dotp \left(2\vec{i} +5\vec{j}\right)$\\
Rewriting the vectors in component form and then using the definition of the dot product gives
\begin{align*}
\left(3\vec{i} -2\vec{j}\right) \dotp \left(2\vec{i} +5\vec{j}\right) &= \vector{3, -2} \dotp \vector{2, 5}\\
                                                                      &= (3)(2)+(-2)(5)\\
                                                                      & = 6-10\\
                                                                      &= -4
\end{align*}
\end{example}

Note that the dot product of two vectors is a real number.  For this reason, the dot product is sometimes called the {\bf scalar product}.

\begin{problem}(Problem 1)
Find the indicated dot product: $\left(4\vec{i} + \vec{j}\right) \dotp \left(5\vec{i} -7\vec{j}\right) =\answer{13}$.\\
\end{problem}

It is interesting to compute the dot product of a vector with itself.  Let $\vec{v} = \vector{x, y}$. Then
\[
\vec{v} \dotp \vec{v} = x^2 + y^2 = |\vec{v}|^2
\]
We see that the dot product of a vector with itself gives the square of the magnitude of the vector.

It is also interesting to compute the dot product between two unit vectors in polar form.
If $\vec{u}_1 = \vector{\cos \alpha, \sin \alpha}$ and $\vec{u}_2 = \vector{\cos \beta, \sin \beta}$ then
\[
\vec{u}_1 \dotp \vec{u_2} = \cos \alpha \cos \beta + \sin \alpha \sin \beta = \cos(\alpha - \beta)
\]
We see here that the dot product of unit vectors gives the cosine of the angle between the vectors.

\begin{image}
\begin{tikzpicture}
\draw[<->] (-3, 0)--(3,0);
\draw[<->] (0,-3)--(0,3);
\draw[thin] (0,0) circle (2);
\draw[->, thick, blue] (0,0) -- (-1.732, 1) node[above left]{$\vec{u_1}$};
\draw[->, thick, red] (0,0) -- (1.414, 1.414) node[above right]{$\vec{u_2}$};
\draw[thin] (0.5 ,0.5) arc (45:150: 0.7) node[midway, above]{$\alpha - \beta$};
\draw[blue, thin] (0.3, 0) arc (0:150: 0.3) node[midway, above]{$\alpha$};
\draw[red, thin] (0.5, 0) arc (0:45: 0.5) node[midway, right]{$\beta$};
\node at(0, -3.5) {$\vec{u}_1 = \vector{\cos \alpha, \sin \alpha}$ and $\vec{u}_2 = \vector{\cos \beta, \sin \beta}$};
\draw[thin] (2, 0.2) -- (2, -0.2) node[below right]{$1$};
\draw[thin] (0.2, 2) -- (-0.2, 2) node[above left]{$1$};
\end{tikzpicture}
\end{image}

In general, the dot product of two vectors is related to both the angle between the vectors and their magnitudes.

\begin{proposition}
Given two non-zero vectors in $\R^2$, $\vec{v}_1 = \vector{x_1, y_1}$ and $\vec{v}_2 = \vector{x_2, y_2}$, we have
\[
\vec{v}_1 \dotp \vec{v}_2 = |\vec{v}_1| \cdot |\vec{v}_2| \cos \theta
\]
where $\theta$ is the angle between the two vectors.
\begin{proof}
Write the vectors in polar form: $\vec{v}_1 = |\vec{v}_1| \vector{\cos \alpha, \sin \alpha}$ and 
$\vec{v}_2 = |\vec{v}_2| \vector{\cos \beta, \sin \beta}$. To calculate the dot product, first multiply by the magnitudes:
\begin{align*}
\vec{v}_1 \dotp \vec{v}_2 & = \left(|\vec{v}_1| \vector{\cos \alpha, \sin \alpha}\right) \dotp \left( |\vec{v}_2| \vector{\cos \beta, \sin \beta} \right)\\
                            &= \vector{|\vec{v}_1|\cos \alpha, |\vec{v}_1|\sin \alpha} \dotp   \vector{|\vec{v}_2|\cos \beta, |\vec{v}_2|\sin \beta} \\
                            &= |\vec{v}_1| \cos \alpha |\vec{v}_2|\cos \beta + |\vec{v}_1|\sin \alpha |\vec{v}_2|\sin \beta\\
                            &= |\vec{v}_1|\cdot |\vec{v}_2| \left(\cos \alpha \cos \beta + \sin \alpha \sin \beta\right)\\
                            &= |\vec{v}_1| \cdot|\vec{v}_2| \cos(\alpha - \beta)\\
                            &= |\vec{v}_1| \cdot |\vec{v}_2| \cos \theta
\end{align*}
where $\theta = \alpha - \beta$ is the angle between $\vec{v}_1$ and $\vec{v}_2$.
\end{proof}
\end{proposition}
\begin{remark} The angle $\theta$ between two vectors is always taken to be between $0^\circ$ and $180^\circ$, i.e.
\[
0^\circ \leq \theta \leq 180^\circ \quad \text{or, in radians} \quad 0 \leq \theta \leq \pi
\]
\end{remark}

\begin{example}[Example 2] Use the dot product to find the angle between the vectors 
\[
\vec v = \vector{-2,5} \quad \text{and} \quad \vec w = \vector{5,12}
\]
Round the answer to the nearest tenth of a degree.\\
Solving the dot product formula in the previous proposition for $\cos \theta$ gives
\[
\cos \theta = \frac{\vec v \dotp \vec w}{|\vec v| \cdot |\vec w|}
\]
where $\theta$ is the angle between the vectors. The numerator is
\[
\vec v \dotp \vec w = (-2)(5) + (5)(12) = 50
\]
and the magnitudes in the denominator are 
\[
|\vec v| = \sqrt{ 4 + 25} = \sqrt{29} \quad \text{and} \quad |\vec w| = \sqrt{25 +144 } = 13
\]
Hence, the cosine of the angle is
\[
\cos \theta = \frac{50}{13\sqrt{29}}
\]
and $\theta$ is found using the inverse cosine:
\[
\theta = \cos^{-1}\left(\frac{50}{13\sqrt{29}} \right) \approx 44.4^\circ
\]
to the nearest tenth of a degree.
\end{example}

\begin{problem}(Problem 2) Use the dot product to find the angle between the vectors 
\[
\vec v = \vector{3,-6} \quad \text{and} \quad \vec w = \vector{2,4}
\]
Round the answer to the nearest tenth of a degree.\\
\[
\vec v \dotp \vec w = \answer{-18} \quad |v| = \answer{3\sqrt{5}} \quad |w| = \answer{2\sqrt{5}}
\]
\[
\cos \theta = \answer{\frac{-3}{5}} \quad \theta \approx \answer{126.7} \text{ degrees}
\]
\end{problem}


In three dimensions, a vector does not have a polar form, but we can still use the dot product to find the angle between two vectors.
\begin{definition}[Dot Product in $\R^3$]
If $\vec{v}_1$ and $\vec{v}_2$ are vectors in $\R^3$ given by
\[
\vec{v}_1 = \vector{x_1, y_1, z_1} \text{  and   } \;\vec{v}_2 = \vector{x_2, y_2, z_2}
\]
then the dot product $\vec{v}_1 \dotp \vec{v}_2$ is defined by
\[
\vec{v}_1 \dotp \vec{v}_2 = x_1x_2 + y_1y_2 + z_1z_2
\]
\end{definition}

\begin{example}[Example 3]
Compute the indicated dot products:\\
a) $\vector{1, 2, 3} \dotp \vector{5, -2, 4} = (1)(5)+ (2)(-2) + (3)(4) = 13$\\
b) $\left(3\vec{i} - 2\vec{k}\right) \dotp \left(4\vec{j} + 5\vec{k}\right) = (3)(0) + (0)(4) + (-2)(5) = -10$
\end{example}

\begin{problem}(Problem 3)
Compute the indicated dot products:\\
a) $\vector{6, -1, 4} \dotp \vector{-3, 7, 2} = \answer{-17}$\\
b) $\left(2\vec{i} + 4\vec{j} - 3\vec{k}\right) \dotp \left(\vec{i} -5\vec{j}\right) = \answer{-18}$
\end{problem}

\begin{proposition}[Properties of the Dot Product]
Let $\vec{v} , \vec{v}_1 $ and $ \vec{v}_2$ be vectors in $\R^3$ and let $c$ be any scalar. 
Then the following properties hold for the dot product:\\
i) $\vec{v}_1 \dotp \vec{v}_2 =\vec{v}_2 \dotp \vec{v}_1$\\
ii) $\left(c\vec{v}_1\right) \dotp \vec{v}_2 =\vec{v}_1 \dotp \left(c\vec{v}_2\right) = c\left(\vec{v}_1 \dotp \vec{v}_2\right)$\\
iii)  $\vec{v} \dotp \left(\vec{v}_1 +\vec{v}_2\right) =\vec{v} \dotp\vec{v}_1 +\vec{v}\dotp \vec{v}_2$\\
iv) $\vec{v} \dotp \vec{0} = 0$\\
v) $\vec{v} \dotp \vec{v} = |\vec{v}|^2$
\end{proposition}

Suppose that $\vec{v}_1=\vector{x_1, y_1, z_1}$ and $\vec{v}_2=\vector{x_2, y_2, z_2}$  are vectors in $\R^3$ placed in standard position.  
Then the initial point of both vectors is the origin, $(0,0,0)$ and the final points are
$(x_1, y_1, z_1)$ and $(x_2, y_2, z_2)$.  Together, these three points form a triangle in $\R^3$.  
The angle between the vectors $\vec{v}_1$ and $\vec{v}_2$ is the angle in this triangle with vertex at the origin.

\begin{proposition}
Let $\vec{v}_1 = \vector{x_1, y_1, z_1}$ and $\vec{v}_2 = \vector{x_2, y_2, z_2}$
be vectors in $\R^3$ and let $\theta$ denote the angle between these vectors.  Then, just as in $\R^2$, the dot product can be expressed in terms of the 
magnitudes of the vectors $\vec{v}_1$ and $\vec{v}_2$ and the angle $\theta$ between them as follows:
\[
\vec{v}_1\dotp\vec{v}_2 = |\vec{v}_1|\cdot |\vec{v}_2| \cos\theta
\]
\begin{proof}
According to the \link[law of cosines]{https://en.wikipedia.org/wiki/Law_of_cosines}, 
\[
|\vec{v}_1 - \vec{v}_2|^2 = |\vec{v}_1|^2 + |\vec{v}_2|^2 - 2|\vec{v}_1|\cdot| \vec{v}_2|\cos \theta
\]
See the figure below. By property $(v)$ of the preceding proposition, 
\[
|\vec{v}_1 - \vec{v}_2|^2 = \left(\vec{v}_1 - \vec{v}_2\right)\dotp \left(\vec{v}_1 - \vec{v}_2\right)
\]
which then by properties $(ii)$ and $(iii)$ of that same proposition then equals
\[
\vec{v}_1 \dotp \vec{v}_1 + \vec{v}_2 \dotp \vec{v}_2 - 2\left(\vec{v}_1 \dotp \vec{v}_2\right)
\]
Using property $(v)$ again, this can be rewritten using magnitudes as follows
\[
|\vec{v}_1|^2 + |\vec{v}_2|^2 - 2\left(\vec{v}_1 \dotp \vec{v}_2\right)
\]
Comparing this with right hand side of the law of cosines we can see that
\[
- 2\left(\vec{v}_1 \dotp \vec{v}_2\right) = - 2|\vec{v}_1|\cdot| \vec{v}_2|\cos \theta
\]
Diving through by $-2$ gives the final result.

\begin{image}
\begin{tikzpicture}
\draw[blue, ->, thick] (0,0) -- (3, 1) node[midway, below]{$\vec{v}_1$};
\draw[blue, ->, thick] (0,0) -- (1, 4) node[midway, left]{$\vec{v}_2$};
\draw[red, ->, thick] (1,4) -- (3, 1) node[midway,right]{$\vec{v}_1 - \vec{v}_2$};
\draw[blue] (0.8,.28) ++(2:0) arc (20:70:1);
\path[blue] (0,0.33) ++(10:0.4cm) node{$\theta$};
%\draw[blue] (0.5, 0.166) arc (20:70:.6) node[midway, above right]{$\theta$};
\node at (1.5, -1){Law of Cosines:};
\node at (1.5, -1.5){ $|\vec{v}_1 - \vec{v}_2|^2 = |\vec{v}_1|^2 + |\vec{v}_2|^2 - 2|\vec{v}_1|\cdot| \vec{v}_2|\cos \theta$};
%\draw[white] (-3, -1.5) -- (7, -1.5);
\end{tikzpicture}
\end{image}

\end{proof}

\end{proposition}

From the above proposition, we see that the dot product of two vectors is zero under special conditions, i.e.,  if 
\[
\vec{v}_1 \dotp \vec{v}_2 = 0
\]
then either
\[
|\vec{v}_1| = 0, |\vec{v}_2| = 0 \; \text{or } \; \cos \theta = 0
\]
Noting that $\cos \theta = 0$ precisely when $\theta = 90^\circ$, we can conclude the following
\begin{corollary} 
Two non-zero vectors $\vec{v}_1$ and $\vec{v}_2$ are perpendicular if and only if $\vec{v}_1 \dotp \vec{v}_2 = 0$.
\end{corollary}
Perpendicular vectors are commonly called {\bf orthogonal}.\\

\begin{example}[Example 4]
Find the value of $z$ that makes the vectors orthogonal:
\[
\vec v = \vector{-1,-2, 4} \quad \text{and} \quad \vec w = \vector{2,3, z}
\]
Computing the dot product gives
\[
\vec v \dotp \vec w = (-1)(2) + (-2)(3) + (4)(z) = -8+4z
\]
The corollary above tells us that the vectors $\vec v$ and $\vec w$ are orthogonal in $\R^3$ if and only if their 
dot product is zero. Solving $-8 + 4z = 0$ for $z$
gives $z = 2$. Hence the vectors $\vec v = \vector{-1,-2, 4}$ and $\vec w = \vector{2,3, 2}$ are orthogonal.
\end{example}

\begin{problem}(Problem 4a) 
Find the value of $y$ that makes the vectors orthogonal:
\[
\vec v = \vector{3,8, 7} \quad \text{and} \quad \vec w = \vector{-4,y, 5}
\]
\[
y = \answer{-\frac{23}{8}}
\]
\end{problem}

\begin{problem}(Problem 4b) Find two vectors, not scalar multiples of one another, that are both orthogonal to the 
vector $\vector{1,1,1}$ in $\R^3$.
\end{problem}


The shortest distance between two points in $\R^3$ is a straight line.  This basic fact is illustrated in the following proposition.

\begin{proposition}[Triangle Inequality]
Let $\vec{v}$ and $\vec{w}$ be vectors in $\R^3$. Then the following inequality holds:
\[
|\vec{v} + \vec{w}| \leq |\vec{v}| +|\vec{w}| 
\]



\begin{proof}
Let $\theta$ be the angle between $\vec v$ and $\vec w$. Using the properties of the dot product in $\R^3$, we can write the following
\begin{align*}
|\vec{v} + \vec{w}|^2 &= \left(\vec{v} + \vec{w}\right) \dotp \left(\vec{v} + \vec{w}\right)\\
                        &= \vec{v} \dotp \vec{v} + \vec{w}\dotp \vec{w} + 2\vec{v} \dotp \vec{w}\\
                        &= |\vec{v}|^2 +|\vec{w}|^2 + 2|\vec{v}|\cdot |\vec{w}| \cos \theta\\
                        &\leq |\vec{v}|^2 +|\vec{w}|^2 + 2|\vec{v}|\cdot |\vec{w}|\\
                        &= \left(|\vec{v}| + |\vec{w}|\right)^2
\end{align*}
The result follows immediately by taking the square root.
          
\begin{image}
\begin{tikzpicture}
\draw[blue, ->, thick] (0,0) -- (3, 1) node[midway, below]{$\vec{v}$};
\draw[red, ->, thick] (0,0) -- (1, 4) node[midway, left]{$\vec{v} +\vec{w}$};
\draw[blue, <-, thick] (1,4) -- (3, 1) node[midway,right]{$\vec{w}$};
%\draw[blue, ->, thick] (0,0) -- (2, 1) node[midway, below]{$\vec{v}$};
%\draw[blue, ->, thick] (2, 1) -- (3, 3) node[midway, right]{$\vec{w}$};
%\draw[red, ->, thick] (0,0) -- (3,3) node[midway, above left]{$\vec{v} + \vec{w}$};
\node at (1.5, -1){Triangle Inequality: $|\vec{v} + \vec{w}| \leq |\vec{v}| +|\vec{w}| $};
\end{tikzpicture}
\end{image}

              
\end{proof}

\end{proposition}

Note that the proof gives us a little bit more information about the inequality. 
If $\cos \theta = 1$, then the inequality becomes equality.  This happens precisely 
when the vectors $\vec{v}$ and $\vec{w}$ are in the same direction, i.e., when $\theta = 0$. 
The result holds for vectors in $\R^2$ as well using the same proof.

\section{Projection}

An important application of the dot product is to compute the projection of one vector onto another.
The diagrams below show the projection of the vector $\vec{w}$ onto the vector $\vec{v}$. 
This projection vector is denoted $\proj_{\vec{v}} \vec{w}$
and it is parallel to $\vec{v}$.

\begin{image}
\begin{tikzpicture}
\draw[red, ->, thick] (0,0) -- (3, 0) node[right]{$\vec{v}$};
\draw[blue, ->, thick] (0,0) -- (2, 4) node[midway, left]{$\vec{w}$};
\draw[blue, ->, thin] (0,0) --(2,0) node[midway, below]{$\proj_{\vec{v}} \vec{w}$};
\draw[blue, thin, dashed] (2, 4) -- (2, 0);
\draw[blue, thin] (1.7, 0) -- ++(0, 0.3) -- + (0.3, 0);
\node at (1.5, -1){Case 1: $\vec{v} \dotp \vec{w} > 0$};

\draw[red, ->, thick] (8,0) -- (11, 0) node[right]{$\vec{v}$};
\draw[red, thick, dashed] (8,0) -- (6, 0);
\draw[blue, ->, thick] (8,0) -- (6, 4) node[midway, right]{$\vec{w}$};
\draw[blue, ->] (8,0) -- (6, 0) node[below, midway]{$\proj_{\vec{v}} \vec{w}$};
\draw[blue, thin, dashed] (6, 4) -- (6, 0);

\draw[blue, thin] (6.3, 0) -- ++(0, 0.3) -- + (-0.3, 0);
\node at (8.5, -1){Case 2: $\vec{v} \dotp \vec{w} < 0$};
\end{tikzpicture}
\end{image}

\begin{proposition}[Projection Vector]
Given vectors $\vec{v} \neq \vec{0}$ and $\vec{w}$ in $\R^3$ (or $\R^2$), the projection of $\vec{w}$ onto $\vec{v}$ is given by
\[
\proj_{\vec{v}} \vec{w} = \left(\frac{\vec{v} \dotp \vec{w}}{\vec{v} \dotp \vec{v}}\right) \vec{v}
\]


\begin{proof}
Note that the projection vector is parallel to $\vec{v}$ so it has the form $\lambda \vec{v}$ for some scalar $\lambda$.
To determine the value of $\lambda$, we will use the fact that the vectors $\vec{v}$ and $\vec{w} - \proj_{\vec{v}} \vec{w}$ are orthogonal, 
so that their dot product is zero. We have
\begin{align*}
0 &= \vec{v} \dotp \left(\vec{w} - \proj_{\vec{v}} \vec{w} \right)\\
  &= \vec{v} \dotp \left(\vec{w} - \lambda \vec{v}\right)\\
  &= \vec{v} \dotp \vec{w} - \lambda \left( \vec{v} \dotp \vec{v} \right)
\end{align*}
Solving for $\lambda$ gives
\[
\lambda = \frac{\vec{v} \dotp \vec{w}}{\vec{v} \dotp \vec{v}}
\]
as desired.
\end{proof}
\end{proposition}

\begin{example}[Example 5]
Let $\vec v = \vector{3,4,7}$ and $\vec w = \vector{8,6,-5}$. Find $\proj_{\vec{v}} \vec{w}$\\
Computing the relevant dot products gives
\[
\vec v \dotp \vec w = 24+24 -35 = 13 \quad \text{and} \quad \vec v \dotp \vec v =9 + 16+49 = 74
\]
Hence
\begin{align*}
\proj_{\vec{v}} \vec{w} &= \left(\frac{\vec{v} \dotp \vec{w}}{\vec{v} \dotp \vec{v}}\right) \vec{v}\\
                        &= \left(\frac{13}{74}\right) \vec v\\
                        &= \frac{13}{74} \vector{3,4,7}\\
                        &= \vector{\frac{39}{74},\frac{26}{37},\frac{91}{74}}
\end{align*}
\end{example}

\begin{problem}(Problem 5)
Let $\vec v = \vector{1,2,3}$ and $\vec w = \vector{2,5,10}$. Find $\proj_{\vec{v}} \vec{w}$\\
\[
\vec v \dotp \vec w = \answer{42} \quad \vec v \dotp \vec v = \answer{129} 
\]
\[
 \proj_{\vec{v}} \vec{w} = \vector{\answer{\frac{42}{129}},\answer{\frac{84}{129}},\answer{\frac{126}{129}}}
\]
\end{problem}


\end{document}


\begin{tikzpicture}[->]
\draw (0,0) -- (xyz cs:x=1);
\draw (0,0) -- (xyz cs:y=1);
\draw (0,0) -- (xyz cs:z=1);
\end{tikzpicture}

\begin{tikzpicture}
\draw (-1,0) -- +(3.5,0);
\draw (1,0) ++(210:2cm) -- +(30:4cm);
\draw (1,0) +(0:1cm) arc (0:30:1cm);
\draw (1,0) +(180:1cm) arc (180:210:1cm);
\path (1,0) ++(15:.75cm) node{$\alpha$};
\path (1,0) ++(15:-.75cm) node{$\beta$};
\end{tikzpicture}
