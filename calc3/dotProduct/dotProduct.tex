\documentclass[handout]{ximera}

%% You can put user macros here
%% However, you cannot make new environments



\newcommand{\ffrac}[2]{\frac{\text{\footnotesize $#1$}}{\text{\footnotesize $#2$}}}
\newcommand{\vasymptote}[2][]{
    \draw [densely dashed,#1] ({rel axis cs:0,0} -| {axis cs:#2,0}) -- ({rel axis cs:0,1} -| {axis cs:#2,0});
}


%\usepackage{tcolorbox} %%Needed for Derivative Definition supposedly and product rule, natural exp log, quotient rule, inverse trig, rates of change


% \graphicspath{{./}{firstExample/}}
% \usepackage{forest}
\usepackage{amsmath}
\usepackage{amssymb}
\usepackage{array}
\usepackage[makeroom]{cancel} %% for strike outs
\usepackage{pgffor} %% required for integral for loops
\usepackage{tikz}
\usepackage{tikz-cd}
\usepackage{tkz-euclide}
\usetikzlibrary{shapes.multipart}


% \usetkzobj{all}
\tikzstyle geometryDiagrams=[ultra thick,color=blue!50!black]


\usetikzlibrary{arrows}
\tikzset{>=stealth,commutative diagrams/.cd,
  arrow style=tikz,diagrams={>=stealth}} %% cool arrow head
\tikzset{shorten <>/.style={ shorten >=#1, shorten <=#1 } } %% allows shorter vectors

\usetikzlibrary{backgrounds} %% for boxes around graphs
\usetikzlibrary{shapes,positioning}  %% Clouds and stars
\usetikzlibrary{matrix} %% for matrix
\usepgfplotslibrary{polar} %% for polar plots
\usepgfplotslibrary{fillbetween} %% to shade area between curves in TikZ



%\usepackage[width=4.375in, height=7.0in, top=1.0in, papersize={5.5in,8.5in}]{geometry}
%\usepackage[pdftex]{graphicx}
%\usepackage{tipa}
%\usepackage{txfonts}
%\usepackage{textcomp}
%\usepackage{amsthm}
%\usepackage{xy}
%\usepackage{fancyhdr}
%\usepackage{xcolor}
%\usepackage{mathtools} %% for pretty underbrace % Breaks Ximera
%\usepackage{multicol}



\newcommand{\RR}{\mathbb R}
\newcommand{\R}{\mathbb R}
\newcommand{\C}{\mathbb C}
\newcommand{\N}{\mathbb N}
\newcommand{\Z}{\mathbb Z}
\newcommand{\dis}{\displaystyle}
%\renewcommand{\d}{\,d\!}
\renewcommand{\d}{\mathop{}\!d}
\newcommand{\dd}[2][]{\frac{\d #1}{\d #2}}
\newcommand{\pp}[2][]{\frac{\partial #1}{\partial #2}}
\renewcommand{\l}{\ell}
\newcommand{\ddx}{\frac{d}{\d x}}
\newcommand{\ppx}{\frac{\partial}{\partial x}}
\newcommand{\ppy}{\frac{\partial}{\partial y}}

\newcommand{\zeroOverZero}{\ensuremath{\boldsymbol{\tfrac{0}{0}}}}
\newcommand{\inftyOverInfty}{\ensuremath{\boldsymbol{\tfrac{\infty}{\infty}}}}
\newcommand{\zeroOverInfty}{\ensuremath{\boldsymbol{\tfrac{0}{\infty}}}}
\newcommand{\zeroTimesInfty}{\ensuremath{\small\boldsymbol{0\cdot \infty}}}
\newcommand{\inftyMinusInfty}{\ensuremath{\small\boldsymbol{\infty - \infty}}}
\newcommand{\oneToInfty}{\ensuremath{\boldsymbol{1^\infty}}}
\newcommand{\zeroToZero}{\ensuremath{\boldsymbol{0^0}}}
\newcommand{\inftyToZero}{\ensuremath{\boldsymbol{\infty^0}}}


\newcommand{\numOverZero}{\ensuremath{\boldsymbol{\tfrac{\#}{0}}}}
\newcommand{\dfn}{\textbf}
%\newcommand{\unit}{\,\mathrm}
\newcommand{\unit}{\mathop{}\!\mathrm}
%\newcommand{\eval}[1]{\bigg[ #1 \bigg]}
\newcommand{\eval}[1]{ #1 \bigg|}
\newcommand{\seq}[1]{\left( #1 \right)}
\renewcommand{\epsilon}{\varepsilon}
\renewcommand{\iff}{\Leftrightarrow}

\DeclareMathOperator{\arccot}{arccot}
\DeclareMathOperator{\arcsec}{arcsec}
\DeclareMathOperator{\arccsc}{arccsc}
\DeclareMathOperator{\si}{Si}
\DeclareMathOperator{\proj}{proj}
\DeclareMathOperator{\scal}{scal}
\DeclareMathOperator{\cis}{cis}
\DeclareMathOperator{\Arg}{Arg}
%\DeclareMathOperator{\arg}{arg}
\DeclareMathOperator{\Rep}{Re}
\DeclareMathOperator{\Imp}{Im}
\DeclareMathOperator{\sech}{sech}
\DeclareMathOperator{\csch}{csch}
\DeclareMathOperator{\Log}{Log}

\newcommand{\tightoverset}[2]{% for arrow vec
  \mathop{#2}\limits^{\vbox to -.5ex{\kern-0.75ex\hbox{$#1$}\vss}}}
\newcommand{\arrowvec}{\overrightarrow}
\renewcommand{\vec}{\mathbf}
\newcommand{\veci}{{\boldsymbol{\hat{\imath}}}}
\newcommand{\vecj}{{\boldsymbol{\hat{\jmath}}}}
\newcommand{\veck}{{\boldsymbol{\hat{k}}}}
\newcommand{\vecl}{\boldsymbol{\l}}
\newcommand{\utan}{\vec{\hat{t}}}
\newcommand{\unormal}{\vec{\hat{n}}}
\newcommand{\ubinormal}{\vec{\hat{b}}}

\newcommand{\dotp}{\bullet}
\newcommand{\cross}{\boldsymbol\times}
\newcommand{\grad}{\boldsymbol\nabla}
\newcommand{\divergence}{\grad\dotp}
\newcommand{\curl}{\grad\cross}
%% Simple horiz vectors
\renewcommand{\vector}[1]{\left\langle #1\right\rangle}


\outcome{In this section we define the dot product and we use it to find the angle between vectors.}

\title{1.4 The Dot Product}



\begin{document}

\begin{abstract}
In this section we define the dot product and we use it to find the angle between vectors.
\end{abstract}
 
\maketitle
The dot product is a special operation that helps us to find the angle between two vectors.
\begin{definition}
If $\avec{v_1}$ and $\avec{v_2}$ are vectors in $\R^2$ given by
\[
\avec{v_1} = \vector{x_1, y_1} \text{  and  } \avec{v_2} = \vector{x_2, y_2}
\]
then the dot product $\avec{v_1} \dotp \avec{v_2}$ is defined by
\[
\avec{v_1} \dotp \avec{v_2} = x_1x_2 + y_1y_2
\]
\end{definition}

\begin{example}
Find the indicated dot product: $\left(3\avec{i} -2\avec{j}\right) \dotp \left(2\avec{i} +5\avec{j}\right)$.\\
According to the definition, we have
\[
\left(3\avec{i} -2\avec{j}\right) \dotp \left(2\avec{i} +5\avec{j}\right) = (3)(2)+(-2)(5) = 6-10 = -4
\]
\end{example}

Note that the dot product of two vectors is a real number.  For this reason, the dot product is sometimes called the scalar product.

\begin{problem}
Find the indicated dot product: $\left(4\avec{i} + \avec{j}\right) \dotp \left(5\avec{i} -7\avec{j}\right) =\answer{13}$.\\
\end{problem}

It is interesting to compute the dot product of a vector with itself.  Let $\avec{v} = \vector{x, y}$. Then
\[
\avec{v} \dotp \avec{v} = x^2 + y^2 = |\avec{v}|^2
\]
We see that the dot product of a vector with itself gives the square of the magnitude of the vector.

It is also interesting to compute the dot product between two unit vectors in polar form.
If $\avec{u_1} = \vector{\cos \alpha, \sin \alpha}$ and $\avec{u_2} = \vector{\cos \beta, \sin \beta}$ then
\[
\avec{u_1} \dotp \avec{u_2} = \cos \alpha \cos \beta + \sin \alpha \sin \beta = \cos(\alpha - \beta)
\]
We see here that the dot product of unit vectors gives the cosine of the angle between the vectors.

\begin{image}
\begin{tikzpicture}
\draw[<->] (-3, 0)--(3,0);
\draw[<->] (0,-3)--(0,3);
\draw[thin, red] (0,0) circle (2);
\draw[->, thick, blue] (0,0) -- (-1.732, 1) node[above left]{$\avec{u_1}$};
\draw[->, thick, blue] (0,0) -- (1.414, 1.414) node[above right]{$\avec{u_2}$};
\draw[blue, thin] (0.3 ,0.3) arc (45:150: 0.42) node[midway, above]{$\alpha - \beta$};
\node at(0, -3.5) {$\avec{u_1} = \vector{\cos \alpha, \sin \alpha}$ and $\avec{u_2} = \vector{\cos \beta, \sin \beta}$};
\draw[thin] (2, 0.2) -- (2, -0.2) node[below right]{$1$};
\draw[thin] (0.2, 2) -- (-0.2, 2) node[above left]{$1$};
\end{tikzpicture}
\end{image}

In general, the dot product of two vectors is related to both the angle between the vectors and thier magnitudes.

\begin{proposition}
Given two non-zero vectors in $\R^2$, $\avec{v_1} = \vector{x_1, y_1}$ and $\avec{v_2} = \vector{x_2, y_2}$, we have
\[
\avec{v_1} \dotp \avec{v_2} = |\avec{v_1}| \cdot |\avec{v_2}| \cos \theta
\]
where $\theta$ is the angle between the two vectors.
\begin{proof}
Write the vectors in polar form: $\avec{v_1} = |\avec{v_1}| \vector{\cos \alpha, \sin \alpha}$ and 
$\avec{v_2} = |\avec{v_2}| \vector{\cos \beta, \sin \beta}$. To calculuate the dot product, first multiply by the magnitudes:
\begin{align*}
\avec{v_1} \cdot \avec{v_2} & = \left(|\avec{v_1}| \vector{\cos \alpha, \sin \alpha}\right) \dotp \left( |\avec{v_2}| \vector{\cos \beta, \sin \beta} \right)\\
                            &= \vector{|\avec{v_1}|\cos \alpha, |\avec{v_1}|\sin \alpha} \dotp   \vector{|\avec{v_2}|\cos \beta, |\avec{v_2}|\sin \beta} \\
                            &= |\avec{v_1}| \cos \alpha |\avec{v_2}|\cos \beta + |\avec{v_1}|\sin \alpha |\avec{v_2}|\sin \beta\\
                            &= |\avec{v_1}|\cdot |\avec{v_2}| \left(\cos \alpha \cos \beta + \sin \alpha \sin \beta\right)\\
                            &= |\avec{v_1}| \cdot|\avec{v_2}| \cos(\alpha - \beta)\\
                            &= |\avec{v_1}| \cdot |\avec{v_2}| \cos(\theta)
\end{align*}
where $\theta = \alpha - \beta$ is the angle between $\avec{v_1}$ and $\avec{v_2}$.
\end{proof}
\end{proposition}

\end{document}




