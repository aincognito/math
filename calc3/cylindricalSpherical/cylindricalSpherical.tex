\documentclass[handout]{ximera}

%% You can put user macros here
%% However, you cannot make new environments



\newcommand{\ffrac}[2]{\frac{\text{\footnotesize $#1$}}{\text{\footnotesize $#2$}}}
\newcommand{\vasymptote}[2][]{
    \draw [densely dashed,#1] ({rel axis cs:0,0} -| {axis cs:#2,0}) -- ({rel axis cs:0,1} -| {axis cs:#2,0});
}


%\usepackage{tcolorbox} %%Needed for Derivative Definition supposedly and product rule, natural exp log, quotient rule, inverse trig, rates of change


% \graphicspath{{./}{firstExample/}}
% \usepackage{forest}
\usepackage{amsmath}
\usepackage{amssymb}
\usepackage{array}
\usepackage[makeroom]{cancel} %% for strike outs
\usepackage{pgffor} %% required for integral for loops
\usepackage{tikz}
\usepackage{tikz-cd}
\usepackage{tkz-euclide}
\usetikzlibrary{shapes.multipart}


% \usetkzobj{all}
\tikzstyle geometryDiagrams=[ultra thick,color=blue!50!black]


\usetikzlibrary{arrows}
\tikzset{>=stealth,commutative diagrams/.cd,
  arrow style=tikz,diagrams={>=stealth}} %% cool arrow head
\tikzset{shorten <>/.style={ shorten >=#1, shorten <=#1 } } %% allows shorter vectors

\usetikzlibrary{backgrounds} %% for boxes around graphs
\usetikzlibrary{shapes,positioning}  %% Clouds and stars
\usetikzlibrary{matrix} %% for matrix
\usepgfplotslibrary{polar} %% for polar plots
\usepgfplotslibrary{fillbetween} %% to shade area between curves in TikZ



%\usepackage[width=4.375in, height=7.0in, top=1.0in, papersize={5.5in,8.5in}]{geometry}
%\usepackage[pdftex]{graphicx}
%\usepackage{tipa}
%\usepackage{txfonts}
%\usepackage{textcomp}
%\usepackage{amsthm}
%\usepackage{xy}
%\usepackage{fancyhdr}
%\usepackage{xcolor}
%\usepackage{mathtools} %% for pretty underbrace % Breaks Ximera
%\usepackage{multicol}



\newcommand{\RR}{\mathbb R}
\newcommand{\R}{\mathbb R}
\newcommand{\C}{\mathbb C}
\newcommand{\N}{\mathbb N}
\newcommand{\Z}{\mathbb Z}
\newcommand{\dis}{\displaystyle}
%\renewcommand{\d}{\,d\!}
\renewcommand{\d}{\mathop{}\!d}
\newcommand{\dd}[2][]{\frac{\d #1}{\d #2}}
\newcommand{\pp}[2][]{\frac{\partial #1}{\partial #2}}
\renewcommand{\l}{\ell}
\newcommand{\ddx}{\frac{d}{\d x}}
\newcommand{\ppx}{\frac{\partial}{\partial x}}
\newcommand{\ppy}{\frac{\partial}{\partial y}}

\newcommand{\zeroOverZero}{\ensuremath{\boldsymbol{\tfrac{0}{0}}}}
\newcommand{\inftyOverInfty}{\ensuremath{\boldsymbol{\tfrac{\infty}{\infty}}}}
\newcommand{\zeroOverInfty}{\ensuremath{\boldsymbol{\tfrac{0}{\infty}}}}
\newcommand{\zeroTimesInfty}{\ensuremath{\small\boldsymbol{0\cdot \infty}}}
\newcommand{\inftyMinusInfty}{\ensuremath{\small\boldsymbol{\infty - \infty}}}
\newcommand{\oneToInfty}{\ensuremath{\boldsymbol{1^\infty}}}
\newcommand{\zeroToZero}{\ensuremath{\boldsymbol{0^0}}}
\newcommand{\inftyToZero}{\ensuremath{\boldsymbol{\infty^0}}}


\newcommand{\numOverZero}{\ensuremath{\boldsymbol{\tfrac{\#}{0}}}}
\newcommand{\dfn}{\textbf}
%\newcommand{\unit}{\,\mathrm}
\newcommand{\unit}{\mathop{}\!\mathrm}
%\newcommand{\eval}[1]{\bigg[ #1 \bigg]}
\newcommand{\eval}[1]{ #1 \bigg|}
\newcommand{\seq}[1]{\left( #1 \right)}
\renewcommand{\epsilon}{\varepsilon}
\renewcommand{\iff}{\Leftrightarrow}

\DeclareMathOperator{\arccot}{arccot}
\DeclareMathOperator{\arcsec}{arcsec}
\DeclareMathOperator{\arccsc}{arccsc}
\DeclareMathOperator{\si}{Si}
\DeclareMathOperator{\proj}{proj}
\DeclareMathOperator{\scal}{scal}
\DeclareMathOperator{\cis}{cis}
\DeclareMathOperator{\Arg}{Arg}
%\DeclareMathOperator{\arg}{arg}
\DeclareMathOperator{\Rep}{Re}
\DeclareMathOperator{\Imp}{Im}
\DeclareMathOperator{\sech}{sech}
\DeclareMathOperator{\csch}{csch}
\DeclareMathOperator{\Log}{Log}

\newcommand{\tightoverset}[2]{% for arrow vec
  \mathop{#2}\limits^{\vbox to -.5ex{\kern-0.75ex\hbox{$#1$}\vss}}}
\newcommand{\arrowvec}{\overrightarrow}
\renewcommand{\vec}{\mathbf}
\newcommand{\veci}{{\boldsymbol{\hat{\imath}}}}
\newcommand{\vecj}{{\boldsymbol{\hat{\jmath}}}}
\newcommand{\veck}{{\boldsymbol{\hat{k}}}}
\newcommand{\vecl}{\boldsymbol{\l}}
\newcommand{\utan}{\vec{\hat{t}}}
\newcommand{\unormal}{\vec{\hat{n}}}
\newcommand{\ubinormal}{\vec{\hat{b}}}

\newcommand{\dotp}{\bullet}
\newcommand{\cross}{\boldsymbol\times}
\newcommand{\grad}{\boldsymbol\nabla}
\newcommand{\divergence}{\grad\dotp}
\newcommand{\curl}{\grad\cross}
%% Simple horiz vectors
\renewcommand{\vector}[1]{\left\langle #1\right\rangle}


\outcome{In this section we compute triple integrals using cylindrical and spherical coordinates.}

\title{4.5 Cylindrical and Spherical Coordinates}



\begin{document}

\begin{abstract}
In this section we compute triple integrals using cylindrical and spherical coordinates.
\end{abstract}
 
\maketitle

\section{Cylindrical Coordinates}
Cylindrical coordinates are an extension of polar coordinates to $\R^3$ by directly adding the $z$ component of a point.
Thus, a point in $\R^3$ can be represented in cylindrical coordinates as $(r, \theta, z)$ 
where $r$ and $\theta$ are the corresponding polar coordinates of the point 
$(x,y)$ in $\R^2$.  The point $(x,y)$ in $\R^2$ can be thought of as the projection of the point $(x,y,z)$ onto the $xy$-plane.
See the figure below.

\begin{image}
\begin{tikzpicture}
\draw[thick, ->] (0,0,0) -- (0,0, 5) node[below]{$x$};
\draw[thick, ->] (0,0,0) -- (0, 5, 0) node[above]{$z$};
\draw[thick, ->] (0,0,0) -- (5,0,0) node[right]{$y$};
%\draw[thick, ->] (0, 0) -- (-2.3, -2.3) node[below, left]{$x$};
\draw[thick, blue!70!white] (0,0) -- (2,3) node[above right, black]{$(r, \theta, z)$};
\draw[dashed] (2, 3) -- (0, 3) node[left]{$z$};
\draw[dashed] (2,3) -- (2, -1) node[right]{$(r, \theta)$} ;
\draw[dashed] (0,0) -- (2,-1) node[midway, below]{$r$};
\draw (-.35, -.35) arc(-135:-28:.495)node[midway, below]{$\theta$};
\filldraw (2, -1) circle (0.05);
\filldraw[blue!70!white] (2, 3) circle (0.05);


\end{tikzpicture}
\end{image}


The relationships between the rectangular coordinates $(x,y,z)$ and the cylindrical coordinates $(r, \theta, z)$ carry over from 
polar coordinates, i.e.,
\[
x = r\cos \theta \quad y = r\sin\theta \quad \text{and} \quad z = z
\]
\[
x^2 + y^2 = r^2 \quad \text{and} \quad \tan \theta = \frac{y}{x}
\]

In cylindrical coordinates, the differential $dV$ is a combination of the polar $dA$ and the rectangular $dz$:
\[
dV = r \, dr \, d\theta \, dz
\]

\begin{example}[Example 1]
Find the volume inside the hyperboloid of one sheet given by
\[
z^2 =  x^2 + y^2 -1
\]
from $z = -2$ to $z = 2$.\\
The hyperboloid can be written in cylindrical coordinates as
\[
z^2 = r^2 -1
\]
For each $z$ between $-2$ and $2$, the polar radius $r$ varies between 
\[
0 \quad \text{and} \quad \sqrt{1 + z^2}
\]
and $\theta$ varies from $0$ to $2\pi$.
Integrating with respect to $r$ first, we have
\begin{align*}
\iiint_R 1 \, dV &= \int_0^{2\pi} \int_{-2}^2 \int_0^{\sqrt{1+z^2}} r \, dr \, dz \, d\theta\\
                 &= 2\pi  \int_{-2}^2 \frac{r^2}{2} \bigg|_0^{\sqrt{1+z^2}}  \, dz\\
                 &= \pi \int_{-2}^2 (1+z^2) \, dz\\
                 &= 2\pi \int_0^2 (1+z^2) \, dz\\
                 &= 2\pi \left(z + \frac{z^3}{3}\right)\bigg|_0^2\\
                 &= 2\pi \left(2 + \frac83\right)\\
                 &= \frac{28\pi}{3}
\end{align*}
            
\begin{image}
\begin{tikzpicture}
\draw[thick,smooth, domain = 1:2.236, samples = 50] plot (\x, {sqrt(\x*\x -1)});
\draw[thick,smooth, domain = 1:2.236, samples = 50] plot (\x, {-sqrt(\x*\x -1)});
\draw[thick,smooth, domain = -2.236:-1, samples = 50] plot (\x, {sqrt(\x*\x -1)});
\draw[thick,smooth, domain = -2.236:-1, samples = 50] plot (\x, {-sqrt(\x*\x -1)});
\draw[thick] (0, 2) ellipse (2.236 and 0.25);
\draw[thick] (0, -2) ellipse (2.236 and 0.25);
\draw[thick, <->] (0, -3) -- (0,3) node[above]{$z$};
\draw[thick, <->] (-3, 0) -- (3,0) node[right]{$y$};
\draw[thick, <->] (2.1, 1.1) -- (-2.1, -1.1) node[below left]{$x$};
\draw[thin] (0.2, 2) -- (-0.2, 2) node[left]{$2$};
\draw[thin] (0.2, -2) -- (-0.2, -2) node[left]{$-2$};


\end{tikzpicture}
\end{image}

                 
\end{example}

\begin{problem}(Problem 1)
Find the volume inside the hyperboloid of one sheet given by
\[
z^2 =  \frac{x^2}{4} + \frac{y^2}{4} -1
\]
between the planes $z = -1$ and $z = 1$.\\
\[
\text{Volume} = \answer{32\pi/3}
\]
\end{problem}


Here is a video solution of problem 1:\\
\begin{foldable}
\youtube{N0dAy9eZuXI}
\end{foldable}



\begin{example}[Example 2]
Compute the triple integral $\iiint_R xyz \, dV$ where $R$ is the region inside the right circular cylinder $ x^2 + y^2 = 4$ from $z = 0 $ to $z = 3$.\\
The cylinder can be described in cylindrical coordinates by $r = 2$. Using cylindrical coordinates to compute the triple integral gives
\begin{align*}
\iiint_R xyz \, dV &= \int_0^3 \int_0^{2\pi} \int_0^2 (r\cos \theta)(r\sin\theta)z r \, dr\, d\theta\, dz \\
                   &= \left(\int_0^3 z\, dz\right) \left( \int_0^{2\pi} \cos \theta \sin \theta \, d\theta\right) \left( \int_0^2 r^3 \, dr\right)\\
                   &= \left(\frac{z^2}{2} \bigg|_0^3\right) \left(\frac12 \sin^2 \theta \bigg|_0^{2\pi} \right) \cdot \left( \frac{r^4}{4} \bigg|_0^2 \right)\\
                   &= (9/2) (0) (4) = 0
\end{align*}
\end{example}


\begin{problem}(Problem 2)
Compute the triple integral $\iiint_R xyz \, dV$ where $R$ is the region in the first octant ($x\geq 0, y \geq 0, z \geq 0$) and 
inside the right circular cylinder $ x^2 + y^2 = 9$ from $z = 0 $ to $z = 2$.\\
\[ 
\iiint_R xyz \, dV = \answer{81/4}
\]
\end{problem}


Here is a video solution of problem 2:\\
\begin{foldable}
\youtube{6kO_2bTwcU0}
\end{foldable}


\begin{example}[Example 3]
A solid lies between the cone $z = 2\sqrt{x^2 + y^2}$ and the paraboloid $z = 8 - x^2 - y^2$ and has a 
density at each point that is proportional to its distance from the $z$-axis. Find the mass of the solid.\\

Noting that the cone and the paraboloid can be written in cylindrical coordinates as
\[
 z = 2r \quad \text{and} \quad  z = 8 - r^2
\]
respectively. The intersection of the cone and the paraboloid is
\[
2r = 8 - r^2 \implies r = 2
\]
The mass of the solid is found by integrating the density function over the region in space that the object occupies.
The density, $\delta$, can be expressed in cylindrical coordinates as $\delta(x, y, z) = k \sqrt{x^2+y^2} = kr$.
Hence, the triple integral that gives the mass can be evaluated as
\begin{align*}
\iiint_R kr \, dV &=  \int_0^{2\pi} \int_0^2 \int_{2r}^{8-r^2} kr \cdot r \, dz \, dr \, d\theta \\
                  &= 2\pi \int_0^2 kr^2 (8 - r^2 - 2r) \,dr\\
                  &= 2k\pi \left(\frac{8r^3}{3} - \frac{r^4}{2} - \frac{r^5}{5} \right)\bigg|_0^2 \\
                  &= 2k\pi \left(\frac{64}{3} - 8 - \frac{32}{5}\right)\\
                  &= \frac{208k\pi}{15}
\end{align*}
\end{example}

\begin{problem}(Problem 3)
A solid lies between the cone $z = 4\sqrt{x^2 + y^2}$ and the paraboloid $z = 30 - 2x^2 - 2y^2$ and has a 
density at each point that is three times to its distance from the $z$-axis. Find the mass of the solid.\\
\[
\text{Mass} = \answer{2754\pi/5 }
\]
\end{problem}

Here is a video solution of problem 3:\\
\begin{foldable}
\youtube{All685W1euM}
\end{foldable}



\section{Spherical Coordinates}
Spherical coordinates locate a point in $\R^3$ using the distance from the origin to the point, $\rho$,  
and two angles, $\theta$ and $\phi$. The angle $\theta$ is the same as the polar $\theta$ from polar and cylindrical coordinates.
The new angle $\phi$ , called the angle of declination, is the angle between the $z$-axis and the vector from the origin to the point.
Note that the angle of declination satisfies $0 \leq \phi \leq \pi$.
See the figure below.
\begin{image}
\begin{tikzpicture}
\draw[thick, ->] (0,0,0) -- (0,0, 5) node[below]{$x$};
\draw[thick, ->] (0,0,0) -- (0, 5, 0) node[above]{$z$};
\draw[thick, ->] (0,0,0) -- (5,0,0) node[right]{$y$};
%\draw[thick, ->] (0, 0) -- (-2.3, -2.3) node[below, left]{$x$};
\draw[thick, blue!70!white] (0,0) -- (2,3) node[above right, black]{$(\rho, \theta, \phi)$};
\node at (1, 1.2) {$\rho$};
%\draw[dashed] (2, 3) -- (0, 3) node[left]{$z$};
\draw[dashed] (2,3) -- (2, -1) node[right]{$(r, \theta)$} ;
\draw[dashed] (0,0) -- (2,-1) node[midway, below]{$r$};
\draw (-.35, -.35) arc(-135:-28:.495)node[midway, below]{$\theta$};
\draw (0, 2) arc(90:56.4:2)node[midway, above]{$\phi$};
\filldraw (2, -1) circle (0.05);
\filldraw[blue!70!white] (2, 3) circle (0.05);

\end{tikzpicture}
\end{image}

Spherical coordinates can be related to rectangular and cylindrical coordinates as follows. First, the angle $\theta$ is the same in 
both spherical and cylindrical coordinates. Next, the spherical $\rho$ can be related to the cylindrical $\theta$ by
\[
\cos\left(\frac{\pi}{2} - \phi\right) = \frac{r}{\rho}
\]
Since, $\cos(\pi/2 - \phi) = \sin \phi$, we have
\[
r = \rho \sin \phi
\]
Using the relations $x = r\cos \theta$ and $y = r\sin \theta$, we can relate the spherical $\rho$ to the rectangular $x$ and $y$ by
\[
x = \rho \sin \phi \cos \theta \quad \text{and} \quad y = \rho \sin \phi \sin \theta
\]
Lastly, the rectangular $z$ can be related to the spherical coordinates by
\[
\cos \phi = \frac{z}{\rho}
\]
and thus
\[
z = \rho \cos \phi
\]

When computing triple integrals, the volume element $dV$ can be expressed in terms of spherical coordinates by computing the Jacobian:
\[
\begin{vmatrix}
\frac{\partial x}{\partial \rho} & \frac{\partial x}{\partial \theta} &\frac{\partial x}{\partial \phi}\\[10pt]
\frac{\partial y}{\partial \rho} & \frac{\partial y}{\partial \theta} &\frac{\partial y}{\partial \phi}\\[10pt]
\pp[z]{\rho} & \frac{\partial z}{\partial \theta} &\frac{\partial z}{\partial \phi}
\end{vmatrix}
%\]
%\[
%\begin{vmatrix}
%x_{\rho} & x_{\theta} &x_{\phi}\\
%y_{\rho} & y_{\theta} &y_{\phi}\\
%z_{\rho} & z_{\theta} &z_{\phi}\\
%\end{vmatrix}
= 
\begin{vmatrix}
\sin \phi \cos \theta& -\rho \sin \phi \sin \theta &\rho \cos \phi \cos \theta\\ 
\sin \phi \sin \theta& \rho \sin \phi \cos\theta &\rho \cos \phi \sin \theta\\ 
 \cos \phi & 0 &-\rho \sin \phi\\
\end{vmatrix}
\]
\begin{align*}
&= \sin \phi \cos \theta \left[-\rho^2 \sin^2\phi\cos\theta \right]\\
 &\quad+\rho \sin \phi \sin \theta\left[-\rho\sin^2\phi\sin\theta- \rho\cos^2\phi\sin\theta\right]\\
 & \quad+ \rho \cos \phi \cos \theta \left[-\rho \sin\phi\cos\phi\cos\theta\right]\\
 &= -\rho^2 \sin^3\phi \cos^2\theta -\rho^2 \sin^3\phi \sin^2\theta \\
 & \quad - \rho^2 \cos^2\phi \sin\phi \sin^2\theta - \rho^2 \cos^2\phi \sin\phi \cos^2\theta\\
  &= -\rho^2\sin^3\phi - \rho^2\cos^2\phi\sin\phi\\
 &=-\rho^2 \sin \phi
\end{align*}
 
Finally, the volume element is given by
\[
dV = |J(\rho, \theta, \phi)| = \rho^2 \sin \phi
\]
Note that $\sin \phi \geq 0$ since $0 \leq \phi \leq \pi$.

\begin{example}[Example 4]
The density of a spherical object of radius 1 is given by $\delta(\rho, \theta, \phi) = \rho + \phi$. Find the mass of the object.\\
Since the object is spherical with radius 1, the bounds on its coordinates are
\[
0\leq \rho \leq 1, \quad 0 \leq \theta \leq 2\pi \quad \text{and} \quad 0\leq \phi \leq \pi
\]
Using the fact that the volume element in spherical coordinates is 
\[
dV = \rho^2 \sin\phi \, d\rho\, d\theta\, d\phi
\]
we can compute the mass of the object as follows:
\begin{align*}
\text{Mass} &= \int_0^\pi \int_0^{2\pi}  \int_0^1 (\rho + \phi) \rho^2 \sin \phi  \, d\rho\, d\theta\, d\phi\\
            &= 2\pi \int_0^\pi \int_0^1 (\rho^3 + \rho^2\phi)  \sin \phi  \, d\rho\, d\phi\\
            &= 2\pi \int_0^\pi \left(\tfrac14\rho^4 + \tfrac13\rho^3 \phi\right) \bigg|_0^1 \sin \phi \, d\phi\\
            &=2 \pi \int_0^\pi \left(\tfrac14 \sin \phi + \tfrac13 \phi \sin \phi\right) \, d\phi\\
            &= 2\pi \left(-\tfrac14 \cos \phi -\tfrac13 \phi \cos \phi + \tfrac13\sin\phi \right)\bigg|_0^\pi\\
            &=2\pi \left(\tfrac14 + \tfrac13 \pi +\tfrac14 \right)\\
            &= \pi + \tfrac{2 }{3}\pi^2
\end{align*}
\end{example}

\begin{problem}(Problem 4)
The density of a spherical object of radius 2 is given by $\delta(\rho, \theta, \phi) = 4\rho + 3\phi$. Find the mass of the object.\\
\[
\text{Mass} = \answer{64\pi + 16\pi^2}
\]
\end{problem}
            

Here is a video solution of problem 4:\\
\begin{foldable}
\youtube{ozMbEcV28Bo}
\end{foldable}


\begin{example}[Example 5]
Find the volume of the solid bounded between the sphere $\rho = 4$ and the cone $\phi = \frac{\pi}{3}$.\\
The coordinates of the solid satisfy the inequalities
\[
0 \leq \rho \leq 4, \quad 0 \leq \phi \leq \frac{\pi}{3} \quad \text{and} \quad 0\leq \theta \leq 2\pi
\]
The volume of the solid can be found by integrating the function $1$ over the solid as follows:
\begin{align*}
\text{Volume} &= \int_0^{\pi/3} \int_0^{2\pi}  \int_0^4 \rho^2 \sin \phi \, d\rho\, d\theta\, d\phi\\
              &= 2\pi \left(\int_0^{\pi/3} \sin \phi \, d\phi\right) \cdot \left( \int_0^4 \rho^2 \, d\rho\right) \\
              &= 2\pi \left( -\cos \phi \bigg|_0^{\pi/3} \right) \cdot \left( \frac{\rho^3}{3}\bigg|_0^4\right)\\
              &= 2\pi \left(1 - \frac12\right) \cdot \left( \frac{64}{3}\right)\\
              &= \frac{64\pi}{3}
\end{align*}

\begin{image}
\begin{tikzpicture}
\filldraw[thick, blue!70!white, fill = blue!10!white] (0,0) -- (30:4) arc (30:150:4) -- cycle;
%\draw (-3.464,2) -- (0,0) -- (3.464, 2);
%\draw (0,0) arc (0:180:4);
\draw[thick, blue!70!white] (0, 2) ellipse (3.464 and 0.25);
\draw[<->, thick] (0, -1) -- (0, 5) node[above]{$z$};
\draw[thin] (0, 1) arc (90:30:1) node[midway, above]{$\pi/3$};
\draw[thick] (0.25, 4) --(-0.25, 4) node[above left]{$4$};
\draw[thick, <->] (-3, 0) -- (3, 0) node[right]{$y$};
\draw[thick, ->] (0,0) -- (-1, -1) node[below left]{$x$};
\draw[thick, ->, darkgray] (0,0) -- (1,1);
\end{tikzpicture}
\end{image}

\end{example}

\begin{problem}(Problem 5)
Find the volume of the solid bounded between the sphere $\rho = 3$ and the cone $\phi = \frac{\pi}{4}$.\\
\[
\text{Volume}= \answer{18\pi(1 - 1/\sqrt 2)}
\]
\end{problem}  

Here is a video solution of problem 5:\\
\begin{foldable}
\youtube{MzSZB6M8GTg}
\end{foldable}

                      
\end{document}
