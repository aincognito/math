\documentclass[handout]{ximera}

%% You can put user macros here
%% However, you cannot make new environments



\newcommand{\ffrac}[2]{\frac{\text{\footnotesize $#1$}}{\text{\footnotesize $#2$}}}
\newcommand{\vasymptote}[2][]{
    \draw [densely dashed,#1] ({rel axis cs:0,0} -| {axis cs:#2,0}) -- ({rel axis cs:0,1} -| {axis cs:#2,0});
}


%\usepackage{tcolorbox} %%Needed for Derivative Definition supposedly and product rule, natural exp log, quotient rule, inverse trig, rates of change


% \graphicspath{{./}{firstExample/}}
% \usepackage{forest}
\usepackage{amsmath}
\usepackage{amssymb}
\usepackage{array}
\usepackage[makeroom]{cancel} %% for strike outs
\usepackage{pgffor} %% required for integral for loops
\usepackage{tikz}
\usepackage{tikz-cd}
\usepackage{tkz-euclide}
\usetikzlibrary{shapes.multipart}


% \usetkzobj{all}
\tikzstyle geometryDiagrams=[ultra thick,color=blue!50!black]


\usetikzlibrary{arrows}
\tikzset{>=stealth,commutative diagrams/.cd,
  arrow style=tikz,diagrams={>=stealth}} %% cool arrow head
\tikzset{shorten <>/.style={ shorten >=#1, shorten <=#1 } } %% allows shorter vectors

\usetikzlibrary{backgrounds} %% for boxes around graphs
\usetikzlibrary{shapes,positioning}  %% Clouds and stars
\usetikzlibrary{matrix} %% for matrix
\usepgfplotslibrary{polar} %% for polar plots
\usepgfplotslibrary{fillbetween} %% to shade area between curves in TikZ



%\usepackage[width=4.375in, height=7.0in, top=1.0in, papersize={5.5in,8.5in}]{geometry}
%\usepackage[pdftex]{graphicx}
%\usepackage{tipa}
%\usepackage{txfonts}
%\usepackage{textcomp}
%\usepackage{amsthm}
%\usepackage{xy}
%\usepackage{fancyhdr}
%\usepackage{xcolor}
%\usepackage{mathtools} %% for pretty underbrace % Breaks Ximera
%\usepackage{multicol}



\newcommand{\RR}{\mathbb R}
\newcommand{\R}{\mathbb R}
\newcommand{\C}{\mathbb C}
\newcommand{\N}{\mathbb N}
\newcommand{\Z}{\mathbb Z}
\newcommand{\dis}{\displaystyle}
%\renewcommand{\d}{\,d\!}
\renewcommand{\d}{\mathop{}\!d}
\newcommand{\dd}[2][]{\frac{\d #1}{\d #2}}
\newcommand{\pp}[2][]{\frac{\partial #1}{\partial #2}}
\renewcommand{\l}{\ell}
\newcommand{\ddx}{\frac{d}{\d x}}
\newcommand{\ppx}{\frac{\partial}{\partial x}}
\newcommand{\ppy}{\frac{\partial}{\partial y}}

\newcommand{\zeroOverZero}{\ensuremath{\boldsymbol{\tfrac{0}{0}}}}
\newcommand{\inftyOverInfty}{\ensuremath{\boldsymbol{\tfrac{\infty}{\infty}}}}
\newcommand{\zeroOverInfty}{\ensuremath{\boldsymbol{\tfrac{0}{\infty}}}}
\newcommand{\zeroTimesInfty}{\ensuremath{\small\boldsymbol{0\cdot \infty}}}
\newcommand{\inftyMinusInfty}{\ensuremath{\small\boldsymbol{\infty - \infty}}}
\newcommand{\oneToInfty}{\ensuremath{\boldsymbol{1^\infty}}}
\newcommand{\zeroToZero}{\ensuremath{\boldsymbol{0^0}}}
\newcommand{\inftyToZero}{\ensuremath{\boldsymbol{\infty^0}}}


\newcommand{\numOverZero}{\ensuremath{\boldsymbol{\tfrac{\#}{0}}}}
\newcommand{\dfn}{\textbf}
%\newcommand{\unit}{\,\mathrm}
\newcommand{\unit}{\mathop{}\!\mathrm}
%\newcommand{\eval}[1]{\bigg[ #1 \bigg]}
\newcommand{\eval}[1]{ #1 \bigg|}
\newcommand{\seq}[1]{\left( #1 \right)}
\renewcommand{\epsilon}{\varepsilon}
\renewcommand{\iff}{\Leftrightarrow}

\DeclareMathOperator{\arccot}{arccot}
\DeclareMathOperator{\arcsec}{arcsec}
\DeclareMathOperator{\arccsc}{arccsc}
\DeclareMathOperator{\si}{Si}
\DeclareMathOperator{\proj}{proj}
\DeclareMathOperator{\scal}{scal}
\DeclareMathOperator{\cis}{cis}
\DeclareMathOperator{\Arg}{Arg}
%\DeclareMathOperator{\arg}{arg}
\DeclareMathOperator{\Rep}{Re}
\DeclareMathOperator{\Imp}{Im}
\DeclareMathOperator{\sech}{sech}
\DeclareMathOperator{\csch}{csch}
\DeclareMathOperator{\Log}{Log}

\newcommand{\tightoverset}[2]{% for arrow vec
  \mathop{#2}\limits^{\vbox to -.5ex{\kern-0.75ex\hbox{$#1$}\vss}}}
\newcommand{\arrowvec}{\overrightarrow}
\renewcommand{\vec}{\mathbf}
\newcommand{\veci}{{\boldsymbol{\hat{\imath}}}}
\newcommand{\vecj}{{\boldsymbol{\hat{\jmath}}}}
\newcommand{\veck}{{\boldsymbol{\hat{k}}}}
\newcommand{\vecl}{\boldsymbol{\l}}
\newcommand{\utan}{\vec{\hat{t}}}
\newcommand{\unormal}{\vec{\hat{n}}}
\newcommand{\ubinormal}{\vec{\hat{b}}}

\newcommand{\dotp}{\bullet}
\newcommand{\cross}{\boldsymbol\times}
\newcommand{\grad}{\boldsymbol\nabla}
\newcommand{\divergence}{\grad\dotp}
\newcommand{\curl}{\grad\cross}
%% Simple horiz vectors
\renewcommand{\vector}[1]{\left\langle #1\right\rangle}


\pgfplotsset{compat=1.13}

\outcome{Compute line integrals}

\title{4.6 Line Integrals and Work}

\begin{document}

\begin{abstract}
We compute integrals of vector-valued functions along curves.
\end{abstract}

\maketitle


\begin{definition}[Line Integral]
Let $C$ be a curve in $\R^2$ parameterized by $\vec r(t) = \vector{x(t), y(t)}$ for $ \; a\leq t \leq b$ and let $\vec f(x, y)$ be a vector-valued function defined along $C$.
Then the integral of $f$ along $C$ is defined by
\[
\int_C \vec f(x, y) \dotp d\vec r = \int_a^b \vec f(x(t), y(t)) \dotp \vec r\,'(t) dt
\]
\end{definition}
%\begin{remark}
%The value of a line integral depends on the orientation of the curve, $\vec r(t)$. 
 
 
\begin{example}[example 1]
Compute $\dis \int_C \vector{x, -y} \dotp d\vec r$ where $C$ is the line segment from $(-1, 0)$ to $(1,1)$.\\
We first parameterize the line segment. Since the equation of the line in the $xy$-plane is $y = \frac12x + \frac12$,
one possible parametrization of the line segment is
\[
\vec r(t) = \vector{x(t), y(t)} = \vector{t , \left(\frac{t}{2} + \frac12\right)}, \; -1 \leq t \leq 1
\]
The derivative of $\vec r(t)$ is 
\[
\vec r\,'(t) = \vector{1, \frac{1}{2}}
\]
Next, $f(x, y) = \vector{x, -y}$ and hence
\[
\vec f(x(t), y(t)) = \vector{x(t), -y(t)} = \vector{t , -\left(\frac{t}{2} + \frac12\right)}
\]
We are now ready to compute the contour integral:
\begin{align*}
\int_C \vec f(x, y) \dotp d\vec r &= \int_{-1}^1 \vec f(x(t), y(t)) \dotp \vec r\,'(t) \, dt\\
                             &= \int_{-1}^1 \vector{t , -\left(\frac{t}{2} + \frac12\right)} \dotp  \vector{1, \frac{1}{2}} \, dt\\[6pt]
                             &= \int_{-1}^1 \left(t -\frac{t}{4} - \frac14\right) \, dt\\[6pt]
                             &= \int_{-1}^1 \left(\frac34 t - \frac14\right) \, dt\\[6pt]
                             &= \left(\frac38 t^2 - \frac14 t\right)\bigg|_{-1}^1\\[6pt]
                         &= \left(\frac38 - \frac14\right) - \left(\frac38 + \frac14\right)\\[6 pt]
                        &= -\frac12
\end{align*}

An alternative method for parameterizing a line segment was discussed in chapter 1: if $P$ and $Q$ are the vector representation of points, the the line segment from 
$P$ to $Q$ is:
\[
\vec r(t) = (1-t)P + tQ, \; 0 \leq t \leq 1
\]
Using $(-1, 0)$ and $(1,1)$ for $P$ and $Q$, respectively, we can parameterize the line segment as
\[
\vec r(t) = (1-t) \vector{-1, 0} + t\vector{1,1} = \vector{2t-1, t}, \; 0\leq t \leq 1
\]
Thus,
\[
x(t) = 2t-1 \quad \text{and} \quad y(t) = t
\]
and the vector function is
\[
\vec f(x(t), y(t)) = \vector{x(t), -y(t)} = \vector{2t-1, -t}
\]
We can now compute the line integral
\begin{align*}
\int_C \vec f(x, y) \dotp d\vec r &= \int_0^1 \vec f(x(t), y(t)) \dotp \vec r\,'(t) \, dt\\
                             &= \int_0^1 \vector{2t-1, t} \dotp  \vector{2, 1} \, dt\\[6pt]
                             &= \int_0^1 \left(4t-2 -t\right) \, dt\\[6pt]
                             &= \int_0^1 \left(3t-2\right) \, dt\\[6pt]
                             &= \left(\frac32 t^2 - 2 t\right)\bigg|_0^1\\[6pt]
                        &= \frac32 - 2 = -\frac12
\end{align*}
as before.
\end{example}
\begin{remark}
A line integral is independent of the choice of the parameterization for the curve $C$. 
\end{remark}

If the vector-valued function $\vec f(x,y)$ the force on an object at the point $(x,y)$, then the line integral
\[
\int_C \vec f(x,y) \dotp  d\vec r
\]
represents the \textbf{work} done by (or against) an object that travels along the curve $C$.

\begin{example}[Example 2]
The force due to gravity by a mass at the origin of $\R^2$ (two dimensions for simplicity) is given by
\[
\vec f(x,y) = \frac{k}{(x^2 + y^2)^{3/2}} \vector{-x, -y}
\]
where $k$ is a positive constant that depends on the masses of the objects involved. Find the work done 
by this force as an object moves in a straight line from the point $(0,1)$ to the point $(1,1)$.\\
The line segment from $(0,1)$ to $(1,1)$ can be parameterized by
\[
C: \vec r(t) = \vector{t, 1} \; \text{for}   \; 0 \leq t \leq 1
\]
The work done is the value of the line integral:
\begin{align*}
\text{Work} &= \int_C \vec f(x,y) \dotp d\vec r\\
            &= \int_0^1 \vec f(x(t),y(t)) \dotp \vec r\,'(t) \, dt\\
            &= \int_0^1 \vec f(t, 1) \dotp \vec r\,'(t) \, dt\\
            &= \int_0^1 \frac{k}{(t^2 + 1^2)^{3/2}} \vector{-t,-1} \dotp \vector{1,0} \, dt\\
            &= \int_0^1 -\frac{kt}{(t^2 + 1^2)^{3/2}} \, dt\\
            &= \frac{k}{(t^2 +1)^{1/2}} \bigg|_0^1 \\
            &= k\left(\frac{1}{\sqrt 2} - 1\right)
\end{align*}
Note that this answer is negative.  This means that work was done \textbf{against} the force rather than by the force.
\end{example}


\end{document}


\begin{problem}(problem 1a)
Use the parametrization given in the remark above to compute the integral in example 1.
\end{problem}

\begin{problem}(problem 1b)
Let $C$ be the line segment from $0$ to $1+i$. Compute \\
$\dis \int_C \overline{z} \, dz = \answer{1} $\\
\end{problem}

\begin{problem}(problem 1c)
Let $C= C(0, 1)$ be the unit circle traversed in the counter-clockwise direction. Compute \\
$\dis \int_C \overline{z} \, dz = \answer{2\pi i} $\\
\begin{hint}
To parametrize the circle $C(z_0, r)$ traversed in the counter-clockwise direction, use
\[
\gamma(t) = z_0 + re^{it}, \, 0 \leq t \leq 2\pi
\]
\end{hint}
\end{problem}


\begin{proposition}
If the contour $C$ has two different parametrizations, $\gamma_1(t)\,$ with  $\, a_1 \leq t \leq b_1$ and $\gamma_2(t)\,$ with $ \, a_2 \leq t \leq b_2$, then
\[
\int_C f(z) \, dz = \int_{a_1}^{b_1} f(\gamma_1(t)) \gamma_1'(t) \, dt = \int_{a_2}^{b_2} f(\gamma_2(t)) \gamma_2'(t) \, dt
\]
That is, the value of a contour integral is independent of the parametrization used to describe the contour.
\end{proposition}

What happens if we reverse the direction of a contour , $C$?


If $C$ is a contour parameterized by $\gamma(t) = x(t) + iy(t), \, a \leq t \leq b$ then the contour $-C$ which represents the 
same points as $C$ but traced in the opposite 
direction has parametrization $\zeta(s) = x(a+b-s) + iy(a+b-s), \, a \leq s \leq b$.


\begin{proposition}
Suppose $f$ is continuous along a contour $C$ %parameterized by $C: \gamma(t) = x(t) + iy(t), \, a \leq t \leq b$
and suppose the contour $-C$ represents the same points as $C$ traced in the opposite direction, 
then 
\[
\int_{-C} f(z) \, dz = -\int_C f(z) \, dz
\]
\end{proposition}
\begin{proof}
Using the substitution $t = a+b-s$, we have
\begin{align*}
\int_{-C} f(z) \, dz &= \int_a^b f(\zeta(s)) \zeta'(s) \, ds \\
&= \int_a^b f\big(x(a+b-s) + iy(a+b-s)\big) \big[x'(a+b-s)(-1)+ iy'(a+b-s)(-1)\big]\, ds\\
                     &= \int_b^a f\big(x(t)+iy(t)\big) \big[x'(t)+ iy'(t)\big] \, dt\\
                     &= -\int_a^b f(\gamma(t)) \gamma'(t) \, dt\\
                   &= - \int_C f(z) \, dz
\end{align*}
\end{proof}

\begin{example}[example 2]
Let $C$ be the semi-circle parametrized by $\gamma(t) = 2e^{it}, \, 0\leq t \leq \pi$. Compute
\[
\int_C \frac{1}{z} \, dz \quad \text{and} \quad \int_{-C }\frac{1}{z} \, dz
\]
and verify the the above proposition.\\
Note that $\gamma'(t) = 2ie^{it}$. We begin with the integral along $C$:
\[
\int_C \frac{1}{z} \, dz = \int_0^\pi \frac{1}{\gamma(t)} \gamma'(t) \, dt = \int_0^\pi \frac{2ie^{it}}{2e^{it}} \, dt
                        =\int_0^\pi i \, dt \,= \, it\, \bigg|_0^\pi \, = \pi i
\]
Next we integrate along $-C$.  To do this we parametrize $-C$ as $\zeta(t) = 2e^{i(\pi - t)}, \, 0\leq t \leq \pi$.
Note that $\zeta'(t) = -2ie^{i(\pi - t)}$. Thus
\[
\int_{-C} \frac{1}{z} \, dz = \int_0^\pi \frac{1}{\zeta(t)} \zeta'(t) \, dt = \int_0^\pi \frac{-2ie^{i(\pi-t)}}{2e^{i(\pi -t)}} \, dt
                        =\int_0^\pi -i \, dt
                        = -it \bigg|_0^\pi
                        = -\pi i
\]
From our calculations, we see that $\dis \int_C \frac{1}{z} \, dz = -\int_{-C} \frac{1}{z} \, dz$
thereby verifying the conclusion of the previous proposition.
\end{example}


\begin{problem}(problem 2a)
Let $C= C(0, 1)$ be the unit circle traversed in the counter-clockwise direction. Compute \\
$\dis \int_C \frac{1}{z} \, dz = \answer{2\pi i} $\\
\begin{hint}
To parametrize the circle $C(z_0, r)$ traversed in the counter-clockwise direction, use
\[
\gamma(t) = z_0 + re^{it}, \, 0 \leq t \leq 2\pi
\]
\end{hint}
\end{problem}

\begin{problem}(problem 2b)
Let $C= C(i, 2)$ be the circle centered at $i$ of radius $2$, traversed in the clockwise direction. Compute \\
$\dis \int_C \frac{1}{z-i} \, dz = \answer{-2\pi i} $\\
\begin{hint}
To parametrize the circle $C(z_0, r)$ traversed in the counter-clockwise direction, use
\[
\gamma(t) = z_0 + re^{it}, \, 0 \leq t \leq 2\pi
\]
\end{hint}
\end{problem}

\begin{problem}(problem 2c)
Let $C= C(z_0, r)$ be the circle centered at $z_0$ of radius $r$, traversed in the counter-clockwise direction. Compute \\
$\dis \int_C \frac{1}{z-z_0} \, dz = \answer{2\pi i} $\\
\begin{hint}
To parametrize the circle $C(z_0, r)$ traversed in the counter-clockwise direction, use
\[
\gamma(t) = z_0 + re^{it}, \, 0 \leq t \leq 2\pi
\]
\end{hint}
\end{problem}

Compare your answers to problems 2b and 2c. Notice that they verify the proposition.

\begin{problem}(problem 2d)
Let $C= C(z_0, r)$ be the circle centered at $z_0$ of radius $r$, traversed in the counter-clockwise direction. Compute \\
$\dis \int_C \frac{1}{(z-z_0)^2} \, dz = \answer{0} $\\
\begin{hint}
To parametrize the circle $C(z_0, r)$ traversed in the counter-clockwise direction, use
\[
\gamma(t) = z_0 + re^{it}, \, 0 \leq t \leq 2\pi
\]
\end{hint}
\end{problem}

\begin{proposition}
Suppose $f(t) = u(t) + iv(t)$ is a continuous function of the real variable $t$. Then
\[
\left| \int_a^b f(t) \, dt \right| \leq \int_a^b \left| f(t) \right| \, dt
\]
\end{proposition}

\begin{proposition}[$ML$-formula]
Let $C$ be a contour and suppose $f$ is continuous along $C$. Then
\[
\left| \int_C f(z) \, dz \right| \leq ML
\]
where $M$ is the maximum value of $|f(z)|$ along $C$ and $L$ is the length of $C$. 
\end{proposition}   

\begin{proof}
Suppose $C$ is parametrized by $\gamma(t), \, a \leq t \leq b$. Since $|f(z)| \leq M$ for all $z \in C$, the previous proposition implies
\[
\left| \int_C f(z) \, dz \right| = \left| \int_a^b f(\gamma(t))\gamma'(t) \, dt \right| \leq  \int_a^b \left|f(\gamma(t))\right| \cdot \left|\gamma'(t)\right| \, dt \leq M \int_a^b \left|\gamma'(t)\right| \, dt
\]
But this last integral is exactly $L$, the length of the curve $C$ and hence
\[
\left| \int_C f(z) \, dz \right| \leq M \int_a^b \left|\gamma'(t)\right| \, dt = ML
\]
\end{proof}

\begin{example}[example 3]
Use the $ML$-formula to estimate the modulus of the integral: $\dis \int_C z^\frac12 \, dz$,
where $C$ is the circle centered at $3+4i$ of radius $4$.\\
The length $L$ of the circle, $C$, is $L = 2\pi \cdot \text{radius} = 8\pi$. 
In polar form, $z^\frac12 = r^\frac12 e^{i\theta/2}$ and hence $|z^\frac12| = r^\frac12 = |z|^\frac12$.
From the triangle inequality, the modulus of any point $z$ on $C=C(3+4i, 4)$ can be estimated by
\[
|z| = |z-(3+4i) + (3+4i)| \leq |z-(3+4i)| + |3+4i| = 4+5 = 9.
\]
Hence for any $z$ on the circle, $|z^\frac12| \leq 9^\frac12 = 3 = M$.
Finally, by the $ML$-formula,
\[
\left| \int_C z^\frac12 \, dz \right| \leq ML = (3)(8\pi) = 24\pi.
\]
\end{example}

\begin{problem}(problem 3)
Use the $ML$-formula to estimate the modulus of the integral: $\dis \int_C e^z \, dz$,
where $C$ is the circle centered at $i$ of radius $2$.\\
\begin{hint}
$|e^z| = e^x$
\end{hint}
The maximum modulus of $e^z$ on $C$ is $M = \answer{e^2}$\\
The length of $C$ is $L = \answer{4\pi}$\\
By the $ML$-formula, $\dis \int_C e^z \, dz \leq \answer{4e^2 \pi}$
\end{problem}
If the function $f(z)$ is analytic at every point on the contour $C$, then the computation of the integral along $C$
can be simplified as long as an anti-derivative of $f$ can be found.

\begin{theorem}[Fundamental Theorem of Calculus]
Suppose $C$ is a contour with initial point $z_1$ and terminal point $z_2$. If $f$ is analytic along $C$ and 
if $F$ is an anti-derivative of $f$ along $C$, i.e., $F' = f$ at every $z \in C$,
then
\[
\int_C f(z) \, dz = F(z_2) -F(z_1)
\]
\end{theorem}

\begin{example}[example 4]
Let $C$ be any contour from $0$ to $i$. Compute $\dis \int_C \cos z \, dz$.\\
Since $\cos z$ is an entire function, it is analytic at every point on $C$. Moreover, since
\[
\frac{d}{dz} \sin z = \cos z \quad \text{for all} \,\, z \in \C
\]
we have an anti-derivative for $\cos z$ for any $z$ on $C$. Thus
\[
\int_C \cos z \, dz = \sin(i) - \sin(0) = \sin(i) = i\sinh(1).
\]
\end{example}

\begin{problem}(problem 4)
Let $C$ be any contour from $1$ to $1+i$ that does not cross the negative real axis or go through the origin. 
Compute $\dis \int_C \frac{1}{z} \, dz$.\\
\[
\int_C \frac{1}{z} \, dz = \answer{\frac12 \ln 2+ i\pi/4}.
\]
\begin{hint}
Use the principal branch of the logarithm, $\Log z$, as the anti-derivative on $C$.
\end{hint}
\end{problem}


\end{document}


                         
                         



                         












