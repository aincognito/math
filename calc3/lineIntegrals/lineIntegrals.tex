\documentclass[handout]{ximera}

%% You can put user macros here
%% However, you cannot make new environments



\newcommand{\ffrac}[2]{\frac{\text{\footnotesize $#1$}}{\text{\footnotesize $#2$}}}
\newcommand{\vasymptote}[2][]{
    \draw [densely dashed,#1] ({rel axis cs:0,0} -| {axis cs:#2,0}) -- ({rel axis cs:0,1} -| {axis cs:#2,0});
}


\graphicspath{{./}{firstExample/}}
\usepackage{forest}
\usepackage{amsmath}
\usepackage{amssymb}
\usepackage{array}
\usepackage[makeroom]{cancel} %% for strike outs
\usepackage{pgffor} %% required for integral for loops
\usepackage{tikz}
\usepackage{tikz-cd}
\usepackage{tkz-euclide}
\usetikzlibrary{shapes.multipart}


%\usetkzobj{all}
\tikzstyle geometryDiagrams=[ultra thick,color=blue!50!black]


\usetikzlibrary{arrows}
\tikzset{>=stealth,commutative diagrams/.cd,
  arrow style=tikz,diagrams={>=stealth}} %% cool arrow head
\tikzset{shorten <>/.style={ shorten >=#1, shorten <=#1 } } %% allows shorter vectors

\usetikzlibrary{backgrounds} %% for boxes around graphs
\usetikzlibrary{shapes,positioning}  %% Clouds and stars
\usetikzlibrary{matrix} %% for matrix
\usepgfplotslibrary{polar} %% for polar plots
\usepgfplotslibrary{fillbetween} %% to shade area between curves in TikZ



%\usepackage[width=4.375in, height=7.0in, top=1.0in, papersize={5.5in,8.5in}]{geometry}
%\usepackage[pdftex]{graphicx}
%\usepackage{tipa}
%\usepackage{txfonts}
%\usepackage{textcomp}
%\usepackage{amsthm}
%\usepackage{xy}
%\usepackage{fancyhdr}
%\usepackage{xcolor}
%\usepackage{mathtools} %% for pretty underbrace % Breaks Ximera
%\usepackage{multicol}



\newcommand{\RR}{\mathbb R}
\newcommand{\R}{\mathbb R}
\newcommand{\C}{\mathbb C}
\newcommand{\N}{\mathbb N}
\newcommand{\Z}{\mathbb Z}
\newcommand{\dis}{\displaystyle}
%\renewcommand{\d}{\,d\!}
\renewcommand{\d}{\mathop{}\!d}
\newcommand{\dd}[2][]{\frac{\d #1}{\d #2}}
\newcommand{\pp}[2][]{\frac{\partial #1}{\partial #2}}
\renewcommand{\l}{\ell}
\newcommand{\ddx}{\frac{d}{\d x}}

\newcommand{\zeroOverZero}{\ensuremath{\boldsymbol{\tfrac{0}{0}}}}
\newcommand{\inftyOverInfty}{\ensuremath{\boldsymbol{\tfrac{\infty}{\infty}}}}
\newcommand{\zeroOverInfty}{\ensuremath{\boldsymbol{\tfrac{0}{\infty}}}}
\newcommand{\zeroTimesInfty}{\ensuremath{\small\boldsymbol{0\cdot \infty}}}
\newcommand{\inftyMinusInfty}{\ensuremath{\small\boldsymbol{\infty - \infty}}}
\newcommand{\oneToInfty}{\ensuremath{\boldsymbol{1^\infty}}}
\newcommand{\zeroToZero}{\ensuremath{\boldsymbol{0^0}}}
\newcommand{\inftyToZero}{\ensuremath{\boldsymbol{\infty^0}}}


\newcommand{\numOverZero}{\ensuremath{\boldsymbol{\tfrac{\#}{0}}}}
\newcommand{\dfn}{\textbf}
%\newcommand{\unit}{\,\mathrm}
\newcommand{\unit}{\mathop{}\!\mathrm}
%\newcommand{\eval}[1]{\bigg[ #1 \bigg]}
\newcommand{\eval}[1]{ #1 \bigg|}
\newcommand{\seq}[1]{\left( #1 \right)}
\renewcommand{\epsilon}{\varepsilon}
\renewcommand{\iff}{\Leftrightarrow}

\DeclareMathOperator{\arccot}{arccot}
\DeclareMathOperator{\arcsec}{arcsec}
\DeclareMathOperator{\arccsc}{arccsc}
\DeclareMathOperator{\si}{Si}
\DeclareMathOperator{\proj}{proj}
\DeclareMathOperator{\scal}{scal}
\DeclareMathOperator{\cis}{cis}
\DeclareMathOperator{\Arg}{Arg}
%\DeclareMathOperator{\arg}{arg}
\DeclareMathOperator{\Rep}{Re}
\DeclareMathOperator{\Imp}{Im}
\DeclareMathOperator{\sech}{sech}
\DeclareMathOperator{\csch}{csch}
\DeclareMathOperator{\Log}{Log}

\newcommand{\tightoverset}[2]{% for arrow vec
  \mathop{#2}\limits^{\vbox to -.5ex{\kern-0.75ex\hbox{$#1$}\vss}}}
\newcommand{\arrowvec}{\overrightarrow}
\renewcommand{\vec}{\mathbf}
\newcommand{\veci}{{\boldsymbol{\hat{\imath}}}}
\newcommand{\vecj}{{\boldsymbol{\hat{\jmath}}}}
\newcommand{\veck}{{\boldsymbol{\hat{k}}}}
\newcommand{\vecl}{\boldsymbol{\l}}
\newcommand{\utan}{\vec{\hat{t}}}
\newcommand{\unormal}{\vec{\hat{n}}}
\newcommand{\ubinormal}{\vec{\hat{b}}}

\newcommand{\dotp}{\bullet}
\newcommand{\cross}{\boldsymbol\times}
\newcommand{\grad}{\boldsymbol\nabla}
\newcommand{\divergence}{\grad\dotp}
\newcommand{\curl}{\grad\cross}
%% Simple horiz vectors
\renewcommand{\vector}[1]{\left\langle #1\right\rangle}


\pgfplotsset{compat=1.13}

\outcome{Compute line integrals}

\title{4.6 Line Integrals and Work}

\begin{document}

\begin{abstract}
We compute integrals of vector-valued functions along curves.
\end{abstract}

\maketitle


\begin{definition}[Line Integral]
Let $C$ be a curve in $\R^2$ parameterized by $\vec r(t) = \vector{x(t), y(t)}$ for $ \; a\leq t \leq b$ and let $\vec f(x, y)$ be a vector-valued function defined along $C$.
Then the integral of $f$ along $C$ is defined by
\[
\int_C \vec f(x, y) \dotp d\vec r = \int_a^b \vec f(x(t), y(t)) \dotp \vec r\,'(t) dt
\]
\end{definition}
%\begin{remark}
%The value of a line integral depends on the orientation of the curve, $\vec r(t)$. 
 
 
\begin{example}[example 1]
Compute $\dis \int_C \vector{x, -y} \dotp d\vec r$ where $C$ is the line segment from $(-1, 0)$ to $(1,1)$.\\
We first parameterize the line segment. Since the equation of the line in the $xy$-plane is $y = \frac12x + \frac12$,
one possible parametrization of the line segment is
\[
\vec r(t) = \vector{x(t), y(t)} = \vector{t , \left(\frac{t}{2} + \frac12\right)}, \; -1 \leq t \leq 1
\]
The derivative of $\vec r(t)$ is 
\[
\vec r\,'(t) = \vector{1, \frac{1}{2}}
\]
Next, $f(x, y) = \vector{x, -y}$ and hence
\[
\vec f(x(t), y(t)) = \vector{x(t), -y(t)} = \vector{t , -\left(\frac{t}{2} + \frac12\right)}
\]
We are now ready to compute the contour integral:
\begin{align*}
\int_C \vec f(x, y) \dotp d\vec r &= \int_{-1}^1 \vec f(x(t), y(t)) \dotp \vec r\,'(t) \, dt\\
                             &= \int_{-1}^1 \vector{t , -\left(\frac{t}{2} + \frac12\right)} \dotp  \vector{1, \frac{1}{2}} \, dt\\[6pt]
                             &= \int_{-1}^1 \left(t -\frac{t}{4} - \frac14\right) \, dt\\[6pt]
                             &= \int_{-1}^1 \left(\frac34 t - \frac14\right) \, dt\\[6pt]
                             &= \left(\frac38 t^2 - \frac14 t\right)\bigg|_{-1}^1\\[6pt]
                         &= \left(\frac38 - \frac14\right) - \left(\frac38 + \frac14\right)\\[6 pt]
                        &= -\frac12
\end{align*}

An alternative method for parameterizing a line segment was discussed in chapter 1: if $P$ and $Q$ are the vector representation of points, the the line segment from 
$P$ to $Q$ is:
\[
\vec r(t) = (1-t)P + tQ, \; 0 \leq t \leq 1
\]
Using $(-1, 0)$ and $(1,1)$ for $P$ and $Q$, respectively, we can parameterize the line segment as
\[
\vec r(t) = (1-t) \vector{-1, 0} + t\vector{1,1} = \vector{2t-1, t}, \; 0\leq t \leq 1
\]
Thus,
\[
x(t) = 2t-1 \quad \text{and} \quad y(t) = t
\]
and the vector function is
\[
\vec f(x(t), y(t)) = \vector{x(t), -y(t)} = \vector{2t-1, -t}
\]
We can now compute the line integral
\begin{align*}
\int_C \vec f(x, y) \dotp d\vec r &= \int_0^1 \vec f(x(t), y(t)) \dotp \vec r\,'(t) \, dt\\
                             &= \int_0^1 \vector{2t-1, t} \dotp  \vector{2, 1} \, dt\\[6pt]
                             &= \int_0^1 \left(4t-2 -t\right) \, dt\\[6pt]
                             &= \int_0^1 \left(3t-2\right) \, dt\\[6pt]
                             &= \left(\frac32 t^2 - 2 t\right)\bigg|_0^1\\[6pt]
                        &= \frac32 - 2 = -\frac12
\end{align*}
as before.
\end{example}
\begin{remark}
A line integral is independent of the choice of the parameterization for the curve $C$. 
\end{remark}

\begin{problem}(Problem 1)
Compute $\dis \int_C \vector{x, -y} \dotp d\vec r$ where $C$ is the line segment from $(1,1)$ to $(-1, 0)$.
Note that this is the same line segment as in Example 1, but traversed in the opposite direction.\\
\[
\int_C \vector{x, -y} \dotp d\vec r = \answer{1/2}
\]
\end{problem}

\begin{remark}
As Example 1 and Problem 1 demonstrate, the line integral is sensitive to the orientation of the curve $C$.
Reversing the direction gives the negative of the line integral.
\end{remark}

\subsection{Work}

If the vector-valued function $\vec f(x,y)$ represents the force on an object at the point $(x,y)$, then the line integral
\[
\int_C \vec f(x,y) \dotp  d\vec r
\]
represents the \textbf{work} done by the force (or against the force, if the value of the integral is negative) on an object that travels along the curve $C$.

\begin{example}[Example 2]
The force due to gravity by a mass at the origin of $\R^2$ (two dimensions for simplicity) is given by
\[
\vec f(x,y) = \frac{k}{(x^2 + y^2)^{3/2}} \vector{-x, -y}
\]
where $k$ is a positive constant that depends on the masses of the objects involved. Find the work done 
by this force as an object moves in a straight line from the point $(0,1)$ to the point $(1,1)$.\\
The line segment from $(0,1)$ to $(1,1)$ can be parameterized by
\[
C: \vec r(t) = \vector{t, 1} \; \text{for}   \; 0 \leq t \leq 1
\]
The work done is the value of the line integral:
\begin{align*}
\text{Work} &= \int_C \vec f(x,y) \dotp d\vec r\\
            &= \int_0^1 \vec f(x(t),y(t)) \dotp \vec r\,'(t) \, dt\\
            &= \int_0^1 \vec f(t, 1) \dotp \vec r\,'(t) \, dt\\
            &= \int_0^1 \frac{k}{(t^2 + 1^2)^{3/2}} \vector{-t,-1} \dotp \vector{1,0} \, dt\\
            &= \int_0^1 -\frac{kt}{(t^2 + 1^2)^{3/2}} \, dt\\
            &= \frac{k}{(t^2 +1)^{1/2}} \bigg|_0^1 \\
            &= k\left(\frac{1}{\sqrt 2} - 1\right)
\end{align*}
Note that this answer is negative.  This means that work was done \textbf{against} the force rather than by the force.
\end{example}

\begin{problem}(Problem 2a)
Use the function given in Example 2 to describe the force due to gravity.
Find the work done by this force on an object as the object moves in a straight line from the point $(-1, 0)$ to the point $(-1,1)$.\\
\[
\text{Work} = \answer{k(\frac{1}{\sqrt 2} - 1}
\]
Was work done by gravity or against gravity in this situation? 
\end{problem}

\begin{problem}(Problem 2b)
Use the function given in Example 2 to describe the force due to gravity.
Find the work done on an object by this force as the object moves in a straight line from the point $(1, -1)$ to the point $(0, -1)$.\\
\[
\text{Work} = \answer{k(1-\frac{1}{\sqrt 2}}
\]
Was work done by gravity or against gravity in this situation? 
\end{problem}

Notice that the answers in Example 2 and Problems 2a and 2b are all similar due to rotational symmetry.

\begin{example}[Example 3]
The force due to gravity by a mass at the origin of $\R^2$ (two dimensions for simplicity) is given by
\[
\vec f(x,y) = \frac{k}{(x^2 + y^2)^{3/2}} \vector{-x, -y}
\]
where $k$ is a positive constant that depends on the masses of the objects involved. Find the work done on an object
by this force as the object moves in a quarter circle from the point $(2, 0)$ to the point $(0, 2)$.\\
The quarter circle from $(2, 0)$ to $(0, 2)$ can be parameterized by
\[
C: \vec r(t) = \vector{2\cos t, 2\sin t} \; \text{for}   \; 0 \leq t \leq \pi/2
\]
The work done is the value of the line integral:
\begin{align*}
\text{Work} &= \int_C \vec f(x,y) \dotp d\vec r\\
            &= \int_0^{\pi/2} \vec f(x(t),y(t)) \dotp \vec r\,'(t) \, dt\\
            &= \int_0^{\pi/2} \vec f(2\cos t, 2\sin t) \dotp \vec r\,'(t) \, dt\\
            &= \int_0^{\pi/2} \frac{k}{[(2\cos t)^2 + (2\sin t)^2)^{3/2}} \vector{-2\cos t, 2 \sin t} \dotp \vector{-2\sin t, 2 \cos t} \, dt\\
            &= \int_0^{\pi/2} 0 \, dt\\
            &= 0
\end{align*}
Note that this answer is zero because the direction of motion is orthogonal to the force (dot product is zero) and 
therefore no work is done by or against the force along this path.
\end{example}

\begin{problem}(Problem 3)
Use the function given in Example 3 to describe the force due to gravity.
Use a line integral to determine the work done on an object
by this force as the object moves in a semi-circle from the point $(0, 3)$ to the point $(0, -3)$.\\
\[
\text{Work} = \answer{0}
\]

\end{problem}


\end{document}

