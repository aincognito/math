\documentclass[handout]{ximera}

%% You can put user macros here
%% However, you cannot make new environments



\newcommand{\ffrac}[2]{\frac{\text{\footnotesize $#1$}}{\text{\footnotesize $#2$}}}
\newcommand{\vasymptote}[2][]{
    \draw [densely dashed,#1] ({rel axis cs:0,0} -| {axis cs:#2,0}) -- ({rel axis cs:0,1} -| {axis cs:#2,0});
}


%\usepackage{tcolorbox} %%Needed for Derivative Definition supposedly and product rule, natural exp log, quotient rule, inverse trig, rates of change


% \graphicspath{{./}{firstExample/}}
% \usepackage{forest}
\usepackage{amsmath}
\usepackage{amssymb}
\usepackage{array}
\usepackage[makeroom]{cancel} %% for strike outs
\usepackage{pgffor} %% required for integral for loops
\usepackage{tikz}
\usepackage{tikz-cd}
\usepackage{tkz-euclide}
\usetikzlibrary{shapes.multipart}


% \usetkzobj{all}
\tikzstyle geometryDiagrams=[ultra thick,color=blue!50!black]


\usetikzlibrary{arrows}
\tikzset{>=stealth,commutative diagrams/.cd,
  arrow style=tikz,diagrams={>=stealth}} %% cool arrow head
\tikzset{shorten <>/.style={ shorten >=#1, shorten <=#1 } } %% allows shorter vectors

\usetikzlibrary{backgrounds} %% for boxes around graphs
\usetikzlibrary{shapes,positioning}  %% Clouds and stars
\usetikzlibrary{matrix} %% for matrix
\usepgfplotslibrary{polar} %% for polar plots
\usepgfplotslibrary{fillbetween} %% to shade area between curves in TikZ



%\usepackage[width=4.375in, height=7.0in, top=1.0in, papersize={5.5in,8.5in}]{geometry}
%\usepackage[pdftex]{graphicx}
%\usepackage{tipa}
%\usepackage{txfonts}
%\usepackage{textcomp}
%\usepackage{amsthm}
%\usepackage{xy}
%\usepackage{fancyhdr}
%\usepackage{xcolor}
%\usepackage{mathtools} %% for pretty underbrace % Breaks Ximera
%\usepackage{multicol}



\newcommand{\RR}{\mathbb R}
\newcommand{\R}{\mathbb R}
\newcommand{\C}{\mathbb C}
\newcommand{\N}{\mathbb N}
\newcommand{\Z}{\mathbb Z}
\newcommand{\dis}{\displaystyle}
%\renewcommand{\d}{\,d\!}
\renewcommand{\d}{\mathop{}\!d}
\newcommand{\dd}[2][]{\frac{\d #1}{\d #2}}
\newcommand{\pp}[2][]{\frac{\partial #1}{\partial #2}}
\renewcommand{\l}{\ell}
\newcommand{\ddx}{\frac{d}{\d x}}
\newcommand{\ppx}{\frac{\partial}{\partial x}}
\newcommand{\ppy}{\frac{\partial}{\partial y}}

\newcommand{\zeroOverZero}{\ensuremath{\boldsymbol{\tfrac{0}{0}}}}
\newcommand{\inftyOverInfty}{\ensuremath{\boldsymbol{\tfrac{\infty}{\infty}}}}
\newcommand{\zeroOverInfty}{\ensuremath{\boldsymbol{\tfrac{0}{\infty}}}}
\newcommand{\zeroTimesInfty}{\ensuremath{\small\boldsymbol{0\cdot \infty}}}
\newcommand{\inftyMinusInfty}{\ensuremath{\small\boldsymbol{\infty - \infty}}}
\newcommand{\oneToInfty}{\ensuremath{\boldsymbol{1^\infty}}}
\newcommand{\zeroToZero}{\ensuremath{\boldsymbol{0^0}}}
\newcommand{\inftyToZero}{\ensuremath{\boldsymbol{\infty^0}}}


\newcommand{\numOverZero}{\ensuremath{\boldsymbol{\tfrac{\#}{0}}}}
\newcommand{\dfn}{\textbf}
%\newcommand{\unit}{\,\mathrm}
\newcommand{\unit}{\mathop{}\!\mathrm}
%\newcommand{\eval}[1]{\bigg[ #1 \bigg]}
\newcommand{\eval}[1]{ #1 \bigg|}
\newcommand{\seq}[1]{\left( #1 \right)}
\renewcommand{\epsilon}{\varepsilon}
\renewcommand{\iff}{\Leftrightarrow}

\DeclareMathOperator{\arccot}{arccot}
\DeclareMathOperator{\arcsec}{arcsec}
\DeclareMathOperator{\arccsc}{arccsc}
\DeclareMathOperator{\si}{Si}
\DeclareMathOperator{\proj}{proj}
\DeclareMathOperator{\scal}{scal}
\DeclareMathOperator{\cis}{cis}
\DeclareMathOperator{\Arg}{Arg}
%\DeclareMathOperator{\arg}{arg}
\DeclareMathOperator{\Rep}{Re}
\DeclareMathOperator{\Imp}{Im}
\DeclareMathOperator{\sech}{sech}
\DeclareMathOperator{\csch}{csch}
\DeclareMathOperator{\Log}{Log}

\newcommand{\tightoverset}[2]{% for arrow vec
  \mathop{#2}\limits^{\vbox to -.5ex{\kern-0.75ex\hbox{$#1$}\vss}}}
\newcommand{\arrowvec}{\overrightarrow}
\renewcommand{\vec}{\mathbf}
\newcommand{\veci}{{\boldsymbol{\hat{\imath}}}}
\newcommand{\vecj}{{\boldsymbol{\hat{\jmath}}}}
\newcommand{\veck}{{\boldsymbol{\hat{k}}}}
\newcommand{\vecl}{\boldsymbol{\l}}
\newcommand{\utan}{\vec{\hat{t}}}
\newcommand{\unormal}{\vec{\hat{n}}}
\newcommand{\ubinormal}{\vec{\hat{b}}}

\newcommand{\dotp}{\bullet}
\newcommand{\cross}{\boldsymbol\times}
\newcommand{\grad}{\boldsymbol\nabla}
\newcommand{\divergence}{\grad\dotp}
\newcommand{\curl}{\grad\cross}
%% Simple horiz vectors
\renewcommand{\vector}[1]{\left\langle #1\right\rangle}


\outcome{In this section we compute double integrals using polar coordiantes.}

\title{4.3 Polar Integrals}



\begin{document}

\begin{abstract}
In this section we compute double integrals using polar coordiantes.
\end{abstract}
 
\maketitle
The area of a \textbf{sector} of a circle corresponding to the angle $\theta$ and radius $r$ is a proportion of the area of the circle itself:  
\[
\text{Area of sector}  = \frac{\theta}{2\pi} \cdot \text{Area of circle}
\]
See the figure below.
\begin{image}
\begin{tikzpicture}
\filldraw[fill = blue!30!white, draw = black, thick] (0,0) -- (3,0) arc (0: 60: 3) -- cycle node[midway, above left]{$r$};
\node at (0.38, 0.25) {$\theta$};
%(2.6,3/2) -- (3.464, 2) 
%arc (30: 45: 4) --  (2.12, 2.12) node[midway, above left]{$\Delta r$} arc (45:30: 3) node[midway, below left]{$\Delta \theta$};
%\draw[thin] (3, 1.75) arc (30: 32: 3.5);
%\draw[thin] (3, 1.75) arc (30: 28: 3.5) node[below right]{$r^*$};

%\draw[thin, dashed] (0,0) -- (2.6, 1.5);
%\draw[thin, dashed] (0,0) -- (2.12, 2.12);
%\node at (1.5, 1.25) ;
%\draw[thick] (3,0) node[below]{$2$} arc (0: 90: 3) node[left]{$2$};
%\draw[thick, <->] (-0.75, 0) -- (4, 0) node[right]{$x$};
%\draw[thick, <->] (0,-.75) -- (0, 4) node[above]{$y$};
%\draw[thick] (1.8,0) node[below]{$x$} -- (1.8, 2.4) node[above right]{$y = \sqrt{4 - x^2}$};

\node at (1.4, -.5) {$A = \frac12 r^2 \theta$};
\end{tikzpicture}
\end{image}

Integrating in polar coordinates requires 
knowledge of the area of a \textbf{polar rectangle}.
A polar rectangle is a region in the $xy$-plane has the form 
\[
r_1 \leq r\leq r_2\quad \text{and} \quad \theta_1 \leq \theta \leq \theta_2
\]
\begin{image}
\begin{tikzpicture}
\filldraw[fill = blue!30!white, draw = black, thick] (2.6,3/2)  -- (3.464, 2) arc (30: 45: 4) --  (2.12, 2.12)  arc (45:30: 3);

\draw[thin, dashed] (3,0) node[below]{$r_1$} arc(0:30:3);
\draw[thin, dashed] (4,0) node[below]{$r_2$} arc(0:30:4);

\draw[thin, dashed] (0,0) -- (2.6, 1.5);
\draw[thin, dashed] (0,0) -- (2.12, 2.12);
\draw[thick] (0,0) -- (4.5,0);
\draw (0.7, 0) arc (0:30:0.7) node[midway, right]{$\theta_1$};
\draw (1.5, 0) arc (0:45:1.5) node[midway, right]{$\theta_2$};

%\node at (1.5, 1.25) ;
%\draw[thick] (3,0) node[below]{$2$} arc (0: 90: 3) node[left]{$2$};
%\draw[thick, <->] (-0.75, 0) -- (4, 0) node[right]{$x$};
%\draw[thick, <->] (0,-.75) -- (0, 4) node[above]{$y$};
%\draw[thick] (1.8,0) node[below]{$x$} -- (1.8, 2.4) node[above right]{$y = \sqrt{4 - x^2}$};
\node at (2, -.75) {A polar rectangle};
\end{tikzpicture}
\end{image}

Its area is computed as the difference of two sectors: 
\begin{align*}
S_1&: 0 \leq r\leq r_1, \; \theta_1 \leq \theta \leq \theta_2 \quad\text{and} \\
S_2&: 0 \leq r\leq r_2, \; \theta_1 \leq \theta \leq \theta_2
\end{align*}
and is given by
\begin{align*}
A &= \frac12 (\theta_2 - \theta_1) r_2^2 - \frac12 (\theta_2 -\theta_1) r_1^2\\
  &= \frac12 (\theta_2 - \theta_1) (r_2^2 - r_1^2)\\
  &= \frac12 (\theta_2 - \theta_1) (r_2 - r_1)(r_2 + r_1)\\
  &= \frac12(r_2 + r_1) \Delta \theta \Delta r \\
  &= r^* \Delta r \Delta \theta
\end{align*}
where
\[
r^* = \frac{r_1 + r_2}{2}, \; \Delta r = r_2 - r_1 \; \text{and} \Delta \theta = \theta_2 - \theta_1
\]
See the figure below.

\begin{image}
\begin{tikzpicture}
\filldraw[fill = blue!30!white, draw = black, thick] (2.6,3/2) -- (3.464, 2)
arc (30: 45: 4) --  (2.12, 2.12) node[midway, above left]{$\Delta r$} arc (45:30: 3) node[midway, below left]{$\Delta \theta$};
\draw[thin] (3, 1.75) arc (30: 32: 3.5);
\draw[thin] (3, 1.75) arc (30: 28: 3.5) node[below right]{$r^*$};
 
\draw[thin, dashed] (0,0) -- (2.6, 1.5);
\draw[thin, dashed] (0,0) -- (2.12, 2.12);
%\node at (1.5, 1.25) ;
%\draw[thick] (3,0) node[below]{$2$} arc (0: 90: 3) node[left]{$2$};
%\draw[thick, <->] (-0.75, 0) -- (4, 0) node[right]{$x$};
%\draw[thick, <->] (0,-.75) -- (0, 4) node[above]{$y$};
%\draw[thick] (1.8,0) node[below]{$x$} -- (1.8, 2.4) node[above right]{$y = \sqrt{4 - x^2}$};
%\node at (1.7, -.25) {The area of the sector is:};
\node at (1.7, -.25) {$A = r^* \Delta r \Delta \theta$};
\end{tikzpicture}
\end{image}

%node[below]{$r_1$}

Suppose that the region $R$ is a polar rectangle. 
When converting a double integral over $R$ into polar coordinates, the formula for the area of a polar rectangle suggests that
\[
\iint_R f(x,y) \, dA = \int_{\theta_1}^{\theta_2} \int_{r_1}^{r_2} f(r, \theta) r \, dr\, d\theta
\]


\begin{example}[Example 1]
Compute $\iint_R \sqrt{x^2 + y^2} \, dA$ where $R$ is the quarter circle in the first quadrant bounded by $x=0, y=0$ and $x^2 + y^2 = 4$

\begin{image}
\begin{tikzpicture}
\filldraw[blue!30!white] (0,0) -- (3,0) arc (0: 90: 3) -- cycle;
\draw[thick] (3,0) node[below]{$2$} arc (0: 90: 3) node[left]{$2$};
\draw[thick, <->] (-0.75, 0) -- (4, 0) node[right]{$x$};
\draw[thick, <->] (0,-.75) -- (0, 4) node[above]{$y$};
\draw[thick] (1.8,0) node[below]{$x$} -- (1.8, 2.4) node[above right]{$y = \sqrt{4 - x^2}$};
\node at (1.7, -1.25) {For each $x$ between $0$ and $2$,};
\node at (1.7, -1.75) {$y$ varies from $0$ to $\sqrt{4 - x^2}$};
\end{tikzpicture}
\end{image}
Writing this integral as interated integrals will eave us with a difficult anti-differentiation problem regardless of the order chosen.
However, the integral is fairly easy to compute using polar coordinates. The region $R$ can be seen as a polar rectangle with 
\[
0\leq r \leq 2 \quad \text{and} \quad 0 \leq \theta \leq \frac{\pi}{2}
\]
Moreover, the function $f(x,y) = \sqrt{x^2 + y^2}$ transforms into a simpler function in polar coordinates. With
\[
x = r\cos \theta \quad \text{and} \quad y = r\sin \theta
\]
the function becomes
\[
\sqrt{x^2 +y^2} = \sqrt{r^2 \cos^2 \theta + r^2 \sin^2 \theta} = \sqrt{r^2} = r
\]
since $r \geq 0$.
Using polar coordinates, we have
\begin{align*}
\iint_R \sqrt{x^2 + y^2} \, dA & = \int_0^{\pi/2} \int_0^2 r \cdot r \, dr \, d\theta\\
                 & = \int_0^{\pi/2}  \frac{r^3}{3} \bigg|_{0}^2 \, d\theta\\
                 & = \int_0^{\pi/2} \frac83\, d\theta\\
                 &= \frac{4\pi}{3}
\end{align*}  

\end{example}



\begin{example}[Example 2]
Compute $\iint_R \tan^{-1}\left(\frac{y}{x}\right) \, dA$ where $R$ is polar rectangle $ 2 \leq r \leq 3$ and $-\pi/2 \leq \theta \leq \pi/2$.\\

\begin{image}
\begin{tikzpicture}
\draw[thick, fill = blue!30!white] (0, -2) node[left]{$-2$} -- (0, -3) node[left]{$-3$} arc (-90:90:3) node[left]{$3$} -- (0,2) node[left]{$2$} arc (90:-90:2);
\draw[thick, <->] (-.75, 0) -- (4, 0) node[right]{$x$};
\draw[thick, <->] (0,-3.5) -- (0, 3.5) node[above]{$y$};

\node at (1.25, -3.75) {Polar rectangle};
\node at (1.25, -4.25) {$0\leq r \leq 2$ and $-\pi/2 \leq \theta \leq \pi/2$};
\end{tikzpicture}
\end{image}

Since $x = r\cos \theta$ and $y = r\sin \theta$, we have
\[
\frac{y}{x} = \frac{r\sin \theta}{r\cos\theta} = \tan \theta
\]
Since, $-\pi/2 \leq \theta \leq \pi/2$,
\[
\tan^{-1} \left(\frac{y}{x}\right) = \tan^{-1} (\tan \theta) = \theta
\]
In polar coordinates, the integral becomes
\begin{align*}
\iint_R \tan^{-1}\left(\frac{y}{x}\right) \, dA & = \int_{-\pi/2}^{\pi/2} \int_2^3 \theta \cdot r \, dr \, d\theta\\
                                                & = \int_{-\pi/2}^{\pi/2}  \theta \cdot \frac{r^2}{2}\bigg|_2^3 \, d\theta\\
                                                & = \frac52 \int_{-\pi/2}^{\pi/2}  \theta \, d\theta\\
                                                &= \frac52 \frac{\theta^2}{2} \bigg|_{-\pi/2}^{\pi/2}\\
                                                &= \frac52 \left(\frac{\pi^2}{4} - \frac{\pi^2}{4}\right)\\
                                                &= 0
\end{align*}  

\end{example}


\begin{example}[Example 3]
Find the volume of the region below the paraboloid $z = 1 - x^2 - y^2$ and above the $xy$-plane.\\

\begin{image}
\begin{tikzpicture}
%\draw[thick] (0,0) ellipse (4 and 1) ;
\draw[thick, dashed] (4,0) arc(0:180:4 and 1);
\draw[thick] (4,0) arc(0:-180:4 and 1);

\node at (1.75, -1.25){$x^2 + y^2 = 1$};
%\draw (-4,0) parabola bend +(0, 4) ++ (4, 0);
\draw[thick] plot[smooth, domain= -4: 4] (\x, {4-(\x/2)^2});
\draw[thick, <->] (0,-2) -- (0, 5) node[above]{$z$};
\draw[thick, <->] (-5,0) -- (5, 0) node[right]{$y$};
\draw[thick, <->] (1.5*1.732, 1.5) -- (-1.5 * 1.732, -1.5) node[below, left]{$x$};
\node at (3, 3.5) {$z = 1 - x^2 - y^2$};

\end{tikzpicture}
\end{image}

\begin{align*}
\iint_R (1-x^2 - y^2) \, dA &= \int_0^{2\pi} \int_0^1 (1-r^2) \cdot r \, dr \, d\theta\\
                            &= 2\pi \left(\frac{r^2}{2} - \frac{r^4}{4}\right) \bigg|_0^1\\
                            &= 2\pi \cdot \frac14 = \frac{\pi}{2}
\end{align*}
\end{example}


\end{document}
