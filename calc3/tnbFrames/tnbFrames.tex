\documentclass[handout]{ximera}

%% You can put user macros here
%% However, you cannot make new environments



\newcommand{\ffrac}[2]{\frac{\text{\footnotesize $#1$}}{\text{\footnotesize $#2$}}}
\newcommand{\vasymptote}[2][]{
    \draw [densely dashed,#1] ({rel axis cs:0,0} -| {axis cs:#2,0}) -- ({rel axis cs:0,1} -| {axis cs:#2,0});
}


\graphicspath{{./}{firstExample/}}
\usepackage{forest}
\usepackage{amsmath}
\usepackage{amssymb}
\usepackage{array}
\usepackage[makeroom]{cancel} %% for strike outs
\usepackage{pgffor} %% required for integral for loops
\usepackage{tikz}
\usepackage{tikz-cd}
\usepackage{tkz-euclide}
\usetikzlibrary{shapes.multipart}


%\usetkzobj{all}
\tikzstyle geometryDiagrams=[ultra thick,color=blue!50!black]


\usetikzlibrary{arrows}
\tikzset{>=stealth,commutative diagrams/.cd,
  arrow style=tikz,diagrams={>=stealth}} %% cool arrow head
\tikzset{shorten <>/.style={ shorten >=#1, shorten <=#1 } } %% allows shorter vectors

\usetikzlibrary{backgrounds} %% for boxes around graphs
\usetikzlibrary{shapes,positioning}  %% Clouds and stars
\usetikzlibrary{matrix} %% for matrix
\usepgfplotslibrary{polar} %% for polar plots
\usepgfplotslibrary{fillbetween} %% to shade area between curves in TikZ



%\usepackage[width=4.375in, height=7.0in, top=1.0in, papersize={5.5in,8.5in}]{geometry}
%\usepackage[pdftex]{graphicx}
%\usepackage{tipa}
%\usepackage{txfonts}
%\usepackage{textcomp}
%\usepackage{amsthm}
%\usepackage{xy}
%\usepackage{fancyhdr}
%\usepackage{xcolor}
%\usepackage{mathtools} %% for pretty underbrace % Breaks Ximera
%\usepackage{multicol}



\newcommand{\RR}{\mathbb R}
\newcommand{\R}{\mathbb R}
\newcommand{\C}{\mathbb C}
\newcommand{\N}{\mathbb N}
\newcommand{\Z}{\mathbb Z}
\newcommand{\dis}{\displaystyle}
%\renewcommand{\d}{\,d\!}
\renewcommand{\d}{\mathop{}\!d}
\newcommand{\dd}[2][]{\frac{\d #1}{\d #2}}
\newcommand{\pp}[2][]{\frac{\partial #1}{\partial #2}}
\renewcommand{\l}{\ell}
\newcommand{\ddx}{\frac{d}{\d x}}

\newcommand{\zeroOverZero}{\ensuremath{\boldsymbol{\tfrac{0}{0}}}}
\newcommand{\inftyOverInfty}{\ensuremath{\boldsymbol{\tfrac{\infty}{\infty}}}}
\newcommand{\zeroOverInfty}{\ensuremath{\boldsymbol{\tfrac{0}{\infty}}}}
\newcommand{\zeroTimesInfty}{\ensuremath{\small\boldsymbol{0\cdot \infty}}}
\newcommand{\inftyMinusInfty}{\ensuremath{\small\boldsymbol{\infty - \infty}}}
\newcommand{\oneToInfty}{\ensuremath{\boldsymbol{1^\infty}}}
\newcommand{\zeroToZero}{\ensuremath{\boldsymbol{0^0}}}
\newcommand{\inftyToZero}{\ensuremath{\boldsymbol{\infty^0}}}


\newcommand{\numOverZero}{\ensuremath{\boldsymbol{\tfrac{\#}{0}}}}
\newcommand{\dfn}{\textbf}
%\newcommand{\unit}{\,\mathrm}
\newcommand{\unit}{\mathop{}\!\mathrm}
%\newcommand{\eval}[1]{\bigg[ #1 \bigg]}
\newcommand{\eval}[1]{ #1 \bigg|}
\newcommand{\seq}[1]{\left( #1 \right)}
\renewcommand{\epsilon}{\varepsilon}
\renewcommand{\iff}{\Leftrightarrow}

\DeclareMathOperator{\arccot}{arccot}
\DeclareMathOperator{\arcsec}{arcsec}
\DeclareMathOperator{\arccsc}{arccsc}
\DeclareMathOperator{\si}{Si}
\DeclareMathOperator{\proj}{proj}
\DeclareMathOperator{\scal}{scal}
\DeclareMathOperator{\cis}{cis}
\DeclareMathOperator{\Arg}{Arg}
%\DeclareMathOperator{\arg}{arg}
\DeclareMathOperator{\Rep}{Re}
\DeclareMathOperator{\Imp}{Im}
\DeclareMathOperator{\sech}{sech}
\DeclareMathOperator{\csch}{csch}
\DeclareMathOperator{\Log}{Log}

\newcommand{\tightoverset}[2]{% for arrow vec
  \mathop{#2}\limits^{\vbox to -.5ex{\kern-0.75ex\hbox{$#1$}\vss}}}
\newcommand{\arrowvec}{\overrightarrow}
\renewcommand{\vec}{\mathbf}
\newcommand{\veci}{{\boldsymbol{\hat{\imath}}}}
\newcommand{\vecj}{{\boldsymbol{\hat{\jmath}}}}
\newcommand{\veck}{{\boldsymbol{\hat{k}}}}
\newcommand{\vecl}{\boldsymbol{\l}}
\newcommand{\utan}{\vec{\hat{t}}}
\newcommand{\unormal}{\vec{\hat{n}}}
\newcommand{\ubinormal}{\vec{\hat{b}}}

\newcommand{\dotp}{\bullet}
\newcommand{\cross}{\boldsymbol\times}
\newcommand{\grad}{\boldsymbol\nabla}
\newcommand{\divergence}{\grad\dotp}
\newcommand{\curl}{\grad\cross}
%% Simple horiz vectors
\renewcommand{\vector}[1]{\left\langle #1\right\rangle}


\outcome{Compute curvature.}

\title{2.6 TNB Frames}



\begin{document}

\begin{abstract}
In this section we determine the unit Normal and unit Binormal vectors.
\end{abstract}

\maketitle


The unit tangent vector $\vec T$ to a smooth space curve $\vec r(t)$ is given by
\[
\vec T(t) = \frac{\vec r\,'(t)}{|\vec r\,'(t)|}
\]
Since $|\vec T(t)| = 1$ for all $t$, we have
\[
\vec T(t) \dotp \vec T\,'(t) = 0
\]
which means that the vector $\vec T\,'(t)$ is orthogonal to the unit tangent vector $\vec T(t)$.

\begin{definition}[Unit Normal Vector]
The unit normal vector $\vec N(t)$ is defined as
\[
\vec N(t) = \frac{\vec T\,'(t)}{|\vec T\,'(t)|}
\]
\end{definition}

\begin{remark} The unit normal vector $\vec N$ points in the direction that the curve 
$\vec r(t)$ is bending.
\end{remark}

\begin{image}
\begin{tikzpicture}
\draw[thick, ->] (0, 0) -- (0.01,0.4);
\draw[thick] (0, 0) to [out = 90, in = 225] (.8, 2.1);
\draw[thick, blue, ->] (.8,2.1) -- (1.5, 2.8) node[above]{$\vec T(t)$};
\draw[thick, red, ->] (.8,2.1) -- (1.5, 1.4) node[below]{$\vec N(t)$};
\draw[thick, ->] (.8, 2.1) to [out = 45, in = 180] (2, 2.5);
\draw[thick, ] (1.95, 2.5) to [out = 0, in = 120] (4, 1);

\draw[thick, blue, ->] (4,1) -- (4.5, .15) node[right]{$\vec T(t)$};
\draw[thick, red, ->] (4,1) -- (3.15, 0.5) node[below]{$\vec N(t)$};


 %TN vecs
\draw[thick, ->] (4, 1) to [out = 300, in = 110] (4.5, -0.5);
\draw[thick] (4.48, -0.46) to [out = 290, in = 180] (5.5, -1.5); %TN vecs
\draw[thick, ->] (5.5, -1.5) to [out = 0, in = 250] (7, 1) node[right]{$\vec r(t)$}; %TN vecs

\draw[thick, blue, ->] (5.5, -1.5) -- (6.5, -1.5) node[right]{$\vec T(t)$};
\draw[thick, red, ->] (5.5, -1.5) -- (5.5, -0.5) node[right]{$\vec N(t)$};

\node at (3.25, -2.5) {The unit tangent and unit normal vectors};

\end{tikzpicture}
\end{image}


\begin{example}[example 1]
Find the unit tangent and unit normal vectors to the spiral helix $\vec r(t) =  \vector{2\cos(3t), 2\sin(3t), 5t}$ at the point $(-2, 0, 5\pi)$.\\
From the $z$-coordinate of the point, we can see that the point $(-2, 0, 5\pi)$ corresponds to $t = \pi$.\\
The tangent vector is
\[
\vec r \,'(t) = \vector{-6 \sin(3t), 6\cos(3t), 5}
\]
and the unit tangent vector is
\[
\vec T(t) = \frac{\vec r\,'(t)}{|\vec r\,'(t)|} = \frac{\vector{-6 \sin(3t), 6\cos(3t), 5}}{\sqrt{61}}
\]

To compute the unit normal vecotr, we differentiate the unit tangent vector:
\[
\vec T\,'(t) = \frac{1}{\sqrt{61}} \vector{-18 \cos(3t), -18 \sin(3t), 0}
\]
The unit normal vector is given by
\[
\vec N(t) = \frac{\vec T\,'(t)}{|\vec T\,'(t)|} = \frac{1}{18} \vector{-18 \cos(3t), -18 \sin(3t), 0} = \vector{-\cos(3t), -\sin(3t), 0}
\]
Note that this vector points towards the $z$-axis.\\
Finally, at the point $(-2, 0, 5\pi)$, we have
\[
\vec T(\pi) = \vector{0, -\frac{6}{\sqrt{61}}, \frac{5}{\sqrt{61}}}
\]
\[
\vec N(\pi) = \vector{1, 0, 0}
\]
\end{example}

The unit binormal vector, $\vec B$, is orthogonal to both the unit tangent vector, $\vec T$, and the unit normal vector, $\vec N$. To define it, we use the cross product.

\begin{definition}[Unit Binormal Vector]
The unit binormal vector is defined by
\[
\vec B(t) = \vec T(t) \cross \vec N(t)
\]
\end{definition}

\begin{remark}
The unit binormal vector is orthogonal to both $\vec T$ and $\vec N$. 
\end{remark}

\begin{remark}
The unit binormal vector $\vec B$ is indeed a unit vector since $\vec T$ and $\vec N$ are orthogonal unit vectors:
\[
|\vec B| = |\vec T| \cdot |\vec N| \sin \theta = 1 \cdot 1 \cdot 1 = 1.
\]
\end{remark}

\begin{example}[Example 2]
Find the unit binormal vector to the spiral helix $\vec r(t) =  \vector{2\cos(3t), 2\sin(3t), 5t}$ at the point $(-2, 0, 5\pi)$.\\
In example 1, we found that
\[
\vec T(t) = \frac{1}{\sqrt{61}} \vector{-6 \sin(3t), 6\cos(3t), 5} \quad \text{and} \quad \vec N(t) = \vector{-\cos(3t), -\sin(3t), 0}
\]

Hence
\begin{align*}
\vec B(t) &= \vec T(t) \cross \vec N(t)\\
          &= \frac{1}{\sqrt{61}} \vector{-6 \sin(3t), 6\cos(3t), 5} \cross \vector{-\cos(3t), -\sin(3t), 0}\\
          &= \frac{1}{\sqrt{61}} \vector{5 \sin(3t), -5 \cos(3t), 6} \quad \text{(verify)}
\end{align*}
It is a simple matter to check that 
\[
\vec T \dotp \vec B = 0, \quad \vec N \dotp \vec B = 0, \quad \text{and} \quad |\vec B| = 1
\]
for all $t$.\\
Finally, at the point $(-2, 0, 5\pi)$ we have $t = \pi$ and
\[
\vec B(\pi) = \vector{0, \frac{5}{\sqrt {61}}, \frac{6}{\sqrt{61}}}
\]
\end{example}

The three vectors $\vec T(t), \vec N(t)$ and $\vec B(t)$ form what is called a TNB frame at each point on the curve $\vec r(t)$.

\end{document}
