\documentclass[handout]{ximera}

%% You can put user macros here
%% However, you cannot make new environments



\newcommand{\ffrac}[2]{\frac{\text{\footnotesize $#1$}}{\text{\footnotesize $#2$}}}
\newcommand{\vasymptote}[2][]{
    \draw [densely dashed,#1] ({rel axis cs:0,0} -| {axis cs:#2,0}) -- ({rel axis cs:0,1} -| {axis cs:#2,0});
}


\graphicspath{{./}{firstExample/}}
\usepackage{forest}
\usepackage{amsmath}
\usepackage{amssymb}
\usepackage{array}
\usepackage[makeroom]{cancel} %% for strike outs
\usepackage{pgffor} %% required for integral for loops
\usepackage{tikz}
\usepackage{tikz-cd}
\usepackage{tkz-euclide}
\usetikzlibrary{shapes.multipart}


%\usetkzobj{all}
\tikzstyle geometryDiagrams=[ultra thick,color=blue!50!black]


\usetikzlibrary{arrows}
\tikzset{>=stealth,commutative diagrams/.cd,
  arrow style=tikz,diagrams={>=stealth}} %% cool arrow head
\tikzset{shorten <>/.style={ shorten >=#1, shorten <=#1 } } %% allows shorter vectors

\usetikzlibrary{backgrounds} %% for boxes around graphs
\usetikzlibrary{shapes,positioning}  %% Clouds and stars
\usetikzlibrary{matrix} %% for matrix
\usepgfplotslibrary{polar} %% for polar plots
\usepgfplotslibrary{fillbetween} %% to shade area between curves in TikZ



%\usepackage[width=4.375in, height=7.0in, top=1.0in, papersize={5.5in,8.5in}]{geometry}
%\usepackage[pdftex]{graphicx}
%\usepackage{tipa}
%\usepackage{txfonts}
%\usepackage{textcomp}
%\usepackage{amsthm}
%\usepackage{xy}
%\usepackage{fancyhdr}
%\usepackage{xcolor}
%\usepackage{mathtools} %% for pretty underbrace % Breaks Ximera
%\usepackage{multicol}



\newcommand{\RR}{\mathbb R}
\newcommand{\R}{\mathbb R}
\newcommand{\C}{\mathbb C}
\newcommand{\N}{\mathbb N}
\newcommand{\Z}{\mathbb Z}
\newcommand{\dis}{\displaystyle}
%\renewcommand{\d}{\,d\!}
\renewcommand{\d}{\mathop{}\!d}
\newcommand{\dd}[2][]{\frac{\d #1}{\d #2}}
\newcommand{\pp}[2][]{\frac{\partial #1}{\partial #2}}
\renewcommand{\l}{\ell}
\newcommand{\ddx}{\frac{d}{\d x}}

\newcommand{\zeroOverZero}{\ensuremath{\boldsymbol{\tfrac{0}{0}}}}
\newcommand{\inftyOverInfty}{\ensuremath{\boldsymbol{\tfrac{\infty}{\infty}}}}
\newcommand{\zeroOverInfty}{\ensuremath{\boldsymbol{\tfrac{0}{\infty}}}}
\newcommand{\zeroTimesInfty}{\ensuremath{\small\boldsymbol{0\cdot \infty}}}
\newcommand{\inftyMinusInfty}{\ensuremath{\small\boldsymbol{\infty - \infty}}}
\newcommand{\oneToInfty}{\ensuremath{\boldsymbol{1^\infty}}}
\newcommand{\zeroToZero}{\ensuremath{\boldsymbol{0^0}}}
\newcommand{\inftyToZero}{\ensuremath{\boldsymbol{\infty^0}}}


\newcommand{\numOverZero}{\ensuremath{\boldsymbol{\tfrac{\#}{0}}}}
\newcommand{\dfn}{\textbf}
%\newcommand{\unit}{\,\mathrm}
\newcommand{\unit}{\mathop{}\!\mathrm}
%\newcommand{\eval}[1]{\bigg[ #1 \bigg]}
\newcommand{\eval}[1]{ #1 \bigg|}
\newcommand{\seq}[1]{\left( #1 \right)}
\renewcommand{\epsilon}{\varepsilon}
\renewcommand{\iff}{\Leftrightarrow}

\DeclareMathOperator{\arccot}{arccot}
\DeclareMathOperator{\arcsec}{arcsec}
\DeclareMathOperator{\arccsc}{arccsc}
\DeclareMathOperator{\si}{Si}
\DeclareMathOperator{\proj}{proj}
\DeclareMathOperator{\scal}{scal}
\DeclareMathOperator{\cis}{cis}
\DeclareMathOperator{\Arg}{Arg}
%\DeclareMathOperator{\arg}{arg}
\DeclareMathOperator{\Rep}{Re}
\DeclareMathOperator{\Imp}{Im}
\DeclareMathOperator{\sech}{sech}
\DeclareMathOperator{\csch}{csch}
\DeclareMathOperator{\Log}{Log}

\newcommand{\tightoverset}[2]{% for arrow vec
  \mathop{#2}\limits^{\vbox to -.5ex{\kern-0.75ex\hbox{$#1$}\vss}}}
\newcommand{\arrowvec}{\overrightarrow}
\renewcommand{\vec}{\mathbf}
\newcommand{\veci}{{\boldsymbol{\hat{\imath}}}}
\newcommand{\vecj}{{\boldsymbol{\hat{\jmath}}}}
\newcommand{\veck}{{\boldsymbol{\hat{k}}}}
\newcommand{\vecl}{\boldsymbol{\l}}
\newcommand{\utan}{\vec{\hat{t}}}
\newcommand{\unormal}{\vec{\hat{n}}}
\newcommand{\ubinormal}{\vec{\hat{b}}}

\newcommand{\dotp}{\bullet}
\newcommand{\cross}{\boldsymbol\times}
\newcommand{\grad}{\boldsymbol\nabla}
\newcommand{\divergence}{\grad\dotp}
\newcommand{\curl}{\grad\cross}
%% Simple horiz vectors
\renewcommand{\vector}[1]{\left\langle #1\right\rangle}


\outcome{Compute curvature.}

\title{2.6 TNB Frames}



\begin{document}

\begin{abstract}
In this section we determine the unit Normal and Binormal vectors.
\end{abstract}

\maketitle


The unit tangent vector $\vec T$ to a smooth space curve $\vec r(t)$ is given by
\[
\vec T(t) = \frac{\vec r\,'(t)}{|\vec r\,'(t)|}
\]
Since $|\vec T(t)| = 1$ for all $t$, we have
\[
\vec T(t) \dotp \vec T\,'(t) = 0
\]
which means that the vector $\vec T\,'(t)$ is orthogonal to the unit tangent vector $\vec T(t)$.

\begin{definition}[Unit Normal Vector]
The unit normal vector $\vec N(t)$ is defined as
\[
\vec N(t) = \frac{\vec T\,'(t)}{|\vec T\,'(t)|}
\]
\end{definition}

\begin{remark} The unit normal vector $\vec N$ points in the direction that the curve 
$\vec r(t)$ is bending.
\end{remark}

\begin{image}
\begin{tikzpicture}
\draw[thick, ->] (0, 0) -- (0.01,0.4);
\draw[thick] (0, 0) to [out = 90, in = 225] (.8, 2.1);
\draw[thick, blue, ->] (.8,2.1) -- (1.5, 2.8) node[above]{$\vec T(t)$};
\draw[thick, red, ->] (.8,2.1) -- (1.5, 1.4) node[below]{$\vec N(t)$};
\draw[thick, ->] (.8, 2.1) to [out = 45, in = 180] (2, 2.5);
\draw[thick, ] (1.95, 2.5) to [out = 0, in = 120] (4, 1);

\draw[thick, blue, ->] (4,1) -- (4.5, .15) node[right]{$\vec T(t)$};
\draw[thick, red, ->] (4,1) -- (3.15, 0.5) node[below]{$\vec N(t)$};


 %TN vecs
\draw[thick, ->] (4, 1) to [out = 300, in = 110] (4.5, -0.5);
\draw[thick] (4.48, -0.46) to [out = 290, in = 180] (5.5, -1.5); %TN vecs
\draw[thick, ->] (5.5, -1.5) to [out = 0, in = 250] (7, 1) node[right]{$\vec r(t)$}; %TN vecs

\draw[thick, blue, ->] (5.5, -1.5) -- (6.5, -1.5) node[right]{$\vec T(t)$};
\draw[thick, red, ->] (5.5, -1.5) -- (5.5, -0.5) node[right]{$\vec N(t)$};

\node at (3.25, -2.5) {The unit tangent and unit normal vectors};

\end{tikzpicture}
\end{image}


\begin{example}[example 1]
Find the unit tangent and unit normal vectors to the spiral helix $\vec r(t) =  \vector{2\cos(3t), 2\sin(3t), 5t}$ at the point $(-2, 0, 5\pi)$.\\
From the $z$-coordinate of the point, we can see that the point $(-2, 0, 5\pi)$ corresponds to $t = \pi$.\\
The tangent vector is
\[
\vec r \,'(t) = \vector{-6 \sin(3t), 6\cos(3t), 5}
\]
and the unit tangent vector is
\[
\vec T(t) = \frac{\vec r\,'(t)}{|\vec r\,'(t)|} = \frac{\vector{-6 \sin(3t), 6\cos(3t), 5}}{\sqrt{61}}
\]

To compute the unit normal vector, we differentiate the unit tangent vector:
\[
\vec T\,'(t) = \frac{1}{\sqrt{61}} \vector{-18 \cos(3t), -18 \sin(3t), 0}
\]
The unit normal vector is given by
\[
\vec N(t) = \frac{\vec T\,'(t)}{|\vec T\,'(t)|} = \frac{1}{18} \vector{-18 \cos(3t), -18 \sin(3t), 0} = \vector{-\cos(3t), -\sin(3t), 0}
\]
Note that this vector points towards the $z$-axis.\\
Finally, at the point $(-2, 0, 5\pi)$, we have
\[
\vec T(\pi) = \vector{0, -\frac{6}{\sqrt{61}}, \frac{5}{\sqrt{61}}} \quad \text{and} \quad \vec N(\pi) = \vector{1, 0, 0}
\]
\end{example}


\begin{problem}(Problem 1)
Find the unit tangent and unit normal vectors to the spiral helix $\vec r(t) = \vector{4t, 5 \sin(2t),5\cos(2t)}$ at the point $(\pi, 5, 0)$.\\
\[
\vec T(t) = \vector{\answer{\frac{4}{\sqrt{116}}}, \answer{\frac{10}{\sqrt{116}}\cos(2t)}, \answer{-\frac{10}{\sqrt{116}}\sin(2t)}}
\]
\[
\vec N(t) = \vector{\answer{0}, \answer{-\sin(2t)}, \answer{-\cos(2t)}}
\]
The value of $t$ corresponding to the point $(\pi, 5, 0)$ is $t = \answer{\pi /4}$\\

At this point, the unit tangent vector is $\displaystyle \vec T = \vector{\answer{\frac{4}{\sqrt{116}}}, \answer{0}, \answer{-\frac{10}{\sqrt{116}}}}$\\
At this point, the unit normal vector is $\displaystyle \vec N = \vector{\answer{0}, \answer{-1}, \answer{0}}$\\

\end{problem}





The binormal vector, $\vec B$, is orthogonal to both the unit tangent vector, $\vec T$, and the unit normal vector, $\vec N$. To define it, we use the cross product.

\begin{definition}[Binormal Vector]
The binormal vector is defined by
\[
\vec B(t) = \vec T(t) \cross \vec N(t)
\]
\end{definition}

\begin{remark}
The binormal vector is orthogonal to both $\vec T$ and $\vec N$. 
\end{remark}

\begin{remark}
The binormal vector $\vec B$ is a \textbf{unit} vector since $\vec T$ and $\vec N$ are orthogonal unit vectors:
\[
|\vec B| = |\vec T \cross \vec N| = |\vec T| \cdot |\vec N| \sin \theta = 1.
\]
\end{remark}

\begin{example}[Example 2]
Find the binormal vector to the spiral helix $\vec r(t) =  \vector{2\cos(3t), 2\sin(3t), 5t}$ at the point $(-2, 0, 5\pi)$.\\
In example 1, we found that
\[
\vec T(t) = \frac{1}{\sqrt{61}} \vector{-6 \sin(3t), 6\cos(3t), 5} \quad \text{and} \quad \vec N(t) = \vector{-\cos(3t), -\sin(3t), 0}
\]

Hence
\begin{align*}
\vec B(t) &= \vec T(t) \cross \vec N(t)\\
          &= \frac{1}{\sqrt{61}} \vector{-6 \sin(3t), 6\cos(3t), 5} \cross \vector{-\cos(3t), -\sin(3t), 0}\\
          &= \frac{1}{\sqrt{61}} \vector{5 \sin(3t), -5 \cos(3t), 6} \quad \text{(verify)}
\end{align*}
It is a simple matter to check that 
\[
\vec T \dotp \vec B = 0, \quad \vec N \dotp \vec B = 0, \quad \text{and} \quad |\vec B| = 1
\]
for all $t$.\\
Finally, at the point $(-2, 0, 5\pi)$ we have $t = \pi$ and
\[
\vec B(\pi) = \vector{0, \frac{5}{\sqrt {61}}, \frac{6}{\sqrt{61}}}
\]
\end{example}

\begin{problem}(Problem 2)
Find the binormal vector to the spiral helix $\vec r(t) = \vector{4t, 5 \sin(2t),5\cos(2t)}$ at the point $(\pi, 5, 0)$\\
\[
\vec B(\pi/4) = \vector{\answer{-\frac{10}{\sqrt{116}}}, \answer{0}, \answer{-\frac{4}{\sqrt{116}}}}
\]


\end{problem}


The three vectors $\vec T(t), \vec N(t)$ and $\vec B(t)$ form what is called a TNB frame at each point on the curve $\vec r(t)$.

\section{The Osculating Plane and the Circle of Curvature}

Associated with each point along a space curve is a plane called the osculating plane.  This is the plane that best fits the space curve at that point.  \
For example, if the space curve happens to lie entirely in one plane, then this plane will be the osculating plane for each point on the curve.

\begin{definition}[Osculating Plane]
The osculating plane at any point on the curve $\vec r(t)$ is the plane determined by the vectors $\vec T(t)$ and $\vec N(t)$.
\end{definition}

\begin{remark}
Since the osculating plane is determined by the vectors $\vec T(t)$ and $\vec N(t)$, the normal vector to the osculating plane is $\vec B(t)$.
\end{remark}

\begin{example}[Example 3]
Find an equation for the osculating plane to the spiral helix $\vec r(t) =  \vector{2\cos(3t), 2\sin(3t), 5t}$ at the point $(-2, 0, 5\pi)$.\\
To find an equation of the osculating plane, we need the normal vector to the plane at the point $(2, 0, 5\pi)$.
According to the remark following the definition above, the normal vector to the osculating plane is the binormal vector found in example 2:
\[
\vec B(\pi) =   \vector{0, \frac{5}{\sqrt {61}}, \frac{6}{\sqrt{61}}}
\]
Hence the equation of the osculating plane is
\[
0x + \frac{5}{\sqrt {61}}y +\frac{6}{\sqrt{61}}z = d
\]
where $d$ is determined from the given point (which is in the plane).
The final answer is thus
\[
\frac{5}{\sqrt {61}}y +\frac{6}{\sqrt{61}}z = \frac{30\pi}{\sqrt{61}} \quad \text{or} \quad 5y+6z = 30\pi
\]
\end{example}

\begin{problem}(Problem 3)
Find an equation for the osculating plane to the spiral helix $\vec r(t) = \vector{4t, 5 \sin(2t),5\cos(2t)}$ at the point $(\pi, 5, 0)$.\\
\[
\answer{5x + 2z } = 5\pi
\]
\end{problem}

\begin{definition}[Circle of Curvature]
The circle of curvature (or osculating circle) at a point on the space curve $\vec r(t)$ is the circle in 
the osculating plane that has the same tangent vector and curvature as $\vec r(t)$.  
Further, the circle lies on the same side of the curve as the normal vector $\vec N$.
\end{definition}

\begin{example}[Example 4]
Find the center and radius of the circle of curvature of the spiral helix $\vec r(t) =  \vector{2\cos(3t), 2\sin(3t), 5t}$ at the point $(-2, 0, 5\pi)$.\\

Note that the tangent line to a point on a circle is perpendicular to its radius at that point. 
Since the circle of curvature lies in the plane created by the unit tangent and unit normal vectors (the osculating plane),
its center must lie on the line through the point of tangency in the direction of the unit normal vector, $\vec N$.\\

In example 3 of section 2.5, we found the curvature of the spiral helix to be 
\[
\kappa = \frac{18}{61}
\]
Since the curvature of a circle is the reciprocal of its radius, the circle of curvature has a radius of $R = \frac{61}{18}$.\\

In example 1, we determined that the unit normal to the helix at the point $(-2, 0, 5\pi)$ is
\[
\vec N = \vector{1, 0, 0}
\]
The center of the circle of curvature is the terminal point of the position vector 
\[
\vector{-2, 0, 5\pi} + \frac{61}{18}\vec N = \vector{-2, 0, 5\pi} + \vector{\frac{61}{18}, 0, 0} = \vector{\frac{25}{18} , 0, 5\pi}
\]
Thus the circle of curvature has center $(25/18, 0, 5\pi)$ and radius $61/18$ and goes through the point $(-2, 0, 5\pi)$.
\end{example}

\begin{problem}(Problem 4)
Find the center and radius of the circle of curvature of the spiral helix $\vec r(t) = \vector{4t, 5 \sin(2t),5\cos(2t)}$ at the point $(\pi, 5, 0)$.\\

The center is $\left(\answer{\pi}, \answer{-4/5}, \answer{0}\right)$\\
The radius is $R = \answer{29/5}$
\end{problem}


\section{Motion in Space}

We now use TNB frames to as an aid to describing the velocity and acceleration of the path of a particle traveling in space.

\begin{definition}[Velocity]
The velocity vector of a particle traveling along the space curve $\vec r(t)$ is given by
\[
\vec v(t) = \vec r\,'(t)
\]
The \textbf{speed} of the particle is the scalar quantity given by the magnitude of the velocity.
\end{definition}

\begin{remark}
We have the following relationship between speed, velocity and the unit tangent vector,
\[
\vec v(t) = |\vec v(t)| \vec T(t)
\]
That is, the velocity vector is speed times the unit tangent vector.
\end{remark}

\begin{definition}[Acceleration]
The acceleration vector of a particle traveling along the space curve $\vec r(t)$ is given by
\[
\vec a(t) = \vec v\,'(t) = \vec r\,''(t)
\]
\end{definition}


\begin{proposition}[Acceleration in terms of $\vec T$ and $\vec N$]
The acceleration vector $\vec a(t)$ associated with the space curve $\vec r(t)$ can be decomposed into a tangential component and a normal component as follows:
\[
\vec a(t) = |\vec v(t)|' \vec T(t) + \kappa |\vec v(t)|^2 \vec N(t)
\]
where $\vec T$ is the unit tangent vector, $\vec N$ is the unit normal vector and $|\vec v(t)|'$ is the rate of change of speed.
\end{proposition}
\begin{proof}
Using the fact in the remark above, we can write the acceleration as follows:
\[
\vec a(t) = \frac{d}{dt} \vec v(t) =  \frac{d}{dt} |\vec v(t)| \vec T(t)
\]
By the product rule we have
\[
\frac{d}{dt} |\vec v(t)| \vec T(t) = |\vec v(t)|' \vec T(t) + |\vec v(t)| \vec T\,'(t)
\]
By the definition of $\vec N$ as $\vec T\,' / |\vec T\,'|$, we have
\[
\vec a(t) = |\vec v(t)|' \vec T(t) + |\vec v(t)|\cdot |T\,'(t)| \vec N(t)
\]
Finally, the second form of curvature says
\[
\kappa = \frac{|\vec T\,'(t)|}{|\vec r\,'(t)|} = \frac{|\vec T\,'(t)|}{|\vec v(t)|}
\]
so that the coefficient of $N$ can be rewritten as
\[
|\vec v(t)|\cdot |T\,'(t)| = |\vec v(t)|\cdot |\vec v(t)| \kappa = |\vec v(t)|^2 \kappa
\]
Thus,
\[
\vec a(t) = |\vec v(t)|' \vec T(t) + \kappa |\vec v(t)|^2 \vec N(t)
\]
as claimed.
\end{proof}


\begin{example}[Example 5]
Consider a particle moving along the spiral helix $\vec r(t) =  \vector{2\cos(3t), 2\sin(3t), 5t}$.
Find the velocity and acceleration vectors and decompose the acceleration vector into its tangential and normal components.\\
The velocity vector is
\[
\vec v(t) = \vec r\,'(t) = \vector{-6\sin(3t), 6\cos(3t), 5}
\]
The acceleration vector is
\[
\vec a(t) = \vec v\,'(t) = \vector{-18\cos(3t), -18\sin(3t), 0}
\]
To decompose the acceleration into its tangential and normal components, we need the magnitude of the velocity vector (i.e. the speed):
\[
|\vec v(t)| = \sqrt{[-6\sin(3t)]^2 + [6\cos(3t)]^2 + 5^2} = \sqrt{36 + 25} = \sqrt{61}
\]
Hence the particle is moving with constant speed and $|\vec v(t)|'$ = 0.
Recall from section 2.5, example 3 that the curvature of this spiral helix is
\[
\kappa = \frac{18}{61}
\]
We can now decompose $\vec a(t)$ into its tangential and normal components:
\begin{align*}
\vec a(t) &= |\vec v(t)|' \vec T(t) + \kappa |\vec v(t)|^2 \vec N(t) \\
          &= 0\cdot \vec T(t) + \frac{18}{61} \left(\sqrt{61}\right)^2 \vec N(t) \\
          &= 0 \cdot \vec T(t) + 18\cdot \vec N(t)\\
\end{align*}

\end{example}

\begin{problem}(Problem 5)
Consider a particle moving along the spiral helix $\vec r(t) =  \vector{4t, 5 \sin(2t),5\cos(2t)}$.
Find the velocity and acceleration vectors and decompose the acceleration vector into its tangential and normal components.\\
\[
\text{velocity,} \; \vec v(t) = \vector{\answer{4},\answer{10\cos(2t)},\answer{-10\sin(2t)}}
\]
\[
\text{speed,} \;  |\vec v(t)| = \answer{\sqrt{116}}
\]
\[
\text{acceleration,} \;  \vec a(t) = \answer{0} \vec T(t) + \answer{20} \vec N(t)
\]

\end{problem}

\end{document}
