\documentclass[handout]{ximera}

%% You can put user macros here
%% However, you cannot make new environments



\newcommand{\ffrac}[2]{\frac{\text{\footnotesize $#1$}}{\text{\footnotesize $#2$}}}
\newcommand{\vasymptote}[2][]{
    \draw [densely dashed,#1] ({rel axis cs:0,0} -| {axis cs:#2,0}) -- ({rel axis cs:0,1} -| {axis cs:#2,0});
}


\graphicspath{{./}{firstExample/}}
\usepackage{forest}
\usepackage{amsmath}
\usepackage{amssymb}
\usepackage{array}
\usepackage[makeroom]{cancel} %% for strike outs
\usepackage{pgffor} %% required for integral for loops
\usepackage{tikz}
\usepackage{tikz-cd}
\usepackage{tkz-euclide}
\usetikzlibrary{shapes.multipart}


%\usetkzobj{all}
\tikzstyle geometryDiagrams=[ultra thick,color=blue!50!black]


\usetikzlibrary{arrows}
\tikzset{>=stealth,commutative diagrams/.cd,
  arrow style=tikz,diagrams={>=stealth}} %% cool arrow head
\tikzset{shorten <>/.style={ shorten >=#1, shorten <=#1 } } %% allows shorter vectors

\usetikzlibrary{backgrounds} %% for boxes around graphs
\usetikzlibrary{shapes,positioning}  %% Clouds and stars
\usetikzlibrary{matrix} %% for matrix
\usepgfplotslibrary{polar} %% for polar plots
\usepgfplotslibrary{fillbetween} %% to shade area between curves in TikZ



%\usepackage[width=4.375in, height=7.0in, top=1.0in, papersize={5.5in,8.5in}]{geometry}
%\usepackage[pdftex]{graphicx}
%\usepackage{tipa}
%\usepackage{txfonts}
%\usepackage{textcomp}
%\usepackage{amsthm}
%\usepackage{xy}
%\usepackage{fancyhdr}
%\usepackage{xcolor}
%\usepackage{mathtools} %% for pretty underbrace % Breaks Ximera
%\usepackage{multicol}



\newcommand{\RR}{\mathbb R}
\newcommand{\R}{\mathbb R}
\newcommand{\C}{\mathbb C}
\newcommand{\N}{\mathbb N}
\newcommand{\Z}{\mathbb Z}
\newcommand{\dis}{\displaystyle}
%\renewcommand{\d}{\,d\!}
\renewcommand{\d}{\mathop{}\!d}
\newcommand{\dd}[2][]{\frac{\d #1}{\d #2}}
\newcommand{\pp}[2][]{\frac{\partial #1}{\partial #2}}
\renewcommand{\l}{\ell}
\newcommand{\ddx}{\frac{d}{\d x}}

\newcommand{\zeroOverZero}{\ensuremath{\boldsymbol{\tfrac{0}{0}}}}
\newcommand{\inftyOverInfty}{\ensuremath{\boldsymbol{\tfrac{\infty}{\infty}}}}
\newcommand{\zeroOverInfty}{\ensuremath{\boldsymbol{\tfrac{0}{\infty}}}}
\newcommand{\zeroTimesInfty}{\ensuremath{\small\boldsymbol{0\cdot \infty}}}
\newcommand{\inftyMinusInfty}{\ensuremath{\small\boldsymbol{\infty - \infty}}}
\newcommand{\oneToInfty}{\ensuremath{\boldsymbol{1^\infty}}}
\newcommand{\zeroToZero}{\ensuremath{\boldsymbol{0^0}}}
\newcommand{\inftyToZero}{\ensuremath{\boldsymbol{\infty^0}}}


\newcommand{\numOverZero}{\ensuremath{\boldsymbol{\tfrac{\#}{0}}}}
\newcommand{\dfn}{\textbf}
%\newcommand{\unit}{\,\mathrm}
\newcommand{\unit}{\mathop{}\!\mathrm}
%\newcommand{\eval}[1]{\bigg[ #1 \bigg]}
\newcommand{\eval}[1]{ #1 \bigg|}
\newcommand{\seq}[1]{\left( #1 \right)}
\renewcommand{\epsilon}{\varepsilon}
\renewcommand{\iff}{\Leftrightarrow}

\DeclareMathOperator{\arccot}{arccot}
\DeclareMathOperator{\arcsec}{arcsec}
\DeclareMathOperator{\arccsc}{arccsc}
\DeclareMathOperator{\si}{Si}
\DeclareMathOperator{\proj}{proj}
\DeclareMathOperator{\scal}{scal}
\DeclareMathOperator{\cis}{cis}
\DeclareMathOperator{\Arg}{Arg}
%\DeclareMathOperator{\arg}{arg}
\DeclareMathOperator{\Rep}{Re}
\DeclareMathOperator{\Imp}{Im}
\DeclareMathOperator{\sech}{sech}
\DeclareMathOperator{\csch}{csch}
\DeclareMathOperator{\Log}{Log}

\newcommand{\tightoverset}[2]{% for arrow vec
  \mathop{#2}\limits^{\vbox to -.5ex{\kern-0.75ex\hbox{$#1$}\vss}}}
\newcommand{\arrowvec}{\overrightarrow}
\renewcommand{\vec}{\mathbf}
\newcommand{\veci}{{\boldsymbol{\hat{\imath}}}}
\newcommand{\vecj}{{\boldsymbol{\hat{\jmath}}}}
\newcommand{\veck}{{\boldsymbol{\hat{k}}}}
\newcommand{\vecl}{\boldsymbol{\l}}
\newcommand{\utan}{\vec{\hat{t}}}
\newcommand{\unormal}{\vec{\hat{n}}}
\newcommand{\ubinormal}{\vec{\hat{b}}}

\newcommand{\dotp}{\bullet}
\newcommand{\cross}{\boldsymbol\times}
\newcommand{\grad}{\boldsymbol\nabla}
\newcommand{\divergence}{\grad\dotp}
\newcommand{\curl}{\grad\cross}
%% Simple horiz vectors
\renewcommand{\vector}[1]{\left\langle #1\right\rangle}


\outcome{Use Lagrange multipliers to find maxima and minima.}

\title{3.8 Lagrange Multipliers}



\begin{document}

\begin{abstract}
In this section we use Lagrange multipliers to find absolute maxima and minima.
\end{abstract}

\maketitle

We now turn to the problem of finding the absolute maximum and absolute minimum values
of a function of several variables on a closed and bounded subset of its domain.
To say that a set is bounded means that it is a subset of some disk.  And to say that it is closed means that it contains 
all of its boundary points. 
Any disk around a boundary point contains points that are in the set as well as points that are not in the set.
Points that are in the set $S$ which are not boundary points are called interior points.

The unit circle is a closed and bounded subset of $\R^2$.  Every point on the circle is a boundary point.
The unit disk is a bounded set, whose boundary is the unit circle. Every point within the unit disk is an interior point.
The unit disk can be described by the inequality $x^2 + y^2 < 1$.
The closed unit disk is the unit disk together with the unit circle and it can be described by $x^2 + y^2 \leq 1$.

Let $S$ be a closed and bounded set in $\R^2$ and $f(x,y)$ a function defined on $S$. 
We wish to find the absolute maximum and absolute minimum values of $f(x,y)$ on $S$.

\begin{theorem}[Extreme Value Theorem]
Let $f(x,y)$ be a continuous function defined on a closed and bounded set $S$.
Then $f(x,y)$ has an absolute maximum and absolute minimum value on $S$.
\end{theorem}

\begin{theorem}
The absolute maximum and absolute minimum values of a continuous function $f(x,y)$ on a closed and bounded set $S$ 
occur at either a critical point or a boundary point.
\end{theorem}

In the last section, we learned how to find critical points. 
In this section, the focus will be on locating boundary points where the maximum and minimum values can occur.

In practice, the boundary of the set $S$ will be a curve in $\R^2$ (for the unit disk, it was the unit circle).
We will assume that the boundary curve is smooth and that it can be parameterized by $x = x(t), y = y(t)$.
Then, along the boundary, the function will have the form, $f(x, y) = f(x(t), y(t))$, so that $f$ is a function of $t$.
At a point on the boundary where $f$ has either a max or a min, $f$ will have a critical number at that point (with respect to the variable $t$).
This means that $f\,'(t) = 0$ at this point. From the chain rule
\[
0 = f\,'(t) = f_x(x, y) x'(t) + f_y(x,y) y'(t) = \grad f(x,y) \cdot \vector{x'(t), y'(t)}
\]
This means that at the max or min, the gradient of $f$ is orthogonal to the boundary curve.
On the other hand, this boundary curve can be written in the form $g(x,y) = 0$. For example, the unit circle can be written as $g(x,y) = x^2 + y^2 - 1 = 0$,
so that we can think of the boundary curve as the level curve of the surface $z = g(x,y)$. We have already seen, that the gradient of $g$ is orthogonal
to its level curves.  Therefore, $\grad g(x,y)$ is also orthogonal to the boundary curve of the region $S$.
Since both $\grad f(x,y)$ and $\grad g(x,y)$ are orthogonal to the boundary curve at the max and min, they must be parallel at these points.
That means that at the maximum and minimum values of $f(x,y)$ along the boundary curve $g(x,y) = 0$ of the set $S$, 
\[
\grad f(x,y) = \lambda \grad g(x,y)
\]
where $\lambda$ is a scalar.

\begin{theorem}[Lagrange Multipliers]
The absolute maximum and minimum values of $f(x,y)$ on the bounded curve $g(x,y) = 0$ occur at 
points where 
\[
\grad f(x,y) = \lambda \grad g(x,y)
\]
for some scalar $\lambda$.
\end{theorem}

\begin{example}[Example 1]
Find the absolute maximum and absolute minimum values of $f(x,y) = 8x - 6y$ subject to $x^2 + y^2 = 1$\\
We calculate the gradients of $f(x,y) = 8x - 6y$ and $g(x,y) = x^2 + y^2$
\[
\grad f(x,y) = \vector{8, -6} \quad \text{and} \quad \grad g(x,y) = \vector{2x, 2y}
\]
According to the theory of Lagrange multipliers, the absolute maximum and absolute minimum of $f(x,y) = 8x - 6y$
along the curve $g(x,y) = x^2 + y^2 = 1$ occur at points where $\grad f(x,y)$ and $\grad g(x,y)$ are parallel.  In other words at points where
\[
\grad f(x,y) = \lambda \grad g(x,y)
\]
for some constant $\lambda$.
This equation gives the system
\[
8 = 2\lambda x \quad \text{and} -6 = 2\lambda y
\]
Solving both of these equations for $\lambda$ gives
\[
\lambda = \frac{8}{2x} = \frac{4}{x} \quad \text{and} \quad \lambda = -\frac{6}{2y} = -\frac{3}{y}
\]
Setting these expressions for $\lambda$ equal to each other gives
\[
\frac{4}{x} = -\frac{3}{y} \quad \text{so} \quad y = -\frac34 x
\]
Now, we substitute this expression for $y$ into the constraint curve $x^2 + y^2 = 1$ to obtain
\[
x^2 + \left(-\frac34 x\right)^2 = 1
\]
and solving for $x$, we have
\[
x^2 = \frac{16}{25} \quad \text{so} \quad x = \pm \frac45
\]
The corresponding $y$-coordinates on the unit circle come from 
\[
y = -\frac34 x = \mp \frac45 = \mp \frac35
\]
We now have two points on the unit circle where the absolute maximum and absolute minimum values of $f(x,y) - 8x - 6y$ can occur.
They are
\[
\left(\frac45, -\frac35\right) \quad \text{and} \quad \left(-\frac45, \frac35\right)
\]
Plugging these points into $f$ gives
\[
f\left(\frac45, -\frac35\right) = 10 \quad \text{and} f\left(-\frac45, \frac35\right) = -10
\]
Hence, the absolute maximum value of $8x -6y$ on the unit circle is $10$ and the absolute minimum is $-10$.
\end{example}

\begin{problem}(Problem 1)
Find the absolute maximum and absolute minimum values of $f(x,y) = 5x -12y$ subject to $x^2 + y^2 = 1$\\
The absolute maximum is $\answer{13}$\\
The absolute minimum is $\answer{-13}$
\end{problem}

\begin{example}[Example 2]
Find the absolute maximum and absolute minimum values of $f(x,y) = 2x^2 + 3y^2 - 4x - 5$ subject to $x^2 + y^2 \leq 16$.\\
The critical points of $f$ are at
\[
\grad f(x,y) = \vector{4x -4, 6y} = \vector{0,0}
\]
 which gives the lone critical point, $(1,0)$ which is inside the disk $x^2 + y^2 \leq 16$.\\
 Extrema on the boundary curve $x^2 + y^2 = 16$ will occur when
 \[
 \grad f(x,y) = \lambda \grad g(x,y) = \lambda \vector{2x, 2y}
 \]
 yielding the system
 \[
 4x - 4 = 2\lambda x, \; 6y = 2\lambda y
 \]
 The second equation is satisfied if $y = 0$ or $\lambda = 3$.  In the case where $y = 0$ this means $x^2 = 16$, so $x = \pm 4$.
 In the case where $\lambda = 3$, the first equation becomes
 \[
 4x - 4 = 6x 
 \]
 which implies that $x = -2$.  Then, on the boundary curve we have $(-2)^2 + y^2 = 16$ so that $y = \pm \sqrt{12}$.
 We now have a complete list of candidates for the location of the absolute extremes:
 \[
 (1,0), \; (-4, 0), \; (4,0), \; (-2, -2\sqrt3), \;\text{and} \; (-2, 2\sqrt 3)
 \]
 Plugging these into $f(x,y) = 2x^2 + 3y^2 - 4x - 5$ gives
 \begin{align*}
 f(1,0) &= 2-4-5=-7\\
 f(-4, 0) &= 32 +16 -5 = 43\\
 f(4, 0) &= 32 -16 -5 = 41\\
 f(-2, -2\sqrt3) &= 8 + 36 +8-5 = 47\\
 f(-2, 2\sqrt3) &= 8 + 36 +8-5 = 47
 \end{align*}
 Hence, the absolute maximum is $47$ and it occurs at two different boundary points, $(-2, \pm 2\sqrt 3)$,
 and the absolute minimum is $-7$ occurring at the interior critical point $(1,0)$.
 
 
\end{example}

\begin{problem}(Problem 2)
Find the absolute maximum and absolute minimum values of $f(x,y) = x^2 + 2y^2 - 8y - 1$ subject to $x^2 + y^2 = 25$\\
The absolute maximum is $\answer{89}$\\
The absolute minimum is $\answer{-9}$
\end{problem}

\begin{example}[Example 3]
Find the absolute maximum and absolute minimum values of $f(x,y) = xy$ subject to $x^2 + 4y^2 \leq 1$\\
The critical points of $f$ are at
\[
\grad f(x,y) = \vector{y, x} = \vector{0,0}
\]
 which gives the lone critical point, $(0,0)$ which is inside the ellipse $x^2 + 4y^2 \leq 1$.\\
 Extrema on the boundary curve $x^2 + 4y^2 = 1$ will occur when
 \[
 \grad f(x,y) = \lambda \grad g(x,y) = \lambda \vector{2x, 8y}
 \]
 yielding the system
 \[
 y = 2\lambda x, \; x = 8\lambda y
 \]
 Solving these equations for $\lambda$ gives
 \[
 \lambda = \frac{y}{2x} \quad \text{and} \quad \lambda = \frac{x}{8y}
 \]
 Setting these equal yields
 \[
 \frac{y}{2x} = \frac{x}{8y} \quad \text{so} \quad x^2 = 4y^2
 \]
Substituting this into the constrain curve gives
\[
2x^2 = 1 \quad \text{so} \quad x = \pm \frac{1}{\sqrt 2}
\]
The associated values of $y$ are 
\[
y = \pm \frac{1}{\sqrt 8} = \pm \frac{1}{2\sqrt 2}
\]
Thus, there are four points on the ellipse $x^2 + 4y^2$ where the absolute max and absolute min can occur.
We now have a complete list of candidates for the location of the absolute extremes:
 \[
 (0,0), \; \left(-\frac{1}{\sqrt 2},-\frac{1}{2\sqrt 2} \right), \; \left(-\frac{1}{\sqrt 2},\frac{1}{2\sqrt 2} \right),
  \; \left(\frac{1}{\sqrt 2},-\frac{1}{2\sqrt 2} \right), \;\text{and} \; \left(\frac{1}{\sqrt 2},\frac{1}{2\sqrt 2} \right)
 \]
 Plugging these into $f(x,y) = xy$ gives
 \begin{align*}
 f(0,0) &= 0\\
 f\left(-\frac{1}{\sqrt 2},-\frac{1}{2\sqrt 2} \right) &= \frac14\\
 f\left(-\frac{1}{\sqrt 2},\frac{1}{2\sqrt 2} \right) &= -\frac14\\
 f\left(\frac{1}{\sqrt 2},-\frac{1}{2\sqrt 2} \right) &= -\frac14\\
 f\left(\frac{1}{\sqrt 2},\frac{1}{2\sqrt 2} \right) &= \frac14
 \end{align*}
 Hence, the absolute maximum is $-1/4$ and it occurs at two of the 
 boundary points, $\left(\pm \frac{1}{\sqrt 2}, \pm \frac{1}{2\sqrt 2}\right)$,
 and the absolute minimum is $-1/4$ occurring at the other two 
 boundary points, $\left(\pm \frac{1}{\sqrt 2}, \mp \frac{1}{2\sqrt 2}\right)$.
 
 
\end{example}

\begin{problem}(Problem 3)
Find the absolute maximum and absolute minimum values of $f(x,y) = x^{1/3}y^{2/3}$ subject to $x^2 + y^2 \leq 12$\\
The absolute maximum is $\answer{2^{4/3}}$\\
The absolute minimum is $\answer{-2^{4/3}}$
\end{problem}

\begin{theorem}[Lagrange Multipliers (for 3 variables)]
The absolute maximum and minimum values of $f(x,y, z)$ on the bounded surface $g(x,y) = 0$ occur at 
points where 
\[
\grad f(x,y, z) = \lambda \grad g(x,y, z)
\]
for some scalar $\lambda$.
\end{theorem}

\begin{example}[Example 4]
Find the absolute maximum and absolute minimum values of $f(x,y, z) = x^2 + y^2 + z^2$ subject to $x^4 + y^4 + z^4  = 5$.\\
\end{example}

\begin{problem}(Problem 4a)
Find the absolute maximum and absolute minimum values of $f(x,y, z) = 2x  -4y  +4z$ subject to $x^2 + y^2 + z^2 \leq 1$.\\
The absolute maximum is $\answer{6}$\\
The absolute minimum is $\answer{-6}$
\end{problem}

\begin{problem}(Problem 4b)
Find the absolute maximum and absolute minimum values of $f(x,y, z) = 4x + 8y - 16z$ subject to $4x^2 + y^2 + z^2 = 4$.\\
The absolute maximum is $\answer{22/3}$\\
The absolute minimum is $\answer{-22/3}$
\end{problem}


\end{document}
