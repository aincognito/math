\documentclass[handout]{ximera}

%% You can put user macros here
%% However, you cannot make new environments



\newcommand{\ffrac}[2]{\frac{\text{\footnotesize $#1$}}{\text{\footnotesize $#2$}}}
\newcommand{\vasymptote}[2][]{
    \draw [densely dashed,#1] ({rel axis cs:0,0} -| {axis cs:#2,0}) -- ({rel axis cs:0,1} -| {axis cs:#2,0});
}


\graphicspath{{./}{firstExample/}}
\usepackage{forest}
\usepackage{amsmath}
\usepackage{amssymb}
\usepackage{array}
\usepackage[makeroom]{cancel} %% for strike outs
\usepackage{pgffor} %% required for integral for loops
\usepackage{tikz}
\usepackage{tikz-cd}
\usepackage{tkz-euclide}
\usetikzlibrary{shapes.multipart}


%\usetkzobj{all}
\tikzstyle geometryDiagrams=[ultra thick,color=blue!50!black]


\usetikzlibrary{arrows}
\tikzset{>=stealth,commutative diagrams/.cd,
  arrow style=tikz,diagrams={>=stealth}} %% cool arrow head
\tikzset{shorten <>/.style={ shorten >=#1, shorten <=#1 } } %% allows shorter vectors

\usetikzlibrary{backgrounds} %% for boxes around graphs
\usetikzlibrary{shapes,positioning}  %% Clouds and stars
\usetikzlibrary{matrix} %% for matrix
\usepgfplotslibrary{polar} %% for polar plots
\usepgfplotslibrary{fillbetween} %% to shade area between curves in TikZ



%\usepackage[width=4.375in, height=7.0in, top=1.0in, papersize={5.5in,8.5in}]{geometry}
%\usepackage[pdftex]{graphicx}
%\usepackage{tipa}
%\usepackage{txfonts}
%\usepackage{textcomp}
%\usepackage{amsthm}
%\usepackage{xy}
%\usepackage{fancyhdr}
%\usepackage{xcolor}
%\usepackage{mathtools} %% for pretty underbrace % Breaks Ximera
%\usepackage{multicol}



\newcommand{\RR}{\mathbb R}
\newcommand{\R}{\mathbb R}
\newcommand{\C}{\mathbb C}
\newcommand{\N}{\mathbb N}
\newcommand{\Z}{\mathbb Z}
\newcommand{\dis}{\displaystyle}
%\renewcommand{\d}{\,d\!}
\renewcommand{\d}{\mathop{}\!d}
\newcommand{\dd}[2][]{\frac{\d #1}{\d #2}}
\newcommand{\pp}[2][]{\frac{\partial #1}{\partial #2}}
\renewcommand{\l}{\ell}
\newcommand{\ddx}{\frac{d}{\d x}}

\newcommand{\zeroOverZero}{\ensuremath{\boldsymbol{\tfrac{0}{0}}}}
\newcommand{\inftyOverInfty}{\ensuremath{\boldsymbol{\tfrac{\infty}{\infty}}}}
\newcommand{\zeroOverInfty}{\ensuremath{\boldsymbol{\tfrac{0}{\infty}}}}
\newcommand{\zeroTimesInfty}{\ensuremath{\small\boldsymbol{0\cdot \infty}}}
\newcommand{\inftyMinusInfty}{\ensuremath{\small\boldsymbol{\infty - \infty}}}
\newcommand{\oneToInfty}{\ensuremath{\boldsymbol{1^\infty}}}
\newcommand{\zeroToZero}{\ensuremath{\boldsymbol{0^0}}}
\newcommand{\inftyToZero}{\ensuremath{\boldsymbol{\infty^0}}}


\newcommand{\numOverZero}{\ensuremath{\boldsymbol{\tfrac{\#}{0}}}}
\newcommand{\dfn}{\textbf}
%\newcommand{\unit}{\,\mathrm}
\newcommand{\unit}{\mathop{}\!\mathrm}
%\newcommand{\eval}[1]{\bigg[ #1 \bigg]}
\newcommand{\eval}[1]{ #1 \bigg|}
\newcommand{\seq}[1]{\left( #1 \right)}
\renewcommand{\epsilon}{\varepsilon}
\renewcommand{\iff}{\Leftrightarrow}

\DeclareMathOperator{\arccot}{arccot}
\DeclareMathOperator{\arcsec}{arcsec}
\DeclareMathOperator{\arccsc}{arccsc}
\DeclareMathOperator{\si}{Si}
\DeclareMathOperator{\proj}{proj}
\DeclareMathOperator{\scal}{scal}
\DeclareMathOperator{\cis}{cis}
\DeclareMathOperator{\Arg}{Arg}
%\DeclareMathOperator{\arg}{arg}
\DeclareMathOperator{\Rep}{Re}
\DeclareMathOperator{\Imp}{Im}
\DeclareMathOperator{\sech}{sech}
\DeclareMathOperator{\csch}{csch}
\DeclareMathOperator{\Log}{Log}

\newcommand{\tightoverset}[2]{% for arrow vec
  \mathop{#2}\limits^{\vbox to -.5ex{\kern-0.75ex\hbox{$#1$}\vss}}}
\newcommand{\arrowvec}{\overrightarrow}
\renewcommand{\vec}{\mathbf}
\newcommand{\veci}{{\boldsymbol{\hat{\imath}}}}
\newcommand{\vecj}{{\boldsymbol{\hat{\jmath}}}}
\newcommand{\veck}{{\boldsymbol{\hat{k}}}}
\newcommand{\vecl}{\boldsymbol{\l}}
\newcommand{\utan}{\vec{\hat{t}}}
\newcommand{\unormal}{\vec{\hat{n}}}
\newcommand{\ubinormal}{\vec{\hat{b}}}

\newcommand{\dotp}{\bullet}
\newcommand{\cross}{\boldsymbol\times}
\newcommand{\grad}{\boldsymbol\nabla}
\newcommand{\divergence}{\grad\dotp}
\newcommand{\curl}{\grad\cross}
%% Simple horiz vectors
\renewcommand{\vector}[1]{\left\langle #1\right\rangle}


\outcome{Define limits, derivatives and integrals of vector-valued functions.}

\title{2.2 Calculus of Space Curves}



\begin{document}

\begin{abstract}
In this section we define limits, derivatives and integrals of vector-valued functions.
\end{abstract}

\maketitle

\section{Limits}
Limits of vector-valued functions are computed componentwise. 
\begin{definition}[Limit of a Vector-Valued Function]
If $\vec r(t) = \vector{x(t), y(t), z(t)}$ then
\[
\lim_{t \to c} \vec r(t) = \vector{\lim_{t \to c}x(t), \lim_{t \to c}y(t),\lim_{t \to c} z(t)}
\]
provided the limit of each component exists.
\end{definition}

\begin{example}[Example 1]
Compute the indicated limit of the vector valued function:
\[
\lim_{t \to 0} \vector{\frac{\sin t}{t}, \frac{\ln t}{t}, \frac{e^t - t -1}{t^2}}
\]
Recall \textbf{L'Hopital's Rule}:
\[
\text{If}\; \lim_{x \to c} \frac{f(x)}{g(x)} = \frac00 \; \text{or} \; \frac{\infty}{\infty} 
\]
\[
\text{then} \; \lim_{x \to c} \frac{f(x)}{g(x)} = \lim_{x \to c} \frac{f'(x)}{g'(x)}
\]
as long as $f$ and $g$ are differentiable in an open interval containing $x = c$ (but not necessarily at $x = c$ itself)
and $g'(c) \neq 0$. We can apply this to the limit in each of the components:
\[
\lim_{t \to 0} \frac{\sin t}{t} = \frac00 = \lim_{t\to 0} \cos t = \cos 0  = 1
\]
\[
\lim_{t \to 0} \frac{\ln(1+t)}{t} = \frac00 = \lim_{t\to 0} \frac{1}{1+t} = 1
\]
\[
\lim_{t \to 0} \frac{e^t - t - 1}{t^2} = \frac00 = \lim_{t\to 0} \frac{e^t - 1}{2t} = \frac00 = \frac{e^t}{2}  = \frac12
\]
Hence the limit of the vector-valued function is the vector $\vector{1, 1, 1/2}$.
\end{example}

\begin{problem}(Problem 1)
Compute the limit if the vector-valued function:
\[
\lim_{t \to 1} \vector{\frac{t^5 - 1}{t^3 - 1}, \frac{\tan(t-1)}{t-1}, \frac{1 - \sqrt t}{t-1}}
\]
\end{problem}

\section{Continuity}

Continuity of a vector-valued function is defined in terms of its components.

\begin{definition}[Continuity of Vector-Valued Functions]
A vector-valued function $\vec r(t) = \vector{x(t), y(t), z(t)}$ is
\textbf{continuous} at $t = c$ if
\[
\lim_{t \to c} \vec r(t) = \vec r(c)
\]
\end{definition}

\begin{example}[Example 2]
Verify that the vector-valued function $\vec r(t) = \vector{\sin t, t \cos t, e^{-t}}$ is continuous at $t = 0$.\\
Since
\[
\lim_{t \to 0} \sin t =  \sin 0 = 0
\]
\[
\lim_{t \to 0} t \cos t =  0 \cdot \cos 0 = 0
\]
\[
\lim_{t \to 0} e^{-t} =  e^0 = 1,
\]
we have
\[
\lim_{t \to 0} \vec r(t) = \vector{0, 0, 1} = \vec r(0)
\]
and hence the vector-valued function $\vec r(t)$ is continuous at $t = 0$.

\end{example}

\begin{problem}(Problem 2a)
Show that the space curve is continuous at $t = 0$:
\[
\vec r(t) = \vector{|t|,\sqrt[3] t, \sec(t)}
\]
\end{problem}

Since continuity is determined componentwise, we can take advantage of our knowledge of 
continuous functions of a single variable.

\begin{proposition}[Continuity]
If $x(t), y(t)$ and $z(t)$ are continuous at $t = c$, then the vector-valued function $\vec r(t) = \vector{x(t), y(t), z(t)}$
is continuous at $t = c$.
\end{proposition}
\begin{proof}
The proof follows directly from the definitions of continuity of vector-valued functions, continuity of functions of a single 
variable, and the componentwise computation of limits of vector-valued functions.
\end{proof}

\begin{problem}(Problem 2b)
Explain why the space curve is continuous on the interval $(0, \infty)$:
\[
\vec r(t) = \vector{ \sqrt t, \ln t, t^{-1}}
\]
\end{problem}

\section{Differentiation}
The derivative of a vector valued function is defined in a manner analogous to the derivative of a function of a single variable.

\begin{definition}[Derivative of a Vector-Valued Function]
Let $\vec r(t)$ be a vector-valued function. Its \textbf{derivative} is defined by
\[
\frac{d\vec r}{dt} = \vec r\, '(t) = \lim_{h \to 0} \frac{\vec r(t+h) - \vec r(t)}{h}
\]
provided the limit exists.
\end{definition}

\begin{remark}
The vector 
\[
\frac{\vec r(a+h) - \vec r(a)}{h}
\]
is a scalar multiple of the vector between the points $\vec r(a+h)$ and $\vec r(a)$ on the space 
curve defined by the vector-valued function $\vec r(t)$. 
Hence, we can see that the vector $\vec r\,'(a)$, if it exists, 
will be a vector that is tangent to the curve $\vec r(t)$ at $t = a$. 
\end{remark}
Below, we see the space curve $\vec r(t)$ with the vectors $\vec r(a+h) - \vec r(a)$ and $\vec r\,'(a)$.
\begin{image}
\begin{tikzpicture}
\draw[thick] (0,1) to [out =60, in = 180] (4, 4) to [out = 0, in = 135] (6, 3) to [out = -45, in = 90] (7, 2) node[right]{$\vec r(t)$};
\draw[blue!70!white, thick, ->] (0,0) node[below, left]{$O$} -- (4, 4) node[midway, above left]{$\vec r(a)$};
\draw[blue!70!white, thick, ->] (0,0) -- (6, 3) node[midway, below right]{$\vec r(a+h)$};
\draw[red!70!black, thick, ->] (4,4) -- (6, 3) node[above, right]{$\vec r(a+h)- \vec r(a)$};
\draw[green!60!black, thick, ->] (4,4) -- (5.3, 4) node[midway, above]{$\vec r\,'(a)$};
%\draw[thick] (0,1) .. controls (2, 3) .. (4, 4) ..controls (6, 5) ..  (8,5) .. controls (9, 4) .. (10, 2) node[right];
\end{tikzpicture}
\end{image}


\begin{definition}[Tangent Vector and Tangent Line]
If the vector-valued function $\vec r(t)$ is differentiable at $t = a$ then the vector $\vec r\,'(a)$ is called the 
\textbf{tangent vector} to the curve determined by $\vec r(t)$ at the point corresponding to $t = a$. 
Furthermore, the equation of the \textbf{tangent line} to this curve at $ t= a$ is given by
\[
\vector{x, y, z} = \vec r(a) + t\,\vec r\,'(a)
\]
\end{definition}

\begin{example}[Example 3] 
Find the unit tangent vector to the spiral helix
\[
\vec r(t) = \vector{\cos(2t), \sin(2t), t}
\]
at $t = \pi/2$.\\
The tangent vector is given by
\[
\vec r\,'(t) = \vector{-2\sin(2t), 2\cos(2t), 1}
\]
At $t = \pi/2$ we have
\[
\vec r\,'(\pi/2) = \vector{-2\sin(\pi), 2\cos(\pi), 1} = \vector{0, -2, 1}
\]
The unit tangent vector is
\[
\frac{r\,'(\pi/2)}{|r\,'(\pi/2)|} = \frac{1}{\sqrt 5}\vector{0, -2, 1} = \vector{0, -\frac{2}{\sqrt 5}, \frac{1}{\sqrt 5}}
\]

\end{example}

\begin{problem}(Problem 3a)
Find the unit tangent vector to the space curve
\[
\vec r(t) = \vector{2t, te^t, \tan(2t)}
\]
at $t = 0$.\\
The tangent vector is $\vector{\answer{2}, \answer{1}, \answer{2}}$\\
The unit tangent vector is $\vector{\answer{2/3}, \answer{1/3}, \answer{2/3}}$
\end{problem}

\begin{problem}(Problem 3b)
Find the equation of the tangent line to the twisted cubic
\[
\vec r(t) = \vector{t, t^2, t^3}
\]
at $t = 1$.\\
The vector form of the tangent line is $\vector{x, y, z} = \vector{1,1,1} + t\vector{1, \answer{2}, \answer{3}}$\\
\end{problem}

In applications where the space curve $\vec r(t)$ represents the path of a particle, the tangent vector $\vec r\,'(t)$ is sometimes referred to as the \textbf{velocity} vector
of the particle and its magnitude, $|\vec r\,'(t)|$ is the \textbf{speed} of the particle.

\begin{example}[Example 4]
Find the speed of a particle whose position is given by $\vec r(t) = \vector{\cos(t), \sin(t), t}$ at $t = \pi$.\\
The tangent vector at $ t = \pi$ is
\[
\vec r\,'(\pi) = \vector{-\sin(\pi), \cos(\pi), 1} = \vector{0, -1, 1}
\]
and the speed of the curve a t $t = \pi$ is
\[
\text{speed} = |\vec r\,'(\pi)| =|\vector{0, -1, 1}| = \sqrt{0^2 + (-1)^2 + 1^2} = \sqrt 2
\]
\end{example}

\begin{problem}(Problem 4) 
Find the speed of the particle with position $\vec r(t) = \vector{te^t, t^2e^{2t}, t^3e^{3t}}$ at $t = 0$.\\
\[
\text{speed} = |\vec r\,'(0)| = \answer{1}
\]
\end{problem}

\section{Integration}

As expected, the integral of a vector-valued function is computed componentwise.

\begin{definition}[Integral of a Vector-Valued Function]
If $\vec r = \vector{x(t), y(t), z(t)}$ is defined and continuous on the interval $[a, b]$, then the definite integral is defined by
\[
\int_a^b \vec r(t) \, dt = \int_a^b \vector{x(t), y(t), z(t)} \, dt =  \vector{\int_a^b x(t) \, dt,\int_a^b  y(t) \, dt,\int_a^b  z(t) \, dt}
\]

and the indefinite integral is defined by

\[
\int \vec r(t) \, dt = \int\vector{x(t), y(t), z(t)} \, dt =  \vector{\int x(t) \, dt,\int  y(t) \, dt,\int  z(t) \, dt}
\]
\end{definition}

\begin{remark}
The constant of integration of a vector-valued function is a vector.
\end{remark}

\begin{example}[Example 5]
Let $\vec r(t) = \vector{te^t,t\cos(t^2),\csc^2(t)}.$ Compute the indefinite integral:
\[
\int \vec r(t) \, dt
\]
We will compute the integral of the components separately:
To integrate $te^t$, we use integration by parts with $u = t$ and $dv = e^t \, dt$:
\begin{align*}
\int te^t \, dt &= \int u \, dv = uv - \int v \, du\\
                 &= te^t - \int e^t \, dt\\
                 & = te^t - e^t + C_1\\
                 & = (t-1)e^t + C_1
\end{align*}
To integrate $t\cos(t^2)$ we use $u$-substitution with $ u = t^2$:
\begin{align*}
\int t\cos(t^2) \, dt &= \frac12 \int \cos(u) \, du \\
                 &= \frac12 \sin(u) + C_2\\
                 & = \frac12 \sin(t^2) + C_2
\end{align*}
To integrate $\csc^2(t)$, note that
\[
\frac{d}{dt} \cot(t) = -\csc^2(t)
\]
We have
\[
\int \csc^2(t) \, dt = -\cot(t) + C_3
\]
Hence, the indefinite integral of $\vec r(t)$ is
\[
\int \vec r(t) \, dt = \vector{(t-1)e^t,\frac12 \sin(t^2),-\cot(t)} + \vector{C_1, C_2, C_3}
\]

\end{example}

\begin{problem}(Problem 5a)
Let $\vec r(t) = \vector{t^2\ln(t),t\sqrt{1+t^2},\frac{1}{1+t^2}}.$ Compute the indefinite integral:
\[
\int \vec r(t) \, dt = \vector{\answer{\frac13 t^3 \ln(t) - \frac19 t^3},\answer{\frac13 (1+t^2)^3/2},\answer{\tan^{-1}(t)}} + \vector{C_1, C_2, C_3}
\]

\end{problem}

\begin{problem}(Problem 5b)
Let $\vec r(t) = \vector{\cos(2\pi t),|t|,e^{-t}}$. Compute the definite integral:
\[
\int_0^1 \vec r(t) \, dt = \vector{\answer{0}, \answer{1/2}, \answer{1 - 1/e}}
\]
\end{problem}




\end{document}
