\documentclass[handout]{ximera}

%% You can put user macros here
%% However, you cannot make new environments



\newcommand{\ffrac}[2]{\frac{\text{\footnotesize $#1$}}{\text{\footnotesize $#2$}}}
\newcommand{\vasymptote}[2][]{
    \draw [densely dashed,#1] ({rel axis cs:0,0} -| {axis cs:#2,0}) -- ({rel axis cs:0,1} -| {axis cs:#2,0});
}


\graphicspath{{./}{firstExample/}}
\usepackage{forest}
\usepackage{amsmath}
\usepackage{amssymb}
\usepackage{array}
\usepackage[makeroom]{cancel} %% for strike outs
\usepackage{pgffor} %% required for integral for loops
\usepackage{tikz}
\usepackage{tikz-cd}
\usepackage{tkz-euclide}
\usetikzlibrary{shapes.multipart}


%\usetkzobj{all}
\tikzstyle geometryDiagrams=[ultra thick,color=blue!50!black]


\usetikzlibrary{arrows}
\tikzset{>=stealth,commutative diagrams/.cd,
  arrow style=tikz,diagrams={>=stealth}} %% cool arrow head
\tikzset{shorten <>/.style={ shorten >=#1, shorten <=#1 } } %% allows shorter vectors

\usetikzlibrary{backgrounds} %% for boxes around graphs
\usetikzlibrary{shapes,positioning}  %% Clouds and stars
\usetikzlibrary{matrix} %% for matrix
\usepgfplotslibrary{polar} %% for polar plots
\usepgfplotslibrary{fillbetween} %% to shade area between curves in TikZ



%\usepackage[width=4.375in, height=7.0in, top=1.0in, papersize={5.5in,8.5in}]{geometry}
%\usepackage[pdftex]{graphicx}
%\usepackage{tipa}
%\usepackage{txfonts}
%\usepackage{textcomp}
%\usepackage{amsthm}
%\usepackage{xy}
%\usepackage{fancyhdr}
%\usepackage{xcolor}
%\usepackage{mathtools} %% for pretty underbrace % Breaks Ximera
%\usepackage{multicol}



\newcommand{\RR}{\mathbb R}
\newcommand{\R}{\mathbb R}
\newcommand{\C}{\mathbb C}
\newcommand{\N}{\mathbb N}
\newcommand{\Z}{\mathbb Z}
\newcommand{\dis}{\displaystyle}
%\renewcommand{\d}{\,d\!}
\renewcommand{\d}{\mathop{}\!d}
\newcommand{\dd}[2][]{\frac{\d #1}{\d #2}}
\newcommand{\pp}[2][]{\frac{\partial #1}{\partial #2}}
\renewcommand{\l}{\ell}
\newcommand{\ddx}{\frac{d}{\d x}}

\newcommand{\zeroOverZero}{\ensuremath{\boldsymbol{\tfrac{0}{0}}}}
\newcommand{\inftyOverInfty}{\ensuremath{\boldsymbol{\tfrac{\infty}{\infty}}}}
\newcommand{\zeroOverInfty}{\ensuremath{\boldsymbol{\tfrac{0}{\infty}}}}
\newcommand{\zeroTimesInfty}{\ensuremath{\small\boldsymbol{0\cdot \infty}}}
\newcommand{\inftyMinusInfty}{\ensuremath{\small\boldsymbol{\infty - \infty}}}
\newcommand{\oneToInfty}{\ensuremath{\boldsymbol{1^\infty}}}
\newcommand{\zeroToZero}{\ensuremath{\boldsymbol{0^0}}}
\newcommand{\inftyToZero}{\ensuremath{\boldsymbol{\infty^0}}}


\newcommand{\numOverZero}{\ensuremath{\boldsymbol{\tfrac{\#}{0}}}}
\newcommand{\dfn}{\textbf}
%\newcommand{\unit}{\,\mathrm}
\newcommand{\unit}{\mathop{}\!\mathrm}
%\newcommand{\eval}[1]{\bigg[ #1 \bigg]}
\newcommand{\eval}[1]{ #1 \bigg|}
\newcommand{\seq}[1]{\left( #1 \right)}
\renewcommand{\epsilon}{\varepsilon}
\renewcommand{\iff}{\Leftrightarrow}

\DeclareMathOperator{\arccot}{arccot}
\DeclareMathOperator{\arcsec}{arcsec}
\DeclareMathOperator{\arccsc}{arccsc}
\DeclareMathOperator{\si}{Si}
\DeclareMathOperator{\proj}{proj}
\DeclareMathOperator{\scal}{scal}
\DeclareMathOperator{\cis}{cis}
\DeclareMathOperator{\Arg}{Arg}
%\DeclareMathOperator{\arg}{arg}
\DeclareMathOperator{\Rep}{Re}
\DeclareMathOperator{\Imp}{Im}
\DeclareMathOperator{\sech}{sech}
\DeclareMathOperator{\csch}{csch}
\DeclareMathOperator{\Log}{Log}

\newcommand{\tightoverset}[2]{% for arrow vec
  \mathop{#2}\limits^{\vbox to -.5ex{\kern-0.75ex\hbox{$#1$}\vss}}}
\newcommand{\arrowvec}{\overrightarrow}
\renewcommand{\vec}{\mathbf}
\newcommand{\veci}{{\boldsymbol{\hat{\imath}}}}
\newcommand{\vecj}{{\boldsymbol{\hat{\jmath}}}}
\newcommand{\veck}{{\boldsymbol{\hat{k}}}}
\newcommand{\vecl}{\boldsymbol{\l}}
\newcommand{\utan}{\vec{\hat{t}}}
\newcommand{\unormal}{\vec{\hat{n}}}
\newcommand{\ubinormal}{\vec{\hat{b}}}

\newcommand{\dotp}{\bullet}
\newcommand{\cross}{\boldsymbol\times}
\newcommand{\grad}{\boldsymbol\nabla}
\newcommand{\divergence}{\grad\dotp}
\newcommand{\curl}{\grad\cross}
%% Simple horiz vectors
\renewcommand{\vector}[1]{\left\langle #1\right\rangle}


\outcome{Define limits, derivatives and integrals of vector-valued functions.}

\title{2.2 Calculus of Space Curves}



\begin{document}

\begin{abstract}
In this section we define limits, derivatives and integrals of vector-valued functions.
\end{abstract}

\maketitle

\begin{definition}[Limit of a Vector-Valued Function]
Limits of vector-valued functions are computed componentwise. If $\vec r(t) = \vector{f(t), g(t), h(t)}$ then
\[
\lim_{t \to c} \vec r(t) = \vector{\lim_{t \to c}f(t), \lim_{t \to c}g(t),\lim_{t \to c} h(t)}
\]
provided the limit of each component exists.
\end{definition}

\begin{example}[Example 1]
Compute the indicated limit of the vector valued function:
\[
\lim_{t \to 0} \vector{\frac{\sin t}{t}, \frac{\ln t}{t}, \frac{e^t - t -1}{t^2}}
\]
Recall \textbf{L'Hopital's Rule}:
\[
\text{If}\; \lim_{x \to c} \frac{f(x)}{g(x)} = \frac00 \; \text{or} \; \frac{\infty}{\infty} 
\]
\[
\text{then} \; \lim_{x \to c} \frac{f(x)}{g(x)} = \lim_{x \to c} \frac{f'(x)}{g'(x)}
\]
as long as $f$ and $g$ are differentiable in an open interval containing $x = c$ (but not necessarily at $x = c$ itself)
and $g'(c) \neq 0$. We can apply this to the limit in each of the components:
\[
\lim_{t \to 0} \frac{\sin t}{t} = \frac00 = \lim_{t\to 0} \cos t = \cos 0  = 1
\]
\[
\lim_{t \to 0} \frac{\ln(1+t)}{t} = \frac00 = \lim_{t\to 0} \frac{1}{1+t} = 1
\]
\[
\lim_{t \to 0} \frac{e^t - t - 1}{t^2} = \frac00 = \lim_{t\to 0} \frac{e^t - 1}{2t} = \frac00 = \frac{e^t}{2}  = \frac12
\]
Hence the limit of the vector-valued function is the vector $\vector{1, 1, 1/2}$.
\end{example}

\begin{problem}(Problem 1)
Compute the limit if the vector-valued function:
\[
\lim_{t \to 1} \vector{\frac{x^5 - 1}{x^3 - 1}, \frac{\tan(t-1)}{t-1}, \frac{1 - \sqrt x}{x-1}}
\]
\end{problem}

\textbf{Continuity} of a vector-valued function is defined in terms of its components.

\begin{definition}[Continuity of Vector-Valued Functions]
A vector-valued function $\vec r(t) = \vector{f(t), g(t), h(t)}$ is
\textbf{continuous} at $t = c$ if
\[
\lim_{t \to c} \vec r(t) = \vec r(c)
\]
\end{definition}

\begin{example}[Example 1]
Verify that the vector-valued function $\vec r(t) = \vector{\sin t, t \cos t, e^{-t}}$ is continuous at $t = 0$.\\
Since
\[
\lim_{t \to 0} \sin t =  \sin 0 = 0
\]
\[
\lim_{t \to 0} t \cos t =  0 \cdot \cos 0 = 0
\]
\[
\lim_{t \to 0} e^{-t} =  e^0 = 1,
\]
we have
\[
\lim_{t \to 0} \vec r(t) = \vector{0, 0, 1} = \vec r(0)
\]
and hence the vector-valued function $\vec r(t)$ is continuous at $t = 0$.

\end{example}

Since continuity is determined componentwise, we can take advantage of our knowledge of 
continuous functions of a single variable.

\begin{proposition}[Continuity]
If $f, g$ and $h$ are continuous at $c$, then the vector-valued function $\vec r(t) = \vector{f(t), g(t), h(t)}$
is continuous at $t = c$
\end{proposition}
\begin{proof}
The proof follows directly from the definitions of continuity of functions of a single 
variable and vector-valued functions and the componentwise computation of limits of vector-valued functions.
\end{proof}

The derivative of a vector valued function is defined in a manner analogous to the derivative of a function of a single variable.

\begin{definition}[Derivative of a Vector-Valued Function]
Let $\vec r(t)$ be a vector-valued function. Its \textbf{derivative} is defined by
\[
\frac{d\vec r}{dt} = \vec r\, '(t) = \lim_{h \to 0} \frac{\vec r(t+h) - \vec r(t)}{h}
\]
provided the limit exists.
\end{definition}

\begin{remark}
The vector 
\[
\frac{\vec r(t+h) - \vec r(t)}{h}
\]
is a scalar multiple of the vector between the points $\vec r(t+h)$ and $\vec r(t)$ on the space 
curve defined by the vector-valued function $\vec r(t)$. 
Hence, we can see that the vector $\vec r\,'(t)$, if it exists, 
will be a vector that is tangent to the curve $\vec r(t)$. 
\end{remark}

\begin{definition}[Tangent Vector and Tangent Line]
If the vector-valued function $\vec r(t)$ is differentiable at $t = t_0$ then the vector $\vec r\,'(t_0)$ is called the 
\textbf{tangent vector} to the curve determined by $\vec r(t)$ at the point corresponding to $t = t_0$. 
Furthermore, the equation of the \textbf{tangent line} to this curve at $ t= t_0$ is given by
\[
\vector{x, y, z} = \vec r(t_0) + t\,\vec r\,'(t_0)
\]
\end{definition}




\end{document}

