\documentclass[handout]{ximera}


%% You can put user macros here
%% However, you cannot make new environments



\newcommand{\ffrac}[2]{\frac{\text{\footnotesize $#1$}}{\text{\footnotesize $#2$}}}
\newcommand{\vasymptote}[2][]{
    \draw [densely dashed,#1] ({rel axis cs:0,0} -| {axis cs:#2,0}) -- ({rel axis cs:0,1} -| {axis cs:#2,0});
}


\graphicspath{{./}{firstExample/}}
\usepackage{forest}
\usepackage{amsmath}
\usepackage{amssymb}
\usepackage{array}
\usepackage[makeroom]{cancel} %% for strike outs
\usepackage{pgffor} %% required for integral for loops
\usepackage{tikz}
\usepackage{tikz-cd}
\usepackage{tkz-euclide}
\usetikzlibrary{shapes.multipart}


%\usetkzobj{all}
\tikzstyle geometryDiagrams=[ultra thick,color=blue!50!black]


\usetikzlibrary{arrows}
\tikzset{>=stealth,commutative diagrams/.cd,
  arrow style=tikz,diagrams={>=stealth}} %% cool arrow head
\tikzset{shorten <>/.style={ shorten >=#1, shorten <=#1 } } %% allows shorter vectors

\usetikzlibrary{backgrounds} %% for boxes around graphs
\usetikzlibrary{shapes,positioning}  %% Clouds and stars
\usetikzlibrary{matrix} %% for matrix
\usepgfplotslibrary{polar} %% for polar plots
\usepgfplotslibrary{fillbetween} %% to shade area between curves in TikZ



%\usepackage[width=4.375in, height=7.0in, top=1.0in, papersize={5.5in,8.5in}]{geometry}
%\usepackage[pdftex]{graphicx}
%\usepackage{tipa}
%\usepackage{txfonts}
%\usepackage{textcomp}
%\usepackage{amsthm}
%\usepackage{xy}
%\usepackage{fancyhdr}
%\usepackage{xcolor}
%\usepackage{mathtools} %% for pretty underbrace % Breaks Ximera
%\usepackage{multicol}



\newcommand{\RR}{\mathbb R}
\newcommand{\R}{\mathbb R}
\newcommand{\C}{\mathbb C}
\newcommand{\N}{\mathbb N}
\newcommand{\Z}{\mathbb Z}
\newcommand{\dis}{\displaystyle}
%\renewcommand{\d}{\,d\!}
\renewcommand{\d}{\mathop{}\!d}
\newcommand{\dd}[2][]{\frac{\d #1}{\d #2}}
\newcommand{\pp}[2][]{\frac{\partial #1}{\partial #2}}
\renewcommand{\l}{\ell}
\newcommand{\ddx}{\frac{d}{\d x}}

\newcommand{\zeroOverZero}{\ensuremath{\boldsymbol{\tfrac{0}{0}}}}
\newcommand{\inftyOverInfty}{\ensuremath{\boldsymbol{\tfrac{\infty}{\infty}}}}
\newcommand{\zeroOverInfty}{\ensuremath{\boldsymbol{\tfrac{0}{\infty}}}}
\newcommand{\zeroTimesInfty}{\ensuremath{\small\boldsymbol{0\cdot \infty}}}
\newcommand{\inftyMinusInfty}{\ensuremath{\small\boldsymbol{\infty - \infty}}}
\newcommand{\oneToInfty}{\ensuremath{\boldsymbol{1^\infty}}}
\newcommand{\zeroToZero}{\ensuremath{\boldsymbol{0^0}}}
\newcommand{\inftyToZero}{\ensuremath{\boldsymbol{\infty^0}}}


\newcommand{\numOverZero}{\ensuremath{\boldsymbol{\tfrac{\#}{0}}}}
\newcommand{\dfn}{\textbf}
%\newcommand{\unit}{\,\mathrm}
\newcommand{\unit}{\mathop{}\!\mathrm}
%\newcommand{\eval}[1]{\bigg[ #1 \bigg]}
\newcommand{\eval}[1]{ #1 \bigg|}
\newcommand{\seq}[1]{\left( #1 \right)}
\renewcommand{\epsilon}{\varepsilon}
\renewcommand{\iff}{\Leftrightarrow}

\DeclareMathOperator{\arccot}{arccot}
\DeclareMathOperator{\arcsec}{arcsec}
\DeclareMathOperator{\arccsc}{arccsc}
\DeclareMathOperator{\si}{Si}
\DeclareMathOperator{\proj}{proj}
\DeclareMathOperator{\scal}{scal}
\DeclareMathOperator{\cis}{cis}
\DeclareMathOperator{\Arg}{Arg}
%\DeclareMathOperator{\arg}{arg}
\DeclareMathOperator{\Rep}{Re}
\DeclareMathOperator{\Imp}{Im}
\DeclareMathOperator{\sech}{sech}
\DeclareMathOperator{\csch}{csch}
\DeclareMathOperator{\Log}{Log}

\newcommand{\tightoverset}[2]{% for arrow vec
  \mathop{#2}\limits^{\vbox to -.5ex{\kern-0.75ex\hbox{$#1$}\vss}}}
\newcommand{\arrowvec}{\overrightarrow}
\renewcommand{\vec}{\mathbf}
\newcommand{\veci}{{\boldsymbol{\hat{\imath}}}}
\newcommand{\vecj}{{\boldsymbol{\hat{\jmath}}}}
\newcommand{\veck}{{\boldsymbol{\hat{k}}}}
\newcommand{\vecl}{\boldsymbol{\l}}
\newcommand{\utan}{\vec{\hat{t}}}
\newcommand{\unormal}{\vec{\hat{n}}}
\newcommand{\ubinormal}{\vec{\hat{b}}}

\newcommand{\dotp}{\bullet}
\newcommand{\cross}{\boldsymbol\times}
\newcommand{\grad}{\boldsymbol\nabla}
\newcommand{\divergence}{\grad\dotp}
\newcommand{\curl}{\grad\cross}
%% Simple horiz vectors
\renewcommand{\vector}[1]{\left\langle #1\right\rangle}


\outcome{Estimate limits using numerical information}

\title{1.1 Velocity}

\begin{document}

\begin{abstract}
We compute average velocity to estimate instantaneous velocity.
\end{abstract}

\maketitle

\section{Average Velocity}
We will consider an object moving along a straight line, such as a vertically free-falling object. If the position of the object at time $t$
is denoted by $s(t)$, then the \textbf{average velocity} of the object from time $t = a$ to $t = b$ is given by 
\[
v_{ave} = \frac{\Delta s}{\Delta t} = \frac{s(b) - s(a)}{b-a}.
\]
In other words, average velocity is the change in position divided by the change in time. 
If the position is measured in feet and the time is measured in seconds, then average velocity will be 
measured in feet per second (ft/sec). The change in position, $s(b) - s(a)$ is also known as \textbf{displacement}.


\begin{example}[example 1]
The height of a falling object at time $t$ seconds is given by $s(t) = 600-16t^2$ feet.
Find the average velocity of the object over the time interval $[2, 5]$.\\
We compute $s(5), s(2)$ and the displacement, $s(5) - s(2)$:
\[
s(5) = 600 - 16\cdot 5^2 = 200 \text{ft}, \; s(2) = 600 - 16 \cdot 2^2 = 536 \text{ft}, \; \text {and} \; s(5) - s(2) = -336 \text{ft}.
\]
The average velocity is
\[
v_{ave} = \frac{s(5) - s(2)}{5-2} = \frac{-336}{3} = - 112 \text{ft/sec}.
\]
Note that the negative sign in the average velocity indicates that the object is 
falling.
\end{example}



\begin{problem}(problem 1)
The height of a vertically free-falling object at time $t$ seconds is given by $s(t) = 300 - 16t^2$ feet.
Find the average velocity of the object from time $t = 2$ seconds to time $t = 4$ seconds.\\
The positions at times $t = 2$ and $t= 4$ are
\[
s(2) = \answer{236}ft \; \text{and} \; s(4) = \answer{44}ft.
\]
The displacement is 
\[
s(4) - s(2) = \answer{-192}ft.
\]
The average velocity of the object over the time interval $[2,4]$ is 
\[
v_{ave} = \answer{-96}ft/sec.
\]

\end{problem}


Average velocity has a connection with the slope of a line. Recall that the slope of a line between two points, $(x_1, y_1)$
and $(x_2, y_2)$ is given by
\[
m = \frac{\text{rise}}{\text{run}} = \frac{\Delta y}{\Delta x} = \frac{y_2 - y_1}{x_2 -x_1}.
\]
Thus, if we make a graph of position versus time, with time on the horizontal axis and position on the vertical axis, then the average velocity 
of a vertically free-falling object from time $t = a$ to time $t = b$ is the same as the slope of the line segment
connecting the points $(a, s(a))$ and $(b, s(b))$. See the figure below.

\begin{image}
\begin{tikzpicture}

\node at (2,5) {Average Velocity and Slope};
\draw[->, thick] (0,0) --(4.5,0)  node[below] {t};
\draw[thin, dashed] (1,3) -- (1,0) node[below] {a};
\draw[thin, dashed, gray] (3,1) -- (3,0) node[below, black] {b};

\draw[->, thick] (0,0) -- (0,4.5) node[left] {$s(t)$};
\draw[thin, dashed] (1,3) -- (0,3) node[left] {$s(a)$};
\draw[thin, dashed, gray] (3,1) -- (0,1) node[left, black] {$s(b)$};
\filldraw (1,3) circle(.05) node[right] {$(a,s(a))$};
\filldraw (3,1) circle(.05) node[right] {$(b,s(b))$};
\draw[thin, blue] (1,3) --(3,1) node[right, midway] {$m = \frac{s(b) -s(a)}{b-a} = $ average velocity};

\node at (2,-1) {The slope of the blue line is the average velocity};
\node at (2, -1.5) {of the object from time $t = a$ to time $t = b$.};

\end{tikzpicture}
\end{image}


\begin{example}[example 2]
In the last example, with $s(t) = 600 - 16t^2$, we found that $s(2) = 536$ft and $s(5) = 200$ft. If we plot the points
$(2, 536)$ and $(5, 200)$ then the slope of the segment connecting them is the average velocity of the falling object from time
$t = 2$ seconds to time $t = 5$ seconds, as shown in the figure below.


\begin{image}
\begin{tikzpicture}

\node at (2,4.5) {Average Velocity and Slope};
\draw[->, thick] (0,0) --(4,0)  node[below] {t};
\draw[thin, dashed] (1,3) -- (1,0) node[below] {2};
\draw[thin, dashed, gray] (3,1) -- (3,0) node[below, black] {5};

\draw[->, thick] (0,0) -- (0,4) node[left] {$s(t)$};
\draw[thin, dashed] (1,3) -- (0,3) node[left] {$536$};
\draw[thin, dashed, gray] (3,1) -- (0,1) node[left, black] {$200$};
\filldraw (1,3) circle(.05) node[right] {$(2,536)$};
\filldraw (3,1) circle(.05) node[right] {$(5,200)$};
\draw[thin, blue] (1,3) --(3,1) node[midway,right] {$m = -112$ft/sec};

\node at (2,-1) {The slope of the blue line is the average velocity};
\node at (2, -1.5) {of the object from time $t = 2$ to time $t = 5$.};

\end{tikzpicture}
\end{image}

\end{example}


\begin{problem}(problem 2)
Find the slope of the line between the points $(2,236)$ and $(4,44)$.\\
\[
m = \answer{-96}.
\]
\end{problem}

\section{Instantaneous Velocity}
To determine the velocity of an object at a particular moment in time, i.e., 
the \textbf{instantaneous velocity}, we find the average velocity 
over smaller and smaller time intervals.

\begin{example}[example 3]
The height of a falling object at time $t$ seconds is given by $s(t) = 600-16t^2$ feet.
Find the average velocity of the object over the time 
intervals $[4, 5], [4.5, 5], [4.9, 5]$ and $[5, 5.1]$.\\
First we will compute the height (in feet) at time $t = 4, 4.5, 4.9, 5$ and $5.1$ seconds:
  
\[
\begin{array}{ c | c | c | c | c | c }
   t& 4 & 4.5 & 4.9 & 5 & 5.1 \\ 
	\hline
	s(t) & 344 & 276 & 215.84 & 200 & 183.84
\end{array}
\]

The average velocities (in ft/sec) are:
\[
\begin{array}{ c | c  }
   \text{interval}& v_{ave} \\ 
	\hline
	[4,5] & -144 \\
	\hline
	[4.5,5] & -152 \\
	\hline
	[4.9,5] & -158.4 \\
	\hline
	[5, 5.1] & -161.6
\end{array}
\]

We can now estimate the instantaneous velocity of the object at time $t = 5$ seconds.  Based on the 
average velocities over the smallest time intervals, namely, $-158.4$ft/sec 
and $-161.6$ft/sec, it seems that $-160$ft/sec would be a reasonable estimate 
for the instantaneous velocity of the falling object at time $t = 5$ seconds.
To improve our estimate, or have more confidence in it, we could compute the average velocity over 
even smaller time intervals.

\end{example}

\begin{problem}(problem 3a)
The height of a vertically free-falling object at time $t$ seconds is given by
\[
s(t) = 200 - 16t^2 \; \text{feet}.
\]
Find the average velocity of the object over each of time intervals given below and use your 
results to estimate the instantaneous velocity
of the object at time, $t = 3$ seconds.

\[
\begin{array}{ c | c  }
   \text{interval}& v_{ave} \\ 
	\hline
	[2.9, 3] & \answer{-94.4}ft/sec \\
	\hline
	[2.99, 3] & \answer{-95.84}ft/sec \\
	\hline
	[3, 3.1] & \answer{-97.6}ft/sec \\
	\hline
	[3, 3.01] & \answer{-96.16}ft/sec
\end{array}
\]


Instantaneous velocity at $t = 3$ seconds: $\answer{-96}ft/sec$.

\end{problem}


\begin{problem}(problem 3b)
The height of a vertical projectile at time $t$ seconds is given by
\[
s(t) = 50t - 16t^2 \text{feet}.
\]
Find the average velocity of the object over each of time intervals given below and use your 
results to estimate the instantaneous velocity
of the object at time, $t = 1$ second.

\[
\begin{array}{ c | c  }
   \text{interval}& v_{ave} \\ 
	\hline
	[0.9, 1] & \answer{19.6}ft/sec \\
	\hline
	[0.99, 1] & \answer{18.16}ft/sec \\
	\hline
	[1, 1.1] & \answer{16.4}ft/sec \\
	\hline
	[1, 1.01] & \answer{17.84}ft/sec
\end{array}
\]


Instantaneous velocity at $t = 1$ second: $\answer{18}ft/sec$.\\
The positive answer to this problem indicates that the projectile was rising at time $t= 1$ second.

\end{problem}

\section{The Tangent Line}
We saw that the average velocity of a falling object can be represented by the slope of a line. 
The instantaneous velocity of a falling object can also be represented by the slope of a line.
The graph below represents the position function for a falling object.  
The slope of the blue \textbf{secant line} represents the average velocity of the object over the time interval $[a,b]$.
The slope of the red \textbf{tangent line} represents the instantaneous velocity of the object at time $t = b$.

\begin{center}
\begin{tikzpicture}
\begin{axis}[axis x line= none, axis y line = none, title={Average vs Instantaneous Velocity}] 

\addplot[domain=0:4, samples = 100, color=black, thick]{4-x^2/4};
\addplot[domain=1:3, samples = 100, color = blue, thick]{-x+19/4};
\addplot[domain=2:4, samples = 100, color = red, thick]{-3*x/2 + 25/4};
%(1, 15/4) and (3,7/4)  m = -1
%(3,7/4) m = -3/2
%\addplot[<->] coordinates {(0,-2) (0,3.5)}; %y-axis
%\addplot[<->] coordinates {(-2.3,0) (2.3,0)}; %x-axis

\addplot[->, thick] coordinates {(0,0) (0,4.5)}; %y-axis
\addplot[->, thick] coordinates {(0,0) (4.5,0)}; %x-axis

\addplot[dashed] coordinates {(1,0) (1,3.75 )};
\addplot[dashed] coordinates {(3,0 ) (3,1.75)};
%\node at (axis cs: 1.1,2.3){$y=2-x^2$};

\node at (axis cs: -.25,4.4){$s(t)$};
\node at (axis cs: 4.4,-.2){$t$};
\node at (axis cs: 1,-.2){$a$};
\node at (axis cs: 3,-.2){$b$};

\node[blue] at (axis cs: 1.3,2.7){$m = v_{ave}$};
\node[red] at (axis cs: 3.4,3.2){$m = $ inst. velocity};

\end{axis}
\end{tikzpicture}
\end{center}


In \textbf{differential} calculus, we study the tangent line and its applications.


\end{document}





