\documentclass[handout]{ximera}

%% You can put user macros here
%% However, you cannot make new environments



\newcommand{\ffrac}[2]{\frac{\text{\footnotesize $#1$}}{\text{\footnotesize $#2$}}}
\newcommand{\vasymptote}[2][]{
    \draw [densely dashed,#1] ({rel axis cs:0,0} -| {axis cs:#2,0}) -- ({rel axis cs:0,1} -| {axis cs:#2,0});
}


%\usepackage{tcolorbox} %%Needed for Derivative Definition supposedly and product rule, natural exp log, quotient rule, inverse trig, rates of change


% \graphicspath{{./}{firstExample/}}
% \usepackage{forest}
\usepackage{amsmath}
\usepackage{amssymb}
\usepackage{array}
\usepackage[makeroom]{cancel} %% for strike outs
\usepackage{pgffor} %% required for integral for loops
\usepackage{tikz}
\usepackage{tikz-cd}
\usepackage{tkz-euclide}
\usetikzlibrary{shapes.multipart}


% \usetkzobj{all}
\tikzstyle geometryDiagrams=[ultra thick,color=blue!50!black]


\usetikzlibrary{arrows}
\tikzset{>=stealth,commutative diagrams/.cd,
  arrow style=tikz,diagrams={>=stealth}} %% cool arrow head
\tikzset{shorten <>/.style={ shorten >=#1, shorten <=#1 } } %% allows shorter vectors

\usetikzlibrary{backgrounds} %% for boxes around graphs
\usetikzlibrary{shapes,positioning}  %% Clouds and stars
\usetikzlibrary{matrix} %% for matrix
\usepgfplotslibrary{polar} %% for polar plots
\usepgfplotslibrary{fillbetween} %% to shade area between curves in TikZ



%\usepackage[width=4.375in, height=7.0in, top=1.0in, papersize={5.5in,8.5in}]{geometry}
%\usepackage[pdftex]{graphicx}
%\usepackage{tipa}
%\usepackage{txfonts}
%\usepackage{textcomp}
%\usepackage{amsthm}
%\usepackage{xy}
%\usepackage{fancyhdr}
%\usepackage{xcolor}
%\usepackage{mathtools} %% for pretty underbrace % Breaks Ximera
%\usepackage{multicol}



\newcommand{\RR}{\mathbb R}
\newcommand{\R}{\mathbb R}
\newcommand{\C}{\mathbb C}
\newcommand{\N}{\mathbb N}
\newcommand{\Z}{\mathbb Z}
\newcommand{\dis}{\displaystyle}
%\renewcommand{\d}{\,d\!}
\renewcommand{\d}{\mathop{}\!d}
\newcommand{\dd}[2][]{\frac{\d #1}{\d #2}}
\newcommand{\pp}[2][]{\frac{\partial #1}{\partial #2}}
\renewcommand{\l}{\ell}
\newcommand{\ddx}{\frac{d}{\d x}}
\newcommand{\ppx}{\frac{\partial}{\partial x}}
\newcommand{\ppy}{\frac{\partial}{\partial y}}

\newcommand{\zeroOverZero}{\ensuremath{\boldsymbol{\tfrac{0}{0}}}}
\newcommand{\inftyOverInfty}{\ensuremath{\boldsymbol{\tfrac{\infty}{\infty}}}}
\newcommand{\zeroOverInfty}{\ensuremath{\boldsymbol{\tfrac{0}{\infty}}}}
\newcommand{\zeroTimesInfty}{\ensuremath{\small\boldsymbol{0\cdot \infty}}}
\newcommand{\inftyMinusInfty}{\ensuremath{\small\boldsymbol{\infty - \infty}}}
\newcommand{\oneToInfty}{\ensuremath{\boldsymbol{1^\infty}}}
\newcommand{\zeroToZero}{\ensuremath{\boldsymbol{0^0}}}
\newcommand{\inftyToZero}{\ensuremath{\boldsymbol{\infty^0}}}


\newcommand{\numOverZero}{\ensuremath{\boldsymbol{\tfrac{\#}{0}}}}
\newcommand{\dfn}{\textbf}
%\newcommand{\unit}{\,\mathrm}
\newcommand{\unit}{\mathop{}\!\mathrm}
%\newcommand{\eval}[1]{\bigg[ #1 \bigg]}
\newcommand{\eval}[1]{ #1 \bigg|}
\newcommand{\seq}[1]{\left( #1 \right)}
\renewcommand{\epsilon}{\varepsilon}
\renewcommand{\iff}{\Leftrightarrow}

\DeclareMathOperator{\arccot}{arccot}
\DeclareMathOperator{\arcsec}{arcsec}
\DeclareMathOperator{\arccsc}{arccsc}
\DeclareMathOperator{\si}{Si}
\DeclareMathOperator{\proj}{proj}
\DeclareMathOperator{\scal}{scal}
\DeclareMathOperator{\cis}{cis}
\DeclareMathOperator{\Arg}{Arg}
%\DeclareMathOperator{\arg}{arg}
\DeclareMathOperator{\Rep}{Re}
\DeclareMathOperator{\Imp}{Im}
\DeclareMathOperator{\sech}{sech}
\DeclareMathOperator{\csch}{csch}
\DeclareMathOperator{\Log}{Log}

\newcommand{\tightoverset}[2]{% for arrow vec
  \mathop{#2}\limits^{\vbox to -.5ex{\kern-0.75ex\hbox{$#1$}\vss}}}
\newcommand{\arrowvec}{\overrightarrow}
\renewcommand{\vec}{\mathbf}
\newcommand{\veci}{{\boldsymbol{\hat{\imath}}}}
\newcommand{\vecj}{{\boldsymbol{\hat{\jmath}}}}
\newcommand{\veck}{{\boldsymbol{\hat{k}}}}
\newcommand{\vecl}{\boldsymbol{\l}}
\newcommand{\utan}{\vec{\hat{t}}}
\newcommand{\unormal}{\vec{\hat{n}}}
\newcommand{\ubinormal}{\vec{\hat{b}}}

\newcommand{\dotp}{\bullet}
\newcommand{\cross}{\boldsymbol\times}
\newcommand{\grad}{\boldsymbol\nabla}
\newcommand{\divergence}{\grad\dotp}
\newcommand{\curl}{\grad\cross}
%% Simple horiz vectors
\renewcommand{\vector}[1]{\left\langle #1\right\rangle}


\pgfplotsset{compat=1.13}

\outcome{Mapping properties of complex functions}

\title{2.4 Complex Mappings}

\begin{document}

\begin{abstract}
We determine the image of special sets.
\end{abstract}

\maketitle

\section{Linear Functions}
A linear function has the form $f(z) = az+b$ where $a, b \in \C$ and $a \neq 0$ (if $a=0$, then $f$ is a constant function).
The effect of a linear function on a set in the complex plane is to scale, rotate and translate that set.  The constant term $b$ acts as a translation vector, 
and the modulus and argument of $a$ are responsible for producing scaling and rotation, respectively.

\begin{example}[example 1]
Find the image of the unit square under the mapping $f(z) = (1+i)z + i$.\\
In this case, $a = 1+i$ which has modulus $\sqrt 2$ and argument $\pi/4$, and $b = i$.
The effect on the square of multiplying by $1+i$ is to scale the square by a factor of $\sqrt 2$, rotate counter-clockwise by $\pi/4$ and then translate it
by $i$. The scaling and the rotation can be done in either order, but the translation must come last.


\begin{image}
\begin{tikzpicture}
\draw[<->, thick] (-3,0)--(3,0) node[right]{x};

\draw[<->, thick] (0, -3) node[right]{$z$-plane} --(0,3) node[right]{iy};

\draw[thick,blue, fill= blue!20] (0,0) -- (1.2,0) node[below, black]{1} -- (1.2,1.2) -- (0, 1.2) node[left, black]{i}  -- (0,0);

%\draw[mark=*,mark size=1pt,mark options={color=blue}] plot coordinates {(-0.5,.4)} node[right, blue]{$z_0 = x_0 + iy_0$};
%\draw[mark=*,mark size=1pt,mark options={color=blue}] plot coordinates {(-0.5,1.4)} node[right, blue]{$z_0+ 2\pi i$};


%\draw[mark=*,mark size=1pt,mark options={color=blue}] plot coordinates {(-0.5,-.6)} node[right, blue]{$z_0- 2\pi i$};
%\draw[mark=*,mark size=1pt,mark options={color=blue}] plot coordinates {(-0.5,-1.6)} node[right, blue]{$z_0- 4\pi i$};


\draw[<->, thick] (6,0)--(12,0)node[right]{u};



\draw[thick,blue, fill= blue!20] (9,.5) node[left, black]{i}-- (10,1.5) node[right, black]{$1+2i$} -- (9,2.5) node[left, black]{3i} -- 
(8, 1.5) node[left, black]{$-1+2i$}  -- (9,.5);
\draw[<->, thick] (9, -3) node[right]{$w$-plane}--(9,3) node[right]{iv};
%\draw[mark=*,mark size=1pt,mark options={color=blue}] plot coordinates {(8.5,1.3)} node[right, blue]{$w_0$};
%\draw[dashed] (7,0) -- (8.5, 1.3) node[above, midway]{$e^{x_0}$};
%\draw (7.4,0) arc (0:42:.4) node[midway, right]{$y_0$} ;
\draw[->, thick, blue] (3.5,1)--(5.5, 1) node[above,midway, blue]{$w = f(z)$};
\node at (4.5, -4){\large The effect of $f(z) = (1+i)z + i$ on the unit square};
\end{tikzpicture}
\end{image}


\end{example}


\begin{problem}(problem 1)
Find the image of the unit square under the given mapping:
\begin{align*}
i) & f(z) = iz\\
ii) & f(z) = 2z\\
iii) & f(z) = z+1\\
iv) & f(z) = (1-i)z +i
\end{align*}
\end{problem}



\begin{example}[example 2]
Find the image of the sector of an annulus given by:
\[
S = \left\{r\cis \theta : \frac12 < r < 2, \;\frac{\pi}{6} < \theta < \frac{\pi}{2}\right\}
\]
under the mapping $f(z) = z^2$.\\
The effect of squaring is to square moduli and double arguments. Hence the image, $f(S)$ is the sector
\[
f(S) = \left\{r\cis \theta : \frac14 < r < 4, \;\frac{\pi}{3} < \theta < \pi \right\}
\]


\begin{image}
\begin{tikzpicture}
\draw[dashed, blue, fill=blue!20] (90:.75) node[left]{$\frac{i}{2}$} arc (90:30:.75) coordinate (alpha) -- (30:2.25) arc (30:90:2.25) node[left]{$2i$} -- cycle;
\draw[->, thin, black] (1.5,0) arc (0:30:1.5) node[right, midway]{$\pi/6$};
\draw[<->, thick] (-3,0)--(3,0) node[right]{$x$};

\draw[<->, thick] (0, -3) node[right]{$z$-plane} --(0,3) node[right]{$iy$};

\draw[dashed, blue, fill=blue!20, shift ={(9,0)}] (180:.5) node[below]{$-\frac{1}{4}$} arc (180:60:.5) 
coordinate (alpha) -- (60:2.5) arc (60:180:2.5) node[below]{$-4$} -- cycle;


\draw[->, thin, black, shift={(9,0)}] (1.5,0) arc (0:60:1.5) node[right, midway]{$\pi/3$};

\draw[<->, thick] (6,0)--(12,0)node[right]{$u$};


\draw[<->, thick] (9, -3) node[right]{$w$-plane}--(9,3) node[right]{$iv$};

\draw[->, thick, blue] (3.5,1)--(5.5, 1) node[above,midway, blue]{$w = z^2$};
\node at (4.5, -4){\large The effect of $f(z) = z^2$ on a sector of an annulus};
\end{tikzpicture}
\end{image}

\end{example}


\begin{problem}(problem 2)
Find the image of the given set under the mapping $f(z) = z^2$:
\begin{align*}
i) & S = \left\{r\cis \theta : 1 < r < 2, \;0 < \theta < \frac{\pi}{4}\right\} \\
ii) & S = \left\{r\cis \theta : 2 < r < 3, \;\frac{\pi}{2} < \theta < \frac{3\pi}{4}\right\}\\
iii) & \mbox{Upper Half Plane}\; = \left\{ x+iy: y> 0\right\}\\
iv) & \mbox{Quadrant IV}\; = \left\{ x+iy: x> 0, y< 0\right\}
\end{align*}
\end{problem}



\begin{example}[example 3]
Find the image of vertical line $\Rep z = 1$ under the mapping $f(z) = z^2$.\\
Instead of a polar analysis, we will use rectangular coordinates with $x = 1$. We have
\[
w = u+iv = (1+iy)^2 = (1-y^2) + 2iy
\]
Thus $u = 1-y^2$ and $v = 2y$, where $y \in R$. Eliminating $y$ gives
\[
u= 1-\frac{v^2}{4}
\]
which is a parabola in the $uv$-plane which opens to the left and has vertex at $(1,0)$.


\begin{image}
\begin{tikzpicture}

\draw[<->, thick] (-3,0)--(3,0) node[right]{$x$};

\draw[<->, thick] (0, -3) node[right]{$z$-plane} --(0,3) node[right]{$iy$};

\draw[blue, <->] (1, -2.5) -- (1, 2.5) ;
\node[blue] at (1.6, 1){$x = 1$};

\draw[<->,blue, rotate=-90, shift={(0,9)}] (-3,-1.25) parabola bend (0,1) (3,-1.25) ;
\node[blue] at (11,1){$u = 1-\frac{v^2}{4}$};

\draw[<->, thick] (6,0)--(12,0)node[right]{$u$};


\draw[<->, thick] (9, -3) node[right]{$w$-plane}--(9,3) node[right]{$iv$};

\draw[->, thick, blue] (3.5,1)--(5.5, 1) node[above,midway, blue]{$w = z^2$};
\node at (4.5, -4){\large The effect of $f(z) = z^2$ on a vertical line};
\end{tikzpicture}
\end{image}

\end{example}


\begin{problem}(problem 3)
Find the image of the given line under the mapping $f(z) = z^2$:
\begin{align*}
i) & \Rep z = 2 \\
ii) & \Rep z = -1\\
iii) & \Imp z = 1\\
iv) & \Imp z = -2
\end{align*}
\end{problem}


\section{Conjugates and Reciprocals}
The function $f(z) = \overline{z}$ maps a set in the complex plane to its mirror image in the real axis.

\begin{example}[example 4]
Find the image of the closed disk $D(z_0, r)$ under the mapping $f(z) = 2\overline{z}$.\\
The image of the center of the disk is $f(z_0) = 2\overline{z_0}$ and so the image of the disk
$D(z_0, r)$ is the disk of twice the radius, $D(\overline{z_0}, 2r)$.

\begin{image}
\begin{tikzpicture}

\draw[<->, thick] (-3,0)--(3,0) node[right]{$x$};

\draw[<->, thick] (0, -3) node[right]{$z$-plane} --(0,3) node[right]{$iy$};

\draw[dashed, blue, fill = blue!20] (1.25, 1.25) circle (0.75) ;
\draw[mark=*,mark size=1pt,mark options={color=blue}] plot coordinates {(1.25, 1.25)} node[left, blue]{$z_0$};
\draw[blue] (1.25, 1.25)--(1.78, 1.78) node[right, midway]{$r$};


\draw[dashed, blue, fill= blue!20] (10.25, -1.25)circle (1.5) ;
\draw[mark=*,mark size=1pt,mark options={color=blue}] plot coordinates {(10.25, -1.25)} node[left, blue]{$\overline{z_0}$};
\draw[blue] (10.25, -1.25)--(11.31, -.19) node[right, midway]{$2r$};

\draw[->, thick, blue] (3.5,1)--(5.5, 1) node[above,midway, blue]{$w = 2\overline{z}$};
\draw[<->, thick] (6,0)--(12,0)node[right]{$u$};


\draw[<->, thick] (9, -3) node[right]{$w$-plane}--(9,3) node[right]{$iv$};


\node at (4.5, -4){\large The effect of $f(z) = 2\overline{z}$ on a disk};
\end{tikzpicture}
\end{image}

\end{example}


\begin{problem}(problem 4)
Find the image of the disk $D(1+i, 1)$ under the given mapping:
\begin{align*}
i) & f(z) = \overline{z} \\
ii) & f(z) = -2\overline{z} \\
iii) & f(z) = 3i\overline{z} \\
iv) & f(z) = i+\overline{z} 
\end{align*}
\end{problem}


To understand the mapping properties of $f(z) = \frac{1}{z}$ it is useful to rewrite it in the following way:
\[
\frac{1}{z} = \frac{1}{z} \cdot \frac{\overline{z}}{\overline{z}}= \frac{\overline{z}}{|z|^2}
\]
Thus the mapping $f(z) = \frac{1}{z}$ has both a reflecting property like the conjugate mapping and a scaling property
due to the division of the positive real number $|z|^2$. As a result of the scaling, the image of a point inside the unit circle 
will map to a point outside of the circle and vice versa. This is shown in the figure below.

\begin{image}
\begin{tikzpicture}
\filldraw[dashed, bottom color = red!30, top color = blue!30] (0,0) circle (1) ;
\draw[mark=*,mark size=1pt,] plot coordinates {(.3, .75)} ;
\node at (.45,.55) {$z_0$};
\draw[<->, thick] (-3,0)--(3,0) node[right]{$x$};

\draw[<->, thick] (0, -3) node[right]{$z$-plane} --(0,3) node[right]{$iy$};
\draw[dashed, fill = white] (0,0)circle (.05) ;



\filldraw[draw = none, top color = red!30, bottom color = blue!30] (6.2,-2.8) rectangle (11.8,2.8) ;
\draw[dashed, fill = white] (9,0)circle (1) ;
\draw[mark=*,mark size=1pt] plot coordinates {(10, -2.5)} ;
\node at (10.3,-2.2) {$1/z_0$};
\draw[->, thick, blue] (3.5,1)--(5.5, 1) node[above,midway, blue]{$w = \frac{1}{z}$};
\draw[<->, thick] (6,0)--(12,0)node[right]{$u$};


\draw[<->, thick] (9, -3) node[right]{$w$-plane}--(9,3) node[right]{$iv$};


\node at (4.5, -4){\large The effect of $f(z) = \frac{1}{z}$ on the open punctured unit disk};
\end{tikzpicture}
\end{image}

The color coding in the figure is intended to convey reflection in the real axis. A blue point in the upper half plane
inside the disk maps to a blue point in the lower half plane outside the disk. Similarly, a red point in the lower half plane inside the disk maps to a
red point outside the disk.

\begin{example}[example 5]
Find the image of the vertical ray $\Rep z = \frac12, \Imp z > 0$ under the mapping $f(z) = \frac{1}{z}$.\\
We write $1/z$ in rectangular form:
\[
\frac{1}{z} = \frac{\overline{z}}{|z|^2} = \frac{x-iy}{x^2 + y^2}
\]
Substituting $x = 1/2$, we have
\[
u = \frac{2}{1+4y^2} \quad \mbox{and} \quad v = -\frac{4y}{1+4y^2}
\]
Note $v < 0$ since $y > 0$. Furthermore, observe that $u$ and $v$ are related by $v = -2uy$. 
To eliminate $y$ from this last equation, we rewrite the expression for $u$, which gives
\[
y = \sqrt{\frac{2-u}{4u}} 
\]
Substituting into $v = -2uy$ gives
\[
v = -2u\sqrt{\frac{2-u}{4u}} = -\sqrt{u(2-u)} = -\sqrt{2u-u^2}
\]
To see what type of curve this is, square both sides :
\[
v^2 = 2u - u^2
\]
and so $u^2 - 2u + v^2 = 0$.  Completing the square we get
\[
(u-1)^2 + v^2 = 1
\]
which is the circle $C(1,1)$. Recalling that $v <0$
we conclude that the image of the ray $x = 1/2, y >0$ is a semi circle in the lower half plane with 
center at $1$ and radius $1$.


\begin{image}
\begin{tikzpicture}

%z-plane
\draw[<->, thick] (-2.5,0)--(2.5,0) node[right]{$x$}; %real axis
\draw[<->, thick] (0, -2.5) node[right]{$z$-plane} --(0,2.5) node[left]{$iy$}; %imaginary axis

\draw[blue, thick, ->] (0.5, 0) -- (0.5, 2.5) ; %domain
\draw[thick,blue, fill = white] (0.5,0) circle (0.07) ;
\node[blue, thick] at (1.5, 1.5){$x = 1, y > 0$}; %domain label


%w-plane
\draw[<->, thick] (5,0)--(10,0)node[right]{$u$}; %real axis
\draw[<->, thick] (7.5, -2.5) node[right]{$w$-plane}--(7.5,2.5) node[left]{$iv$}; %imaginary axis

\draw[blue, thick, ->] (9.5,0) arc (360:180:1); %image
\draw[thick,blue, fill = white] (9.5,0) circle (0.07) ;
\node[blue, thick] at (10, -1.5){$(u-1)^2 + v^2 = 1, v< 0$}; %image label

\node at (3.75, -3.5){\large The effect of $f(z) = \frac{1}{z}$ on a vertical ray}; %caption
\end{tikzpicture}
\end{image}

\end{example}


\begin{problem}(problem 5)
Find the image of the following sets under the mapping $f(z) = \frac{1}{z}$:
\begin{align*}
i) & \Rep z = \frac12, \Imp z \leq 0 \\
ii) & \Rep z = 2, \Imp z > 0 \\
iii) & \Imp z = 3, \Rep z \geq 0 \\
iv) & \Imp z = \frac13, \Rep z < 0 
\end{align*}
\end{problem}



\end{document}








