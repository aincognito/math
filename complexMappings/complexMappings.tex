\documentclass[handout]{ximera}

%% You can put user macros here
%% However, you cannot make new environments



\newcommand{\ffrac}[2]{\frac{\text{\footnotesize $#1$}}{\text{\footnotesize $#2$}}}
\newcommand{\vasymptote}[2][]{
    \draw [densely dashed,#1] ({rel axis cs:0,0} -| {axis cs:#2,0}) -- ({rel axis cs:0,1} -| {axis cs:#2,0});
}


\graphicspath{{./}{firstExample/}}
\usepackage{forest}
\usepackage{amsmath}
\usepackage{amssymb}
\usepackage{array}
\usepackage[makeroom]{cancel} %% for strike outs
\usepackage{pgffor} %% required for integral for loops
\usepackage{tikz}
\usepackage{tikz-cd}
\usepackage{tkz-euclide}
\usetikzlibrary{shapes.multipart}


%\usetkzobj{all}
\tikzstyle geometryDiagrams=[ultra thick,color=blue!50!black]


\usetikzlibrary{arrows}
\tikzset{>=stealth,commutative diagrams/.cd,
  arrow style=tikz,diagrams={>=stealth}} %% cool arrow head
\tikzset{shorten <>/.style={ shorten >=#1, shorten <=#1 } } %% allows shorter vectors

\usetikzlibrary{backgrounds} %% for boxes around graphs
\usetikzlibrary{shapes,positioning}  %% Clouds and stars
\usetikzlibrary{matrix} %% for matrix
\usepgfplotslibrary{polar} %% for polar plots
\usepgfplotslibrary{fillbetween} %% to shade area between curves in TikZ



%\usepackage[width=4.375in, height=7.0in, top=1.0in, papersize={5.5in,8.5in}]{geometry}
%\usepackage[pdftex]{graphicx}
%\usepackage{tipa}
%\usepackage{txfonts}
%\usepackage{textcomp}
%\usepackage{amsthm}
%\usepackage{xy}
%\usepackage{fancyhdr}
%\usepackage{xcolor}
%\usepackage{mathtools} %% for pretty underbrace % Breaks Ximera
%\usepackage{multicol}



\newcommand{\RR}{\mathbb R}
\newcommand{\R}{\mathbb R}
\newcommand{\C}{\mathbb C}
\newcommand{\N}{\mathbb N}
\newcommand{\Z}{\mathbb Z}
\newcommand{\dis}{\displaystyle}
%\renewcommand{\d}{\,d\!}
\renewcommand{\d}{\mathop{}\!d}
\newcommand{\dd}[2][]{\frac{\d #1}{\d #2}}
\newcommand{\pp}[2][]{\frac{\partial #1}{\partial #2}}
\renewcommand{\l}{\ell}
\newcommand{\ddx}{\frac{d}{\d x}}

\newcommand{\zeroOverZero}{\ensuremath{\boldsymbol{\tfrac{0}{0}}}}
\newcommand{\inftyOverInfty}{\ensuremath{\boldsymbol{\tfrac{\infty}{\infty}}}}
\newcommand{\zeroOverInfty}{\ensuremath{\boldsymbol{\tfrac{0}{\infty}}}}
\newcommand{\zeroTimesInfty}{\ensuremath{\small\boldsymbol{0\cdot \infty}}}
\newcommand{\inftyMinusInfty}{\ensuremath{\small\boldsymbol{\infty - \infty}}}
\newcommand{\oneToInfty}{\ensuremath{\boldsymbol{1^\infty}}}
\newcommand{\zeroToZero}{\ensuremath{\boldsymbol{0^0}}}
\newcommand{\inftyToZero}{\ensuremath{\boldsymbol{\infty^0}}}


\newcommand{\numOverZero}{\ensuremath{\boldsymbol{\tfrac{\#}{0}}}}
\newcommand{\dfn}{\textbf}
%\newcommand{\unit}{\,\mathrm}
\newcommand{\unit}{\mathop{}\!\mathrm}
%\newcommand{\eval}[1]{\bigg[ #1 \bigg]}
\newcommand{\eval}[1]{ #1 \bigg|}
\newcommand{\seq}[1]{\left( #1 \right)}
\renewcommand{\epsilon}{\varepsilon}
\renewcommand{\iff}{\Leftrightarrow}

\DeclareMathOperator{\arccot}{arccot}
\DeclareMathOperator{\arcsec}{arcsec}
\DeclareMathOperator{\arccsc}{arccsc}
\DeclareMathOperator{\si}{Si}
\DeclareMathOperator{\proj}{proj}
\DeclareMathOperator{\scal}{scal}
\DeclareMathOperator{\cis}{cis}
\DeclareMathOperator{\Arg}{Arg}
%\DeclareMathOperator{\arg}{arg}
\DeclareMathOperator{\Rep}{Re}
\DeclareMathOperator{\Imp}{Im}
\DeclareMathOperator{\sech}{sech}
\DeclareMathOperator{\csch}{csch}
\DeclareMathOperator{\Log}{Log}

\newcommand{\tightoverset}[2]{% for arrow vec
  \mathop{#2}\limits^{\vbox to -.5ex{\kern-0.75ex\hbox{$#1$}\vss}}}
\newcommand{\arrowvec}{\overrightarrow}
\renewcommand{\vec}{\mathbf}
\newcommand{\veci}{{\boldsymbol{\hat{\imath}}}}
\newcommand{\vecj}{{\boldsymbol{\hat{\jmath}}}}
\newcommand{\veck}{{\boldsymbol{\hat{k}}}}
\newcommand{\vecl}{\boldsymbol{\l}}
\newcommand{\utan}{\vec{\hat{t}}}
\newcommand{\unormal}{\vec{\hat{n}}}
\newcommand{\ubinormal}{\vec{\hat{b}}}

\newcommand{\dotp}{\bullet}
\newcommand{\cross}{\boldsymbol\times}
\newcommand{\grad}{\boldsymbol\nabla}
\newcommand{\divergence}{\grad\dotp}
\newcommand{\curl}{\grad\cross}
%% Simple horiz vectors
\renewcommand{\vector}[1]{\left\langle #1\right\rangle}


\pgfplotsset{compat=1.13}

\outcome{Mapping properties of complex functions}

\title{2.4 Complex Mappings}

\begin{document}

\begin{abstract}
We determine the image of special sets.
\end{abstract}

\maketitle


\section{Linear Functions}
A linear function has the form $f(z) = az+b$ where $a, b \in \C$ and $a \neq 0$ (if $a=0$, then $f$ is a constant function).
The effect of a linear function on a set in the complex plane is to scale, rotate and translate that set.  The constant term $b$ acts as a translation vector, 
and the modulus and argument of $a$ are responsible for producing scaling and rotation, respectively.

\begin{example}[example 1]
Find the image of the unit square under the mapping $f(z) = (1+i)z + i$.\\
In this case, $a = 1+i$ which has modulus $\sqrt 2$ and argument $\pi/4$, and $b = i$.
The effect on the square of multiplying by $1+i$ is to scale the square by a factor of $\sqrt 2$, rotate counter-clockwise by $\pi/4$ and then translate it
by $i$. Due to the order of operations, the scaling and the rotation are done first (in either order), and the translation is done last.


\begin{image}
\begin{tikzpicture}
\draw[<->, thick] (-3,0)--(3,0) node[right]{x};

\draw[<->, thick] (0, -3) node[right]{$z$-plane} --(0,3) node[right]{iy};

\draw[thick,blue, fill= blue!20] (0,0) -- (1.2,0) node[below, black]{1} -- (1.2,1.2) -- (0, 1.2) node[left, black]{i}  -- (0,0);

%\draw[mark=*,mark size=1pt,mark options={color=blue}] plot coordinates {(-0.5,.4)} node[right, blue]{$z_0 = x_0 + iy_0$};
%\draw[mark=*,mark size=1pt,mark options={color=blue}] plot coordinates {(-0.5,1.4)} node[right, blue]{$z_0+ 2\pi i$};


%\draw[mark=*,mark size=1pt,mark options={color=blue}] plot coordinates {(-0.5,-.6)} node[right, blue]{$z_0- 2\pi i$};
%\draw[mark=*,mark size=1pt,mark options={color=blue}] plot coordinates {(-0.5,-1.6)} node[right, blue]{$z_0- 4\pi i$};


\draw[<->, thick] (6,0)--(12,0)node[right]{u};



\draw[thick,blue, fill= blue!20] (9,.5) node[left, black]{i}-- (10,1.5) node[right, black]{$1+2i$} -- (9,2.5) node[left, black]{3i} -- 
(8, 1.5) node[left, black]{$-1+2i$}  -- (9,.5);
\draw[<->, thick] (9, -3) node[right]{$w$-plane}--(9,3) node[right]{iv};
%\draw[mark=*,mark size=1pt,mark options={color=blue}] plot coordinates {(8.5,1.3)} node[right, blue]{$w_0$};
%\draw[dashed] (7,0) -- (8.5, 1.3) node[above, midway]{$e^{x_0}$};
%\draw (7.4,0) arc (0:42:.4) node[midway, right]{$y_0$} ;
\draw[->, thick, blue] (3.5,1)--(5.5, 1) node[above,midway, blue]{$w = f(z)$};
\node at (4.5, -4){\large The effect of $f(z) = (1+i)z + i$ on the unit square};
\end{tikzpicture}
\end{image}


\end{example}


\begin{problem}(problem 1)
Find the image of the unit square under the given mapping:
\begin{align*}
i) &  \; f(z) = iz\\
ii) &  \; f(z) = 2z\\
iii) &  \; f(z) = z+1\\
iv) &  \; f(z) = (1-i)z +i
\end{align*}
\end{problem}

Here is a video solution of one part of problem 1, part iv:\\
\begin{foldable}
\youtube{vGQIDeIxNgs}
\end{foldable}

\section{Squaring}

If $z = r \cis \theta$, then $z^2 = r^2 \cis 2\theta$.  
Hence, the effect of squaring is to square the modulus and double the 
argument of a complex number, $z$.

\begin{example}[example 2]
Find the image of the sector of an annulus given by:
\[
S = \left\{r\cis \theta : \frac12 < r < 2, \;\frac{\pi}{6} < \theta < \frac{\pi}{2}\right\}
\]
under the mapping $f(z) = z^2$.\\
The moduli of the complex numbers in the region $S$ are between $\frac12$ and $2$. 
Hence, the moduli of their squares are between $\left(\tfrac12\right)^2$ and $2^2$. 
Additionally, the arguments of the points in $S$ are between $\tfrac{\pi}{6}$ and $\tfrac{\pi}{2}$,
and thus, the arguments of their images are between $\tfrac{\pi}{3}$ and $\pi$. 
Putting these facts together yields the image, $f(S)$, which is also a sector of an annulus:
\[
f(S) = \left\{r\cis \theta : \frac14 < r < 4, \;\frac{\pi}{3} < \theta < \pi \right\}
\]


\begin{image}
\begin{tikzpicture}
\draw[dashed, blue, fill=blue!20] (90:.75) node[left]{$\frac{i}{2}$} arc (90:30:.75) coordinate (alpha) -- (30:2.25) arc (30:90:2.25) node[left]{$2i$} -- cycle;
\draw[->, thin, black] (1.5,0) arc (0:30:1.5) node[right, midway]{$\pi/6$};
\draw[<->, thick] (-3,0)--(3,0) node[right]{$x$};

\draw[<->, thick] (0, -3) node[right]{$z$-plane} --(0,3) node[right]{$iy$};

\draw[dashed, blue, fill=blue!20, shift ={(9,0)}] (180:.5) node[below]{$-\frac{1}{4}$} arc (180:60:.5) 
coordinate (alpha) -- (60:2.5) arc (60:180:2.5) node[below]{$-4$} -- cycle;


\draw[->, thin, black, shift={(9,0)}] (1.5,0) arc (0:60:1.5) node[right, midway]{$\pi/3$};

\draw[<->, thick] (6,0)--(12,0)node[right]{$u$};


\draw[<->, thick] (9, -3) node[right]{$w$-plane}--(9,3) node[right]{$iv$};

\draw[->, thick, blue] (3.5,1)--(5.5, 1) node[above,midway, blue]{$w = z^2$};
\node at (4.5, -4){\large The effect of $f(z) = z^2$ on a sector of an annulus};
\end{tikzpicture}
\end{image}

\end{example}


\begin{problem}(problem 2)
Find the image of the given set under the mapping $f(z) = z^2$:
\begin{align*}
i) & \; S = \left\{r\cis \theta : 1 < r < 2, \;0 < \theta < \frac{\pi}{4}\right\} \\
ii) &  \; S = \left\{r\cis \theta : 2 < r < 3, \;\frac{\pi}{2} < \theta < \frac{3\pi}{4}\right\}\\
iii) & \;  \mbox{Upper Half Plane}\; = \left\{ x+iy: y> 0\right\}\\
iv) &  \; \mbox{Quadrant IV}\; = \left\{ x+iy: x> 0, y< 0\right\}
\end{align*}
\end{problem}

Here is a video solution of one part of problem 2, part iv:\\
\begin{foldable}
\youtube{DWtxTPhwcfo}
\end{foldable}

\begin{example}[example 3]
Find the image of vertical line $\Rep z = 1$ under the mapping $f(z) = z^2$.\\
Instead of a polar analysis, we will use rectangular coordinates with $x = 1$. We have
\[
w = u+iv = (1+iy)^2 = (1-y^2) + 2iy
\]
Thus $u = 1-y^2$ and $v = 2y$, where $y \in R$. Eliminating $y$ gives
\[
u= 1-\frac{v^2}{4}
\]
which is a parabola in the $uv$-plane which opens to the left and has vertex at $(1,0)$.


\begin{image}
\begin{tikzpicture}

\draw[<->, thick] (-3,0)--(3,0) node[right]{$x$};

\draw[<->, thick] (0, -3) node[right]{$z$-plane} --(0,3) node[right]{$iy$};

\draw[blue, thick, <->] (1, -2.5) -- (1, 2.5) ;
\node[blue] at (1.6, 1){$x = 1$};

\draw[<->,thick, blue, rotate=-90, shift={(0,9)}] (-3,-1.25) parabola bend (0,1) (3,-1.25) ;
\node[blue] at (11,1){$u = 1-\frac{v^2}{4}$};

\draw[<->, thick] (6,0)--(12,0)node[right]{$u$};


\draw[<->, thick] (9, -3) node[right]{$w$-plane}--(9,3) node[right]{$iv$};

\draw[->, thick, blue] (3.5,1)--(5.5, 1) node[above,midway, blue]{$w = z^2$};
\node at (4.5, -4){\large The effect of $f(z) = z^2$ on a vertical line};
\end{tikzpicture}
\end{image}

\end{example}


\begin{problem}(problem 3)
Find the image of the given line under the mapping $f(z) = z^2$:
\begin{align*}
i) &  \; \Rep z = 2 \\
ii) & \;  \Rep z = -1\\
iii) & \;  \Imp z = 1\\
iv) &  \; \Imp z = -2
\end{align*}
\end{problem}

Here is a video solution of one part of problem 3, part iv:\\
\begin{foldable}
\youtube{gDtj9E8rkpA}
\end{foldable}

\section{Conjugates and Reciprocals}
The function $f(z) = \overline{z}$ maps a set in the complex plane to its mirror image in the real axis.

\begin{example}[example 4]
Find the image of the closed disk $D(z_0, r)$ under the mapping $f(z) = 2\overline{z}$.\\
The image of the center of the disk is $f(z_0) = 2\overline{z_0}$ and so the image of the disk
$D(z_0, r)$ is the disk of twice the radius, $D(2\overline{z_0}, 2r)$.

\begin{image}
\begin{tikzpicture}

\draw[<->, thick] (-3,0)--(3,0) node[right]{$x$};

\draw[<->, thick] (0, -3) node[right]{$z$-plane} --(0,3) node[right]{$iy$};

\draw[dashed, blue, fill = blue!20] (1.25, 1.25) circle (0.75) ;
\draw[mark=*,mark size=1pt,mark options={color=blue}] plot coordinates {(1.25, 1.25)} node[left, blue]{$z_0$};
\draw[blue] (1.25, 1.25)--(1.78, 1.78) node[right, midway]{$r$};


\draw[dashed, blue, fill= blue!20] (11.5, -2.5)circle (1.5) ;
\draw[mark=*,mark size=1pt,mark options={color=blue}] plot coordinates {(11.25, -2.5)} node[left, blue]{$2\overline{z_0}$};
\draw[blue] (11.25, -2.5)--(12.36, -1.33) node[right, midway]{$2r$};

\draw[->, thick, blue] (3.5,1)--(5.5, 1) node[above,midway, blue]{$w = 2\overline{z}$};
\draw[<->, thick] (6,0)--(12,0)node[right]{$u$};


\draw[<->, thick] (9, -3) node[left]{$w$-plane}--(9,3) node[right]{$iv$};


\node at (4.5, -4){\large The effect of $f(z) = 2\overline{z}$ on a disk};
\end{tikzpicture}
\end{image}

\end{example}


\begin{problem}(problem 4)
Find the image of the disk $D(1+i, 1)$ under the given mapping:
\begin{align*}
i) & \;  f(z) = \overline{z} \\
ii) &  \; f(z) = -2\overline{z} \\
iii) & \;  f(z) = 3i\overline{z} \\
iv) & \;  f(z) = i+\overline{z} 
\end{align*}
\end{problem}

The function $f(z) = \frac{1}{z}$ maps the punctured complex plane, $\C \setminus \{0\}$, to itself. Moreover, it is a bijection whose inverse is itself. 
To understand the mapping properties of $f(z) = \frac{1}{z}$ it is useful to 
rewrite it in the following way:
\[
\frac{1}{z} = \frac{1}{z} \cdot \frac{\overline{z}}{\overline{z}}= \frac{\overline{z}}{|z|^2}
\]
Thus the mapping $f(z) = \frac{1}{z}$ has both a reflecting property like the 
conjugate mapping and a scaling property
due to the division of the positive real number $|z|^2$. 
As a result of the scaling, the image of a point inside the unit circle 
will map to a point outside the circle and vice versa. This is shown in the figure below.

\begin{image}
\begin{tikzpicture}


\begin{scope}
\draw[clip, draw = none] (1,0) arc (0:180:2) --cycle;
\draw[outer color = blue!50, inner color = white, shading = radial] (-1,0) circle (2);
%\draw[ color = blue!50,shading = radial] (-2,0) circle (2);
\end{scope}

 

 \begin{scope}
\draw[clip, draw = none] (1,0) arc (0:-180:2) --cycle;
\draw[outer color = red!50, inner color = white] (-1,0) circle (2);
\end{scope}

\draw[<->, thick] (-4,0)--(2,0) node[right]{$x$};
\draw[<->, thick] (-1, -3) node[right]{$z$-plane} --(-1,3) node[right]{$iy$};

\begin{scope}
\draw[clip, draw = none] (6.2,0) rectangle (11.8,2.8);
\filldraw[draw = none, inner color = red, outer color = red!10, shading = radial] (6.2,-2.8) rectangle (11.8,2.8) ;
\end{scope}

\begin{scope}
\draw[clip, draw = none] (6.2,-2.8) rectangle (11.8,0);
\filldraw[draw = none, inner color = blue, outer color = blue!10, shading = radial] (6.2,-2.8) rectangle (11.8,2.8) ;
\end{scope}

\draw[dashed, fill = white] (9,0)circle (1) ;
\draw[white, thick] (6.2, -2.8) rectangle (11.8, 2.8);
\draw[thick, color = blue!20] (6.5,0)--(8,0);
\draw[thin, color = blue!20] (10,0)--(11.3,0);

\draw[<->, thick] (6,0)--(12,0)node[right]{$u$};


\draw[<->, thick] (9, -3) node[right]{$w$-plane}--(9,3) node[right]{$iv$};

\draw[->, thick, blue] (3,1)--(5, 1) node[above,midway, blue]{$w = \frac{1}{z}$};

\draw[dashed, fill = white] (-1,0)circle (.05) ;
\draw[dashed] (-1,0)circle (2) ;

\node at (4, -4){\large The effect of $f(z) = \frac{1}{z}$ on the open punctured unit disk};

\draw[mark=*,mark size=1pt,] plot coordinates {(-.3, 1.25)} ;
\node at (-.45,1) {$z_0$};

\draw[mark=*,mark size=1pt] plot coordinates {(9.8, -1.4)} ;
\node at (10.2,-1.4) {$\frac{1}{z_0}$};

\end{tikzpicture}
\end{image}




The color coding in the figure is intended to convey reflection in the real axis. A blue point in the upper half plane
inside the disk maps to a blue point in the lower half plane outside the disk. Similarly, a red point in the lower half plane inside the disk maps to a
red point outside the disk.

\begin{example}[example 5]
Find the image of the vertical ray $\Rep z = \frac12, \Imp z > 0$ under the mapping $f(z) = \frac{1}{z}$.\\
The equation $w = 1/z$ is equivalent to the equation $z = 1/w$ where $w$ and $z$ are both non-zero.
We write $1/w$ in rectangular form:
\[
z = \frac{1}{w} = \frac{\overline{w}}{|w|^2} = \frac{u-iv}{u^2 + v^2}
\]
Thus,
\[
x = \frac{u}{u^2+v^2} \quad \text{and} \quad y = -\frac{v}{u^2+v^2}
\]
Later, we will use the fact that $v$ is negative since $y$ is positive. First, we substitute $x = 1/2$ into the first equation, yielding:
\[
\frac12 = \frac{u}{u^2+v^2}
\]
Rearranging gives:
\[
u^2 - 2u + v^2 = 0
\]
Finally, we complete the square by adding $1$ to both sides to obtain:
\[
(u-1)^2 + v^2 = 1
\]
This is the equation of a circle in the $w$-plane with center at $(1,0)$ and radius $1$, i.e., $C(1,1)$.

Recalling that $v <0$,
we conclude that the image of the ray $x = 1/2, y >0$ is a semicircle in the lower half plane with 
center at $1$ and radius $1$. See the figure below.


\begin{image}
\begin{tikzpicture}

%z-plane
\draw[<->, thick] (-2.5,0)--(2.5,0) node[right]{$x$}; %real axis
\draw[<->, thick] (0, -2.5) node[right]{$z$-plane} --(0,2.5) node[left]{$iy$}; %imaginary axis

\draw[blue, thick, ->] (0.5, 0) -- (0.5, 2.5) ; %domain
\draw[thick,blue, fill = white] (0.5,0) circle (0.07) ;
\node[blue, thick] at (1.5, 1.5){$x = 1/2, \;y > 0$}; %domain label

\draw[->, thick, blue] (3,-1.5)--(5, -1.5) node[above,midway, blue]{$w = \frac{1}{z}$};
%w-plane
\draw[<->, thick] (5,0)--(10,0)node[right]{$u$}; %real axis
\draw[<->, thick] (7.5, -2.5) node[right]{$w$-plane}--(7.5,2.5) node[left]{$iv$}; %imaginary axis

\draw[blue, thick, ->] (9.5,0) arc (360:180:1); %image
\draw[thick,blue, fill = white] (9.5,0) circle (0.07) ;
\node[blue, thick] at (10, -1.5){$(u-1)^2 + v^2 = 1, v< 0$}; %image label

\node at (3.75, -3.5){\large The effect of $f(z) = \frac{1}{z}$ on a vertical ray}; %caption
\end{tikzpicture}
\end{image}

\end{example}


\begin{problem}(problem 5)
Find the image of the following sets under the mapping $f(z) = \frac{1}{z}$:
\begin{align*}
i)\; & \Imp z = \frac13, \Rep z < 0 \\
ii)\; & \Rep z = 2, \Imp z \leq 0 \\
iii)\; & \Rep z + \Imp z = 1 \\
iv)\; & \mbox{The ray:} \; \theta = \theta_0
\end{align*}
\end{problem}

Here is a video solution of problem 5, part iii:\\
\begin{foldable}
\youtube{feXi4bWWHi0}
\end{foldable}

\subsection{Stereographic Projection}


The mapping $f(z) = \frac{1}{z}$ in some sense inverts the unit disk, with the singularity at the origin corresponding to a ring of points far from 0.
In the complex plane this entire ring is considered to be ``$\infty$". Stereographic projection enables us to identify the complex plane with a punctured 
sphere $S\setminus N$ where $N$ is the north pole of the sphere (sorry Santa) and further, to identify ``$\infty$" with the north pole, $N$.
%(North Pole, like batteries, not included).

At the heart of stereographic projection lies a one-to-one, onto map $\varphi: \C\to S\setminus N $ where $S$ is the sphere with 
center at $\left(0,0,\tfrac12\right)$ and radius $\tfrac12$, so that $N = (0,0,1)$. 




\begin{image}
\begin{tikzpicture}

\draw (0,2) circle (2);
%\draw (-2,2) .. controls (0, 1.5) .. (2,2);
%\draw[dashed] (2,2) -- (-2,2);
\begin{scope}
\draw[clip, draw = none] (-2,2) rectangle (2,-4);
\draw (0,2) ellipse (2 and 0.5);
\end{scope}

\begin{scope}
\draw[clip, draw = none] (-2,2) rectangle (2,4);
\draw[dashed] (0,2) ellipse (2 and 0.5);
\end{scope}

\draw (-5,-1) -- (-3,1) --(5,1) -- (3,-1) --cycle;
\node at (0.3, 4.3) {$N$};
\node at (.2, -.2) {$0$};
\draw (0, 4) -- (-2, -0.5) node[below right] {$z_0$};
\node at (4.2, -.3) {$\C$};
\node at (2.2, 3.3) {$S\setminus N$};
\draw[fill=white] (0,4) circle (0.03);
%\draw[mark=*,mark size=1pt,] plot coordinates {(0, 4)} ;
\draw[mark=*,mark size=1pt,] plot coordinates {(0, 0)} ;
\draw[mark=*,mark size=1pt,] plot coordinates {(-2, -0.5)} ;
\draw[mark=*,mark size=1pt,red] plot coordinates {(-1.08, 1.58)}  ;
\node[red] at (-0.7, 1.7) {$w_0$};
\node at (0, -1.7) {\Large Stereographic Projection: $\varphi (z_0) = w_0$};
\draw[->] (0,0) -- (0, 5);
\end{tikzpicture}
\end{image}



To find the precise formula for the mapping $\varphi$, recall that a line segment in three dimensions can be expressed in vector form as
\[
\gamma(t) = {\bf u} + t{\bf v},  \; 0 \leq t \leq 1
\]

This segment goes from ${\bf u}$ to ${\bf u} + {\bf v}$ as $t$ goes from $0$ to $1$:

\begin{image}
\begin{tikzpicture}

\draw[thin, dashed, blue!80, ->] (0,0) -- (2, 2) node[midway, left] {${\bf u}$};
\draw[blue, ->, thick] (2,2) -- (5,3) node[midway, blue, above] {$\bf{v}$};
\draw[mark=*,mark size=1pt, blue!80] plot coordinates {(0, 0)} node[below left]{$0$} ;
\draw[mark=*,mark size=1pt, blue] plot coordinates {(2, 2)} ;
\draw[mark=*,mark size=1pt, blue] plot coordinates {(5, 3)} ;
\draw[mark=*,mark size=1pt,red] plot coordinates {(3, 2.33)} node[below right] {$\gamma(t) = {\bf u}+ t{\bf v}$} ;
\node at (2.5, -0.8){The segment from ${\bf u} \;\; \mbox{to}\;\; {\bf u}+{\bf v}$};
\end{tikzpicture}
\end{image}

In our case, the vector ${\bf u}$ is $(x,y,0)$ and the vector ${\bf v}$ goes from $(x, y, 0)$ to $(0, 0, 1)$, so ${\bf v} = (-x, -y, 1)$. 
Now, we wish to find $t$ so that the point
\[
{\bf u}+ t{\bf v} = (x- tx,y- ty, t) = (x(1-t), y(1-t), t)
\]
lies on the sphere, $S$, given by

\[
x^2 + y^2 + \left(z-\frac12 \right)^2 = \left(\frac12\right)^2
\]
Substituting gives
\[
x^2(1-t)^2 + y^2(1-t)^2 + \left(t-\frac12 \right)^2 = \left(\frac12\right)^2
\]
which is equivalent to 
\[
x^2(1-t)^2 + y^2(1-t)^2 + t(t-1) = 0
\]
Dividing by $t-1$ gives
\[
x^2(t-1) + y^2(t-1) + t = 0
\]
Solving this linear equation for $t$ we have
\[
t= \frac{x^2 + y^2}{1+x^2+y^2}
\]
Noting that $\displaystyle 1-t = \frac{1}{1+x^2 +y^2}$, we can define $\varphi$:
\[
\varphi(x+iy)   = \left( \frac{x}{1+x^2 +y^2}, \frac{y}{1+x^2 +y^2}, \frac{x^2 + y^2}{1+x^2+y^2} \right)
\]

The bijection $\varphi$ can be extended naturally to include $\infty$ in its domain: $\varphi(\infty) = N$.

\begin{question} The mapping $\varphi$ is a bijection from $\C$ to the punctured sphere $S \setminus N$.
What is the formula for $\varphi^{-1} : S \setminus N \to \C$?
\end{question}



\section{Exponential and Logs}
Next, we turn to the exponential function, $f(z) = e^z$ and its mapping properties.
Recall that $e^z$ is $2\pi i$ periodic.  As a result, the image of the band of complex numbers satisfying
\(-\pi < \Imp z \leq \pi\) is its full image, $\C \setminus \{0\}$.

\begin{example}[example 6]
Find the image of the vertical segment $x = 2, -\pi < y \leq \pi$ under the mapping $f(z) = e^z$.\\
For $z$ on this segment, 
\[
e^z = e^2 \cis y
\]
and since $y$ takes on all values in an interval of length $2\pi$, $\cis y$ will take on all values on the unit circle.
Hence, the image of the segment is the circle centered at the origin with radius $e^2$.
\begin{image}
\begin{tikzpicture}

%z-plane
\draw[<->, thick] (-2.5,0)--(2.5,0) node[right]{$x$}; %real axis
\draw[<->, thick] (0, -2.5) node[left]{$z$-plane} --(0,2.5) node[left]{$iy$}; %imaginary axis

\draw[blue, thick] (1, -1.2) -- (1, 1.2) ; %domain
\draw[thick,blue, fill = white] (1,-1.2) circle (0.07) ;
\draw[thick,blue, fill = blue] (1, 1.2) circle (0.07) ;
\node[blue, thick] at (2, 1.5){$x = 2, -\pi < y \leq \pi$}; %domain label

\draw[->, thick, blue] (3,-1.5)--(5, -1.5) node[above,midway, blue]{$w = e^z$};
%w-plane
\draw[<->, thick] (5,0)--(10,0)node[right]{$u$}; %real axis
\draw[<->, thick] (7.5, -2.5) node[right]{$w$-plane}--(7.5,2.5) node[left]{$iv$}; %imaginary axis

\draw[blue, thick, ->] (7.5,0) circle (2.2); %image
\draw[dashed,thin,blue!80] (7.5,0) -- (9.1, 1.6) node[midway, below right]{$e^2$} ;
\node[blue, thick] at (10.2, -1.7){$u^2 + v^2 = e^4$}; %image label

\node at (3.75, -3.5){\large The effect of $f(z) = \frac{1}{z}$ on a vertical segment}; %caption
\end{tikzpicture}
\end{image}
An alternative method to arrive at the conclusion of the last example is to write
\[
u = e^2\cos  y \quad \mbox{and} \quad v = e^2 \sin y
\]
and use the Pythagorean trigonometric identity $\cos^2 \theta + \sin^2 \theta= 1$ to obtain:
\[
u^2 + v^2 = (e^2)^2 = e^4
\]
To see that the image is this entire circle, note that since $-\pi < y \leq \pi$ we have
$-e^2 \leq u \leq e^2$ and similarly for $v$.
\end{example}


\begin{problem}(problem 6)
Find the image of the given line segment under the map $w = e^z$:
\begin{align*}
i)\; & \Rep z = 1,\; 0<\Imp z < \pi \\
ii)\; & \Rep z = -1,\; \pi/2 < \Imp z < 3\pi/2
\end{align*}
\end{problem}

Here is a video solution of one part of problem 6, part ii:\\
\begin{foldable}
\youtube{V7-6_7eb9JE}
\end{foldable}

\begin{example}[example 7]
Find the image of the horizontal line $y = \frac{\pi}{4}$ under the mapping $f(z) = e^z$.\\
For $z$ on this horizontal line, 
\[
e^z = e^x \cis \left(\frac{\pi}{4}\right)
\]
where $-\infty < x < \infty$. Such a point lies on the polar ray $\theta = \pi/4$ at a distance $e^x$ units from the origin.
Since $e^x$ goes from $0$ to $\infty$ as $x$ goes from $-\infty$ to $\infty$, the image of the horizontal line is the polar
ray, $\theta = \pi/4$ with $r>0$.

\begin{image}
\begin{tikzpicture}

%z-plane
\draw[<->, thick] (-2.5,0)--(2.5,0) node[right]{$x$}; %real axis
\draw[<->, thick] (0, -2.5) node[right]{$z$-plane} --(0,2.5) node[left]{$iy$}; %imaginary axis

\draw[blue, thick, <->] (-2, 1.6) -- (2, 1.6) ; %domain

\node[blue, thick] at (1.5, 1.1){$ y = \pi/4$}; %domain label

\draw[->, thick, blue] (3,-1.5)--(5, -1.5) node[above,midway, blue]{$w = e^z$};
%w-plane
\draw[<->, thick] (5,0)--(10,0)node[right]{$u$}; %real axis
\draw[<->, thick] (7.5, -2.5) node[right]{$w$-plane}--(7.5,2.5) node[left]{$iv$}; %imaginary axis

\draw[blue, thick, ->] (7.5,0) -- (9.5,2); %image
\draw[thick,blue, fill = white] (7.5,0) circle (0.07) ;
\node[blue, thick] at (10, .8){$\theta = \pi/4, r>0$}; %image label

\node at (3.75, -3.5){\large The effect of $f(z) = e^z$ on a horizontal line}; %caption
\end{tikzpicture}
\end{image}
An alternative method to arrive at the conclusion of the last example is to write
\[
u = e^x\cos\left(\frac{\pi}{4}\right) = \frac{e^x}{\sqrt 2} = e^x\sin\left(\frac{\pi}{4}\right) = v
\]
where $u$ and $v$ are both strictly positive. This gives the portion of the line $u = v$ in the first quadrant.

\end{example}



\begin{problem}(problem 7)
Find the image of the line $\Imp z = \pi/3$ under the map $w = e^z$.
\end{problem}

The next example involves the principal logarithm which we recall is defined by:
\[
\Log z = \ln|z| + i\Arg z
\]
\begin{example}[example 8]
Find the image of the semicircle $z = 3\cis \theta, \; \pi/2 < \theta \leq 3\pi/2$ 
under the mapping $f(z) = \Log z$.\\
On the given semicircle, $|z| = 3$, so 
\[
w = \Log z = \ln 3 + i\Arg \theta
\]
Thus $u = \ln 3$ and so the image lies on a vertical line. By definition of $\Log z$, 
the values of $v$ must lie in the interval $\left(-\pi, \pi\right]$.
For $\pi/2 <\theta\leq \pi$ we have $\pi/2 < \Arg z \leq  \pi$ and for $\pi < \theta \leq 3\pi/2$ we have $-\pi/2 \geq \Arg z  > -\pi$.
Hence, the range of values for $v$ is $(\pi/2, \pi] \cup [-\pi/2, -\pi)$


\begin{image}
\begin{tikzpicture}

%z-plane
\draw[<->, thick] (-2.5,0)--(2.5,0) node[right]{$x$}; %real axis
\draw[<->, thick] (0, -2.5) node[right]{$z$-plane} --(0,2.5) node[left]{$iy$}; %imaginary axis

\draw[blue, thick, ->] (0,1.5) node[right]{$3i$} arc (90:140:1.5) ; %domain
\draw[blue, thick] (-1.0605, 1.0605) arc (135:180:1.5) node[anchor = north east]{$-3$}; %domain
\draw[red, thick, ->] (-1.5,0) arc (180:230:1.5); %domain
\draw[red, thick] (-1.0605,-1.0605) arc (225:270:1.5) node[right]{$-3i$}; %domain

\draw[thick,blue, fill = white] (0,1.5) circle (0.04) ;
\draw[thick,blue, fill = blue] (-1.5,0) circle (0.04) ;
\draw[thick,red, fill=red] (0,-1.5) circle (0.04) ;
\node[blue, thick] at (-1.5, 1.6){$x^2 + y^2 = 9$}; %domain label

\draw[->, thick, blue] (3,-1.5)--(5, -1.5) node[above,midway, blue]{$w = \Log z$};

%w-plane
\draw[<->, thick] (5,0)--(10,0)node[right]{$u$}; %real axis
\draw[<->, thick] (7.5, -2.5) node[right]{$w$-plane}--(7.5,2.5) node[left]{$iv$}; %imaginary axis

\draw[blue, thick, ->] (8.4, 1) -- (8.4, 1.6) ; %image
\draw[blue, thick] (8.4, 1.5) -- (8.4, 2) node[right]{$u = \ln 3$}; %image
\draw[thick,blue, fill=blue] (8.4,2) circle (0.04) ;
\draw[thick,blue, fill=white] (8.4,1) circle (0.04) ;

\draw[red, thick, ->] (8.4, -2) -- (8.4, -1.4) ; %image
\draw[red, thick] (8.4, -1.5) -- (8.4, -1) ; %image
\draw[thick,red, fill=red] (8.4,-1) circle (0.04) ;
\draw[thick,red, fill=white] (8.4,-2) circle (0.04) ;
%\node[red, thick] at (9.5, 1.5){$x = 1, y > 0$}; %image label

\draw[thin] (7.6, 2) -- (7.4, 2) node[left]{$\pi i$};
\draw[thin] (7.6, -2) -- (7.4, -2) node[left]{$-\pi i$};



\node at (3.75, -3.5){\large The effect of $f(z) = \Log z $ on a semi-circle}; %caption
\end{tikzpicture}
\end{image}


\end{example}

\begin{problem}(problem 8)
Find the image of the given set under the map $w = \Log z$:
\begin{align*}
i)\; & \mbox{The unit circle,}\; C(0,1) \\
ii)\; & \mbox{The ray}\; \theta = 7\pi/4
\end{align*}
\end{problem}

Here is a video solution of one part of problem 8, part ii:\\
\begin{foldable}
\youtube{C6C_B5RKSAg}
\end{foldable}

Below is a Geogebra applet that shows the image of a line under $1/z, e^z$ and $z^2$.  
Move the red line and observe the effects! 
(The Geogebra activity can be accessed by going to geogebra.org and typing
Ptnx5Z3m into the search bar.)
\begin{center}
\geogebra{Ptnx5Z3m}{600}{600}
\end{center}


Below is a Geogebra applet that shows the image of a circle under $1/z, e^z$ and $z^2$.  
Move the red circle and observe the effects! 
(The Geogebra activity can be accessed by going to geogebra.org and typing WuXRYxUm into the search bar.)
\begin{center}
\geogebra{WuXRYxUm}{600}{600}
\end{center}




\section{Sine and Cosine}

\begin{example}[example 9]
Find the image of the horizontal line $\Imp z =1$ under the mapping $f(z) = \cos z$.\\
Recall
\[
\cos z = \cos x \cosh y - i \sin x \sinh y
\]
Writing $w = u+iv = \cos z$ and substituting $y = 1$ gives
\[
u = \cos x \cosh 1 \quad \mbox{and} \quad v = -\sin x \sinh 1
\]
We square $u$ and $v$ to take advantage of the Pythagorean Identity,
\[
\frac{u^2}{\cosh^2 1} + \frac{v^2}{\sinh^2 1} = \cos^2 x + \sin^2 x = 1
\]
This is the equation of an ellipse with $u$-intercepts $u = \pm \cosh 1$
and $v$-intercepts $v = \pm \sinh 1$.


\begin{image}
\begin{tikzpicture}

\draw[<->, thick] (-3,0)--(3,0) node[right]{$x$};

\draw[<->, thick] (0, -3) node[right]{$z$-plane} --(0,3) node[right]{$iy$};

\draw[blue, thick, <->] (-2.5, 1) -- (2.5, 1) ;
\node[blue] at (1, 1.6){$y = 1$};

\draw[thick, blue] (9,0) ellipse (1.54 and 1.17);


\node[blue] at (11.5,2){$\frac{u^2}{\cosh^2 1} + \frac{v^2}{\sinh^2 1} = 1$};

\draw[<->, thick] (6,0)--(12,0)node[right]{$u$};


\draw[<->, thick] (9, -3) node[right]{$w$-plane}--(9,3) node[right]{$iv$};

\draw[->, thick, blue] (3.5,-1.5)--(5.5, -1.5) node[above,midway, blue]{$w = \cos z$};
\node at (4.5, -4){\large The effect of $f(z) = \cos z$ on a horizontal line};
\end{tikzpicture}
\end{image}


\end{example}


\begin{problem}(problem 9)
Find the image of the given line under the given map:
\begin{align*}
i)\; & \Rep z = \pi/4; \; \cos z \\
ii)\; & \Imp z = -2; \; \cos z \\
iii)\; & \Rep z = \pi/6; \; \sin z\\
iv)\; & \Imp z = 1; \; \sin z
\end{align*}
\end{problem}

Here is a video solution of one part of problem 9, part iii:\\
\begin{foldable}
\youtube{nBjfyuz5LS8}
\end{foldable}

\end{document}







