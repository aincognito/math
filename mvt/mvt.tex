\documentclass[handout]{ximera}

%% You can put user macros here
%% However, you cannot make new environments



\newcommand{\ffrac}[2]{\frac{\text{\footnotesize $#1$}}{\text{\footnotesize $#2$}}}
\newcommand{\vasymptote}[2][]{
    \draw [densely dashed,#1] ({rel axis cs:0,0} -| {axis cs:#2,0}) -- ({rel axis cs:0,1} -| {axis cs:#2,0});
}


\graphicspath{{./}{firstExample/}}
\usepackage{forest}
\usepackage{amsmath}
\usepackage{amssymb}
\usepackage{array}
\usepackage[makeroom]{cancel} %% for strike outs
\usepackage{pgffor} %% required for integral for loops
\usepackage{tikz}
\usepackage{tikz-cd}
\usepackage{tkz-euclide}
\usetikzlibrary{shapes.multipart}


%\usetkzobj{all}
\tikzstyle geometryDiagrams=[ultra thick,color=blue!50!black]


\usetikzlibrary{arrows}
\tikzset{>=stealth,commutative diagrams/.cd,
  arrow style=tikz,diagrams={>=stealth}} %% cool arrow head
\tikzset{shorten <>/.style={ shorten >=#1, shorten <=#1 } } %% allows shorter vectors

\usetikzlibrary{backgrounds} %% for boxes around graphs
\usetikzlibrary{shapes,positioning}  %% Clouds and stars
\usetikzlibrary{matrix} %% for matrix
\usepgfplotslibrary{polar} %% for polar plots
\usepgfplotslibrary{fillbetween} %% to shade area between curves in TikZ



%\usepackage[width=4.375in, height=7.0in, top=1.0in, papersize={5.5in,8.5in}]{geometry}
%\usepackage[pdftex]{graphicx}
%\usepackage{tipa}
%\usepackage{txfonts}
%\usepackage{textcomp}
%\usepackage{amsthm}
%\usepackage{xy}
%\usepackage{fancyhdr}
%\usepackage{xcolor}
%\usepackage{mathtools} %% for pretty underbrace % Breaks Ximera
%\usepackage{multicol}



\newcommand{\RR}{\mathbb R}
\newcommand{\R}{\mathbb R}
\newcommand{\C}{\mathbb C}
\newcommand{\N}{\mathbb N}
\newcommand{\Z}{\mathbb Z}
\newcommand{\dis}{\displaystyle}
%\renewcommand{\d}{\,d\!}
\renewcommand{\d}{\mathop{}\!d}
\newcommand{\dd}[2][]{\frac{\d #1}{\d #2}}
\newcommand{\pp}[2][]{\frac{\partial #1}{\partial #2}}
\renewcommand{\l}{\ell}
\newcommand{\ddx}{\frac{d}{\d x}}

\newcommand{\zeroOverZero}{\ensuremath{\boldsymbol{\tfrac{0}{0}}}}
\newcommand{\inftyOverInfty}{\ensuremath{\boldsymbol{\tfrac{\infty}{\infty}}}}
\newcommand{\zeroOverInfty}{\ensuremath{\boldsymbol{\tfrac{0}{\infty}}}}
\newcommand{\zeroTimesInfty}{\ensuremath{\small\boldsymbol{0\cdot \infty}}}
\newcommand{\inftyMinusInfty}{\ensuremath{\small\boldsymbol{\infty - \infty}}}
\newcommand{\oneToInfty}{\ensuremath{\boldsymbol{1^\infty}}}
\newcommand{\zeroToZero}{\ensuremath{\boldsymbol{0^0}}}
\newcommand{\inftyToZero}{\ensuremath{\boldsymbol{\infty^0}}}


\newcommand{\numOverZero}{\ensuremath{\boldsymbol{\tfrac{\#}{0}}}}
\newcommand{\dfn}{\textbf}
%\newcommand{\unit}{\,\mathrm}
\newcommand{\unit}{\mathop{}\!\mathrm}
%\newcommand{\eval}[1]{\bigg[ #1 \bigg]}
\newcommand{\eval}[1]{ #1 \bigg|}
\newcommand{\seq}[1]{\left( #1 \right)}
\renewcommand{\epsilon}{\varepsilon}
\renewcommand{\iff}{\Leftrightarrow}

\DeclareMathOperator{\arccot}{arccot}
\DeclareMathOperator{\arcsec}{arcsec}
\DeclareMathOperator{\arccsc}{arccsc}
\DeclareMathOperator{\si}{Si}
\DeclareMathOperator{\proj}{proj}
\DeclareMathOperator{\scal}{scal}
\DeclareMathOperator{\cis}{cis}
\DeclareMathOperator{\Arg}{Arg}
%\DeclareMathOperator{\arg}{arg}
\DeclareMathOperator{\Rep}{Re}
\DeclareMathOperator{\Imp}{Im}
\DeclareMathOperator{\sech}{sech}
\DeclareMathOperator{\csch}{csch}
\DeclareMathOperator{\Log}{Log}

\newcommand{\tightoverset}[2]{% for arrow vec
  \mathop{#2}\limits^{\vbox to -.5ex{\kern-0.75ex\hbox{$#1$}\vss}}}
\newcommand{\arrowvec}{\overrightarrow}
\renewcommand{\vec}{\mathbf}
\newcommand{\veci}{{\boldsymbol{\hat{\imath}}}}
\newcommand{\vecj}{{\boldsymbol{\hat{\jmath}}}}
\newcommand{\veck}{{\boldsymbol{\hat{k}}}}
\newcommand{\vecl}{\boldsymbol{\l}}
\newcommand{\utan}{\vec{\hat{t}}}
\newcommand{\unormal}{\vec{\hat{n}}}
\newcommand{\ubinormal}{\vec{\hat{b}}}

\newcommand{\dotp}{\bullet}
\newcommand{\cross}{\boldsymbol\times}
\newcommand{\grad}{\boldsymbol\nabla}
\newcommand{\divergence}{\grad\dotp}
\newcommand{\curl}{\grad\cross}
%% Simple horiz vectors
\renewcommand{\vector}[1]{\left\langle #1\right\rangle}


\outcome{Learn the Mean Value Theorem and apply it to examples}

\title{3.5 Mean Value Theorem}

\begin{document}

\begin{abstract}
We apply the Mean Value Theorem.
\end{abstract}

\maketitle


\section{Rolle's Theorem}

We begin with a special case of the Mean Value Theorem known as Rolle's Theorem.
This theorem uses the Extreme Value Theorem to guarantee the existence of a critical number for a differentiable 
function on and interval in which the function is equal at the endpoints.

\begin{theorem}[Rolle's Theorem]
Suppose the function $f(x)$ is continuous on the closed interval $[a,b]$ and differentiable on the open interval $(a,b)$.
If $f(a) = f(b)$, then there exists a number $c$ in the open interval $(a,b)$ such that 
\[
f'(c) = 0.
\]
\end{theorem}

\begin{image}
\begin{tikzpicture}
\begin{axis}[axis x line=  center, axis y line = none, xmin= -2.5, xmax=2.5, xtick={-2, 0, 2}, xticklabels={$a$,$c$,$b$},
title={Rolle's Theorem for $f(x) $ on $[a,b]$}]
\addplot[domain=-2.3:2.3, 
    samples=100, color=black, thick]{-0.25*x^2 + 1.5 };
    \node at (axis cs: 1.8,1.25) {$y = f(x)$};
\addplot[smooth,mark=*,blue,dashed] plot coordinates {(-2,0.5)  (2,0.5)} node[below, midway]{$f(a) = f(b)$};
\addplot[domain=-1.2:1.2, 
    samples=100, color=red]{1.5};
\addplot[smooth,mark=*,red] plot coordinates {(0, 1.5)} node[above] {$f'(c) = 0$};
\addplot[domain=-2.5:2.5, 
    samples=100, color=black]{0};
\end{axis}
\end{tikzpicture}
\end{image}
If $f$ is a constant function on $[a,b]$, then $f'(x)= 0$ for all $x$ in the open interval $(a,b)$ and so the theorem is true in this case.
If $f$ is not constant on the closed interval $[a,b]$, the rationale for the verity of the theorem is more complicated.
We argue as follows.
Since $f$ is assumed to be continuous on the closed interval $[a,b]$, we can apply the Extreme Value Theorem (section 3.2) to conclude
that $f$ has absolute an absolute max and absolute min on $[a,b]$.
Furthermore, the hypotheses of the theorem require that $f(a) = f(b)$, so the endpoints of a non-constant function cannot be both the max and the min.
We have therefore deduced that 
$f$, must have at least one of its max or min in the open interval $(a,b)$.
By Fermat's Theorem (section 3.2), this absolute extreme occurs at a critical number, $x= c$. 
Finally, since $f$ is differentiable on $(a,b)$ (by hypothesis), we can conclude 
that $f'(c) = 0$. 
Thus, the theorem is true in this case as well, and Rolle's Theorem is proved.


\begin{example} [example 1] Use Rolle's theorem to show that the function $f(x) = \sin(x)$ has a critical number in the interval $(0, \pi)$.\\
The function $f(x) = \sin(x)$ is differentiable (and hence also continuous) on the interval $(-\infty, \infty)$ and since
\[
\sin 0 = \sin \pi = 0,
\]
we can apply Rolle's Theorem to $f$ on the interval $[0, \pi]$. The theorem states that there exists a number $c$ between $0$ and $\pi$ such that 
$f'(c) = 0$. That is, $f(x) = \sin(x)$ has a critical number in the interval $(0, \pi)$.
Moreover, we can find $c$ by solving the equation
\[
f'(x) = \cos(x) = 0
\]
This equation has infinitely many solutions: $x = \pi/2 + k\pi$ where $k$ is any integer.
The solution $x = \pi/2$ (corresponding to $k = 0$) is in the interval $(0, \pi)$.
\end{example}

\begin{problem} [problem 1] 
Use Rolle's theorem to show that the function $f(x) = x^2 - \sqrt x$ has a critical number in the interval $(0, 1)$.
Then find the value of this critical number.\\
The critical number is 
\[ 
x = \answer{1/\sqrt[3]{16}}
\]
\end{problem} 

For the next example we need to recall that a {\it root} of a polynomial, $p(x)$, is a value $x = a$, such that $p(a) = 0$.
We also need the fact that a quadratic polynomial has at most two roots, since if
\[
ax^2 + bx + c = 0
\]
then by the quadratic formula
\[
x = \frac{-b \pm \sqrt{b^2 - 4ac}}{2a}
\]

\begin{example}[example 2] Use Rolle's Theorem  to show that a cubic polynomial can have at most three roots.\\



Let $p(x)$ be a cubic polynomial, i.e., 
\[
p(x) = ax^3 + bx^2 + cx + d, \;\; \text{where} \;\; a\neq 0.
\]
We will argue by {\it contradiction} to demonstrate that $p(x)$ can have at most three roots.
Consider a cubic polynomial $p(x)$ with four roots.
Label these roots in ascending order: $x_1 <  x_2 < x_3 < x_4$.
By definition of root, we have
\[
p(x_1) = p(x_2) = p(x_3) = p(x_4) = 0
\]
We apply Rolle's Theorem to $p(x)$ on each of the three intervals $[x_1, x_2], [x_2, x_3]$ and $[x_3, x_4]$ to conclude that 
there are values $c_1 < c_2 < c_3$ in these respective intervals such that
\[
p'(c_1) = p'(c_2) = p'(c_3) = 0
\]
The above equations are telling us that $p'$ has three distinct roots.
However, since the degree of $p(x)$ is $3$, the Power Rule tells us that the degree of $p'(x)$ must be $2$.
The quadratic formula tells us that a quadratic polynomial can have at most two roots.
We see that the assumption that the cubic polynomial $p(x)$ has four roots has led to the absurdity that 
the quadratic polynomial $p'(x)$ has three roots.
This contradiction implies that a cubic polynomial having four roots cannot exist.
From this we conclude that a cubic polynomial can have at most three roots.

See the figure below which shows polynomial with four roots and the corresponding values, $c_1, c_2$ and $c_3$ given by Rolle's Theorem.

\begin{image}
\begin{tikzpicture}
\begin{axis}[axis x line=  center, axis y line = none, xmin= -0.5, xmax=6.5, xtick={0, 1, 2, 3, 4, 5, 6}, 
xticklabels={$x_1$,$c_1$,$x_2$,$c_2$,$x_3$,$c_3$,$x_4$}, title={The roots of  $p(x)$ are $x_1, x_2, x_3$ and $x_4$}] 
\addplot[domain=-0.5:6.5, 
    samples=100, color=black]{0};
\addplot [samples=100,smooth,domain=0:6] {sin(deg(1.57*x))} ;
\addplot[smooth,mark=*,blue, dashed] plot coordinates {(0,0)   (2,0)  (4,0) (6,0)} ;
\addplot[smooth,mark=*,red] plot coordinates {(1, 1)} ;
\addplot[smooth,mark=*,red] plot coordinates {(3, -1)} ;
\addplot[smooth,mark=*,red] plot coordinates {(5, 1)} ;
\addplot[domain=0.3:1.7, 
    samples=100, color=red]{1} node[above]{$p'(c_1) = 0$};
\addplot[domain=2.3:3.7, 
    samples=100, color=red]{-1} node[below]{$p'(c_2) = 0$};
\addplot[domain=2.9:3, color=white]{-1.4} ;
\addplot[domain=4.3:5.7, 
    samples=100, color=red]{1} node[above]{$p'(c_3) = 0$};
    \node at (axis cs:3, -1.75) {The roots of $p'(x)$ are at $c_1, c_2$ and $c_3$};
    \node at (axis cs:3, -1.75) {By Rolle's Theorem, $p'(x)$ has three roots};
    \addplot[smooth,mark=*,white] plot coordinates {(3, -2)} ;  
\end{axis}
\end{tikzpicture}
\end{image}

\end{example}

\begin{remark} The Fundamental Theorem of Algebra states that a polynomial of degree $n$ can have at most $n$ roots.
	The proof of this result requires \link[mathematical induction]{https://en.wikipedia.org/wiki/Mathematical_induction}.
\end{remark}

\begin{problem}[problem 2]
Use Rolle's Theorem and the fact that a polynomial of degree three has at most three 
roots to prove that a polynomial of degree four has at most four roots.
\begin{hint} 
	Use the argument in example 2
\end{hint}
\end{problem}

\section{The Mean Value Theorem}


The Mean Value Theorem is one of the most far-reaching theorems in calculus. It states that for a continuous 
and differentiable function, the average rate of change over an interval is attained as an 
instantaneous rate of change at some point inside the interval. The precise mathematical statement is as follows.\\

\begin{theorem}[Mean Value Theorem]
Suppose that the function $f(x)$ is continuous on the closed interval $[a,b]$ and differentiable on the 
open interval $(a,b)$. Then there exists a number $c$ between $x = a$ and $x = b$ such that
\[f'(c) = \frac{f(b) - f(a)}{b-a}\]

\end{theorem}



Geometrically, the left-hand side of the conclusion of the MVT represents the slope of the tangent line to $f(x)$ at $x = c$ 
Meanwhile, the right-hand side represents the slope of the secant line connecting the points $(a, f(a))$ and $(b, f(b))$. 
Since their slopes are equal, these two lines are parallel.
Conceptually, the left-hand side represents the instantaneous rate of change of $f(x)$ at $x = c$ while the
right-hand side represents the average rate of change of $f(x)$ over the interval from $x=a$ to $x=b$. 
Thus, the MVT says that at some point, the instantaneous rate of change will be equal to the average rate of change.

\begin{image}
\begin{tikzpicture}
\begin{axis}[axis x line=  center, axis y line = none, xtick={-1, 0.5, 2}, xticklabels={$a$,$c$,$b$}, 
legend pos=outer north east, title={MVT for $f(x) $ on $[a,b]$}]
\addplot[domain=-1.4:2.4, 
    samples=100, color=black, thick]{-x^2 + 5 };
\addplot[smooth,mark=*,blue, thin] plot coordinates {(-1,4)  (2,1)};
\addplot[domain=-1:2, 
    samples=100, color=blue, thin]{-x + 3};
\addplot[domain=-0.2:1.2, 
    samples=100, color=red, thin]{-x + 5.25 };
\addplot[smooth,mark=*,red] plot coordinates {(0.5, 4.75)};
\legend{$y = f(x)$, , secant line, tangent line, };
\addplot[dashed] coordinates{(-1,0)(-1,4)};
\addplot[dashed] coordinates{(0.5,0)(0.5,4.75)};
\addplot[dashed] coordinates{(2,0)(2,1)};
node[label=below]{some label};
\end{axis}
\end{tikzpicture}
\end{image}

As an example of the conceptual interpretation of the theorem, consider a car that averages a speed of, say, 47.4 miles per hour on a long trip. 
Because the distance traveled is a continuous and differentiable function of time, the Mean Value Theorem tells us that the car must have been 
traveling at exactly 47.4 miles per hour at some time during the trip.

The Mean Value Theorem is an {\it existence} theorem because it asserts that there exists at least one value $x=c$ inside the interval $(a,b)$
that satisfies the equation 
\[f'(c) = \frac{f(b) - f(a)}{b-a}.\]
In our examples, we will determine this special value. 

\begin{example}[example 3]
Verify that the function $f(x) = x^2$ satisfies the hypotheses of the Mean Value Theorem
on the interval $[0,3]$. Then find all values of $x$ that satisfy the conclusion of the theorem.\\
Since $x^2$ is a polynomial, it is continuous and differentiable everywhere. Therefore, $f$ is continuous on the closed interval $[0, 3]$ and differentiable on the open interval $(0, 3)$. 
Thus, $f(x) = x^2$ satisfies the hypotheses of the MVT on $[0, 3]$. We now know that the equation
\[
f'(x) = \frac{f(b) - f(a)}{b-a}
\]
has at least one solution in the interval $(0, 3)$. To find the solution(s), 
we first compute the value of the right-hand side using $a = 0$ and $b = 3$:
\[\frac{f(b) - f(a)}{b-a} = \frac{f(3) - f(0)}{3-0} \]
\[= \frac{3^2 - 0^2}{3}= \frac{9 - 0}{3} = 3.\]
Next, we compute the derivative, $f'(x) = 2x$.
The conclusion of the MVT guarantees that the equation
\[2x = 3\]
has at least one solution in the open interval $(0,3)$.
This is easily verified, since if $2x = 3$, then 
\[x = \frac{3}{2},
\]
which is in $(0, 3)$.
Furthermore, $x 3/2$ is the mid-point of the interval $(0, 3)$.
When applying the MVT to a quadratic polynomial on any interval $[a, b]$ the value of $x$ that satisfies the theorem will always be the mid-point!

In the figure below, the secant line is in blue, 
and the tangent line at $x = 3/2$ is in red. The Mean Value Theorem asserts that these lines are parallel, and the figure makes this clear.

\begin{image}
\begin{tikzpicture}
\begin{axis}[axis lines = center, legend pos=outer north east, title={MVT for $f(x) = x^2$ on $[0,3]$}]
\addplot[domain=0:3, 
    samples=100, color=black, thick]{x^2};
\addplot[smooth,mark=*,blue, thin] plot coordinates {(0,0)  (3,9)};
\addplot[domain=0:3, 
    samples=100, color=blue, thin]{3*x};
\addplot[domain=0.8:2.2, 
    samples=100, color=red, thin]{3*x - 2.25 };
\addplot[smooth,mark=*,red] plot coordinates {(1.5, 2.25)};
\legend{$y=x^2$, , \text{secant line}, \text{tangent line}, }
\end{axis}
\end{tikzpicture}
\end{image}


\end{example}

\begin{problem}(problem 3a)
  Is the function $f(x) = x^2 + 2$ continuous on the closed interval $[1,3]$ and differentiable on the open interval $(1,3)$? Why?
  If so, find all values of $x$ in $(1,3)$ that satisfy the conclusion of the Mean Value Theorem for $f$ on the interval $[1,3]$.
	
    \begin{hint}
      Compute $f'(x)$ and $\dfrac{f(3) - f(1)}{3-1}$
    \end{hint}
		\begin{hint}
		  Solve the equation: $f'(x) = \dfrac{f(3) - f(1)}{3-1}$ for $x$
		\end{hint}
		
		The value of $x$ is:
		 $\answer{2}$
     What do you notice about this value of $x$ in relation to the given interval? Is this always the case?
\end{problem}

\begin{problem}(problem 3b)
  Is the function $f(x) = x^2 -3x + 5$ continuous on the closed interval $[-1,2]$ and differentiable on the open interval $(-1,2)$? Why?
  If so, find all values of $x$ in $(-1,2)$ that satisfy the conclusion of the Mean Value Theorem for $f$ on the interval $[-1,2]$.
	
    \begin{hint}
      Compute $f'(x)$ and $\dfrac{f(2) - f(-1)}{2-(-1)}$
    \end{hint}
		\begin{hint}
		  Solve the equation: $f'(x) = \dfrac{f(2) - f(-1)}{2-(-1)}$ for $x$
		\end{hint}
		
		The value of $x$ is:
		 $\answer{1/2}$
 What do you notice about this value of $x$ in relation to the given interval? Is this always the case?
\end{problem}

\begin{problem}(problem 3c)
  Is the function $f(x) = 3x^2 -5x + 8$ continuous on the closed interval $[1,6]$ and differentiable on the open interval $(1,6)$? Why?
  If so, find all values of $x$ in $(1,6)$ that satisfy the conclusion of the Mean Value Theorem for $f$ on the interval $[1,6]$.
	
    \begin{hint}
      Compute $f'(x)$ and $\dfrac{f(6) - f(1)}{6-1}$
    \end{hint}
		\begin{hint}
		  Solve the equation: $f'(x) = \dfrac{f(6) - f(1)}{6-1}$ for $x$
		\end{hint}
		
		The value of $x$ is:
		 $\answer{7/2}$
     What do you notice about this value of $x$ in relation to the given interval? Is this always the case?
\end{problem}

\begin{problem}(problem 3d)
  Is the function $f(x) = x^3 + 2x -9$ continuous on the closed interval $[-2,2]$ and differentiable on the open interval $(-2,2)$? Why?
  If so, find all values of $x$ in $(-2,2)$ that satisfy the conclusion of the Mean Value Theorem for $f$ on the interval $[-2,2]$.
  	
    \begin{hint}
      Compute $f'(x)$ and $\dfrac{f(2) - f(-2)}{2-(-2)}$
    \end{hint}
		\begin{hint}
		  Solve the equation: $f'(x) = \dfrac{f(2) - f(-2)}{2-(-2)}$ for $x$
		\end{hint}
		
		The values of $x$ in ascending order are:
		 $\answer{-2/\sqrt 3}$ and $\answer{2/\sqrt 3}$
     are these values in the interval $(-2,2)$?
\end{problem}




\begin{example}[example 4]
Verify that the function $f(x) = \sqrt x$ satisfies the hypotheses of the Mean Value Theorem
on the interval $[0,4]$. Then find all values of $x$ that satisfy the conclusion of the theorem for $f$ on this interval.
The function $f(x) = \sqrt x$ is continuous on the interval $[0, \infty)$  and differentiable on the interval $(0, \infty)$. 
Hence, it is continuous on the closed interval $[0, 4]$ and differentiable on the open interval $(0, 4)$. 
To find the values of $x$ which satisfy the conclusion of the theorem, we first compute
\[\frac{f(b) - f(a)}{b-a} = \frac{\sqrt 4 - \sqrt 0}{4-0} = \frac{2}{4} = \frac{1}{2}.\]
Next, we compute
\[f'(x) = \frac{d}{dx} \sqrt x = \frac{d}{dx} x^{1/2} = (1/2)x^{-1/2} = \frac{1}{2\sqrt x}.\]
Finally, we solve the equation
\[\frac{1}{2\sqrt x} = \frac{1}{2}\]
Take the reciprocal of both sides:
\[2\sqrt x = 2\]
Divide both sides by $2$: 
\[\sqrt x = 1\]
Squaring both sides yields the unique solution
\[ x=1\]
Note that the value $x = 1$ is in the interval $(0,4)$.

In the figure below, the secant line is in blue, 
and the tangent line at $x = 1$ is in red. The Mean Value Theorem asserts that these lines are parallel, and the figure makes this clear.

\begin{image}
\begin{tikzpicture}
\begin{axis}[axis lines = center, legend pos=outer north east, title={MVT for $f(x) = \sqrt x$ on $[0,4]$}]
\addplot[domain=0:4, 
    samples=100, color=black]{sqrt(x)};
\addplot[smooth,mark=*,blue] plot coordinates {(0,0)  (4,2)};
\addplot[domain=0:3, 
    samples=100, color=blue]{0.5*x};
\addplot[domain=0:3, 
    samples=100, color=red]{0.5*x + 0.5 };
\addplot[smooth,mark=*,red] plot coordinates {(1, 1)};
\legend{$y=\sqrt x$, , secant line, tangent line, }
\end{axis}
\end{tikzpicture}
\end{image}


\end{example}

\begin{problem}(problem 4a)
  Is the function $f(x) = \sqrt x$ continuous on the closed interval $[0,9]$ and differentiable on the open interval $(0,9)$? Why?
  If so, find all values of $x$ in $(0,9)$ that satisfy the conclusion of the Mean Value Theorem for $f$ on the interval $[0,9]$.
	
    \begin{hint}
      Compute $f'(x)$ and $\dfrac{f(9) - f(0)}{9-0}$
    \end{hint}
		\begin{hint}
		  Solve $f'(x) = \dfrac{f(9) - f(0)}{9-0}$
		\end{hint}
		
		The value of $x$ is:
		 $\answer{9/4}$
     Is this value in the interval $(0,9)$?
\end{problem}


\begin{problem}(problem 4b)
  Is the function $f(x) = \sqrt x$ continuous on the closed interval $[4,16]$ and differentiable on the open interval $(4,16)$? Why?
  If so, find all values of $x$ in $(4,16)$ that satisfy the conclusion of the Mean Value Theorem for $f$ on the interval $[4,16]$.
	
    \begin{hint}
      Compute $f'(x)$ and $\dfrac{f(16) - f(4)}{16-4}$
    \end{hint}
		\begin{hint}
		  Solve $f'(x) = \dfrac{f(16) - f(4)}{16-4}$
		\end{hint}
		
		The value of $x$ is:
		 $\answer{9}$
     Is this value in the interval $(4,16)$?
\end{problem}


\begin{problem}(problem 4c)
  Is the function $f(x) = \sqrt{2x+1}$ continuous on the closed interval $[0,4]$ and differentiable on the open interval $(0,4)$? Why?
  If so, find all values of $x$ in $(0,4)$ that satisfy the conclusion of the Mean Value Theorem for $f$ on the interval $[0,4]$.


  Given that the function $f(x) = $ is continuous on the closed interval $[0,4]$ and differentiable on the open interval $(0,4)$,
  find the values of $x$ which satisfy the conclusion of the Mean Value Theorem for $f$ on the interval $[0,4]$.
	
    \begin{hint}
      Compute $f'(x)$ and $\dfrac{f(4) - f(0)}{4-0}$
    \end{hint}
		\begin{hint}
		  Solve $f'(x) = \dfrac{f(4) - f(0)}{4-0}$
		\end{hint}
		
		The value of $x$ is:
		 $\answer{3/2}$
     Is this value in the interval $(0,4)$?
\end{problem}



\begin{example}[example 5]
Verify that the function $f(x) = e^x$ satisfies the hypotheses of the Mean Value Theorem
on the interval $[0,2]$ and find all values of $x$ that satisfy the conclusion of the theorem for $f$ on this interval.
The function $f(x) = e^x$ is continuous and differentiable on the interval $(-\infty, \infty)$. 
Hence, $f$ is continuous on the closed interval $[0, 2]$ and differentiable on the open interval $(0, 2)$, and the hypotheses of the theorem are satisfied.
To find the values of $x$ that satisfy the conclusion of the Mean Value Theorem, we compute
\[\frac{f(b) - f(a)}{b-a} = \frac{e^2 - e^0}{2-0} = \frac{e^2 - 1}{2}\]
and, we compute
\[f'(x) = \frac{d}{dx} e^x = e^x.\]
We then set these equal and solve the resulting equation for $x$:
\[e^x = \frac{e^2 - 1}{2}\]
This gives the unique solution
\[ x = \ln(\frac{e^2 - 1}{2}).\]

Note that the value $\ln(\frac{e^2 - 1}{2}) \approx 1.1614$ is in the interval $(0,2)$.

In the figure below, the secant line is in blue, 
and the tangent line at $x = \ln(\frac{e^2 - 1}{2})$ is in red. The Mean Value Theorem asserts that these lines are parallel, and the figure makes this clear.

\begin{image}
\begin{tikzpicture}
\begin{axis}[axis lines = center, legend pos=outer north east, title={MVT for $f(x) = e^x$ on $[0,2]$}]
\addplot[domain=0:2, 
    samples=100, color=black]{e^x};
\addplot[smooth,mark=*,blue] plot coordinates {(0,1)  (2,e^2)};
\addplot[domain=0:2, 
    samples=100, color=blue]{3.194528049*x + 1};
\addplot[domain=0:2, 
    samples=100, color=red]{3.194528049*x - 0.51572257};
\addplot[smooth,mark=*,red] plot coordinates {(1.161439362, 3.194528049)};
\legend{$y=e^x$, , secant line, tangent line, }
\end{axis}
\end{tikzpicture}
\end{image}

\end{example}


\begin{problem}(problem 5)
  Is the function $f(x) = e^{2x}$ continuous on the closed interval $[0,1]$ and differentiable on the open interval $(0,1)$? Why?
  If so, find all values of $x$ in $(0,1)$ that satisfy the conclusion of the Mean Value Theorem for $f$ on the interval $[0,1]$.

    \begin{hint}
      Compute both $f'(x)$ and $\dfrac{f(1) - f(0)}{1-0}$
    \end{hint}
		\begin{hint}
		  Solve the equation $f'(x) = \dfrac{f(1) - f(0)}{1-0}$ for $x$
		\end{hint}
		
		The value of $x$ is:
		 $\answer{(1/2) \ln((e^2 -1)/2)}$
		 Is this value in the interval $(0,1)$?
\end{problem}


\begin{example}[example 6]
Verify that the function $f(x) = \ln(x)$ satisfies the hypotheses of the Mean Value Theorem
on the interval $[1,e]$, and we will find the values of $x$ that satisfies the Mean Value Equation for $f$ in $[1,e]$.\\
The function $\ln(x)$ is continuous and differentiable on the interval $(0, \infty)$. 
Hence, it is continuous on the closed interval $[1, e]$ and differentiable on the open interval $(1, e)$ and the hypotheses of the theorem are satisfied.
To find the values of $x$ that satisfy the conclusion of the Mean Value Theorem, we compute
\[
\frac{f(b) - f(a)}{b-a} = \frac{\ln(e) - \ln(1)}{e-1} = \frac{1 - 0}{e-1} = \frac{1}{e-1}
\]
Next, we compute
\[f'(x) = \frac{d}{dx} \ln(x) = \frac{1}{x}\]
Finally, we solve the equation
\[\frac{1}{x} = \frac{1}{e-1}\]
Taking the reciprocal of both sides yields the unique solution
\[ x = e-1\]

Note that the value $x = e-1\approx 1.72$ is in the interval $(1,e)$.

In the figure below, the secant line is in blue, 
and the tangent line at $x = e-1$ is in red. 
The Mean Value Theorem asserts that these lines are parallel, and the figure
makes this clear.

\begin{image}
\begin{tikzpicture}
\begin{axis}[axis lines = center, legend pos=outer north east, title={MVT for $f(x) = \ln(x)$ on $[1,e]$}]
\addplot[domain=1:e, 
    samples=100, color=black]{ln(x)};
\addplot[smooth,mark=*,blue] plot coordinates {(1,0)  (e,1)};
\addplot[domain=1:e, 
    samples=100, color=blue]{0.581976707*x - 0.581976707};
\addplot[domain=1:e, 
    samples=100, color=red]{0.581976707*x - 0.458675145};
\addplot[smooth,mark=*,red] plot coordinates {(1.718281828, 0.541324855)};
\legend{$y=\ln(x)$, , secant line, tangent line, }
\end{axis}
\end{tikzpicture}
\end{image}


\end{example}

\begin{problem}(problem 6a) 
  Is the function $f(x) = ln(x)$ continuous on the closed interval $[1,e]$ and differentiable on the open interval $(1,e)$? Why?
  If so, find all values of $x$ that satisfy the conclusion of the Mean Value Theorem for $f$ on the interval $[1,e]$.
	
    \begin{hint}
      Compute $f'(x)$ and $\dfrac{f(e) - f(1)}{e-1}$
    \end{hint}
    \begin{hint}
    $\ln 1 = 0$ and $\ln e = 1$
    \end{hint}
		\begin{hint}
		  Solve the equation $f'(x) = \dfrac{f(e) - f(1)}{e-1}$ for $x$
		\end{hint}
    \begin{hint}
    Begin by taking the reciprocal of both sides
    \end{hint}
		
		The value of $x$ is:
		 $\answer{e-1}$
     Is this value in the interval $(1,e)$?
\end{problem}


\begin{problem}(problem 6b)
  Is the function $f(x) = \sin(x)$ continuous on the closed interval $[0, 3\pi]$ and differentiable on the open interval $(0, 3\pi)$? Why?
  Find all values of $x$ that satisfy the conclusion of the Mean Value Theorem for $f$ on the interval $[0, 3\pi]$.
	
    \begin{hint}
      Compute $f'(x)$ and $\dfrac{f(3\pi) - f(0)}{3\pi - 0}$
    \end{hint}
		\begin{hint}
		  Solve the equation $f'(x) = \dfrac{f(3\pi) - f(0)}{3\pi - 0}$ for $x$
		\end{hint}
    \begin{hint}
    Only consider solutions in the interval $(0, 3\pi)$
    \end{hint}
		
		The values of $x$ are (in ascending order):
		 $\answer{\pi /2}$, $\answer{3\pi /2}$ and $\answer{5\pi /2}$
     Are these values in the interval $(0, 3\pi)$?
\end{problem}

\begin{problem}(problem 6c)
  Is the function $f(x) = |x|$ continuous on the closed interval $[-2,2]$ and differentiable on the open interval $(-2,2)$? Why or why not?
  Determine if there are any values of $x$ that satisfy the conclusion of the Mean Value Theorem for $f$ on the interval $[-2,2]$.
	
    \begin{hint}
      Compute $f'(x)$ and $\dfrac{f(2) - f(-2)}{2 - (-2)}$
    \end{hint}
		\begin{hint}
		  Solve the equation $f'(x) = \dfrac{f(2) - f(-2)}{2 - (-2)}$ for $x$ if possible
		\end{hint}
		
		Were there any values of $x$? Does this contradict the Mean Value Theorem?
\end{problem}

\begin{center}
\begin{foldable}
\unfoldable{Here is a detailed, lecture style video on the Mean Value Theorem:}
\youtube{suJx3pB_cVI}
\end{foldable}
\end{center}



\end{document}






