\documentclass[handout]{ximera}

%% You can put user macros here
%% However, you cannot make new environments



\newcommand{\ffrac}[2]{\frac{\text{\footnotesize $#1$}}{\text{\footnotesize $#2$}}}
\newcommand{\vasymptote}[2][]{
    \draw [densely dashed,#1] ({rel axis cs:0,0} -| {axis cs:#2,0}) -- ({rel axis cs:0,1} -| {axis cs:#2,0});
}


\graphicspath{{./}{firstExample/}}
\usepackage{forest}
\usepackage{amsmath}
\usepackage{amssymb}
\usepackage{array}
\usepackage[makeroom]{cancel} %% for strike outs
\usepackage{pgffor} %% required for integral for loops
\usepackage{tikz}
\usepackage{tikz-cd}
\usepackage{tkz-euclide}
\usetikzlibrary{shapes.multipart}


%\usetkzobj{all}
\tikzstyle geometryDiagrams=[ultra thick,color=blue!50!black]


\usetikzlibrary{arrows}
\tikzset{>=stealth,commutative diagrams/.cd,
  arrow style=tikz,diagrams={>=stealth}} %% cool arrow head
\tikzset{shorten <>/.style={ shorten >=#1, shorten <=#1 } } %% allows shorter vectors

\usetikzlibrary{backgrounds} %% for boxes around graphs
\usetikzlibrary{shapes,positioning}  %% Clouds and stars
\usetikzlibrary{matrix} %% for matrix
\usepgfplotslibrary{polar} %% for polar plots
\usepgfplotslibrary{fillbetween} %% to shade area between curves in TikZ



%\usepackage[width=4.375in, height=7.0in, top=1.0in, papersize={5.5in,8.5in}]{geometry}
%\usepackage[pdftex]{graphicx}
%\usepackage{tipa}
%\usepackage{txfonts}
%\usepackage{textcomp}
%\usepackage{amsthm}
%\usepackage{xy}
%\usepackage{fancyhdr}
%\usepackage{xcolor}
%\usepackage{mathtools} %% for pretty underbrace % Breaks Ximera
%\usepackage{multicol}



\newcommand{\RR}{\mathbb R}
\newcommand{\R}{\mathbb R}
\newcommand{\C}{\mathbb C}
\newcommand{\N}{\mathbb N}
\newcommand{\Z}{\mathbb Z}
\newcommand{\dis}{\displaystyle}
%\renewcommand{\d}{\,d\!}
\renewcommand{\d}{\mathop{}\!d}
\newcommand{\dd}[2][]{\frac{\d #1}{\d #2}}
\newcommand{\pp}[2][]{\frac{\partial #1}{\partial #2}}
\renewcommand{\l}{\ell}
\newcommand{\ddx}{\frac{d}{\d x}}

\newcommand{\zeroOverZero}{\ensuremath{\boldsymbol{\tfrac{0}{0}}}}
\newcommand{\inftyOverInfty}{\ensuremath{\boldsymbol{\tfrac{\infty}{\infty}}}}
\newcommand{\zeroOverInfty}{\ensuremath{\boldsymbol{\tfrac{0}{\infty}}}}
\newcommand{\zeroTimesInfty}{\ensuremath{\small\boldsymbol{0\cdot \infty}}}
\newcommand{\inftyMinusInfty}{\ensuremath{\small\boldsymbol{\infty - \infty}}}
\newcommand{\oneToInfty}{\ensuremath{\boldsymbol{1^\infty}}}
\newcommand{\zeroToZero}{\ensuremath{\boldsymbol{0^0}}}
\newcommand{\inftyToZero}{\ensuremath{\boldsymbol{\infty^0}}}


\newcommand{\numOverZero}{\ensuremath{\boldsymbol{\tfrac{\#}{0}}}}
\newcommand{\dfn}{\textbf}
%\newcommand{\unit}{\,\mathrm}
\newcommand{\unit}{\mathop{}\!\mathrm}
%\newcommand{\eval}[1]{\bigg[ #1 \bigg]}
\newcommand{\eval}[1]{ #1 \bigg|}
\newcommand{\seq}[1]{\left( #1 \right)}
\renewcommand{\epsilon}{\varepsilon}
\renewcommand{\iff}{\Leftrightarrow}

\DeclareMathOperator{\arccot}{arccot}
\DeclareMathOperator{\arcsec}{arcsec}
\DeclareMathOperator{\arccsc}{arccsc}
\DeclareMathOperator{\si}{Si}
\DeclareMathOperator{\proj}{proj}
\DeclareMathOperator{\scal}{scal}
\DeclareMathOperator{\cis}{cis}
\DeclareMathOperator{\Arg}{Arg}
%\DeclareMathOperator{\arg}{arg}
\DeclareMathOperator{\Rep}{Re}
\DeclareMathOperator{\Imp}{Im}
\DeclareMathOperator{\sech}{sech}
\DeclareMathOperator{\csch}{csch}
\DeclareMathOperator{\Log}{Log}

\newcommand{\tightoverset}[2]{% for arrow vec
  \mathop{#2}\limits^{\vbox to -.5ex{\kern-0.75ex\hbox{$#1$}\vss}}}
\newcommand{\arrowvec}{\overrightarrow}
\renewcommand{\vec}{\mathbf}
\newcommand{\veci}{{\boldsymbol{\hat{\imath}}}}
\newcommand{\vecj}{{\boldsymbol{\hat{\jmath}}}}
\newcommand{\veck}{{\boldsymbol{\hat{k}}}}
\newcommand{\vecl}{\boldsymbol{\l}}
\newcommand{\utan}{\vec{\hat{t}}}
\newcommand{\unormal}{\vec{\hat{n}}}
\newcommand{\ubinormal}{\vec{\hat{b}}}

\newcommand{\dotp}{\bullet}
\newcommand{\cross}{\boldsymbol\times}
\newcommand{\grad}{\boldsymbol\nabla}
\newcommand{\divergence}{\grad\dotp}
\newcommand{\curl}{\grad\cross}
%% Simple horiz vectors
\renewcommand{\vector}[1]{\left\langle #1\right\rangle}


\outcome{Learn the Mean Value Theorem and apply it to examples}

\title{3.5 Mean Value Theorem}

\begin{document}

\begin{abstract}
We apply the Mean Value Theorem.
\end{abstract}

\maketitle

\section{Rolle's Theorem}
We begin with a special case of the Mean Value Theorem known as Rolle's Theorem.
This theorem uses the Extreme Value Theorem to guarantee a critical number of a differentiable function under certain circumstances.

\begin{theorem}[Rolle's Theorem]
Suppose the function $f(x)$ is continuous on the closed interval $[a,b]$ and differentiable on the open interval $(a,b)$.
If $f(a) = f(b)$, then there exists a number $c$ between $a$ and $b$ such that 
\[
f'(c) = 0.
\]
\end{theorem}

\begin{image}
\begin{tikzpicture}
\begin{axis}[axis x line=  center, axis y line = none, xmin= -2.5, xmax=2.5, xtick={-2, 0, 2}, xticklabels={$a$,$c$,$b$},
title={Rolle's Theorem for $f(x) $ on $[a,b]$}]
\addplot[domain=-2.3:2.3, 
    samples=100, color=black, thick]{-0.25*x^2 + 1.5 };
    \node at (axis cs: 1.8,1.25) {$y = f(x)$};
\addplot[smooth,mark=*,blue,dashed] plot coordinates {(-2,0.5)  (2,0.5)} node[below, midway]{$f(a) = f(b)$};
\addplot[domain=-1.2:1.2, 
    samples=100, color=red]{1.5};
\addplot[smooth,mark=*,red] plot coordinates {(0, 1.5)} node[above] {$f'(c) = 0$};
\addplot[domain=-2.5:2.5, 
    samples=100, color=black]{0};
\end{axis}
\end{tikzpicture}
\end{image}
Since the function is continuous on the interval $[a,b]$, the Extreme Value Theorem applies.
Thus $f(x)$ has an absolute minimum and an absolute maximum on the interval.
Since $f(a) = f(b)$, $f(x)$ must have an absolute extreme on the interval $(a,b)$.
Recall that $f(x)$ is differentiable on $(a,b)$ which implies that the derivative is zero at this extreme.

 critical number on the interval $(a,b)$. 
At this value of our differentiable function, the derivative is zero.

(and hence a critical number)

\begin{example} Show that the function $f(x) = \sin(x)$ has a critical number in the interval $(0, \pi)$.\\
The function $\sin(x)$ is differentiable (and hence also continuous) on the interval $(-\infty, \infty)$ and since
\[
\sin 0 = \sin \pi = 0,
\]
we can apply Rolle's Theorem to $\sin(x)$ on the interval $[0, \pi]$. The theorem says that there exists a number $c$ between $0$ and $\pi$ such that 
$f'(c) = 0$. That is, $f(x) = \sin(x)$ has a critical number in the interval $(0, \pi)$.
Moreover, we can find $c$ since $f'(x) = \cos(x)$ and the equation $\cos(x) = 0$ has $x = \pi/2$ as a solution in the interval.
\end{example}

\begin{example} Use Rolle's Theorem  to show that a cubic polynomial can have at most 3 roots.\\
Recall that a root of a polynomial, $p(x)$, is a value $x = a$, such that $p(a) = 0$.

Also recall that a quadratic polynomial has at most two roots, since if
\[
ax^2 + bx + c = 0
\]
then
\[
x = \frac{-b \pm \sqrt{b^2 - 4ac}}{2a}.
\]
Let $p(x)$ be a cubic polynomial, i.e., 
\[
p(x) = ax^3 + bx^2 + cx + d, \;\; \text{where} \;\; a\neq 0.
\]
We will argue by contradiction to demonstrate that $p(x)$ can have at most 3 roots.
%So we would like to show that an equation of the form $p(x) = 0$, where $p(x)$ is a cubic polynomial, 
%We  suppose that $(p(x)$ can have 4 roots and draw from that an erroneous conclusion.
%This contradiction implies that our supposition ``$p(x)$ can have 4 roots" is false.  Therefore we can conclude that $p(x)$ has at most 3 roots.

 %that a cubic polynomial can have at most 3 roots.
Suppose that a cubic polynomial, $p(x)$, can have 4 roots.  From smallest to largest, we label the 4 roots $x_1, x_2, x_3$ and $x_4$.
Then, by definition of root,
\[
p(x_1) = p(x_2) = p(x_3) = p(x_4) = 0.
\]
We can now apply Rolle's Theorem to $p(x)$ on $[x_1, x_2], [x_2, x_3]$ and $[x_3, x_4]$ to conclude
that $p'(x)$ has a root on each of these intervals. 

That is, there exist numbers $c_1, c_2$ and $c_3$ such that
\[
p'(c_1) = p'(c_2) = p'(c_3) = 0,
\]
where: $x_1 < c_1 < x_2, \; x_2 < c_2 < x_3 \;$ and $\; x_3 < c_3 < x_4$.

\begin{image}
\begin{tikzpicture}
\begin{axis}[axis x line=  center, axis y line = none, xmin= -0.5, xmax=6.5, xtick={0, 1, 2, 3, 4, 5, 6}, 
xticklabels={$x_1$,$c_1$,$x_2$,$c_2$,$x_3$,$c_3$,$x_4$}, title={By Rolle's Theorem, $p'(x)$ has 3 roots}]
\addplot[domain=-0.5:6.5, 
    samples=100, color=black]{0};
\addplot [samples=100,smooth,domain=0:6] {sin(deg(1.57*x))} ;
\addplot[smooth,mark=*,blue, dashed] plot coordinates {(0,0)   (2,0)  (4,0) (6,0)} ;
\addplot[smooth,mark=*,red] plot coordinates {(1, 1)} ;
\addplot[smooth,mark=*,red] plot coordinates {(3, -1)} ;
\addplot[smooth,mark=*,red] plot coordinates {(5, 1)} ;
\addplot[domain=0.3:1.7, 
    samples=100, color=red]{1} node[above]{$p'(c_1) = 0$};
\addplot[domain=2.3:3.7, 
    samples=100, color=red]{-1} node[below]{$p'(c_2) = 0$};
\addplot[domain=2.9:3, color=white]{-1.4} ;
\addplot[domain=4.3:5.7, 
    samples=100, color=red]{1} node[above]{$p'(c_3) = 0$};
\end{axis}
\end{tikzpicture}
\end{image}

This would mean that $p'(x)$ has 3 roots, which is not possible: 
 $p'(x)$ is a quadratic polynomial, and so $p'(x)$ has at most 2 roots.
 
This contradiction implies that a cubic polynomial cannot have 4 roots. Thus, a cubic polynomial can have at most 3 roots.

In fact, a similar argument can be used to show that a polynomial of degree $n$ can have at most $n$ roots for any whole number, $n$.  
%This can be proved similarly.
%This can be proven using an argument similar to this one, combined with an advanced technique known as mathematical induction.
\end{example}




\section{The Mean Value Theorem}


The Mean Value Theorem is one of the most far-reaching theorems in calculus. It states that for a continuous 
and differentiable function, the average rate of change over an interval is attained as an 
instantaneous rate of change at some point inside the interval. The precise mathematical statement is as follows.\\

\begin{theorem}[Mean Value Theorem]
Suppose that the function $f(x)$ is continuous on the closed interval $[a,b]$ and differentiable on the 
open interval $(a,b)$. Then there exists a number $c$ between $x = a$ and $x = b$ such that
\[f'(c) = \frac{f(b) - f(a)}{b-a}\]

\end{theorem}



Geometrically, the left-hand side of the conclusion of the MVT represents the slope of the tangent line to $f(x)$ at $x = c$ 
Meanwhile, the right-hand side represents the slope of the secant line connecting the points $(a, f(a))$ and $(b, f(b))$. 
Since their slopes are equal, these two lines are parallel.
Conceptually, the left-hand side represents the instantaneous rate of change of $f(x)$ at $x = c$ while the
right-hand side represents the average rate of change of $f(x)$ over the interval from $x=a$ to $x=b$. 
Thus, the MVT says that at some point, the instantaneous rate of change will be equal to the average rate of change.

\begin{image}
\begin{tikzpicture}
\begin{axis}[axis x line=  center, axis y line = none, xtick={-1, 0.5, 2}, xticklabels={$a$,$c$,$b$}, 
legend pos=outer north east, title={MVT for $f(x) $ on $[a,b]$}]
\addplot[domain=-1.4:2.4, 
    samples=100, color=black, thick]{-x^2 + 5 };
\addplot[smooth,mark=*,blue, thin] plot coordinates {(-1,4)  (2,1)};
\addplot[domain=-1:2, 
    samples=100, color=blue, thin]{-x + 3};
\addplot[domain=-0.2:1.2, 
    samples=100, color=red, thin]{-x + 5.25 };
\addplot[smooth,mark=*,red] plot coordinates {(0.5, 4.75)};
\legend{$y = f(x)$, , secant line, tangent line, };
\addplot[dashed] coordinates{(-1,0)(-1,4)};
\addplot[dashed] coordinates{(0.5,0)(0.5,4.75)};
\addplot[dashed] coordinates{(2,0)(2,1)};
node[label=below]{some label};
\end{axis}
\end{tikzpicture}
\end{image}

As an example of the conceptual interpretation of the theorem, consider a car that averages  a speed of 57.4 miles per hour on a long trip. 
By the MVT the car must have been 
traveling at exactly 57.4 miles per hour at some instant during the trip.

The MVT is considered an existence theorem because it asserts that there exists at least one value $c$ inside the interval $(a,b)$
that satisfies the equation 
\[f'(c) = \frac{f(b) - f(a)}{b-a}.\]
In our examples, we will determine the exact value of $c$. 

\begin{example}[example 1]
Verify that the function $f(x) = x^2$ satisfies the hypotheses of the Mean Value Theorem
on the interval $[0,3]$. Then find all values that satisfy the conclusion of the theorem.\\
Since $x^2$ is a polynomial, it is continuous on the closed interval $[0, 3]$ and differentiable on the open interval $(0, 3)$. 
Thus, $f(x) = x^2$ satisfies the hypotheses of the MVT on $[0, 3]$. We now know that the equation
\[
f'(x) = \frac{f(b) - f(a)}{b-a}
\]
has at least one solution in the interval $(0, 3)$. To find the solution(s), 
we first compute the value of the right-hand side using $a = 0$ and $b = 3$:
\[\frac{f(b) - f(a)}{b-a} = \frac{f(3) - f(0)}{3-0} \]
\[= \frac{3^2 - 0^2}{3}= \frac{9 - 0}{3} = 3.\]
Since, $f'(x) = 2x$,
the conclusion of the MVT guarantees that the equation
\[2x = 3\]
has at least one solution in the open interval $(0,3)$.
This is easily verified, since if $2x = 3$, then 
\[x = \frac{3}{2},
\]
which is in the open interval $(0, 3)$.
Furthermore, note that $3/2$ is actually the mid-point of the interval $(0, 3)$.
When applying the MVT to a quadratic polynomial on any interval $[a, b]$ the point $c$ will always be the mid-point!
% and so the special value, $c$, guaranteed to exist by the MVT,
%is $\frac32$ in this example.


\begin{image}
\begin{tikzpicture}
\begin{axis}[axis lines = center, legend pos=outer north east, title={MVT for $f(x) = x^2$ on $[0,3]$}]
\addplot[domain=0:3, 
    samples=100, color=black, thick]{x^2};
\addplot[smooth,mark=*,blue, thin] plot coordinates {(0,0)  (3,9)};
\addplot[domain=0:3, 
    samples=100, color=blue, thin]{3*x};
\addplot[domain=0.8:2.2, 
    samples=100, color=red, thin]{3*x - 2.25 };
\addplot[smooth,mark=*,red] plot coordinates {(1.5, 2.25)};
\legend{$y=x^2$, , \text{secant line}, \text{tangent line}, }
\end{axis}
\end{tikzpicture}
\end{image}

In the above figure, the blue line in the secant line for $f(x) = x^2$ on the interval $[0, 3]$, 
and the red line is the tangent line at $x = 3/2$. The Mean Value Theorem asserts that these lines are parallel.
\end{example}

\begin{problem}(problem 1a)
  Given that the function $f(x) = x^2 + 2$ satisfies the hypotheses of the MVT on the
	interval $[1,3]$, find the value of $c$ in the open interval $(1,3)$ which satisfies 
	the conclusion of the theorem.
	
    \begin{hint}
      Compute $f'(x)$ and $\dfrac{f(3) - f(1)}{3-1}$
    \end{hint}
		\begin{hint}
		  Solve $f'(x) = \dfrac{f(3) - f(1)}{3-1}$
		\end{hint}
		
		The value of $c$ is:
		 $\answer{2}$
\end{problem}

\begin{problem}(problem 1b)
  Given that the function $f(x) = x^2 -3x + 5$ satisfies the hypotheses of the MVT on the
	interval $[-1,2]$, find the value of $c$ in the open interval $(-1,2)$ which satisfies 
	the conclusion of the theorem.
	
    \begin{hint}
      Compute $f'(x)$ and $\dfrac{f(2) - f(-1)}{2-(-1)}$
    \end{hint}
		\begin{hint}
		  Solve $f'(x) = \dfrac{f(2) - f(-1)}{2-(-1)}$
		\end{hint}
		
		The value of $c$ is:
		 $\answer{1/2}$
\end{problem}

\begin{problem}(problem 1c)
  Given that the function $f(x) = 3x^2 -5x + 8$ satisfies the hypotheses of the MVT on the
	interval $[1,6]$, find the value of $c$ in the open interval $(1,6)$ which satisfies 
	the conclusion of the theorem.
	
    \begin{hint}
      Compute $f'(x)$ and $\dfrac{f(6) - f(1)}{6-1}$
    \end{hint}
		\begin{hint}
		  Solve $f'(x) = \dfrac{f(6) - f(1)}{6-1}$
		\end{hint}
		
		The value of $c$ is:
		 $\answer{7/2}$
\end{problem}

\begin{problem}(problem 1d)
  Given that the function $f(x) = x^3 + 2x -9$ satisfies the hypotheses of the MVT on the
	interval $[-2,2]$, find the values of $c$ in the open interval $(-2,2)$ which satisfy 
	the conclusion of the theorem.
	
    \begin{hint}
      Compute $f'(x)$ and $\dfrac{f(2) - f(-2)}{2-(-2)}$
    \end{hint}
		\begin{hint}
		  Solve $f'(x) = \dfrac{f(2) - f(-2)}{2-(-2)}$
		\end{hint}
		
		The values of $c$ in ascending order are:
		 $\answer{-2/\sqrt 3}$ and $\answer{2/\sqrt 3}$
\end{problem}




\begin{example}[example 2]
We will verify that the function $f(x) = \sqrt x$ satisfies the hypotheses of the MVT
on the interval $[0,4]$ and we will find the special value of $c$ that satisfies the Mean Value Equation.
The function $\sqrt x$ is continuous on the interval $[0, \infty)$  and differentiable on the interval $(0, \infty)$. 
Hence, it is continuous on the closed interval $[0, 4]$ and differentiable on the open interval $(0, 4)$. 
Then, by the MVT,  the Mean Value Equation holds for some number 
$c$ in the interval $(0, 4)$. To find $c$, we first compute
\[\frac{f(b) - f(a)}{b-a} = \frac{\sqrt 4 - \sqrt 0}{4-0} = \frac{2}{4} = \frac{1}{2}.\]
Next we compute
\[f'(x) = \frac{d}{dx} \sqrt x = \frac{d}{dx} x^{1/2} = (1/2)x^{-1/2} = \frac{1}{2\sqrt x}.\]
Finally we solve the equation
\[\frac{1}{2\sqrt x} = \frac{1}{2}\]
which gives
\[\sqrt x = 1\]
and so 
\[ x=1.\]
Note that the value $x = 1$ is in the interval $(0,4)$ and so the special value, $c$, guaranteed to exist by the MVT,
is $1$ in this example.


\begin{image}
\begin{tikzpicture}
\begin{axis}[axis lines = center, legend pos=outer north east, title={MVT for $f(x) = \sqrt x$ on $[0,4]$}]
\addplot[domain=0:4, 
    samples=100, color=black]{sqrt(x)};
\addplot[smooth,mark=*,blue] plot coordinates {(0,0)  (4,2)};
\addplot[domain=0:3, 
    samples=100, color=blue]{0.5*x};
\addplot[domain=0:3, 
    samples=100, color=red]{0.5*x + 0.5 };
\addplot[smooth,mark=*,red] plot coordinates {(1, 1)};
\legend{$y=\sqrt x$, , secant line, tangent line, }
\end{axis}
\end{tikzpicture}
\end{image}

In the above figure, the blue line in the secant line for $f(x) = \sqrt x$ on the interval $[0, 4]$, 
and the red line is the tangent line at $x = 1$. The Mean Value Equation asserts that these lines are parallel, and this
is clear in the figure.
\end{example}

\begin{problem}(problem 2a)
  Given that the function $f(x) = \sqrt x$ satisfies the hypotheses of the MVT on the
	interval $[0,9]$, find the value of $c$ in the open interval $(0,9)$ which satisfies 
	the conclusion of the theorem.
	
    \begin{hint}
      Compute $f'(x)$ and $\dfrac{f(9) - f(0)}{9-0}$
    \end{hint}
		\begin{hint}
		  Solve $f'(x) = \dfrac{f(9) - f(0)}{9-0}$
		\end{hint}
		
		The value of $c$ is:
		 $\answer{9/4}$
\end{problem}


\begin{problem}(problem 2b)
  Given that the function $f(x) = \sqrt x$ satisfies the hypotheses of the MVT on the
	interval $[4,16]$, find the value of $c$ in the open interval $(4,16)$ which satisfies 
	the conclusion of the theorem.
	
    \begin{hint}
      Compute $f'(x)$ and $\dfrac{f(16) - f(4)}{16-4}$
    \end{hint}
		\begin{hint}
		  Solve $f'(x) = \dfrac{f(16) - f(4)}{16-4}$
		\end{hint}
		
		The value of $c$ is:
		 $\answer{9}$
\end{problem}


\begin{problem}(problem 2c)
  Given that the function $f(x) = \sqrt{2x+1}$ satisfies the hypotheses of the MVT on the
	interval $[0,4]$, find the value of $c$ in the open interval $(0,4)$ which satisfies 
	the conclusion of the theorem.
	
    \begin{hint}
      Compute $f'(x)$ and $\dfrac{f(4) - f(0)}{4-0}$
    \end{hint}
		\begin{hint}
		  Solve $f'(x) = \dfrac{f(4) - f(0)}{4-0}$
		\end{hint}
		
		The value of $c$ is:
		 $\answer{3/2}$
\end{problem}



\begin{example}[example 3]
We will verify that the function $f(x) = e^x$ satisfies the hypotheses of the MVT
on the interval $[0,2]$ and we will find the special value of $c$ that satisfies the Mean Value Equation.
The function $e^x$ is continuous and differentiable on the interval $(-\infty, \infty)$. 
Hence, it is continuous on the closed interval $[0, 2]$ and differentiable on the open interval $(0, 2)$. 
Then, by the MVT,  the Mean Value Equation holds for some number 
$c$ in the interval $(0, 2)$. To find $c$, we first compute
\[\frac{f(b) - f(a)}{b-a} = \frac{e^2 - e^0}{2-0} = \frac{e^2 - 1}{2}.\]
Next we compute
\[f'(x) = \frac{d}{dx} e^x = e^x.\]
Finally we solve the equation
\[e^x = \frac{e^2 - 1}{2}\]
which gives
\[ x = \ln(\frac{e^2 - 1}{2}) \approx 1.16.\]

Note that the value $\ln(\frac{e^2 - 1}{2})$ is in the interval $(0,2)$ and so the 
special value, $c$, guaranteed to exist by the MVT, is $\ln(\frac{e^2 - 1}{2})$ in this example.


\begin{image}
\begin{tikzpicture}
\begin{axis}[axis lines = center, legend pos=outer north east, title={MVT for $f(x) = e^x$ on $[0,2]$}]
\addplot[domain=0:2, 
    samples=100, color=black]{e^x};
\addplot[smooth,mark=*,blue] plot coordinates {(0,1)  (2,e^2)};
\addplot[domain=0:2, 
    samples=100, color=blue]{3.194528049*x + 1};
\addplot[domain=0:2, 
    samples=100, color=red]{3.194528049*x - 0.51572257};
\addplot[smooth,mark=*,red] plot coordinates {(1.161439362, 3.194528049)};
\legend{$y=e^x$, , secant line, tangent line, }
\end{axis}
\end{tikzpicture}
\end{image}

In the above figure, the blue line in the secant line for $f(x) = e^x$ on the interval $[0, 2]$, 
and the red line is the tangent line at $x = \ln(\frac{e^2 - 1}{2})$. 
The Mean Value Equation asserts that these lines are parallel, and this
is clear in the figure.
\end{example}


\begin{problem}(problem 3)
  Given that the function $f(x) = e^{2x}$ satisfies the hypotheses of the MVT on the
	interval $[0,1]$, find the value of $c$ in the open interval $(0,1)$ which satisfies 
	the conclusion of the theorem.
	
    \begin{hint}
      Compute $f'(x)$ and $\dfrac{f(1) - f(0)}{1-0}$
    \end{hint}
		\begin{hint}
		  Solve $f'(x) = \dfrac{f(1) - f(0)}{1-0}$
		\end{hint}
		
		The value of $c$ is:
		 $\answer{(1/2) \ln((e^2 -1)/2)}$
\end{problem}


\begin{example}[example 4] %example 4
We will verify that the function $f(x) = \ln(x)$ satisfies the hypotheses of the MVT
on the interval $[1,e]$ and we will find the special value of $c$ that satisfies the Mean Value Equation.
The function $\ln(x)$ is continuous and differentiable on the interval $(0, \infty)$. 
Hence, it is continuous on the closed interval $[1, e]$ and differentiable on the open interval $(1, e)$. 
Then, by the MVT,  the Mean Value Equation holds for some number 
$c$ in the interval $(1, e)$. To find $c$, we first compute
\[\frac{f(b) - f(a)}{b-a} = \frac{\ln(e) - \ln(1)}{e-1} = \frac{1 - 0}{e-1} = \frac{1}{e-1}.\]
Next we compute
\[f'(x) = \frac{d}{dx} \ln(x) = \frac{1}{x}.\]
Finally we solve the equation
\[\frac{1}{x} = \frac{1}{e-1}\]
which gives
\[ x = e-1 \approx 1.72.\]

Note that the value $x = e-1$ is in the interval $(1,e)$ and so the special value, $c$, 
guaranteed to exist by the MVT,
is $e-1$ in this example.


\begin{image}
\begin{tikzpicture}
\begin{axis}[axis lines = center, legend pos=outer north east, title={MVT for $f(x) = \ln(x)$ on $[1,e]$}]
\addplot[domain=1:e, 
    samples=100, color=black]{ln(x)};
\addplot[smooth,mark=*,blue] plot coordinates {(1,0)  (e,1)};
\addplot[domain=1:e, 
    samples=100, color=blue]{0.581976707*x - 0.581976707};
\addplot[domain=1:e, 
    samples=100, color=red]{0.581976707*x - 0.458675145};
\addplot[smooth,mark=*,red] plot coordinates {(1.718281828, 0.541324855)};
\legend{$y=\ln(x)$, , secant line, tangent line, }
\end{axis}
\end{tikzpicture}
\end{image}

In the above figure, the blue line in the secant line for $f(x) = \ln(x)$ on the interval $[1, e]$, 
and the red line is the tangent line at $x = e-1$. 
The Mean Value Equation asserts that these lines are parallel, and this
is clear in the figure.
\end{example}

\begin{problem}(problem 4a) 
  Given that the function $f(x) = \ln(x)$ satisfies the hypotheses of the MVT on the
	interval $[1,e]$, find the value of $c$ in the open interval $(1,e)$ which satisfies 
	the conclusion of the theorem.
	
    \begin{hint}
      Compute $f'(x)$ and $\dfrac{f(e) - f(1)}{e-1}$
    \end{hint}
		\begin{hint}
		  Solve $f'(x) = \dfrac{f(e) - f(1)}{e-1}$
		\end{hint}
		
		The value of $c$ is:
		 $\answer{e-1}$
\end{problem}
%example with |x| to show MVE does not always have a solution

\begin{problem}(problem 4b)
  Given that the function $f(x) = \sin(x)$ satisfies the hypotheses of the MVT on the
	interval $[0, \pi]$, find the value of $c$ in the open interval $(0, \pi)$ which satisfies 
	the conclusion of the theorem.
	
    \begin{hint}
      Compute $f'(x)$ and $\dfrac{f(\pi) - f(0)}{\pi - 0}$
    \end{hint}
		\begin{hint}
		  Solve $f'(c) = \dfrac{f(\pi) - f(0)}{\pi - 0}$
		\end{hint}
		
		The value of $c$ is:
		 $\answer{\pi /2}$
\end{problem}

\begin{center}
\begin{foldable}
\unfoldable{Here is a detailed, lecture style video on the Mean Value Theorem:}
\youtube{suJx3pB_cVI}
\end{foldable}
\end{center}



\end{document}






