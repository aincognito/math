\documentclass[handout]{ximera}

%% You can put user macros here
%% However, you cannot make new environments



\newcommand{\ffrac}[2]{\frac{\text{\footnotesize $#1$}}{\text{\footnotesize $#2$}}}
\newcommand{\vasymptote}[2][]{
    \draw [densely dashed,#1] ({rel axis cs:0,0} -| {axis cs:#2,0}) -- ({rel axis cs:0,1} -| {axis cs:#2,0});
}


\graphicspath{{./}{firstExample/}}
\usepackage{forest}
\usepackage{amsmath}
\usepackage{amssymb}
\usepackage{array}
\usepackage[makeroom]{cancel} %% for strike outs
\usepackage{pgffor} %% required for integral for loops
\usepackage{tikz}
\usepackage{tikz-cd}
\usepackage{tkz-euclide}
\usetikzlibrary{shapes.multipart}


%\usetkzobj{all}
\tikzstyle geometryDiagrams=[ultra thick,color=blue!50!black]


\usetikzlibrary{arrows}
\tikzset{>=stealth,commutative diagrams/.cd,
  arrow style=tikz,diagrams={>=stealth}} %% cool arrow head
\tikzset{shorten <>/.style={ shorten >=#1, shorten <=#1 } } %% allows shorter vectors

\usetikzlibrary{backgrounds} %% for boxes around graphs
\usetikzlibrary{shapes,positioning}  %% Clouds and stars
\usetikzlibrary{matrix} %% for matrix
\usepgfplotslibrary{polar} %% for polar plots
\usepgfplotslibrary{fillbetween} %% to shade area between curves in TikZ



%\usepackage[width=4.375in, height=7.0in, top=1.0in, papersize={5.5in,8.5in}]{geometry}
%\usepackage[pdftex]{graphicx}
%\usepackage{tipa}
%\usepackage{txfonts}
%\usepackage{textcomp}
%\usepackage{amsthm}
%\usepackage{xy}
%\usepackage{fancyhdr}
%\usepackage{xcolor}
%\usepackage{mathtools} %% for pretty underbrace % Breaks Ximera
%\usepackage{multicol}



\newcommand{\RR}{\mathbb R}
\newcommand{\R}{\mathbb R}
\newcommand{\C}{\mathbb C}
\newcommand{\N}{\mathbb N}
\newcommand{\Z}{\mathbb Z}
\newcommand{\dis}{\displaystyle}
%\renewcommand{\d}{\,d\!}
\renewcommand{\d}{\mathop{}\!d}
\newcommand{\dd}[2][]{\frac{\d #1}{\d #2}}
\newcommand{\pp}[2][]{\frac{\partial #1}{\partial #2}}
\renewcommand{\l}{\ell}
\newcommand{\ddx}{\frac{d}{\d x}}

\newcommand{\zeroOverZero}{\ensuremath{\boldsymbol{\tfrac{0}{0}}}}
\newcommand{\inftyOverInfty}{\ensuremath{\boldsymbol{\tfrac{\infty}{\infty}}}}
\newcommand{\zeroOverInfty}{\ensuremath{\boldsymbol{\tfrac{0}{\infty}}}}
\newcommand{\zeroTimesInfty}{\ensuremath{\small\boldsymbol{0\cdot \infty}}}
\newcommand{\inftyMinusInfty}{\ensuremath{\small\boldsymbol{\infty - \infty}}}
\newcommand{\oneToInfty}{\ensuremath{\boldsymbol{1^\infty}}}
\newcommand{\zeroToZero}{\ensuremath{\boldsymbol{0^0}}}
\newcommand{\inftyToZero}{\ensuremath{\boldsymbol{\infty^0}}}


\newcommand{\numOverZero}{\ensuremath{\boldsymbol{\tfrac{\#}{0}}}}
\newcommand{\dfn}{\textbf}
%\newcommand{\unit}{\,\mathrm}
\newcommand{\unit}{\mathop{}\!\mathrm}
%\newcommand{\eval}[1]{\bigg[ #1 \bigg]}
\newcommand{\eval}[1]{ #1 \bigg|}
\newcommand{\seq}[1]{\left( #1 \right)}
\renewcommand{\epsilon}{\varepsilon}
\renewcommand{\iff}{\Leftrightarrow}

\DeclareMathOperator{\arccot}{arccot}
\DeclareMathOperator{\arcsec}{arcsec}
\DeclareMathOperator{\arccsc}{arccsc}
\DeclareMathOperator{\si}{Si}
\DeclareMathOperator{\proj}{proj}
\DeclareMathOperator{\scal}{scal}
\DeclareMathOperator{\cis}{cis}
\DeclareMathOperator{\Arg}{Arg}
%\DeclareMathOperator{\arg}{arg}
\DeclareMathOperator{\Rep}{Re}
\DeclareMathOperator{\Imp}{Im}
\DeclareMathOperator{\sech}{sech}
\DeclareMathOperator{\csch}{csch}
\DeclareMathOperator{\Log}{Log}

\newcommand{\tightoverset}[2]{% for arrow vec
  \mathop{#2}\limits^{\vbox to -.5ex{\kern-0.75ex\hbox{$#1$}\vss}}}
\newcommand{\arrowvec}{\overrightarrow}
\renewcommand{\vec}{\mathbf}
\newcommand{\veci}{{\boldsymbol{\hat{\imath}}}}
\newcommand{\vecj}{{\boldsymbol{\hat{\jmath}}}}
\newcommand{\veck}{{\boldsymbol{\hat{k}}}}
\newcommand{\vecl}{\boldsymbol{\l}}
\newcommand{\utan}{\vec{\hat{t}}}
\newcommand{\unormal}{\vec{\hat{n}}}
\newcommand{\ubinormal}{\vec{\hat{b}}}

\newcommand{\dotp}{\bullet}
\newcommand{\cross}{\boldsymbol\times}
\newcommand{\grad}{\boldsymbol\nabla}
\newcommand{\divergence}{\grad\dotp}
\newcommand{\curl}{\grad\cross}
%% Simple horiz vectors
\renewcommand{\vector}[1]{\left\langle #1\right\rangle}


\pgfplotsset{compat=1.13}

\outcome{Learn properties of the complex conjugate}

\title{1.3 Complex Conjugate}

\begin{document}

\begin{abstract}
We learn properties of the complex conjugate.
\end{abstract}

\maketitle



We have seen that the complex conjugate is defined by
\[
\overline{a+bi} = a-bi
\]

The conjugate of the conjugate is the original complex number:

\[
\overline{\overline{a+bi}} = \overline{a-bi} = a+bi
\]

The conjugate of a real number is itself:
\[
\overline{a}=\overline{a+0i} = a-0i = a
\]

The conjugate of an imaginary number is its negative:
\[
\overline{bi}=\overline{0+bi} = 0-bi = -bi
\]


\section{Real and Imaginary Part}


If we add a complex number and it's conjugate, we get
\[
(a+bi) + (\overline{a+bi}) = 2a = 2\Rep(a+bi)
\]
Thus, we have a formula for the real part of a complex number in terms of its conjugate:
\[
\Rep(a+bi) = \frac12 \left[(a+bi) + (\overline{a+bi})\right]
\]
Similarly, subtracting the conjugate gives 
\[
a+bi - \overline{a+bi} = 2bi,
\]
and so
\[
\Imp(a+bi) = -\frac{i}{2}\left[(a+bi) + (\overline{a+bi})\right]
\]
 
\section{Modulus}

If we multiply a complex number by its conjugate, we get the square of the modulus:
\[
(a+bi)(\overline{a+bi}) = (a+bi)(a-bi) = a^2 + b^2 = |a+bi|^2
\]
Thus, we have a formula for the modulus of a complex number in terms of its conjugate:
\[
|a+bi|= \sqrt{(a+bi)(\overline{a+bi})}
\]


\section{Multiplicative Inverse}

For a non-zero complex number, $a+bi$, its multiplicative inverse is its conjugate divided by the square of its modulus:
\[
\frac{1}{a+bi} = \frac{1}{a+bi}\cdot \frac{\overline{a+bi}}{\overline{a+bi}} = \frac{\overline{a+bi}}{\ |a+bi|^2}
\]



\section{Addition and Multiplication}

The conjugate of a sum is the sum of the conjugates:
\[
\overline{(a+bi) + (c+di)} = (a+c) - (b+d)i = \overline{a+bi }+\overline{c+di}
\]


The conjugate of a product is the product of the conjugates:
\[
\overline{(a+bi) \cdot (c+di)} = \overline{a+bi }\cdot\overline{c+di}
\]
To see this, we can calculate the left and right sides separately and see that they are the same:
\[
\overline{(a+bi) \cdot (c+di)}= \overline{(ac-bd) + (ad+bc)i} = (ac-bd) - (ad+bc)i
\]
and
\[
\overline{a+bi }\cdot\overline{c+di} = (a-bi)(c-di) = (ac-bd) + (-ad-bc)i = (ac-bd) - (ad +bc)i
\]


\begin{problem}(problem 1) 
Prove that the conjugate of a difference is the difference of the conjugates.
\end{problem}


\begin{problem}(problem 2) 
Prove that the conjugate of a quotient is the quotient of the conjugates.
\end{problem}

Here is a video solution for problem 2:\\
\begin{foldable}
\youtube{59J6em9bRb0}
\end{foldable}
\end{document}



