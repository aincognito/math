\documentclass[handout]{ximera}

%% You can put user macros here
%% However, you cannot make new environments



\newcommand{\ffrac}[2]{\frac{\text{\footnotesize $#1$}}{\text{\footnotesize $#2$}}}
\newcommand{\vasymptote}[2][]{
    \draw [densely dashed,#1] ({rel axis cs:0,0} -| {axis cs:#2,0}) -- ({rel axis cs:0,1} -| {axis cs:#2,0});
}


\graphicspath{{./}{firstExample/}}
\usepackage{forest}
\usepackage{amsmath}
\usepackage{amssymb}
\usepackage{array}
\usepackage[makeroom]{cancel} %% for strike outs
\usepackage{pgffor} %% required for integral for loops
\usepackage{tikz}
\usepackage{tikz-cd}
\usepackage{tkz-euclide}
\usetikzlibrary{shapes.multipart}


%\usetkzobj{all}
\tikzstyle geometryDiagrams=[ultra thick,color=blue!50!black]


\usetikzlibrary{arrows}
\tikzset{>=stealth,commutative diagrams/.cd,
  arrow style=tikz,diagrams={>=stealth}} %% cool arrow head
\tikzset{shorten <>/.style={ shorten >=#1, shorten <=#1 } } %% allows shorter vectors

\usetikzlibrary{backgrounds} %% for boxes around graphs
\usetikzlibrary{shapes,positioning}  %% Clouds and stars
\usetikzlibrary{matrix} %% for matrix
\usepgfplotslibrary{polar} %% for polar plots
\usepgfplotslibrary{fillbetween} %% to shade area between curves in TikZ



%\usepackage[width=4.375in, height=7.0in, top=1.0in, papersize={5.5in,8.5in}]{geometry}
%\usepackage[pdftex]{graphicx}
%\usepackage{tipa}
%\usepackage{txfonts}
%\usepackage{textcomp}
%\usepackage{amsthm}
%\usepackage{xy}
%\usepackage{fancyhdr}
%\usepackage{xcolor}
%\usepackage{mathtools} %% for pretty underbrace % Breaks Ximera
%\usepackage{multicol}



\newcommand{\RR}{\mathbb R}
\newcommand{\R}{\mathbb R}
\newcommand{\C}{\mathbb C}
\newcommand{\N}{\mathbb N}
\newcommand{\Z}{\mathbb Z}
\newcommand{\dis}{\displaystyle}
%\renewcommand{\d}{\,d\!}
\renewcommand{\d}{\mathop{}\!d}
\newcommand{\dd}[2][]{\frac{\d #1}{\d #2}}
\newcommand{\pp}[2][]{\frac{\partial #1}{\partial #2}}
\renewcommand{\l}{\ell}
\newcommand{\ddx}{\frac{d}{\d x}}

\newcommand{\zeroOverZero}{\ensuremath{\boldsymbol{\tfrac{0}{0}}}}
\newcommand{\inftyOverInfty}{\ensuremath{\boldsymbol{\tfrac{\infty}{\infty}}}}
\newcommand{\zeroOverInfty}{\ensuremath{\boldsymbol{\tfrac{0}{\infty}}}}
\newcommand{\zeroTimesInfty}{\ensuremath{\small\boldsymbol{0\cdot \infty}}}
\newcommand{\inftyMinusInfty}{\ensuremath{\small\boldsymbol{\infty - \infty}}}
\newcommand{\oneToInfty}{\ensuremath{\boldsymbol{1^\infty}}}
\newcommand{\zeroToZero}{\ensuremath{\boldsymbol{0^0}}}
\newcommand{\inftyToZero}{\ensuremath{\boldsymbol{\infty^0}}}


\newcommand{\numOverZero}{\ensuremath{\boldsymbol{\tfrac{\#}{0}}}}
\newcommand{\dfn}{\textbf}
%\newcommand{\unit}{\,\mathrm}
\newcommand{\unit}{\mathop{}\!\mathrm}
%\newcommand{\eval}[1]{\bigg[ #1 \bigg]}
\newcommand{\eval}[1]{ #1 \bigg|}
\newcommand{\seq}[1]{\left( #1 \right)}
\renewcommand{\epsilon}{\varepsilon}
\renewcommand{\iff}{\Leftrightarrow}

\DeclareMathOperator{\arccot}{arccot}
\DeclareMathOperator{\arcsec}{arcsec}
\DeclareMathOperator{\arccsc}{arccsc}
\DeclareMathOperator{\si}{Si}
\DeclareMathOperator{\proj}{proj}
\DeclareMathOperator{\scal}{scal}
\DeclareMathOperator{\cis}{cis}
\DeclareMathOperator{\Arg}{Arg}
%\DeclareMathOperator{\arg}{arg}
\DeclareMathOperator{\Rep}{Re}
\DeclareMathOperator{\Imp}{Im}
\DeclareMathOperator{\sech}{sech}
\DeclareMathOperator{\csch}{csch}
\DeclareMathOperator{\Log}{Log}

\newcommand{\tightoverset}[2]{% for arrow vec
  \mathop{#2}\limits^{\vbox to -.5ex{\kern-0.75ex\hbox{$#1$}\vss}}}
\newcommand{\arrowvec}{\overrightarrow}
\renewcommand{\vec}{\mathbf}
\newcommand{\veci}{{\boldsymbol{\hat{\imath}}}}
\newcommand{\vecj}{{\boldsymbol{\hat{\jmath}}}}
\newcommand{\veck}{{\boldsymbol{\hat{k}}}}
\newcommand{\vecl}{\boldsymbol{\l}}
\newcommand{\utan}{\vec{\hat{t}}}
\newcommand{\unormal}{\vec{\hat{n}}}
\newcommand{\ubinormal}{\vec{\hat{b}}}

\newcommand{\dotp}{\bullet}
\newcommand{\cross}{\boldsymbol\times}
\newcommand{\grad}{\boldsymbol\nabla}
\newcommand{\divergence}{\grad\dotp}
\newcommand{\curl}{\grad\cross}
%% Simple horiz vectors
\renewcommand{\vector}[1]{\left\langle #1\right\rangle}


\outcome{Compute indefinite integrals}

\title{0.2 Indefinite Integrals}

\begin{document}

\begin{abstract}
We will compute indefinite Integrals
\end{abstract}

\maketitle

\begin{definition}[Indefinite Integral]
\[
\int f(x) \; dx = F(x) + C,
\]
where $F' = f$ and $C$ is an arbitrary constant.

\end{definition}


\section{Basic Indefinite Integrals}

We now present a list of basic indefinite integrals.  In this context, the word \textbf{basic} means that the formula shown
comes directly from its corresponding differentiation formula.


\begin{enumerate}


\item[1.] Constant functions:
$\displaystyle
\int k \; dx = kx + C,
$
where $k$ is any constant.

\item[2.] Power functions ($n \neq -1$):
$\displaystyle
\int x^n \; dx = \frac{x^{n+1}}{n+1} + C
$


\item[3.] Power functions ($n = -1$):
$\displaystyle
 \int x^{-1} \, dx = \int \frac{1}{x} \; dx = \ln|x| + C.
$

\item[4.] Trigonometric functions:

\begin{enumerate}

\item[a.] $\displaystyle{\int \cos x \; dx = \sin x + C}$
\item[b.] $\displaystyle{\int \sin x \; dx = -\cos x + C}$
\item[c.] $\displaystyle{\int \sec^2 x \; dx = \tan(x) + C}$
\item[d.] $\displaystyle{\int \sec x \tan x \; dx = \sec x + C}$
\item[e.] $\displaystyle{\int \csc^2 x \; dx = -\cot x + C}$
\item[f.] $\displaystyle{\int \csc x \cot x \; dx = -\csc x + C}$
 
\end{enumerate}


\item[5.] Natural exponential function:
$\displaystyle
\int e^x \; dx = e^x + C.
$

\item[6.] General exponential function:
$\displaystyle
\int a^x \; dx = \frac{a^x}{\ln a} + C,
$
where $a > 1$.

\item[7.] Inverse trigonometric functions:

\begin{enumerate}

\item[a.] $\displaystyle{\int \frac{1}{\sqrt{1-x^2}} \; dx = \sin^{-1} x + C}$
\item[b.] $\displaystyle{\int \frac{1}{1+x^2} \; dx = \tan^{-1} x + C}$

\end{enumerate}


\end{enumerate}


\section{Minor modifications}

In many applications, the independent variable is multiplied by a constant, $a$. 
We list the effects of such a modification on indefinite integrals.

\begin{enumerate}

\item[8.] $\displaystyle{\int e^{ax} \; dx = \frac{1}{a} \cdot e^{ax} + C}$


\item[9.] $\displaystyle{\int \cos(ax) \; dx = \frac{1}{a} \cdot \sin(ax) + C}$

\item[10.] $\displaystyle{\int \sin(ax) \; dx = -\frac{1}{a} \cdot \cos(ax) + C}$

\item[11.] $\displaystyle{\int \sec^2(ax) \; dx = \frac{1}{a} \cdot \tan(ax) + C}$

\item[12.] $\displaystyle{\int \sec(ax)\tan(ax) \; dx = \frac{1}{a} \cdot \sec(ax) + C}$

\item[13.] $\displaystyle{\int \frac{1}{a^2 + x^2} \; dx = \frac{1}{a} \cdot \tan^{-1}\left(\frac{x}{a}\right) + C}$

\end{enumerate}



\section{Problems}






\begin{problem}(problem 1)
Compute
\begin{hint}
\[
\frac{d}{dx} \cos x = -\sin x
\]
\end{hint}
\begin{hint}
\begin{center}
Do not add the +C to your answer
\end{center}
\end{hint}

\[
\int \sin(x) \ dx =
\answer[given]{-\cos x} \ + C
\]
\end{problem}

\begin{problem}(problem 2)
Compute
\begin{hint}
\[
\frac{d}{dx} \sin(3x) = 3\cos(3x)
\]
\end{hint}
\begin{hint}
The answer is a minor modification of $\sin(3x)$
\end{hint}
\begin{hint}
\begin{center}
Do not add the +C to your answer
\end{center}
\end{hint}

\[
\int \cos(3x) \ dx =
\answer[given]{\frac13 \sin(3x)} \ + C
\]
\end{problem}



\begin{problem}(problem 3)
Compute 
%\[
%\int \csc^2(x) \ dx.
%\]

\begin{hint}
\[
\frac{d}{dx} \cot x = -\csc^2 x
\]
\end{hint}
\begin{hint}
\begin{center}
Do not add the +C to your answer
\end{center}
\end{hint}

\[
\int \csc^2 x \ dx =
\answer[given]{-\cot x} \ + C
\]
\end{problem}

\begin{problem}(problem 4)
Compute 
%\[
%\int \sec^2(4x) \ dx.
%\]

\begin{hint}
\[
\frac{d}{dx} \tan(4x) = 4\sec^2(4x)
\]
\end{hint}
\begin{hint}
The answer is a minor modification of $\tan(4x)$
\end{hint}
\begin{hint}
\begin{center}
Do not add the +C to your answer
\end{center}
\end{hint}

\[
\int \sec^2(4x) \ dx =
\answer[given]{\frac14\tan(4x)} \ + C
\]
\end{problem}




\begin{problem}(problem 5)
Compute 
%\[
%\int \csc(x)\cot(x) \ dx.
%\]

\begin{hint}
\[
\frac{d}{dx} \csc x = -\csc x\cot x
\]
\end{hint}
\begin{hint}
\begin{center}
Do not add the +C to your answer
\end{center}
\end{hint}

\[
\int \csc x\cot x \ dx =
\answer[given]{-\csc x} \ + C
\]
\end{problem}





\begin{problem}(problem 6)
Compute 
%\[
%\int 2^x \ln(2) \ dx.
%\]

\begin{hint}
\[
\frac{d}{dx} 2^x = 2^x \ln 2
\]
\end{hint}
\begin{hint}
\begin{center}
Do not add the +C to your answer
\end{center}
\end{hint}

\[
\int 2^x \ln 2 \ dx =
\answer[given]{2^x} \ + C
\]
\end{problem}

\begin{problem}(problem 7)
Compute 
%\[
%\int e^{5x} \ dx.
%\]

\begin{hint}
\[
\frac{d}{dx} e^{5x} = 5e^{5x}
\]
\end{hint}
\begin{hint}
The answer is a minor modification of $e^{5x}$
\end{hint}
\begin{hint}
\begin{center}
Do not add the +C to your answer
\end{center}
\end{hint}

\[
\int e^{5x} \ dx =
\answer[given]{\frac15 e^{5x}} \ + C
\]
\end{problem}

\begin{problem}(problem 8)
Compute 
%\[
%\int e^{-x/2} \ dx.
%\]

\begin{hint}
\[
\frac{d}{dx} e^{-x/2} = -\frac12 e^{-x/2}
\]
\end{hint}
\begin{hint}
The answer is a minor modification of $e^{-x/2}$
\end{hint}
\begin{hint}
\begin{center}
Do not add the +C to your answer
\end{center}
\end{hint}

\[
\int e^{-x/2} \ dx =
\answer[given]{-2 e^{-x/2}} \ + C
\]
\end{problem}



\begin{problem}(problem 9)
Compute 
%\[
%\int 3x^2 \ dx.
%\]

\begin{hint}
\[
\frac{d}{dx} x^3 = 3x^2
\]
\end{hint}
\begin{hint}
\begin{center}
Do not add the +C to your answer
\end{center}
\end{hint}

\[
\int 3x^2 \ dx =
\answer[given]{x^3} \ + C
\]
\end{problem}





\begin{problem}(problem 10)
Compute 
%\[
%\int \frac{1}{2\sqrt w} \ dw.
%\]

\begin{hint}
\[
\frac{d}{dx} \sqrt x = \frac{1}{2\sqrt x}
\]
\end{hint}
\begin{hint}
Note that the variable is $w$
\end{hint}
\begin{hint}
Write your answer as $w$ to a power.
\end{hint}
\begin{hint}
\begin{center}
Do not add the +C to your answer
\end{center}
\end{hint}

\[
\int \frac{1}{2\sqrt w} \ dw =
\answer[given]{w^{1/2}} \ + C
\]
\end{problem}





\begin{problem}(problem 11)
Compute 
%\[
%\int \frac{1}{1+u^2} \ du.
%\]

\begin{hint}
\[
\frac{d}{dx} \tan^{-1} x = \frac{1}{1+x^2}
\]
\end{hint}
\begin{hint}
Note that the variable is $u$
\end{hint}
\begin{hint}
\begin{center}
Do not add the +C to your answer
\end{center}
\end{hint}

\[
\int \frac{1}{1+u^2} \ du =
\answer[given]{\tan^{-1}(u)} \ + C
\]
\end{problem}




\begin{problem}(problem 12)
Compute 
%\[
%\int \frac{1}{\sqrt{1 - t^2}} \ dt.
%\]

\begin{hint}
\[
\frac{d}{dx} \sin^{-1} x = \frac{1}{\sqrt{1 - x^2}}
\]
\end{hint}
\begin{hint}
Note that the variable is $t$
\end{hint}
\begin{hint}
\begin{center}
Do not add the +C to your answer
\end{center}
\end{hint}

\[
\int \frac{1}{\sqrt{1 - t^2}} \ dt =
\answer[given]{\sin^{-1}(t)} \ + C
\]
\end{problem}




\begin{problem}(problem 13)
Compute 
%\[
%\int x^4 \ dx.
%\]

\begin{hint}
Use the Power Rule with $n=4$
\end{hint}
\begin{hint}
The Power Rule says $\int x^n \ dx = \frac{x^{n+1}}{n+1} +$ C
\end{hint}
\begin{hint}
\begin{center}
Do not add the +C to your answer
\end{center}
\end{hint}

\[
\int x^4 \ dx =
\answer[given]{x^5 /5} \ + C
\]
\end{problem}


\begin{problem}(problem 14)
Compute 
%\[
%\int x^6 \ dx.
%\]

\begin{hint}
Use the Power Rule with $n=6$
\end{hint}
\begin{hint}
The Power Rule says $\int x^n \ dx = \frac{x^{n+1}}{n+1} +$ C
\end{hint}
\begin{hint}
\begin{center}
Do not add the +C to your answer
\end{center}
\end{hint}

\[
\int x^6 \ dx =
\answer[given]{x^7 /7} \ + C
\]
\end{problem}



\begin{problem}(problem 15)
Compute 
%\[
%\int 1 \ dt.
%\]


\begin{hint}
Use the Power Rule with $n=0$
\end{hint}
\begin{hint}
The Power Rule says $\int x^n \ dx = \frac{x^{n+1}}{n+1} +$ C
\end{hint}
\begin{hint}
Note that the variable is $t$
\end{hint}
\begin{hint}
\begin{center}
Do not add the +C to your answer
\end{center}
\end{hint}

\[
\int 1 \ dt =
\answer[given]{t} \ + C
\]
\end{problem}


\begin{problem}(problem 16)
Compute 



\begin{hint}
Use the Power Rule with $n=0$
\end{hint}
\begin{hint}
The Power Rule says $\int x^n \ dx = \frac{x^{n+1}}{n+1} +$ C
\end{hint}
\begin{hint}
Note that the variable is $u$
\end{hint}
\begin{hint}
\begin{center}
Do not add the +C to your answer
\end{center}
\end{hint}

\[
\int 1 \ du =
\answer[given]{u} \ + C
\]
\end{problem}




\begin{problem}(problem 17)
Compute 
%\[
%\int t \ dt.
%\]

\begin{hint}
Use the Power Rule with $n=1$
\end{hint}
\begin{hint}
The Power Rule says $\int x^n \ dx = \frac{x^{n+1}}{n+1} +$ C
\end{hint}
\begin{hint}
Note that the variable is $t$
\end{hint}
\begin{hint}
\begin{center}
Do not add the +C to your answer
\end{center}
\end{hint}

\[
\int t \ dt =
\answer[given]{t^2 /2} \ + C
\]
\end{problem}



\begin{problem}(problem 18)
Compute 
%\[
%\int u \ du.
%\]

\begin{hint}
Use the Power Rule with $n=1$
\end{hint}
\begin{hint}
The Power Rule says $\int x^n \ dx = \frac{x^{n+1}}{n+1} +$ C
\end{hint}
\begin{hint}
Note that the variable is $u$
\end{hint}
\begin{hint}
\begin{center}
Do not add the +C to your answer
\end{center}
\end{hint}

\[
\int u \ du =
\answer[given]{u^2 /2} \ + C
\]
\end{problem}


\begin{problem}(problem 19)
Compute 
%\[
%\int \sqrt[3]x \ dx.
%\]

\begin{hint}
Rational exponents: $\sqrt[3]x = x^{1/3}$
\end{hint}
\begin{hint}
Use the Power Rule with $n=1/3$
\end{hint}
\begin{hint}
The Power Rule says $\int x^n \ dx = \frac{x^{n+1}}{n+1} +$ C
\end{hint}
\begin{hint}
\begin{center}
Do not add the +C to your answer
\end{center}
\end{hint}

\[
\int \sqrt[3]x \ dx =
\answer[given]{3x^{4/3} /4} \ + C
\]
\end{problem}


\begin{problem}(problem 20)
Compute 
%\[
%\int \sqrt[4]x \ dx.
%\]

\begin{hint}
Rational exponents: $\sqrt[4]x = x^{1/4}$
\end{hint}
\begin{hint}
Use the Power Rule with $n=1/4$
\end{hint}
\begin{hint}
The Power Rule says $\int x^n \ dx = \frac{x^{n+1}}{n+1} +$ C
\end{hint}
\begin{hint}
\begin{center}
Do not add the +C to your answer
\end{center}
\end{hint}

\[
\int \sqrt[4]x \ dx =
\answer[given]{4x^{5/4} /5} \ + C
\]
\end{problem}




\begin{problem}(problem 21)
Compute 
%\[
%\int \frac{1}{x^4} \ dx.
%\]

\begin{hint}
Negative exponents: $\frac{1}{x^4} = x^{-4}$
\end{hint}
\begin{hint}
Use the Power Rule with $n=-4$
\end{hint}
\begin{hint}
The Power Rule says $\int x^n \ dx = \frac{x^{n+1}}{n+1} +$ C
\end{hint}
\begin{hint}
\begin{center}
Do not add the +C to your answer
\end{center}
\end{hint}

\[
\int \frac{1}{x^4} \ dx =
\answer[given]{-x^{-3} /3} \ + C
\]
\end{problem}


\begin{problem}(problem 22)
Compute 
%\[
%\int \frac{1}{x^7} \ dx.
%\]

\begin{hint}
Negative exponents: $\frac{1}{x^7} = x^{-7}$
\end{hint}
\begin{hint}
Use the Power Rule with $n=-7$
\end{hint}
\begin{hint}
The Power Rule says $\int x^n \ dx = \frac{x^{n+1}}{n+1} +$ C
\end{hint}
\begin{hint}
\begin{center}
Do not add the +C to your answer
\end{center}
\end{hint}

\[
\int \frac{1}{x^7} \ dx =
\answer[given]{-x^{-6} /6} \ + C
\]
\end{problem}





\begin{problem}(problem 23)
Compute 
%\[
%\int \frac{1}{\sqrt[3]x} \ dx.
%\]

\begin{hint}
Negative rational exponents: $\frac{1}{\sqrt[3]x} = x^{-1/3}$
\end{hint}
\begin{hint}
Use the Power Rule with $n=-1/3$
\end{hint}
\begin{hint}
The Power Rule says $\int x^n \ dx = \frac{x^{n+1}}{n+1} +$ C
\end{hint}
\begin{hint}
\begin{center}
Do not add the +C to your answer
\end{center}
\end{hint}

\[
\int \frac{1}{\sqrt[3]x} \ dx =
\answer[given]{3x^{2/3} /2} \ + C
\]
\end{problem}



\begin{problem}(problem 24)
Compute 
%\[
%\int \frac{1}{\sqrt[6]x} \ dx.
%\]

\begin{hint}
Negative rational exponents: $\frac{1}{\sqrt[6]x} = x^{-1/6}$
\end{hint}
\begin{hint}
Use the Power Rule with $n=-1/6$
\end{hint}
\begin{hint}
The Power Rule says $\int x^n \ dx = \frac{x^{n+1}}{n+1} +$ C
\end{hint}
\begin{hint}
\begin{center}
Do not add the +C to your answer
\end{center}
\end{hint}

\[
\int \frac{1}{\sqrt[6]x} \ dx =
\answer[given]{6x^{5/6} /5} \ + C
\]
\end{problem}



\begin{problem}(problem 25)
Compute 
%\[
%\int \frac{1}{t} \ dt.
%\]

\begin{hint}
Negative exponents: $\frac{1}{x} = x^{-1}$
\end{hint}
\begin{hint}
The Power Rule does not apply when $n = -1$
\end{hint}
\begin{hint}
Recall $\frac{d}{dx} \ln|x| = \frac{1}{x}$
\end{hint}
\begin{hint}
\begin{center}
Do not add the +C to your answer
\end{center}
\end{hint}

\[
\int \frac{1}{t} \ dt =
\answer[given]{\ln|t|} \ + C
\]
\end{problem}




\end{document}









