\documentclass[handout]{ximera}

%% You can put user macros here
%% However, you cannot make new environments



\newcommand{\ffrac}[2]{\frac{\text{\footnotesize $#1$}}{\text{\footnotesize $#2$}}}
\newcommand{\vasymptote}[2][]{
    \draw [densely dashed,#1] ({rel axis cs:0,0} -| {axis cs:#2,0}) -- ({rel axis cs:0,1} -| {axis cs:#2,0});
}


\graphicspath{{./}{firstExample/}}
\usepackage{forest}
\usepackage{amsmath}
\usepackage{amssymb}
\usepackage{array}
\usepackage[makeroom]{cancel} %% for strike outs
\usepackage{pgffor} %% required for integral for loops
\usepackage{tikz}
\usepackage{tikz-cd}
\usepackage{tkz-euclide}
\usetikzlibrary{shapes.multipart}


%\usetkzobj{all}
\tikzstyle geometryDiagrams=[ultra thick,color=blue!50!black]


\usetikzlibrary{arrows}
\tikzset{>=stealth,commutative diagrams/.cd,
  arrow style=tikz,diagrams={>=stealth}} %% cool arrow head
\tikzset{shorten <>/.style={ shorten >=#1, shorten <=#1 } } %% allows shorter vectors

\usetikzlibrary{backgrounds} %% for boxes around graphs
\usetikzlibrary{shapes,positioning}  %% Clouds and stars
\usetikzlibrary{matrix} %% for matrix
\usepgfplotslibrary{polar} %% for polar plots
\usepgfplotslibrary{fillbetween} %% to shade area between curves in TikZ



%\usepackage[width=4.375in, height=7.0in, top=1.0in, papersize={5.5in,8.5in}]{geometry}
%\usepackage[pdftex]{graphicx}
%\usepackage{tipa}
%\usepackage{txfonts}
%\usepackage{textcomp}
%\usepackage{amsthm}
%\usepackage{xy}
%\usepackage{fancyhdr}
%\usepackage{xcolor}
%\usepackage{mathtools} %% for pretty underbrace % Breaks Ximera
%\usepackage{multicol}



\newcommand{\RR}{\mathbb R}
\newcommand{\R}{\mathbb R}
\newcommand{\C}{\mathbb C}
\newcommand{\N}{\mathbb N}
\newcommand{\Z}{\mathbb Z}
\newcommand{\dis}{\displaystyle}
%\renewcommand{\d}{\,d\!}
\renewcommand{\d}{\mathop{}\!d}
\newcommand{\dd}[2][]{\frac{\d #1}{\d #2}}
\newcommand{\pp}[2][]{\frac{\partial #1}{\partial #2}}
\renewcommand{\l}{\ell}
\newcommand{\ddx}{\frac{d}{\d x}}

\newcommand{\zeroOverZero}{\ensuremath{\boldsymbol{\tfrac{0}{0}}}}
\newcommand{\inftyOverInfty}{\ensuremath{\boldsymbol{\tfrac{\infty}{\infty}}}}
\newcommand{\zeroOverInfty}{\ensuremath{\boldsymbol{\tfrac{0}{\infty}}}}
\newcommand{\zeroTimesInfty}{\ensuremath{\small\boldsymbol{0\cdot \infty}}}
\newcommand{\inftyMinusInfty}{\ensuremath{\small\boldsymbol{\infty - \infty}}}
\newcommand{\oneToInfty}{\ensuremath{\boldsymbol{1^\infty}}}
\newcommand{\zeroToZero}{\ensuremath{\boldsymbol{0^0}}}
\newcommand{\inftyToZero}{\ensuremath{\boldsymbol{\infty^0}}}


\newcommand{\numOverZero}{\ensuremath{\boldsymbol{\tfrac{\#}{0}}}}
\newcommand{\dfn}{\textbf}
%\newcommand{\unit}{\,\mathrm}
\newcommand{\unit}{\mathop{}\!\mathrm}
%\newcommand{\eval}[1]{\bigg[ #1 \bigg]}
\newcommand{\eval}[1]{ #1 \bigg|}
\newcommand{\seq}[1]{\left( #1 \right)}
\renewcommand{\epsilon}{\varepsilon}
\renewcommand{\iff}{\Leftrightarrow}

\DeclareMathOperator{\arccot}{arccot}
\DeclareMathOperator{\arcsec}{arcsec}
\DeclareMathOperator{\arccsc}{arccsc}
\DeclareMathOperator{\si}{Si}
\DeclareMathOperator{\proj}{proj}
\DeclareMathOperator{\scal}{scal}
\DeclareMathOperator{\cis}{cis}
\DeclareMathOperator{\Arg}{Arg}
%\DeclareMathOperator{\arg}{arg}
\DeclareMathOperator{\Rep}{Re}
\DeclareMathOperator{\Imp}{Im}
\DeclareMathOperator{\sech}{sech}
\DeclareMathOperator{\csch}{csch}
\DeclareMathOperator{\Log}{Log}

\newcommand{\tightoverset}[2]{% for arrow vec
  \mathop{#2}\limits^{\vbox to -.5ex{\kern-0.75ex\hbox{$#1$}\vss}}}
\newcommand{\arrowvec}{\overrightarrow}
\renewcommand{\vec}{\mathbf}
\newcommand{\veci}{{\boldsymbol{\hat{\imath}}}}
\newcommand{\vecj}{{\boldsymbol{\hat{\jmath}}}}
\newcommand{\veck}{{\boldsymbol{\hat{k}}}}
\newcommand{\vecl}{\boldsymbol{\l}}
\newcommand{\utan}{\vec{\hat{t}}}
\newcommand{\unormal}{\vec{\hat{n}}}
\newcommand{\ubinormal}{\vec{\hat{b}}}

\newcommand{\dotp}{\bullet}
\newcommand{\cross}{\boldsymbol\times}
\newcommand{\grad}{\boldsymbol\nabla}
\newcommand{\divergence}{\grad\dotp}
\newcommand{\curl}{\grad\cross}
%% Simple horiz vectors
\renewcommand{\vector}[1]{\left\langle #1\right\rangle}


\pgfplotsset{compat=1.13}

\outcome{Introduce complex trigonometric functions}

\title{2.3 Complex Trigonometric Functions}

\begin{document}

\begin{abstract}
We define and discuss the complex trigonometric functions.
\end{abstract}

\maketitle

\section{The Complex Cosine}

To define $f(z) = \cos z$ we will use 
\link[Maclaurin series]{https://en.wikipedia.org/wiki/Taylor_series} 
and the 
\link[sum identity for the cosine]{https://en.wikipedia.org/wiki/List_of_trigonometric_identities}.


% Link   
%                \link[Euler's identity]{https://en.wikipedia.org/wiki/Euler's_identity}
%command

The series of interest are:
\begin{equation*}
\begin{aligned}[c]
&\sin(x) =\sum_{n=0}^\infty (-1)^n \frac{x^{2n+1}}{(2n+1)!}\\[15pt]
&\sinh(x)=\sum_{n=0}^\infty \frac{x^{2n+1}}{(2n+1)!}\\
\end{aligned}
\quad \quad\quad
\begin{aligned}[c]
&\cos(x) =\sum_{n=0}^\infty (-1)^n \frac{x^{2n}}{(2n)!}\\[15pt]
&\cosh(x)=\sum_{n=0}^\infty \frac{x^{2n}}{(2n)!}\\
\end{aligned}
\end{equation*}


%\begin{tikzpicture}
%\node[draw,brown!50!green!50] at (0,2) {$\sin(x) =\sum_{n=0}^\infty (-1)^n \frac{x^{2n+1}}{(2n+1)!}$};
%\node[draw,brown!70!purple!80] at (5,2) {$\cos(x) =\sum_{n=0}^\infty (-1)^n \frac{x^{2n}}{(2n)!}$};
%\node[draw,blue!50!brown!50] at (0,0) {$\sinh(x)=\sum_{n=0}^\infty \frac{x^{2n+1}}{(2n+1)!}$};
%\node[draw,blue] at (5,0) {$\cosh(x)=\sum_{n=0}^\infty \frac{x^{2n}}{(2n)!}$};
%\end{tikzpicture}




and the sum identity for the cosine is:
\[
\cos(\alpha + \beta) = \cos\alpha \cos \beta - \sin \alpha \sin \beta
\]
We get the ball rolling by allowing an imaginary term in the sum identity:
\[
\cos(x+iy) = \cos(x) \cos(iy) - \sin(x) \sin(iy)
\]
Next, we define the sine and cosine of a purely imaginary angle using their respective power series:
%Substituting $iy$ for $x$ in the power series for the sine and cosine
%yields a connection between  trig functions of purely imaginary angles and the hyperbolic functions: 
\[
\cos(iy) = \sum_{n=0}^\infty (-1)^n \frac{(iy)^{2n}}{(2n)!} 
\]
and
\[
\sin(iy) = \sum_{n=0}^\infty (-1)^n \frac{(iy)^{2n+1}}{(2n+1)!} 
\]
These power series can be simplified into hyperbolic functions (!) by noting that $(-1)^n i^{2n} = 1$ for all $n$: 
\[
\cos(iy) = \sum_{n=0}^\infty  \frac{y^{2n}}{(2n)!} = \cosh(y)
\]
and
\[
\sin(iy) = i\sum_{n=0}^\infty \frac {y^{2n+1}}{(2n+1)!}= i\sinh(y)
\]

Substituting these into the sum identity establishes
%the expressions for $\cos(iy)$ and $\sin(iy)$ into the sum identity for the cosine leads to
\[
\cos(x+iy) = \cos(x) \cosh(y) - i\sin(x) \sinh(y)
\]
We use this as our definition of the complex cosine.
\begin{definition}[The Complex Cosine]
For all $z \in \C$ we define the complex cosine by
%$f(z) = \cos z: \C \to \C$
\[
\cos(z) = \cos(x+iy) = \cos(x) \cosh(y) - i\sin(x) \sinh(y)
\]
\end{definition}


\section{The Complex Sine}

\begin{question}
The sum identity for the sine function states that
\[
\sin(\alpha + \beta) = \sin \alpha \cos \beta + \cos \alpha \sin \beta
\]
for all angles $\alpha$ and $\beta$.

%Allowing an imaginary angle in the sum identity using an imaginary angle:
%How would you define the complex sine function in light of the identity for the sin of a sum? \\
\begin{hint}
\[
\sin{z} = \sin(x+iy) = \sin(x) \cos(iy) + \cos(x) \sin(iy) =
\]
\end{hint}
\begin{hint}
\[
 \cos(iy) = \cosh(y)
\]
\end{hint}
\begin{hint}
\[
\sin(iy) = i \sinh(y)
\]
\end{hint}


Based on the results obtained in the method for defining the complex cosine, which of the following is the definition 
of the complex sine function $f(z) = \sin z$?
\begin{multipleChoice}
\choice{$\sin(x) \sinh(y) - i\cos(x) \cosh(y)$}\\
\choice{$\sin(x) \sinh(y) + i\cos(x) \cosh(y)$}\\
\choice{$\sin(x) \cosh(y) - i\cos(x) \sinh(y)$}\\
\choice[correct]{$\sin(x) \cosh(y) + i\cos(x) \sinh(y)$ }
\end{multipleChoice}
\end{question}


\begin{example}[example 1]
Solve for $z: \; \sin(z) = 2$.\\
Since
\[
\sin(z) = \sin(x) \cosh(y) + i\cos(x) \sinh(y)
\]
we need 
\[
\sin(x) \cosh(y) =2 \;\; \mbox{and} \;\; \cos(x) \sinh(y) = 0
\]
simultaneously. The second equation gives
\[
x = \pi/2 + n\pi \;\; \mbox{or} \;\; y = 0
\]
We proceed on a case by case basis. If $y = 0$, then since
\[
\cosh 0 = 1
\]
we would be led to $\sin(x) = 2$ which has no solutions.\\
Next, if $x = \pi/2 + n\pi$ where $n=2k$ is even, then 
\[
\sin(x) = \sin\left(\frac{\pi}{2} + 2k\pi\right) = 1
\]
and we arrive at 
\[
\cosh(y) = 2
\]
which has two solutions (verify)
\[
y = \ln\left(2\pm \sqrt 3 \right)
\]
Finally, if $x = \pi/2 + n\pi$ where $n=2k+1$ is odd, then 
\[
\sin(x) = \sin\left(\frac{\pi}{2} + (2k+1)\pi\right) = -1
\]
and we arrive at 
\[
\cosh(y) = -2
\]
which has no solutions (verify).
Thus the equation $\sin(z) = 2$ has as its solutions
\[
z = \left[\frac{\pi}{2} + 2k\pi\right] + i\left[\ln\left(2\pm \sqrt 3\right)\right]
\]
where $k$ is any integer.
\end{example}

\begin{problem}(problem 1)
Solve the following equations for $z$.
\begin{align*}
i) & \; \sin(z) = -5\\
ii) & \;  \sin(z) = i\\
iii) & \;  \cos(z) = 5\\
iv) & \;  \cos(z) = -2i\\
\end{align*}
\end{problem}


Here is a video solution of one part of problem 1, part iv:\\
\begin{foldable}
\youtube{CX9sOecXTkw}
\end{foldable}


\section{Identities}
\subsection{Periodicity}
Since the real sine and cosine functions are $2\pi$ periodic, so are their complex extensions.
\begin{proposition}
\[
\sin(z + 2\pi) = \sin(z) \;\; \mbox{and} \;\; \cos(z+2\pi) = \cos(z)
\]
\end{proposition}
The periodicity follows immediately from the definition:
\[
\sin(z + 2\pi) = \sin(x+2\pi)\cosh(y) + i\cos(x+2\pi) \sinh(y) = \sin(x)\cosh(y) + i\cos(x) \sinh(y) =\sin(z)
\]
The periodicity of the cosine is proved similarly (verify).\\
\subsection{Evenness and Oddness}
Recall that $\sin(x)$ and $\sinh(x)$ are odd functions and that $\cos(x)$ and $\cosh(x)$ are even functions.
As a result, $\sin(z)$ is an odd function and $\cos(z)$ is an even function.

\begin{proposition}
\[
\sin(-z) = -\sin(z) \;\; \mbox{and} \;\; \cos(-z) = \cos(z)
\]
\end{proposition}
To obtain the first equation, we have
\begin{align*}
\sin(-z) &= \sin(-x-iy) = \sin(-x)\cosh(-y) + i\cos(-x) \sinh(-y)\\
  &= -\sin(x)\cosh(y) -i\cos(x) \sinh(y) = -\sin(z)
\end{align*}
The evenness of the complex cosine is demonstrated similarly (verify).

\subsection{Shift by $\pi$}
\begin{proposition}
\[
\sin(z + \pi) = -\sin(z) \;\; \mbox{and} \;\; \cos(z+\pi) = -\cos(z)
\]
\end{proposition}
The proof of these equations follows directly from their real counterparts (verify).

\subsection{Relation to the Complex Exponential}
\begin{proposition}
\[
\sin(z) = \frac{e^{iz} - e^{-iz}}{2i} \;\; \mbox{and} \;\;  \cos(z) = \frac{e^{iz} + e^{-iz}}{2}
\]
\end{proposition}
To prove the first equation, we begin with
\[
e^{iz} = e^{-y+ix} = e^{-y}\cos(x) + ie^{-y}\sin(x)
\]
and
\[
e^{-iz} = e^{y-ix} = e^{y}\cos(x) - ie^{y}\sin(x)
\]
Subtracting the second of these from the first, we get
\begin{align*}
e^{iz} - e^{-iz} &= \left[e^{-y}-e^y\right]\cos(x) + i\left[e^{-y}+e^y\right]\sin(x)\\
&=-2\cos(x) \sinh(y) +2i \sin(x)\cosh(y)\\
& = 2i\left[\sin(x)\cosh(y) + i \cos(x)\sinh(y)\right]\\
&=2i\sin(z)
\end{align*}
The result follows by dividing by $2i$.
The proof of the second equation is similar (verify).

Here is a video demonstration of the equation $\cos(z) = \frac{e^{iz}+e^{-iz}}{2}$:\\
\begin{foldable}
\youtube{0zfQWxKLbo0}
\end{foldable}

\begin{corollary}(Pythagorean Identity)
  For all $z \in \C$
  \[\cos^2(z) + \sin^2(z) = 1\]
  
\end{corollary}
\begin{proof}
  Using the exponential forms of sine and cosine gives:
  \begin{align*}
    \cos^2(z) + \sin^2(z) &= \left(\frac{e^{iz} + e^{-iz}}{2}\right)^2 + \left(\frac{e^{iz} - e^{-iz}}{2i}  \right)^2\\
                         &= \frac{e^{2iz} + 2 + e^{-2iz}}{4} + \frac{e^{2iz} - 2 + e^{-2iz}}{-4}\\
                         &= \frac{e^{2iz} + 2 + e^{-2iz}}{4} - \frac{e^{2iz} - 2 + e^{-2iz}}{4}\\
                         &= \frac{44} = 1.                
  \end{align*}
\end{proof}
\subsection{Co-functions}
\begin{proposition}
\[
\sin\left(\frac{\pi}{2}-z\right) = \cos(z) \;\; \mbox{and} \;\; \cos\left(\frac{\pi}{2}-z\right) = \sin(z)
\]
\end{proposition}
These identities follow directly from their counterparts for the real sine and cosine functions.

\subsection{Sum and Difference Identities}
\begin{proposition}
\[
\sin(z_1 + z_2) = \sin(z_1)\cos(z_2) + \cos(z_1)\sin(z_2)
\]
\[
\sin(z_1 - z_2) = \sin(z_1)\cos(z_2) - \cos(z_1)\sin(z_2)
\]
\[
\cos(z_1 + z_2) = \cos(z_1)\cos(z_2) - \sin(z_1)\sin(z_2)
\]
\[
\cos(z_1 - z_2) = \cos(z_1)\cos(z_2) + \sin(z_1)\sin(z_2)
\]
\end{proposition}
To prove the first equation, we rewrite the right hand side using the complex exponential. The first term is
\begin{align*}
\sin(z_1)\cos(z_2) &= \frac{e^{iz_1} - e^{- iz_1}}{2i} \cdot \frac{e^{iz_2} + e^{-iz_2}}{2}\\
&= \frac{1}{4i}\left[e^{i(z_1+ z_2)} + e^{i(z_1 - z_2)} - e^{-i(z_1 - z_2)} - e^{-i(z_1+ z_2)}\right]
\end{align*}
The second term is
\begin{align*}
\cos(z_1)\sin(z_2) &= \frac{e^{iz_1} + e^{- iz_1}}{2} \cdot \frac{e^{iz_2} - e^{-iz_2}}{2i}\\
&= \frac{1}{4i}\left[e^{i(z_1+ z_2)} - e^{i(z_1 - z_2)} + e^{-i(z_1 - z_2)} - e^{-i(z_1+ z_2)}\right]
\end{align*}
Adding these, we get
\[
\sin(z_1)\cos(z_2) + \cos(z_1)\sin(z_2)  = \frac{1}{4i}\left[2e^{i(z_1+ z_2)} - 2e^{-i(z_1+ z_2)}\right]
\]
\[
= \sin(z_1 + z_2)
\]


%&= e^{-y_1 -y_2}\cis(x) - e^{y_1 +y_2}\cis(-x)\\
%&=\left[e^{-y_1 -y_2} - e^{y_1 +y_2}\right]\cos(x) + 

%\end{align*}

The second equation follows from the first by replacing $z_2$ with $-z_2$ and using evenness and oddness.
The third and fourth equations are proved in the same manner as the first and second (verify).


\section{The Other Trig functions}
The other four trigonometric functions are defined in terms of the sine and cosine.

\begin{definition}
\[
\tan z = \frac{\sin(z)}{\cos(z)} \;\; \mbox{and} \;\; \cot(z) = \frac{\cos(z)}{\sin(z)}
\]
\[
\sec z = \frac{1}{\cos(z)} \;\; \mbox{and} \;\; \csc(z) = \frac{1}{\cos(z)}
\]
\end{definition}
The functions $\tan(z)$ and $\cot(z)$ are $\pi$-periodic
and the functions $\sec(z)$ and $\csc(z)$ are $2\pi$-periodic (verify).
The Pythagorean Identity for the sine and cosine gives rise to two other Pythagorean identities:
\[
1+ \tan^2(z) = \sec^2(z)
\]
and
\[
1 + \cot^2(z) = \csc^2(z)
\]

\end{document}







