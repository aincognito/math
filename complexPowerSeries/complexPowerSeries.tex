\documentclass[handout]{ximera}

%% You can put user macros here
%% However, you cannot make new environments



\newcommand{\ffrac}[2]{\frac{\text{\footnotesize $#1$}}{\text{\footnotesize $#2$}}}
\newcommand{\vasymptote}[2][]{
    \draw [densely dashed,#1] ({rel axis cs:0,0} -| {axis cs:#2,0}) -- ({rel axis cs:0,1} -| {axis cs:#2,0});
}


%\usepackage{tcolorbox} %%Needed for Derivative Definition supposedly and product rule, natural exp log, quotient rule, inverse trig, rates of change


% \graphicspath{{./}{firstExample/}}
% \usepackage{forest}
\usepackage{amsmath}
\usepackage{amssymb}
\usepackage{array}
\usepackage[makeroom]{cancel} %% for strike outs
\usepackage{pgffor} %% required for integral for loops
\usepackage{tikz}
\usepackage{tikz-cd}
\usepackage{tkz-euclide}
\usetikzlibrary{shapes.multipart}


% \usetkzobj{all}
\tikzstyle geometryDiagrams=[ultra thick,color=blue!50!black]


\usetikzlibrary{arrows}
\tikzset{>=stealth,commutative diagrams/.cd,
  arrow style=tikz,diagrams={>=stealth}} %% cool arrow head
\tikzset{shorten <>/.style={ shorten >=#1, shorten <=#1 } } %% allows shorter vectors

\usetikzlibrary{backgrounds} %% for boxes around graphs
\usetikzlibrary{shapes,positioning}  %% Clouds and stars
\usetikzlibrary{matrix} %% for matrix
\usepgfplotslibrary{polar} %% for polar plots
\usepgfplotslibrary{fillbetween} %% to shade area between curves in TikZ



%\usepackage[width=4.375in, height=7.0in, top=1.0in, papersize={5.5in,8.5in}]{geometry}
%\usepackage[pdftex]{graphicx}
%\usepackage{tipa}
%\usepackage{txfonts}
%\usepackage{textcomp}
%\usepackage{amsthm}
%\usepackage{xy}
%\usepackage{fancyhdr}
%\usepackage{xcolor}
%\usepackage{mathtools} %% for pretty underbrace % Breaks Ximera
%\usepackage{multicol}



\newcommand{\RR}{\mathbb R}
\newcommand{\R}{\mathbb R}
\newcommand{\C}{\mathbb C}
\newcommand{\N}{\mathbb N}
\newcommand{\Z}{\mathbb Z}
\newcommand{\dis}{\displaystyle}
%\renewcommand{\d}{\,d\!}
\renewcommand{\d}{\mathop{}\!d}
\newcommand{\dd}[2][]{\frac{\d #1}{\d #2}}
\newcommand{\pp}[2][]{\frac{\partial #1}{\partial #2}}
\renewcommand{\l}{\ell}
\newcommand{\ddx}{\frac{d}{\d x}}
\newcommand{\ppx}{\frac{\partial}{\partial x}}
\newcommand{\ppy}{\frac{\partial}{\partial y}}

\newcommand{\zeroOverZero}{\ensuremath{\boldsymbol{\tfrac{0}{0}}}}
\newcommand{\inftyOverInfty}{\ensuremath{\boldsymbol{\tfrac{\infty}{\infty}}}}
\newcommand{\zeroOverInfty}{\ensuremath{\boldsymbol{\tfrac{0}{\infty}}}}
\newcommand{\zeroTimesInfty}{\ensuremath{\small\boldsymbol{0\cdot \infty}}}
\newcommand{\inftyMinusInfty}{\ensuremath{\small\boldsymbol{\infty - \infty}}}
\newcommand{\oneToInfty}{\ensuremath{\boldsymbol{1^\infty}}}
\newcommand{\zeroToZero}{\ensuremath{\boldsymbol{0^0}}}
\newcommand{\inftyToZero}{\ensuremath{\boldsymbol{\infty^0}}}


\newcommand{\numOverZero}{\ensuremath{\boldsymbol{\tfrac{\#}{0}}}}
\newcommand{\dfn}{\textbf}
%\newcommand{\unit}{\,\mathrm}
\newcommand{\unit}{\mathop{}\!\mathrm}
%\newcommand{\eval}[1]{\bigg[ #1 \bigg]}
\newcommand{\eval}[1]{ #1 \bigg|}
\newcommand{\seq}[1]{\left( #1 \right)}
\renewcommand{\epsilon}{\varepsilon}
\renewcommand{\iff}{\Leftrightarrow}

\DeclareMathOperator{\arccot}{arccot}
\DeclareMathOperator{\arcsec}{arcsec}
\DeclareMathOperator{\arccsc}{arccsc}
\DeclareMathOperator{\si}{Si}
\DeclareMathOperator{\proj}{proj}
\DeclareMathOperator{\scal}{scal}
\DeclareMathOperator{\cis}{cis}
\DeclareMathOperator{\Arg}{Arg}
%\DeclareMathOperator{\arg}{arg}
\DeclareMathOperator{\Rep}{Re}
\DeclareMathOperator{\Imp}{Im}
\DeclareMathOperator{\sech}{sech}
\DeclareMathOperator{\csch}{csch}
\DeclareMathOperator{\Log}{Log}

\newcommand{\tightoverset}[2]{% for arrow vec
  \mathop{#2}\limits^{\vbox to -.5ex{\kern-0.75ex\hbox{$#1$}\vss}}}
\newcommand{\arrowvec}{\overrightarrow}
\renewcommand{\vec}{\mathbf}
\newcommand{\veci}{{\boldsymbol{\hat{\imath}}}}
\newcommand{\vecj}{{\boldsymbol{\hat{\jmath}}}}
\newcommand{\veck}{{\boldsymbol{\hat{k}}}}
\newcommand{\vecl}{\boldsymbol{\l}}
\newcommand{\utan}{\vec{\hat{t}}}
\newcommand{\unormal}{\vec{\hat{n}}}
\newcommand{\ubinormal}{\vec{\hat{b}}}

\newcommand{\dotp}{\bullet}
\newcommand{\cross}{\boldsymbol\times}
\newcommand{\grad}{\boldsymbol\nabla}
\newcommand{\divergence}{\grad\dotp}
\newcommand{\curl}{\grad\cross}
%% Simple horiz vectors
\renewcommand{\vector}[1]{\left\langle #1\right\rangle}


\pgfplotsset{compat=1.13}

\outcome{Determine the radius of convergence of complex power series}

\title{4.3 Power Series}

\begin{document}

\begin{abstract}
We determine radius of convergence of complex power series.
\end{abstract}

\maketitle


\begin{definition} 
A {\bf complex power series centered at} ${\bf z_0}$ is a complex series of the form
\[
\sum_{k=0}^\infty c_k(z-z_0)^k = c_0 + c_1 (z-z_0) + c_2 (z-z_0)^2 + c_3 (z-z_0)^3 + \cdots
\]
\end{definition}
Since a complex power series contains a variable, the question is, for which values of the variable, $z$, 
does the complex power series converge? This question is answered by using the root test, as described in the following theorem.

\begin{theorem}
Let $\dis L = \limsup_{k \to 0} \sqrt[k]{|c_k|}$ and let $\dis \rho = \frac{1}{L}$.
Then complex power series centered at $z_0$ given by
\[
\sum_{k=0}^\infty c_k(z-z_0)^k
\]
will converge for 
$z \in D(z_0, \rho)$ and diverge for $|z-z_0| > \rho$.\\
The radius, $\rho$, is called the radius of convergence of the complex power series.\\
If $L = 0$, then $\rho$ is understood to be infinity, i.e., the complex power series converges for all $z$.\\
If $L = \infty$, then the power series converges only at its center, $z_0$.\\
A complex power series may converge or diverge points on the circle $C(z_0, \rho)$.
\end{theorem}
\begin{remark}
The limit $L$, can also be computed from the ratio test as 
\[
L = \limsup_{k \to \infty} \left|\frac{c_{k+1}}{c_k}\right|
\]
provided $c_k \neq 0$ for all but finitely many values of $k$
\end{remark}


\begin{proof}
From the root test, the series  $\dis \sum_{k=0}^\infty c_k(z-z_0)^k$ converges if $z$ satisfies the inequality
\[
\limsup_{k \to 0} \sqrt[k]{|c_k(z-z_0)^k|} = |z-z_0| \limsup_{k \to 0} \sqrt[k]{|c_k|} = |z-z_0| \cdot L < 1
\]
If $L =0$, then the series converges for all $z$. If $L \neq 0$, then the series converges if $|z-z_0| < \frac{1}{L} = \rho$.
Moreover, the root test also shows that if $|z-z_0|  > \frac{1}{L} = \rho$ then the series diverges.
\end{proof}

\begin{example}[example 1]
Find the disk of convergence of the complex power series: $\dis \sum_{k=1}^\infty \frac{i^k}{k}(z+i)^k$\\
We first compute the limit superior from the root test:
\[
L = \limsup_{k \to \infty} \sqrt[k]{|c_k|} = \limsup_{k \to \infty} \sqrt[k]{\left|\frac{i^k}{k}\right|} = \limsup_{k \to \infty} \frac{1}{\sqrt[k]{k}} = 1
\]
Therefore the radius of convergence is $\rho = 1/L = 1$ and since the center of the power series is $-i$, the disk of convergence is $D(-i, 1)$.
\end{example}

\begin{problem}(problem 1)
Find the disk of convergence of the complex power series: $\dis \sum_{k=1}^\infty \frac{e^{ik}}{k^2}(z-2i)^k$\\
The center of the power series is $\; \answer{2i}$\\
The limit superior of the coefficients from the root test (or ratio test) is:
\[
\limsup_{k \to \infty} \sqrt[k]{|c_k|} = \; \answer{1}
\]
The disk of convergence is:
\begin{multipleChoice}
\choice{$D(-2i, 1)$}\\
\choice{$D(2i, -1)$}\\
\choice[correct]{$D(2i, 1)$}\\
\end{multipleChoice}
\end{problem}

On its disk of convergence, a power series is an analytic function.
\begin{theorem}
Suppose the complex power series
\[
f(z) = \sum_{k=0}^\infty c_k(z-z_0)^k
\]
converges on $D(z_0, \rho)$ for some $\rho > 0$.
Then $f$ is differentiable on $D(z_0, \rho)$ and 
\[
f'(z) = \sum_{k=1}^\infty kc_k(z-z_0)^{k-1}
\]
Moreover, $f'$ has the same disk of convergence as $f$.
\end{theorem}
\begin{remark}
The formula for $f'$ shows that a power series can be differentiated term by term.
\end{remark}
\begin{corollary}
Suppose the complex power series
\[
f(z) = \sum_{k=0}^\infty c_k(z-z_0)^k
\]
converges on $D(z_0, \rho)$ for some $\rho > 0$.
Then $f$ is infinitely differentiable on $D(z_0, \rho)$
\end{corollary}

\begin{proof}
Since $f'$ is a power series with the same radius of convergence as $f$, the theorem can be applied to $f'$. Thus,
$f'$ is differentiable and $f''$ has the same radius of convergence as $f'$. Continuing inductively, $f$
has derivatives of al orders, i.e., $f$ is infinitely differentiable on $D(z_0, \rho)$.
\end{proof}

\begin{corollary}
Suppose the complex power series
\[
f(z) = \sum_{k=0}^\infty c_k(z-z_0)^k
\]
converges on $D(z_0, \rho)$ for some $\rho > 0$. Then we have the following formula for the coefficients:
\[
c_k = \frac{f^{(k)}(z_0)}{k!}
\]
\end{corollary}
\begin{proof}
Since $f$ is infinitely differentiable and the derivative is computed term by term, we have
\[
f^{(j)}(z) = \sum_{k=j}^\infty k(k-1)(k-2)\cdots(k-j+1)c_k(z-z_0)^{k-j}
\]
Plugging in $z_0$ for $z$ and noting that all of the terms of the series are zero except the first (where $k = j$),
we obtain
\[
f^{(j)}(z_0) = j!c_j
\]
Note that we treat the zero power in the first term as 1 when $z=z_0$.
\end{proof}

\begin{example}[example 2]
Find the derivative of the power series function $\dis f(z) = \sum_{k=0}^\infty \frac{z^k}{k!}$\\
The ratio test shows that this series converges for all $z \in \C$. It's derivative is
\[
f'(z) = \sum_{k=1}^\infty\frac{kz^{k-1}}{k!}= \sum_{k=1}^\infty \frac{z^{k-1}}{(k-1)!}
\]
Re-indexing so that $k$ starts at $0$ instead of $1$ (i.e. replacing $k-1$ with $k$), yields
\[
f'(z) = \sum_{k=0}^\infty \frac{z^k}{k!}
\]
But this is the same series as the original function! What function is this?
\end{example}

\begin{problem}(problem 2a)
Find the derivative of the power series function $\dis f(z) = \sum_{k=0}^\infty z^k$\\
$\dis f'(z) = \sum_{k=0}^\infty \answer{(k+1)z^k}$
\end{problem}

\begin{problem}(problem 2b)
Find the derivative of the power series function $\dis f(z) = \sum_{k=0}^\infty \frac{(-1)^k}{(2k+1)!}z^{2k+1}$\\
$\dis f'(z) = \sum_{k=0}^\infty \answer{\frac{(-1)^k}{(2k)!}z^{2k}}$
\end{problem}



\end{document}




