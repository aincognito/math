\documentclass[handout]{ximera}

%% You can put user macros here
%% However, you cannot make new environments



\newcommand{\ffrac}[2]{\frac{\text{\footnotesize $#1$}}{\text{\footnotesize $#2$}}}
\newcommand{\vasymptote}[2][]{
    \draw [densely dashed,#1] ({rel axis cs:0,0} -| {axis cs:#2,0}) -- ({rel axis cs:0,1} -| {axis cs:#2,0});
}


\graphicspath{{./}{firstExample/}}
\usepackage{forest}
\usepackage{amsmath}
\usepackage{amssymb}
\usepackage{array}
\usepackage[makeroom]{cancel} %% for strike outs
\usepackage{pgffor} %% required for integral for loops
\usepackage{tikz}
\usepackage{tikz-cd}
\usepackage{tkz-euclide}
\usetikzlibrary{shapes.multipart}


%\usetkzobj{all}
\tikzstyle geometryDiagrams=[ultra thick,color=blue!50!black]


\usetikzlibrary{arrows}
\tikzset{>=stealth,commutative diagrams/.cd,
  arrow style=tikz,diagrams={>=stealth}} %% cool arrow head
\tikzset{shorten <>/.style={ shorten >=#1, shorten <=#1 } } %% allows shorter vectors

\usetikzlibrary{backgrounds} %% for boxes around graphs
\usetikzlibrary{shapes,positioning}  %% Clouds and stars
\usetikzlibrary{matrix} %% for matrix
\usepgfplotslibrary{polar} %% for polar plots
\usepgfplotslibrary{fillbetween} %% to shade area between curves in TikZ



%\usepackage[width=4.375in, height=7.0in, top=1.0in, papersize={5.5in,8.5in}]{geometry}
%\usepackage[pdftex]{graphicx}
%\usepackage{tipa}
%\usepackage{txfonts}
%\usepackage{textcomp}
%\usepackage{amsthm}
%\usepackage{xy}
%\usepackage{fancyhdr}
%\usepackage{xcolor}
%\usepackage{mathtools} %% for pretty underbrace % Breaks Ximera
%\usepackage{multicol}



\newcommand{\RR}{\mathbb R}
\newcommand{\R}{\mathbb R}
\newcommand{\C}{\mathbb C}
\newcommand{\N}{\mathbb N}
\newcommand{\Z}{\mathbb Z}
\newcommand{\dis}{\displaystyle}
%\renewcommand{\d}{\,d\!}
\renewcommand{\d}{\mathop{}\!d}
\newcommand{\dd}[2][]{\frac{\d #1}{\d #2}}
\newcommand{\pp}[2][]{\frac{\partial #1}{\partial #2}}
\renewcommand{\l}{\ell}
\newcommand{\ddx}{\frac{d}{\d x}}

\newcommand{\zeroOverZero}{\ensuremath{\boldsymbol{\tfrac{0}{0}}}}
\newcommand{\inftyOverInfty}{\ensuremath{\boldsymbol{\tfrac{\infty}{\infty}}}}
\newcommand{\zeroOverInfty}{\ensuremath{\boldsymbol{\tfrac{0}{\infty}}}}
\newcommand{\zeroTimesInfty}{\ensuremath{\small\boldsymbol{0\cdot \infty}}}
\newcommand{\inftyMinusInfty}{\ensuremath{\small\boldsymbol{\infty - \infty}}}
\newcommand{\oneToInfty}{\ensuremath{\boldsymbol{1^\infty}}}
\newcommand{\zeroToZero}{\ensuremath{\boldsymbol{0^0}}}
\newcommand{\inftyToZero}{\ensuremath{\boldsymbol{\infty^0}}}


\newcommand{\numOverZero}{\ensuremath{\boldsymbol{\tfrac{\#}{0}}}}
\newcommand{\dfn}{\textbf}
%\newcommand{\unit}{\,\mathrm}
\newcommand{\unit}{\mathop{}\!\mathrm}
%\newcommand{\eval}[1]{\bigg[ #1 \bigg]}
\newcommand{\eval}[1]{ #1 \bigg|}
\newcommand{\seq}[1]{\left( #1 \right)}
\renewcommand{\epsilon}{\varepsilon}
\renewcommand{\iff}{\Leftrightarrow}

\DeclareMathOperator{\arccot}{arccot}
\DeclareMathOperator{\arcsec}{arcsec}
\DeclareMathOperator{\arccsc}{arccsc}
\DeclareMathOperator{\si}{Si}
\DeclareMathOperator{\proj}{proj}
\DeclareMathOperator{\scal}{scal}
\DeclareMathOperator{\cis}{cis}
\DeclareMathOperator{\Arg}{Arg}
%\DeclareMathOperator{\arg}{arg}
\DeclareMathOperator{\Rep}{Re}
\DeclareMathOperator{\Imp}{Im}
\DeclareMathOperator{\sech}{sech}
\DeclareMathOperator{\csch}{csch}
\DeclareMathOperator{\Log}{Log}

\newcommand{\tightoverset}[2]{% for arrow vec
  \mathop{#2}\limits^{\vbox to -.5ex{\kern-0.75ex\hbox{$#1$}\vss}}}
\newcommand{\arrowvec}{\overrightarrow}
\renewcommand{\vec}{\mathbf}
\newcommand{\veci}{{\boldsymbol{\hat{\imath}}}}
\newcommand{\vecj}{{\boldsymbol{\hat{\jmath}}}}
\newcommand{\veck}{{\boldsymbol{\hat{k}}}}
\newcommand{\vecl}{\boldsymbol{\l}}
\newcommand{\utan}{\vec{\hat{t}}}
\newcommand{\unormal}{\vec{\hat{n}}}
\newcommand{\ubinormal}{\vec{\hat{b}}}

\newcommand{\dotp}{\bullet}
\newcommand{\cross}{\boldsymbol\times}
\newcommand{\grad}{\boldsymbol\nabla}
\newcommand{\divergence}{\grad\dotp}
\newcommand{\curl}{\grad\cross}
%% Simple horiz vectors
\renewcommand{\vector}[1]{\left\langle #1\right\rangle}


\outcome{Compute integrals involving powers and products of trigonometric functions}

\title{1.3 Trigonometric Integrals}

\begin{document}

\begin{abstract}
We compute integrals involving powers and products of trigonometric functions.
\end{abstract}

\maketitle

\section{Trigonometric Identities}

The following are the Pythagorean Trigonometric Identities 
(named for \link[Pythagoras of Samos]{https://en.wikipedia.org/wiki/Pythagoras}) which hold for all angles,
$\theta$, in the domains of the functions involved:
\[
\sin^2\theta + \cos^2\theta = 1,
\]
\[
1 + \tan^2\theta = \sec^2\theta,
\]
and
\[
1 + \cot^2\theta = \csc^2\theta.
\]


Next, we have the half-angle formulas:

\[
\sin^2\theta = \frac{1-\cos(2\theta)}{2},
\]

and

\[
\cos^2\theta = \frac{1+\cos(2\theta)}{2}.
\]
We will find the half-angle formulas useful for integrating even powers of sine and cosine.

\section{Integrating Powers of Sine and Cosine}

In this section, we compute integrals of the form
\[
\int \sin^m x \cos^n x \; dx
\]
The method we use depends on whether $m$ and $n$ are even or odd.

\subsection{Case 1: $m$ or $n$ is odd}
In this case, the technique involves preparing and performing a $u$-substitution. The key is to use the Pythagorean 
Identity $\sin^2 \theta + \cos^2 \theta = 1$ to convert
sines to cosines or vice-versa.

\begin{example}[example 1]
Compute
\[
\int \sin^4 x \cos x \, dx
\]
This problem is solved using the substitution $u = \sin(x)$, so that $du = \cos(x) \; dx$.
\begin{align*}
\int \sin^4 x \cos x \, dx &= \int u^4 \, du\\
                             &= \frac{u^5}{5} + C\\
                             &= \frac15 \sin^5 x + C
\end{align*}

\end{example}



\begin{problem}(problem 1a) 
Compute $\displaystyle{\int \sin^6 x\cos x \, dx}$\\

Let $u = \answer{\sin x}$, \; then $du = \answer{\cos x dx}$.\\

Substituting gives
\[
\int \sin^6 x\cos x \, dx = \int \answer{u^6} \; du
\]

Finally
\[
\int \sin^{6} x\cos x \, dx = \answer{1/7 \sin^7 x} + C
\]

\end{problem}



\begin{problem}(problem 1b) 
Compute $\displaystyle{\int \sin x \cos^4 x \, dx}$\\

Let $u = \answer{\cos x}$, \; then $du = \answer{-\sin x dx}$.\\

Substituting gives 
\[
\int \sin x\cos^4 x \, dx = \int \answer{-u^4} \; du
\]

Finally 
\[
\int \sin x\cos^4 x \, dx = \answer{-1/5 \cos^5 x} + C
\]

\end{problem}



\begin{problem}(problem 1c)
Compute $\displaystyle{\int \sin^{10}(3x)\cos(3x) \, dx}$\\

Let $u = \answer{\sin(3x)}$, \; then $du = \answer{3\cos(3x) dx}$.\\

Substituting gives 
\[
\int \sin^{10}(3x)\cos(3x) \, dx = \int \answer{\frac13 u^{10}} \; du
\]

Finally
\[
\int \sin^{10}(3x)\cos(3x) \, dx = \answer{\frac{1}{33} \sin^{11}(3x)} + C
\]

\end{problem}



In the next example, we need to prepare for the $u$-substitution. 



\begin{example}[example 2]
Compute $\displaystyle{\int \sin^4 x\cos^3 x \; dx}$\\


Prepare for substitution by rewriting the odd power of cosine.
\[
\cos^3 x = \cos^2 x \cos x = \left[1 - \sin^2 x\right] \cos x
\]

Rewrite the original integral.
\[
\int \sin^4 x\cos^3 x \; dx = \int \sin^4 x\left[1 - \sin^2 x\right] \cos(x) \; dx
\]
Make the substitution \, $u = \sin x$,  \, $du = \cos x \, dx$.
\begin{align*}
\int \sin^4 x\cos^3 x \; dx &= \int u^4 (1-u^2) \; du\\
   &= \int (u^4 -u^6) \; du\\
  &= \frac{u^5}{5} - \frac{u^7}{7} + C \\
  &= \frac15 \sin^5 x - \frac17 \sin^7 x + C
\end{align*}
  
\end{example}



\begin{problem}(problem 2a)
Compute $\displaystyle{\int \sin^6 x\cos^3 x \; dx}$\\

Rewrite the odd power of cosine as

\begin{multipleChoice}
\choice{$\cos^3 x = \left[\sin^2 x- 1\right] \cos x$} 
\choice[correct]{$\cos^3 x = \left[1 - \sin^2 x\right] \cos x$}
\choice{$\cos^3 x = \left[1 - \sin x\right]^2 \cos x$}
\end{multipleChoice}

Next, let $u = \answer{\sin x}$ so that $du = \answer{\cos x dx}$.\\

Substituting gives 
\[
\int \sin^6 x\cos^3 x \; dx= \int \answer{u^6(1-u^2)} \; du
\]

Finally 
\[
\int \sin^6 x\cos^3 x \; dx= \answer{\sin^7 x/7 -  \sin^9 x/9} + C
\]

\end{problem}



\begin{problem}{\color{gray}(problem 2b)} 
Compute $\displaystyle{\int \sin^3 x\cos^2 x \; dx}$\\

Rewrite the odd power of sine as:

\begin{multipleChoice}
\choice{$\sin^3 x = \left[1 - \cos x\right]^2 \sin x$}
\choice{$\sin^3 x = \left[\cos^2 x- 1\right] \sin x$} 
\choice[correct]{$\sin^3 x = \left[1 - \cos^2 x\right] \sin x$}
\end{multipleChoice}

Next, let $u = \answer{\cos x}$ so that $du = \answer{-\sin x dx}$.\\

Substituting gives
\[
\int \sin^3 x\cos^2 x \; dx= \int \answer{-u^2(1-u^2)} \; du.
\]

Finally
\[
\int \sin^3 x\cos^2 x \; dx= \answer{\cos^5 x/5 - \cos^3 x/3} + C.
\]

\end{problem}



\begin{problem}{\color{gray}(problem 2c)}
Compute $\displaystyle{\int 3\sin^8(2x)\cos^3(2x) \; dx}$\\

Rewrite the odd power of cosine as:

\begin{multipleChoice}
\choice[correct]{$\cos^3(2x) = \left[1 - \sin^2(2x)\right] \cos(2x)$}
\choice{$\cos^3(2x) = \left[\sin^2(2x)- 1\right] \cos(2x)$} 
\choice{$\cos^3(2x) = \left[1 - \sin(2x)\right]^2 \cos(2x)$}
\end{multipleChoice}

Next, let $u = \answer{\sin(2x)}$ so that $du = \answer{2\cos(2x) dx}$.\\

Substituting gives
\[
\int 3\sin^8(2x)\cos^3(2x) \; dx= \int \answer{3/2 u^8(1-u^2)} \; du
\]

Finally
\[
\int 3\sin^8(2x)\cos^3(2x) \; dx= \answer{\frac16 \sin^9(2x)- \frac{3}{22} \sin^{11}(2x)} + C
\]
\end{problem}


\begin{problem}(problem 2d)
Compute $\displaystyle{\int \cos^3(4x) \; dx}$\\

Rewrite the odd power of cosine as:

\begin{multipleChoice}
\choice{$\cos^3(4x) = \left[\sin^2(4x)- 1\right] \cos(4x)$} 
\choice[correct]{$\cos^3(4x) = \left[1 - \sin^2(4x)\right] \cos(4x)$}
\choice{$\cos^3(4x) = \left[1 - \sin(4x)\right]^2 \cos(4x)$}
\end{multipleChoice}

Next, let $u = \answer{\sin(4x)}$ so that $du = \answer{4\cos(4x) dx}$.\\

Substituting gives
\[
\int \cos^3(4x) \; dx= \int \answer{1/4 (1-u^2)} \; du
\]

Finally
\[
\int \cos^3(4x) \; dx= \answer{\frac14 \sin(4x) - \frac{1}{12} \sin^3(4x) } + C
\]

\end{problem}





\begin{example}[example 3]
Compute $\displaystyle{\int \sin^8 x\cos^5 x \; dx}$\\

Prepare for substitution by rewriting the odd power of cosine.
\[
\cos^5 x = \cos^4 x \cos x = \left[1 - \sin^2 x\right]^2 \cos x
\]

Rewrite the original integral.
\[
\int \sin^8 x\cos^5 x \; dx = \int \sin^8 x\left[1 - \sin^2 x\right]^2 \cos x \; dx.
\]

Make the substitution $u = \sin x, \; du = \cos x \, dx$.
\begin{align*}
\int \sin^8 x\cos^5 x \; dx &= \int u^8 (1-u^2)^2 \; du\\
   &= \int u^8 (1-2u^2 + u^4) \; du\\
   &= \int (u^8 -2u^{10} + u^{12}) \; du\\
  &= \tfrac19 u^9 - \tfrac{2}{11}u^{11} + \tfrac{1}{13} u^{13} + C \\
  &= \tfrac19 \sin^9 x - \tfrac{2}{11} \sin^{11} x  + \tfrac{1}{13} \sin^{13} x + C.
\end{align*}
  
\end{example}



\begin{problem}(problem 3a) 
Compute $\displaystyle{\int \sin^5 x\cos^6 x \; dx}$\\

Rewrite the odd power of sine as:

\begin{multipleChoice}
\choice{$\sin^5 x = \left[1 - \cos^4 x\right] \sin x$}
\choice{$\sin^5 x = \left[1 + \cos^4 x\right] \sin x$} 
\choice[correct]{$\sin^5 x = \left[1 - 2\cos^2 x + \cos^4 x\right] \sin x$}
\end{multipleChoice}

Next, let $u = \answer{\cos x}$ so that $du = \answer{-\sin x dx}$.\\

Substituting gives
\[
\int \sin^5 x\cos^6 x \; dx= \int \answer{-u^6(1-2u^2 + u^4)} \; du
\]

Finally
\[
\int \sin^5 x\cos^6 x \; dx= \answer{-1/7 \cos^7 x + 2/9 \cos^9 x - 1/11 \cos^{11} x} + C
\]

\end{problem}



\begin{problem}{\color{gray}(problem 3b)} 
Compute $\displaystyle{\int \sin^2(6x)\cos^5(6x) \; dx}$\\

Rewrite the odd power of cosine as:

\begin{multipleChoice}
\choice[correct]{$\cos^5(6x) = \left[1 - 2\sin^2(6x) + \sin^4(6x)\right] \cos(6x)$}
\choice{$\cos^5(6x) = \left[1 - \sin^4(6x)\right] \cos(6x)$}
\choice{$\cos^5(6x) = \left[1 + \sin^4(6x)\right] \cos(6x)$} 
\end{multipleChoice}

Next, let $u = \answer{\sin(6x)}$ so that $du = \answer{6 \cos(6x) dx}$.\\

Substituting gives
\[
\int \sin^2(6x)\cos^5(6x) \; dx= \int \answer{1/6 u^2(1-2u^2 + u^4)} \; du
\]

Finally
\[
\int \sin^2(6x)\cos^5(6x) \; dx= \answer{1/18 \sin^3(6x) - 1/15 \sin^5(6x) + 1/42 \sin^7(6x)} + C
\]
\end{problem}



\begin{problem}{\color{gray}(problem 3c)} 
Compute $\displaystyle{\int \sin^5 x \; dx}$\\

Rewrite the odd power of sine as:

\begin{multipleChoice}
\choice{$\sin^5 x = \left[1 - \cos^4 x\right] \sin x$}
\choice{$\sin^5 x = \left[1 + \cos^4 x\right] \sin x$} 
\choice[correct]{$\sin^5 x = \left[1 - 2\cos^2 x + \cos^4 x\right] \sin x$}
\end{multipleChoice}

Next, let $u = \answer{\cos x}$ so that $du = \answer{-\sin x dx}$.\\

Substituting gives
\[
\int \sin^5 x \; dx= \int \answer{-(1-2u^2 + u^4)} \; du
\]

Finally
\[
\int \sin^5 x \; dx= \answer{- \cos(x) + 2/3 \cos^3(x) - 1/5 \cos^5(x)} + C
\]

\end{problem}

In the next example, both powers are odd.  In this case, we have to choose which one to rewrite.



\begin{example}[example 4]
Compute $\displaystyle{\int \sin^7 x\cos^5 x \; dx}$\\

Prepare for substitution by rewriting the smaller odd power.
\[
\cos^5 x = \cos^4 x \cos x = \left[1-\sin^2 x\right]^2 \cos x
\]

Rewrite the original integral.
\[
\int \sin^7 x\cos^5 x \; dx = \int \sin^7 x\left[1 - \sin^2 x\right]^2 \cos x \; dx
\]

Make the substitution $u = \sin x, \; du = \cos x \, dx$.
\begin{align*}
\int \sin^7 x\cos^5 x \; dx  &= \int u^7 (1-u^2)^2 \; du\\
   &= \int u^7(1 - 2u^2 + u^4) \; du\\
   &= \int (u^7 - 2u^9 + u^{11}) \; du\\
  &= \tfrac18 u^8 - \tfrac15 u^{10} + \tfrac{1}{12}u^{12}+ C \\
  &= \tfrac18 \sin^8 x - \tfrac15 \sin^{10} x + \tfrac{1}{12} \sin^{12} x+ C
\end{align*}

\end{example}





\begin{problem}{\color{gray}(problem 4a)}
Compute $\displaystyle{\int \sin^5 x\cos^7 x \; dx}$

Which trig function should be rewritten?

\begin{multipleChoice}
\choice[correct]{$\sin^5 x$}
\choice{$\cos^7 x$}
\end{multipleChoice}

Rewrite the odd power of sine as:
\begin{multipleChoice}
\choice{$\sin^5 x = \left[1 - \cos^4 x\right] \sin x$}
\choice{$\sin^5 x = \left[\cos^4 x- 1\right] \sin x$} 
\choice[correct]{$\sin^5 x = \left[1 - 2\cos^2 x + \cos^4 x\right] \sin x$}
\end{multipleChoice}

Next, let $u = \answer{\cos x}$ so that $du = \answer{-\sin x dx}$.\\

Substituting gives
\[
\int \sin^5 x\cos^7 x \; dx= \int \answer{-(1-2u^2 + u^4)u^7} \; du
\]

Finally
\[
\int \sin^5 x\cos^7 x \; dx= \answer{-\cos^8 x/8 + \cos^{10}(x)/5 - \cos^{12}(x)/12} + C
\]
\end{problem}



\begin{problem}(problem 4b)
Compute $\displaystyle{\int \sin^9 x\cos^3 x \; dx}$\\

Which trig function should be rewritten?
\begin{multipleChoice}
\choice{$\sin^9 x$}
\choice[correct]{$\cos^3 x$}
\end{multipleChoice}

Rewrite the odd power of cosine as:
\begin{multipleChoice}
\choice{$\cos^3 x = \left[\sin^2 x- 1\right] \cos x$} 
\choice[correct]{$\cos^3 x = \left[1 - \sin^2 x\right] \cos x$}
\choice{$\cos^3 x = \left[1 - \sin x\right]^2 \cos x$}
\end{multipleChoice}

Next, let $u = \answer{\sin x}$ so that $du = \answer{\cos x dx}$.\\

Substituting gives
\[
\int \sin^9 x\cos^3 x \; dx= \int \answer{u^9(1 - u^2)} \; du
\]

Finally
\[
\int \sin^9 x\cos^3 x \; dx= \answer{\sin^{10}(x)/(10) - \sin^{12}(x)/(12) } + C
\]
\end{problem}




\subsection{Case 2: $m$ and $n$ are both even}
In this section, we compute integrals of the form
\[
\int \sin^m x \cos^n x \; dx
\]
where $m$ and $n$ are both even.\\
We will use the half-angle formulas
\[
\sin^2 \theta = \frac{1 - \cos 2\theta}{2} \; \text{and} \; \cos^2 \theta = \frac{1 + \cos 2\theta}{2}
\]

\begin{example}[example 5]
Compute $\displaystyle{\int \sin^2 x \; dx}$

Rewrite the integral using a half-angle formula.

\begin{align*}
\int \sin^2 x \; dx &= \int \left[\frac12- \frac12\cos(2x)\right]  \; dx\\
  &= \frac{x}{2}  - \frac{1}{4}\sin(2x) + C.
\end{align*}
\end{example}

\begin{problem}(problem 5)
Compute $\displaystyle{\int \cos^2 x \; dx}$
\begin{multipleChoice}
\choice[correct]{$\frac{x}{2}  + \frac{1}{4}\sin(2x) + C$} 
\choice{$\frac{x}{2}  - \frac{1}{4}\cos(2x) + C$}
\choice{$\frac{x}{2}  + \frac{1}{4}\cos(2x) + C$}
\end{multipleChoice}
\end{problem}

\begin{example}[example 6]
Compute the indefinite integral:
\[
\int \sin^4 x \; dx.
\]
Rewrite the integral using the half-angle formula for sine.

\begin{align*}
\int \sin^4 x \; dx &= \int \left[\tfrac12- \tfrac12\cos(2x)\right] \cdot \left[\tfrac12 - \tfrac12\cos(2x)\right] \; dx\\
  &=  \int \left[\tfrac14 - \tfrac12\cos(2x) + \tfrac14\cos^2(2x)\right] \; dx \\
  &= \int \left( \tfrac14 - \tfrac12 \cos(2x) + \tfrac14 
  \left[\tfrac12 + \tfrac12\cos(4x)\right]\right) \; dx  \;\; \text{(half-angle again)} \\
  &= \int \left[\tfrac14 - \tfrac12 \cos(2x) +\tfrac18 + \tfrac18\cos(4x)\right] \; dx\\
  &=  \int \left[\tfrac38 - \tfrac12 \cos(2x) + \tfrac18\cos(4x)\right] \; dx \;\; \text{(now we can integrate)}\\
  &= \tfrac38 x - \tfrac14 \sin(2x) + \tfrac{1}{32}\sin(4x) + C.
\end{align*}
\end{example}




\begin{problem}(problem 6a)
Compute $\displaystyle{\int \cos^4 x \; dx}$


Use a half angle formula to rewrite: $\cos^4 x=$

\begin{multipleChoice}
\choice{$\tfrac14  + \tfrac14\cos^2(2x)$}
\choice{$\tfrac14 - \tfrac12\cos(2x) + \tfrac14\cos^2(2x)$}
\choice[correct]{$\tfrac14 + \tfrac12\cos(2x) + \tfrac14\cos^2(2x)$}
\end{multipleChoice}

Use a half-angle formula again to obtain, $\cos^4 x = $

\begin{multipleChoice}
\choice{$\tfrac18 + \tfrac12\cos(2x) + \tfrac18 \cos(4x)$}
\choice{$\tfrac14 + \tfrac12\cos(2x) + \tfrac18 \cos(4x)$}
\choice[correct]{$\tfrac38 + \tfrac12\cos(2x) + \tfrac18 \cos(4x)$}
\end{multipleChoice}

Finally,
\[
\int \cos^4 x \; dx = \answer{\frac{3x}{8} +\frac14 \sin(2x) + \frac{1}{32}\sin(4x)} + C
\]
\end{problem}



\begin{problem}(problem 6b)
Compute $\displaystyle{\int \sin^2(3x)\cos^2(3x) \; dx}$


Use the half-angle formulas to rewrite: $\sin^2(3x) \cos^2(3x)=$

\begin{multipleChoice}
\choice{$\tfrac14 + \tfrac14\cos^2(6x)$}
\choice{$\tfrac14 - \tfrac14\sin^2(6x)$}
\choice[correct]{$\tfrac14 - \tfrac14\cos^2(6x)$}
\end{multipleChoice}

Use a half-angle formula again to obtain, $\sin^2(3x) \cos^2(3x)=$

\begin{multipleChoice}
\choice{$\tfrac38  - \tfrac18 \cos(12x)$}
\choice{$\tfrac18  + \tfrac18 \cos(12x)$}
\choice[correct]{$\tfrac18  - \tfrac18 \cos(12x)$}
\end{multipleChoice}

Finally,
\[
\int \sin^2(3x)\cos^2(3x) \; dx = \answer{\frac18 x - \frac{1}{96} \sin(12x)} + C.
\]

\end{problem}


%Rewrite the integral using the half-angle formulas.

%\begin{align*}
%\int \sin^2(x)\cos^2(x) \; dx &= \int \left[\tfrac12- \tfrac12\cos(2x)\right] \cdot \left[\tfrac12 + \tfrac12\cos(2x)\right] \; dx\\
%  &=  \int \left[\tfrac14 - \tfrac14\cos^2(2x)\right] \; dx \\
%  &= \int \left( \tfrac14 - \tfrac14 \left[\tfrac12 + \tfrac12\cos(4x)\right]\right) \; dx  \;\; \textup{(half-angle again)} \\
%  &= \int \left[\tfrac14 -\tfrac18 - \tfrac18\cos(4x)\right] \; dx\\
%  &=  \int \left[\tfrac18 - \tfrac18\cos(4x)\right] \; dx \;\; \textup{(now we can integrate)}\\
%  &= \tfrac18 x  - \tfrac{1}{32}\sin(4x) + C.
%\end{align*}
%\end{example}



\section{Integrating Powers of Secant and Tangent}
In this section, we compute integrals of the form
\[
\int \sec^m x\tan^n x \; dx,
\]
where either $m$ is even or $n$ is odd. We will use the identity
\[
1+\tan^2 \theta = \sec^2 \theta
\]
to prepare a $u$-substitution.


\subsection{Case 1: $m$ is even}
\begin{example}[example 7]
Compute $\displaystyle{\int \tan^4 x \sec^2 x\;dx}$

Make the substitution $u = \tan x, \; du = \sec^2 x \, dx$.

\begin{align*}
\int \tan^4 x \sec^2 x\;dx &= \int u^4 du\\
&=  \frac{u^5}{5} + C  \\
&= \frac15\tan^5 x  + C
\end{align*}
\end{example} 




\begin{problem}(problem 7)
Compute $\displaystyle{\int \tan^{10}(3x)\sec^2(3x) \, dx}$\\

Let $u = \answer{\tan(3x)}$, \; then $du = \answer{3\sec^2(3x) dx}$.\\

Substituting gives 
\[
\int \tan^{10}(3x)\sec^2(3x) \, dx = \int \answer{\frac13 u^{10}} \; du
\]

Finally
\[
\int \tan^{10}(3x)\sec^2(3x) \, dx = \answer{\frac{1}{33} \tan^{11}(3x)} + C
\]

\end{problem}




\begin{example}[example 8]
Compute $\displaystyle{\int \tan^4 x \sec^6 x\;dx}$

Prepare for $u$-substitution by rewriting the even power of secant.
\[
\sec^6 x = \sec^4 x \sec^2 x = \left[1+\tan^2 x\right]^2 \sec^2 x
\]

Rewrite the original integral.

\[
\int \tan^4 x \sec^6 x\;dx = \int \tan^4 x \left[1+\tan^2 x\right]^2 \sec^2 x \; dx
\]

Make the substitution $u = \tan x, \; du = \sec^2 x \, dx$.


\begin{align*}
\int \tan^4 x \sec^6 x\;dx &= \int u^4 (1+u^2)^2 du\\
&= \int u^4(1+2u^2 + u^4) \; du \\
&= \int (u^4 + 2u^6 + u^8) \; du \\
&=  \tfrac15 u^5 + \tfrac27 u^7 + \tfrac19 u^9 + C  \\
&= \tfrac15\tan^5 x  + \tfrac27 \tan^7 x + \tfrac19 \tan^9 x + C
\end{align*}
\end{example} 



\begin{problem}(problem 8a)
Compute $\displaystyle{\int \tan^6 x\sec^4 x \; dx}$\\

Rewrite the even power of secant as

\begin{multipleChoice}
\choice{$\sec^4 x = \left[\tan^2 x- 1\right] \sec^2 x$} 
\choice[correct]{$\sec^4 x = \left[1 + \tan^2 x\right] \sec^2 x$}
\choice{$\sec^4 x = \left[1 - \tan^2 x\right] \sec^2 x$}
\end{multipleChoice}

Next, let $u = \answer{\tan x}$ so that $du = \answer{\sec^2 x dx}$.\\

Substituting gives 
\[
\int \tan^6 x\sec^4 x \; dx= \int \answer{u^6(1+u^2)} \; du
\]

Finally 
\[
\int \tan^6 x\sec^4 x \; dx= \answer{\tan^7(x)/7 +  \tan^9(x)/9} + C
\]

\end{problem}



\begin{problem}(problem 8b)
Compute the indefinite integral:
\[
\int \tan^{1/3} x \sec^4 x\;dx = 
\]

Rewrite the even power of secant as

\begin{multipleChoice}
\choice{$\sec^4 x = \left[\tan^2 x- 1\right] \sec^2 x$} 
\choice[correct]{$\sec^4 x = \left[1 + \tan^2 x\right] \sec^2 x$}
\choice{$\sec^4 x = \left[1 - \tan^2 x\right] \sec^2 x$}
\end{multipleChoice}

Next, let $u = \answer{\tan x}$ so that $du = \answer{\sec^2 x dx}$.\\

Substituting gives 
\[
\int \tan^{1/3} x\sec^4 x \; dx= \int \answer{u^{1/3}(1+u^2)} \; du
\]

Finally 
\[
\int \tan^{1/3} x\sec^4 x \; dx= \answer{\frac34 \tan^{4/3}(x) + \frac{3}{10} \tan^{10/3}(x)} + C.
\]

\end{problem}

\begin{problem}(problem 8c)
Compute the indefinite integral:
\[
\int  \sec^6(3x)\;dx
\]

Rewrite the even power of secant as

\begin{multipleChoice}
\choice{$\sec^6(3x) = \left[1  + \tan^4(3x) \right] \sec^2(3x)$} 
\choice{$\sec^6(3x) = \left[1 - 2\tan^2(3x) + \tan^4(3x) \right] \sec^2(3x)$}
\choice[correct]{$\sec^6(3x) = \left[1 + 2\tan^2(3x) + \tan^4(3x) \right] \sec^2(3x)$}
\end{multipleChoice}

Next, let $u = \answer{\tan(3x)}$ so that $du = \answer{3\sec^2(3x) dx}$.\\

Substituting gives 
\[
\int \sec^6(3x) \; dx= \int \answer{(1+ 2u^2 +u^4)/3} \; du
\]

Finally 
\[
\int \sec^6(3x) \; dx= \answer{\frac13 \tan(3x) + \frac29 \tan^3(3x) + \frac{1}{15}\tan^5(3x)} + C.
\]


\end{problem}




\subsection{Case 2: $n$ is odd}


\begin{example}[example 9]

Compute
\[
\int \sec^5 x \tan x \, dx.
\]
With an odd power of tangent, we prepare the substitution $u = \sec x, \; du = \sec x\tan x \, dx$ by rewriting the integral as
\[
\int \sec^5 x \tan x \, dx = \int \sec^4 x \sec x \tan x \, dx.
\]
Next we perform the substitution:
\begin{align*}
\int \sec^5 x \tan x \, dx &= \int u^4 \, du\\
                             &= \frac{u^5}{5} + C\\
                             &= \frac15 \sec^5 x + C.
\end{align*}

\end{example}


\begin{problem}(problem 9)
Compute $\displaystyle{\int \sec^3 x \tan x \, dx}$\\
Let $u = \answer{\sec x}$ then \, $du = \answer{\sec(x)\tan(x) dx}$\\
Substituting gives 
\[
\int \sec^3 x \tan x \, dx = \int \answer{u^2} \, du
\]
Finally
\[
\int \sec^3 x \tan x \, dx = \answer{1/3 \sec^3(x)} + C
\]

\end{problem}


\begin{example}[example 10]
Compute the indefinite integral:
\[
\int \tan^5 x \sec^3 x\;dx.
\]
Rewrite the integral to prepare for the substitution, $u = \sec x, \;du = \sec x \tan x \; dx$.

\[
\int \tan^5 x \sec^3 x\;dx = \int \tan^4 x \sec^2 x \sec x \tan x\;dx.
\]

Next, convert tangents to secants, using $1+ \tan^2 \theta = \sec^2 \theta$

\[
\tan^4 x = \left[\sec^2 x - 1\right]^2
\]

The integral becomes

\[
 \int \tan^4 x \sec^2 x \sec x \tan x\;dx = \int \left[\sec^2 x -1\right]^2 \sec^2 x \sec x \tan x\;dx
\]

Now we are ready to substitute, $u = \sec x, \; du = \sec x \tan x \, dx$

\begin{align*}
\int \tan^5 x \sec^3 x\;dx &= \int \left(u^2 -1\right)^2 u^2 \; du \\
                             &= \int \left(u^4 - 2u^2 +1\right) u^2 \; du \\
                             &= \int \left(u^6 - 2u^4 +u^2\right) \; du \\
                             &=  \frac{u^7}{7} - \frac{2u^5}{5} + \frac{u^3}{3} + C  \\
                             &= \frac17\sec^7 x - \frac25 \sec^5 x + \frac13 \sec^3 x + C
\end{align*}
\end{example}



\begin{problem}(problem 10a)
Compute the indefinite integral:
\[
\int \tan^3 x \sec^5 x\;dx
\]
First rewrite the integral as
\[
\int \tan^3 x \sec^5 x\;dx = \int \tan^2 x \sec^4 x \sec x \tan x \; dx
\]
Next, rewrite $\tan^2 x$ as
\begin{multipleChoice}
\choice{$1+\sec^2 x$}
\choice{$1-\sec^2 x$}
\choice[correct]{$\sec^2 x -1$}
\end{multipleChoice}

Let $u = \answer{\sec(x)}$, then \; $du = \answer{\sec(x) \tan(x) \, dx}$

Substituting gives
\[
\int \tan^3 x \sec^5 x\;dx = \int \answer{(u^2 - 1)u^4} \; du
\]

Finally
\[
\int \tan^3 x \sec^5 x\;dx = \answer{\frac17 \sec^7(x) - \frac15 \sec^5(x)} + C
\]
\end{problem}




\begin{problem}(problem 10b)
Compute the indefinite integral:
\[
\int \tan^5 x \sec^7 x\;dx = 
\]
First rewrite the integral as
\[
\int \tan^5 x \sec^7 x\;dx = \int \tan^4 x \sec^6 x \sec x \tan x \; dx
\]
Next, rewrite $\tan^4 x$ as
\begin{multipleChoice}
\choice{$\left[\sec^2 x + 1\right]^2$}
\choice[correct]{$\left[\sec^2 x -1\right]^2$}
\choice{$\left[1-\sec^2 x\right]^2$}
\end{multipleChoice}

Let $u = \answer{\sec x}$, then \; $du = \answer{\sec(x) \tan(x) \, dx}$

Substituting gives
\[
\int \tan^5 x \sec^7 x\;dx = \int \answer{(u^4 -2u^2 + 1) u^6} \; du
\]

Finally
\[
\int \tan^5 x \sec^7 x\;dx =  \answer{\frac{1}{11} \sec^{11}(x) - \frac29 \sec^9(x) + \frac17 \sec^7(x)} + C.
\]
\end{problem}





\section{Odd Powers of Secant}
\begin{example}
Compute the indefinite integral:
\[
\int \sec x \; dx.
\]
This integral is computed by using a $u$-substitution after rewriting the integrand
by multiplying by 1 in the following way:
\[
\int \sec x \; dx = \int \sec x \cdot \frac{\sec x + \tan x}{\sec x + \tan x} \; dx.
\]
After distributing $\sec x$ we get
\[
\int \sec x \; dx = \int \frac{\sec^2 x + \sec x\tan x}{\sec x + \tan x} \; dx.
\]
Now we observe that the numerator is the derivative of the denominator which suggests the substitution
$u = \sec x + \tan x$. We have $du = [ \sec x\tan x + \sec^2 x] \; dx$ and the integral becomes
\begin{align*}
\int \sec x \; dx &= \int \frac{\sec^2 x + \sec x\tan x}{\sec x + \tan x} \; dx\\
&= \int \frac{1}{u} \; du\\
&= \ln|u| + C\\
&= \ln|\sec x + \tan x|.
\end{align*}

\end{example}

%The integral of $\sec^3(x)$ was handled in the Integration by parts section
%yielding
%\[
%\int \sec^3(x) \; dx = \frac12 \sec(x) \tan(x) + \frac12 \ln|\sec(x) + \tan(x)| + C.
%\]
Furthermore, IBP also gave a reduction formula for higher powers of $\sec(x)$:
\[
\int \sec^n x \; dx =  \frac{1}{n-1}\sec^{n-2} x\tan x + \frac{n-2}{n-1}\int \sec^{n-2} x \; dx. 
\]
This formula is particularly useful if
the power is odd, since if $n$ is odd, $n-2$ is also odd.  Therefore, if we repeat the formula enough times, 
we will eventually be left with $\int \sec x \, dx$
which we have just solved.

\begin{example}
Use the above reduction of powers formula twice to compute the indefinite integral
\[
\int \sec^5 x \; dx.
\]
We have $n =5$, and the formula gives
\[
\int \sec^5 x \; dx = \frac14 \sec^3 x\tan x + \frac34 \int \sec^3 x \; dx.
\]
We will use the reduction of powers formula once again, this time with $n=3$:
\begin{align*}
\int \sec^5 x\; dx &= \frac14 \sec^3 x\tan x + \frac34 \int \sec^3 x \; dx\\
&= \frac14 \sec^3 x\tan x + \frac34\left(\frac12\sec x\tan x + \frac12\int \sec x \; dx\right)\\
&=\frac14 \sec^3 x\tan x + \frac38\sec x\tan x + \frac38 \int \sec x \; dx \\
&=\frac14 \sec^3 x\tan x + \frac38\sec x\tan x + \frac38\ln|\sec x + \tan x| + C
\end{align*}
\end{example}



%\begin{center}
%\begin{foldable}
%\unfoldable{Here is a video of Example 1}
%\youtube{Yy6QXnFlnXs} %vid of example 1
%\end{foldable}
%\end{center}





\begin{center}
\begin{foldable}
\unfoldable{Here are two detailed, lecture style videos on trigonometric integrals:}
\youtube{qbRIlTVP_2E}
\youtube{G0R-dItF4EA}
\end{foldable}
\end{center}





\end{document}






