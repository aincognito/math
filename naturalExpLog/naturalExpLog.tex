\documentclass[handout]{ximera}
% \usepackage{tcolorbox}
%% You can put user macros here
%% However, you cannot make new environments



\newcommand{\ffrac}[2]{\frac{\text{\footnotesize $#1$}}{\text{\footnotesize $#2$}}}
\newcommand{\vasymptote}[2][]{
    \draw [densely dashed,#1] ({rel axis cs:0,0} -| {axis cs:#2,0}) -- ({rel axis cs:0,1} -| {axis cs:#2,0});
}


\graphicspath{{./}{firstExample/}}
\usepackage{forest}
\usepackage{amsmath}
\usepackage{amssymb}
\usepackage{array}
\usepackage[makeroom]{cancel} %% for strike outs
\usepackage{pgffor} %% required for integral for loops
\usepackage{tikz}
\usepackage{tikz-cd}
\usepackage{tkz-euclide}
\usetikzlibrary{shapes.multipart}


%\usetkzobj{all}
\tikzstyle geometryDiagrams=[ultra thick,color=blue!50!black]


\usetikzlibrary{arrows}
\tikzset{>=stealth,commutative diagrams/.cd,
  arrow style=tikz,diagrams={>=stealth}} %% cool arrow head
\tikzset{shorten <>/.style={ shorten >=#1, shorten <=#1 } } %% allows shorter vectors

\usetikzlibrary{backgrounds} %% for boxes around graphs
\usetikzlibrary{shapes,positioning}  %% Clouds and stars
\usetikzlibrary{matrix} %% for matrix
\usepgfplotslibrary{polar} %% for polar plots
\usepgfplotslibrary{fillbetween} %% to shade area between curves in TikZ



%\usepackage[width=4.375in, height=7.0in, top=1.0in, papersize={5.5in,8.5in}]{geometry}
%\usepackage[pdftex]{graphicx}
%\usepackage{tipa}
%\usepackage{txfonts}
%\usepackage{textcomp}
%\usepackage{amsthm}
%\usepackage{xy}
%\usepackage{fancyhdr}
%\usepackage{xcolor}
%\usepackage{mathtools} %% for pretty underbrace % Breaks Ximera
%\usepackage{multicol}



\newcommand{\RR}{\mathbb R}
\newcommand{\R}{\mathbb R}
\newcommand{\C}{\mathbb C}
\newcommand{\N}{\mathbb N}
\newcommand{\Z}{\mathbb Z}
\newcommand{\dis}{\displaystyle}
%\renewcommand{\d}{\,d\!}
\renewcommand{\d}{\mathop{}\!d}
\newcommand{\dd}[2][]{\frac{\d #1}{\d #2}}
\newcommand{\pp}[2][]{\frac{\partial #1}{\partial #2}}
\renewcommand{\l}{\ell}
\newcommand{\ddx}{\frac{d}{\d x}}

\newcommand{\zeroOverZero}{\ensuremath{\boldsymbol{\tfrac{0}{0}}}}
\newcommand{\inftyOverInfty}{\ensuremath{\boldsymbol{\tfrac{\infty}{\infty}}}}
\newcommand{\zeroOverInfty}{\ensuremath{\boldsymbol{\tfrac{0}{\infty}}}}
\newcommand{\zeroTimesInfty}{\ensuremath{\small\boldsymbol{0\cdot \infty}}}
\newcommand{\inftyMinusInfty}{\ensuremath{\small\boldsymbol{\infty - \infty}}}
\newcommand{\oneToInfty}{\ensuremath{\boldsymbol{1^\infty}}}
\newcommand{\zeroToZero}{\ensuremath{\boldsymbol{0^0}}}
\newcommand{\inftyToZero}{\ensuremath{\boldsymbol{\infty^0}}}


\newcommand{\numOverZero}{\ensuremath{\boldsymbol{\tfrac{\#}{0}}}}
\newcommand{\dfn}{\textbf}
%\newcommand{\unit}{\,\mathrm}
\newcommand{\unit}{\mathop{}\!\mathrm}
%\newcommand{\eval}[1]{\bigg[ #1 \bigg]}
\newcommand{\eval}[1]{ #1 \bigg|}
\newcommand{\seq}[1]{\left( #1 \right)}
\renewcommand{\epsilon}{\varepsilon}
\renewcommand{\iff}{\Leftrightarrow}

\DeclareMathOperator{\arccot}{arccot}
\DeclareMathOperator{\arcsec}{arcsec}
\DeclareMathOperator{\arccsc}{arccsc}
\DeclareMathOperator{\si}{Si}
\DeclareMathOperator{\proj}{proj}
\DeclareMathOperator{\scal}{scal}
\DeclareMathOperator{\cis}{cis}
\DeclareMathOperator{\Arg}{Arg}
%\DeclareMathOperator{\arg}{arg}
\DeclareMathOperator{\Rep}{Re}
\DeclareMathOperator{\Imp}{Im}
\DeclareMathOperator{\sech}{sech}
\DeclareMathOperator{\csch}{csch}
\DeclareMathOperator{\Log}{Log}

\newcommand{\tightoverset}[2]{% for arrow vec
  \mathop{#2}\limits^{\vbox to -.5ex{\kern-0.75ex\hbox{$#1$}\vss}}}
\newcommand{\arrowvec}{\overrightarrow}
\renewcommand{\vec}{\mathbf}
\newcommand{\veci}{{\boldsymbol{\hat{\imath}}}}
\newcommand{\vecj}{{\boldsymbol{\hat{\jmath}}}}
\newcommand{\veck}{{\boldsymbol{\hat{k}}}}
\newcommand{\vecl}{\boldsymbol{\l}}
\newcommand{\utan}{\vec{\hat{t}}}
\newcommand{\unormal}{\vec{\hat{n}}}
\newcommand{\ubinormal}{\vec{\hat{b}}}

\newcommand{\dotp}{\bullet}
\newcommand{\cross}{\boldsymbol\times}
\newcommand{\grad}{\boldsymbol\nabla}
\newcommand{\divergence}{\grad\dotp}
\newcommand{\curl}{\grad\cross}
%% Simple horiz vectors
\renewcommand{\vector}[1]{\left\langle #1\right\rangle}

\outcome{Compute derivatives involving exp(x) and ln(x).}


\title{2.3 Derivatives of Natural Exponential and Log}

%\newcommand{\ffrac}[2]{\frac{\mbox{\footnotesize $#1$}}{\mbox{\footnotesize $#2$}}}
%\newcommand{\vasymptote}[2][]{
 %   \draw [densely dashed,#1] ({rel axis cs:0,0} -| {axis cs:#2,0}) -- ({rel axis cs:0,1} -| {axis cs:#2,0});
%}


\begin{document}

\begin{abstract}
In this section we compute derivatives involving $e^x$ and $\ln(x)$.
\end{abstract}

\maketitle


\begin{image}
\begin{tikzpicture}
\begin{axis}[axis x line=  center, axis y line = center,
title={The graphs of the natural exponential and logarithmic functions}]
 
 
\addplot[domain=-4:1.7, samples = 100, color=blue, thick]{e^x} node[right] {$y = e^x$};
\addplot[domain=0.02:6, samples = 100, color=red, thick]{ln(x)} node[anchor = southwest,above left] {$y = \ln(x)$};
\addplot[domain=-4:6, samples = 100, dashed]{x};



%\addplot[&lt;-&gt;] coordinates {(-0.8,-0.5) (-0.8, 2.8)} ; %y-axis
%\addplot[&lt;-&gt;] coordinates {(-1.3,0) (4.3,0)} node[right] {$x$}; %x-axis
%\addplot[thin] coordinates {(1.57,0.1) (1.57, -0.1)} node[below] {$2$};
%\node at (axis cs: 2,2.1){$y = f(x)$};
%\node at (axis cs: 2,-.6){$f(x)$ is continuous at $x = 2$};
%\addplot[smooth,mark=*,blue] plot coordinates {(1.57,1.5)};
 
\end{axis}
\end{tikzpicture}
\end{image}

\section{Derivative of $e^x$}

We begin by computing the derivative of the exponential function $f(x) = e^x$.  
Recall that $e$ is a number whose value is approximately 2.72.
Like $\pi$ and $\sqrt 2$, $e$ is an irrational number, meaning that its decimal representation 
is non-terminating and non-repeating. The significance of the number $e$ in mathematics
is underscored by the simplicity of the differentiation formula we are about to discover. 
The result hinges on a limit that we analyzed in the Numerical Limits section:
\[
\lim_{x \to 0} \frac{e^x - 1}{x} = 1.
\]

The derivative of $f(x) = e^x$ is found as follows:

\begin{center}
$\begin{aligned}
f'(x) &= \lim_{h \to 0} \frac{f(x+h)-f(x)}{h} \\[5pt]
&= \lim_{h \to 0}\frac{e^{x+h}-e^x}{h}\\[5pt]
&= \lim_{h \to 0} \frac{e^x e^h-e^x}{h}\\[5pt]
&= \lim_{h \to 0} \frac{e^x (e^h-1)}{h}\\[5pt]
&= e^x \cdot 1 = e^x.\\[-5pt]
\end{aligned}$
\end{center}

Thus the derivative of $e^x$ is itself!

\begin{proposition}[Derivative of the Natural Exponential]
\[
\frac{d}{dx} e^x = e^x
\]
\end{proposition}


\begin{example}[example 1] Find the equation of the tangent line to the graph of $y = 3e^x - 5$ at $x = 0$.\\
The point of tangency is $(0, f(0)) = (0, -2)$ since $3e^0 -5 = 3-5 = -2$. 
The derivative is $y' = 3e^x$, so the slope of the tangent line is $m = f'(0) = 3e^0 = 3$.
Using the point-slope formula, $y - y_1 = m(x-x_1)$ we can make the equation of the tangent line:
\[
y - (-2) = 3(x-0)
\]
which can be rewritten in slope-intercept form as
\[
y = 3x-2
\]
\end{example}


\begin{problem}(problem 1a)
Find the equation of the tangent line to the graph of $f(x) = e^x$ at $x = 0$.
\begin{hint}
Use the derivative to find the slope, $m$
\end{hint}
\begin{hint}
The point of tangency is $(0, f(0))$
\end{hint}
\begin{hint}
Use the point slope form: $y-y_0 = m(x-x_0)$
\end{hint}

The equation of the tangent line is \ $y = \answer{x+1}$
\end{problem}

\begin{problem}(problem 1b)
Find the equation of the tangent line to the graph of $f(x) = e^x$ at $x = 1$.
\begin{hint}
Use the derivative to find the slope, $m$
\end{hint}
\begin{hint}
The point of tangency is $(1, f(1))$
\end{hint}
\begin{hint}
Use the point slope form: $y-y_0 = m(x-x_0)$
\end{hint}

The equation of the tangent line is \ $y = \answer{ex}$
\end{problem}


\begin{problem}(problem 1c)
Find the following derivatives involving $e^x$:

i) $\frac{d}{dx} \left(5e^x\right) = \answer{5e^x}.$\\
ii) $\frac{d}{dx} \left(-2e^x\right) = \answer{-2e^x}.$\\
iii) $\frac{d}{dx} \left(\frac{e^x}{2}\right) = \answer{\frac{e^x}{2}}.$\\
iv) $\frac{d}{dx} \left(x^2 + e^x\right) = \answer{2x + e^x}.$\\
v) $\frac{d}{dx} \left(\sqrt x - 3e^x\right) = \answer{(1/2)x^{-1/2} - 3e^x}.$\\
vi) $\frac{d}{dx} \left(4e^x + \frac{1}{x^2}\right) = \answer{4e^x -\frac{2}{x^3}}.$


\end{problem}



\section{Derivative of $\ln(x)$}


Next, we use the definition of the derivative to find the derivative of the natural logarithm.



We will need the special limit 

\[
\lim_{x \to 0} \frac{\ln(1 + x)}{x} = 1.
\]

We can use a table of values to verify this limit:
\[
\begin{array}{ c | c | c | c | c }
  x & 0.1 & 0.01  & 0.001 & 0.0001 \\ 
	\hline
	 \frac{\ln(1+x)}{x} & 0.953 & 0.995 & 0.9995 & 0.99995
\end{array}
\]
and
\[
\begin{array}{ c | c | c | c | c }
  x & -0.1 & -0.01  & -0.001 & -0.0001 \\ 
	\hline
	 \frac{\ln(1+x)}{x} & 1.054 & 1.005 & 1.0005 & 1.00005
\end{array}
\]
Now we are prepared to use the definition to find the derivative of $\ln(x)$:

If $f(x) = \ln(x)$ then\\[10pt]
\begin{center}
$\begin{aligned}
f'(x) &= \lim_{h \to 0} \frac{f(x+h)-f(x)}{h}\\[5pt]
&= \lim_{h \to 0}\frac{\ln(x+h)-\ln(x)}{h}\\[5pt]
&= \lim_{h \to 0} \frac{\ln(\frac{x+h}{x})}{h}\\[5pt]
&= \lim_{h \to 0}\frac{\ln(1 + \frac{h}{x})}{h}\\[5pt]
&= \lim_{k \to 0} \frac{\ln(1 + k)}{kx} \\[5pt]
&= \frac{1}{x}.
\end{aligned}$
\end{center}

Note that we made the substitution $k =\frac{h}{x}$ so that $h = kx$ and  also note 
that $h\to 0$ is equivalent to $k\to 0$.
To recap, 
\[
\frac{d}{dx} \ln(x) = \frac{1}{x}
\]
which gives a memorable mathematical relationship between the \textbf{transcendental}, natural logarithm function 
the \textbf{algebraic}, reciprocal function.



\begin{proposition}[Derivative of the Natural Logarithm]
\[
\frac{d}{dx} \ln(x) = \frac{1}{x}
\]
\end{proposition}



\begin{example}[example 2]
 Find $f'(x)$ if $f(x) = \frac{\ln(x)}{\ln(4)}$.\\
 Using the constant multiple rule with $c = \frac{1}{\ln(4)}$, 
 we get 
\[
f'(x) = \tfrac{1}{\ln(4)} \cdot \frac{1}{x} = \frac{1}{x\ln(4)}.
\]
\end{example}



\begin{problem}(problem 2)
Find the following derivatives involving $\ln(x)$:

2a) $\frac{d}{dx} \left(5\ln(x)\right) = \answer{5/x}.$\\
2b) $\frac{d}{dx} \left(-2\ln(x)\right) = \answer{-2/x}.$\\
2c) $\frac{d}{dx} \left(\frac{\ln(x)}{2}\right) = \answer{1/(2x)}.$\\
2d) $\frac{d}{dx} \left(x^3 + \ln(x)\right) = \answer{3x^2 + 1/x}.$\\
2e) $\frac{d}{dx} \left(\sqrt x - 3\ln(x)\right) = \answer{(1/2)x^{-1/2} - 3/x}.$\\
2f) $\frac{d}{dx} \left(4\ln(x) + \frac{1}{x^2}\right) = \answer{4/x -\frac{2}{x^3}}.$


\end{problem}





In the formula $\tfrac{d}{dx} \ln(x) = \tfrac{1}{x}$, the domains of the functions do not match.  
The domain of $1/x$ is all non-zero $x$, but the domain of 
$\ln(x)$  is only $x >0$.  This leads us to ask if there is a function defined for $x<0$ whose derivative is $1/x$. 
The answer is yes and the function is 
$\ln|x|$. In other words,
\[
\dd{x} \ln|x| = \frac{1}{x}.
\]
This result follows from the definition of the derivative exactly as it was used above.  
The absolute value bars will eventually drop in the computation because $1+h/x$ becomes positive as $h \to 0$.
 
\begin{example}[example 3]
Find the slope of the tangent line to the graph of $y = \ln|x|$ at $x = -3$.\\
The slope of the tangent line is $m = f'(-3)$ and since the derivative is $f'(x) = \frac{1}{x}$,
the slope is $-\frac13$.
\end{example}
 
\begin{problem}(problem 3)
Find the slope of the tangent line to the graph of $y= x^2 + \ln|x|$ at $x = -2$.
\[
m = \answer{-9/2}
\]
\end{problem}
 
  
%\begin{center}
\begin{foldable}
\unfoldable{Below is a graph of $f(x) = \ln|x|$ (in blue) and its derivative, $f'(x) = 1/x$ (in purple).
Notice that the slope of the tangent line to $f(x)$ (in red) is the height of the corresponding 
point on $f'(x)$.}
\includeinteractive{lnwithderiv.js}
\end{foldable}
%\end{center}


\end{document}




