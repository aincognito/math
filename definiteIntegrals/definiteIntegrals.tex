\documentclass[handout]{ximera}
\usepgfplotslibrary{fillbetween}
%% You can put user macros here
%% However, you cannot make new environments



\newcommand{\ffrac}[2]{\frac{\text{\footnotesize $#1$}}{\text{\footnotesize $#2$}}}
\newcommand{\vasymptote}[2][]{
    \draw [densely dashed,#1] ({rel axis cs:0,0} -| {axis cs:#2,0}) -- ({rel axis cs:0,1} -| {axis cs:#2,0});
}


\graphicspath{{./}{firstExample/}}
\usepackage{forest}
\usepackage{amsmath}
\usepackage{amssymb}
\usepackage{array}
\usepackage[makeroom]{cancel} %% for strike outs
\usepackage{pgffor} %% required for integral for loops
\usepackage{tikz}
\usepackage{tikz-cd}
\usepackage{tkz-euclide}
\usetikzlibrary{shapes.multipart}


%\usetkzobj{all}
\tikzstyle geometryDiagrams=[ultra thick,color=blue!50!black]


\usetikzlibrary{arrows}
\tikzset{>=stealth,commutative diagrams/.cd,
  arrow style=tikz,diagrams={>=stealth}} %% cool arrow head
\tikzset{shorten <>/.style={ shorten >=#1, shorten <=#1 } } %% allows shorter vectors

\usetikzlibrary{backgrounds} %% for boxes around graphs
\usetikzlibrary{shapes,positioning}  %% Clouds and stars
\usetikzlibrary{matrix} %% for matrix
\usepgfplotslibrary{polar} %% for polar plots
\usepgfplotslibrary{fillbetween} %% to shade area between curves in TikZ



%\usepackage[width=4.375in, height=7.0in, top=1.0in, papersize={5.5in,8.5in}]{geometry}
%\usepackage[pdftex]{graphicx}
%\usepackage{tipa}
%\usepackage{txfonts}
%\usepackage{textcomp}
%\usepackage{amsthm}
%\usepackage{xy}
%\usepackage{fancyhdr}
%\usepackage{xcolor}
%\usepackage{mathtools} %% for pretty underbrace % Breaks Ximera
%\usepackage{multicol}



\newcommand{\RR}{\mathbb R}
\newcommand{\R}{\mathbb R}
\newcommand{\C}{\mathbb C}
\newcommand{\N}{\mathbb N}
\newcommand{\Z}{\mathbb Z}
\newcommand{\dis}{\displaystyle}
%\renewcommand{\d}{\,d\!}
\renewcommand{\d}{\mathop{}\!d}
\newcommand{\dd}[2][]{\frac{\d #1}{\d #2}}
\newcommand{\pp}[2][]{\frac{\partial #1}{\partial #2}}
\renewcommand{\l}{\ell}
\newcommand{\ddx}{\frac{d}{\d x}}

\newcommand{\zeroOverZero}{\ensuremath{\boldsymbol{\tfrac{0}{0}}}}
\newcommand{\inftyOverInfty}{\ensuremath{\boldsymbol{\tfrac{\infty}{\infty}}}}
\newcommand{\zeroOverInfty}{\ensuremath{\boldsymbol{\tfrac{0}{\infty}}}}
\newcommand{\zeroTimesInfty}{\ensuremath{\small\boldsymbol{0\cdot \infty}}}
\newcommand{\inftyMinusInfty}{\ensuremath{\small\boldsymbol{\infty - \infty}}}
\newcommand{\oneToInfty}{\ensuremath{\boldsymbol{1^\infty}}}
\newcommand{\zeroToZero}{\ensuremath{\boldsymbol{0^0}}}
\newcommand{\inftyToZero}{\ensuremath{\boldsymbol{\infty^0}}}


\newcommand{\numOverZero}{\ensuremath{\boldsymbol{\tfrac{\#}{0}}}}
\newcommand{\dfn}{\textbf}
%\newcommand{\unit}{\,\mathrm}
\newcommand{\unit}{\mathop{}\!\mathrm}
%\newcommand{\eval}[1]{\bigg[ #1 \bigg]}
\newcommand{\eval}[1]{ #1 \bigg|}
\newcommand{\seq}[1]{\left( #1 \right)}
\renewcommand{\epsilon}{\varepsilon}
\renewcommand{\iff}{\Leftrightarrow}

\DeclareMathOperator{\arccot}{arccot}
\DeclareMathOperator{\arcsec}{arcsec}
\DeclareMathOperator{\arccsc}{arccsc}
\DeclareMathOperator{\si}{Si}
\DeclareMathOperator{\proj}{proj}
\DeclareMathOperator{\scal}{scal}
\DeclareMathOperator{\cis}{cis}
\DeclareMathOperator{\Arg}{Arg}
%\DeclareMathOperator{\arg}{arg}
\DeclareMathOperator{\Rep}{Re}
\DeclareMathOperator{\Imp}{Im}
\DeclareMathOperator{\sech}{sech}
\DeclareMathOperator{\csch}{csch}
\DeclareMathOperator{\Log}{Log}

\newcommand{\tightoverset}[2]{% for arrow vec
  \mathop{#2}\limits^{\vbox to -.5ex{\kern-0.75ex\hbox{$#1$}\vss}}}
\newcommand{\arrowvec}{\overrightarrow}
\renewcommand{\vec}{\mathbf}
\newcommand{\veci}{{\boldsymbol{\hat{\imath}}}}
\newcommand{\vecj}{{\boldsymbol{\hat{\jmath}}}}
\newcommand{\veck}{{\boldsymbol{\hat{k}}}}
\newcommand{\vecl}{\boldsymbol{\l}}
\newcommand{\utan}{\vec{\hat{t}}}
\newcommand{\unormal}{\vec{\hat{n}}}
\newcommand{\ubinormal}{\vec{\hat{b}}}

\newcommand{\dotp}{\bullet}
\newcommand{\cross}{\boldsymbol\times}
\newcommand{\grad}{\boldsymbol\nabla}
\newcommand{\divergence}{\grad\dotp}
\newcommand{\curl}{\grad\cross}
%% Simple horiz vectors
\renewcommand{\vector}[1]{\left\langle #1\right\rangle}


\outcome{Compute the value of a definite integral using the fundamental theorem.}

\title{4.5 Definite Integrals}

%\newcommand{\ffrac}[2]{\frac{\mbox{\footnotesize $#1$}}{\mbox{\footnotesize $#2$}}}
%\newcommand{\vasymptote}[2][]{\draw [densely dashed,#1] 
%({rel axis cs:0,0} -| {axis cs:#2,0}) -- ({rel axis cs:0,1} -| {axis cs:#2,0});}


\begin{document}

\begin{abstract}
In this section we learn to compute the value of a definite integral using the fundamental theorem of calculus.
\end{abstract}

\maketitle



\section{The Definite Integral}

We begin with the definition of the definite integral.
Recall that a Riemann Sum for the function $f(x)$ on the interval $[a,b]$ using $n$ sample points, $x_i^*$, taken from intervals of width $\Delta x = (b-a)/n$ is given by
\[\sum_{i = 1}^n f(x_i^*)\Delta x.\]
For a non-negative function, a Riemann Sum gives area of $n$ rectangles. Due to the connection between these rectangles and the function $f(x)$,
this sum approximates the area under the curve.  Observing that the approximation approaches the exact value as the number of rectangles increases,
we define the exact area under the curve as a limit of Riemann Sums and we call this limit the definite integral.

\begin{definition}[Definite Integral]
Let $f(x)$ be a continuous function on the closed interval $[a,b]$ and let 
$\Delta x = (b-a)/n$ and $x_i^*$ be a sample point for the interval $[x_{i-1}, x_i]$ where $x_i = a+i\Delta x$. Then we define the \textbf{definite integral} of $f(x)$ over the interval $[a,b]$ to be
\[\int_a^b f(x) \ dx \equiv \lim_{n\to \infty} \sum_{i=1}^n f(x_i^*)\Delta x.\]
\end{definition} 
%The expression $\int_a^b f(x) \, dx$ is called a definite integral. 
The numbers $a$ and $b$ on the integral sign
are called the \textbf{endpoints of integration} and the function $f(x)$ is called the \textbf{integrand}.  
For a non-negative function $f(x)$ on the interval $[a,b]$, the definite integral gives us the exact area under the curve.

If the the graph of the function $f(x)$ is a line or semi-circle, then we may be able to compute the definite integral 
by referring to a familiar geometric formula rather than referring to a Riemann Sum.

%Some integrals can be computed from elementary formulas for the area of specific geometric figures, like triangles, rectangles and circles.

\begin{center}
\textbf{Definite Integrals that represent areas of familiar regions}
\end{center}


\begin{example}[example 1]
Use the fact that the definite integral gives the area under the curve to compute
\[\int_{1}^4 4 \ dx.\]
The graph of the function $f(x) = 4$ is a horizontal line. The region below the curve on the interval $[1,4]$ is a rectangle with base $4-1=3$
and height $4$.  Thus the area is $12$ and the definite integral is 12, i.e.,
\[\int_1^4 4 \ dx =12.\]

\begin{image}
\begin{tikzpicture}
\begin{axis}[axis x line=  bottom, axis y line = left,xtick={0, 1, 2, 3, 4, 5}, ytick={0, 1, 2, 3, 4, 5}, title={The Area Under $f(x)=4$ on $[1,4]$}]
\addplot[name path = A, domain=1:4, samples=100, color=black]{4};
\addplot[name path = B, domain=1:4, samples=100, color=black]{0};
\addplot[blue!10] fill between[of=A and B];
\addplot[domain=0:1, samples=100, color=black]{0};
\addplot[domain=4:5, samples=100, color=black]{0};
\addplot[thin, samples = 100, color=black] coordinates {(1,0) (1,4)};
\addplot[thin, samples = 100, color=black] coordinates {(4,0) (4,4)};


%\addplot[thin, samples = 100, color=black] coordinates {(e,0) (e,1/e)};
%xtick={0, $\pi/4$,$\pi/2$},
\node at (axis cs: 2.5,2){Area = $12$};
\end{axis}
%\legend{$y = f(x)$, , secant line, tangent line, };
\end{tikzpicture}
\end{image}

\end{example}


\begin{problem}(problem 1)
Use geometry to find the value of the definite integral.
\begin{hint}
The definite integral gives the area under the curve
\end{hint}
\[\int_{-2}^{3} 2 \ dx = \answer{10}.\]
\end{problem}


\begin{example}[example 2]
Use the fact that the definite integral gives the area under the curve to compute
\[\int_0^3 2x \ dx.\]
The graph of the function $f(x) = 2x$ is a line with slope $2$ through the origin. 
The region under the graph on the interval $[0,3]$ is a triangle with base $3-0 = 3$ and height $6$. 
Therefore the area is $\frac12 bh = \frac12(3)(6) = 9$ and so
\[\int_0^3 2x \ dx =9.\]


\begin{image}
\begin{tikzpicture}
\begin{axis}[axis x line=  bottom, axis y line = left,xtick={0, 1, 2, 3}, ytick={0, 1, 2, 3, 4,5, 6}, 
title={The Area Under $f(x)=2x$ on $[0,3]$}]
\addplot[name path = A, domain=0:3, 
    samples=100, color=black]{2*x};
\addplot[thin, samples = 100, color=black] coordinates {(3,0) (3,6)};
\addplot[name path = B, domain=0:3, samples=100, color=black]{0};
\addplot[blue!10] fill between[of=A and B];
%\addplot[thin, samples = 100, color=black] coordinates {(e,0) (e,1/e)};
%xtick={0, $\pi/4$,$\pi/2$},
\node at (axis cs: 2,1.3){Area = $9$};
\end{axis}
%\legend{$y = f(x)$, , secant line, tangent line, };
\end{tikzpicture}
\end{image}

\end{example}

\begin{problem}(problem 2)
Use geometry to find the value of the definite integral.
\begin{hint}
The definite integral gives the area under the curve
\end{hint}
\[\int_{0}^{2} (2-x) \ dx = \answer{2}.\]
\end{problem}


\begin{example}[example 3]
Use the fact that the definite integral gives the area under the curve to compute
\[\int_0^2 (x+1) \ dx.\]
The graph of the function $f(x) = x+1$ is a line with slope $1$ and $y$-intercept $1$. 
The region under the curve on the interval $[0, 2]$ is a trapezoid.  
The definite integral represents the area of this trapezoid. The area of this trapezoid can be computed two different ways.  
First, we can use the formula $A = \frac12 (b_1 + b_2)h$. 
In this case, $b_1 = 1, b_2 = 3$ and $h = 2$, so that $A = \frac12 (1+3)(2) = 4$.  
Secondly, we can observe that this particular trapezoid can be decomposed into a 
rectangle with base $2$ and height $1$ and a triangle with base $2$ and height $2$. 
Combining the areas of these figures also gives that the area of the trapezoid is $4$.
Hence the definite integral is $4$, i.e.,
\[\int_0^2 (x+1) \ dx = 4.\]
 




\begin{image}
\begin{tikzpicture}
\begin{axis}[axis x line=  bottom, axis y line = left,xtick={0, 1, 2, 3}, ytick={0, 1, 2, 3, 4,5, 6}, 
title={The Area Under $f(x)=x+1$ on $[0,2]$}]
\addplot[name path = A, domain=0:2, 
    samples=100, color=black]{x+1};
\addplot[thin, samples = 100, color=black] coordinates {(2,0) (2,3)};
\addplot[name path = B, domain=0:2, samples=100, color=black]{0};
\addplot[blue!10] fill between[of=A and B];

%\addplot[thin, samples = 100, color=black] coordinates {(e,0) (e,1/e)};
%xtick={0, $\pi/4$,$\pi/2$},
\node at (axis cs: 1,1){Area = $4$};
\end{axis}
%\legend{$y = f(x)$, , secant line, tangent line, };
\end{tikzpicture}
\end{image}

\end{example}


\begin{problem}(problem 3)
Use geometry to find the value of the definite integral $\displaystyle{\int_{1}^{2} x \ dx}.$
\begin{hint}
The definite integral gives the area under the curve
\end{hint}
\[\int_{1}^{2} x \ dx = \answer{3/2}.\]
\end{problem}

For the next example, we need to know that for a number $r>0$,
the graph of the equation
\[x^2 + y^2 = r^2\]
is a circle centered at the origin with radius $r$.
Solving this equation for $y$, we get:
\[y = \pm \sqrt{r^2 - x^2}.\]
The graph of $y = \sqrt{r^2 -x^2}$ is the upper semi-circle and the
graph of $y = -\sqrt{r^2 -x^2}$ is the lower semi-circle.

\begin{example}[example 4]
Use the fact that the definite integral gives the area under the curve to compute
\[\int_{-2}^2 \sqrt{4-x^2} \ dx.\]
The graph of the equation $x^2 + y^2 = r^2$ is a circle with center at the origin and radius, $r$.
Solving this equation for $y$ yields,
\[
y = \pm \sqrt{r^2 - x^2}.
\]
Taking the positive square root gives the equation of the upper semi-circle and taking the negative square root gives 
the equation of the lower semi-circle.
Hence, the graph of the function $f(x) = \sqrt{4-x^2}$ is an upper semi-circle with center at the origin and radius $2$, shown below. 
The definite integral in question represents the area of this semi-circle, which is given by
\[
A = \frac12 \pi r^2 = \frac12 \pi (2^2) = 2\pi.
\]
Therefore definite integral is $2\pi$, i.e.,
\[\int_{-2}^2 \sqrt{4-x^2} \ dx =2\pi.\]



\begin{image}
\begin{tikzpicture}
\begin{axis}[axis x line=  middle, axis y line =middle,xtick={-1, -2, 0, 1, 2}, ytick={0, 1, 2}, 
title={The area under $f(x)=\sqrt{4-x^2}$ on $[-2, 2]$}]
\addplot[name path = A, domain=-2:2, 
    samples=100, color=black]{(4-x^2)^(1/2)};
\addplot[->, thin, samples = 100, color=black] coordinates {(0,2.2) (0,-.7)};
\addplot[<-, name path = B, domain=-2.1:2.1, samples=100, color=black]{0};
\addplot[blue!10] fill between[of=A and B];

\addplot[thick, color=black] coordinates {(-.05, 2) (.05, 2)};
\addplot[ultra thin, color=black] coordinates {(-.05, 1) (.05, 1)};
\addplot[ultra thin, color=black] coordinates {(1,-.05) (1, .05)};
\addplot[ultra thin, color=black] coordinates {(-1,-.05) (-1,.05)};
%\addplot[thin, samples = 100, color=black] coordinates {(e,0) (e,1/e)};
%xtick={0, $\pi/4$,$\pi/2$},\addplot[ultra thin, color=black] coordinates {(-.05, 1) (.05, 1)};
\node at (axis cs: 0,.5){Area = $2\pi$};
%\addplot[thin, samples = 100, color=blue!10] coordinates {(0,.3) (0,.7)};
\end{axis}
%\legend{$y = f(x)$, , secant line, tangent line, };
\end{tikzpicture}
\end{image}

\end{example}

\begin{problem}(problem 4)
Use geometry to find the value of the definite integral.
\begin{hint}
The definite integral gives the area under the curve
\end{hint}
\[\int_{-3}^{3} \sqrt{9-x^2} \ dx = \answer{9\pi/2}.\]
\end{problem}


The above examples were very special in that the region under the curve had a familiar shape which
made calculating the definite integral an exercise in using elementary geometry formulas. 
In general, the area cannot be computed in this elementary manner.  
The alternative is to use a limit of Riemann Sums.  It turns out that this method is quite laborious (as we will show in the example below), but 
fortunately, there is a beautiful result which reduces the solution process to finding anti-derivatives. 
We will explore this \textbf{Fundamental Theorem of Calculus} in the next section.

\begin{example}[example 5]
Use a Riemann Sum to compute the Definite Integral $\displaystyle{\int_0^1 x^2 \; dx}$.

This integral represents the area under the parabola $y = x^2$ over the interval $[0,1]$. 
We will compute this definite integral using the Riemann Sums, as indicated in the definition of the definite integral.
First, we subdivide the interval $[0,1]$ into $n$ equal sub-intervals, each of length 
$\Delta x= 1/n$ whose endpoints have the form $x_i = i\Delta x = i/n$.  Next, we choose the sample points $x_i^*$
to be right endpoints, so that $x_i^* = x_i = i/n$. Then the Riemann Sum becomes
\[
\sum_{i = 1}^n f(x_i^*) \Delta x =  \sum_{i = 1}^n \left(x_i^*\right)^2 \Delta x = \sum_{i = 1}^n \left(\frac{i}{n}\right)^2 \cdot \frac{1}{n}.
\]
We will evaluate this sum before letting $n\to \infty$. Using the formula
\[
\sum_{i = 1}^n i^2 = \frac{n(n+1)(2n+1)}{6},
\]
our Riemann Sum becomes
\[
\sum_{i = 1}^n \left(\frac{i}{n}\right)^2 \cdot \frac{1}{n} = \frac{1}{n^3} \sum_{i = 1}^n i^2 = \frac{n(n+1)(2n+1)}{6n^3}.
\]
Finally, we can compute the definite integral by taking the limit as $n \to \infty$ and noting that the polynomials in the formula above are both of degree 3:
\[
\int_0^1 x^2 \; dx \equiv \lim_{n \to \infty} \sum_{i = 1}^n \left(x_i^*\right)^2 \Delta x = \lim_{n \to \infty} \frac{n(n+1)(2n+1)}{6n^3} = \frac{1}{3},
\]
by simplifying the ratio of the lead coefficients, $2/6$.
As mentioned earlier, there is a more elegant method for solving this problem using the Fundamental Theorem of Calculus, presented in the next section.

\end{example}



\end{document}









