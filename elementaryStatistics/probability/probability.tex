\documentclass[handout]{ximera}

%% You can put user macros here
%% However, you cannot make new environments



\newcommand{\ffrac}[2]{\frac{\text{\footnotesize $#1$}}{\text{\footnotesize $#2$}}}
\newcommand{\vasymptote}[2][]{
    \draw [densely dashed,#1] ({rel axis cs:0,0} -| {axis cs:#2,0}) -- ({rel axis cs:0,1} -| {axis cs:#2,0});
}


%\usepackage{tcolorbox} %%Needed for Derivative Definition supposedly and product rule, natural exp log, quotient rule, inverse trig, rates of change


% \graphicspath{{./}{firstExample/}}
% \usepackage{forest}
\usepackage{amsmath}
\usepackage{amssymb}
\usepackage{array}
\usepackage[makeroom]{cancel} %% for strike outs
\usepackage{pgffor} %% required for integral for loops
\usepackage{tikz}
\usepackage{tikz-cd}
\usepackage{tkz-euclide}
\usetikzlibrary{shapes.multipart}


% \usetkzobj{all}
\tikzstyle geometryDiagrams=[ultra thick,color=blue!50!black]


\usetikzlibrary{arrows}
\tikzset{>=stealth,commutative diagrams/.cd,
  arrow style=tikz,diagrams={>=stealth}} %% cool arrow head
\tikzset{shorten <>/.style={ shorten >=#1, shorten <=#1 } } %% allows shorter vectors

\usetikzlibrary{backgrounds} %% for boxes around graphs
\usetikzlibrary{shapes,positioning}  %% Clouds and stars
\usetikzlibrary{matrix} %% for matrix
\usepgfplotslibrary{polar} %% for polar plots
\usepgfplotslibrary{fillbetween} %% to shade area between curves in TikZ



%\usepackage[width=4.375in, height=7.0in, top=1.0in, papersize={5.5in,8.5in}]{geometry}
%\usepackage[pdftex]{graphicx}
%\usepackage{tipa}
%\usepackage{txfonts}
%\usepackage{textcomp}
%\usepackage{amsthm}
%\usepackage{xy}
%\usepackage{fancyhdr}
%\usepackage{xcolor}
%\usepackage{mathtools} %% for pretty underbrace % Breaks Ximera
%\usepackage{multicol}



\newcommand{\RR}{\mathbb R}
\newcommand{\R}{\mathbb R}
\newcommand{\C}{\mathbb C}
\newcommand{\N}{\mathbb N}
\newcommand{\Z}{\mathbb Z}
\newcommand{\dis}{\displaystyle}
%\renewcommand{\d}{\,d\!}
\renewcommand{\d}{\mathop{}\!d}
\newcommand{\dd}[2][]{\frac{\d #1}{\d #2}}
\newcommand{\pp}[2][]{\frac{\partial #1}{\partial #2}}
\renewcommand{\l}{\ell}
\newcommand{\ddx}{\frac{d}{\d x}}
\newcommand{\ppx}{\frac{\partial}{\partial x}}
\newcommand{\ppy}{\frac{\partial}{\partial y}}

\newcommand{\zeroOverZero}{\ensuremath{\boldsymbol{\tfrac{0}{0}}}}
\newcommand{\inftyOverInfty}{\ensuremath{\boldsymbol{\tfrac{\infty}{\infty}}}}
\newcommand{\zeroOverInfty}{\ensuremath{\boldsymbol{\tfrac{0}{\infty}}}}
\newcommand{\zeroTimesInfty}{\ensuremath{\small\boldsymbol{0\cdot \infty}}}
\newcommand{\inftyMinusInfty}{\ensuremath{\small\boldsymbol{\infty - \infty}}}
\newcommand{\oneToInfty}{\ensuremath{\boldsymbol{1^\infty}}}
\newcommand{\zeroToZero}{\ensuremath{\boldsymbol{0^0}}}
\newcommand{\inftyToZero}{\ensuremath{\boldsymbol{\infty^0}}}


\newcommand{\numOverZero}{\ensuremath{\boldsymbol{\tfrac{\#}{0}}}}
\newcommand{\dfn}{\textbf}
%\newcommand{\unit}{\,\mathrm}
\newcommand{\unit}{\mathop{}\!\mathrm}
%\newcommand{\eval}[1]{\bigg[ #1 \bigg]}
\newcommand{\eval}[1]{ #1 \bigg|}
\newcommand{\seq}[1]{\left( #1 \right)}
\renewcommand{\epsilon}{\varepsilon}
\renewcommand{\iff}{\Leftrightarrow}

\DeclareMathOperator{\arccot}{arccot}
\DeclareMathOperator{\arcsec}{arcsec}
\DeclareMathOperator{\arccsc}{arccsc}
\DeclareMathOperator{\si}{Si}
\DeclareMathOperator{\proj}{proj}
\DeclareMathOperator{\scal}{scal}
\DeclareMathOperator{\cis}{cis}
\DeclareMathOperator{\Arg}{Arg}
%\DeclareMathOperator{\arg}{arg}
\DeclareMathOperator{\Rep}{Re}
\DeclareMathOperator{\Imp}{Im}
\DeclareMathOperator{\sech}{sech}
\DeclareMathOperator{\csch}{csch}
\DeclareMathOperator{\Log}{Log}

\newcommand{\tightoverset}[2]{% for arrow vec
  \mathop{#2}\limits^{\vbox to -.5ex{\kern-0.75ex\hbox{$#1$}\vss}}}
\newcommand{\arrowvec}{\overrightarrow}
\renewcommand{\vec}{\mathbf}
\newcommand{\veci}{{\boldsymbol{\hat{\imath}}}}
\newcommand{\vecj}{{\boldsymbol{\hat{\jmath}}}}
\newcommand{\veck}{{\boldsymbol{\hat{k}}}}
\newcommand{\vecl}{\boldsymbol{\l}}
\newcommand{\utan}{\vec{\hat{t}}}
\newcommand{\unormal}{\vec{\hat{n}}}
\newcommand{\ubinormal}{\vec{\hat{b}}}

\newcommand{\dotp}{\bullet}
\newcommand{\cross}{\boldsymbol\times}
\newcommand{\grad}{\boldsymbol\nabla}
\newcommand{\divergence}{\grad\dotp}
\newcommand{\curl}{\grad\cross}
%% Simple horiz vectors
\renewcommand{\vector}[1]{\left\langle #1\right\rangle}


\pgfplotsset{compat=1.13}

\outcome{Learn the laws of probability}

\title{1.1 Probability}


%\documentclass[11pt,fleqn]{article}

%\usepackage{amssymb,amsmath,amsthm,amsfonts}
%\usepackage{textcomp}
%\usepackage{tikz}
%\usepackage{color}
%\usepackage{xcolor}
%\usepackage{graphicx}
%\usepackage{indentfirst}
%\usepackage{multirow}
%\usepackage{bbm}
%\usepackage{float}
%\usepackage[fleqn,tbtags]{mathtools}
%\usepackage{hyperref}
%\usepackage[hidelinks]{hyperref}

%\hypersetup{
 %   colorlinks = true,
 %   linkcolor = blue,
 %   linkbordercolor = {white}%,
 %   }

%\setlength{\oddsidemargin}{0in}
%\setlength{\evensidemargin}{0in}
%\setlength{\textwidth}{6in}
%\setlength{\textheight}{8in}

\newcommand {\be} {\begin{equation}}
\newcommand {\ee} {\end{equation}}
\newcommand {\by} {\begin{eqnarray}}
\newcommand {\ey} {\end{eqnarray}}
\newcommand {\bs} {\begin{eqnarray*}}
\newcommand {\es} {\end{eqnarray*}}
\newcommand {\bc} {\begin{center}}
\newcommand {\ec} {\end{center}}
\newcommand {\bi} {\begin{itemize}}
\newcommand {\ei} {\end{itemize}}
\newcommand {\bn} {\begin{enumerate}}
\newcommand {\en} {\end{enumerate}}
\newcommand{\ds}{\displaystyle}

\renewcommand \baselinestretch{1}
\renewcommand{\theequation} {\arabic{section}.\arabic{equation}}


\begin{document}

\begin{abstract}
We will learn the laws of probability to compute the probability of an event.
\end{abstract}

\maketitle
\section*{Unit 1 (Part I): Probability}

\bi

\item \textbf{{\color{red} Question: What is Probability?}}

\bi
 \item Probability is the study of randomness.
\ei

\item \textbf{{\color{red} Question: What does it mean to say something is ``Random''?}}

 \bi
  \item Each outcome has some chance of occurring.
  \item We know the possible outcomes, just uncertain which will occur.
  \item \textbf{{\color{green} Key: A pattern (distribution) of the outcomes emerges only in the ``Long-run''.}}
 \ei

\item \underline{Three Types of Probability:}
  \bi
   \item Subjective - based on opinions not fact or experiments.
    \item Empirical - based on an experiment where the certain ``event (A)'' of interest is performed numerous times.
      \bi
        \item[] $P(A) =$ $\#$ of times $A$ occurred / $\#$ of trials.
      \ei
    \item Theoretical - based on the assumption that each outcome of an event is equally likely to occur.
       \bi
        \item[] $P(A) =$ Number of ways $A$ can occur / number of possible outcomes.
       \ei
   \ei

 \item \textbf{{\color{blue} View Class applet: ``Empirical vs. Theoretical''}}

 \vspace{1.5in}

 \item \underline{Recap: Applet}
     \bi
      \item[]  empirical prob. (proportion) $\rightarrow$ theoretical probability as $n \rightarrow \infty$
      \item[]  Sums [$\sum_{i=1}^n X_i$] or counts $\nrightarrow$ (No single point) as $n \rightarrow \infty$
     \ei

 \item \underline{Basic Definitions:}

   \bi
	  \item Probability Experiment - an action whose outcomes can be determined and recorded.
    \item Sample Space $(S)$ - Set of all possible outcomes.
    \item Event $(E)$ - is an particular outcome or a set of outcomes.
   \ei
	
	\begin{example}
	Roll a six-sided die and record the number on the top face.\\
	This is a probability experiment because the outcomes can be determined and we can 
	record them as numbers from 1 to 6.
	The sample space of the experiment is 
	\[
	S = \{ 1, 2, 3, 4, 5, 6\}
	\]
	Any subset of the sample space $S$ is an event. For example, we can write
	\[
	E_1 = \{1, 2, 3\}
	\]
	or
	\[
	E_2: \mbox{roll a prime number}
	\]
	The event $E_2$ can also be described using set notation as
	\[
	E_2 = \{2, 3, 5\} \;\; \mbox{(note: 1 is not considered a prime number)}
	\]
	A single outcome is also considered an event:
	\[
	E_3: \mbox{roll a} \; 6, \;\mbox{i.e.}, \; E_3 = \{6\}
	\]
	An event consisting of a single outcome is sometimes called a simple event.
	
	\end{example}
	
	\begin{problem}
	Consider the probability experiment of rolling a six-sided die and recording the number on the top face. Describe each of the following events as a subset of the sample space $S = \{1, 2, 3, 4, 5, 6\}$.\\
	$E_1: \mbox{an even number}$
	\begin{multipleChoice}
	\choice{$E_1 = \{2\}$}\\
	\choice{$E_1 = \{2, 4\}$}\\
	\choice[correct]{$E_1 = \{2, 4, 6\}$}
	\end{multipleChoice}
	
	$E_2: \mbox{at least 3}$
	\begin{multipleChoice}
	\choice{$E_2 = \{3, 6\}$}\\
	\choice{$E_2 = \{4, 5, 6\}$}\\
	\choice[correct]{$E_2 = \{3, 4, 5, 6\}$}
	\end{multipleChoice}
	
	$E_3: \mbox{at most 4}$
	\begin{multipleChoice}
	\choice{$E_3 = \{1, 2, 3\}$}\\
	\choice[correct]{$E_3 = \{1, 2, 3, 4\}$}\\
	\choice{$E_3 = \{1, 4\}$}
	\end{multipleChoice}
	
	\end{problem}
	
	
	
	\begin{example}
	Flip a coin twice and record the sequence of heads and tails.\\
	This is a probability experiment with sample space consisting of four outcomes:
	\[
	S = \{HH, HT, TH, TT\}
	\]
	One possible event is
	\[
	E: \mbox{at least one heads}
	\]
	As a subset of the sample space this event can be written as
	\[
	E = \{HH, HT, TH\}
	\]
	The event
	\[
	F: \mbox{two tails}
	\]
	is a simple event consisting of the single outcome TT.
	
	\end{example}
	
	\begin{example}
	Select a card from a deck of 52 ordinary playing cards and record the cards type and suit.\\
	The sample space of this probability experiment consists of the 52 individual cards in the deck described using a type and a suit.  Some examples of individual outcomes are:
	\[
	K\heartsuit, 5 \clubsuit \; \mbox {and} \; A\spadesuit
	\]
	An example event is
	\[
	E: \mbox{select a Queen}
	\]
	which can be represented as a subset of the sample space by
	\[
	E = \{Q\heartsuit, Q\diamondsuit, Q\clubsuit, Q\spadesuit\}
	\]
	Another example even is
	\[
 F: \mbox{select a club}
\]
This event can be represented by the set
\[
F = \{2\clubsuit,3\clubsuit,\dots,10\clubsuit,J\clubsuit,Q\clubsuit,K\clubsuit, A\clubsuit\}
\]
Note the use of the dots to continue the counting pattern from 3 to 10.
	\end{example}
	

 \newpage

 \item \underline{\textbf{{\color{blue} How do we use Probability in Statistics?:}}}
   \bi
    \item \underline{Rare Event Rule}: If, \underline{\makebox[3in][c]{ }}, the probability of a particular observed event is extremely small, we conclude that the assumption is probably \underline{\makebox[2in][c]{ }}.
   \ei
  \item[] \underline{Motivating Example}: Let $E$ denote the event of a student picking the number 3 from the set $A = \{1,2,3,4\}$. (a) What is the Theoretical Probability? (b) What is the Empirical Probability? (c) What is the Status Quo (Null Hypothesis)? (d) Does the Rare Event Rule apply in this Situation?

\newpage


 \item[] \underline{Example 1:} Suppose a family has 3 children where boy and girl are equally likely. What is the probability of having at least 2 girls?


  \vspace{3.5in}


 \item \underline{Probability Rules:}

   \bi
    \item[1.] $P(S) = 1$
    \item[2.] $P(A) + P(\bar{A}) = 1$
    \item[3.] $0 \leq P(A) \leq 1$
   \ei

 \item[1.] \underline{Complement Rule}:

    \vspace{1.5in}
    
    \newpage

 \item[2.] \underline{Addition Rule}:

    \vspace{2.75in}

 \item[3.] \underline{Disjoint (Mutually Exclusive)}:

    \vspace{2.75in}

\newpage

 \item[] \underline{Example 2:} Is the following Grade distribution valid?
\begin{center}
 \begin{tabular}{|c c c c c|}
 \hline
 A & B & C & D & F \\
 \hline\hline
 .40 & .20 & .30 & .07 & .05 \\
 \hline
\end{tabular}
\end{center}

\vspace{1.5in}

 \item[] \underline{Example 3:} Let $S = \{1,2,3,4,5,6,7\}$, $A = \{1,2,3\}$, $B = \{3,7\}$, $C = \{1,2,4,6\}$. Develop equations to calculate the following probabilities.
     \bi
      \item[a)] $P(\bar{C})$

       \vspace{1.75in}

      \item[b)] $P(A \cup B)$

       \vspace{1.75in}

      \item[c)] $P(A \cap C)$

       \vspace{1.75in}

      \item[d)] $P(A \cap \bar{B})$

       \vspace{1.75in}
     \ei


 \item \underline{Conditional Probability:}
   \bi
    \item For any two events $A$ and $B$, the conditional probability of $A$ given $B$ is defined by
      \bs
        P(A|B) = \frac{P(A \cap B)} {P(B)}.
      \es

    \vspace{2.25in}
   \ei
 \item \underline{Independence:}
  \bi
   \item Two events $A$ and $B$ are said to be \textit{independent} if the outcome of one event does not affect the outcome of the other event.
   \item To verify that two events are independent
     \bi
      \item[1.] $P(A|B) = P(A)$
      \item[2.] $P(A \cap B) = P(A) \cdot P(B)$
     \ei
  \ei

\vspace{.2in}

\newpage

\item[] \underline{Example 4:} Consider a Basket with 3 red chips and 5 blue chips. Two chips are selected at random \textbf{without} replacement.
     \bi
      \item[a)] What is the probability of a red chip on the 1st draw?
       \vspace{1.95in}
      \item[b)] What is the probability of a red chip on the 2nd draw?
        \vspace{1.95in}
     \ei


 \item[] \underline{Example 5:} Consider a standard deck of cards. $K = \{$ drawing a king $\}$, $A = \{$ drawing an Ace $\}$. Find the following probabilities.
    \bi
      \item[a)] $P(K)$

      \vspace{1.4in}

      \item[b)] $P(A)$

      \vspace{1.4in}

      \item[c)] $P(K|A)$

      \vspace{1.4in}

      \item[d)] Are the two events independent?

      \vspace{1.4in}

      \item[e)] How does the results in part (c) change when we use ``with'' replacement?

      \vspace{1.4in}

      \item[f)] $P(K \cap A)$

      \vspace{1.4in}

    \ei


\item[] \underline{Mutually Exclusive:}
   \bi
    \item Two events are said to be mutually exclusive if the two events have nothing in common or cannot occur at the same time.
    \item To verify that two events are mutually exclusive (disjoint)
     \bi
      \item[1.] $P(A|B) = 0$
      \item[2.] $P(A \cap B) = 0$
     \ei
   \ei

 \item[] \underline{Example 6:} Based on the data below, is there an association between speeding tickets and gender?

   \vspace{4.5in}

 \item[] \underline{General Probability Equations:}
  \bi
   \item[1.] $P(\bar{A}) = 1 - P(A)$
   \item[2.] $P(A \cup B) = P(A) + P(B) - P(A \cap B)$
   \item[3.] $P(A \cap \bar{B}) = P(A) - P(A \cap B)$
   \item[4.] $P(A | B) = P(A \cap B)/P(B) $
   \item[5.] $P(A \cap B) = P(A) \cdot P(B|A)$
 \ei


 \ei


\end{document}






 \item[] \underline{Example 5:} If $P(A) = 3/4, P(\bar{B}) = 1/2,$ can $A$ and $B$ be disjoint?

  \vspace{.75in}

  \item[] \underline{Example 6:} If $A$ and $B$ are independent show that $A$ and $\bar{B}$ are independent.

  \vspace{1in}

   \item[] \underline{Example 7:} (Total Law of Probability). A bag contains 4 white chips and 3 red chips. A second bag contains 3 white chips and 5 red chips. One chip is selected from the 1st bag and placed into the second bag. What is the probability that a red chip is drawn from the 2nd bag?


   \vspace{2.25in}


    \item[] \underline{Example 8:} (Bayes Rule). Suppose that we have two bags each containing red and white chips. One bag contains three times as many white chips as red. The other bag contains three times as many red chips as white. Suppose we choose one of these bags at random. For this bag we select five balls at random, replacing each ball after it has been selected. The result is that we find 4 white balls and one red. What is the probability that we were using the bag with mainly white balls?


\newpage

\item[] \underline{General Probability Equations:}
  \bi
   \item[1.] $P(\bar{A}) = 1 - P(A)$
   \item[2.] $P(A \cup B) = P(A) + P(B) - P(A \cap B)$
   \item[3.] $P(A \cap \bar{B}) = P(A) - P(A \cap B)$
   \item[4.] $P(A | B) = P(A \cap B)/P(B) $
   \item[5.] $P(A \cap B) = P(A) \cdot P(B|A)$
   \item[6.] $P(A) = \sum_{i=1}^k P(B_i) \cdot P(A | B_i)$
   \item[7.] $P(B_j|A) = P(B_j) \cdot P(A|B_j)/P(A)$
 \ei





 \ei



\end{document}







