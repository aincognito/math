\documentclass[handout]{ximera}
% \usepackage{tcolorbox}
%% You can put user macros here
%% However, you cannot make new environments



\newcommand{\ffrac}[2]{\frac{\text{\footnotesize $#1$}}{\text{\footnotesize $#2$}}}
\newcommand{\vasymptote}[2][]{
    \draw [densely dashed,#1] ({rel axis cs:0,0} -| {axis cs:#2,0}) -- ({rel axis cs:0,1} -| {axis cs:#2,0});
}


%\usepackage{tcolorbox} %%Needed for Derivative Definition supposedly and product rule, natural exp log, quotient rule, inverse trig, rates of change


% \graphicspath{{./}{firstExample/}}
% \usepackage{forest}
\usepackage{amsmath}
\usepackage{amssymb}
\usepackage{array}
\usepackage[makeroom]{cancel} %% for strike outs
\usepackage{pgffor} %% required for integral for loops
\usepackage{tikz}
\usepackage{tikz-cd}
\usepackage{tkz-euclide}
\usetikzlibrary{shapes.multipart}


% \usetkzobj{all}
\tikzstyle geometryDiagrams=[ultra thick,color=blue!50!black]


\usetikzlibrary{arrows}
\tikzset{>=stealth,commutative diagrams/.cd,
  arrow style=tikz,diagrams={>=stealth}} %% cool arrow head
\tikzset{shorten <>/.style={ shorten >=#1, shorten <=#1 } } %% allows shorter vectors

\usetikzlibrary{backgrounds} %% for boxes around graphs
\usetikzlibrary{shapes,positioning}  %% Clouds and stars
\usetikzlibrary{matrix} %% for matrix
\usepgfplotslibrary{polar} %% for polar plots
\usepgfplotslibrary{fillbetween} %% to shade area between curves in TikZ



%\usepackage[width=4.375in, height=7.0in, top=1.0in, papersize={5.5in,8.5in}]{geometry}
%\usepackage[pdftex]{graphicx}
%\usepackage{tipa}
%\usepackage{txfonts}
%\usepackage{textcomp}
%\usepackage{amsthm}
%\usepackage{xy}
%\usepackage{fancyhdr}
%\usepackage{xcolor}
%\usepackage{mathtools} %% for pretty underbrace % Breaks Ximera
%\usepackage{multicol}



\newcommand{\RR}{\mathbb R}
\newcommand{\R}{\mathbb R}
\newcommand{\C}{\mathbb C}
\newcommand{\N}{\mathbb N}
\newcommand{\Z}{\mathbb Z}
\newcommand{\dis}{\displaystyle}
%\renewcommand{\d}{\,d\!}
\renewcommand{\d}{\mathop{}\!d}
\newcommand{\dd}[2][]{\frac{\d #1}{\d #2}}
\newcommand{\pp}[2][]{\frac{\partial #1}{\partial #2}}
\renewcommand{\l}{\ell}
\newcommand{\ddx}{\frac{d}{\d x}}
\newcommand{\ppx}{\frac{\partial}{\partial x}}
\newcommand{\ppy}{\frac{\partial}{\partial y}}

\newcommand{\zeroOverZero}{\ensuremath{\boldsymbol{\tfrac{0}{0}}}}
\newcommand{\inftyOverInfty}{\ensuremath{\boldsymbol{\tfrac{\infty}{\infty}}}}
\newcommand{\zeroOverInfty}{\ensuremath{\boldsymbol{\tfrac{0}{\infty}}}}
\newcommand{\zeroTimesInfty}{\ensuremath{\small\boldsymbol{0\cdot \infty}}}
\newcommand{\inftyMinusInfty}{\ensuremath{\small\boldsymbol{\infty - \infty}}}
\newcommand{\oneToInfty}{\ensuremath{\boldsymbol{1^\infty}}}
\newcommand{\zeroToZero}{\ensuremath{\boldsymbol{0^0}}}
\newcommand{\inftyToZero}{\ensuremath{\boldsymbol{\infty^0}}}


\newcommand{\numOverZero}{\ensuremath{\boldsymbol{\tfrac{\#}{0}}}}
\newcommand{\dfn}{\textbf}
%\newcommand{\unit}{\,\mathrm}
\newcommand{\unit}{\mathop{}\!\mathrm}
%\newcommand{\eval}[1]{\bigg[ #1 \bigg]}
\newcommand{\eval}[1]{ #1 \bigg|}
\newcommand{\seq}[1]{\left( #1 \right)}
\renewcommand{\epsilon}{\varepsilon}
\renewcommand{\iff}{\Leftrightarrow}

\DeclareMathOperator{\arccot}{arccot}
\DeclareMathOperator{\arcsec}{arcsec}
\DeclareMathOperator{\arccsc}{arccsc}
\DeclareMathOperator{\si}{Si}
\DeclareMathOperator{\proj}{proj}
\DeclareMathOperator{\scal}{scal}
\DeclareMathOperator{\cis}{cis}
\DeclareMathOperator{\Arg}{Arg}
%\DeclareMathOperator{\arg}{arg}
\DeclareMathOperator{\Rep}{Re}
\DeclareMathOperator{\Imp}{Im}
\DeclareMathOperator{\sech}{sech}
\DeclareMathOperator{\csch}{csch}
\DeclareMathOperator{\Log}{Log}

\newcommand{\tightoverset}[2]{% for arrow vec
  \mathop{#2}\limits^{\vbox to -.5ex{\kern-0.75ex\hbox{$#1$}\vss}}}
\newcommand{\arrowvec}{\overrightarrow}
\renewcommand{\vec}{\mathbf}
\newcommand{\veci}{{\boldsymbol{\hat{\imath}}}}
\newcommand{\vecj}{{\boldsymbol{\hat{\jmath}}}}
\newcommand{\veck}{{\boldsymbol{\hat{k}}}}
\newcommand{\vecl}{\boldsymbol{\l}}
\newcommand{\utan}{\vec{\hat{t}}}
\newcommand{\unormal}{\vec{\hat{n}}}
\newcommand{\ubinormal}{\vec{\hat{b}}}

\newcommand{\dotp}{\bullet}
\newcommand{\cross}{\boldsymbol\times}
\newcommand{\grad}{\boldsymbol\nabla}
\newcommand{\divergence}{\grad\dotp}
\newcommand{\curl}{\grad\cross}
%% Simple horiz vectors
\renewcommand{\vector}[1]{\left\langle #1\right\rangle}


\outcome{Compute the derivative of a composition}

\title{2.15 Rates of Change}



\begin{document}

\begin{abstract}
In this section we interpret the derivative as an instantaneous rate of change.
\end{abstract}

\maketitle

\section{Rates of Change}

We begin by comparing the notion instantaneous rate of change to average rate of change. 
For two points $(x_1, y_1)$ and $(x_2, y_2)$ with $x_1 < x_2$ on the graph of a function
$y = f(x)$, the slope of the secant line connecting them is 
\[m = \frac{\text{rise}}{\text{run}} = \frac{\Delta y}{\Delta x} = \frac{y_2 - y_1}{x_2 - x_1}.\]
This slope can be interpreted as the average rate of change of $y$ with respect to $x$ over the interval $[x_1, x_2]$.

The instantaneous rate of change of $y$ with respect to $x$ at $x = x_1$ is denoted by
\[\frac{dy}{dx}\Big{|}_{x=x_1},\]
and it is obtained by letting $x_2$ approach $x_1$ in the formula for $\frac{\Delta y}{\Delta x}.$
Graphically, 
\[\frac{dy}{dx}\Big{|}_{x=x_1}\]
is the slope of the tangent line to the graph of $y = f(x)$ at $x=x_1$.
Generally speaking, $\frac{\Delta y}{\Delta x}$
represents the average rate of change of $y$ relative to $x$ whereas
$\frac{dy}{dx}$
represents the instantaneous rate of change of $y$ relative to $x$.

%More generally, $\frac{dy}{dx}$ is a formula for the instantaneous rate of change of $y$ with respect to $x$.

\begin{center}
\bf{Examples of Rates of Change}
\end{center}
 
\begin{example}[example 1]

If $y = f(t)$ is the distance traveled by an object as a function of time, then 
\[\frac{\Delta y}{\Delta t}\]
is the average speed of the object and
\[\frac{dy}{dt}\]
is the instantaneous speed of the object.
\end{example}

\begin{example} If $V = f(t)$ is the volume of water in a tank as a function of time, then 
\[\frac{\Delta V}{\Delta t}\]
is the average rate that water is entering (or leaving, if this is negative) the tank and
\[\frac{dV}{dt}\]
is the instantaneous rate that water is entering (or leaving) the tank.
 \end{example}

\begin{example}[example 2]
 If $C = f(x)$ is the cost of producing $x$ units then $\frac{dC}{dx}$ represents the instantaneous rate of change 
of cost with respect to the number of units produced.  This is called the {\it marginal cost} of production. 
\end{example}


\begin{example}[example 3]
 If $P(t)$ represents population at time $t$, then 
\[\frac{\Delta P}{\Delta t}\]
is the average growth rate of the population and
\[\frac{dP}{dt}\]
is the instantaneous growth rate of the population. If the population of a certain bacteria is 50 (thousand) and it doubles every hour, 
then what is the population after 3 hours?  How fast is the population growing at $t = 3$ hours? First, from the description of the population, 
we have $P(t) = 50 \cdot 2^t$.  Hence the population after 3 hours is given by $P(3) = 50\cdot 2^3 = 400$ (thousand) bacteria.
The growth rate at time $t=3$ is given by
\[\frac{dP}{dt}\Big{|}_{t=3}.\]
Computing the derivative, we have 
\[\frac{dP}{dt}= 50 \cdot 2^t \cdot \ln(2),\]
and hence,
\[\frac{dP}{dt}\Big{|}_{t=3} = 50 \cdot 2^3 \cdot \ln(2) = 400\ln(2) \approx 277.26 \mbox{(thousand) bacteria per hour}.\]
\end{example}

\begin{example}[example 4]
 If $Q(t)$ represents the quantity of a (medical) drug in the bloodstream at time $t$, then 
\[\frac{\Delta Q}{\Delta t}\]
is the average rate of change of the quantity in the bloodstream and
\[\frac{dQ}{dt}\]
is the instantaneous rate of change of the quantity in the bloodstream.


\end{example}

\end{document}
