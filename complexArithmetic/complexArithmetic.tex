\documentclass[handout]{ximera}

%% You can put user macros here
%% However, you cannot make new environments



\newcommand{\ffrac}[2]{\frac{\text{\footnotesize $#1$}}{\text{\footnotesize $#2$}}}
\newcommand{\vasymptote}[2][]{
    \draw [densely dashed,#1] ({rel axis cs:0,0} -| {axis cs:#2,0}) -- ({rel axis cs:0,1} -| {axis cs:#2,0});
}


\graphicspath{{./}{firstExample/}}
\usepackage{forest}
\usepackage{amsmath}
\usepackage{amssymb}
\usepackage{array}
\usepackage[makeroom]{cancel} %% for strike outs
\usepackage{pgffor} %% required for integral for loops
\usepackage{tikz}
\usepackage{tikz-cd}
\usepackage{tkz-euclide}
\usetikzlibrary{shapes.multipart}


%\usetkzobj{all}
\tikzstyle geometryDiagrams=[ultra thick,color=blue!50!black]


\usetikzlibrary{arrows}
\tikzset{>=stealth,commutative diagrams/.cd,
  arrow style=tikz,diagrams={>=stealth}} %% cool arrow head
\tikzset{shorten <>/.style={ shorten >=#1, shorten <=#1 } } %% allows shorter vectors

\usetikzlibrary{backgrounds} %% for boxes around graphs
\usetikzlibrary{shapes,positioning}  %% Clouds and stars
\usetikzlibrary{matrix} %% for matrix
\usepgfplotslibrary{polar} %% for polar plots
\usepgfplotslibrary{fillbetween} %% to shade area between curves in TikZ



%\usepackage[width=4.375in, height=7.0in, top=1.0in, papersize={5.5in,8.5in}]{geometry}
%\usepackage[pdftex]{graphicx}
%\usepackage{tipa}
%\usepackage{txfonts}
%\usepackage{textcomp}
%\usepackage{amsthm}
%\usepackage{xy}
%\usepackage{fancyhdr}
%\usepackage{xcolor}
%\usepackage{mathtools} %% for pretty underbrace % Breaks Ximera
%\usepackage{multicol}



\newcommand{\RR}{\mathbb R}
\newcommand{\R}{\mathbb R}
\newcommand{\C}{\mathbb C}
\newcommand{\N}{\mathbb N}
\newcommand{\Z}{\mathbb Z}
\newcommand{\dis}{\displaystyle}
%\renewcommand{\d}{\,d\!}
\renewcommand{\d}{\mathop{}\!d}
\newcommand{\dd}[2][]{\frac{\d #1}{\d #2}}
\newcommand{\pp}[2][]{\frac{\partial #1}{\partial #2}}
\renewcommand{\l}{\ell}
\newcommand{\ddx}{\frac{d}{\d x}}

\newcommand{\zeroOverZero}{\ensuremath{\boldsymbol{\tfrac{0}{0}}}}
\newcommand{\inftyOverInfty}{\ensuremath{\boldsymbol{\tfrac{\infty}{\infty}}}}
\newcommand{\zeroOverInfty}{\ensuremath{\boldsymbol{\tfrac{0}{\infty}}}}
\newcommand{\zeroTimesInfty}{\ensuremath{\small\boldsymbol{0\cdot \infty}}}
\newcommand{\inftyMinusInfty}{\ensuremath{\small\boldsymbol{\infty - \infty}}}
\newcommand{\oneToInfty}{\ensuremath{\boldsymbol{1^\infty}}}
\newcommand{\zeroToZero}{\ensuremath{\boldsymbol{0^0}}}
\newcommand{\inftyToZero}{\ensuremath{\boldsymbol{\infty^0}}}


\newcommand{\numOverZero}{\ensuremath{\boldsymbol{\tfrac{\#}{0}}}}
\newcommand{\dfn}{\textbf}
%\newcommand{\unit}{\,\mathrm}
\newcommand{\unit}{\mathop{}\!\mathrm}
%\newcommand{\eval}[1]{\bigg[ #1 \bigg]}
\newcommand{\eval}[1]{ #1 \bigg|}
\newcommand{\seq}[1]{\left( #1 \right)}
\renewcommand{\epsilon}{\varepsilon}
\renewcommand{\iff}{\Leftrightarrow}

\DeclareMathOperator{\arccot}{arccot}
\DeclareMathOperator{\arcsec}{arcsec}
\DeclareMathOperator{\arccsc}{arccsc}
\DeclareMathOperator{\si}{Si}
\DeclareMathOperator{\proj}{proj}
\DeclareMathOperator{\scal}{scal}
\DeclareMathOperator{\cis}{cis}
\DeclareMathOperator{\Arg}{Arg}
%\DeclareMathOperator{\arg}{arg}
\DeclareMathOperator{\Rep}{Re}
\DeclareMathOperator{\Imp}{Im}
\DeclareMathOperator{\sech}{sech}
\DeclareMathOperator{\csch}{csch}
\DeclareMathOperator{\Log}{Log}

\newcommand{\tightoverset}[2]{% for arrow vec
  \mathop{#2}\limits^{\vbox to -.5ex{\kern-0.75ex\hbox{$#1$}\vss}}}
\newcommand{\arrowvec}{\overrightarrow}
\renewcommand{\vec}{\mathbf}
\newcommand{\veci}{{\boldsymbol{\hat{\imath}}}}
\newcommand{\vecj}{{\boldsymbol{\hat{\jmath}}}}
\newcommand{\veck}{{\boldsymbol{\hat{k}}}}
\newcommand{\vecl}{\boldsymbol{\l}}
\newcommand{\utan}{\vec{\hat{t}}}
\newcommand{\unormal}{\vec{\hat{n}}}
\newcommand{\ubinormal}{\vec{\hat{b}}}

\newcommand{\dotp}{\bullet}
\newcommand{\cross}{\boldsymbol\times}
\newcommand{\grad}{\boldsymbol\nabla}
\newcommand{\divergence}{\grad\dotp}
\newcommand{\curl}{\grad\cross}
%% Simple horiz vectors
\renewcommand{\vector}[1]{\left\langle #1\right\rangle}


\pgfplotsset{compat=1.13}

\outcome{Learn the arithmetic of complex numbers}

\title{1.1 Complex Arithmetic}

\begin{document}

\begin{abstract}
We perform arithmetic operations on complex numbers.
\end{abstract}

\maketitle

\section{Complex Numbers}

What is a complex number?  The core component of a complex number is the imaginary number $i = \sqrt{-1}$.

\begin{definition}
A {\bf complex number} is a number of the form $a+bi$ where $a,b \in \mathbb{R}$. 
The set of all complex numbers is $\mathbb{C}= \{a+bi \,|\, a,b \in \mathbb{R}\}$. 
The {\bf real part} of $\,a+bi\,$ is `$a$' and the {\bf imaginary part} of $\,a+bi\,$ is `$b$'.
\end{definition}

\begin{remark}
The imaginary part of a complex number is a real number.
\end{remark}

We use the symbols $\Rep$ and $\Imp$ to denote the real and imaginary parts of a complex number:
\[
\Rep(a+bi) = a \qquad \mbox{and} \qquad \Imp(a+bi) = b
\]


\begin{example}[Example 1] 
The following are complex numbers: $2+3i, \; -2i$ and $\sqrt 2$.
Their real and imaginary parts are:
\begin{align*}
(i)& \quad \Rep(2+3i) = 2 \quad \mbox{and} \quad \;\Imp(2+3i) = 3 \\
(ii)& \quad \Rep(-2i) = 0 \quad \;\;\; \mbox{and} \quad \;\Imp(-2i) = -2 \\
(iii)& \quad \Rep\left(\sqrt 2\right) = \sqrt 2 \quad \;\mbox{and} \quad \;\Imp\left(\sqrt 2\right) = 0 
\end{align*}

\end{example}


\begin{remark}
A real number is a complex number, i.e., $\mathbb{R} \subset \mathbb{C}$. 
Moreover, the imaginary part of a real number is $0$.
\end{remark}
\begin{remark}
If the real part of a complex number is $0$, then we say that the complex number is purely imaginary. 
\end{remark}


\begin{problem}(Problem 1a)
Find the real and imaginary parts of the following complex numbers: $-3i,\; \pi$ and $-2+i$.
\begin{align*}
(i) \quad&\Rep(-3i) = \answer{0} \qquad \quad \; \mbox{and} \quad \; \Imp(-3i) = \answer{-3}\\
(ii) \quad&\Rep(\pi) = \answer{\pi} \qquad \qquad \; \mbox{and} \quad \;\Imp(\pi) = \answer{0}\\
(iii) \quad&\Rep(-2+i) = \answer{-2} \qquad \mbox{and} \quad \; \Imp(-2+i) = \answer{1}\\
\end{align*}

\end{problem}



\begin{problem}(Problem 1b)
Label each complex number as either real, purely imaginary or neither:\\
(i) \quad $-3i$ \wordChoice{\choice{real}\choice[correct]{purely imaginary}
                           \choice{neither}} \quad
(ii) \quad $\pi$  \wordChoice{\choice[correct]{real}\choice{purely imaginary}
                           \choice{neither}} \quad
(iii) \quad $-2+i$  \wordChoice{\choice{real}\choice{purely imaginary}
								\choice[correct]{neither}} \quad
(iv) \quad $0$  \wordChoice{\choice{real}\choice{purely imaginary}
                           \choice{neither}\choice[correct]{both!}}						
%\wordChoice{\choice[correct]{Yes}
%\choice{No}}
\end{problem}

Two complex numbers are equal if their real and imaginary parts are both equal. 



\begin{example}[Example 2]
The complex numbers $\frac12 - \frac13 i$ and $0.5 - 0.{\overline 3}i$ are equal since
$\frac12 = 0.5$ and $-\frac13 = -0.{\overline 3}$.
\end{example}

\begin{problem}(Problem 2)
Which of the following complex numbers are equal?

\begin{selectAll}
\choice[correct]{$1 + 0.125i$ \quad and \quad $0.{\overline 9} + \frac18 i$}
\choice[correct]{$\cos\left(\tfrac{\pi}{6}\right) + i\sin\left(\tfrac{\pi}{6}\right)$ \quad and \quad $\frac{\sqrt 3}{2} + \frac12 i$}
\choice{$e + \pi i$ \quad and \quad $2.72 + 3.14i$}
\end{selectAll}

\end{problem}



\section{Addition of Complex Numbers}

Complex numbers are added by adding the real and imaginary parts independently:
\[
(a+bi) + (c+di) = (a+c) + (b+d)i
\]

Complex addition is both commutative and associative. This is a consequence of the fact that the addition of real numbers is both commutative and associative.

\begin{example}[Example 3]
Compute the following sums:
\begin{align*}
(i)& \quad (3+4i) + (5+6i) = 8+10i \\
(ii)& \quad (3+2i) + (3-2i) = 6 \\
(iii)& \quad (4+i) - (4-i) = 2i
\end{align*}

\end{example}


\begin{problem}(Problem 3)
Compute the following sums:
\begin{align*}
(i)& \quad (2+3i) + (4+5i) = \answer{6+8i} \\
(ii)& \quad(5+2i) + (5-2i) = \answer{10}\\
(iii)& \quad(6-3i) - (6+3i) = \answer{-6i}
\end{align*}
\end{problem}

\section{Multiplication of Complex Numbers}
Complex numbers are multiplied in the same way as binomials, using the fact that $i^2 = -1$:
\[
(a+bi)(c+di) = ac + adi + bci + (bi)(di) = (ac-bd) + (ad+bc)i
\]

Complex multiplication is both commutative and associative since the multiplication of real numbers is also both commutative and associative.

\begin{example}[Example 4]
Compute the following products:
\begin{align*}
(i)& \quad (3+4i)  (5+6i) = 15 + 18i + 20i - 24 = -9 + 38i \\
(ii)& \quad (3+2i)  (3-2i) = 9-6i+6i+4=13 \\
(iii)& \quad 5(2i) = 10i
\end{align*}

\end{example}


\begin{problem}(Problem 4)
Compute the following products:
\begin{align*}
(i)& \quad (2+3i)  (4+5i) = \answer{-7+22i}\\
(ii)& \quad (3-i)  (2-3i) = \answer{3-11i}\\
(iii)& \quad (4+i)  (4-i) = \answer{17}\\
(iv)& \quad -4(3i)= \answer{-12i}
\end{align*}
\end{problem}


\section{The Powers of $i$}
An alternative to stating that $i = \sqrt {-1}$ is to say $i^2 = -1$. This leads us to compute the powers of $i$.

\begin{example}[Example 5]
Compute the each of the following powers: $i^3, i^4$ and $i^{75}$.
\begin{align*}
(i)& \quad i^3 = (i^2)(i) = (-1)(i) =-i\\
(ii)& \quad i^4 = (i^2)(i^2) = (-1)(-1) = 1\\
(iii)& \quad i^{75} = (i^{72})(i^3) = (i^4)^{18} (i^3) = i^3 = -i
\end{align*}
Note that in part $(iii)$, a large power of $i$ was written in terms of $i^4$ which is $1$.
In general $i^{4n+k} = i^k$, where $k$ and $n$ are integers.
%Note that in part $(iii), i^{72} =1$ since $i^4 =1$ and $72$ is a multiple of $4$.
\end{example}


\begin{problem}(Problem 5)
Compute the following powers of $i$:
\begin{align*}
(i)& \quad i^5 = \answer{i}\\
(ii)& \quad i^{50} = \answer{-1}\\
(iii)& \quad i^{103}= \answer{-i}
\end{align*}
\end{problem}



\section{The Complex Conjugate}

\begin{definition}[Complex Conjugate]
The {\bf complex conjugate} of a complex number $a+bi$ is defined as
\[
\overline {a+bi} = a-bi
\]
\end{definition}

\begin{example}[Example 6]
Find the complex conjugate:
\begin{align*}
(i)& \quad \overline {2+3i} = 2-3i \\
(ii)& \quad \overline {1-i} = 1+i \\
(iii)& \quad \overline {75} = 75 \\
(iv)& \quad \overline {4i} = -4i
\end{align*}
\end{example}


\begin{problem}(Problem 6)
Find the complex conjugate:
\begin{align*}
(i)& \quad {\overline {3+2i}} = \answer{3-2i} \\
(ii)& \quad {\overline {4-5i}} = \answer{4+5i}\\
(iii)& \quad {\overline {26}}= \answer{26}\\
(iv)& \quad {\overline {-7i}}= \answer{7i}
\end{align*}
\end{problem}


If we multiply a complex number by its conjugate, we get a real number:
\[
(a+bi)  \overline{(a+bi)} = (a+bi)(a-bi) = a^2 - (bi)^2 = a^2 +b^2
\]

\begin{example}[Example 7]
Multiply the complex conjugates:
\begin{align*}
(i)& \quad (2+5i) \overline {(2+5i)} = (2+5i)(2-5i) = 2^2 + 5^2 = 29\\
(ii)& \quad (1-i) \overline {(1-i)} = (1-i)(1+i) = 1^2 + (-1)^2 = 2
\end{align*}
\end{example}

\begin{problem}(Problem 7)
Multiply the complex conjugates:
\begin{align*}
(i)& \quad (6+2i) \overline{(6+2i)} = \answer{40} \\
(ii)& \quad (3-4i) \overline {(3-4i)} = \answer{25}
\end{align*}
\end{problem}


\section{Division of Complex Numbers}
To divide complex numbers, we use the complex conjugate as follows:
\[
\frac{a+bi}{c+di} = \frac{a+bi}{c+di}\cdot\frac{c-di}{c-di} = \frac{(a+bi)(c-di)}{c^2 + d^2} = \left(\frac{ac+bd}{c^2 + d^2}\right) + \left(\frac{bc-ad}{c^2 + d^2}\right)i
\]


\begin{example}[Example 8]
Divide the complex numbers:
\[
\frac{3+2i}{4-5i} 
\]
Begin by multiplying the numerator and denominator by the conjugate of the denominator, 
which is $\overline {4-5i} = 4+5i$.
We have
\begin{align*}
\frac{3+2i}{4-5i} &= \frac{3+2i}{4-5i} \cdot \frac{4+5i}{4+5i} \\[9 pt]
                  &= \frac{(3+2i)(4+5i)}{(4-5i)(4+5i)}\\[9pt]
									&= \frac{(3+2i)(4+5i)}{4^2 +(-5)^2}\\[8pt]
                  &= \frac{2+23i}{41}\\[7pt]
                  &= \frac{2}{41} + \frac{23}{41}i
\end{align*}
\end{example}


\begin{problem}(Problem 8)
Divide the complex numbers:
\begin{align*}
(i)& \quad  \frac{7+3i}{8+5i} = \answer{\frac{71}{89} -\frac{11}{89}i}\\
(ii)& \quad\frac{1+i}{1-i} = \answer{i}\\
(iii)& \quad\frac{4+i}{i} = \answer{1-4i}\\
(iv)& \quad\frac{1}{6+8i} = \answer{\frac{6}{100} - \frac{8}{100}i}
(v)&\quad \frac{1}{i} = \answer{-i}
\end{align*}
\end{problem}


\begin{problem}(Problem 9)
Find the following quantities:
\begin{align*}
(i)& \quad \Rep\left[(4+3i)\overline{(2+i)}\right] = \answer{11}\\
(ii)& \quad \Imp\left[(1+i)^3\right] = \answer{2}\\
(iii)& \quad \overline{\left(\frac{1}{i^7}\right)} = \answer{-i}\\
(iv)& \quad \overline{\overline{(5-2i)}} = \answer{5-2i}
\end{align*}
\end{problem}

Here is a video solution to a problem on dividing complex numbers:

\begin{foldable}
\unfoldable
\youtube{ziIfu42uvok}
\end{foldable}

\end{document}












