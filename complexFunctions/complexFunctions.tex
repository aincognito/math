\documentclass[handout]{ximera}

%% You can put user macros here
%% However, you cannot make new environments



\newcommand{\ffrac}[2]{\frac{\text{\footnotesize $#1$}}{\text{\footnotesize $#2$}}}
\newcommand{\vasymptote}[2][]{
    \draw [densely dashed,#1] ({rel axis cs:0,0} -| {axis cs:#2,0}) -- ({rel axis cs:0,1} -| {axis cs:#2,0});
}


\graphicspath{{./}{firstExample/}}
\usepackage{forest}
\usepackage{amsmath}
\usepackage{amssymb}
\usepackage{array}
\usepackage[makeroom]{cancel} %% for strike outs
\usepackage{pgffor} %% required for integral for loops
\usepackage{tikz}
\usepackage{tikz-cd}
\usepackage{tkz-euclide}
\usetikzlibrary{shapes.multipart}


%\usetkzobj{all}
\tikzstyle geometryDiagrams=[ultra thick,color=blue!50!black]


\usetikzlibrary{arrows}
\tikzset{>=stealth,commutative diagrams/.cd,
  arrow style=tikz,diagrams={>=stealth}} %% cool arrow head
\tikzset{shorten <>/.style={ shorten >=#1, shorten <=#1 } } %% allows shorter vectors

\usetikzlibrary{backgrounds} %% for boxes around graphs
\usetikzlibrary{shapes,positioning}  %% Clouds and stars
\usetikzlibrary{matrix} %% for matrix
\usepgfplotslibrary{polar} %% for polar plots
\usepgfplotslibrary{fillbetween} %% to shade area between curves in TikZ



%\usepackage[width=4.375in, height=7.0in, top=1.0in, papersize={5.5in,8.5in}]{geometry}
%\usepackage[pdftex]{graphicx}
%\usepackage{tipa}
%\usepackage{txfonts}
%\usepackage{textcomp}
%\usepackage{amsthm}
%\usepackage{xy}
%\usepackage{fancyhdr}
%\usepackage{xcolor}
%\usepackage{mathtools} %% for pretty underbrace % Breaks Ximera
%\usepackage{multicol}



\newcommand{\RR}{\mathbb R}
\newcommand{\R}{\mathbb R}
\newcommand{\C}{\mathbb C}
\newcommand{\N}{\mathbb N}
\newcommand{\Z}{\mathbb Z}
\newcommand{\dis}{\displaystyle}
%\renewcommand{\d}{\,d\!}
\renewcommand{\d}{\mathop{}\!d}
\newcommand{\dd}[2][]{\frac{\d #1}{\d #2}}
\newcommand{\pp}[2][]{\frac{\partial #1}{\partial #2}}
\renewcommand{\l}{\ell}
\newcommand{\ddx}{\frac{d}{\d x}}

\newcommand{\zeroOverZero}{\ensuremath{\boldsymbol{\tfrac{0}{0}}}}
\newcommand{\inftyOverInfty}{\ensuremath{\boldsymbol{\tfrac{\infty}{\infty}}}}
\newcommand{\zeroOverInfty}{\ensuremath{\boldsymbol{\tfrac{0}{\infty}}}}
\newcommand{\zeroTimesInfty}{\ensuremath{\small\boldsymbol{0\cdot \infty}}}
\newcommand{\inftyMinusInfty}{\ensuremath{\small\boldsymbol{\infty - \infty}}}
\newcommand{\oneToInfty}{\ensuremath{\boldsymbol{1^\infty}}}
\newcommand{\zeroToZero}{\ensuremath{\boldsymbol{0^0}}}
\newcommand{\inftyToZero}{\ensuremath{\boldsymbol{\infty^0}}}


\newcommand{\numOverZero}{\ensuremath{\boldsymbol{\tfrac{\#}{0}}}}
\newcommand{\dfn}{\textbf}
%\newcommand{\unit}{\,\mathrm}
\newcommand{\unit}{\mathop{}\!\mathrm}
%\newcommand{\eval}[1]{\bigg[ #1 \bigg]}
\newcommand{\eval}[1]{ #1 \bigg|}
\newcommand{\seq}[1]{\left( #1 \right)}
\renewcommand{\epsilon}{\varepsilon}
\renewcommand{\iff}{\Leftrightarrow}

\DeclareMathOperator{\arccot}{arccot}
\DeclareMathOperator{\arcsec}{arcsec}
\DeclareMathOperator{\arccsc}{arccsc}
\DeclareMathOperator{\si}{Si}
\DeclareMathOperator{\proj}{proj}
\DeclareMathOperator{\scal}{scal}
\DeclareMathOperator{\cis}{cis}
\DeclareMathOperator{\Arg}{Arg}
%\DeclareMathOperator{\arg}{arg}
\DeclareMathOperator{\Rep}{Re}
\DeclareMathOperator{\Imp}{Im}
\DeclareMathOperator{\sech}{sech}
\DeclareMathOperator{\csch}{csch}
\DeclareMathOperator{\Log}{Log}

\newcommand{\tightoverset}[2]{% for arrow vec
  \mathop{#2}\limits^{\vbox to -.5ex{\kern-0.75ex\hbox{$#1$}\vss}}}
\newcommand{\arrowvec}{\overrightarrow}
\renewcommand{\vec}{\mathbf}
\newcommand{\veci}{{\boldsymbol{\hat{\imath}}}}
\newcommand{\vecj}{{\boldsymbol{\hat{\jmath}}}}
\newcommand{\veck}{{\boldsymbol{\hat{k}}}}
\newcommand{\vecl}{\boldsymbol{\l}}
\newcommand{\utan}{\vec{\hat{t}}}
\newcommand{\unormal}{\vec{\hat{n}}}
\newcommand{\ubinormal}{\vec{\hat{b}}}

\newcommand{\dotp}{\bullet}
\newcommand{\cross}{\boldsymbol\times}
\newcommand{\grad}{\boldsymbol\nabla}
\newcommand{\divergence}{\grad\dotp}
\newcommand{\curl}{\grad\cross}
%% Simple horiz vectors
\renewcommand{\vector}[1]{\left\langle #1\right\rangle}


\pgfplotsset{compat=1.13}

\outcome{Learn about complex functions}

\title{2.1 Complex Functions}

\begin{document}

\begin{abstract}
We introduce functions of a complex variable.
\end{abstract}

\maketitle

\section{Complex Functions}

A complex function has the form $w = f(z)$ where $z$ and $w$ are complex variables.
Let $u(x,y) = \Rep(f)$ and $v(x,y) = \Imp(f)$. Then we can write 
\[
w = f(z) = u(x,y) + iv(x,y)
\]



\begin{example}[Example 1] 
Let $w = f(z) = z^2$. Find $u(x,y)$ and $v(x,y)$.\\
Since $z = x+iy$, 
\[
z^2 = (x+iy)^2 = \left(x^2 - y^2\right) + i(2xy)
\]
Thus $u(x,y) = x^2 - y^2$ and $v(x,y) = 2xy$.

\end{example}




\begin{problem}(Problem 1a)
Find the real and imaginary parts of the function $w = f(z) = \overline{z}$
\[
u(x,y) = \answer{x} \;\; \mbox{and} \;\; v(x,y) = \answer{-y}
\]

\end{problem}


\begin{problem}(Problem 1b)
Find the real and imaginary parts of the function $w = f(z) = 2iz+i$     
\[
u(x,y) = \answer{-2y} \;\; \mbox{and} \;\; v(x,y) = \answer{2x+1}
\]

\end{problem}


\begin{problem}(Problem 1c)
Find the real and imaginary parts of the function $w = f(z) = z^3$   
\[
u(x,y) = \answer{x^3 - 3xy^2} \;\; \mbox{and} \;\; v(x,y) = \answer{3x^2y -y^3}
\]

\end{problem}

Here is a video solution to problem 1c:\\
\begin{foldable}
\youtube{uIXdl2bXmu8}
\end{foldable}

\begin{problem}(Problem 1d)
Find the real and imaginary parts of the function $w = f(z) = 1/z$   
\[
u(x,y) = \answer{\frac{x}{x^2 + y^2}} \;\; \mbox{and} \;\; v(x,y) = \answer{-\frac{y}{x^2 + y^2}}
\]

\end{problem}

\section{The Complex Exponential}

To define $e^z$ we will use the power series representations of $e^x, \sin(x)$ and $\cos(x)$:
\[
e^x = \sum_{n=0}^\infty \frac{x^n}{n!},
\]
\[
\sin(x) = \sum_{n=0}^\infty (-1)^n \frac{x^{2n+1}}{(2n+1)!},
\]
and
\[
\cos(x) = \sum_{n=0}^\infty (-1)^n \frac{x^{2n}}{(2n)!}.
\]

We would like the complex exponential function to preserve as many properties of the real exponential function as possible. 
In particular, we would like the complex exponential to have the fundamental property $e^{z_1}\cdot e^{z_2} = e^{z_1 + z_2}$.
In particular, we would like
\[
e^z = e^{x+iy} = e^x \cdot e^{iy}
\]
This reduces the challenge to defining $e^{iy}$. To do this we make a ``formal" power series expansion:
\begin{align*}
e^{iy} &= \sum_{n=0}^\infty \frac{(iy)^n}{n!} = \sum_{n=0}^\infty \frac{i^ny^n}{n!}\\
       &= 1+iy -\frac{y^2}{2!} -i\frac{y^3}{3!} + \frac{y^4}{4!} + i\frac{y^5}{5!} - \frac{y^6}{6!} -i \frac{y^7}{7!} + \cdots\\
       &= \sum_{n=0}^\infty (-1)^n\frac{y^{2n}}{(2n)!} + i\sum_{n=0}^\infty (-1)^n\frac{y^{2n+1}}{(2n+1)!}\\
       &= \cos(y) + i\sin(y)\\
       &= \cis(y)
\end{align*}


In light of this calculation, we define the complex exponential function as
\[
e^z \equiv e^x \cis(y) 
\]
where $z = x+iy$.

\begin{example}[Example 2- Euler's Identity]
Consider the function $f(z) = e^z$. Compute $f(\pi i)$.\\
Writing $\pi i$ as $0+\pi i$, we use the definition of $e^z$ with $x = 0 $ and $y= \pi$.
We have 

\[ f(\pi i) = e^{\pi i} = e^0 \cis \pi = \cos(\pi) + i\sin(\pi) = -1 \]

This gives us \link[Euler's identity]{https://en.wikipedia.org/wiki/Euler's_identity},
\[
e^{\pi i} +1 = 0
\]

\end{example}


\begin{problem}(Problem 2)
Calculate each of the following:\\
i) $e^0 = \answer{1}$\\
ii) $e^{\pi i /4} = \answer{1/\sqrt 2 + i/\sqrt 2}$\\
iii) $e^{2\pi i}= \answer{1}$\\
iv) $e^3 = \answer{e^3}$\\
v) $e^{3+2\pi i} = \answer{e^3}$\\
\end{problem} 

\begin{remark}
Since $e^{2\pi i} = 1$, the function $e^z$ is $2\pi i$ periodic.
\end{remark}

%re-write intro sentence to u(x,y) and v(x,y): 
Since $e^z = e^x \cis y$, we have $e^z = u(x,y) + iv(x,y)$ with
\[
u(x,y) = e^x\cos(y) \;\; \mbox{and} \;\; v(x,y)= e^x\sin(y)
\]




\begin{example}[Example 3] 
Let $w = f(z) = e^{2z}$. Find $u(x,y)$ and $v(x,y)$.\\
Since $z = x+iy$, 
\[
e^{2z} = e^{2x+2iy} = e^{2x}\cis(2y)
\]
Thus $u(x,y) = e^{2x}\cos(2y)$ and $v(x,y) = e^{2x}\sin(2y)$.

\end{example}



\begin{problem}(Problem 3a)
Find the real and imaginary parts of the function $w = f(z) = e^{3z}$
\[
u(x,y) = \answer{e^{3x}\cos(3y)} \;\; \mbox{and} \;\; v(x,y) = \answer{e^{3x}\sin(3y)}
\]

\end{problem}

\begin{problem}(Problem 3b)
Find the real and imaginary parts of the function $w = f(z) = e^{-z}$
\[
u(x,y) = \answer{e^{-x}\cos(y)} \;\; \mbox{and} \;\; v(x,y) = \answer{-e^{-x}\sin(y)}
\]

\end{problem}

\begin{problem}(Problem 3c)
Find the real and imaginary parts of the function $w = f(z) = e^{\overline{z}}$
\[
u(x,y) = \answer{e^x\cos(y)} \;\; \mbox{and} \;\; v(x,y) = \answer{-e^x\sin(y)}
\]
\end{problem}

Here is a video solution to problem 3c:\\
\begin{foldable}
\youtube{uI8oS4IuzRA}
\end{foldable}


\begin{example}[Example 4] 
Solve the equation $e^z = -4$ for $z$.\\
Since $e^z = e^x \cis y$ and $-4 = 4\cis \pi$,
we have 
\[
e^x = 4 \;\; \mbox{and} \;\; y = \pi+2k\pi
\]
where $k \in Z$. Thus
\[
x = \ln 4 \;\; \mbox{and} \;\; y = (2k+1)\pi
\]
\end{example}



\begin{problem}(Problem 4a)
Solve for $z: e^z = -3$\\
\begin{hint}
There are infinitely many answers
\end{hint}
\begin{hint}
Use $k$ to represent an arbitrary integer
\end{hint}
$z = \answer{\ln 3 + i(2k\pi + \pi)}$
\end{problem}


\begin{problem}(Problem 4b)
Solve for $z: e^z = i$\\
\begin{hint}
There are infinitely many answers
\end{hint}
\begin{hint}
Use $k$ to represent an arbitrary integer
\end{hint}
$z = \answer{i(2k\pi + \pi/2)}$
\end{problem}


\begin{problem}(Problem 4c)
Solve for $z: e^z = 1+i$\\
\begin{hint}
There are infinitely many answers
\end{hint}
\begin{hint}
Use $k$ to represent an arbitrary integer
\end{hint}
$z = \answer{\ln \sqrt 2 +i(2k\pi + \pi/4)}$
\end{problem}

Here is a video solution to problem 4c:\\
\begin{foldable}
\youtube{j7DSlNngP8I}
\end{foldable}


\section{The Principal Argument Function}

Recall the polar form of a complex number:
\[
z = r\cis\theta
\]
where $r = |z|$ and $\theta$ is an angle co-terminal with the vector from $0$ to $z$. Such an angle is called an {\bf argument}
of the complex number. If $\theta$ is an argument of $z$, then any angle of the form $\theta +2k\pi$ where $k$ is an integer, is also an argument of $z$.

We define the set
\[
\arg(z) = \{\theta + 2k\pi | k \in \mathbb{Z}\}
\]
where $\theta$ is any argument of $z$.
To create an argument function, we must select one member of the set $\arg(z)$ for each $z \in \mathbb{C}$.
The principal argument function, $\Arg(z), z \neq 0$, is defined by
\[
\Arg(z) = \theta, \;\mbox{where}\; \theta\in\arg(z)\;\mbox{and}\; -\pi < \theta \leq \pi
\]

\begin{remark}
$\Arg(0)$ is not defined.
\end{remark}

\begin{example}[Example 5] 
Let $w = f(z) = \Arg(z)$. Find $u(x,y)$ and $v(x,y)$.\\ 
First, since $\Arg(z)$ is a real number, we have $u(x,y) = Arg(z)$ and $v(x,y) = 0$.
To find $u(x,y)$ we use the inverse tangent function.
If $z=x+iy$ is in the first or fourth quadrant, then 
\[
\Arg(z) = \tan^{-1}\left(\frac{y}{x}\right).
\]
If $z$ is in the second quadrant, then 
\[
\Arg(z) = \tan^{-1}\left(\frac{y}{x}\right) + \pi.
\]
If $z$ is in the third quadrant, then 
\[
\Arg(z) = \tan^{-1}\left(\frac{y}{x}\right) - \pi.
\]
If $z =x$ is a real number, then $Arg(z) = 0$ if $x>0$ and  $Arg(z) = \pi$ if $x<0$.\\
If $z = iy$ is a purely imaginary number, then $Arg(z) = \pi/2$ if $y>0$ and $Arg(z) = -\pi/2$ if $y<0$.
\end{example}


\begin{problem}(Problem 5a)
Find the following principal arguments:\\
$i) \Arg(\pi) = \answer{0}$\\
$ii) \Arg(-2\pi) = \answer{\pi}$\\
$iii) \Arg(\pi i) = \answer{\pi/2}$\\
$iv) \Arg(-i) = \answer{-\pi/2}$\\
$v) \Arg(1-i) = \answer{-\pi/4}$\\
$vi) \Arg(-1+i\sqrt 3) = \answer{2\pi/3}$
\end{problem}

\begin{problem}(Problem 5b)
Recall that a geometric property of complex multiplication is the addition of angles:
\[
z_1 \cdot z_2 = r_1 \cis\theta_1 \cdot r_2 \cis\theta_2 = r_1r_2\cis\left(\theta_1+\theta_2\right)
\]
Now, let's consider $\Arg(z_1 \cdot z_2)$ and ruminate on the following question.
Does the equation 
\[
\Arg(z_1\cdot z_2) = \Arg(z_1) + \Arg(z_2)
\]
hold for all non-zero complex numbers $z_1$ and $z_2$?
\begin{multipleChoice}
\choice{Yes}
\choice[correct]{No}
\end{multipleChoice}
Which of the following pairs of complex numbers can be used as a counter-example?
\begin{selectAll}
\choice{$1+i$ and $i-1$}
\choice{$1+i$ and $1-i$}
\choice[correct]{$1+i$ and $-1$}
\choice[correct]{$-i$ and $-1-i$}
\end{selectAll}

\end{problem}



\section{Complex Roots}

The Principal Argument function plays a role in creating a Principal $n^{th}$ root function.
We begin with the square root. Just as there are two square roots of a positive real number, there are two square roots of a non-zero complex number.
For the real square root, we distinguish these roots as positive and negative. However, this concept does not apply to non-real 
complex numbers. To find roots, we use the polar form $z = r\cis\theta = |z|e^{i\theta}$.

\begin{example}[Example 6] 
Find the two square roots of $4i$. \\
The modulus of $4i$ is $4$ and the Principal Argument is $\pi/2$, so in polar form
\[
4i = 4\cis(\pi/2) = 4e^{\pi i/2}
\]
Since squaring a complex number squares the modulus and doubles the argument, 
to find a square root, we should take the (positive) square root of the modulus and halve the argument.
Thus
\[
\sqrt{4i} = 2\cis(\pi/4) = 2e^{\pi i/4} = \sqrt 2 + i\sqrt 2
\]
The other square root is the negative of this one, but we will compute it directly using a different argument for $4i$.
The idea of this method generalizes to $n^{th}$ roots.
We can also write $4i$ in polar form as $4\cis(5\pi/2)$.  Thus the other square root is
\[
\sqrt{4i} = 2\cis(5\pi/4) = -\sqrt 2 - i\sqrt 2
\]

\end{example}


\begin{definition}[Principal Square Root]
If $z \neq 0$, the function
\[
f(z) = z^{1/2} =  |z|^{1/2} \cis\left(\frac{\Arg z}{2}\right) = |z|^{1/2} e^{i\left(\frac{\Arg z}{2}\right)}
\]
is called the Principal Square Root function.
\end{definition}


\begin{remark} In the preceding example, since $\Arg 4i = \pi/2$, the Principal Square Root is
\[
(4i)^{\frac12} = 2\cis(\pi/4) = 2e^{\pi i /4} = \sqrt 2 + i\sqrt 2
\] 
\end{remark}


\begin{problem}(Problem 6)
Find the Principal Square Root of the complex number. Write your answer in the form $a + bi$.\\
i) $(2i)^{\frac12} = \answer{1+i}$\\
ii) $(-144)^{\frac12} = \answer{12i}$\\
iii) $(2 + 2i\sqrt 3)^{\frac12} = \answer{\sqrt 3 + i}$\\
%4 cis 60 ^1/2 = 2 cis 30
iv) $(-18+ 18i\sqrt 3)^{\frac12} = \answer{3 + 3i\sqrt 3}$\\
% 36 cis 120 ^1/2 = 6 cis 60
\end{problem}


The Principal $n^{th}$ Root function is defined by taking the positive $n^{th}$ root of the modulus and 
$(1/n)^{th}$ of the Principal Argument.

\begin{definition}[Principal $n^{th}$ Root function]
If $n$ is a positive integer and $z \neq 0$, the function
\[
f(z) = z^{\frac1n} =  |z|^{\frac1n} \cis\left(\frac{\Arg z}{n}\right) = |z|^{\frac1n} e^{i\frac{\Arg(z)}{n}}
\]
is called the Principal $n^{th}$ Root function.
\end{definition}


\begin{example}[Example 7]
Find the Principal Cube Root of $-8i$ and then find all of the other cube roots. Include them in a sketch.\\
Since $|-8i| = 8$ and $\Arg(-8i) = -\pi/2$, we have
\[
(-8i)^{\frac13} = 8^{\frac13} \cis(-\pi/6) = \sqrt 3 - i
\]
To find the other cube roots, we add multiples of $2\pi$ to the argument before dividing by $3$.
This gives
\begin{align*}
\sqrt[3]{-8i} &= 2 \cis\left[\frac13\left(\frac{-\pi}{2} + 2k\pi\right)\right] \\
              &= 2 \cis\left(-\frac{\pi}{6} + \frac{2k\pi}{3}\right)
\end{align*}
     This gives distinct values if $k = 0, 1$ or $2$. If $k=0$ we get the Principal Cube Root.
     If $k = 1$ we get
\[
\sqrt[3]{-8i}= 2 \cis\left(\frac{\pi}{2}\right)  = 2i 
\]
and if $k = 2$, we get
\[
\sqrt[3]{-8i}= 2 \cis\left(\frac{7\pi}{6}\right) = -\sqrt 3 - i 
 \]    
\begin{image}
\begin{tikzpicture}
\draw [blue, <->](-2.5, 0) -- (2.5,0);
\draw[blue, <->] (0,2.5) -- (0, -2.5) node[ below=10pt, blue]{\large The Cube Roots of $-8i$  };
%\node @(0,-2.5){The three cube roots of $-8i$}
\draw[mark=*,mark size=1pt,mark options={color=blue}] plot coordinates {(0,2)} node[right, blue]{$2i$};
\draw[mark=*,mark size=1pt,mark options={color=red}] plot coordinates {(1.732,-1)} node[below right, red]{$\sqrt 3 - i $};
\draw[mark=*,mark size=1pt,mark options={color=blue}] plot coordinates {(-1.732,-1)} node[below left, blue]{$-\sqrt 3 - i $};
\draw [red, dashed](0, 0) -- (1.732, -1);
\draw [blue, dashed](0, 0) -- (-1.732, -1);
\draw [blue, dashed, thick](0, 0) -- (0,2);
%\node (test) [rectangle, draw] {this node \\ has \\ four \\lines};(Principal Cube Root in red)
\end{tikzpicture}
\end{image}

\end{example}


\begin{problem}(problem 7a)
Find and sketch the 3 cube roots of each of the following:\\
\begin{align*}
i) & \;1 \; \mbox{(these are called roots of unity)} \\
ii)& \;27 \\
iii) & \;-8 \\
iv) & \;i
\end{align*}
\end{problem}

\begin{problem}(problem 7b)
Find and sketch the 4 fourth roots of each of the following:\\
\begin{align*}
i) &\; 1 \; \mbox{(these are called roots of unity)} \\
ii)&\; 16 \\
iii) &\; -81 \\
iv) & \;i
\end{align*}
\end{problem}

\begin{problem}(problem 7c)
Find and sketch the 5 fifth roots of each of the following:\\
\begin{align*}
i) & \;1 \; \mbox{(these are called roots of unity)} \\
ii)& \;-32 \\
iii) & \;i
\end{align*}
\end{problem}

Here is a video solution to problem 7c, part iii:\\
\begin{foldable}
\youtube{F6u_niiyexs}
\end{foldable}

\end{document}

 












