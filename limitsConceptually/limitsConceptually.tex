\documentclass[handout]{ximera}

%% You can put user macros here
%% However, you cannot make new environments



\newcommand{\ffrac}[2]{\frac{\text{\footnotesize $#1$}}{\text{\footnotesize $#2$}}}
\newcommand{\vasymptote}[2][]{
    \draw [densely dashed,#1] ({rel axis cs:0,0} -| {axis cs:#2,0}) -- ({rel axis cs:0,1} -| {axis cs:#2,0});
}


\graphicspath{{./}{firstExample/}}
\usepackage{forest}
\usepackage{amsmath}
\usepackage{amssymb}
\usepackage{array}
\usepackage[makeroom]{cancel} %% for strike outs
\usepackage{pgffor} %% required for integral for loops
\usepackage{tikz}
\usepackage{tikz-cd}
\usepackage{tkz-euclide}
\usetikzlibrary{shapes.multipart}


%\usetkzobj{all}
\tikzstyle geometryDiagrams=[ultra thick,color=blue!50!black]


\usetikzlibrary{arrows}
\tikzset{>=stealth,commutative diagrams/.cd,
  arrow style=tikz,diagrams={>=stealth}} %% cool arrow head
\tikzset{shorten <>/.style={ shorten >=#1, shorten <=#1 } } %% allows shorter vectors

\usetikzlibrary{backgrounds} %% for boxes around graphs
\usetikzlibrary{shapes,positioning}  %% Clouds and stars
\usetikzlibrary{matrix} %% for matrix
\usepgfplotslibrary{polar} %% for polar plots
\usepgfplotslibrary{fillbetween} %% to shade area between curves in TikZ



%\usepackage[width=4.375in, height=7.0in, top=1.0in, papersize={5.5in,8.5in}]{geometry}
%\usepackage[pdftex]{graphicx}
%\usepackage{tipa}
%\usepackage{txfonts}
%\usepackage{textcomp}
%\usepackage{amsthm}
%\usepackage{xy}
%\usepackage{fancyhdr}
%\usepackage{xcolor}
%\usepackage{mathtools} %% for pretty underbrace % Breaks Ximera
%\usepackage{multicol}



\newcommand{\RR}{\mathbb R}
\newcommand{\R}{\mathbb R}
\newcommand{\C}{\mathbb C}
\newcommand{\N}{\mathbb N}
\newcommand{\Z}{\mathbb Z}
\newcommand{\dis}{\displaystyle}
%\renewcommand{\d}{\,d\!}
\renewcommand{\d}{\mathop{}\!d}
\newcommand{\dd}[2][]{\frac{\d #1}{\d #2}}
\newcommand{\pp}[2][]{\frac{\partial #1}{\partial #2}}
\renewcommand{\l}{\ell}
\newcommand{\ddx}{\frac{d}{\d x}}

\newcommand{\zeroOverZero}{\ensuremath{\boldsymbol{\tfrac{0}{0}}}}
\newcommand{\inftyOverInfty}{\ensuremath{\boldsymbol{\tfrac{\infty}{\infty}}}}
\newcommand{\zeroOverInfty}{\ensuremath{\boldsymbol{\tfrac{0}{\infty}}}}
\newcommand{\zeroTimesInfty}{\ensuremath{\small\boldsymbol{0\cdot \infty}}}
\newcommand{\inftyMinusInfty}{\ensuremath{\small\boldsymbol{\infty - \infty}}}
\newcommand{\oneToInfty}{\ensuremath{\boldsymbol{1^\infty}}}
\newcommand{\zeroToZero}{\ensuremath{\boldsymbol{0^0}}}
\newcommand{\inftyToZero}{\ensuremath{\boldsymbol{\infty^0}}}


\newcommand{\numOverZero}{\ensuremath{\boldsymbol{\tfrac{\#}{0}}}}
\newcommand{\dfn}{\textbf}
%\newcommand{\unit}{\,\mathrm}
\newcommand{\unit}{\mathop{}\!\mathrm}
%\newcommand{\eval}[1]{\bigg[ #1 \bigg]}
\newcommand{\eval}[1]{ #1 \bigg|}
\newcommand{\seq}[1]{\left( #1 \right)}
\renewcommand{\epsilon}{\varepsilon}
\renewcommand{\iff}{\Leftrightarrow}

\DeclareMathOperator{\arccot}{arccot}
\DeclareMathOperator{\arcsec}{arcsec}
\DeclareMathOperator{\arccsc}{arccsc}
\DeclareMathOperator{\si}{Si}
\DeclareMathOperator{\proj}{proj}
\DeclareMathOperator{\scal}{scal}
\DeclareMathOperator{\cis}{cis}
\DeclareMathOperator{\Arg}{Arg}
%\DeclareMathOperator{\arg}{arg}
\DeclareMathOperator{\Rep}{Re}
\DeclareMathOperator{\Imp}{Im}
\DeclareMathOperator{\sech}{sech}
\DeclareMathOperator{\csch}{csch}
\DeclareMathOperator{\Log}{Log}

\newcommand{\tightoverset}[2]{% for arrow vec
  \mathop{#2}\limits^{\vbox to -.5ex{\kern-0.75ex\hbox{$#1$}\vss}}}
\newcommand{\arrowvec}{\overrightarrow}
\renewcommand{\vec}{\mathbf}
\newcommand{\veci}{{\boldsymbol{\hat{\imath}}}}
\newcommand{\vecj}{{\boldsymbol{\hat{\jmath}}}}
\newcommand{\veck}{{\boldsymbol{\hat{k}}}}
\newcommand{\vecl}{\boldsymbol{\l}}
\newcommand{\utan}{\vec{\hat{t}}}
\newcommand{\unormal}{\vec{\hat{n}}}
\newcommand{\ubinormal}{\vec{\hat{b}}}

\newcommand{\dotp}{\bullet}
\newcommand{\cross}{\boldsymbol\times}
\newcommand{\grad}{\boldsymbol\nabla}
\newcommand{\divergence}{\grad\dotp}
\newcommand{\curl}{\grad\cross}
%% Simple horiz vectors
\renewcommand{\vector}[1]{\left\langle #1\right\rangle}


\outcome{Describe and denote limits}

\title{1.2 Understanding Limits}


\begin{document}

\begin{abstract}
Describe what a limit is and how to denote a limit.
\end{abstract}

\maketitle

\section{Understanding Limits}
 


Limits are the backbone of calculus. A limit tells us the end of an infinite process. For example, consider the following infinite sequence of numbers:
\[ 0.9, 0.99, 0.999, 0.9999, 0.99999, 0.999999, ... \]
This infinite sequence of numbers is becoming arbitrarily close to the number 1, so we say the \textbf{limit} of the sequence is 1.
In calculus, we will be concerned with limits involving functions.
%A function is an input-output device involving numbers. 
The inputs of the function will undergo an infinite process which will then correspond to an infinite process for the outputs.  Our goal will be to determine the limit of the outputs.
Suppose that $f$ represents a function of the input variable $x$.
We denote by $x \to c$ an infinite process where the inputs are becoming arbitrarily close to the value $c$ without ever reaching it.
Then, we denote by
\[ \lim_{x\to c} f(x) \]
the limit of the outputs of the function $f$ as $x$ approaches $c$.
The possiblilities for the limit of the outputs of $f$ are: a numerical value, 
$\infty$, $-\infty$, or that the limit does not exist.
If the numerical value of the limit is $L$, then as the input variable $x$ is approaches the value $c$, the outputs, $f(x)$ approach the value $L$, and we would write
\[ 
\lim_{x\to c} f(x) = L.
\]

If the value of the limit is $\infty$, then as $x \to c$, the outputs $f(x)$ are \textbf{increasing without bound}.  
Similarly, if the limit is $-\infty$, that means the outputs are decreasing without bound.
When none of these conclusions apply, we say that the limit \textbf{does not exist}. 

In addition to writing $x \to c$, we can also write the expression
$x \to c^-$ to indicate that $x$ is approaching $c$ from the left hand side and $x \to c^+$ to 
indicate that $x$ is approaching $c$ from the lright hand side.

The expression $x \to c^-$ can be understood visually using the following diagram:




\begin{center}
\begin{tikzpicture}
\draw[<->, thick] (-3,0) -- (3,0);
\draw[thick] (-1,-.2) -- (-1,.2); %tick mark
\draw[thick] (1,-.2) -- (1,.2);
\node[below] at (-1,-.3) {$x$};
\node[below] at (1,-.3) {$c$};
\draw[->, thick] (-0.7,0.5) -- (0.7,0.5);
\draw[white](0, .6) -- (.1, .6);
\node[below] at (0,-0.8) {$x$ approaches $c$ from the left: $x \to c^-$};


\end{tikzpicture}

\end{center}

Note that $x \to c^-$ implies $x < c$.
Similarly, the expression $x \to c^+$ can be understood visually using the following diagram:



\begin{center}
\begin{tikzpicture}
\draw[<->, thick] (-3,0) -- (3,0);
\draw[thick] (-1,-.2) -- (-1,.2); %tick mark
\draw[thick] (1,-.2) -- (1,.2);
\node[below] at (-1,-.3) {$c$};
\node[below] at (1,-.3) {$x$};
\draw[<-, thick] (-0.7,0.5) -- (0.7,0.5);
\node[below] at (0,-0.8) {$x$ approaches $c$ from the right: $x \to c^+$};
\draw[white](0, .7) -- (.1, .7);
\end{tikzpicture}
\end{center}

Note that $x \to c^+$ implies $x > c$. We refer to the limits

\[\lim_{x \to c^-} f(x)  \quad \text{and} \quad \lim_{x \to c^+} f(x) \]

as \textbf{one-sided limits}, and we refer to 

\[\lim_{x \to c} f(x) \]

as a \textbf{two-sided limit}.


We can also let the inputs, $x$, either increase or decrease without bound, denoted by $x \to \infty$ and $x \to -\infty$ respectively.
These can be represented visually by the following diagrams:



\begin{center}
\begin{tikzpicture}[scale =0.7]
\draw[<->, thick] (-3,0) -- (3,0);
\draw[thick] (0,-.2) -- (0,.2); %tick mark
\node[below] at (0,-.3) {$x$};
\draw[->, thick] (0.3,0.5) -- (1.7,0.5);
\node[below] at (0,-1) {$x$ increases without bound: $x \to \infty$};
\draw[<->, thick] (-3,-3.5) -- (3,-3.5);
\draw[thick] (0,-3.3) -- (0,-3.7); %tick mark
\node[below] at (0,-3.7) {$x$};
\draw[<-, thick] (-1.7,-3) -- (-0.3,-3);
\node[below] at (0,-4.5) {$x$ decreases without bound: $x \to -\infty$};
\draw[white](0, .7) -- (.1, .7);
\end{tikzpicture}
\end{center}

We refer to the limits
\[\lim_{x \to \infty} f(x)  \quad \text{and} \quad \lim_{x \to -\infty} f(x) \]
as \textbf{limits at infinity}.



\end{document}




