\documentclass[handout]{ximera}

%% You can put user macros here
%% However, you cannot make new environments



\newcommand{\ffrac}[2]{\frac{\text{\footnotesize $#1$}}{\text{\footnotesize $#2$}}}
\newcommand{\vasymptote}[2][]{
    \draw [densely dashed,#1] ({rel axis cs:0,0} -| {axis cs:#2,0}) -- ({rel axis cs:0,1} -| {axis cs:#2,0});
}


\graphicspath{{./}{firstExample/}}
\usepackage{forest}
\usepackage{amsmath}
\usepackage{amssymb}
\usepackage{array}
\usepackage[makeroom]{cancel} %% for strike outs
\usepackage{pgffor} %% required for integral for loops
\usepackage{tikz}
\usepackage{tikz-cd}
\usepackage{tkz-euclide}
\usetikzlibrary{shapes.multipart}


%\usetkzobj{all}
\tikzstyle geometryDiagrams=[ultra thick,color=blue!50!black]


\usetikzlibrary{arrows}
\tikzset{>=stealth,commutative diagrams/.cd,
  arrow style=tikz,diagrams={>=stealth}} %% cool arrow head
\tikzset{shorten <>/.style={ shorten >=#1, shorten <=#1 } } %% allows shorter vectors

\usetikzlibrary{backgrounds} %% for boxes around graphs
\usetikzlibrary{shapes,positioning}  %% Clouds and stars
\usetikzlibrary{matrix} %% for matrix
\usepgfplotslibrary{polar} %% for polar plots
\usepgfplotslibrary{fillbetween} %% to shade area between curves in TikZ



%\usepackage[width=4.375in, height=7.0in, top=1.0in, papersize={5.5in,8.5in}]{geometry}
%\usepackage[pdftex]{graphicx}
%\usepackage{tipa}
%\usepackage{txfonts}
%\usepackage{textcomp}
%\usepackage{amsthm}
%\usepackage{xy}
%\usepackage{fancyhdr}
%\usepackage{xcolor}
%\usepackage{mathtools} %% for pretty underbrace % Breaks Ximera
%\usepackage{multicol}



\newcommand{\RR}{\mathbb R}
\newcommand{\R}{\mathbb R}
\newcommand{\C}{\mathbb C}
\newcommand{\N}{\mathbb N}
\newcommand{\Z}{\mathbb Z}
\newcommand{\dis}{\displaystyle}
%\renewcommand{\d}{\,d\!}
\renewcommand{\d}{\mathop{}\!d}
\newcommand{\dd}[2][]{\frac{\d #1}{\d #2}}
\newcommand{\pp}[2][]{\frac{\partial #1}{\partial #2}}
\renewcommand{\l}{\ell}
\newcommand{\ddx}{\frac{d}{\d x}}

\newcommand{\zeroOverZero}{\ensuremath{\boldsymbol{\tfrac{0}{0}}}}
\newcommand{\inftyOverInfty}{\ensuremath{\boldsymbol{\tfrac{\infty}{\infty}}}}
\newcommand{\zeroOverInfty}{\ensuremath{\boldsymbol{\tfrac{0}{\infty}}}}
\newcommand{\zeroTimesInfty}{\ensuremath{\small\boldsymbol{0\cdot \infty}}}
\newcommand{\inftyMinusInfty}{\ensuremath{\small\boldsymbol{\infty - \infty}}}
\newcommand{\oneToInfty}{\ensuremath{\boldsymbol{1^\infty}}}
\newcommand{\zeroToZero}{\ensuremath{\boldsymbol{0^0}}}
\newcommand{\inftyToZero}{\ensuremath{\boldsymbol{\infty^0}}}


\newcommand{\numOverZero}{\ensuremath{\boldsymbol{\tfrac{\#}{0}}}}
\newcommand{\dfn}{\textbf}
%\newcommand{\unit}{\,\mathrm}
\newcommand{\unit}{\mathop{}\!\mathrm}
%\newcommand{\eval}[1]{\bigg[ #1 \bigg]}
\newcommand{\eval}[1]{ #1 \bigg|}
\newcommand{\seq}[1]{\left( #1 \right)}
\renewcommand{\epsilon}{\varepsilon}
\renewcommand{\iff}{\Leftrightarrow}

\DeclareMathOperator{\arccot}{arccot}
\DeclareMathOperator{\arcsec}{arcsec}
\DeclareMathOperator{\arccsc}{arccsc}
\DeclareMathOperator{\si}{Si}
\DeclareMathOperator{\proj}{proj}
\DeclareMathOperator{\scal}{scal}
\DeclareMathOperator{\cis}{cis}
\DeclareMathOperator{\Arg}{Arg}
%\DeclareMathOperator{\arg}{arg}
\DeclareMathOperator{\Rep}{Re}
\DeclareMathOperator{\Imp}{Im}
\DeclareMathOperator{\sech}{sech}
\DeclareMathOperator{\csch}{csch}
\DeclareMathOperator{\Log}{Log}

\newcommand{\tightoverset}[2]{% for arrow vec
  \mathop{#2}\limits^{\vbox to -.5ex{\kern-0.75ex\hbox{$#1$}\vss}}}
\newcommand{\arrowvec}{\overrightarrow}
\renewcommand{\vec}{\mathbf}
\newcommand{\veci}{{\boldsymbol{\hat{\imath}}}}
\newcommand{\vecj}{{\boldsymbol{\hat{\jmath}}}}
\newcommand{\veck}{{\boldsymbol{\hat{k}}}}
\newcommand{\vecl}{\boldsymbol{\l}}
\newcommand{\utan}{\vec{\hat{t}}}
\newcommand{\unormal}{\vec{\hat{n}}}
\newcommand{\ubinormal}{\vec{\hat{b}}}

\newcommand{\dotp}{\bullet}
\newcommand{\cross}{\boldsymbol\times}
\newcommand{\grad}{\boldsymbol\nabla}
\newcommand{\divergence}{\grad\dotp}
\newcommand{\curl}{\grad\cross}
%% Simple horiz vectors
\renewcommand{\vector}[1]{\left\langle #1\right\rangle}


\outcome{Compute definite integrals involving infinity}

\title{1.6 Improper Integrals}

\begin{document}

\begin{abstract}
We will compute definite integrals involving infinity.
\end{abstract}

\maketitle

\section{Introduction}
In this section we will consider definite integrals involving infinity.
There are two fundamentally different types of improper integrals.  The first involves infinity as an ``endpoint"
of integration and the second involves vertical asymptotes at or between the endpoints of integration.
Examples of the first type include
\[
\int_0^\infty e^{-2x} \; dx \; \; \text{ and } \;  \; \int_{-\infty}^\infty \frac{1}{1+x^2}\; dx.
\]
The second of the above examples is called ``doubly improper" since it involves both $\infty$ and $-\infty$.
Examples of the second type include
\[
\int_0^{\pi/2} \tan(x)\; dx \; \; \text{ and } \; \; \int_0^1 \frac{1}{x} \; dx.
\]
If the value of an improper integral is a number, then we say that the integral \textbf{converges}.
Otherwise, we say that the integral \textbf{diverges}.
The method of computing improper integrals involves replacing an endpoint with a variable 
and then taking an appropriate limit.




\section{Infinity as an endpoint}
\begin{definition}[Type 1 Improper Integral]
Suppose $f(x)$ is continuous on the interval $[a, \infty)$. Then we compute the improper integral at infinity as follows:
\[
\int_a^\infty f(x) \; dx = \lim_{b \to \infty} \int_a^b f(x) \; dx.
\]
The improper integral at negative infinity is defined similarly.
\end{definition}


\begin{example}[example 1]
Compute the improper integral
\[
\int_0^\infty e^{-2x} \; dx.
\]

We will begin by replacing $\infty$ in the improper integral with the variable $b$, a typical choice for an upper endpoint of integration.
We compute
\begin{align*}
\int_0^b e^{-2x} \; dx &= \left(-\frac12 e^{-2x} \right) \bigg|_0^b\\
                       &= \left(-\frac12 e^{-2b} \right) - \left(-\frac12 e^{0} \right)\\
                       &=\frac12 -\frac12 e^{-2b}.
\end{align*}
                       
To complete the problem, we now take a limit as $b \to \infty$.
\begin{align*}
\int_0^\infty e^{-2x} \; dx &= \lim_{b \to \infty}\int_0^b e^{-2x} \; dx\\
                       &=\lim_{b \to \infty} \left(\frac12 -\frac12 e^{-2b}\right)\\
                       &= \frac12 -\frac12 e^{-\infty}\\
                       &= \frac12 - 0 \\
                       &= \frac12.
\end{align*}
Since the limit is a number, we say that the integral converges.
More precisely, we can say that the integral converges to $1/2$.

\end{example}




\begin{problem}(problem 1a)
Compute the improper integral
\[
\int_0^\infty 2e^{-5x} \; dx = \answer{2/5}
\]
\begin{multipleChoice}
\choice[correct]{converge}
\choice{diverge}
\end{multipleChoice}
\end{problem}


\begin{problem}(problem 1b)
Compute the improper integral
\[
\int_{-\infty}^1 3e^{x/2} \; dx = \answer{6e^{1/2}}
\]
\begin{multipleChoice}
\choice[correct]{converge}
\choice{diverge}
\end{multipleChoice}
\end{problem}



\begin{example}[example 2]
Compute the improper integral
\[
\int_1^\infty \frac{3}{x^2} \; dx.
\]

We will begin by replacing $\infty$ in the improper integral with the variable $b$, a typical choice for an upper endpoint of integration.
We compute
\begin{align*}
\int_1^b \frac{3}{x^2} \; dx &= \int_1^b 3x^{-2} \; dx\\
                           &= \left(3\cdot \frac{x^{-1}}{-1} \right) \bigg|_1^b\\
                           &= \left(-\frac{3}{x} \right) \bigg|_1^b\\
                       &= \left(-\frac{3}{b} \right) - \left(-\frac{3}{1} \right)\\
                       &=3 - \frac{3}{b}.
\end{align*}
                       
To complete the problem, we now take a limit as $b \to \infty$.
\begin{align*}
\int_1^\infty \frac{3}{x^2} \; dx &= \lim_{b \to \infty}\int_1^b \frac{3}{x^2} \; dx\\
                       &=\lim_{b \to \infty} \left(3 - \frac{3}{b}\right)\\
                       &= 3 -\frac{3}{\infty}\\
                       &= 3 - 0 \\
                       &= 3.
\end{align*}
Since the limit is a number, we say that the integral converges.
More precisely, we can say that the integral converges to $3$.

\end{example}



\begin{problem}(problem 2a)
Compute the improper integral
\[
\int_2^\infty \frac{5}{x^3} \; dx = \answer{5/8} 
\]
\begin{multipleChoice}
\choice[correct]{converge}
\choice{diverge}
\end{multipleChoice}
\end{problem}


\begin{problem}(problem 2b)
Compute the improper integral
\[
\int_1^\infty \frac{5}{(3x+1)^2} \; dx = \answer{5/12}
\]
\begin{multipleChoice}
\choice[correct]{converge}
\choice{diverge}
\end{multipleChoice}
\end{problem}




\begin{example}[example 3]
Compute the improper integral
\[
\int_1^\infty \frac{4}{x} \; dx.
\]

We will begin by replacing $\infty$ in the improper integral with the variable $b$, a typical choice for an upper endpoint of integration.
We compute
\begin{align*}
\int_1^b \frac{4}{x} \; dx &= \int_1^b \frac{4}{x} \; dx\\
                           &= \left(4\ln|x| \right) \bigg|_1^b\\
                       &= \left(4\ln|b| \right) - \left(4\ln|1| \right)\\
                       &=4\ln|b|.
\end{align*}
                       
To complete the problem, we now take a limit as $b \to \infty$.
\begin{align*}
\int_1^\infty \frac{4}{x} \; dx &= \lim_{b \to \infty}\int_1^b \frac{4}{x} \; dx\\
                       &=\lim_{b \to \infty} 4\ln|b|\\
                       &= 4\ln(\infty)\\
                       &= \infty.
\end{align*}
Since the answer is a \textbf{not} a finite number, we say that the integral diverges.

\end{example}


\begin{problem}(problem 3)
Compute the improper integral
\[
\int_3^\infty \frac{7}{\sqrt x} = \answer{\infty}.
\]
Does the improper integral converge or diverge?
\begin{multipleChoice}
\choice{converge}
\choice[correct]{diverge}
\end{multipleChoice}
\end{problem}

Based on the previous examples and problems, we can expect the following theorem about improper integrals of the form
\[
\int_1^\infty \frac{1}{x^p} \; dx
\]

\begin{theorem}[$p$-integrals]
The improper $p$-integral $\int_1^\infty \frac{1}{x^p} \; dx$
 converges if $p>1$ and diverges if $p \leq 1$
The result is the same if lower endpoint is replaced by any other positive number.
\end{theorem}
This theorem will be of value to us in future sections, but for now, we use it to help us with another improper integral.

\begin{example}[example 4]
Use the theory of $p$-integrals to determine if the integral converges or diverges
\[
\int_2^\infty \frac{2}{\sqrt x} \; dx
\]
This is a (multiple of a) $p$-integral with $p = 1/2$. Since $1/2 \leq 1$, this integral diverges.
\end{example}




\begin{problem}(problem 4a)
Use the theory of $p$-integrals to determine if the integral converges or diverges
\[
\int_4^\infty \frac{3}{\sqrt[3] x} \; dx
\]
This is a (multiple of a) $p$-integral with $p = \answer{1/3}$\\
Does the improper integral converge or diverge?
\begin{multipleChoice}
\choice{converge}
\choice[correct]{diverge}
\end{multipleChoice}
\end{problem}





\begin{problem}(problem 4b)
Use the theory of $p$-integrals to determine if the integral converges or diverges
\[
\int_{0.3}^\infty 2x^{-5} \; dx
\]
This is a (multiple of a) $p$-integral with $p = \answer{5}$\\
Does the improper integral converge or diverge?
\begin{multipleChoice}
\choice[correct]{converge}
\choice{diverge}
\end{multipleChoice}
\end{problem}



\begin{example}[example 5]
Use the theory of $p$-integrals to determine whether the improper integral convergs or diverges
\[
\int_2^\infty \frac{1}{x\ln(x)} \; dx.
\]
Using a $u$-substitution with $u = \ln(x)$ and $du = \frac{1}{x} \, dx$, the integral becomes
\[
\int_2^\infty \frac{1}{x\ln(x)} \; dx = \int_{\ln(2)}^\infty \frac{1}{u} \; du.
\]
This is a $p$-integral in the variable $u$ with $p=1$. Since $p \leq 1$, the theorem says that this integral diverges, and hence
\[
\int_2^\infty \frac{1}{x\ln(x)} \; dx \;\; \text{ diverges}.
\]
\end{example}


\begin{problem}(problem 5a)
Use the theory of $p$-integrals to determine if the improper integral converges or diverges:
\[
\int_3^\infty \frac{1}{x\ln^2(x)} \; dx
\]

Make a substitution\\
$u = \answer{\ln(x)} \;\; du = \answer{1/x} \, dx$\\
The endpoints become\\
$x = 3 \implies u = \answer{\ln(3)}$ and $x= \infty \implies u = \answer{\infty}$\\
In terms of $u$, the integral is
\[
\int_3^\infty \frac{1}{x\ln^2(x)} \; dx = \int_{\ln(3)}^\infty \answer{1/u^2} \; du
\]
This is a $p$-integral with $p = \answer{2}$\\
According to the theory of $p$-integrals, this integral
\begin{multipleChoice}
\choice[correct]{converges}
\choice{diverges}
\end{multipleChoice}
\end{problem}


\begin{problem}(problem 5b)
Use the theory of $p$-integrals to determine the values of $p$ for which the improper integral converges
\[
\int_2^\infty \frac{1}{x \ln^p(x)} \; dx
\]
This integral will converge for $p > \answer{1}$.
\end{problem}



\section{Vertical Asymptotes in the Interval of Integration}

In this section we examine definite integrals in which the integrand has a vertical asymptote at an endpoint of integration.

\begin{definition}[Type 2 Improper Integral]
Suppose $f(x)$ is continuous on the interval $(a, b]$ and that $f(x)$ has a vertical asymptote at $x = a$.
Then
\[
\int_a^b f(x) \; dx = \lim_{t \to a^+} \int_t^b f(x) \; dx.
\]
The improper integral is defined similarly if the vertical asymptote is at the upper endpoint of integration.
\end{definition}

\begin{example}[example 6]
Compute the improper integral
\[
\int_0^1 \frac{3}{\sqrt x} \; dx.
\]
The integral is improper because the integrand has a vertical asymptote at $x = 0$. Hence
\begin{align*}
\int_0^1 \frac{3}{\sqrt x} \; dx &= \lim_{a \to 0^+} \int_a^1 \frac{3}{\sqrt x} \; dx\\
                                 &= \lim_{a \to 0^+} \int_a^1 3x^{-1/2} \; dx\\
                                 &= \lim_{a \to 0^+} \left( 6x^{1/2}  \right) \bigg|_a^1\\
                                 &= \lim_{a \to 0^+} \left( 6\sqrt{1} - 6\sqrt{a}  \right) \\
                                 &=  6 - 0   \\
                                 &= 6.
\end{align*}
 Therefore, the improper integral converges to 6.
 \end{example}
 
\begin{problem}(problem 6)
Compute the improper integral
\[
\int_2^6 \frac{4}{\sqrt{x-2}} \; dx = \answer{16}
\]
Does the improper integral converge or diverge?
\begin{multipleChoice}
\choice[correct]{converge}
\choice{diverge}
\end{multipleChoice}
\end{problem}


\begin{example}[example 7]
Compute the improper integral
\[
\int_0^{\pi/2} \tan(x) \; dx.
\]
The integral is improper because the integrand has a vertical asymptote at $x = \pi/2$. Hence
\begin{align*}
\int_0^{\pi/2} \tan(x) \; dx &= \lim_{b \to \frac{\pi}{2}^-} \int_0^b \tan(x) \; dx\\
                                 &= \lim_{b \to \frac{\pi}{2}^-} \int_0^b \frac{\sin(x)}{\cos(x)} \; dx\\
                                 &= \lim_{b \to \frac{\pi}{2}^-}  \ln|\sec(x)|   \bigg|_0^b\\
                                 &= \lim_{b \to \frac{\pi}{2}^-} \big( \ln|\sec(b)| - \ln|\sec(0)|  \big) \\
                                 &=  \infty - 0  \\
                                 &= \infty.
\end{align*}
 Therefore, the improper integral diverges.
 \end{example}
 
 \begin{problem}(problem 7)
Compute the improper integral
\[
\int_0^{\pi/2} \cot(x) \; dx
\]
The integrand has a vertical asymptote at
\begin{multipleChoice}
\choice[correct]{0}
\choice{$\pi/2$}
\end{multipleChoice}
The anti derivative is
\[
\int \cot(x) \; dx = \answer{\ln|\sin(x)|} + C
\]
The improper integral
\begin{multipleChoice}
\choice{converges}
\choice[correct]{diverges}
\end{multipleChoice}
\end{problem}



\begin{example}[example 8]
Compute the improper integral
\[
\int_0^\infty xe^{-x} \; dx
\]
We use integration by parts with $u = x$ and $dv = e^{-x} dx$ to obtain
\begin{align*}
\int_0^\infty xe^{-x} \; dx &= \lim_{b \to \infty} \int_0^b xe^{-x} \; dx\\
                                 &= \lim_{b \to \infty}  \left[ -xe^{-x}\bigg|_0^b + \int_0^b e^{-x} \; dx \right]   \\
                                 &= \lim_{b \to \infty}  -xe^{-x}- e^{-x}\bigg|_0^b  \\
                                 &= \lim_{b \to \infty}  xe^{-x} + e^{-x}\bigg|_b^0  \\
                                 &= \lim_{b \to \infty}  (0+1) - (be^{-b} + e^{-b})    \\
                                 &=  1 - 0  \\
                                 &= 1
\end{align*}
 Therefore, the improper integral converges.
 \begin{remark}
 L'Hopital's rule is used to compute the limit of $be^{-b}$:
 \[
 \lim_{b \to \infty} be^{-b} = \lim_{b \to \infty} \frac{b}{e^b} = \frac{\infty}{\infty}
 \]
 and now, by L'Hopital's rule, we get
 \[
 \lim_{b \to \infty} \frac{b}{e^b} = \lim_{b \to \infty} \frac{1}{e^b} = 0.
 \]
 \end{remark}
 
 \end{example}

           
           

\section{Doubly Improper Integrals}

Doubly improper integrals have the form 
\[
\int_{-\infty}^\infty f(x) \; dx
\]
To compute a doubly improper integral, we let $a$ be any number (typically 0) and split it into two improper integrals:
\[
\int_{-\infty}^\infty f(x) \; dx = \int_{-\infty}^a f(x) \; dx + \int_a^\infty f(x) \; dx
\]
If either of these improper integrals diverges, then we say that the doubly improper integral diverges.
Otherwise, we say that it converges.

\begin{example}[example 9]
Compute the doubly improper integral
\[
\int_{-\infty}^\infty \frac{1}{1+x^2} \; dx
\]
We will rewrite this integral as
\[
\int_{-\infty}^\infty \frac{1}{1+x^2} \; dx = \int_{-\infty}^0 \frac{1}{1+x^2} \; dx + \int_0^\infty \frac{1}{1+x^2} \; dx
\] 
The first integral is
\[
\int_{-\infty}^0 \frac{1}{1+x^2} \; dx = \lim_{a \to -\infty} \int_{a}^0 \frac{1}{1+x^2} \; dx
\]
\[
= \lim_{a \to -\infty} \tan^{-1}(x) \bigg|_a^0 = \lim_{a \to -\infty} \left[\tan^{-1}(0) - \tan^{-1}(a)\right]
\]
\[
= 0 - \left(-\frac{\pi}{2}\right) = \frac{\pi}{2}
\]
Thus the first integral converges. You can verify that the second integral also converges to $\frac{\pi}{2}$,
and so the double improper integral converges to $\pi$, i.e.
\[
\int_{-\infty}^\infty \frac{1}{1+x^2} \; dx = \frac{\pi}{2} + \frac{\pi}{2} = \pi
\]

\end{example}


                    
\begin{center}
\begin{foldable}
\unfoldable{Here is a detailed, lecture style video on improper integrals:}
\youtube{gRJTe1i1muo}
\end{foldable}
\end{center}





\end{document}




