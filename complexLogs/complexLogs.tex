\documentclass[handout]{ximera}

%% You can put user macros here
%% However, you cannot make new environments



\newcommand{\ffrac}[2]{\frac{\text{\footnotesize $#1$}}{\text{\footnotesize $#2$}}}
\newcommand{\vasymptote}[2][]{
    \draw [densely dashed,#1] ({rel axis cs:0,0} -| {axis cs:#2,0}) -- ({rel axis cs:0,1} -| {axis cs:#2,0});
}


\graphicspath{{./}{firstExample/}}
\usepackage{forest}
\usepackage{amsmath}
\usepackage{amssymb}
\usepackage{array}
\usepackage[makeroom]{cancel} %% for strike outs
\usepackage{pgffor} %% required for integral for loops
\usepackage{tikz}
\usepackage{tikz-cd}
\usepackage{tkz-euclide}
\usetikzlibrary{shapes.multipart}


%\usetkzobj{all}
\tikzstyle geometryDiagrams=[ultra thick,color=blue!50!black]


\usetikzlibrary{arrows}
\tikzset{>=stealth,commutative diagrams/.cd,
  arrow style=tikz,diagrams={>=stealth}} %% cool arrow head
\tikzset{shorten <>/.style={ shorten >=#1, shorten <=#1 } } %% allows shorter vectors

\usetikzlibrary{backgrounds} %% for boxes around graphs
\usetikzlibrary{shapes,positioning}  %% Clouds and stars
\usetikzlibrary{matrix} %% for matrix
\usepgfplotslibrary{polar} %% for polar plots
\usepgfplotslibrary{fillbetween} %% to shade area between curves in TikZ



%\usepackage[width=4.375in, height=7.0in, top=1.0in, papersize={5.5in,8.5in}]{geometry}
%\usepackage[pdftex]{graphicx}
%\usepackage{tipa}
%\usepackage{txfonts}
%\usepackage{textcomp}
%\usepackage{amsthm}
%\usepackage{xy}
%\usepackage{fancyhdr}
%\usepackage{xcolor}
%\usepackage{mathtools} %% for pretty underbrace % Breaks Ximera
%\usepackage{multicol}



\newcommand{\RR}{\mathbb R}
\newcommand{\R}{\mathbb R}
\newcommand{\C}{\mathbb C}
\newcommand{\N}{\mathbb N}
\newcommand{\Z}{\mathbb Z}
\newcommand{\dis}{\displaystyle}
%\renewcommand{\d}{\,d\!}
\renewcommand{\d}{\mathop{}\!d}
\newcommand{\dd}[2][]{\frac{\d #1}{\d #2}}
\newcommand{\pp}[2][]{\frac{\partial #1}{\partial #2}}
\renewcommand{\l}{\ell}
\newcommand{\ddx}{\frac{d}{\d x}}

\newcommand{\zeroOverZero}{\ensuremath{\boldsymbol{\tfrac{0}{0}}}}
\newcommand{\inftyOverInfty}{\ensuremath{\boldsymbol{\tfrac{\infty}{\infty}}}}
\newcommand{\zeroOverInfty}{\ensuremath{\boldsymbol{\tfrac{0}{\infty}}}}
\newcommand{\zeroTimesInfty}{\ensuremath{\small\boldsymbol{0\cdot \infty}}}
\newcommand{\inftyMinusInfty}{\ensuremath{\small\boldsymbol{\infty - \infty}}}
\newcommand{\oneToInfty}{\ensuremath{\boldsymbol{1^\infty}}}
\newcommand{\zeroToZero}{\ensuremath{\boldsymbol{0^0}}}
\newcommand{\inftyToZero}{\ensuremath{\boldsymbol{\infty^0}}}


\newcommand{\numOverZero}{\ensuremath{\boldsymbol{\tfrac{\#}{0}}}}
\newcommand{\dfn}{\textbf}
%\newcommand{\unit}{\,\mathrm}
\newcommand{\unit}{\mathop{}\!\mathrm}
%\newcommand{\eval}[1]{\bigg[ #1 \bigg]}
\newcommand{\eval}[1]{ #1 \bigg|}
\newcommand{\seq}[1]{\left( #1 \right)}
\renewcommand{\epsilon}{\varepsilon}
\renewcommand{\iff}{\Leftrightarrow}

\DeclareMathOperator{\arccot}{arccot}
\DeclareMathOperator{\arcsec}{arcsec}
\DeclareMathOperator{\arccsc}{arccsc}
\DeclareMathOperator{\si}{Si}
\DeclareMathOperator{\proj}{proj}
\DeclareMathOperator{\scal}{scal}
\DeclareMathOperator{\cis}{cis}
\DeclareMathOperator{\Arg}{Arg}
%\DeclareMathOperator{\arg}{arg}
\DeclareMathOperator{\Rep}{Re}
\DeclareMathOperator{\Imp}{Im}
\DeclareMathOperator{\sech}{sech}
\DeclareMathOperator{\csch}{csch}
\DeclareMathOperator{\Log}{Log}

\newcommand{\tightoverset}[2]{% for arrow vec
  \mathop{#2}\limits^{\vbox to -.5ex{\kern-0.75ex\hbox{$#1$}\vss}}}
\newcommand{\arrowvec}{\overrightarrow}
\renewcommand{\vec}{\mathbf}
\newcommand{\veci}{{\boldsymbol{\hat{\imath}}}}
\newcommand{\vecj}{{\boldsymbol{\hat{\jmath}}}}
\newcommand{\veck}{{\boldsymbol{\hat{k}}}}
\newcommand{\vecl}{\boldsymbol{\l}}
\newcommand{\utan}{\vec{\hat{t}}}
\newcommand{\unormal}{\vec{\hat{n}}}
\newcommand{\ubinormal}{\vec{\hat{b}}}

\newcommand{\dotp}{\bullet}
\newcommand{\cross}{\boldsymbol\times}
\newcommand{\grad}{\boldsymbol\nabla}
\newcommand{\divergence}{\grad\dotp}
\newcommand{\curl}{\grad\cross}
%% Simple horiz vectors
\renewcommand{\vector}[1]{\left\langle #1\right\rangle}



\outcome{Introduce complex logarithms}

\title{2.2 Complex Logarithms}

\begin{document}

\maketitle



\section{The Multivalued Logarithm}
Consider the function $w = f(z) = e^z = e^x \cis y$. This function is $2\pi i$-periodic, so it is not one-to-one.  Hence, $e^z$ does not have a traditional inverse-
the complex logarithm is multivalued.  
%For a given $w \in \C$, $z=\log w$ gives all of the values of $z$ so that $e^{z} = w$. 
To define the complex log, consider a complex number $w_0$ in the image of $f(z) = e^z$, 
so that $w_0 = e^{z_0}$. If $z_0 = x_0 + iy_0$, then
\[
w_0 = e^{x_0+iy_0} = e^{x_0} \cis (y_0) = e^{x_0} \cis (y_0+2k\pi).  
\]
See the figure below.

\begin{tikzpicture}
    \draw(0,0) -- (3,3)
\end{tikzpicture}

Since $w_0 = e^{x_0}\cis(y_0 + 2k\pi)$, we can see that the modulus of $w_0$ is $|w_0| = e^{x_0}$ and an 
argument of $w_0$ is $y_0$, so that $\arg(w_0) = \{y_0 + 2k\pi\}$

We wish to define the function $\log w_0$ so that the ouputs are the set of points $x_0 + i(y_0 + 2k\pi)$.  
Thus we define $\log w_0 = \ln|w_0| +i \arg w_0$. 
In the definition below, we use $z$ in the place of $w_0$.
\begin{definition}[Multivalued Logarithm]
For $z \neq 0$, we define the multivalued complex logarithm by
\[
\log z = \ln |z| + i \arg z
\]
\end{definition}

\begin{example}[example 1]
Find the values of $\log(1+i)$\\
We have 
\[
\log(1+i) = \ln |1+i| + i \arg(1+i) = \tfrac12\ln 2 + i\left(\frac{\pi}{4} + 2k\pi\right)
\]
where $k \in \Z$. Note that $\ln \sqrt 2 = \ln 2^{1/2} = \frac12\ln 2$.
\end{example}

\begin{problem}(problem 1a)
Find all values of the complex logarithm: $\; \log i$.\\
\begin{multipleChoice}
\choice{$\frac{\pi}{2} + 2k\pi$}\\
\choice{$1 + i\left(\frac{\pi}{2} + 2k\pi\right)$}\\
\choice[correct]{$i\left(\frac{\pi}{2} + 2k\pi\right)$}\\
\choice{$2k\pi i$}
\end{multipleChoice}
\end{problem}


\begin{problem}(problem 1b)
Find all values of the complex logarithm: $\; \log(-3)$\\
\begin{multipleChoice}
\choice{$ (2k+1)\pi + i\ln 3$}\\
\choice[correct]{$\ln 3 + (2k+1)\pi i$}\\
\choice{$3 + (2k+1)\pi i$}\\
\choice{$(2k-1)\pi i$}
\end{multipleChoice}
\end{problem}


\begin{problem}(problem 1c)
Find all values of the complex logarithm: $\; \log(i-1)$\\
\begin{multipleChoice}
\choice{$\left(\frac{3\pi}{4} + 2k \pi \right) + i \ln \sqrt 2$}\\
\choice{$ \ln \sqrt 2 + i\left(\frac{3\pi}{4} + k \pi \right)$}\\
\choice{$\ln 2 + i\left(\frac{3\pi}{4} + 2k \pi \right)$}\\
\choice[correct]{$\tfrac12 \ln 2 + i\left(\frac{3\pi}{4} + 2k \pi \right)$}
\end{multipleChoice}
\end{problem}

Here is a video solution of problem 1c:\\
\begin{foldable}
\youtube{Qt-Aoldk41I}
\end{foldable}

Some familiar properties of real logarithms carry over to the multivalued complex logarithm.

\begin{proposition}
For $z_1, z_2 \neq 0$, 
\[
\log(z_1\cdot z_2) = \log z_1 + \log z_2
\]
and for $z \neq 0$
\[
\log\left(\frac{1}{z}\right) = - \log z
\]
\end{proposition}
The key to the proof of the first property is that the argument of a product is the sum of the arguments:
\[
\arg(z_1 \cdot z_2) = \arg z_1 + \arg z_2
\]
As a result of this property of the multivalued argument, we have
\begin{align*}
\log(z_1\cdot z_2) &= \ln|z_1 z_2| + i\arg(z_1 z_2)\\
&= \ln|z_1| + \ln|z_2| + i\left[\arg z_1 + \arg z_2\right]\\
&= \log z_1 + \log z_2
\end{align*}

The proof of the second property is similar (verify) and uses
\[
\arg\left(\frac{1}{z}\right) = -\arg z
\]

\begin{corollary}
For $z_1, z_2 \neq 0$
\[
\log\left(\frac{z_1}{z_2}\right) = \log z_1 - \log z_2
\]
\end{corollary}

\section{Principal Log}

Remarking that the complex exponential function, $e^z$, is defined in terms of a radial part, $e^x=r$,
and an angular part $e^{iy}= \cis y$, it is natural that the complex logarithm be defined in terms of the real natural logarithm of a modulus,
and an argument function.

\begin{definition}[$\Log z$]
For $z \neq 0$, we define the principal branch of the complex logarithm by
\[
\Log z =  \ln|z| + i\Arg z
\]
where $\Arg z$ is the principal argument of $z$. 
\end{definition}

 

First, we will show that $\Log z$ and $e^z$ are inverses.


\[
e^{\Log z} = e^{\ln |z| + i\Arg z} = e^{\ln |z|} \cis \Arg z = |z| \cis \Arg z =z
\]
in polar form (if $z=r\cis \theta$ then $r = |z|$ and $\theta \in \arg z$).
Furthermore,
\[
\Log e^z = \Log e^x\cis y = \ln e^x + i\Arg(\cis y) = x+iy
\]
if and only if $-\pi < y \leq \pi$.




\begin{example}[Example 2] 
Find $\Log z$ for $z = -5, i, 1+i, 1-i$ and $-3-i\sqrt 3$.\\
\begin{align*}
& \Log(-5) = \ln|-5|+ i \Arg(-5) = \ln 5 + i\pi\\
& \Log(i) = \ln|i|+ i \Arg(i) = \ln 1 + i\pi/2 = i\pi/2\\
& \Log(1+i) = \ln|1+i|+ i \Arg(1+i) = \ln \sqrt 2 + i\pi/4\\
& \Log(1-i) = \ln|1-i|+ i \Arg(1-i) = \ln \sqrt 2 - i\pi/4\\
& \Log(-3-i\sqrt 3) = \ln|-3-i\sqrt 3|+ i \Arg(-3-i\sqrt 3) = \ln 2\sqrt 3 + i7\pi/6
\end{align*}
\end{example}


\begin{problem}(problem 2) 
Find $\Log z$ for $z = -2, -i, -2-2i, -4+4i$ and $1+i\sqrt 3$.\\
\begin{align*}
& \Log(-2) = \answer{\ln 2 + i\pi }\\
& \Log(-i) = \answer{-i\pi/2}\\
& \Log(-2-2i) = \answer{\frac 32 \ln 2 - i3\pi/4}\\
& \Log(-4+4i) = \answer{\frac52 \ln 2 +i3\pi/4}\\
& \Log(1+i\sqrt 3) = \answer{\ln 2+ i\pi/3}
\end{align*}
\end{problem}

Here is a video solution of one part of problem 2:\\
\begin{foldable}
\youtube{__-0tc1zHnc}
\end{foldable}

The properties of the multivalued complex logarithm do not carry over to the principal branch without modification.
\begin{proposition}
For $z_1, z_2 \neq 0$
\[\Log(z_1\cdot z_2) = \Log z_1 + \Log z_2
\]
if and only if 
\[
-\pi < \Arg z_1 + \Arg z_2 \leq \pi\]

and for $z \neq 0$
\[
\Log\left(\frac{1}{z}\right) = - \Log z
\]
if and only if $z$ is not a negative real number.
\end{proposition}
\begin{question}Under what conditions can we expect
\[
\Log\left(\frac{z_1}{z_2}\right) = \Log z_1 - \Log z_2?
\]
\end{question}


\section{Other Branches}

Let $\alpha \in \R$ and consider the interval $(\alpha, \alpha + 2\pi]$. For any non-zero $z \in \C$, there is an element of the
set $\arg z$ which belongs to this interval. Hence, we can define a branch of the complex logarithm as follows.

\begin{definition}
Let $\alpha \in \R$ and let $\theta_\alpha$ be an argument of $z$ in the interval $(\alpha, \alpha + 2\pi]$.
Then we define
\[
\log_\alpha z = \ln|z| +i\theta_\alpha
\]
\end{definition}


\begin{example}[Example 3] 
Find the indicated logarithm. The expression $\arg_\alpha(z)$ is used below to indicate the element of the set 
$\arg z$ which lies in the interval $(\alpha, \alpha + 2\pi]$.\\
\begin{align*}
i)  \, & \log_0(5) = \ln|5|+ i \arg_0(5) = \ln 5 + i2\pi\\
ii) \,  & \log_\pi(i) = \ln|i|+ i \arg_\pi(i) = \ln 1 + i5\pi/2 = i5\pi/2\\
iii) \,  & \log_{2\pi}(-2) = \ln|-2|+ i \arg_{2\pi}(-2) = \ln 2 + i3\pi\\
iv) \,  & \log_0(-1-i)= \ln|-1-i|+ i \arg_0(-1-i) = \ln \sqrt 2 + i5\pi/4\\
v)  \, & \log_{4\pi}(\sqrt 3 + i) = \ln|\sqrt 3 + i|+ i \arg_{4\pi}(\sqrt 3+i) = \ln 2 + i25\pi/6
\end{align*}
\end{example}


\begin{problem}(problem 3) 
Find the following complex logarithms:\\
\begin{align*}
i) \, & \log_0(3) = \answer{\ln 3 + i2\pi }\\
ii)  \, & \log_\pi(-3) = \answer{\ln 3 + i3\pi}\\
iii) \,  & \log_\pi(-i) = \answer{ i7\pi/2}\\
iv)  \, & \log_{2\pi}(1+i) = \answer{\ln \sqrt 2 +i11\pi/4}\\
v)  \, & \log_0(1- i\sqrt 3) = \answer{\ln 2+ 5\pi/3}
\end{align*}
\end{problem}

Here is a video solution of problem 3, part ii:\\
\begin{foldable}
\youtube{2y0fdV8ihG0}
\end{foldable}



\section{Complex Powers}
We define power functions with complex powers. To do so, recall that the real functions $e^x$ and $\ln x$ 
are inverses, so that for $x>0$, we can write
\[
x = e^{\ln x}
\]
Furthermore, a property of logarithms allows us to write
\[
\ln(x^n) = n \ln x
\]
Combining these gives
\[
x^n = e^{n\ln x}
\]
\begin{definition}
For $c \in \C$ and $z \neq 0$, we define
\[
z^c = e^{c \log z}
\]
and the principal branch of the complex power function is given by
\[
z^c = e^{c \Log z}
\]
\end{definition}

\begin{example}[example 4]
Find all values of $i^i$ and highlight the principal value.\\
Using the definition of a complex exponent, we have
\[
i^i = e^{i\log i} = e^{i(\pi/2 + 2k\pi)i} 
\]
\[
= e^{-(\pi/2 + 2k\pi)}
\]
The principal value occurs when $k=0$.  It is
\[
i^i = e^{-\pi/2} \approx 0.20788...
\]
Note that the values of $i^i$ are real!
\end{example}

\begin{problem}(problem 4)
Find all values of each of the exponential expressions and highlight their principal values.\\
\begin{align*}
i) &\; 4^i; \;\; \mbox{Principal value:}\;\; \answer{e^{i\ln 4}}\\
ii) & \;(1+i)^{\pi i}; \;\;\mbox{Principal value:}\;\; \answer{e^{-\pi^2/4 + \pi i \ln(\sqrt2)}}\\
iii) & \;(-1)^{\sqrt 2}; \;\;\mbox{Principal value:}\;\; \answer{e^{i\pi\sqrt 2}}
\end{align*}
\end{problem}

Here is a video solution of problem 4, part iii:\\
\begin{foldable}
\youtube{X2glIMUFc6E}
\end{foldable}

test

\end{document}











