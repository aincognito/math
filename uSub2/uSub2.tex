\documentclass{ximera}
\usepackage{tcolorbox}
%% You can put user macros here
%% However, you cannot make new environments



\newcommand{\ffrac}[2]{\frac{\text{\footnotesize $#1$}}{\text{\footnotesize $#2$}}}
\newcommand{\vasymptote}[2][]{
    \draw [densely dashed,#1] ({rel axis cs:0,0} -| {axis cs:#2,0}) -- ({rel axis cs:0,1} -| {axis cs:#2,0});
}


%\usepackage{tcolorbox} %%Needed for Derivative Definition supposedly and product rule, natural exp log, quotient rule, inverse trig, rates of change


% \graphicspath{{./}{firstExample/}}
% \usepackage{forest}
\usepackage{amsmath}
\usepackage{amssymb}
\usepackage{array}
\usepackage[makeroom]{cancel} %% for strike outs
\usepackage{pgffor} %% required for integral for loops
\usepackage{tikz}
\usepackage{tikz-cd}
\usepackage{tkz-euclide}
\usetikzlibrary{shapes.multipart}


% \usetkzobj{all}
\tikzstyle geometryDiagrams=[ultra thick,color=blue!50!black]


\usetikzlibrary{arrows}
\tikzset{>=stealth,commutative diagrams/.cd,
  arrow style=tikz,diagrams={>=stealth}} %% cool arrow head
\tikzset{shorten <>/.style={ shorten >=#1, shorten <=#1 } } %% allows shorter vectors

\usetikzlibrary{backgrounds} %% for boxes around graphs
\usetikzlibrary{shapes,positioning}  %% Clouds and stars
\usetikzlibrary{matrix} %% for matrix
\usepgfplotslibrary{polar} %% for polar plots
\usepgfplotslibrary{fillbetween} %% to shade area between curves in TikZ



%\usepackage[width=4.375in, height=7.0in, top=1.0in, papersize={5.5in,8.5in}]{geometry}
%\usepackage[pdftex]{graphicx}
%\usepackage{tipa}
%\usepackage{txfonts}
%\usepackage{textcomp}
%\usepackage{amsthm}
%\usepackage{xy}
%\usepackage{fancyhdr}
%\usepackage{xcolor}
%\usepackage{mathtools} %% for pretty underbrace % Breaks Ximera
%\usepackage{multicol}



\newcommand{\RR}{\mathbb R}
\newcommand{\R}{\mathbb R}
\newcommand{\C}{\mathbb C}
\newcommand{\N}{\mathbb N}
\newcommand{\Z}{\mathbb Z}
\newcommand{\dis}{\displaystyle}
%\renewcommand{\d}{\,d\!}
\renewcommand{\d}{\mathop{}\!d}
\newcommand{\dd}[2][]{\frac{\d #1}{\d #2}}
\newcommand{\pp}[2][]{\frac{\partial #1}{\partial #2}}
\renewcommand{\l}{\ell}
\newcommand{\ddx}{\frac{d}{\d x}}
\newcommand{\ppx}{\frac{\partial}{\partial x}}
\newcommand{\ppy}{\frac{\partial}{\partial y}}

\newcommand{\zeroOverZero}{\ensuremath{\boldsymbol{\tfrac{0}{0}}}}
\newcommand{\inftyOverInfty}{\ensuremath{\boldsymbol{\tfrac{\infty}{\infty}}}}
\newcommand{\zeroOverInfty}{\ensuremath{\boldsymbol{\tfrac{0}{\infty}}}}
\newcommand{\zeroTimesInfty}{\ensuremath{\small\boldsymbol{0\cdot \infty}}}
\newcommand{\inftyMinusInfty}{\ensuremath{\small\boldsymbol{\infty - \infty}}}
\newcommand{\oneToInfty}{\ensuremath{\boldsymbol{1^\infty}}}
\newcommand{\zeroToZero}{\ensuremath{\boldsymbol{0^0}}}
\newcommand{\inftyToZero}{\ensuremath{\boldsymbol{\infty^0}}}


\newcommand{\numOverZero}{\ensuremath{\boldsymbol{\tfrac{\#}{0}}}}
\newcommand{\dfn}{\textbf}
%\newcommand{\unit}{\,\mathrm}
\newcommand{\unit}{\mathop{}\!\mathrm}
%\newcommand{\eval}[1]{\bigg[ #1 \bigg]}
\newcommand{\eval}[1]{ #1 \bigg|}
\newcommand{\seq}[1]{\left( #1 \right)}
\renewcommand{\epsilon}{\varepsilon}
\renewcommand{\iff}{\Leftrightarrow}

\DeclareMathOperator{\arccot}{arccot}
\DeclareMathOperator{\arcsec}{arcsec}
\DeclareMathOperator{\arccsc}{arccsc}
\DeclareMathOperator{\si}{Si}
\DeclareMathOperator{\proj}{proj}
\DeclareMathOperator{\scal}{scal}
\DeclareMathOperator{\cis}{cis}
\DeclareMathOperator{\Arg}{Arg}
%\DeclareMathOperator{\arg}{arg}
\DeclareMathOperator{\Rep}{Re}
\DeclareMathOperator{\Imp}{Im}
\DeclareMathOperator{\sech}{sech}
\DeclareMathOperator{\csch}{csch}
\DeclareMathOperator{\Log}{Log}

\newcommand{\tightoverset}[2]{% for arrow vec
  \mathop{#2}\limits^{\vbox to -.5ex{\kern-0.75ex\hbox{$#1$}\vss}}}
\newcommand{\arrowvec}{\overrightarrow}
\renewcommand{\vec}{\mathbf}
\newcommand{\veci}{{\boldsymbol{\hat{\imath}}}}
\newcommand{\vecj}{{\boldsymbol{\hat{\jmath}}}}
\newcommand{\veck}{{\boldsymbol{\hat{k}}}}
\newcommand{\vecl}{\boldsymbol{\l}}
\newcommand{\utan}{\vec{\hat{t}}}
\newcommand{\unormal}{\vec{\hat{n}}}
\newcommand{\ubinormal}{\vec{\hat{b}}}

\newcommand{\dotp}{\bullet}
\newcommand{\cross}{\boldsymbol\times}
\newcommand{\grad}{\boldsymbol\nabla}
\newcommand{\divergence}{\grad\dotp}
\newcommand{\curl}{\grad\cross}
%% Simple horiz vectors
\renewcommand{\vector}[1]{\left\langle #1\right\rangle}


\outcome{Compute an anti-derivative by reversing the chain rule.}

\title{0.3 Substitution}



\begin{document}

\begin{abstract}
In this section we learn to reverse the chain rule by making a substitution.
\end{abstract}

\maketitle

\section{U-Substitution}

We make a u-substitution to fill in the gaps in the following equation, which reverses the chain rule:

\[\int f(g(x))g'(x) \ dx = F(g(x)) + C. \]

If we let $ u = g(x)$ then $du = g'(x) \ dx$ and the integral can be rewritten as
\[\int f(g(x))g'(x) \ dx = \int f(u) \ du \]
which equals
\[F(u) + C\]
and back substituting gives
\[F(g(x)) + C. \]
Now we can see that
\[\int f(g(x))g'(x) \ dx = F(g(x)) + C. \]



\begin{example}[example 1]
Compute 
\[\int 2x\cos(x^2 + 1) \ dx.\]

Let $u = x^2 + 1$.  We have,
\[u = x^2 + 1\]
\[du = 2x \ dx\]

and the integral can be written as 
\[\int 2x\cos(x^2 + 1) \ dx = \int \cos(x^2 + 1) \ 2x\  dx = \int \cos(u) \ du.\]
The last integral can be computed as 
\[\int \cos(u) \ du = \sin(u) + C\]
and by back substituting, we have 
\[\sin(u) + C = \sin(x^2 + 1) + C.\]
Thus, using u-substitution we can conclude that
\[\int 2x\cos(x^2 + 1) \ dx =  \sin(x^2 + 1) + C.\]
\end{example}

\begin{problem}(problem 1a) Compute: $\displaystyle{\int 3x^2\cos(x^3 + 2) \ dx.}$\\
Let $u = \answer{x^3 + 2}$, then $du = \answer{3x^2 dx}$.\\
\begin{hint}
Don't forget the `dx' in your answer for `du'.
\end{hint}
Convert to an intergal in the variable $u$:
\[\int 3x^2\cos(x^3 + 2) \ dx = \int\answer{\cos(u)} \; du\]
The final answer in terms of $x$ is:

\[\int 3x^2\cos(x^3 + 2) \ dx = \answer{\sin(x^3 + 2)} +C.\]
\end{problem}

\begin{problem}(problem 1b) Compute: $\displaystyle{\int 5x^4\sin(x^5 -7) \ dx}$.\\
Let $u = \answer{x^5 - 7}$, then $du = \answer{5x^4 dx}$.\\
\begin{hint}
Don't forget the `dx' in your answer for `du'.
\end{hint}
Convert to an intergal in the variable $u$:
\[\int 5x^4\sin(x^5 -7) \ dx = \int\answer{\sin(u)} \; du\]
The final answer in terms of $x$ is:
\[\int 5x^4\sin(x^5 -7) \ dx = \answer{-\cos(x^5 -7)} +C.\]
\end{problem}


\begin{example} Compute 
\[\int 2x(x^2 + 1)^4 \ dx.\]
Let $u = x^2 + 1$.  We have,
\[u = x^2 + 1\]
\[du = 2x \ dx\]

and the integral can be written as 
\[\int 2x(x^2 + 1)^4 \ dx = \int (x^2 + 1)^4 \ 2x \  dx = \int u^4 \ du.\]
The last integral can be computed as 
\[\int u^4 \ du = \tfrac{u^5}{5} + C\]
and by back substituting, we have 
\[\tfrac{u^5}{5} + C = \tfrac15(x^2 + 1)^5 + C .\]
Thus, using u-substitution we can conclude that
\[\int 2x(x^2 + 1)^4 \ dx  =  \tfrac15(x^2 + 1)^5 + C .\]
\end{example}

\begin{problem}
\begin{hint}
Let $u = x^3 + 2$
\end{hint}
\begin{hint}
Compute $du$
\end{hint}
\[\int 3x^2(x^3 + 2)^6 \ dx = \answer{(x^3 + 2)^7/7} +C.\]
\end{problem}

\begin{problem}
\begin{hint}
Let $u = x^4 -1$
\end{hint}
\begin{hint}
Compute $du$
\end{hint}
\[\int 4x^3(x^4 -1)^8 \ dx = \answer{(x^4 -1)^9/9} +C.\]
\end{problem}

\begin{example} Compute 
\[\int 2xe^{(x^2 + 1)} \ dx.\]
Let $u = x^2 + 1$.  We have,
\[u = x^2 + 1\]
\[du = 2x \ dx\]

\[\int 2xe^{(x^2 + 1)} \ dx = \int e^{(x^2 + 1)} \ 2x\  dx = \int e^u \ du.\]
The last integral can be computed as 
\[\int e^u \ du = e^u + C\]
and by back substituting, we have 
\[e^u + C = e^{(x^2 + 1)} + C.\]
Thus, using u-substitution we can conclude that
\[\int 2xe^{(x^2 + 1)} \ dx =  e^{(x^2 + 1)} + C.\]
\end{example}

\begin{problem}
\begin{hint}
Let $u = x^3 + 2$
\end{hint}
\begin{hint}
Compute $du$
\end{hint}
\[\int 3x^2e^{x^3 + 2} \ dx = \answer{e^{x^3 + 2}} +C.\]
\end{problem}

\begin{problem}
\begin{hint}
Let $u = -x^2$
\end{hint}
\begin{hint}
Compute $du$
\end{hint}
\[\int -2xe^{-x^2} \ dx = \answer{e^{-x^2}} +C.\]
\end{problem}

\begin{example} Compute 
\[\int 2x\sqrt{x^2 + 1} \ dx.\]
Let $u = x^2 + 1$.  We have,
\[u = x^2 + 1\]
\[du = 2x \ dx\]

\[\int 2x\sqrt{x^2 + 1} \ dx = \int \sqrt{x^2 + 1} \cdot 2x\  dx = \int \sqrt{u} \ du.\]
The last integral can be computed as 
\[\int \sqrt u  \ du = \int u^{1/2} \ du = \tfrac{u^{3/2}}{3/2} + C = \tfrac23 u^{3/2} + C\]
and by back substituting, we have 
\[\tfrac23 u^{3/2}  + C = \frac23 (x^2 + 1)^{3/2} + C.\]
Thus, using u-substitution we can conclude that
\[\int 2x\sqrt{x^2 + 1} \ dx =  \tfrac23 (x^2 + 1)^{3/2} + C.\]
\end{example}

\begin{problem}
\begin{hint}
Let $u = x^4 + 2$
\end{hint}
\begin{hint}
Compute $du$
\end{hint}
\[\int 4x^3 \sqrt{x^4 +2} \ dx = \answer{2/3(x^4 +2)^{3/2}} +C.\]
\end{problem}



\begin{example} Compute 
\[\int \frac{2x}{x^2 + 1} \ dx.\]
Let $u = x^2 + 1$.  We have,
\[u = x^2 + 1\]
\[du = 2x \ dx\]

\[\int \frac{2x}{x^2 + 1} \ dx = \int \frac{1}{x^2 + 1} \cdot 2x\  dx = \int \frac{1}{u} \ du.\]
The last integral can be computed as 
\[\int \frac{1}{u} \ du = \ln|u| + C\]
and by back substituting, we have 
\[\ln|u| + C =  \ln|x^2 + 1| + C=\ln(x^2 + 1) + C.\]
Thus, using u-substitution we can conclude that
\[\int \frac{2x}{x^2 + 1} \ dx =  \ln(x^2 + 1) + C.\]
\end{example}


\begin{problem}
\begin{hint}
Let $u = x^4 + 2$
\end{hint}
\begin{hint}
Compute $du$
\end{hint}
\[\int \frac{4x^3}{x^4 +2} \ dx = \answer{\ln|x^4 +2|} +C.\]
\end{problem}



\begin{example} Compute 
\[\int e^x\sin(e^x) \ dx.\]
We let $u = e^x$.  We have,
\[u = e^x\]
\[du = e^x \ dx\]

and the integral can be written as 
\[\int e^x\sin(e^x) \ dx = \int \sin(e^x) \cdot e^x \  dx = \int \sin(u) \ du.\]
The last integral can be computed as 
\[\int \sin(u) \ du = -\cos(u) + C\]
and by back substituting, we have 
\[-\cos(u) + C = -\cos(e^x) + C.\]
Thus, using u-substitution we can conclude that
\[\int e^x\sin(e^x) \ dx =  -\cos(e^x) + C.\]
\end{example}


\begin{problem}
\begin{hint}
Let $u = e^x$
\end{hint}
\begin{hint}
Compute $du$
\end{hint}
\[\int e^x\cos(e^x) \ dx = \answer{\sin(e^x)} +C.\]
\end{problem}

\begin{problem}
\begin{hint}
Let $u = \ln(x)$
\end{hint}
\begin{hint}
Compute $du$
\end{hint}
\[\int \frac{\cos(\ln(x))}{x} \ dx = \answer{\sin(\ln(x))} +C.\]
\end{problem}


\begin{example} Compute 
\[\int 3x^2\sec(x^3)\tan(x^3) \ dx.\]
Let $u = x^3$.  Then we have,
\[u = x^3\]
\[du = 3x^2 \ dx\]

and the integral can be written as 
\begin{align*}
\int 3x^2\sec(x^3)\tan(x^3) \ dx &= \int \sec(x^3)\tan(x^3) \cdot 3x^2 \ dx\\
&= \int \sec(u)\tan(u) \ du.
\end{align*}
The last integral can be computed as 
\[\int \sec(u)\tan(u) \ du = \sec(u) + C\]
and by back substituting, we have 
\[\sec(u) + C = \sec(x^3) + C.\]
Thus, using u-substitution we can conclude that
\[\int 3x^2\sec(x^3)\tan(x^3) \ dx =  \sec(x^3) + C.\]
\end{example}

\begin{problem}
\begin{hint}
Let $u = e^x$
\end{hint}
\begin{hint}
Compute $du$
\end{hint}
\[\int e^x\sec(e^x)\tan(e^x) \ dx = \answer{\sec(e^x)} +C.\]
\end{problem}

\begin{problem}
\begin{hint}
Let $u = x^2$
\end{hint}
\begin{hint}
Compute $du$
\end{hint}
\[\int 2x\csc(x^2)\cot(x^2) \ dx = \answer{-\csc(x^2)} +C.\]
\end{problem}

\begin{example} Compute 
\[\int x^3\cos(x^4) \ dx.\]
Let $u = x^4$.  Then,
\[u = x^4\]
\[du = 4x^3 \ dx\]

and the integral can be written as 
\begin{align*}
\int x^3\cos(x^4) \ dx &= \int \cos(x^4) \cdot x^3\  dx \\
&=  \int \cos(x^4)\cdot \tfrac14 \cdot 4x^3\  dx\\
&=  \int \tfrac14\cos(u) \ du.
\end{align*}
The last integral can be computed as 
\[\int \tfrac14 \cos(u) \ du = \tfrac14 \sin(u) + C\]
and by back substituting, we have 
\[\tfrac14 \sin(u) + C = \tfrac14 \sin(x^4) + C.\]
Thus, using u-substitution we can conclude that
\[\int x^3\cos(x^4) \ dx =  \tfrac14 \sin(x^4) + C.\]
\end{example}

\begin{problem}
\begin{hint}
Let $u = x^2$
\end{hint}
\begin{hint}
Compute $du$
\end{hint}
\[\int x\cos(x^2) \ dx = \answer{\sin(x^2)/2} +C.\]
\end{problem}


\begin{example} Compute 
\[\int xe^{-x^2} \ dx.\]
Let $u = -x^2 $.  Then we have,
\[u = -x^2\]
\[du = -2x \ dx\]
and the integral can be written as
\begin{align*}
\int xe^{-x^2} \ dx &= \int e^{-x^2} \cdot x\  dx \\
&=  \int e^{-x^2}( -\tfrac12)\ (-2x)\  dx \\
&=  \int -\tfrac12 e^u \ du.
\end{align*}
The last integral can be computed as 
\[ \int -\tfrac12 e^u \ du = -\tfrac12 e^u + C\]
and by back substituting, we have 
\[-\tfrac12 e^u + C = -\tfrac12 e^{-x^2} + C.\]
Thus, using u-substitution we can conclude that
\[\int xe^{-x^2} \ dx =  -\tfrac12 e^{-x^2} + C.\]
\end{example}


\begin{problem}
\begin{hint}
Let $u = x^3$
\end{hint}
\begin{hint}
Compute $du$
\end{hint}
\[\int x^2e^{x^3} \ dx = \answer{e^{x^3}/3} +C.\]
\end{problem}


\begin{problem}
\begin{hint}
Let $u = \sqrt x$
\end{hint}
\begin{hint}
Compute $du$
\end{hint}
\[\int \frac{e^{\sqrt x}}{\sqrt x} \ dx = \answer{2e^{\sqrt x}} +C.\]
\end{problem}


\begin{example} Compute 
\[\int \sin(3x) \ dx.\]
Let $u = 3x$.  Then we have,
\[u = 3x\]
\[du = 3 \ dx\]
and the integral can be written as 
\[\int\sin(3x) \ dx =  \int \sin(3x) \cdot \tfrac13\cdot 3 \   dx =   \int \tfrac13 \sin(u) \ du.\]
The last integral can be computed as 
\[ \int \tfrac13 \sin(u) \ du = -\tfrac13 \cos(u) + C\]
and by back substituting, we have 
\[-\tfrac13 \cos(u) + C = -\tfrac13 \cos(3x) + C.\]
Thus, using u-substitution we can conclude that
\[\int \sin(3x) \ dx =  -\tfrac13 \cos(3x) + C.\]
\end{example}

\begin{example} Compute 
\[\int e^{\frac{x}{2}} \ dx.\]
Let $u = \frac{x}{2}$.  Then we have,
\[u = \frac{x}{2}\]
\[du = \tfrac12 \ dx\]
and the integral can be written as 
\[\int e^{\frac{x}{2}} \ dx =  \int e^{\frac{x}{2}} \cdot 2\cdot \tfrac{1}{2}  \   dx =   \int 2e^u \ du.\]
The last integral can be computed as 
\[\int 2e^u \ du = 2 e^u + C\]
and by back substituting, we have 
\[2e^u + C = 2e^{\frac{x}{2}}+ C.\]
Thus, using u-substitution we can conclude that
\[\int e^{x/2} \ dx = 2e^{x/2} + C.\]
\end{example}

\begin{example} Compute 
\[\int (4x+3)^5 \ dx.\]
Let $u = 4x+3$. Then we have,
\[u = 4x+3\]
\[du = 4 \ dx\]
and the integral can be written as 
\[\int (4x+3)^5 \ dx =   \int (4x+3)^5 \cdot \tfrac14\cdot 4  \   dx =   \int \tfrac14 u^5 \ du.\]
The last integral can be computed as 
\[\tfrac14  \int u^5 \ du = \tfrac14  \cdot \tfrac{u^6}{6} + C = \tfrac{u^6}{24} + C\]
and by back substituting, we have 
\[\tfrac{u^6}{24} + C = \tfrac{(4x+3)^6}{24}+ C.\]
Thus, using u-substitution we can conclude that
\[\int (4x+3)^5 \ dx = \tfrac{1}{24}(4x+3)^6 + C.\]
\end{example}

\begin{example} Compute 
\[\int \frac{1}{x\ln^2(x)} \ dx.\]
Let $u = \ln(x)$. Then we have,
\[u = \ln(x)\]
\[du = \tfrac1x \ dx\]
and the integral can be written as 
\[\int \frac{1}{x\ln^2(x)} \ dx = \int \frac{1}{\ln^2(x)} \cdot \frac{1}{x}\  dx = \int \frac{1}{u^2} \ du.\]
The last integral can be computed as 
\[\int \frac{1}{u^2} \ du = \int u^{-2} \ du = \tfrac{u^{-1}}{-1} + C 
= -\frac{1}{u} + C\]
and by back substituting, we have 
\[-\frac{1}{u} + C =  -\frac{1}{\ln(x)} + C.\]
Thus, using u-substitution we can conclude that
\[\int \frac{1}{x\ln^2(x)} \ dx =  -\frac{1}{\ln(x)} + C.\]
\end{example}


\begin{problem}
\begin{hint}
Let $u = \ln(x)$
\end{hint}
\begin{hint}
Compute $du$
\end{hint}
\[\int \frac{1}{x\ln^3(x)} \ dx = \answer{- \ln^{-2}(x)/2} +C.\]
\end{problem}

\begin{example} Compute 
\[\int \sin^4(x)\cos(x) \ dx.\]
Let $u = \sin(x)$. Then we have,
\[u = \sin(x)\]
\[du = \cos(x) \ dx\]
and the integral can be written as 
\[\int \sin^4(x)\cos(x) \ dx  = \int u^4 \ du.\]
The last integral can be computed as 
\[\int u^4 \ du = \tfrac{u^5}{5} + C\]
and by back substituting, we have 
\[\tfrac{u^5}{5} + C = \tfrac15 \sin^5(x) + C.\]
Thus, using u-substitution we can conclude that
\[\int \sin^4(x)\cos(x) \ dx =  \tfrac15 \sin^5(x) + C.\]
\end{example}


\begin{problem}
\begin{hint}
Let $u = \cos(x)$
\end{hint}
\begin{hint}
Compute $du$
\end{hint}
\[\int \cos^3(x)\sin(x) \ dx = \answer{-\cos^4(x)/4} +C.\]
\end{problem}


\begin{problem}
\begin{hint}
Let $u = \tan(x)$
\end{hint}
\begin{hint}
Compute $du$
\end{hint}
\[\int \tan^5(x)\sec^2(x) \ dx = \answer{\tan^6(x)/6} +C.\]
\end{problem}

\begin{example} Compute 
\[\int \tan(x) \ dx.\]
First rewrite the integral:
\[\int \tan(x) \ dx =\int \frac{\sin(x)}{\cos(x)} \ dx.\]
Now, let $u = \cos(x)$; then $du = -\sin(x) \ dx$
and the integral can be written as
\begin{align*}
\int \frac{\sin(x)}{\cos(x)} \ dx &= \int \frac{1}{\cos(x)}\ \sin(x) \  dx \\
 &=  - \int \frac{1}{\cos(x)}\ \big(-\sin(x)\big) \  dx\\
&=-\int \frac{1}{u} \ du.
\end{align*}
The last integral can be computed as 
\[-\int \frac{1}{u} \ du = -\ln|u| + C\]
and by back substituting, we have 
\[-\ln|u| + C = -\ln|\cos(x)| + C = \ln|\sec(x)| +C.\]
Thus, using u-substitution we can conclude that
\[\int \tan(x) \ dx =  \ln|\sec(x)| + C.\]
\end{example}

\begin{problem}
\begin{hint}
Rewrite: $\cot(x) = \frac{\cos(x)}{\sin(x)}$
\end{hint}
\begin{hint}
Let $u = \sin(x)$
\end{hint}
\begin{hint}
Compute $du$
\end{hint}
\[\int \cot(x) \ dx = \answer{\ln|\sin(x)|} +C.\]


\end{problem}


\begin{example} Compute 
\[\int \sec(x) \ dx.\]
First rewrite the integral:
\[\int \sec(x) \ dx =\int \sec(x)\frac{\sec(x)+\tan(x)}{\sec(x)+\tan(x)} \ dx.\]
Distributing in the numerator, we get
\[\int \sec(x)\frac{\sec(x)+\tan(x)}{\sec(x)+\tan(x)} \ dx = \int \frac{\sec^2(x)+\sec(x)\tan(x)}{\sec(x)+\tan(x)} \ dx.\]
Now, let $u = \sec(x) + \tan(x)$, then $du = [\sec(x)\tan(x) + \sec^2(x)] \ dx$ and the integral can be rewritten as
\[ \int \frac{\sec^2(x)+\sec(x)\tan(x)}{\sec(x)+\tan(x)} \ dx = \int \frac{1}{u} \ du.\]
The last integral can be computed as
\[\int \frac{1}{u} \ du = \ln|u| + C\]
and by back substituting, we have 
\[\ln|u| + C = \ln|\sec(x) + \tan(x)| + C.\]
Thus, using u-substitution we can conclude that
\[\int \sec(x) \ dx =  \ln|\sec(x) + \tan(x)| + C.\]
\end{example}


\begin{center}
\begin{foldable}
\unfoldable{Here is a detailed, lecture style video on u-substitution:}
\youtube{5yb7_e9PCQU}
\end{foldable}
\end{center}

\end{document}
