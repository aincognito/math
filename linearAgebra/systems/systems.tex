\documentclass[handout]{ximera}

%% You can put user macros here
%% However, you cannot make new environments



\newcommand{\ffrac}[2]{\frac{\text{\footnotesize $#1$}}{\text{\footnotesize $#2$}}}
\newcommand{\vasymptote}[2][]{
    \draw [densely dashed,#1] ({rel axis cs:0,0} -| {axis cs:#2,0}) -- ({rel axis cs:0,1} -| {axis cs:#2,0});
}


\graphicspath{{./}{firstExample/}}
\usepackage{forest}
\usepackage{amsmath}
\usepackage{amssymb}
\usepackage{array}
\usepackage[makeroom]{cancel} %% for strike outs
\usepackage{pgffor} %% required for integral for loops
\usepackage{tikz}
\usepackage{tikz-cd}
\usepackage{tkz-euclide}
\usetikzlibrary{shapes.multipart}


%\usetkzobj{all}
\tikzstyle geometryDiagrams=[ultra thick,color=blue!50!black]


\usetikzlibrary{arrows}
\tikzset{>=stealth,commutative diagrams/.cd,
  arrow style=tikz,diagrams={>=stealth}} %% cool arrow head
\tikzset{shorten <>/.style={ shorten >=#1, shorten <=#1 } } %% allows shorter vectors

\usetikzlibrary{backgrounds} %% for boxes around graphs
\usetikzlibrary{shapes,positioning}  %% Clouds and stars
\usetikzlibrary{matrix} %% for matrix
\usepgfplotslibrary{polar} %% for polar plots
\usepgfplotslibrary{fillbetween} %% to shade area between curves in TikZ



%\usepackage[width=4.375in, height=7.0in, top=1.0in, papersize={5.5in,8.5in}]{geometry}
%\usepackage[pdftex]{graphicx}
%\usepackage{tipa}
%\usepackage{txfonts}
%\usepackage{textcomp}
%\usepackage{amsthm}
%\usepackage{xy}
%\usepackage{fancyhdr}
%\usepackage{xcolor}
%\usepackage{mathtools} %% for pretty underbrace % Breaks Ximera
%\usepackage{multicol}



\newcommand{\RR}{\mathbb R}
\newcommand{\R}{\mathbb R}
\newcommand{\C}{\mathbb C}
\newcommand{\N}{\mathbb N}
\newcommand{\Z}{\mathbb Z}
\newcommand{\dis}{\displaystyle}
%\renewcommand{\d}{\,d\!}
\renewcommand{\d}{\mathop{}\!d}
\newcommand{\dd}[2][]{\frac{\d #1}{\d #2}}
\newcommand{\pp}[2][]{\frac{\partial #1}{\partial #2}}
\renewcommand{\l}{\ell}
\newcommand{\ddx}{\frac{d}{\d x}}

\newcommand{\zeroOverZero}{\ensuremath{\boldsymbol{\tfrac{0}{0}}}}
\newcommand{\inftyOverInfty}{\ensuremath{\boldsymbol{\tfrac{\infty}{\infty}}}}
\newcommand{\zeroOverInfty}{\ensuremath{\boldsymbol{\tfrac{0}{\infty}}}}
\newcommand{\zeroTimesInfty}{\ensuremath{\small\boldsymbol{0\cdot \infty}}}
\newcommand{\inftyMinusInfty}{\ensuremath{\small\boldsymbol{\infty - \infty}}}
\newcommand{\oneToInfty}{\ensuremath{\boldsymbol{1^\infty}}}
\newcommand{\zeroToZero}{\ensuremath{\boldsymbol{0^0}}}
\newcommand{\inftyToZero}{\ensuremath{\boldsymbol{\infty^0}}}


\newcommand{\numOverZero}{\ensuremath{\boldsymbol{\tfrac{\#}{0}}}}
\newcommand{\dfn}{\textbf}
%\newcommand{\unit}{\,\mathrm}
\newcommand{\unit}{\mathop{}\!\mathrm}
%\newcommand{\eval}[1]{\bigg[ #1 \bigg]}
\newcommand{\eval}[1]{ #1 \bigg|}
\newcommand{\seq}[1]{\left( #1 \right)}
\renewcommand{\epsilon}{\varepsilon}
\renewcommand{\iff}{\Leftrightarrow}

\DeclareMathOperator{\arccot}{arccot}
\DeclareMathOperator{\arcsec}{arcsec}
\DeclareMathOperator{\arccsc}{arccsc}
\DeclareMathOperator{\si}{Si}
\DeclareMathOperator{\proj}{proj}
\DeclareMathOperator{\scal}{scal}
\DeclareMathOperator{\cis}{cis}
\DeclareMathOperator{\Arg}{Arg}
%\DeclareMathOperator{\arg}{arg}
\DeclareMathOperator{\Rep}{Re}
\DeclareMathOperator{\Imp}{Im}
\DeclareMathOperator{\sech}{sech}
\DeclareMathOperator{\csch}{csch}
\DeclareMathOperator{\Log}{Log}

\newcommand{\tightoverset}[2]{% for arrow vec
  \mathop{#2}\limits^{\vbox to -.5ex{\kern-0.75ex\hbox{$#1$}\vss}}}
\newcommand{\arrowvec}{\overrightarrow}
\renewcommand{\vec}{\mathbf}
\newcommand{\veci}{{\boldsymbol{\hat{\imath}}}}
\newcommand{\vecj}{{\boldsymbol{\hat{\jmath}}}}
\newcommand{\veck}{{\boldsymbol{\hat{k}}}}
\newcommand{\vecl}{\boldsymbol{\l}}
\newcommand{\utan}{\vec{\hat{t}}}
\newcommand{\unormal}{\vec{\hat{n}}}
\newcommand{\ubinormal}{\vec{\hat{b}}}

\newcommand{\dotp}{\bullet}
\newcommand{\cross}{\boldsymbol\times}
\newcommand{\grad}{\boldsymbol\nabla}
\newcommand{\divergence}{\grad\dotp}
\newcommand{\curl}{\grad\cross}
%% Simple horiz vectors
\renewcommand{\vector}[1]{\left\langle #1\right\rangle}


\outcome{In this section we solve linear systems of equations.}

\title{1.1 Linear Systems}

\begin{document}

\begin{abstract}
In this section we solve linear systems of equations.
\end{abstract}
 
\maketitle

A {\bf linear equation} in two variables is an equation of the form 
\[
ax + by = c,
\]
where $a, b$ and $c$ are constants and $a$ and $b$ are not both $0$.
Examples of linear equations include:
\[
2x+3y =5, 4x = 7 \;\text{and}\; y = 3x -2
\]

The graph of a linear equation in two variables is a straight line. The graphs of the three linear equations given above are shown below.


\begin{image}
\begin{tikzpicture}
\draw[ ->] (-1,0) -- (3.6, 0) node[right]{$x$};
\draw[ ->] (0,-1) -- (0, 3.6) node[above]{$y$};
\draw[<->,blue] (1.75,-1) -- (1.75,3) node[right]{$4x=7$};
\draw[<->,red] (-1,2.33) -- (3, -0.33) node[right]{$2x+3y=5$};
\draw[<->,green] (1,1) -- (2, 4) node[right]{$y = 3x-2$};
\end{tikzpicture}
\end{image}



\begin{image}
\begin{tikzpicture}
\draw[->] (-1,0) -- (3.6, 0) node[right]{$x$};
\draw[->] (0,-1) -- (0, 3.6) node[above]{$y$};
\draw[<->,blue] (1.75,-1) -- (1.75,4.3) node[left]{$4x=7$};
\draw[<->,red] (-1,2.33) -- (3, -0.33) node[right]{$2x+3y=5$};
\draw[<->,green] (0.5,-0.5) -- (2.25, 4.75) node[right]{$y = 3x-2$};
\end{tikzpicture}
\end{image}


A {\bf system} of linear equations refers to more than one linear equation.
The following is a system of 2 linear equations in 2 variables:
\begin{align}
2x+3y &= 5\\
y &= 3x-2
\end{align}

{\bf Solving} a system of linear equations means finding all coordinate pairs $(x_0, y_0)$ 
which satisfy all of the equations in the system. For example, the coordinate pair $(2,4)$ 
satisfies the second equation in the above system, but not the first- so it 
is {\bf not} a solution to the above system. However, the point $(1,1)$ 
satisfies {\bf both} of the equations in the above system, and so it {\bf is} a 
solution to the system.  Notice that this solution is represented geometrically by the 
point of intersection of the two lines that make up the system (see the figure above). 
Our goal is to systematically produce all of the solutions of a given linear system, if there are any.

Here is an example of the elimination method created by Chatgpt(!).\\

Solve the linear system:

\begin{align*}
x + 2y &= 5 \\
3x - 5y &= 11
\end{align*}

To eliminate the variable $x$, we can multiply the first equation by 3 and the second equation by 1:

\begin{align*}
3(x + 2y) &= 3(5) \\
1(3x - 5y) &= 1(11)
\end{align*}

Simplifying, we get:

\begin{align*}
3x + 6y &= 15 \\
3x - 5y &= 11
\end{align*}

Next, we subtract the second equation from the first equation to eliminate the variable $x$:

\begin{align*}
(3x + 6y) - (3x - 5y) &= 15 - 11 \\
3x + 6y - 3x + 5y &= 4 \\
11y &= 4 \\
y &= \frac{4}{11}
\end{align*}

Substituting the value of $y$ back into the first equation:

\begin{align*}
x + 2\left(\frac{4}{11}\right) &= 5 \\
x + \frac{8}{11} &= 5 \\
x &= 5 - \frac{8}{11} \\
x &= \frac{27}{11}
\end{align*}

Therefore, the solution to the system of linear equations is $x = \frac{27}{11}$ and $y = \frac{4}{11}$.


We can represent the given system of linear equations using an augmented matrix:

\[
\begin{bmatrix}
1 & 2 & \vert & 5 \\
3 & -5 & \vert & 11 \\
\end{bmatrix}
\]

We'll perform row operations to transform the augmented matrix into reduced row-echelon form:

\[
\begin{array}{ccc|c}
R_1 & R_2 &  &  \\
\hline
1 & 2 & \vert & 5 \\
3 & -5 & \vert & 11 \\
\end{array}
\]

We'll start by replacing $R_2$ with $R_2 - 3R_1$ to eliminate the coefficient of $x$ in the second equation:

\[
\begin{array}{ccc|c}
R_1 & R_2 &  &  \\
\hline
1 & 2 & \vert & 5 \\
0 & -11 & \vert & -4 \\
\end{array}
\]

Next, we'll divide $R_2$ by $-11$ to make the leading coefficient $1$:

\[
\begin{array}{ccc|c}
R_1 & R_2 &  &  \\
\hline
1 & 2 & \vert & 5 \\
0 & 1 & \vert & \frac{4}{11} \\
\end{array}
\]

Now, we'll replace $R_1$ with $R_1 - 2R_2$ to eliminate the coefficient of $y$ in the first equation:

\[
\begin{array}{ccc|c}
R_1 & R_2 &  &  \\
\hline
1 & 0 & \vert & \frac{27}{11} \\
0 & 1 & \vert & \frac{4}{11} \\
\end{array}
\]

The augmented matrix is now in reduced row-echelon form. The solution to the system of linear equations is $x = \frac{27}{11}$ and $y = \frac{4}{11}$.



We can represent the given system of linear equations using augmented matrices:

\[
\begin{bmatrix}
1 & 2 \\
3 & -5 \\
\end{bmatrix}
\begin{bmatrix}
x \\
y \\
\end{bmatrix}
=
\begin{bmatrix}
5 \\
11 \\
\end{bmatrix}
\]

We'll perform row operations to transform the augmented matrix into reduced row-echelon form:

\[
\begin{bmatrix}
1 & 2 \\
3 & -5 \\
\end{bmatrix}
\begin{bmatrix}
x \\
y \\
\end{bmatrix}
=
\begin{bmatrix}
5 \\
11 \\
\end{bmatrix}
\]

We'll start by replacing the second row with the second row minus 3 times the first row:

\[
\begin{bmatrix}
1 & 2 \\
0 & -11 \\
\end{bmatrix}
\begin{bmatrix}
x \\
y \\
\end{bmatrix}
=
\begin{bmatrix}
5 \\
-4 \\
\end{bmatrix}
\]

Next, we'll divide the second row by $-11$ to make the leading coefficient $1$:

\[
\begin{bmatrix}
1 & 2 \\
0 & 1 \\
\end{bmatrix}
\begin{bmatrix}
x \\
y \\
\end{bmatrix}
=
\begin{bmatrix}
5 \\
-\frac{4}{11} \\
\end{bmatrix}
\]

Now, we'll replace the first row with the first row minus 2 times the second row:

\[
\begin{bmatrix}
1 & 0 \\
0 & 1 \\
\end{bmatrix}
\begin{bmatrix}
x \\
y \\
\end{bmatrix}
=
\begin{bmatrix}
\frac{27}{11} \\
\frac{4}{11} \\
\end{bmatrix}
\]

The augmented matrix is now in reduced row-echelon form. The solution to the system of linear equations is $x = \frac{27}{11}$ and $y = \frac{4}{11}$.


We can represent the given system of linear equations using matrices:

\[
\begin{bmatrix}
1 & 2 \\
3 & -5 \\
\end{bmatrix}
\begin{bmatrix}
x \\
y \\
\end{bmatrix}
=
\begin{bmatrix}
5 \\
11 \\
\end{bmatrix}
\]

We'll perform row operations to transform the augmented matrix into reduced row-echelon form:

\[
\left[
\begin{array}{cc|c}
1 & 2 & 5 \\
3 & -5 & 11 \\
\end{array}
\right]
\]

We'll start by replacing the second row with the second row minus 3 times the first row:

\[
\left[
\begin{array}{cc|c}
1 & 2 & 5 \\
0 & -11 & -4 \\
\end{array}
\right]
\]

Next, we'll divide the second row by $-11$ to make the leading coefficient $1$:

\[
\left[
\begin{array}{cc|c}
1 & 2 & 5 \\
0 & 1 & \frac{4}{11} \\
\end{array}
\right]
\]

Now, we'll replace the first row with the first row minus 2 times the second row:

\[
\left[
\begin{array}{cc|c}
1 & 0 & \frac{27}{11} \\
0 & 1 & \frac{4}{11} \\
\end{array}
\right]
\]

The augmented matrix is now in reduced row-echelon form. The solution to the system of linear equations is:

\[
\begin{aligned}
x &= \frac{27}{11} \\
y &= \frac{4}{11} \\
\end{aligned}
\]

\end{document}

\end{document}

\end{document}




It is not necessary to place a vector in standard position.  
In practical applications of vectors, it is typical to place the initial point of a vector in a position of interest.  
For example, if the vector represents the velocity of an object, then one would likely choose the initial point to be the location of the object.
Below is a graphical representation of the vector, $\vector{2,2}$ placed in several different positions.
Since each of the arrows below are the same length and point in the same direction, they represent the same vector.
Notice that in each copy of the vector, the final point is two units to the right and two units up from the initial point. 
\begin{image}
\begin{tikzpicture}
%\draw[step=1.0,black,thin] (-2.5,-2.5) grid (2.5,2.5);
%\draw[thick, <->] (-2.5,0) -- (2.5, 0) node[right]{$x$};
%\draw[thick, <->] (0,-2.5) -- (0, 2.5) node[above]{$y$};
%\draw[->, blue] (0,0) -- (1, 1) ;
\draw[->, blue] (-0.5, -0.5) -- (0.5, 0.5);
\draw[->, blue] (1.5, 1.5) -- (2.5, 2.5);
\draw[->, blue] (-2.5, -2.5) -- (-1.5,-1.5);
\draw[->, blue] (-0.5, 1.5) -- (0.5, 2.5);
\draw[->, blue]  (-2.5, -0.5)-- (-1.5, 0.5);
\draw[->, blue] (-2.5, 1.5) -- (-1.5, 2.5);
\draw[->, blue] (1.5, -0.5) -- (2.5, 0.5);
\draw[->, blue]  (-0.5, -2.5)-- (0.5,-1.5);
\draw[->, blue] (1.5, -2.5) -- (2.5, -1.5);
%\draw[->] (-1,0)--(0,1);
\draw[->, blue] (-0.5, 1.5) -- (0.5, 2.5);
%\draw[->, blue] (-2,0) -- (-1, 1) ;
%\draw[->, blue] (0,1) -- (1, 2) ;
%\draw[->, blue] (-2,1) -- (-1, 2) ;
%\draw[->, blue] (1,-2) -- (2, -1) ;
%\draw[->, blue] (0,-1) -- (1, 0) ;
%\draw[->, blue] (-2,-2) -- (-1, -1) ;
%\draw[->, blue] (1,-1) -- (2, 0) ;
%\node at (2.2, -0.3){$2$};
%\node at (-1.8, -0.3){$-2$};
%\node at (-0.3,2){$2$};
%\node at (-0.3, -2){$-2$};
\draw[step= 0.5 cm,color=gray] (-3,-3) grid (3,3);
\node at (0, -3.5) {Multiple copies of the vector $\vector{2,2}$};
\end{tikzpicture}
\end{image}



A special vector is the zero vector, $\vec{0} = \vector{0,0}$.  
This vector has magnitude zero and no direction.  
It is represented graphically by a single point.

\begin{image}
\begin{tikzpicture}
\draw[ <->] (-2,0) -- (2, 0) node[right]{$x$};
\draw[<->] (0,-2) -- (0, 2) node[above]{$y$};
\draw[blue, fill] (0,0) circle (0.03) node[below right]{$\vec{0}$};
%\draw[thin] (1, 0.15) -- (1, -0.15) node[below]{$1$};
%\draw[thin] (0.15,1) -- (-0.15,1) node[left]{$1$};
\end{tikzpicture}
\end{image}

The vector with initial point $P(x_1, y_1)$ and final point $Q(x_2, y_2)$ has components $\avec{PQ} = \vector{x_2 - x_1, y_2-y_1}$


\begin{image}
\begin{tikzpicture}
\draw[ <->] (-2,0) -- (2, 0) node[right]{$x$};
\draw[ <->] (0,-1) node[below]{The vector $\avec{PQ} = \vector{x_2 - x_1, y_2-y_1}$} -- (0, 2) node[above]{$y$};
\draw[->, blue] (-1,0.5) node[left]{$P(x_1,y_1)$} -- (1.5, 1.5) node[right]{$Q(x_2,y_2)$};
\draw[blue, fill] (-1, 0.5) circle (0.03);
%\draw[thin] (2, 0.15) -- (2, -0.15) node[below]{$1$};
%\draw[thin] (0.15,2) -- (-0.15,2) node[left]{$1$};
\end{tikzpicture}
\end{image}

\begin{example}[Example 1]
Find the components of the vector from the point $P(-2, 3)$ to the point $Q(1, 1)$ and draw two copies of this vector- the first from the point $P$ to the point $Q$
and the second in standard position.\\
The components of $\avec{PQ}$ are $\vector{1-(-2), 1-3} = \vector{3, -2}$.
The two copies of the vector are shown below.

\begin{image}
\begin{tikzpicture}
\draw[step=1.0,black,thin] (-3.5,-3.5) grid (3.5,3.5);
\draw[thick, <->] (-3.5,0) -- (3.5, 0) node[right]{$x$};
\draw[thick, <->] (0,-3.5) -- (0, 3.5) node[above]{$y$};
\draw[->, blue, thick] (0,0) -- (3, -2) ;
\draw[blue, fill] (0, 0) circle (0.03);
\draw[->, blue, thick] (-2,3) node[above left]{$P$} -- (1, 1) node[right]{$Q$} ;
\draw[blue, fill] (-2, 3) circle (0.03);
\node at (2.1, -0.3){$2$};
\node at (-2, -0.3){$-2$};
\node at (-0.3,2){$2$};
\node at (-0.3, -2){$-2$};
\node at (0, -3.8) {Two copies of the vector $\avec{PQ} = \vector{3,-2}$};
\end{tikzpicture}
\end{image}

\end{example}

\begin{problem}(Problem 1)
Find the components of the vector from $P(-3, 2)$ to the point $Q(2, -1)$ and draw two copies of this vector- the first from the point $P$ to the point $Q$
and the second in standard position.\\
\end{problem}


\subsection{Magnitude of a Vector}

The magnitude of a vector $\vec{v} = \vector{x,y}$ in $\R^2$ is given by the distance formula:
\[
|\vec{v}| = |\vector{x,y}| = \sqrt{x^2 +y^2}
\]

\begin{example}[Example 2]
Find the magnitude of the vector $\vector{1,1}$.\\
Its magnitude is
\[
|\vector{1,1}| = \sqrt{1^2 +1^2} = \sqrt 2
\]
\end{example}

\begin{problem}(Problem 2)
Find the magnitudes of the following vectors:\\
a) $|\vector{2,3}| = \answer{\sqrt{13}}$\\
b) $|\vector{-4,2}| = \answer{2\sqrt{5}}$\\
c) $|\vector{12,-5}| = \answer{13}$
\end{problem}

The magnitude of the zero vector is $0$ and it is the only vector with magnitude zero
\[
|\vec{0}| = |\vector{0,0}| = \sqrt{0^2 +0^2} = 0
\]

\section{Vector Operations}
In this section, we learn about two fundamental operations involving vectors: scalar multiplication and addition.
The more advanced dot product and cross product operations will be handled in later sections.

\subsection{Scalar Multiplication}
Given a vector $\vec{v} = \vector{x,y}$ in $\R^2$ and a scalar $c$ in $\R$, we can multiply the scalar and the vector in the following
natural way
\[
c\vec{v} = c\vector{x,y} = \vector{cx,cy}
\]
In words, we multiply a vector by a scalar by multiplying each component by the scalar.
The first thing to note is that if the scalar is 0, then the resulting vector will be the zero vector:
\[
0\vec{v} = 0\vector{x,y} = \vector{0,0} = \vec{0}
\]

The effect of scalar multiplication on the magnitude of a vector is given in the following proposition.
\begin{proposition}[Scaling]
Given a vector $\vec{v} = \vector{x,y}$ in $\R^2$ and a scalar $c$ in $\R$,
\[
|c\vec{v}| = |c| |\vec{v}|
\]
\begin{proof}
Write $\vec{v}$ in component form as $\vec{v} = \vector{x,y}$, then
\begin{align*}
|c\vec{v}| &= |c\vector{x,y}|\\
            &= |\vector{cx,cy}|\\
            &= \sqrt{(cx)^2 + (cy)^2}\\
            &= \sqrt{c^2x^2 + c^2y^2}\\
            &= \sqrt{c^2(x^2 + y^2)}\\
            &= \sqrt{c^2} \sqrt{x^2 + y^2}\\
            &= |c| |\vec{v}|
\end{align*}
\end{proof}
\end{proposition}

When $c>0$, the vector $c\vec{v}$ has the same direction as $\vec{v}$, and when $c<0$, the
vector $c\vec{v}$ has the opposite direction as $\vec{v}$, as illustrated below.



\begin{image}
\begin{tikzpicture}
\draw[blue, thick, ->] (-3,0) -- (-2,0) node[right]{$\vec{v}$};
\draw[red, thick, <-] (-3,-1) node[left ]{$-\vec{v}$} -- (-2,-1) ;
\draw[blue, thick, ->] (0,0) -- (2,0) node[right]{$2\vec{v}$};
\draw[red, thick, <-] (0,-1) node[left]{$-2\vec{v}$} -- (2,-1);
\draw[blue, thick, ->] (4,0) -- (4.5,0) node[right]{$\tfrac12\vec{v}$};
\draw[red, thick, <-] (4,-1)node[left]{$-\tfrac12\vec{v}$} -- (4.5,-1) ;
\node at (0.75, -2) {Multiplying $\vec{v}$ by positive and negative scalars};
%\draw[white] (0,-4) -- (6, -4);
\end{tikzpicture}
\end{image}

\subsection{Unit Vectors}

A vector with magnitude 1 is called a {\bf unit vector}. Two special unit vectors in $\R^2$ are the {\bf standard basis} vectors:
\[
\vec{i} = \vector{1,0} \quad \text{and} \quad \vec{j} = \vector{0,1}
\]

The vectors $\vec{i}$ and $\vec{j}$ are shown below.

\begin{image}
\begin{tikzpicture}
\draw[ <->] (-1.5,0) -- (1.8, 0) node[right]{$x$};
\draw[<->] (0,-1.5) -- (0,1.8) node[above]{$y$};
\draw[thick,->, blue!70!white] (0,0) -- (1, 0) node[midway, below ]{$\vec{i}$};
\draw[thick,->, red!70!white] (0,0) -- (0,1) node[midway, left]{$\vec{j}$};
\draw[blue, fill] (0,0) circle (0.01);
\draw[thin] (1, 0.1) -- (1, -0.1) node[below]{$1$};
\draw[thin] (0.1,1) -- (-0.1,1) node[left]{$1$};
\node at (0, -2) {The standard basis vectors in $\R^2$};
\draw[white] (-1, -2.5) -- (1, -2.5);
\end{tikzpicture}
\end{image}

For any angle $\theta$, the identity 
\[
\cos^2 \theta + \sin^2 \theta = 1
\]
 implies that the vector 
 \[
 \vector{\cos \theta, \sin \theta}
 \]
is a unit vector. In $\R^2$, the angle $\theta$ between the positive $x$-axis and a vector placed in standard position can be used to describe the direction of the vector.  
For this reason, unit vectors are also referred to as {\bf direction vectors}.

\begin{image}
\begin{tikzpicture}
\draw[ <->] (-2,0) -- (2, 0) node[right]{$x$};
\draw[<->] (0,-2) -- (0,2) node[above]{$y$};
\draw (0,0) circle (1.5);
\draw[blue!70, ->] (0,0) -- (-1.06, 1.06)  node[above, left]{$\vector{\cos \theta, \sin\theta}$};
\draw[blue!70] (.2,0) arc (0:135:0.2) node[ midway, right]{$\theta$};
\node at (0, -2.5) {A unit or {\it direction} vector};
\end{tikzpicture}
\end{image}

Given a vector $\vec{v}$, we will want to find a unit vector in the direction of $\vec{v}$.
This unit vector will have the form $c\vec{v}$ for some positive scalar $c$.  
\begin{proposition}[Unit Vectors]
The unit vector $\vec{u}$ in the same direction as the vector $\vec{v} \neq \vec{0}$ is given by 
\[
\vec{u} = \frac{1}{|\vec{v}|} \vec{v}
\]
\begin{proof}
Since $|\vec{v}| \neq 0$, the scalar $\displaystyle \frac{1}{|\vec{v}|}$ is a positive real number.  Thus, the vector $\vec{u}$ points 
in the same direction as $\vec{v}$, since it is a positive multiple of $\vec{v}$ . We must now show that $|\vec{u}| = 1$.  
By the previous proposition, we have
\begin{align*}
|\vec{u}| &= \left| \frac{1}{|\vec{v}|} \vec{v} \right|\\
           &= \left| \frac{1}{|\vec{v}|} \right| |\vec{v}|\\
           &= \frac{1}{|\vec{v}|} |\vec{v}|\\
           &= 1
\end{align*}
\end{proof}
\end{proposition}

\begin{example}[Example 3]
Find a unit vector in the same direction as the vector $\vec{v} = \vector{3, -4}$.\\
First we find the magnitude of $\vec{v}$:
\[
|\vec{v}| = |\vector{3, -4}| = \sqrt{3^2 + (-4)^2} = \sqrt{25} = 5
\]
By the proposition above, we have
\[
\vec{u} = \frac{1}{|\vec{v}|} \vec{v} = \frac15 \vector{3, -4} = \vector{\frac35, -\frac45}
\]
where $\vec{u}$ is a unit vector in the same direction as $\vec{v}$.
\end{example}

\begin{problem}(Problem 3a)
Find a unit vector in the same direction as the vector $\vec{v} = \vector{-8, 15}$.\\
$\vec{u} = \vector{\answer{-8/17}, \answer{15/17}}$
\end{problem}

\begin{problem}(Problem 3b)
Find a unit vector in the opposite direction of the vector $\vec{v} = \vector{9, -12}$.\\
$\vec{u} = \vector{\answer{-3/5}, \answer{4/5}}$
\end{problem}

Every non-zero vector in $\R^2$ can be written in terms of its magnitude and a direction angle, i.e. a polar form.

\begin{proposition}[Polar Form]
Let $\vec{v} = \vector{x,y} \neq \vector{0,0}$.  Then there is an angle $\theta$ such that $\vec{v} = |\vec{v}| \vector{\cos \theta, \sin \theta}$.
\begin{proof}
%Since $\cos^2\theta + \sin^2\theta = 1$ for any angle $\theta$, the vector $\vector{\cos \theta, \sin \theta}$ is a unit vector.
By the previous proposition, the vector $|\vec{v}| \vector{\cos \theta, \sin \theta}$ has magnitude $|\vec{v}|$. 
Now, if $\vec{v}$ is drawn in standard position, and if $\theta$ is the (counterclockwise) angle between $\vec{v}$ and the positive $x$-axis, 
then 
\[
\cos \theta = \frac{x}{|\vec v|} \quad \text{and} \quad \sin \theta = \frac{y}{|\vec v|}
\]
Hence, the components of $\vec{v}$ are 
\[
x = |\vec{v}|\cos \theta \quad \text{and} \quad y = |\vec{v}|\sin \theta, 
\]
which means that
\[
\vec v = \vector{x,y} = \vector{|\vec{v}|\cos\theta,|\vec{v}|\sin\theta} 
\]
Factoring out $|\vec v|$ we obtain
\[
\vec v = |\vec{v}|\vector{\cos\theta,\sin\theta}
\]
as desired. See the figure below.
\begin{image}
\begin{tikzpicture}
%\draw[step=1.0,black,thin] (-3.5,-3.5) grid (3.5,3.5);
\draw[thick, <->] (-3.5,0) -- (3.5, 0) node[right]{$x$};
\draw[thick, <->] (0,-3) -- (0, 3.5) node[above]{$y$};
\draw[->, blue, thick] (0,0) -- (-2.5, 2.5) node[above]{$\vec{v} = \vector{|\vec{v}|\cos\theta,|\vec{v}|\sin\theta}$};
\draw[blue, fill] (0, 0) circle (0.03);
\draw[red] (0,0) circle (1);
\draw[red, fill, ->] (-.707, .707) circle (0.03) node[above, left]{$(\cos \theta, \sin\theta)$};
\draw (.3,0) arc (0:135:0.3) node[midway, above right]{$\theta$};
\node at (2.1, -0.3){$2$};
\node at (-2, -0.3){$-2$};
\node at (-0.3,2){$2$};
\node at (-0.3, -2){$-2$};
\node[red] at (1.3, -1.3) {$x^2 + y^2 = 1$};
\node at (0, -3.4) {Polar form: $\vec{v} = |\vec{v}| \vector{\cos \theta, \sin \theta}$};
\end{tikzpicture}
\end{image}


\end{proof}
\end{proposition}

\begin{example}[Example 4]
Find the polar form of the vectors $-2\vec{i}$ and $4\vec{i}-7\vec{j}$.\\
The angle associated with the vector $-2\vec{i}$ is $\pi$ since the vector points in the direction of the negative $x$-axis.
So $-2\vec{i} = |-2\vec{i}| \vector{\cos \pi, \sin \pi} = 2\vector{\cos \pi, \sin \pi} (= 2\vector{-1, 0} = \vector{-2, 0})$.\\
The angle associated with the vector $4\vec{i}-7\vec{j}$ is $\theta = \tan^{-1}(-7/4)$, which is not a well known angle.
The magnitude of $4\vec{i}-7\vec{j}$ is $\sqrt{16 +49} = \sqrt{65}$.  Hence, the polar form of $4\vec{i}-7\vec{j}$ is
$\sqrt{65}\vector{\cos \theta, \sin\theta}$ where $\theta$ is as above. Note that if $\vec{v}$ is in the second or third quadrant, then 
to find the angle we would add $\pi$ to the inverse tangent.
\end{example}

\begin{problem}(Problem 4)
Find the polar form of the vectors $-3\vec{j}$ and $-2\vec{i}+5\vec{j}$.\\
\end{problem}

\subsection{Vector Addition}
Vector addition is done component-wise.  If $\vec{v}_1 = \vector{x_1, y_1}$ and $\vec{v}_2 = \vector{x_2, y_2}$
then their sum is given by
\[
\vec{v}_1+ \vec{v}_2= \vector{x_1+x_2,  y_1+y_2}
\]
Vector subtraction is addition of the negative which results in component-wise subtraction:
\[
\vec{v}_1- \vec{v}_2= \vec{v}_1+ \left(- \vec{v}_2\right)  = \vector{x_1,y_1} + \vector{-x_2, -y_2} = \vector{x_1-x_2,  y_1-y_2}
\]

\begin{example}[Example 5]
Let $\vec{u} = \vector{-4, 2}$ and $\vec{v} = \vector{5, -7}$.  Compute each of the following:\\
a) $\vec{u}+\vec{v} = \vector{1, -5}$\\
b) $\vec{u}-\vec{v} = \vector{-9, 9}$\\
c) $\vec{v}-\vec{u} = \vector{9, -9}$\\
d) $2\vec{u}+3\vec{v} = \vector{7, -17}$\\
\end{example}

\begin{problem}(Problem 5)
Let $\vec{u} = \vector{2, -3}$ and $\vec{v} = \vector{-1, 6}$.  Compute each of the following:\\
a) $\vec{u}+\vec{v} = \vector{\answer{1}, \answer{3}}$\\
b) $\vec{u}-\vec{v} = \vector{\answer{3}, \answer{-9}}$\\
c) $\vec{v}-\vec{u} = \vector{\answer{-3}, \answer{9}}$\\
d) $3\vec{u}+5\vec{v} = \vector{\answer{1}, \answer{21}}$\\
\end{problem}

Graphically, vector addition can be performed using either the parallelogram method or the end to end method as pictured below.
\begin{image}
\begin{tikzpicture}
\draw[blue, ->, thick] (0,0) -- (3, 1) node[midway, below]{$\vec{u}$};
\draw[blue, ->, thick] (0,0) -- (1, 4) node[midway, left]{$\vec{v}$};
\draw[blue, thin, dashed] (3, 1) -- (4, 5);
\draw[blue, thin, dashed] (1, 4) -- (4, 5);
\draw[red, ->, thick] (0,0) -- (4, 5) node[above]{$\vec{u} + \vec{v}$};
\node at (1.5, -1){Parallelogram Method};
\draw[blue, ->, thick] (6,0) -- (9, 1) node[midway, below]{$\vec{u}$};
\draw[blue, ->, thick] (9,1) -- (10, 5) node[midway, right]{$\vec{v}$};
\draw[red, ->, thick] (6,0) -- (10, 5) node[above]{$\vec{u} + \vec{v}$};
\node at (7.5, -1){End to End Method};
\end{tikzpicture}
\end{image}

The difference $\vec{u} - \vec{v}$ is the vector from $\vec{v}$ to $\vec{u}$ as shown below.
\begin{image}
\begin{tikzpicture}
\draw[blue, ->, thick] (0,0) -- (3, 1) node[midway, below]{$\vec{u}$};
\draw[blue, ->, thick] (0,0) -- (1, 4) node[midway, left]{$\vec{v}$};
\draw[red, ->, thick] (1,4) -- (3, 1) node[midway,right]{$\vec{u} - \vec{v}$};
\node at (1.5, -1){The vector $\vec{u} - \vec{v}$};
\draw[white] (-3, -1.5) -- (7, -1.5);
\end{tikzpicture}
\end{image}
Note that in the figure above, if we add the vectors $\vec{v}$ and $\vec{u} -\vec{v}$ using the end to end method, 
we get the vector $\vec{u}$, i.e.,
\[
\vec{v} + \left(\vec{u} -\vec{v}\right) = \vec{u}
\]

Vectors in $\R^2$ can be written using the {\bf standard basis} vectors $\vec{i} = \vector{1,0}$ and $\vec{j} = \vector{0,1}$.  
Given a vector $\vec{v} = \vector{x,y}$, we can use the addition and scalar multiplication operations to rewrite it as
\begin{align*}
\vec{v} &= \vector{x,y}\\
         &= \vector{x,0} + \vector{0,y}\\
         &= x\vector{1,0} + y \vector{0,1}\\
         &= x\vec{i} + y\vec{j}
\end{align*}



\section{Application}
Forces acting on an object have both magnitude and a direction and hence they can be modeled using vectors.

\begin{example}[Example 6]
A box on the floor is being pulled by a rope. The weight of the box is $20$ Newtons, the force due to friction is $8$ Newtons, and the 
force on the rope is $24$ Newtons at an angle of $30^\circ$ to the floor. Find the normal force, $\vec{n}$, acting on the box from the 
floor and resultant force acting on the box, $\vec{F}$.\\
\begin{image}  
\begin{tikzpicture}
\draw (0,0) -- (6,0) node[right]{floor};
\draw (1.5, 0) -- (1.5, 2) -- (4.5, 2) -- (4.5, 0) node[midway, right]{box};
\draw[blue!70!white, ->] (3,0) -- (3, -1) node[below]{weight = $20$N};
\draw[blue!70!white, <-] (1, 1) node[left]{friction = $8$N} -- (1.5,1);
\draw[blue!70!white, ->] (4.5, 2) -- (6, 3) node[right]{rope force = $24$N};
\draw[blue!70!white, ->] (3,2) -- (3, 2.5) node[above]{normal force, $\vec{n}$};
\draw[blue!70!white, dashed] (4.5, 2) -- (5.7, 2) ;
\node[blue!70!white] at (5.3,2.2){$30^\circ$};
\end{tikzpicture}
\end{image}


The resultant force on the box is the sum of the forces acting on the box, where each force is represented by a vector.
The force of the weight of the box has a magnitude of $20$ and points in the direction of the negative $y$-axis.  
This vector is $\vec{w} = \vector{0,-20}$.
The friction force has a magnitude of $8$ and points in the direction of the negative $x$-axis.  This vector is $\vec{f} = \vector{-8,0}$.
The force on the rope has a magnitude of $24$ and points in the direction $30^\circ$ above the $x$-axis. 
Using trigonometry, we can deduce that the horizontal and vertical components of this force vector are $x = 24\cos(30^\circ) = 12\sqrt 3$
and $y = 24\sin(30^\circ) = 12$ respectively. Hence, this force is $\vec r = \vector{12\sqrt 3, 12}$.  See the figure below.

\begin{image}
\begin{tikzpicture}
\draw(0,0) -- (6.8,0) node[midway, below]{$24 \cos(30^\circ)$} ;
\draw (6.8, 0) -- (6.8,4) node[midway,right]{$24\sin(30^\circ)$};
\draw[blue!70!white, ->] (0,0) -- (6.8, 4) node[midway, above left]{$|\vec{r}| = 24$};
\node at (1, 0.25) {$30^\circ$};
\node at (3.4, -1) {The force on the rope is $\vec r = \vector{12\sqrt 3, 12}$};
\draw (6.5, 0) -- (6.5, 0.3) -- (6.8, 0.3);
\end{tikzpicture}
\end{image}

Assuming that the vertical component of the rope force is not enough to overcome the weight of the box (and thus send it airborne), 
the floor will exert an upward normal force on the box.  This force will have the form $\vec{n} = \vector{0, s}$ where the positive 
number $s$ will be determined by the fact that the net vertical component of all forces acting on the box is zero. 
The resultant force, $\vec{F}$, on the box is the sum of the vectors $\vec{f}, \vec{w}, \vec{n}$ and $\vec{r}$:
\begin{align*}
\vec{F} &= \vec{f} + \vec{w} + \vec{n} +  \vec{r} \\
         &= \vector{-8,0} + \vector{0,-20} +\vector{0,s} + \vector{12\sqrt 3, 12} \\
         &= \vector{12\sqrt 3 -8 , s-8}
\end{align*}
Since the vertical component of the resultant force $\vec{F}$ should be zero, we see that $s = 8$. 
Thus $\vec{n} = \vector{0,8} = 8\vec{j}$ so that $\vec{F} = \vector{12\sqrt 3 -8, 0} \approx \vector{12.785,0} = 12.785\vec{i}$. Because of this force, 
the box will slide along the floor to the right.

% at a speed equal to the magnitude of this force: 
%\[
%|\vec{F}| = \left|\vector{12\sqrt 3 -8, 0}\right| = 12\sqrt 3 - 8 \approx 12.785\, \text{meters/second}
%\]




\end{example}
\end{document}








