\documentclass[handout]{ximera}

%% You can put user macros here
%% However, you cannot make new environments



\newcommand{\ffrac}[2]{\frac{\text{\footnotesize $#1$}}{\text{\footnotesize $#2$}}}
\newcommand{\vasymptote}[2][]{
    \draw [densely dashed,#1] ({rel axis cs:0,0} -| {axis cs:#2,0}) -- ({rel axis cs:0,1} -| {axis cs:#2,0});
}


%\usepackage{tcolorbox} %%Needed for Derivative Definition supposedly and product rule, natural exp log, quotient rule, inverse trig, rates of change


% \graphicspath{{./}{firstExample/}}
% \usepackage{forest}
\usepackage{amsmath}
\usepackage{amssymb}
\usepackage{array}
\usepackage[makeroom]{cancel} %% for strike outs
\usepackage{pgffor} %% required for integral for loops
\usepackage{tikz}
\usepackage{tikz-cd}
\usepackage{tkz-euclide}
\usetikzlibrary{shapes.multipart}


% \usetkzobj{all}
\tikzstyle geometryDiagrams=[ultra thick,color=blue!50!black]


\usetikzlibrary{arrows}
\tikzset{>=stealth,commutative diagrams/.cd,
  arrow style=tikz,diagrams={>=stealth}} %% cool arrow head
\tikzset{shorten <>/.style={ shorten >=#1, shorten <=#1 } } %% allows shorter vectors

\usetikzlibrary{backgrounds} %% for boxes around graphs
\usetikzlibrary{shapes,positioning}  %% Clouds and stars
\usetikzlibrary{matrix} %% for matrix
\usepgfplotslibrary{polar} %% for polar plots
\usepgfplotslibrary{fillbetween} %% to shade area between curves in TikZ



%\usepackage[width=4.375in, height=7.0in, top=1.0in, papersize={5.5in,8.5in}]{geometry}
%\usepackage[pdftex]{graphicx}
%\usepackage{tipa}
%\usepackage{txfonts}
%\usepackage{textcomp}
%\usepackage{amsthm}
%\usepackage{xy}
%\usepackage{fancyhdr}
%\usepackage{xcolor}
%\usepackage{mathtools} %% for pretty underbrace % Breaks Ximera
%\usepackage{multicol}



\newcommand{\RR}{\mathbb R}
\newcommand{\R}{\mathbb R}
\newcommand{\C}{\mathbb C}
\newcommand{\N}{\mathbb N}
\newcommand{\Z}{\mathbb Z}
\newcommand{\dis}{\displaystyle}
%\renewcommand{\d}{\,d\!}
\renewcommand{\d}{\mathop{}\!d}
\newcommand{\dd}[2][]{\frac{\d #1}{\d #2}}
\newcommand{\pp}[2][]{\frac{\partial #1}{\partial #2}}
\renewcommand{\l}{\ell}
\newcommand{\ddx}{\frac{d}{\d x}}
\newcommand{\ppx}{\frac{\partial}{\partial x}}
\newcommand{\ppy}{\frac{\partial}{\partial y}}

\newcommand{\zeroOverZero}{\ensuremath{\boldsymbol{\tfrac{0}{0}}}}
\newcommand{\inftyOverInfty}{\ensuremath{\boldsymbol{\tfrac{\infty}{\infty}}}}
\newcommand{\zeroOverInfty}{\ensuremath{\boldsymbol{\tfrac{0}{\infty}}}}
\newcommand{\zeroTimesInfty}{\ensuremath{\small\boldsymbol{0\cdot \infty}}}
\newcommand{\inftyMinusInfty}{\ensuremath{\small\boldsymbol{\infty - \infty}}}
\newcommand{\oneToInfty}{\ensuremath{\boldsymbol{1^\infty}}}
\newcommand{\zeroToZero}{\ensuremath{\boldsymbol{0^0}}}
\newcommand{\inftyToZero}{\ensuremath{\boldsymbol{\infty^0}}}


\newcommand{\numOverZero}{\ensuremath{\boldsymbol{\tfrac{\#}{0}}}}
\newcommand{\dfn}{\textbf}
%\newcommand{\unit}{\,\mathrm}
\newcommand{\unit}{\mathop{}\!\mathrm}
%\newcommand{\eval}[1]{\bigg[ #1 \bigg]}
\newcommand{\eval}[1]{ #1 \bigg|}
\newcommand{\seq}[1]{\left( #1 \right)}
\renewcommand{\epsilon}{\varepsilon}
\renewcommand{\iff}{\Leftrightarrow}

\DeclareMathOperator{\arccot}{arccot}
\DeclareMathOperator{\arcsec}{arcsec}
\DeclareMathOperator{\arccsc}{arccsc}
\DeclareMathOperator{\si}{Si}
\DeclareMathOperator{\proj}{proj}
\DeclareMathOperator{\scal}{scal}
\DeclareMathOperator{\cis}{cis}
\DeclareMathOperator{\Arg}{Arg}
%\DeclareMathOperator{\arg}{arg}
\DeclareMathOperator{\Rep}{Re}
\DeclareMathOperator{\Imp}{Im}
\DeclareMathOperator{\sech}{sech}
\DeclareMathOperator{\csch}{csch}
\DeclareMathOperator{\Log}{Log}

\newcommand{\tightoverset}[2]{% for arrow vec
  \mathop{#2}\limits^{\vbox to -.5ex{\kern-0.75ex\hbox{$#1$}\vss}}}
\newcommand{\arrowvec}{\overrightarrow}
\renewcommand{\vec}{\mathbf}
\newcommand{\veci}{{\boldsymbol{\hat{\imath}}}}
\newcommand{\vecj}{{\boldsymbol{\hat{\jmath}}}}
\newcommand{\veck}{{\boldsymbol{\hat{k}}}}
\newcommand{\vecl}{\boldsymbol{\l}}
\newcommand{\utan}{\vec{\hat{t}}}
\newcommand{\unormal}{\vec{\hat{n}}}
\newcommand{\ubinormal}{\vec{\hat{b}}}

\newcommand{\dotp}{\bullet}
\newcommand{\cross}{\boldsymbol\times}
\newcommand{\grad}{\boldsymbol\nabla}
\newcommand{\divergence}{\grad\dotp}
\newcommand{\curl}{\grad\cross}
%% Simple horiz vectors
\renewcommand{\vector}[1]{\left\langle #1\right\rangle}


\outcome{Compute the derivative of a constant multiple of a function.}

\title{Constant Multiple Rule}

\begin{document}

\begin{abstract}
We find the derivative of a constant multiple of a 
function.
\end{abstract}



\maketitle


\begin{center}
\bf{The Constant Multiple Rule}
\end{center}


\begin{theorem} If $f(x)$ is differentiable and $c$ is any constant, then
\[[cf(x)]' = c f'(x)\]
\end{theorem}

In words, the derivative of a constant times a function is the constant times the derivative of the function.\\



\begin{example} %example 1
 If $f(x) = 5x^2$ then  we use the constant multiple rule with $c = 5$ and we get 
\[
f'(x) = 5(2x)  = 10x.
\]
\end{example}


\begin{problem} %problem #1a
  Compute 
  \[
  \frac{d}{dx} \left(4x^2\right)
  \]
  
    \begin{hint}
      The derivative of $x^2$ is $2x$
    \end{hint}    
		The derivative of $4x^2$ with respect to $x$ is
		 $\answer{8x}$
	
\end{problem}


\begin{problem} %problem #1c
  Compute 
  \[
  \frac{d}{dx} \left(\frac{x^2}{2}\right)
  \]
  
    \begin{hint}
      The derivative of $x^2$ is $2x$
    \end{hint}    
		The derivative of $\frac{x^2}{2}$ with respect to $x$ is
		 $\answer{x}$
	
\end{problem}


\begin{example} %example 2
 If $f(x) = \frac{x^3}{9}$ then  we use the constant multiple rule with $c = \frac19$ and we get 
\[
f'(x) = \tfrac19 \cdot 3x^2 
= \frac{x^2}{3}.
\]
\end{example}

\begin{problem} %problem #2a
  Compute 
  \[
  \frac{d}{dx} \left(5x^3\right)
  \]
  
    \begin{hint}
      Use the Power Rule on $x^3$
    \end{hint}
    \begin{hint}
      The Power Rule says:
      \[
      \frac{d}{dx} x^n = nx^{n-1}
      \]
    \end{hint}
		\begin{hint}
		  Don't forget to multiply by $5$
		\end{hint}
		The derivative of $5x^3$ with respect to $x$ is
		 $\answer{15x^2}$
	
\end{problem}

\begin{problem} %problem #2b
  Compute 
  \[
  \frac{d}{dx} \left(8x^{1/2}\right)
  \]
  
    \begin{hint}
      Use the Power Rule on $x^{1/2}$
    \end{hint}
    \begin{hint}
      The Power Rule says:
      \[
      \frac{d}{dx} x^n = nx^{n-1}
      \]
    \end{hint}
		\begin{hint}
		  Don't forget to multiply by $8$
		\end{hint}
		The derivative of $8x^{1/2}$ with respect to $x$ is
		 $\answer{4x^{-1/2}}$
	
\end{problem}

\begin{problem} %problem #2c
  Compute 
  \[
  \frac{d}{dx} \left(\frac{x^5}{10}\right)
  \]
  
    \begin{hint}
      Use the Power Rule on $x^5$
    \end{hint}
    \begin{hint}
      The Power Rule says:
      \[
      \frac{d}{dx} x^n = nx^{n-1}
      \]
    \end{hint}
		\begin{hint}
		  Don't forget to multiply by $1/10$
		\end{hint}
		
		The derivative of $\frac{x^5}{10}$ with respect to $x$ is
		 $\answer{\frac{x^4}{2}}$
	
\end{problem}


\begin{example} %example 3
 If $f(x) = \frac{4}{x^2}$ then we rewrite $f(x)$ as $4x^{-2}$  and we use the constant multiple rule with $c = 4$, 
giving 
\[f'(x) = 4 (-2x^{-3}) = -8x^{-3} = -\frac{8}{x^3}.
\]
\end{example}

\begin{problem} %problem #3a
  Compute 
  \[
  \frac{d}{dx} \left(\frac{10}{x^5}\right)
  \]
  
    \begin{hint}
		  Rewrite using $\frac{1}{x^n} = x^{-n}$
		\end{hint}
		\begin{hint}
      Use the Power Rule on $x^{-5}$
    \end{hint}
    \begin{hint}
      The Power Rule says:
      \[
      \frac{d}{dx} x^n = nx^{n-1}
      \]
    \end{hint}
		\begin{hint}
		  Don't forget to multiply by 10
		\end{hint}
		
		The derivative of $\frac{10}{x^5}$ with respect to $x$ is
		 $\answer{-50x^{-6}}$
	
\end{problem}


\begin{problem} %problem #3b
  Compute 
  \[
  \frac{d}{dx} \left(6\sqrt x\right)
  \]
  
    \begin{hint}
		  Rewrite using $\sqrt[n] x = x^{1/n}$
		\end{hint}
		\begin{hint}
      Use the Power Rule on $x^{1/2}$
    \end{hint}
    \begin{hint}
      The Power Rule says:
      \[
      \frac{d}{dx} x^n = nx^{n-1}
      \]
    \end{hint}
		\begin{hint}
		  Don't forget to multiply by 6
		\end{hint}
		
		The derivative of $6\sqrt x$ with respect to $x$ is
		 $\answer{3x^{-1/2}}$
	
\end{problem}


\begin{example} %example 4
 If $f(x) = -3\cos(x)$ then we use the constant multiple rule with $c = -3$ and we get 
\[
f'(x) = -3(-\sin(x)) = 3\sin(x).
\]
\end{example}


\begin{problem} %problem #4a
  Compute 
  \[
  \frac{d}{dx} \left[5\sin(x)\right]
  \]
  
    \begin{hint}
      The derivative of $\sin(x)$ is $\cos(x)$
    \end{hint}
		\begin{hint}
		  Don't forget to multiply by 5
		\end{hint}
		The derivative of $5\sin(x)$ with respect to $x$ is
		 $\answer{5\cos(x)}$
	
\end{problem}


\begin{problem} %problem #4b
  Compute 
  \[
  \frac{d}{dx} \left[\pi\cos(x)\right]
  \]
  
    \begin{hint}
      The derivative of $\cos(x)$ is $-\sin(x)$
    \end{hint}
		\begin{hint}
		  Don't forget to multiply by $\pi$
		\end{hint}
		
		The derivative of $\pi\cos(x)$ with respect to $x$ is
		 $\answer{-\pi\sin(x)}$
	
\end{problem}


\begin{example} %example 5
 If $f(x) = \pi\tan^{-1}(x)$ then we use the constant multiple rule with $c = \pi$ and we get 
\[
f'(x) =  \pi \cdot \frac{1}{1+x^2}= \frac{\pi}{1+x^2}.
\]
\end{example}

\begin{problem} %problem #5a
  Compute 
  \[
  \frac{d}{dx} \left[3\sin^{-1}(x)\right]
  \]
  
    \begin{hint}
      The derivative of $\sin^{-1}(x)$ is $\frac{1}{\sqrt{1-x^2}}$
    \end{hint} 
		\begin{hint}
		  Don't forget to multiply by 3
		\end{hint}
		
		The derivative of $3\sin^{-1}(x)$ with respect to $x$ is
		 $\answer{\frac{3}{\sqrt{1-x^2}}}$
	
\end{problem}

\begin{problem} %problem #5b
  Compute 
  \[
  \frac{d}{dx} \left[2\cos^{-1}(x)\right]
  \]
  
    \begin{hint}
      The derivative of $\cos^{-1}(x)$ is $-\frac{1}{\sqrt{1-x^2}}$
    \end{hint}
		\begin{hint}
		  Don't forget to multiply by 2
		\end{hint}
		The derivative of $2\cos^{-1}(x)$ with respect to $x$ is
		 $\answer{-\frac{2}{\sqrt{1-x^2}}}$
	
\end{problem}


\begin{problem} %problem #5c
  Compute 
  \[
  \frac{d}{dx} \left[5\tan^{-1}(x)\right]
  \]
  
    \begin{hint}
      The derivative of $\tan^{-1}(x)$ is $\frac{1}{1+x^2}$
    \end{hint}
		\begin{hint}
		  Don't forget to multiply by 5
		\end{hint}
		
		The derivative of $5\tan^{-1}(x)$ with respect to $x$ is
		 $\answer{\frac{5}{1+x^2}}$
	
\end{problem}


\begin{example} %example 6
 If $f(x) = \frac{2e^x}{5}$ then we use the constant multiple rule with $c = \frac25$ and we get 
\[
f'(x) = \tfrac25 e^x = \frac{2e^x}{5}.
\]
\end{example}


\begin{problem} %problem #6a
  Compute 
  \[
  \frac{d}{dx} \left(\frac{e^x}{2}\right)
  \]
  
    \begin{hint}
      The derivative of $e^x$ is $e^x$
    \end{hint}
		\begin{hint}
		  Don't forget to multiply by $1/2$
		\end{hint}
		
		The derivative of $\frac{e^x}{2}$ with respect to $x$ is
		 $\answer{\frac{e^x}{2}}$
	
\end{problem}


\begin{problem} %problem #6b
  Compute 
  \[
  \frac{d}{dx} \left(3\cdot 2^x\right)
  \]
  
    \begin{hint}
      The derivative of $a^x$ is $a^x \ln(a)$
    \end{hint}
		\begin{hint}
		  Don't forget to multiply by $3$
		\end{hint}
		
		The derivative of $3\cdot 2^x$ with respect to $x$ is
		 $\answer{3\cdot 2^x \ln(2)}$
	
\end{problem}


\begin{example} %example 7
 If $f(x) = \frac{\ln(x)}{\ln(4)}$ then we use the constant multiple rule with $c = \frac{1}{\ln(4)}$ 
and we get 
\[
f'(x) = \tfrac{1}{\ln(4)} \cdot \frac{1}{x} = \frac{1}{x\ln(4)}.
\]
\end{example}


\begin{problem} %problem #7a
  Compute 
  \[
  \frac{d}{dx} \left[8\ln(x)\right]
  \]
  
    \begin{hint}
      The derivative of $\ln(x)$ is $1/x$
    \end{hint}
		\begin{hint}
		  Don't forget to multiply by $8$
		\end{hint}
		
		The derivative of $8\ln(x)$ with respect to $x$ is
		 $\answer{\frac{8}{x}}$
	
\end{problem}


\begin{problem} %problem #7b
  Compute 
  \[
  \frac{d}{dx} \left[2\log(x)\right]
  \]
  
    \begin{hint}
      The derivative of $\log_a(x)$ is $\frac{1}{x\ln(a)}$
    \end{hint}
		\begin{hint}
		  $\log(x) = \log_10(x)$ 
		\end{hint}
		\begin{hint}
		  Don't forget to multiply by $2$
		\end{hint}
		
		The derivative of $2\log(x)$ with respect to $x$ is
		 $\answer{\frac{2}{x\ln(10)}}$
	
\end{problem}

\end{document}



