\documentclass{ximera}


%% You can put user macros here
%% However, you cannot make new environments



\newcommand{\ffrac}[2]{\frac{\text{\footnotesize $#1$}}{\text{\footnotesize $#2$}}}
\newcommand{\vasymptote}[2][]{
    \draw [densely dashed,#1] ({rel axis cs:0,0} -| {axis cs:#2,0}) -- ({rel axis cs:0,1} -| {axis cs:#2,0});
}


%\usepackage{tcolorbox} %%Needed for Derivative Definition supposedly and product rule, natural exp log, quotient rule, inverse trig, rates of change


% \graphicspath{{./}{firstExample/}}
% \usepackage{forest}
\usepackage{amsmath}
\usepackage{amssymb}
\usepackage{array}
\usepackage[makeroom]{cancel} %% for strike outs
\usepackage{pgffor} %% required for integral for loops
\usepackage{tikz}
\usepackage{tikz-cd}
\usepackage{tkz-euclide}
\usetikzlibrary{shapes.multipart}


% \usetkzobj{all}
\tikzstyle geometryDiagrams=[ultra thick,color=blue!50!black]


\usetikzlibrary{arrows}
\tikzset{>=stealth,commutative diagrams/.cd,
  arrow style=tikz,diagrams={>=stealth}} %% cool arrow head
\tikzset{shorten <>/.style={ shorten >=#1, shorten <=#1 } } %% allows shorter vectors

\usetikzlibrary{backgrounds} %% for boxes around graphs
\usetikzlibrary{shapes,positioning}  %% Clouds and stars
\usetikzlibrary{matrix} %% for matrix
\usepgfplotslibrary{polar} %% for polar plots
\usepgfplotslibrary{fillbetween} %% to shade area between curves in TikZ



%\usepackage[width=4.375in, height=7.0in, top=1.0in, papersize={5.5in,8.5in}]{geometry}
%\usepackage[pdftex]{graphicx}
%\usepackage{tipa}
%\usepackage{txfonts}
%\usepackage{textcomp}
%\usepackage{amsthm}
%\usepackage{xy}
%\usepackage{fancyhdr}
%\usepackage{xcolor}
%\usepackage{mathtools} %% for pretty underbrace % Breaks Ximera
%\usepackage{multicol}



\newcommand{\RR}{\mathbb R}
\newcommand{\R}{\mathbb R}
\newcommand{\C}{\mathbb C}
\newcommand{\N}{\mathbb N}
\newcommand{\Z}{\mathbb Z}
\newcommand{\dis}{\displaystyle}
%\renewcommand{\d}{\,d\!}
\renewcommand{\d}{\mathop{}\!d}
\newcommand{\dd}[2][]{\frac{\d #1}{\d #2}}
\newcommand{\pp}[2][]{\frac{\partial #1}{\partial #2}}
\renewcommand{\l}{\ell}
\newcommand{\ddx}{\frac{d}{\d x}}
\newcommand{\ppx}{\frac{\partial}{\partial x}}
\newcommand{\ppy}{\frac{\partial}{\partial y}}

\newcommand{\zeroOverZero}{\ensuremath{\boldsymbol{\tfrac{0}{0}}}}
\newcommand{\inftyOverInfty}{\ensuremath{\boldsymbol{\tfrac{\infty}{\infty}}}}
\newcommand{\zeroOverInfty}{\ensuremath{\boldsymbol{\tfrac{0}{\infty}}}}
\newcommand{\zeroTimesInfty}{\ensuremath{\small\boldsymbol{0\cdot \infty}}}
\newcommand{\inftyMinusInfty}{\ensuremath{\small\boldsymbol{\infty - \infty}}}
\newcommand{\oneToInfty}{\ensuremath{\boldsymbol{1^\infty}}}
\newcommand{\zeroToZero}{\ensuremath{\boldsymbol{0^0}}}
\newcommand{\inftyToZero}{\ensuremath{\boldsymbol{\infty^0}}}


\newcommand{\numOverZero}{\ensuremath{\boldsymbol{\tfrac{\#}{0}}}}
\newcommand{\dfn}{\textbf}
%\newcommand{\unit}{\,\mathrm}
\newcommand{\unit}{\mathop{}\!\mathrm}
%\newcommand{\eval}[1]{\bigg[ #1 \bigg]}
\newcommand{\eval}[1]{ #1 \bigg|}
\newcommand{\seq}[1]{\left( #1 \right)}
\renewcommand{\epsilon}{\varepsilon}
\renewcommand{\iff}{\Leftrightarrow}

\DeclareMathOperator{\arccot}{arccot}
\DeclareMathOperator{\arcsec}{arcsec}
\DeclareMathOperator{\arccsc}{arccsc}
\DeclareMathOperator{\si}{Si}
\DeclareMathOperator{\proj}{proj}
\DeclareMathOperator{\scal}{scal}
\DeclareMathOperator{\cis}{cis}
\DeclareMathOperator{\Arg}{Arg}
%\DeclareMathOperator{\arg}{arg}
\DeclareMathOperator{\Rep}{Re}
\DeclareMathOperator{\Imp}{Im}
\DeclareMathOperator{\sech}{sech}
\DeclareMathOperator{\csch}{csch}
\DeclareMathOperator{\Log}{Log}

\newcommand{\tightoverset}[2]{% for arrow vec
  \mathop{#2}\limits^{\vbox to -.5ex{\kern-0.75ex\hbox{$#1$}\vss}}}
\newcommand{\arrowvec}{\overrightarrow}
\renewcommand{\vec}{\mathbf}
\newcommand{\veci}{{\boldsymbol{\hat{\imath}}}}
\newcommand{\vecj}{{\boldsymbol{\hat{\jmath}}}}
\newcommand{\veck}{{\boldsymbol{\hat{k}}}}
\newcommand{\vecl}{\boldsymbol{\l}}
\newcommand{\utan}{\vec{\hat{t}}}
\newcommand{\unormal}{\vec{\hat{n}}}
\newcommand{\ubinormal}{\vec{\hat{b}}}

\newcommand{\dotp}{\bullet}
\newcommand{\cross}{\boldsymbol\times}
\newcommand{\grad}{\boldsymbol\nabla}
\newcommand{\divergence}{\grad\dotp}
\newcommand{\curl}{\grad\cross}
%% Simple horiz vectors
\renewcommand{\vector}[1]{\left\langle #1\right\rangle}


\outcome{Determine where a function is increasing or decreasing.}

\title{3.3 Increasing and Decreasing Functions}


\begin{document}

\begin{abstract}
In this section, we use the derivative to determine intervals on which a given function is increasing or decreasing.  
We will also determine the local extremes of the function.
\end{abstract}

\maketitle


\section{Increasing and Decreasing Functions}



\begin{definition}
A function $f(x)$ is called \textbf{increasing} on an interval $I$ if given any two numbers, $x_1$ and $x_2$ in $I$ such that 
$x_1 < x_2$, we have $f(x_1) < f(x_2)$.\\ 

Similarly, $f(x)$ is called \textbf{decreasing} on an interval $I$ if given any two numbers,
$x_1$ and $x_2$ in $I$ such that 
$x_1 < x_2$, we have $f(x_1) > f(x_2)$. 
\end{definition}




%As an example of increasing, the function $f(x) = x^2$ is increasing on the interval 
%$(0, \infty)$ since if $0 < x_1 < x_2$ then $(x_1)^2 < (x_2)^2$.

\begin{center}
\begin{tikzpicture}
\begin{axis}[axis x line=middle, axis y line= middle, xlabel={$x$}, ylabel={$y$}, axis equal, xtick={.75, 1.5},
xticklabels={$x_1$, $x_2$}, ytick={1.682, 2.828}, yticklabels={$f(x_1)$, $f(x_2)$}, title={An Increasing Function}]

\addplot[domain=-1:2]{2^x};
\addplot[dashed] coordinates{(0.75, 0)  (0.75, 1.682)};
\addplot[dashed] coordinates{ (0.75, 1.682) (0, 1.682)};
\addplot[dashed] coordinates{(1.5, 0) (1.5, 2.828)};
\addplot[dashed] coordinates{(1.5, 2.828) (0, 2.828)};

\end{axis}
%\addplot[mark=*,fill=white] coordinates {(0,0)}
\node at (4, -1) {$x_1 < x_2 \implies f(x_1) < f(x_2)$};				
\node at (6.5, 5) {$y = f(x)$};
 

\end{tikzpicture}
\hspace{1.5 in}
\begin{tikzpicture}
\begin{axis}[axis x line=middle, axis y line= middle, xlabel={$x$}, ylabel={$y$}, axis equal, xtick={.5, 1.5},
xticklabels={$x_1$, $x_2$}, ytick={0.707, 0.3536}, yticklabels={$f(x_1)$, $f(x_2)$}, title={A Decreasing Function}]

\addplot[domain=-0.5:2]{1/2^x};
\addplot[dashed] coordinates{(0.5, 0)  (0.5, 0.707)};
\addplot[dashed] coordinates{ (0.5, 0.707) (0, 0.707)};
\addplot[dashed] coordinates{(1.5, 0) (1.5, 0.3536)};
\addplot[dashed] coordinates{(1.5, 0.3536) (0, 0.3536)};

\end{axis}
%\addplot[mark=*,fill=white] coordinates {(0,0)}
\node at (4, -1) {$x_1 < x_2 \implies f(x_1) > f(x_2)$};				
\node at (-0.7, 4.5) {$y = f(x)$};
 

\end{tikzpicture}

\end{center}
%align=flush center,text width=8cm
%minor tick ={0.7, 1.7}
%\draw (0.7pt,2pt) -- (0.7pt,-2pt) node[below] {$x_1$};
%[below=1cm]


%As an example of decreasing, the function $x^2$ is decreasing on the interval $(-\infty, 0)$ 
%since if $ x_1 < x_2 < 0$ then $(x_1)^2 > (x_2)^2$.


%\addplot[mark=*,fill=white] coordinates {(0,0)};


%\begin{image}
%\begin{tikzpicture}
%\begin{axis}[axis x line=  center, axis y line = none, xtick={-1, 0.5, 2}, xticklabels={$a$,$c$,$b$}, 
%legend pos=outer north east, title={MVT for $f(x) $ on $[a,b]$}]
%\addplot[domain=-1:2, 
 %   samples=100, color=black]{-x^2 + 5 };
%\addplot[smooth,mark=*,blue] plot coordinates {(-1,4)  (2,1)};
%\addplot[domain=-1:2, 
 %   samples=100, color=blue]{-x + 3};
%\addplot[domain=-1:2, 
%    samples=100, color=red]{-x + 5.25 };
%\addplot[smooth,mark=*,red] plot coordinates {(0.5, 4.75)};
%\legend{$y = f(x)$, , secant line, tangent line, };
%\draw[dashed] {(-1,0) -- (-1,4)};
%\draw[dashed] (0.5,0) -- (0.5,4.75);
%\draw[dashed] (2,0) -- (2,1);
%\end{axis}
%\end{tikzpicture}
%\end{image}




To determine where a function is increasing or decreasing, we use the derivative as in the following theorem.

\begin{theorem}[Increasing/Decreasing]
If $f'(x) > 0$ on an interval $I$, then $f(x)$ is increasing on $I$, and
if $f'(x) < 0$ on an interval $I$, then $f(x)$ is decreasing on $I$.
\end{theorem}


%Proof (Increasing case): Let $x_1$ and $x_2$ be in the interval $I$ with $x_1 < x_2$.  
%Then since $f$ is differentiable on the closed interval 
%$[x_1, x_2]$, it is also continuous there.  Hence we can apply the MVT on to $f(x)$ on $[x_1, x_2]$ and conclude that
%\[f(x_2) - f(x_1) = f'(c) (x_2 - x_1).\]
%Both factors on the right hand side are positive, hence $f(x_2) - f(x_1)$ is positive and $f(x_2)> f(x_1)$.  
%Hence $x_1 < x_2$implies $f(x_1)< f(x_2)$ which means that $f(x)$ is increasing on $I$.

%We now use this theorem to determine intervals on which a given function is either increasing or decreasing.

\begin{example}[example 1]
Determine intervals on which $f(x) = 2x^3 - 3x^2  - 36x + 2$ is increasing or decreasing.\\
According to the theorem, we must determine where $f'(x)$ is positive and  where $f'(x)$ is negative. 
To do this, it is often easiest to first determine where $f'(x) = 0$ or 
$f'(x)$ is undefined. In this example, 
\[f'(x) = 6x^2 - 6x -36\] 
which exists for all $x$. We solve the equation
\[f'(x) = 6x^2 - 6x -36 = 0\] 
which yields 
\[6(x+2)(x-3) = 0\] 
and hence, $x=-2$ or $x=3$. 
Note that these are the critical numbers of $f(x)$.
These two $x$-values break the real number line into three open intervals: $(-\infty, -2), (-2, 3)$ and $(3, \infty)$.
On each of these intervals, $f'$ will be either strictly positive or strictly negative.  
To determine which, we will use \textbf{test points}. 
For the interval, $(-\infty, -2)$ we will use $-3$ as the test point (any number in the interval is acceptable). 
Then we consider the derivative at this point: $f'(-3) = (-3)^2 - (-3) -6 = 6 >0$.
This means that $f'(x) > 0$ for every $x$ value in the interval $(-\infty, -2)$ and by the theorem, 
$f(x)$ is increasing on this interval.

Next, for the interval $(-2, 3)$, we will use $0$ as the test point, and we have $f'(0) = (0)^2 - (0) -6 = -6 <0$. 
This means that $f'(x) < 0$ for every $x$ value in the interval $(-2, 3)$ and by the theorem, 
$f(x)$ is decreasing on this interval.


Finally, for the interval $(3, \infty)$, we will use $4$ as the test point and we have $f'(4) = (4)^2 - (4) -6 = 10>0$. 
This means that $f'(x) > 0$ for every $x$ value in the interval $(3, \infty)$ and by the theorem, 
$f(x)$ is increasing on this interval.


\begin{image}
\begin{tikzpicture}
\node at (.5,1.5) {Sign of $f'(x)$};
\node at (-3, .5) {+++};
\node at (.5, .5) {- - - - -};
\node at (4, .5) {+++};
\draw[thick, <->] (-4, 0) --(5, 0);
\draw[thin] (-2, -.2) --(-2,.2);
\draw[thin] (3,-.2) --(3,.2);
\node at (-2, -.5) {$-2$};
\node at (3, -.5) {$3$};
\end{tikzpicture}
\end{image}


Thus $f(x)$ is increasing on the intervals $(-\infty, -2)$ and $(3, \infty)$,
and it is decreasing on the interval $(-2, 3)$.
\end{example}

The work that was done in the previous example can actually give us slightly more information about $f(x)$.  
We can determine the \textbf{local extremes} of $f(x)$.

\begin{definition}[Local Extremes]
We say that $f(x)$ has a \textbf{local maximum} at $x = a$ if there is an open interval $I$ containing $a$ 
such that $f(x) \leq f(a)$ for all values of $x$ in $I$.  Similarly, we say that $f(x)$ has a 
\textbf{local minimum} at $x = a$ if there is an open interval $I$ containing $a$ 
such that $f(x) \geq f(a)$ for all values of $x$ in $I$.
\end{definition}

Critical numbers can help us find the location and the nature of local extremes and the next theorem tells us how.

\begin{theorem}[First Derivative Test]
  %Suppose $x=a$ is a critical number of the function $f(x)$. \\
	%If $f'(x)$ changes sign at $x_0$, then $f(x)$ has a local 
%extreme at $x=x_0$.  \\
%More specifically, 
If $f'(x)$ changes sign from positive to negative at the critical number $x=a$,
 then  $f(x)$ has a local maximum at $x = a$ and
 if $f'(x)$ changes sign from negative to positive at the critical number $x=a$,
 then $f(x)$ has a local minimum at $x = a$.
\end{theorem}
 
\begin{image}
\begin{tikzpicture}
\node at (5,3.5) {First Derivative Test};
\node at (1.5, 2.5) {Local Maximum at $x=a$};
\draw[<->] (0, 0) --(3, 0);
%\node at (3.2, 0) {$x$};
\draw (1.5,2) parabola (2.75,.25);
\draw (1.5,2) parabola (0.25,.25);
\draw[fill] (1.5,2) circle [radius=0.07];
\node at (-.5, 1) {$f'(x) >0$};
\node at (3.5, 1) {$f'(x) <0$};

\draw (1.5, .1) -- (1.5, -.1) node[below] {$a$};
%\node at (1.5, -.9) {$f''(a) < 0$};

\node at (8, 2.5) {Local Minimum at $x=a$};
\draw[<->] (6.5, 0) --(9.5, 0);
%\node at (3.2, 0) {$x$};

\draw (8,.25) parabola (6.75,2);
\draw (8,.25) parabola (9.25,2);
\draw[fill] (8,.25) circle [radius=0.07];
\draw (8, .1) -- (8, -.1) node[below] {$a$};
\node at (6, 1) {$f'(x) <0$};
\node at (10, 1) {$f'(x) >0$};
%\node at (6.5, -.9) {$f''(a) > 0$};
\end{tikzpicture}
\end{image}




\begin{example}[example 2]
Find the local extremes of $f(x) =2x^3 - 3x^2 - 36x + 2 $.\\
In the previous example, we found that the critical numbers for $f(x)$ were $-2$ and $3$.  We also noted that
$f'(x)$ changed sign from positive to negative at $x = -2$ and from negative to positive at $x=3$.
Hence, by the First Derivative Test, $f(x) = 2x^3 - 6x^2 - 36x + 2$ has a local maximum at $x = -2$ and a local minimum at $x=3$.
\[
\graph{2x^3 - 3x^2 - 36x + 2}
\]
\end{example}



\begin{problem}(problem 2a)
Find the local extremes of the function $f(x) = x^2 - 6x + 2.$\\
\begin{hint}
Find the critical numbers of $f(x)$
\end{hint}
\begin{hint}
Find intervals where $f'(x)$ is positive/negative
\end{hint}

\begin{hint}
Use the First Derivative Test to determine the local extremes.
\end{hint}

If there are none, type ``none".\\
$f(x)$ has a local maximum at $x = \answer{none}.$\\
$f(x)$ has a local minimum at $x = \answer{3}.$
\end{problem}


\begin{problem}(problem 2b)
Find the local extremes of the function $f(x) = x^3 - 3x^2 + 3.$\\
\begin{hint}
Find the critical numbers of $f(x)$
\end{hint}
\begin{hint}
Find intervals where $f'(x)$ is positive/negative
\end{hint}
\begin{hint}
Use the First Derivative Test to determine the local extremes.
\end{hint}

%3x^2 - 6x   3x(x-2)
If there are none, type ``none".\\
$f(x)$ has a local maximum at $x = \answer{0}.$\\
$f(x)$ has a local minimum at $x = \answer{2}.$
\end{problem}


\begin{problem}(problem 2c)
Find the local extremes of the function $f(x) = x^3 - 3x^2 +3x -1.$\\
\begin{hint}
Find the critical numbers of $f(x)$
\end{hint}
\begin{hint}
Find intervals where $f'(x)$ is positive/negative
\end{hint}
\begin{hint}
Use the First Derivative Test to determine the local extremes.
\end{hint}

%3x^2 - 6x +3  x^2 - 2x + 1  x = 1
If there are none, type ``none".\\
$f(x)$ has a local maximum at $x = \answer{none}.$\\
$f(x)$ has a local minimum at $x = \answer{none}.$
\end{problem}


\begin{problem}(problem 2d)
Find the local extremes of the function $f(x) = x^4 - 8x^2 + 1.$
\begin{hint}
Find the critical numbers of $f(x)$
\end{hint}
\begin{hint}
Find intervals where $f(x)$ is increasing/decreasing
\end{hint}
\begin{hint}
Use the First Derivative Test to determine the local extremes.
\end{hint}

If there are none, type ``none".\\
If there is more than one local extreme, list them in ascending order.\\
$f(x)$ has a local maximum at $x = \answer{0}.$\\
$f(x)$ has a local minimum at $x = \answer{-2}.$\\
and at $x = \answer{2}.$
\end{problem}


\begin{problem}(problem 2e)
Find the local extremes of the function $f(x) = 3x^4 +8x^3 - 5.$\\
% f' = x^2 (x+2) = x^3 + 2x^2,  f = x^4/4 +2x^3/3 or 3x^4 +8x^3 - 5
\begin{hint}
Find the critical numbers of $f(x)$
\end{hint}
\begin{hint}
Find intervals where $f'(x)$ is positive/negative
\end{hint}
\begin{hint}
Use the First Derivative Test to determine the local extremes.
\end{hint}

If there are none, type ``none".\\
$f(x)$ has a local maximum at $x = \answer{none}.$\\
$f(x)$ has a local minimum at $x = \answer{-2}.$
\end{problem}



\begin{example}[example 3]
Find intervals on which $f(x) = xe^{3x}$ is increasing or decreasing and find and describe the local extremes.\\
We begin by finding the critical numbers of $f(x)$. By the product and chain rules, 
\[f'(x) = e^{3x} + xe^{3x}\cdot 3 = e^{3x}(1+3x).\]
The derivative exists for all $x$.  Setting the derivative equal to zero gives
\[e^{3x} = 0 \;\; \text{or} \;\; 1+3x = 0.\]
The first equation has no solutions, since $e$ raised to any power is strictly positive
and the second equation has one solution, $x = -1/3$.
This one critical number breaks the real number line into two intervals.  The first is $(-\infty, -1/3)$
and the second is $(-1/3, \infty)$.  In the first interval, we choose the test point $x = -1$:
\[f'(-1) = e^{-3}(1-3) = -\frac{2}{e^3} < 0,\]
and so $f(x)$ is decreasing on the interval $(-\infty, -1/3)$.
In the second interval, we choose the test point $x = 0$:
\[f'(0) = e^{0}(1-0) = 1 > 0,\]
and so $f(x)$ is increasing on the interval $(-1/3, \infty)$.
As for the local extremes, at the critical number $x = -1/3$,
the derivative changes sign from negative to positive, so by the First Derivative Test,
$f(x)$ has a local minimum at $x = -1/3$. There are no other critical numbers, so there are no other local extremes for this function.
\end{example}

\begin{problem}(problem 3a)
Find the local extremes of the function $f(x) = xe^{2x}.$\\
\begin{hint}
Find the critical numbers of $f(x)$
\end{hint}
\begin{hint}
Find intervals where $f'(x)$ is positive/negative
\end{hint}
\begin{hint}
Use the First Derivative Test to determine the local extremes.
\end{hint}

If there are none, type ``none".\\
$f(x)$ has a local maximum at $x = \answer{none}.$\\
$f(x)$ has a local minimum at $x = \answer{-1/2}.$
\end{problem}


\begin{problem}(problem 3b)
Find the local extremes of the function $f(x) = x^2e^{3x}.$\\
\begin{hint}
Find the critical numbers of $f(x)$
\end{hint}
\begin{hint}
Find intervals where $f'(x)$ is positive/negative
\end{hint}
\begin{hint}
Use the First Derivative Test to determine the local extremes.
\end{hint}

If there are none, type ``none".\\
$f(x)$ has a local maximum at $x = \answer{-2/3}.$\\
$f(x)$ has a local minimum at $x = \answer{0}.$
\end{problem}

\begin{problem}(problem 3c)
Find the local extremes of the function $f(x) = \dfrac{3x}{x^2 + 4}.$\\
\begin{hint}
Find the critical numbers of $f(x)$
\end{hint}
\begin{hint}
Find intervals where $f'(x)$ is positive/negative
\end{hint}
\begin{hint}
Use the First Derivative Test to determine the local extremes.
\end{hint}

If there are none, type ``none".\\
$f(x)$ has a local maximum at $x = \answer{2}.$\\
$f(x)$ has a local minimum at $x = \answer{-2}.$
\end{problem}




\begin{example}[example 4]
Use the graph of $f'(x)$ given below to determine intervals on which $f(x)$
is increasing or decreasing.

\begin{image}
\begin{tikzpicture} 
\begin{axis}[ axis x line=middle, axis y line=middle, ymax=5, ymin=-5, xlabel=$x$,ylabel=$y$, title={Graph of the derivative, $y = f'(x)$} ] 
\addplot[blue,mark=none, domain=-3:3,samples=40] {4-x^2}; 
\end{axis} 
\end{tikzpicture}
\end{image}


First, recall that the Increasing/Decreasing Theorem states that $f(x)$ is increasing on intervals where $f'(x) > 0$ and
$f(x)$ is decreasing on intervals where $f'(x) < 0$.
Such intervals can be determined from the graph of $f'(x)$ by noting when it is above or below the $x$-axis.
This graph is above the $x$-axis on the interval $(-2, 2)$ and below the $x$-axis on the intervals $(-\infty, -2)$
and $(2, \infty)$. When the graph of $f'(x)$ is above the $x$-axis, then $f'(x)>0$ and hence, $f(x)$
is increasing. Likewise, when the graph of $f'(x)$ is below the $x$-axis, then $f'(x)<0$ and $f(x)$ is decreasing.
We can now interpret the graph of $f'(x)$ to state that $f(x)$ is increasing on the interval $(-2, 2)$ 
and $f(x)$ is decreasing on the intervals $(-\infty, -2)$ and $(2, \infty)$.



%From the graph of $f'(x)$, above, we can see that $f'(x) > 0$ on the interval $(-2, 2)$ and $f'(x) < 0$ on the intervals  
%Specifically, we have $f(x)$ is increasing 

\end{example}









\begin{center}
\begin{foldable}
\unfoldable{Here are some detailed, lecture style videos on inc/dec functions:}
\youtube{bNVSN7xdZIs}
\youtube{0yGcbq9Bbdo}
\end{foldable}
\end{center}


\end{document}












\begin{problem}
Find the intervals on which the function $f(x) = \dfrac{3x}{x^2 + 4}$ is increasing or decreasing.
\begin{hint}
Find the critical numbers of $f(x)$
\end{hint}
\begin{hint}
Find intervals where $f'(x)$ is positive/negative
\end{hint}
\begin{hint}
$f'(x) > 0 \implies f(x) \; \text{ is increasing}$\\
$f'(x) < 0 \implies f(x) \; \text{ is decreasing}$
\end{hint}

If there are none, type ``none".\\
$f(x)$ has a local maximum at $x = \answer{2}$.\\
$f(x)$ has a local minimum at $x = \answer{-2}$.
\end{problem}
