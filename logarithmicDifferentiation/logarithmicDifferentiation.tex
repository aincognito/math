\documentclass[handout]{ximera}

%% You can put user macros here
%% However, you cannot make new environments



\newcommand{\ffrac}[2]{\frac{\text{\footnotesize $#1$}}{\text{\footnotesize $#2$}}}
\newcommand{\vasymptote}[2][]{
    \draw [densely dashed,#1] ({rel axis cs:0,0} -| {axis cs:#2,0}) -- ({rel axis cs:0,1} -| {axis cs:#2,0});
}


%\usepackage{tcolorbox} %%Needed for Derivative Definition supposedly and product rule, natural exp log, quotient rule, inverse trig, rates of change


% \graphicspath{{./}{firstExample/}}
% \usepackage{forest}
\usepackage{amsmath}
\usepackage{amssymb}
\usepackage{array}
\usepackage[makeroom]{cancel} %% for strike outs
\usepackage{pgffor} %% required for integral for loops
\usepackage{tikz}
\usepackage{tikz-cd}
\usepackage{tkz-euclide}
\usetikzlibrary{shapes.multipart}


% \usetkzobj{all}
\tikzstyle geometryDiagrams=[ultra thick,color=blue!50!black]


\usetikzlibrary{arrows}
\tikzset{>=stealth,commutative diagrams/.cd,
  arrow style=tikz,diagrams={>=stealth}} %% cool arrow head
\tikzset{shorten <>/.style={ shorten >=#1, shorten <=#1 } } %% allows shorter vectors

\usetikzlibrary{backgrounds} %% for boxes around graphs
\usetikzlibrary{shapes,positioning}  %% Clouds and stars
\usetikzlibrary{matrix} %% for matrix
\usepgfplotslibrary{polar} %% for polar plots
\usepgfplotslibrary{fillbetween} %% to shade area between curves in TikZ



%\usepackage[width=4.375in, height=7.0in, top=1.0in, papersize={5.5in,8.5in}]{geometry}
%\usepackage[pdftex]{graphicx}
%\usepackage{tipa}
%\usepackage{txfonts}
%\usepackage{textcomp}
%\usepackage{amsthm}
%\usepackage{xy}
%\usepackage{fancyhdr}
%\usepackage{xcolor}
%\usepackage{mathtools} %% for pretty underbrace % Breaks Ximera
%\usepackage{multicol}



\newcommand{\RR}{\mathbb R}
\newcommand{\R}{\mathbb R}
\newcommand{\C}{\mathbb C}
\newcommand{\N}{\mathbb N}
\newcommand{\Z}{\mathbb Z}
\newcommand{\dis}{\displaystyle}
%\renewcommand{\d}{\,d\!}
\renewcommand{\d}{\mathop{}\!d}
\newcommand{\dd}[2][]{\frac{\d #1}{\d #2}}
\newcommand{\pp}[2][]{\frac{\partial #1}{\partial #2}}
\renewcommand{\l}{\ell}
\newcommand{\ddx}{\frac{d}{\d x}}
\newcommand{\ppx}{\frac{\partial}{\partial x}}
\newcommand{\ppy}{\frac{\partial}{\partial y}}

\newcommand{\zeroOverZero}{\ensuremath{\boldsymbol{\tfrac{0}{0}}}}
\newcommand{\inftyOverInfty}{\ensuremath{\boldsymbol{\tfrac{\infty}{\infty}}}}
\newcommand{\zeroOverInfty}{\ensuremath{\boldsymbol{\tfrac{0}{\infty}}}}
\newcommand{\zeroTimesInfty}{\ensuremath{\small\boldsymbol{0\cdot \infty}}}
\newcommand{\inftyMinusInfty}{\ensuremath{\small\boldsymbol{\infty - \infty}}}
\newcommand{\oneToInfty}{\ensuremath{\boldsymbol{1^\infty}}}
\newcommand{\zeroToZero}{\ensuremath{\boldsymbol{0^0}}}
\newcommand{\inftyToZero}{\ensuremath{\boldsymbol{\infty^0}}}


\newcommand{\numOverZero}{\ensuremath{\boldsymbol{\tfrac{\#}{0}}}}
\newcommand{\dfn}{\textbf}
%\newcommand{\unit}{\,\mathrm}
\newcommand{\unit}{\mathop{}\!\mathrm}
%\newcommand{\eval}[1]{\bigg[ #1 \bigg]}
\newcommand{\eval}[1]{ #1 \bigg|}
\newcommand{\seq}[1]{\left( #1 \right)}
\renewcommand{\epsilon}{\varepsilon}
\renewcommand{\iff}{\Leftrightarrow}

\DeclareMathOperator{\arccot}{arccot}
\DeclareMathOperator{\arcsec}{arcsec}
\DeclareMathOperator{\arccsc}{arccsc}
\DeclareMathOperator{\si}{Si}
\DeclareMathOperator{\proj}{proj}
\DeclareMathOperator{\scal}{scal}
\DeclareMathOperator{\cis}{cis}
\DeclareMathOperator{\Arg}{Arg}
%\DeclareMathOperator{\arg}{arg}
\DeclareMathOperator{\Rep}{Re}
\DeclareMathOperator{\Imp}{Im}
\DeclareMathOperator{\sech}{sech}
\DeclareMathOperator{\csch}{csch}
\DeclareMathOperator{\Log}{Log}

\newcommand{\tightoverset}[2]{% for arrow vec
  \mathop{#2}\limits^{\vbox to -.5ex{\kern-0.75ex\hbox{$#1$}\vss}}}
\newcommand{\arrowvec}{\overrightarrow}
\renewcommand{\vec}{\mathbf}
\newcommand{\veci}{{\boldsymbol{\hat{\imath}}}}
\newcommand{\vecj}{{\boldsymbol{\hat{\jmath}}}}
\newcommand{\veck}{{\boldsymbol{\hat{k}}}}
\newcommand{\vecl}{\boldsymbol{\l}}
\newcommand{\utan}{\vec{\hat{t}}}
\newcommand{\unormal}{\vec{\hat{n}}}
\newcommand{\ubinormal}{\vec{\hat{b}}}

\newcommand{\dotp}{\bullet}
\newcommand{\cross}{\boldsymbol\times}
\newcommand{\grad}{\boldsymbol\nabla}
\newcommand{\divergence}{\grad\dotp}
\newcommand{\curl}{\grad\cross}
%% Simple horiz vectors
\renewcommand{\vector}[1]{\left\langle #1\right\rangle}


\outcome{Use the logarithm to compute the derivative of a function}

\title{2.12 Logarithmic Differentiation}

\begin{document}

\begin{abstract}
We use the logarithm to compute the derivative of a function.
\end{abstract}

\maketitle

\begin{center}
\textbf{Logarithmic Differentiation}
\end{center}

In this section we use the natural logarithm and its properties to help us to compute the 
derivative of some special functions as well as some complicated functions.
From the chain rule, we have
\[
\frac{d}{dx}\left[\ln(g(x))\right] = \frac{g'(x)}{g(x)}.
\]
If we solve this equation for $g'(x)$ we get the \textbf{logarithmic differentiation formula}:
\[
g'(x)= g(x)\cdot\frac{d}{dx}\left[\ln(g(x))\right].
\]
It says that we can find the derivative of a function by multiplying the original 
function by the derivative of its logarithm.
The reason that this formula is potentially useful is that the logarithm function has 
three calculus friendly properties:

\begin{align*}
1.\;\; &\ln(uv) = \ln(u) + \ln(v)\\
2.\;\; &\ln\left(\frac{u}{v}\right) = \ln(u) - \ln(v)\\
3.\;\; &\ln(u^n) = n\ln(u).
\end{align*}


In these pre-calculus formulas, the expressions $u, v$ and $n$ can be functions as well as constants.

\begin{example}[example 1]
Use logarithmic differentiation to find the derivative of 
\[
g(x) = x^x.
\]
We begin by computing the logarithm of $g(x)$ using the properties of the logarithm.
\[
\ln(g(x)) = \ln(x^x) = x\ln(x).
\]
Now we can use the product rule to compute the derivative of $\ln(g(x))$.
We have
\[
\frac{d}{dx} \left[\ln(g(x))\right] = \frac{d}{dx}\left[ x\ln(x)\right]
\]
\[
= 1\cdot \ln(x) + x \cdot \frac{1}{x} = \ln(x) + 1 = 1+ \ln(x).
\]
Finally, we use the logarithmic differentiation formula to obtain $g'(x)$.
We have:
\[
g'(x) = g(x) \frac{d}{dx} \left[\ln(g(x))\right] = x^x [1+\ln(x)].
\]

\end{example}


\begin{problem}(problem 1)
  Compute
  \[
  \frac{d}{dx} \left(x^{2x}\right)
  \]
  
    \begin{hint}
      Use logarithmic differentiation
    \end{hint}
    \begin{hint}
      Use property 3 of logarithms
    \end{hint}
    \begin{hint}
		  Use the product rule to find the derivative of $\ln(g(x))$
		\end{hint}
		\begin{hint}
		  Remember to multiply by the original function
		\end{hint}
    
		The derivative of $x^{2x}$ with respect to $x$ is
		 $\answer{x^{2x} (2+2\ln(x))}$
		
\end{problem}

\begin{example}[example 2]
Use logarithmic differentiation to find the derivative of 
\[
g(x) = x^{\sin(x)}.
\]
We begin by computing the logarithm of $g(x)$ using the properties of the logarithm.
\[
\ln(g(x)) = \ln\left(x^{\sin(x)}\right) = \sin(x)\ln(x).
\]
Now we can use the product rule to compute the derivative of $\ln(g(x))$.
We have
\[
\frac{d}{dx} \left[\ln(g(x))\right] = \frac{d}{dx} \left[\sin(x)\ln(x)\right]
\]
\[
= \cos(x)\cdot \ln(x) + \sin(x) \cdot \frac{1}{x}.
\]
Finally, we use the logarithmic differentiation formula to obtain $g'(x)$.
We have:
\[
g'(x) = g(x) \frac{d}{dx} \left[\ln(g(x))\right] = x^{\sin(x)} \left[\cos(x)\ln(x) + \frac{\sin(x)}{x}\right].
\]

\end{example}


\begin{problem}(problem 2)
  Compute
  \[
  \frac{d}{dx} \left(x^{\tan(x)}\right)
  \]
  
    \begin{hint}
      Use logarithmic differentiation
    \end{hint}
    \begin{hint}
      Use property 3 of logarithms
    \end{hint}
    \begin{hint}
		  Use the product rule to find the derivative of $\ln(g(x))$
		\end{hint}
		\begin{hint}
		  Remember to multiply by the original function
		\end{hint}
    
		The derivative of $x^{\tan(x)}$ with respect to $x$ is
		 $\answer{x^{\tan(x)} (\sec^2(x)\ln(x) + \frac{\tan(x)}{x})}$
		
\end{problem}


\begin{example}[example 3]
Use logarithmic differentiation to find the derivative of 
\[
g(x) = \frac{e^x \sin^4(x)}{\sqrt{x^4 + x^2 + 1}}.
\]
We begin by computing the logarithm of $g(x)$ using the properties of the logarithm.
\begin{align*}
\ln(g(x)) &= \ln\left(\frac{e^x \sin^4(x)}{\sqrt{x^4 + x^2 + 1}}\right)\\
&= \ln\left(e^x \sin^4(x)\right) - \ln\left(\sqrt{x^4 + x^2 + 1}\right)\\
&= \ln(e^x) + \ln(\sin^4(x)) - \ln[(x^4 + x^2 + 1)^{1/2}]\\
&= x + 4\ln(\sin(x)) - \frac12 \ln(x^4 + x^2 + 1).
\end{align*}

Note that we used the fact that the logarithm and the exponential functions are 
inverses in the last line
to write $\ln(e^x) = x$.
Now we can use the chain rule to compute the derivative of $\ln(g(x))$.
We have
\begin{align*}
\frac{d}{dx}\left[ \ln(g(x))\right] &= \frac{d}{dx}\left[ x + 4\ln(\sin(x)) - \frac12 \cdot \ln(x^4 + x^2 + 1)\right]\\
&= 1+ 4 \cdot \frac{\cos(x)}{\sin(x)} - \frac12 \cdot \frac{4x^3 + 2x}{x^4 + x^2 + 1}.
\end{align*}


Finally, we use the logarithmic differentiation formula to obtain $g'(x)$.
With a couple of minor simplifications, we have:
\[
g'(x) = \frac{e^x \sin^4(x)}{\sqrt{x^4 + x^2 + 1}} \left[1+ 4 \cot(x) - 
 \frac{2x^3 + x}{x^4 + x^2 + 1}\right].
\]

\end{example}

\begin{problem}(problem 3a)
  If 
	\[
	g(x) = \frac{\sqrt[4]{x^8 + x^4 + 2}}{e^{2x} \sec^2(x)}
	\]
	then which of the following is equal to $\ln(g(x))$?
  \begin{multipleChoice}
    \choice[correct]{$\frac14 \cdot \ln(x^8 + x^4 + 2) - 2x - 2\ln(\sec(x))$}
    \choice{$\frac14 \cdot \ln(x^8 + x^4 + 2) - 2x + 2\ln(\sec(x))$}
    \choice{$\frac14 \cdot \ln(x^8 + x^4 + 2) - 2x\ln(\sec(x))$}
  \end{multipleChoice}
\end{problem}



\begin{problem}(problem 3b)
  Which of the following is equal to 
  \[
  \frac{d}{dx} \frac{\sqrt[4]{x^8 + x^4 + 2}}{e^{2x} \sec^2(x)}?
  \]

  \begin{multipleChoice}
    \choice{\[\frac{2x^7 + x^3}{x^8+x^4 +2} -2 -2 \tan(x)\]}
    \choice[correct]{\[\frac{\sqrt[4]{x^8 + x^4 + 2}}{e^{2x} \sec^2(x)} \cdot
		                                    \left[\frac{2x^7 + x^3}{x^8+x^4 +2} -2 -2 \tan(x)\right]\]}
    \choice{\[\frac{\sqrt[4]{x^8 + x^4 + 2}}{e^{2x} \sec^2(x)} \cdot
		                                           \left[\frac{2x^7 + x^3}{x^8+x^4 +2} -2 -2 \sec(x)\right]\]}
  \end{multipleChoice}
   
		
\end{problem}

As a final note, even though $\ln(x)$ is only defined for $x>0$, the logarithmic differentiation formula given above 
will yield the correct result even in situations where $g(x) <0$. This is because 
\[
\frac{d}{dx} \ln|x| = \frac{1}{x},
\]
and so, by the chain rule,
\[
\frac{d}{dx} \ln|g(x)| = \frac{g'(x)}{g(x)}.
\]




\end{document}








