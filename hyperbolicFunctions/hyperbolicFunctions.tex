\documentclass[handout]{ximera}

%% You can put user macros here
%% However, you cannot make new environments



\newcommand{\ffrac}[2]{\frac{\text{\footnotesize $#1$}}{\text{\footnotesize $#2$}}}
\newcommand{\vasymptote}[2][]{
    \draw [densely dashed,#1] ({rel axis cs:0,0} -| {axis cs:#2,0}) -- ({rel axis cs:0,1} -| {axis cs:#2,0});
}


%\usepackage{tcolorbox} %%Needed for Derivative Definition supposedly and product rule, natural exp log, quotient rule, inverse trig, rates of change


% \graphicspath{{./}{firstExample/}}
% \usepackage{forest}
\usepackage{amsmath}
\usepackage{amssymb}
\usepackage{array}
\usepackage[makeroom]{cancel} %% for strike outs
\usepackage{pgffor} %% required for integral for loops
\usepackage{tikz}
\usepackage{tikz-cd}
\usepackage{tkz-euclide}
\usetikzlibrary{shapes.multipart}


% \usetkzobj{all}
\tikzstyle geometryDiagrams=[ultra thick,color=blue!50!black]


\usetikzlibrary{arrows}
\tikzset{>=stealth,commutative diagrams/.cd,
  arrow style=tikz,diagrams={>=stealth}} %% cool arrow head
\tikzset{shorten <>/.style={ shorten >=#1, shorten <=#1 } } %% allows shorter vectors

\usetikzlibrary{backgrounds} %% for boxes around graphs
\usetikzlibrary{shapes,positioning}  %% Clouds and stars
\usetikzlibrary{matrix} %% for matrix
\usepgfplotslibrary{polar} %% for polar plots
\usepgfplotslibrary{fillbetween} %% to shade area between curves in TikZ



%\usepackage[width=4.375in, height=7.0in, top=1.0in, papersize={5.5in,8.5in}]{geometry}
%\usepackage[pdftex]{graphicx}
%\usepackage{tipa}
%\usepackage{txfonts}
%\usepackage{textcomp}
%\usepackage{amsthm}
%\usepackage{xy}
%\usepackage{fancyhdr}
%\usepackage{xcolor}
%\usepackage{mathtools} %% for pretty underbrace % Breaks Ximera
%\usepackage{multicol}



\newcommand{\RR}{\mathbb R}
\newcommand{\R}{\mathbb R}
\newcommand{\C}{\mathbb C}
\newcommand{\N}{\mathbb N}
\newcommand{\Z}{\mathbb Z}
\newcommand{\dis}{\displaystyle}
%\renewcommand{\d}{\,d\!}
\renewcommand{\d}{\mathop{}\!d}
\newcommand{\dd}[2][]{\frac{\d #1}{\d #2}}
\newcommand{\pp}[2][]{\frac{\partial #1}{\partial #2}}
\renewcommand{\l}{\ell}
\newcommand{\ddx}{\frac{d}{\d x}}
\newcommand{\ppx}{\frac{\partial}{\partial x}}
\newcommand{\ppy}{\frac{\partial}{\partial y}}

\newcommand{\zeroOverZero}{\ensuremath{\boldsymbol{\tfrac{0}{0}}}}
\newcommand{\inftyOverInfty}{\ensuremath{\boldsymbol{\tfrac{\infty}{\infty}}}}
\newcommand{\zeroOverInfty}{\ensuremath{\boldsymbol{\tfrac{0}{\infty}}}}
\newcommand{\zeroTimesInfty}{\ensuremath{\small\boldsymbol{0\cdot \infty}}}
\newcommand{\inftyMinusInfty}{\ensuremath{\small\boldsymbol{\infty - \infty}}}
\newcommand{\oneToInfty}{\ensuremath{\boldsymbol{1^\infty}}}
\newcommand{\zeroToZero}{\ensuremath{\boldsymbol{0^0}}}
\newcommand{\inftyToZero}{\ensuremath{\boldsymbol{\infty^0}}}


\newcommand{\numOverZero}{\ensuremath{\boldsymbol{\tfrac{\#}{0}}}}
\newcommand{\dfn}{\textbf}
%\newcommand{\unit}{\,\mathrm}
\newcommand{\unit}{\mathop{}\!\mathrm}
%\newcommand{\eval}[1]{\bigg[ #1 \bigg]}
\newcommand{\eval}[1]{ #1 \bigg|}
\newcommand{\seq}[1]{\left( #1 \right)}
\renewcommand{\epsilon}{\varepsilon}
\renewcommand{\iff}{\Leftrightarrow}

\DeclareMathOperator{\arccot}{arccot}
\DeclareMathOperator{\arcsec}{arcsec}
\DeclareMathOperator{\arccsc}{arccsc}
\DeclareMathOperator{\si}{Si}
\DeclareMathOperator{\proj}{proj}
\DeclareMathOperator{\scal}{scal}
\DeclareMathOperator{\cis}{cis}
\DeclareMathOperator{\Arg}{Arg}
%\DeclareMathOperator{\arg}{arg}
\DeclareMathOperator{\Rep}{Re}
\DeclareMathOperator{\Imp}{Im}
\DeclareMathOperator{\sech}{sech}
\DeclareMathOperator{\csch}{csch}
\DeclareMathOperator{\Log}{Log}

\newcommand{\tightoverset}[2]{% for arrow vec
  \mathop{#2}\limits^{\vbox to -.5ex{\kern-0.75ex\hbox{$#1$}\vss}}}
\newcommand{\arrowvec}{\overrightarrow}
\renewcommand{\vec}{\mathbf}
\newcommand{\veci}{{\boldsymbol{\hat{\imath}}}}
\newcommand{\vecj}{{\boldsymbol{\hat{\jmath}}}}
\newcommand{\veck}{{\boldsymbol{\hat{k}}}}
\newcommand{\vecl}{\boldsymbol{\l}}
\newcommand{\utan}{\vec{\hat{t}}}
\newcommand{\unormal}{\vec{\hat{n}}}
\newcommand{\ubinormal}{\vec{\hat{b}}}

\newcommand{\dotp}{\bullet}
\newcommand{\cross}{\boldsymbol\times}
\newcommand{\grad}{\boldsymbol\nabla}
\newcommand{\divergence}{\grad\dotp}
\newcommand{\curl}{\grad\cross}
%% Simple horiz vectors
\renewcommand{\vector}[1]{\left\langle #1\right\rangle}


\pgfplotsset{compat=1.13}

\outcome{Learn the arithmetic of complex numbers}

\title{1.7 Hyperbolic Functions}

\begin{document}

\begin{abstract}
We learn the basic properties of the hyperbolic functions.
\end{abstract}

\maketitle

The hyperbolc sine and hyperbolic cosine functions are defined as
\[
\sinh x = \frac{e^x - e^{-x}}{2} \;\; \mbox{and} \;\; \cosh x = \frac{e^x + e^{-x}}{2}
\]

\begin{image}
\begin{tikzpicture}
\draw[<->, thick] (-3,0)--(3,0);

\draw[<->, thick] (0, 3)--(0,-3) node[ below=10pt, blue]{\large The Hyperbolic Sine, $y =\sinh x$};


\draw[ thick, domain=-1.7:1.7,smooth,variable=\x,blue] plot ({\x},{0.5*e^\x -0.5* e^-\x});


\draw[thin] (1,.2)--(1,-.2) node[below]{$1$};
\draw[thin] (-1,.2)--(-1,-.2) node[below]{$-1$};
\draw[thin] (2,.2)--(2,-.2) node[below]{$2$};
\draw[thin] (-2,.2)--(-2,-.2) node[below]{$-2$};

\draw[thin] (.2,1)--(-.2,1) node[left]{$1$};
\draw[thin] (.2,-1)--(-.2,-1) node[left]{$-1$};
\draw[thin] (.2,2)--(-.2,2) node[left]{$2$};
\draw[thin] (.2,-2)--(-.2,-2) node[left]{$-2$};


\end{tikzpicture}
\end{image}

\begin{image}
\begin{tikzpicture}
\draw[<->, thick] (-3,0)--(3,0);

\draw[<->, thick] (0, 3)--(0,-0.5) node[ below=10pt, blue]{The Hyperbolic Cosine, $y =\cosh x$};


\draw[ thick, domain=-1.7:1.7,smooth,variable=\x,blue] plot ({\x},{0.5*e^\x +0.5* e^-\x});


\draw[thin] (1,.2)--(1,-.2) node[below]{$1$};
\draw[thin] (-1,.2)--(-1,-.2) node[below]{$-1$};
\draw[thin] (2,.2)--(2,-.2) node[below]{$2$};
\draw[thin] (-2,.2)--(-2,-.2) node[below]{$-2$};

\draw[thin] (.2,1)--(-.2,1) node[left]{$1$};
\draw[thin] (.2,2)--(-.2,2) node[left]{$2$};



\end{tikzpicture}
\end{image}

\begin{example}[example 1]
Solve the equation $\cosh x = 6$ for $x$.\\
We use the definition of the hyperbolic cosine:
\[
\cosh x = \frac{e^x + e^{-x}}{2} = 6
\]
Ergo
\[
e^x +e^{-x} = 12
\]
Multiplying by $e^x$, we get
\[
e^{2x} - 12e^x +1 = 0 \;\; \mbox{or} \;\; u^2 - 12u + 1 = 0
\]
where $u = e^x$.  Now the quadratic formula gives
\[
u = \frac{12 \pm \sqrt{144 - 4}}{2} = 6 \pm \sqrt{35}
\]
Finally, we get the two solutions by taking the natural logarithm of $u$:
\[
x = \ln\left(6 \pm \sqrt{35}\right)
\]
These answers are negatives of one another (verify).
\end{example}

\begin{problem}[problem 1a]
Solve the following equation $\cosh x = 3$\\
The answers are (in increasing order):\\
$x = \answer{\ln(3-2\sqrt 2)}$ and $x = \answer{\ln(3+2\sqrt 2)}$
\end{problem}


\begin{problem}[problem 1b]
Solve the following equation $\sinh x = 3$\\
The answer is:\\
$x = \answer{\ln(3+\sqrt{10} 2)}$
\end{problem}

\begin{problem}(problem 1c) 
Find inverses for the hyperbolic sine and hyperbolic cosine functions.
\end{problem}

The hyperbolic sine function is an odd function:
\[
\sinh(-x) = -\sinh(x)
\]
and the hyperbolic cosine function is even:
\[
\cosh(-x) = -\cosh(x)
\]

The hyperbolic sine and cosine satisfy the fundamental identity
\[
\cosh^2 x - \sinh^2 x =1
\]
which means that the point $(\cosh t, \sinh t)$ lies on the (right branch of) the hyperbola
\[
x^2 -y^2 = 1
\]
This is why the functions are refered to as the hyperbolic functions.

The other four hyperbolic functions can be created from the hyperbolic sine and hyperbolic cosine functions:
\[
\tanh x = \frac{\sinh x}{\cosh x},\;\; \coth x = \frac{\cosh x}{\sinh x},
\]
\[
\sech x = \frac{1}{\cosh x}, \;\; \mbox{and}\;\; \csch x = \frac{1}{\sinh x}
\]

\begin{example}[example 2]
Find the derivative of the hyperbolic sine function.\\
The derivative of the hyperbolic sine function is given by
\[
\frac{d}{dx} \sinh(x) = \frac{d}{dx} \left( \frac{e^x -e^{-x}}{2} \right) = \frac{e^x + e^{-x}}{2} = \cosh x
\]
\end{example}

\begin{problem}(problem 2)
Find the derivatives of the other 5 hyperbolic functions.\\
$\frac{d}{dx} \cosh x = \answer{\sinh x}$\\
$\frac{d}{dx} \tanh x = \answer{\sech^2 x}$\\
$\frac{d}{dx} \sech x = \answer{-\sech x \tanh x}$\\
$\frac{d}{dx} \coth x = \answer{-\csch^2 x}$\\
$\frac{d}{dx} \csch x = \answer{-\csch x \coth x}$
\end{problem}


\begin{example}[example 3]
Find the Maclaurin series representation for the hyperbolic cosine.\\
Recall that the Maclaurin series for $e^x$ is given by
\[
e^x = \sum_{n=0}^\infty \frac{x^n}{n!}
\]
Thus
\[
\cosh x = \frac{e^x + e^{-x}}{2} = \frac12 \sum_{n=0}^\infty \frac{x^n}{n!} + \frac12 \sum_{n=0}^\infty \frac{(-x)^n}{n!}
\]
\[
= \sum_{\substack{n=0\\ n\; even}}^\infty \frac{x^n}{n!} = \sum_{n=0}^\infty \frac{x^{2n}}{(2n)!}
\]

\end{example}

\begin{problem}(problem 3)
Find the Maclauring series for the hyperbolic sine function.\\

\begin{multipleChoice}
\choice{$\sinh x=\displaystyle{\sum_{n=0}^\infty (-1)^n \frac{x^{2n+1}}{(2n+1)!}}$}
\choice{$\sinh x=\displaystyle{\sum_{n=0}^\infty  \frac{x^{2n+1}}{ n!}}$}
\choice[correct]{$\sinh x=\displaystyle{\sum_{n=0}^\infty  \frac{x^{2n+1}}{(2n+1)!}}$}
\end{multipleChoice}
\end{problem}


\end{document}

