\documentclass[handout]{ximera}

%% You can put user macros here
%% However, you cannot make new environments



\newcommand{\ffrac}[2]{\frac{\text{\footnotesize $#1$}}{\text{\footnotesize $#2$}}}
\newcommand{\vasymptote}[2][]{
    \draw [densely dashed,#1] ({rel axis cs:0,0} -| {axis cs:#2,0}) -- ({rel axis cs:0,1} -| {axis cs:#2,0});
}


%\usepackage{tcolorbox} %%Needed for Derivative Definition supposedly and product rule, natural exp log, quotient rule, inverse trig, rates of change


% \graphicspath{{./}{firstExample/}}
% \usepackage{forest}
\usepackage{amsmath}
\usepackage{amssymb}
\usepackage{array}
\usepackage[makeroom]{cancel} %% for strike outs
\usepackage{pgffor} %% required for integral for loops
\usepackage{tikz}
\usepackage{tikz-cd}
\usepackage{tkz-euclide}
\usetikzlibrary{shapes.multipart}


% \usetkzobj{all}
\tikzstyle geometryDiagrams=[ultra thick,color=blue!50!black]


\usetikzlibrary{arrows}
\tikzset{>=stealth,commutative diagrams/.cd,
  arrow style=tikz,diagrams={>=stealth}} %% cool arrow head
\tikzset{shorten <>/.style={ shorten >=#1, shorten <=#1 } } %% allows shorter vectors

\usetikzlibrary{backgrounds} %% for boxes around graphs
\usetikzlibrary{shapes,positioning}  %% Clouds and stars
\usetikzlibrary{matrix} %% for matrix
\usepgfplotslibrary{polar} %% for polar plots
\usepgfplotslibrary{fillbetween} %% to shade area between curves in TikZ



%\usepackage[width=4.375in, height=7.0in, top=1.0in, papersize={5.5in,8.5in}]{geometry}
%\usepackage[pdftex]{graphicx}
%\usepackage{tipa}
%\usepackage{txfonts}
%\usepackage{textcomp}
%\usepackage{amsthm}
%\usepackage{xy}
%\usepackage{fancyhdr}
%\usepackage{xcolor}
%\usepackage{mathtools} %% for pretty underbrace % Breaks Ximera
%\usepackage{multicol}



\newcommand{\RR}{\mathbb R}
\newcommand{\R}{\mathbb R}
\newcommand{\C}{\mathbb C}
\newcommand{\N}{\mathbb N}
\newcommand{\Z}{\mathbb Z}
\newcommand{\dis}{\displaystyle}
%\renewcommand{\d}{\,d\!}
\renewcommand{\d}{\mathop{}\!d}
\newcommand{\dd}[2][]{\frac{\d #1}{\d #2}}
\newcommand{\pp}[2][]{\frac{\partial #1}{\partial #2}}
\renewcommand{\l}{\ell}
\newcommand{\ddx}{\frac{d}{\d x}}
\newcommand{\ppx}{\frac{\partial}{\partial x}}
\newcommand{\ppy}{\frac{\partial}{\partial y}}

\newcommand{\zeroOverZero}{\ensuremath{\boldsymbol{\tfrac{0}{0}}}}
\newcommand{\inftyOverInfty}{\ensuremath{\boldsymbol{\tfrac{\infty}{\infty}}}}
\newcommand{\zeroOverInfty}{\ensuremath{\boldsymbol{\tfrac{0}{\infty}}}}
\newcommand{\zeroTimesInfty}{\ensuremath{\small\boldsymbol{0\cdot \infty}}}
\newcommand{\inftyMinusInfty}{\ensuremath{\small\boldsymbol{\infty - \infty}}}
\newcommand{\oneToInfty}{\ensuremath{\boldsymbol{1^\infty}}}
\newcommand{\zeroToZero}{\ensuremath{\boldsymbol{0^0}}}
\newcommand{\inftyToZero}{\ensuremath{\boldsymbol{\infty^0}}}


\newcommand{\numOverZero}{\ensuremath{\boldsymbol{\tfrac{\#}{0}}}}
\newcommand{\dfn}{\textbf}
%\newcommand{\unit}{\,\mathrm}
\newcommand{\unit}{\mathop{}\!\mathrm}
%\newcommand{\eval}[1]{\bigg[ #1 \bigg]}
\newcommand{\eval}[1]{ #1 \bigg|}
\newcommand{\seq}[1]{\left( #1 \right)}
\renewcommand{\epsilon}{\varepsilon}
\renewcommand{\iff}{\Leftrightarrow}

\DeclareMathOperator{\arccot}{arccot}
\DeclareMathOperator{\arcsec}{arcsec}
\DeclareMathOperator{\arccsc}{arccsc}
\DeclareMathOperator{\si}{Si}
\DeclareMathOperator{\proj}{proj}
\DeclareMathOperator{\scal}{scal}
\DeclareMathOperator{\cis}{cis}
\DeclareMathOperator{\Arg}{Arg}
%\DeclareMathOperator{\arg}{arg}
\DeclareMathOperator{\Rep}{Re}
\DeclareMathOperator{\Imp}{Im}
\DeclareMathOperator{\sech}{sech}
\DeclareMathOperator{\csch}{csch}
\DeclareMathOperator{\Log}{Log}

\newcommand{\tightoverset}[2]{% for arrow vec
  \mathop{#2}\limits^{\vbox to -.5ex{\kern-0.75ex\hbox{$#1$}\vss}}}
\newcommand{\arrowvec}{\overrightarrow}
\renewcommand{\vec}{\mathbf}
\newcommand{\veci}{{\boldsymbol{\hat{\imath}}}}
\newcommand{\vecj}{{\boldsymbol{\hat{\jmath}}}}
\newcommand{\veck}{{\boldsymbol{\hat{k}}}}
\newcommand{\vecl}{\boldsymbol{\l}}
\newcommand{\utan}{\vec{\hat{t}}}
\newcommand{\unormal}{\vec{\hat{n}}}
\newcommand{\ubinormal}{\vec{\hat{b}}}

\newcommand{\dotp}{\bullet}
\newcommand{\cross}{\boldsymbol\times}
\newcommand{\grad}{\boldsymbol\nabla}
\newcommand{\divergence}{\grad\dotp}
\newcommand{\curl}{\grad\cross}
%% Simple horiz vectors
\renewcommand{\vector}[1]{\left\langle #1\right\rangle}


\outcome{Determine differentiability}

\title{1.11 Differentiability}

\begin{document}

\begin{abstract}
We determine differentiability at a point
\end{abstract}

\maketitle

\begin{center}
\textbf{Differentiability}
\end{center}


\begin{definition}[Differentiability].  The function $f(x)$ is said to be 
\textbf{differentiable} at the point $x= x_0$
if the following limit exists:
\[
\lim_{h\to 0} \frac{f(x_0+h) -f(x_0)}{h}.
\]
We denote the limit by $f'(x_0)$, the derivative of $f$ at $x_0$.

\end{definition}

Geometrically, the derivative $f'(x_0)$ gives us the \it{slope} of a tangent line.  
Conceptually, it represents an \it{instantaneous} rate of change.
In this section, we will explore the three main reasons that a function is \textbf{not} differentiable at a point. These are
discontinuities, corner points, and vertical tangent lines.

\begin{theorem}
If $f(x)$ is differentiable at $x = x_0$, then $f(x)$ is continuous at $x=x_0$, i.e.,
differentiability implies continuity.
\end{theorem}

An immediate consequence of this theorem is that if $f(x)$ is \textbf{not} continuous at $x = x_0$,
then it cannot be differentiable there.  

\begin{corollary}
If $f(x)$ is not continuous at $x = x_0$ then $f(x)$ is not differentiable at $x = x_0$.
\end{corollary}

If the graph of a function has a removable, jump or infinite discontinuity, 
then the function is not differentiable at the corresponding point.

There are two other common reasons that a function might \textbf{not} be differentiable.

\begin{definition}[Corner Point] The function $f(x)$ has a \textbf{corner point} at $x = x_0$ if
the difference quotient has a jump discontinuity at $x = x_0$, i.e.,
\[
\lim_{h\to 0^-} \frac{f(x_0 +h)-f(x_0)}{h} \text{  and  }  \lim_{h\to 0^+} \frac{f(x_0 +h)-f(x_0)}{h}
\]
are both finite, but they are different.

\end{definition}

\begin{example}[example 1]
The function $f(x) = |x|$ has a corner point at $x = 0$.
The left-hand limit, 
\[
\lim_{h\to 0^-} \frac{f(x_0 +h)-f(x_0)}{h} = \lim_{h\to 0^-} \frac{|h|}{h} = -1
\]
whereas the right-hand limit,
\[
\lim_{h\to 0^+} \frac{f(x_0 +h)-f(x_0)}{h} = \lim_{h\to 0^+} \frac{|h|}{h} = 1.
\]

\[
\graph{abs(x)}
\]
\end{example}

The other common occurrence of a point of non-differentiability is at a vertical tangent line.


\begin{definition}[Vertical Tangent Line] The function $f(x)$ has a \textbf{vertical tangent line} at $x = x_0$ if
the vertical line $x = x_0$ is tangent to the graph of $y = f(x)$ at the point $(x_0, f(x_0))$.
\end{definition}

\begin{example}[example 2]
The function $f(x) = \sqrt[3] x$ has a vertical tangent line at $x = 0$.
\[
\graph{x^{1/3}}
\]
\end{example}


\begin{problem}(problem 1)
Below is the graph of $y = f(x)$.  Answer the questions below the graph.  You can zoom in on the graph if you need to.
\[
\graph{x^2}
\]
Is $f(x)$ differentiable at $x = 1$?
\begin{multipleChoice}
\choice[correct]{Yes}
\choice{No}
\end{multipleChoice}
If yes, is $f'(1)$ positive, negative or zero?
\begin{multipleChoice}
\choice[correct]{$f'(1) > 0$}
\choice{$f'(1) < 0$}
\choice{$f'(1) = 0$}
\end{multipleChoice}
\end{problem}




\begin{problem}(problem 2)
Below is the graph of $y = f(x)$.  Answer the questions below the graph.  You can zoom in on the graph if you need to.
\[
\graph{3-abs(x-1)}
\]
Is $f(x)$ differentiable at $x = 1$?
\begin{multipleChoice}
\choice{Yes}
\choice[correct]{No}
\end{multipleChoice}
If no, why not?
\begin{multipleChoice}
\choice{discontinuity at $x = 1$}
\choice[correct]{corner point at $x = 1$}
\choice{vertical tangent line at $x = 1$}
\end{multipleChoice}
\end{problem}



\begin{problem}(problem 3)
Below is the graph of $y = f(x)$.  Answer the questions below the graph.  You can zoom in on the graph if you need to.
\[
\graph{1 - (x-1)^{1/3}}
\]
Is $f(x)$ differentiable at $x = 1$?
\begin{multipleChoice}
\choice{Yes}
\choice[correct]{No}
\end{multipleChoice}
If no, why not?
\begin{multipleChoice}
\choice{discontinuity at $x = 1$}
\choice{corner point at $x = 1$}
\choice[correct]{vertical tangent line at $x = 1$}
\end{multipleChoice}
\end{problem}



\begin{problem}(problem 4)
Below is the graph of $y = f(x)$.  Answer the questions below the graph.  You can zoom in on the graph if you need to.
\[
\graph{(x-2)^2}
\]
Is $f(x)$ differentiable at $x = 1$?
\begin{multipleChoice}
\choice[correct]{Yes}
\choice{No}
\end{multipleChoice}
If yes, is $f'(1)$ positive, negative or zero?
\begin{multipleChoice}
\choice{$f'(1) > 0$}
\choice[correct]{$f'(1) < 0$}
\choice{$f'(1) = 0$}
\end{multipleChoice}
\end{problem}


\begin{problem}(problem 5)
Below is the graph of $y = f(x)$.  Answer the questions below the graph.  You can zoom in on the graph if you need to.
\[
\graph{1/(x-1)}
\]
Is $f(x)$ differentiable at $x = 1$?
\begin{multipleChoice}
\choice{Yes}
\choice[correct]{No}
\end{multipleChoice}
If no, why not?
\begin{multipleChoice}
\choice[correct]{discontinuity at $x = 1$}
\choice{corner point at $x = 1$}
\choice{vertical tangent line at $x = 1$}
\end{multipleChoice}
\end{problem}



\begin{problem}(problem 6)
Below is the graph of $y = f(x)$.  Answer the questions below the graph.  You can zoom in on the graph if you need to.
\[
\graph{1 + (x-1)^2}
\]
Is $f(x)$ differentiable at $x = 1$?
\begin{multipleChoice}
\choice[correct]{Yes}
\choice{No}
\end{multipleChoice}
If yes, is $f'(1)$ positive, negative or zero?
\begin{multipleChoice}
\choice{$f'(1) > 0$}
\choice{$f'(1) < 0$}
\choice[correct]{$f'(1) = 0$}
\end{multipleChoice}
\end{problem}


\end{document}






