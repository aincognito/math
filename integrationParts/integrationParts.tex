\documentclass{ximera}

%% You can put user macros here
%% However, you cannot make new environments



\newcommand{\ffrac}[2]{\frac{\text{\footnotesize $#1$}}{\text{\footnotesize $#2$}}}
\newcommand{\vasymptote}[2][]{
    \draw [densely dashed,#1] ({rel axis cs:0,0} -| {axis cs:#2,0}) -- ({rel axis cs:0,1} -| {axis cs:#2,0});
}


%\usepackage{tcolorbox} %%Needed for Derivative Definition supposedly and product rule, natural exp log, quotient rule, inverse trig, rates of change


% \graphicspath{{./}{firstExample/}}
% \usepackage{forest}
\usepackage{amsmath}
\usepackage{amssymb}
\usepackage{array}
\usepackage[makeroom]{cancel} %% for strike outs
\usepackage{pgffor} %% required for integral for loops
\usepackage{tikz}
\usepackage{tikz-cd}
\usepackage{tkz-euclide}
\usetikzlibrary{shapes.multipart}


% \usetkzobj{all}
\tikzstyle geometryDiagrams=[ultra thick,color=blue!50!black]


\usetikzlibrary{arrows}
\tikzset{>=stealth,commutative diagrams/.cd,
  arrow style=tikz,diagrams={>=stealth}} %% cool arrow head
\tikzset{shorten <>/.style={ shorten >=#1, shorten <=#1 } } %% allows shorter vectors

\usetikzlibrary{backgrounds} %% for boxes around graphs
\usetikzlibrary{shapes,positioning}  %% Clouds and stars
\usetikzlibrary{matrix} %% for matrix
\usepgfplotslibrary{polar} %% for polar plots
\usepgfplotslibrary{fillbetween} %% to shade area between curves in TikZ



%\usepackage[width=4.375in, height=7.0in, top=1.0in, papersize={5.5in,8.5in}]{geometry}
%\usepackage[pdftex]{graphicx}
%\usepackage{tipa}
%\usepackage{txfonts}
%\usepackage{textcomp}
%\usepackage{amsthm}
%\usepackage{xy}
%\usepackage{fancyhdr}
%\usepackage{xcolor}
%\usepackage{mathtools} %% for pretty underbrace % Breaks Ximera
%\usepackage{multicol}



\newcommand{\RR}{\mathbb R}
\newcommand{\R}{\mathbb R}
\newcommand{\C}{\mathbb C}
\newcommand{\N}{\mathbb N}
\newcommand{\Z}{\mathbb Z}
\newcommand{\dis}{\displaystyle}
%\renewcommand{\d}{\,d\!}
\renewcommand{\d}{\mathop{}\!d}
\newcommand{\dd}[2][]{\frac{\d #1}{\d #2}}
\newcommand{\pp}[2][]{\frac{\partial #1}{\partial #2}}
\renewcommand{\l}{\ell}
\newcommand{\ddx}{\frac{d}{\d x}}
\newcommand{\ppx}{\frac{\partial}{\partial x}}
\newcommand{\ppy}{\frac{\partial}{\partial y}}

\newcommand{\zeroOverZero}{\ensuremath{\boldsymbol{\tfrac{0}{0}}}}
\newcommand{\inftyOverInfty}{\ensuremath{\boldsymbol{\tfrac{\infty}{\infty}}}}
\newcommand{\zeroOverInfty}{\ensuremath{\boldsymbol{\tfrac{0}{\infty}}}}
\newcommand{\zeroTimesInfty}{\ensuremath{\small\boldsymbol{0\cdot \infty}}}
\newcommand{\inftyMinusInfty}{\ensuremath{\small\boldsymbol{\infty - \infty}}}
\newcommand{\oneToInfty}{\ensuremath{\boldsymbol{1^\infty}}}
\newcommand{\zeroToZero}{\ensuremath{\boldsymbol{0^0}}}
\newcommand{\inftyToZero}{\ensuremath{\boldsymbol{\infty^0}}}


\newcommand{\numOverZero}{\ensuremath{\boldsymbol{\tfrac{\#}{0}}}}
\newcommand{\dfn}{\textbf}
%\newcommand{\unit}{\,\mathrm}
\newcommand{\unit}{\mathop{}\!\mathrm}
%\newcommand{\eval}[1]{\bigg[ #1 \bigg]}
\newcommand{\eval}[1]{ #1 \bigg|}
\newcommand{\seq}[1]{\left( #1 \right)}
\renewcommand{\epsilon}{\varepsilon}
\renewcommand{\iff}{\Leftrightarrow}

\DeclareMathOperator{\arccot}{arccot}
\DeclareMathOperator{\arcsec}{arcsec}
\DeclareMathOperator{\arccsc}{arccsc}
\DeclareMathOperator{\si}{Si}
\DeclareMathOperator{\proj}{proj}
\DeclareMathOperator{\scal}{scal}
\DeclareMathOperator{\cis}{cis}
\DeclareMathOperator{\Arg}{Arg}
%\DeclareMathOperator{\arg}{arg}
\DeclareMathOperator{\Rep}{Re}
\DeclareMathOperator{\Imp}{Im}
\DeclareMathOperator{\sech}{sech}
\DeclareMathOperator{\csch}{csch}
\DeclareMathOperator{\Log}{Log}

\newcommand{\tightoverset}[2]{% for arrow vec
  \mathop{#2}\limits^{\vbox to -.5ex{\kern-0.75ex\hbox{$#1$}\vss}}}
\newcommand{\arrowvec}{\overrightarrow}
\renewcommand{\vec}{\mathbf}
\newcommand{\veci}{{\boldsymbol{\hat{\imath}}}}
\newcommand{\vecj}{{\boldsymbol{\hat{\jmath}}}}
\newcommand{\veck}{{\boldsymbol{\hat{k}}}}
\newcommand{\vecl}{\boldsymbol{\l}}
\newcommand{\utan}{\vec{\hat{t}}}
\newcommand{\unormal}{\vec{\hat{n}}}
\newcommand{\ubinormal}{\vec{\hat{b}}}

\newcommand{\dotp}{\bullet}
\newcommand{\cross}{\boldsymbol\times}
\newcommand{\grad}{\boldsymbol\nabla}
\newcommand{\divergence}{\grad\dotp}
\newcommand{\curl}{\grad\cross}
%% Simple horiz vectors
\renewcommand{\vector}[1]{\left\langle #1\right\rangle}

\usetikzlibrary{matrix,positioning}

\outcome{Compute an integral using integration by parts}

\title{Integration by Parts}

\begin{document}

\begin{abstract}
We will compute integrals using the integration by parts technique.
\end{abstract}

\maketitle

%\begin{center}
%\textbf{Integration by Parts}
%\end{center}

We will use the integration by parts (IBP) formula to compute integrals involving products of unrelated factors.


\begin{theorem}[Integration by Parts]
\[
\int u \; dv = uv - \int v \; du
\]
\end{theorem}






\begin{example} %example #1
Compute the integral using IBP:
  \[
  \int x\sin(x) \;dx.
  \]
  
  The integrand is the product of two factors, $x$ and $\sin(x)$.  We consider one factor to differentiate and the other factor to anti-differentiate.
  The factor we prefer to differentiate is $x$ since the derivative of $x$ is a constant and constants are easy to deal with in integrals.
  As for the $\sin(x)$ factor, whether we differentiate it or anti-differentiate it makes little difference.  In fact, only a change in sign occurs,
  as 
  \[\frac{d}{dx}{\sin(x)} = \cos(x)    \text{ \;\;   and \;\;   }   \int \sin(x) \;dx = -\cos(x) + C.\]
  Thus to apply the IBP formula, we let
  \[u=x \text{  and  }  dv = \sin(x) \;dx.\]
  Next, we compute
  \[du = dx  \text{  \;\;  and \;\;   }  v = -\cos(x).\]
  \begin{remark} 
  We do not need to add the constant of integration to v, as any anti-derivative will suffice 
  for our purpose and we choose the constant of integration to be zero.
  \end{remark}
  Now we use the Integration by Parts formula, $\int u\;dv = uv-\int v\; du$:
  \begin{align*}
  \int x\sin(x) &= -x\cos(x) - \int -\cos(x) \;dx\\
                &= -x\cos(x) + \int \cos(x) \;dx\\
                &= -x\cos(x) + \sin(x) + C.
 \end{align*}
 
  We have found an answer, but at this stage we should prove that it is the correct answer by differentiating it.  The result should be $x\sin(x)$.
  Note that the differentiation requires the product rule on the $uv$ term: $-x\cos(x)$.
  We have
  \[
  \frac{d}{dx}{\left[-x\cos(x) + \sin(x) + C\right]} = [-\cos(x) + x\sin(x)] + \cos(x) = x\sin(x)
  \]
  which proves that our answer was correct.
  
    
\end{example}

\begin{center}
\begin{foldable}
\unfoldable{Here is a video of Example 1}
%\youtube{Yy6QXnFlnXs} %vid of example 1
\end{foldable}
\end{center}


\begin{problem} %problem #1
  Compute the integral using IBP:
  \[
  \int xe^x \;dx.
  \]
  
  Let $u = \answer{x}$   and   $dv = \answer{e^x dx}$.\\
  Then $du = \answer{dx}$   and   $v = \answer{e^x}$.\\
  Thus 
  \[
  \int xe^x dx = xe^x - \int \answer{e^x} dx = \answer{xe^x - e^x} + C.
  \]
  


  %\begin{hint}
   %   $\int udv = uv-\int vdu$
  %\end{hint}
    

\end{problem}


\begin{problem} %problem #1
  Compute the integral using IBP:
  \[
  \int x\cos(x) \;dx.
  \]
  
  Let $u = \answer{x}$   and   $dv = \answer{\cos(x) \;dx}$.\\
  Then $du = \answer{dx}$   and   $v = \answer{\sin(x)}$.\\
  Thus 
  \[
  \int x\cos(x) dx = x\sin(x) - \int \answer{\sin(x)} \;dx = \answer{x\sin(x) + \cos(x)} + C.
  \]
  


  %\begin{hint}
   %   $\int udv = uv-\int vdu$
  %\end{hint}
    

\end{problem}


\begin{example}
Compute the integral using IBP:
  \[
  \int 3x\sin(4x) \;dx.
  \]

Let $u = 3x$ and $dv = \sin(4x) \,dx$.\\
Then $du = 3 \,dx$ and $v = -\frac14 \cos(4x)$.
Next, the IBP formula gives
\begin{align*}
  \int 3x\sin(4x) \;dx &= (3x)(-\frac14 \cos(4x)) - \int -\frac14 \cos(4x) \cdot 3 \; dx \\
                       &= -\frac34 x\cos(4x) + \frac34 \int \cos(4x) \; dx \\
                       &= -\frac34 x\cos(4x) + \frac34 \cdot \frac 14 \sin(4x) + C\\
                       &= -\frac34 x\cos(4x) + \frac{3}{16}\sin(4x) + C.
\end{align*}

\end{example}


\begin{problem}
Compute the integral using IBP:
  \[
  \int 2x\sin(5x)\;dx.
  \]

\end{problem}


\begin{problem}
Compute the integral using IBP:
  \[
  \int 6xe^{-3x} \;dx.
  \]

\end{problem}


\begin{problem}
Compute the integral using IBP:
  \[
  \int x\sin(x/2) \;dx.
  \]

\end{problem}


\begin{example}[IBP twice]
Compute the integral using IBP:
  \[
  \int x^2\sin(3x) \;dx.
  \]

\[
\text{Let} \;\; u = x^2 \;\; \text{and} \;\; dv = \sin(3x) \,dx,
\]
\[
\text{then} \;\; du = 2x \,dx \;\; \text{and} \;\; v = -\frac13 \cos(3x).
\]

Next, the IBP formula gives
\begin{align*}
\int x^2\sin(3x) \;dx &= (x^2)\left[-\frac13 \cos(3x)\right] - \int -\frac13 \cos(3x) \cdot 2x \; dx \\
                       &= -\frac13 x^2\cos(3x) + \frac13 \int 2x\cos(3x) \; dx\\                               
\text{we now use IBP} & \text{ again with \; $u = 2x$ \; and \; $dv = \cos(3x) \, dx$}\\
& \qquad \qquad \quad \text{$du = 2 \, dx$ \; and \; $v=\frac13 \sin(3x)$}\\
                       &= -\frac13 x^2\cos(3x) + \frac29x\sin(3x) - \frac13 \int \frac13 \sin(3x)\cdot 2 \; dx\\
                       &= -\frac13 x^2\cos(3x) + \frac29x\sin(3x) + \frac{2}{27} \cos(3x) + C.
\end{align*}

\end{example}


\begin{problem}
Compute the integral using IBP:
  \[
  \int x^2\cos(2x) \;dx.
  \]

\end{problem}

\begin{problem}
Compute the integral using IBP:
  \[
  \int x^2e^{-x} \;dx.
  \]

\end{problem}

\begin{center}
\textbf{Tabular Integration}
\end{center}

Integrals of the form $\int x^n f(x) \,dx$ require IBP $n$ times when $f(x)$ is sine, cosine or exponential function.
A notational shortcut called tabular integration can be used to streamline the process of using IBP multiple times.
As an example, let's revisit the following problem found above:

\begin{example}[Tabular Integration]
Compute the integral using Tabular Integration:
  \[
  \int x^2\sin(3x) \;dx.
  \]

We will construct a table that lists the derivatives of $u$ and the anti-derivatives of $dv$.
We can then obtain the final answer by a series of multiplications and additions or subtractions, as indicated
by the arrows in the table.


\begin{image}[5cm]
\begin{tikzpicture}
 \matrix (m) [matrix of nodes, ampersand replacement=\&,
            column sep = 1.5cm]{
    u     \&                  dv \\\hline
    $x^2$ \& $\sin(3x)$ \\ [8pt]
     $2x$  \& $-\frac13\cos(3x)$ \\ [8pt]
      $2$ \& $-\frac19\sin(3x)$ \\ [8pt]
       $0$ \& $\frac{1}{27}\cos(3x)$ \\
 };
 \draw[-latex] (m-2-1) -- (m-3-2) node[midway,above]{$+$};  % <-- It works!
 \draw[-latex] (m-3-1) -- (m-4-2) node[midway,above]{$-$};  % <-- It works!
 \draw[-latex] (m-4-1) -- (m-5-2) node[midway,above]{$+$};  % <-- It works!
 %\node at (-.6, 1.5){$+$};
 %\node at (-.6, 0.3){$-$};
 %\node at (-.6, -0.9){$+$};
 %\draw (,) -- (,);
\end{tikzpicture}
\end{image}

Thus,
\[
  \int x^2\sin(3x) \;dx = -\frac13 x^2\cos(3x) + \frac29x\sin(3x) + \frac{2}{27} \cos(3x) + C.
  \]

\end{example}



\begin{example}[Tabular Integration]
Compute the integral using Tabular Integration:
  \[
  \int x^4 e^{-2x} \;dx.
  \]
We construct a table with the derivatives of $u = x^4$ and the anti-derivatives of $dv = e^{-2x} \, dx$.
We add arrows which indicate multiplication of an element of the $u$ column with elements in the $dv$ column one row lower.
We adorn the arrows with alternating $+$ and $-$ signs to indicate either addition or subtraction of terms.
\begin{image}[5cm]
\begin{tikzpicture}
 \matrix (m) [matrix of nodes, ampersand replacement=\&,
            column sep = 1.5cm]{
    u     \&                  dv \\\hline
    $x^4$ \& $e^{-2x}$ \\ [8pt]
     $4x^3$  \& $-\frac12 e^{-2x}$ \\ [8pt]
      $12x^2$ \& $\frac14 e^{-2x}$ \\ [8pt]
       $24x$ \& $-\frac18 e^{-2x}$ \\[8pt]
       $24$ \& $\frac{1}{16} e^{-2x}$ \\[8pt]
       $0$ \& $-\frac{1}{32} e^{-2x}$ \\
 };
 \draw[-latex] (m-2-1) -- (m-3-2) node[midway,above]{$+$};  % <-- It works!
 \draw[-latex] (m-3-1) -- (m-4-2) node[midway,above]{$-$};  % <-- It works!
 \draw[-latex] (m-4-1) -- (m-5-2) node[midway,above]{$+$};  % <-- It works!
 \draw[-latex] (m-5-1) -- (m-6-2) node[midway,above]{$-$};  % <-- It works!
 \draw[-latex] (m-6-1) -- (m-7-2) node[midway,above]{$+$};  % <-- It works!
 %\node at (-.6, 1.5){$+$};
 %\node at (-.6, 0.3){$-$};
 %\node at (-.6, -0.9){$+$};
 %\draw (,) -- (,);
\end{tikzpicture}
\end{image}

Thus,
\[
  \int x^4 e^{-2x} \;dx = -\frac12 x^4 e^{-2x} - \frac44x^3 e^{-2x} - \frac{12}{8} x^2 e^{-2x} - \frac{24}{16}x e^{-2x} - \frac{24}{32} e^{-2x} + C.
  \]
\[
= \left(-\frac12 x^4 - x^3  - \frac{3}{2} x^2  - \frac{3}{2}x  - \frac{3}{4}\right) e^{-2x} + C.
\]
\end{example}


\begin{problem}[Tabular Integration]
Compute the integral using Tabular Integration:
  \[
  \int x^3 \cos(5x) \;dx.
  \]

\begin{image}
\begin{tikzpicture}
 \matrix (m) [matrix of nodes, ampersand replacement=\&,
            column sep = 1.5cm]{
    u     \&                  dv \\\hline
    $x^3$ \& $\cos(5x)$ \\ [8pt]
     $\answer{3x^2}$  \& $\answer{\frac15 \sin(5x)}$ \\ [8pt]
      $\answer{6x}$ \& $\answer{-\frac{1}{25} \cos(5x)}$ \\ [8pt]
       $\answer{6}$ \& $\answer{-\frac{1}{125} \sin(5x)}$ \\[8pt]
       $0$ \& $\answer{\frac{1}{625} \cos(5x)}$\\     
 };
 \draw[-latex] (m-2-1) -- (m-3-2) node[midway,above]{$+$};  % <-- It works!
 \draw[-latex] (m-3-1) -- (m-4-2) node[midway,above]{$-$};  % <-- It works!
 \draw[-latex] (m-4-1) -- (m-5-2) node[midway,above]{$+$};  % <-- It works!
 \draw[-latex] (m-5-1) -- (m-6-2) node[midway,above]{$-$};  % <-- It works!
\end{tikzpicture}
\end{image}

Thus,
\[
  \int x^3 \cos(5x) \;dx = \answer{\frac15 x^3\sin(5x) + \frac{3}{25}x^2 \cos(5x) - \frac{6}{125}\sin(5x) -\frac{6}{625}\cos(5x)} +C.
  \]
\end{problem}


\begin{center}
\begin{foldable}
\unfoldable{Here is a detailed, lecture style video on integration by parts:}
\youtube{2jSr68xaAfs}
\end{foldable}
\end{center}





\end{document}


\begin{example} %example #15
Find $h'(x)$ if $h(x) = x^{\sin(x)}$.\\
We will use the fact that the exponential and logarithm functions are inverses,
\[e^{\ln(x)} = x,\]
and the exponent property of logarithms, 
\[\ln(x^n) = n\ln(x),\]
to rewrite $h(x)$.  We have 
\[h(x) = x^{\sin(x)} = e^{\ln(x^{\sin(x)})} = e^{\sin(x)\ln(x)}\]
and we can now compute $h'(x)$ using a combination of the chain rule and product rule.
We can write $h(x)$ as a composition, $f(g(x))$ with 
\[g(x) = \sin(x)\ln(x) \quad \text{and} \quad f(x) = e^x.\]
Then to find $g'(x)$ we us the product rule and we get $g'(x) = \frac{\sin(x)}{x} + \cos(x)\ln(x)$.
Next $f'(x) = e^x$ and 
hence $f'(g(x)) = e^{g(x)} = e^{\sin(x)\ln(x)} = x^{\sin(x)}$.
We can then conclude $h'(x) = f'(g(x))g'(x) = x^{\sin(x)} \left[ \frac{\sin(x)}{x} + \cos(x)\ln(x)\right]$.
\end{example}

%more question formats below













%\begin{verbatim}
\begin{question}
What is your favorite color?
\begin{multipleChoice}
\choice[correct]{Rainbow}
\choice{Blue}
\choice{Green}
\choice{Red}
\end{multipleChoice}
\begin{freeResponse}
Hello
\end{freeResponse}
\end{question}
%\end{verbatim}





\begin{question}
  Which one will you choose?
  \begin{multipleChoice}
    \choice[correct]{I'm correct.}
    \choice{I'm wrong.}
    \choice{I'm wrong too.}
  \end{multipleChoice}
\end{question}


\begin{question}
  Which one will you choose?
  \begin{selectAll}
    \choice[correct]{I'm correct.}
    \choice{I'm wrong.}
    \choice[correct]{I'm also correct.}
    \choice{I'm wrong too.}
  \end{selectAll}
\end{question}


\begin{freeResponse}
What is the chain rule used for?
\end{freeResponse}
