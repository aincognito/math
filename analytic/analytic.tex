\documentclass[handout]{ximera}

%% You can put user macros here
%% However, you cannot make new environments



\newcommand{\ffrac}[2]{\frac{\text{\footnotesize $#1$}}{\text{\footnotesize $#2$}}}
\newcommand{\vasymptote}[2][]{
    \draw [densely dashed,#1] ({rel axis cs:0,0} -| {axis cs:#2,0}) -- ({rel axis cs:0,1} -| {axis cs:#2,0});
}


\graphicspath{{./}{firstExample/}}
\usepackage{forest}
\usepackage{amsmath}
\usepackage{amssymb}
\usepackage{array}
\usepackage[makeroom]{cancel} %% for strike outs
\usepackage{pgffor} %% required for integral for loops
\usepackage{tikz}
\usepackage{tikz-cd}
\usepackage{tkz-euclide}
\usetikzlibrary{shapes.multipart}


%\usetkzobj{all}
\tikzstyle geometryDiagrams=[ultra thick,color=blue!50!black]


\usetikzlibrary{arrows}
\tikzset{>=stealth,commutative diagrams/.cd,
  arrow style=tikz,diagrams={>=stealth}} %% cool arrow head
\tikzset{shorten <>/.style={ shorten >=#1, shorten <=#1 } } %% allows shorter vectors

\usetikzlibrary{backgrounds} %% for boxes around graphs
\usetikzlibrary{shapes,positioning}  %% Clouds and stars
\usetikzlibrary{matrix} %% for matrix
\usepgfplotslibrary{polar} %% for polar plots
\usepgfplotslibrary{fillbetween} %% to shade area between curves in TikZ



%\usepackage[width=4.375in, height=7.0in, top=1.0in, papersize={5.5in,8.5in}]{geometry}
%\usepackage[pdftex]{graphicx}
%\usepackage{tipa}
%\usepackage{txfonts}
%\usepackage{textcomp}
%\usepackage{amsthm}
%\usepackage{xy}
%\usepackage{fancyhdr}
%\usepackage{xcolor}
%\usepackage{mathtools} %% for pretty underbrace % Breaks Ximera
%\usepackage{multicol}



\newcommand{\RR}{\mathbb R}
\newcommand{\R}{\mathbb R}
\newcommand{\C}{\mathbb C}
\newcommand{\N}{\mathbb N}
\newcommand{\Z}{\mathbb Z}
\newcommand{\dis}{\displaystyle}
%\renewcommand{\d}{\,d\!}
\renewcommand{\d}{\mathop{}\!d}
\newcommand{\dd}[2][]{\frac{\d #1}{\d #2}}
\newcommand{\pp}[2][]{\frac{\partial #1}{\partial #2}}
\renewcommand{\l}{\ell}
\newcommand{\ddx}{\frac{d}{\d x}}

\newcommand{\zeroOverZero}{\ensuremath{\boldsymbol{\tfrac{0}{0}}}}
\newcommand{\inftyOverInfty}{\ensuremath{\boldsymbol{\tfrac{\infty}{\infty}}}}
\newcommand{\zeroOverInfty}{\ensuremath{\boldsymbol{\tfrac{0}{\infty}}}}
\newcommand{\zeroTimesInfty}{\ensuremath{\small\boldsymbol{0\cdot \infty}}}
\newcommand{\inftyMinusInfty}{\ensuremath{\small\boldsymbol{\infty - \infty}}}
\newcommand{\oneToInfty}{\ensuremath{\boldsymbol{1^\infty}}}
\newcommand{\zeroToZero}{\ensuremath{\boldsymbol{0^0}}}
\newcommand{\inftyToZero}{\ensuremath{\boldsymbol{\infty^0}}}


\newcommand{\numOverZero}{\ensuremath{\boldsymbol{\tfrac{\#}{0}}}}
\newcommand{\dfn}{\textbf}
%\newcommand{\unit}{\,\mathrm}
\newcommand{\unit}{\mathop{}\!\mathrm}
%\newcommand{\eval}[1]{\bigg[ #1 \bigg]}
\newcommand{\eval}[1]{ #1 \bigg|}
\newcommand{\seq}[1]{\left( #1 \right)}
\renewcommand{\epsilon}{\varepsilon}
\renewcommand{\iff}{\Leftrightarrow}

\DeclareMathOperator{\arccot}{arccot}
\DeclareMathOperator{\arcsec}{arcsec}
\DeclareMathOperator{\arccsc}{arccsc}
\DeclareMathOperator{\si}{Si}
\DeclareMathOperator{\proj}{proj}
\DeclareMathOperator{\scal}{scal}
\DeclareMathOperator{\cis}{cis}
\DeclareMathOperator{\Arg}{Arg}
%\DeclareMathOperator{\arg}{arg}
\DeclareMathOperator{\Rep}{Re}
\DeclareMathOperator{\Imp}{Im}
\DeclareMathOperator{\sech}{sech}
\DeclareMathOperator{\csch}{csch}
\DeclareMathOperator{\Log}{Log}

\newcommand{\tightoverset}[2]{% for arrow vec
  \mathop{#2}\limits^{\vbox to -.5ex{\kern-0.75ex\hbox{$#1$}\vss}}}
\newcommand{\arrowvec}{\overrightarrow}
\renewcommand{\vec}{\mathbf}
\newcommand{\veci}{{\boldsymbol{\hat{\imath}}}}
\newcommand{\vecj}{{\boldsymbol{\hat{\jmath}}}}
\newcommand{\veck}{{\boldsymbol{\hat{k}}}}
\newcommand{\vecl}{\boldsymbol{\l}}
\newcommand{\utan}{\vec{\hat{t}}}
\newcommand{\unormal}{\vec{\hat{n}}}
\newcommand{\ubinormal}{\vec{\hat{b}}}

\newcommand{\dotp}{\bullet}
\newcommand{\cross}{\boldsymbol\times}
\newcommand{\grad}{\boldsymbol\nabla}
\newcommand{\divergence}{\grad\dotp}
\newcommand{\curl}{\grad\cross}
%% Simple horiz vectors
\renewcommand{\vector}[1]{\left\langle #1\right\rangle}


\pgfplotsset{compat=1.13}

\outcome{Determine Analyticity}

\title{3.4 Analytic Functions}

\begin{document}

\begin{abstract}
We determine if a function is analytic using the Cauchy-Riemann equations.
\end{abstract}

\maketitle

\begin{definition}
We say that $f(z)$ is analytic at $z_0$ if $f$ is differentiable on $D(z_0, r)$ for some $r>0$.
Furthermore, $f(z)$ is called analytic on a region $R$ if it is analytic at each point in $R$.
\end{definition}


\begin{example}[example 1]
Recall from section 3.2 that the function $f(z) = \overline{z}^2$ is differentiable 
at $z=0$ and nowhere else. Thus,
$f(z) = \overline{z}^2$ is not differentiable on any disk of positive radius. 
Hence, $f(z) = \overline{z}^2$ is nowhere analytic.
\end{example}

\begin{problem}(problem 1)
Where is the function $f(z) = (x^3 + 3xy^2) +i(3x^2y+y^3)$ analytic?\\
$u_x = \answer{3x^2+3y^2} \quad \text{and} \quad u_y = \answer{6xy}$\\
$v_x = \answer{6xy} \quad \text{and} \quad v_y = \answer{3x^2 + 3y^2}$\\
Complete the sentence:  $f$ is differentiable
\begin{multipleChoice}
\choice{on the real axis only}
\choice{on the imaginary axis only}
\choice[correct]{on both the real and imaginary axes}
\choice{at the origin only}
\choice{nowhere}
\end{multipleChoice}
Complete the sentence:  $f$ is analytic
\begin{multipleChoice}
\choice{on the real axis only}
\choice{on the imaginary axis only}
\choice{on both the real and imaginary axes}
\choice{at the origin only}
\choice[correct]{nowhere}
\end{multipleChoice}
\end{problem}

Here is a video solution to problem 1:\\
\begin{foldable}
\youtube{BvrgO-VIXPM}
\end{foldable}

\begin{example}[example 2]
The function $f(z) = e^z$ is differentiable at every point in $\C$ and hence 
it is analytic everywhere in $\C$.
\end{example}

\begin{problem}(problem 2a)
Where is the function $f(z) = \cos(z)$ analytic?\\
\begin{multipleChoice}
\choice{on punctured plane}
\choice{on the slit plane}
\choice[correct]{on entire plane}
\choice{at the origin only}
\choice{nowhere}
\end{multipleChoice}
\end{problem}

\begin{problem}(problem 2b)
Where is the function $f(z) = \Log(z)$ analytic?\\
\begin{multipleChoice}
\choice{on punctured plane}
\choice[correct]{on the slit plane}
\choice{on entire plane}
\choice{at the origin only}
\choice{nowhere}
\end{multipleChoice}
\end{problem}

\begin{problem}(problem 2c)
Where is the function $f(z) = \frac{1}{z}$ analytic?\\
\begin{multipleChoice}
\choice[correct]{on punctured plane}
\choice{on the slit plane}
\choice{on entire plane}
\choice{at the origin only}
\choice{nowhere}
\end{multipleChoice}
\end{problem}

Like the real variable case, an analytic function whose derivative is zero is a constant.

\begin{theorem}
Suppose $f(z)$ is analytic in a disk, $D$. If $f'(z) =0$ on $D$,
then $f(z)$ is constant on $D$.
\end{theorem}
\begin{proof}
First note that since $f'=0$ on $D$, the partial derivatives $u_x, u_y, v_x$ and $v_y$ 
are all zero on the disk.
Suppose $z_1, z_2 \in D$ with $\Imp z_1 = \Imp z_2$. Then on the horizontal segment 
between $z_1$ and $z_2$, $u_x = 0$ implies that $u$ is constant 
on the segment and $v_x =0$ on the segment implies that $v$ is constant on the segment as well.
Now suppose $z_1, z_2 \in D$ with $\Rep z_1 = \Rep z_2$. Then on the vertical segment 
between $z_1$ and $z_2$, $u_y = 0$ implies that $u$ is constant 
on the segment and $v_y =0$ on the segment implies that $v$ is constant on the segment as well.
Finally, any two points $z_1, z_2 \in D$ can be connected by a horizontal and a 
vertical segment and $f$ is constant on these segments, so $f(z_1) = f(z_2)$.
Since $z_1$ and $z_2$ were arbitrary, $f(z)$ is constant on $D$. 
\end{proof}

In a vane similar to the previous theorem, if the modulus of an analytic function is constant, 
then the function itself is constant.

\begin{theorem}
Suppose $f(z)$ is analytic in a disk $D$. If $|f(z)| = m$ on  $D$, then $f(z)$ is constant on $D$.
\end{theorem}

\begin{proof}
If $m =0$, the $f$ is identically zero, so suppose $m>0$. The definition of modulus gives
\[
u^2(x,y) + v^2(x,y) = m^2
\]
Taking the partial derivative of both sides with respect to $x$ and with respect to $y$ gives the system
\[
2uu_x +2vv_x = 0
\]
\[
2uu_y + 2vv_y = 0
\]
Dividing by $2$ and using the Cauchy-Riemann equations selectively gives
\[
uu_x -vu_y = 0
\]
\[
uu_y + vu_x = 0
\]
Multiplying the first equation by $u$ and the second equation by $v$ gives
\[
u^2u^x - uvu_y=0
\]
\[
uvu_y + v^2u_x = 0
\]
adding these equations together gives
\[
\left(u^2 +v^2\right)u_x = m^2u_x = 0
\]
which gives $u_x = 0$ on $D$. We could have eliminated $u_x$ instead of $u_y$ which would 
have led to $u_y =0$ on $D$.
Hence $u$ is constant on $D$. Moreover, by the Cauchy-Riemann equations $v_x=0$ and $v_y=0$ on $D$ and 
so $v$ is constant on $D$ as well.
Thus $f = u+iv$ is constant on $D$.
\end{proof}

\begin{problem}(linear system)
Solve the system for $u_y$:
\[
uu_x -vu_y = 0
\]
\[
uu_y + vu_x = 0
\]
\begin{multipleChoice}
\choice[correct]{$u_y= 0$}
\choice{$u_y = -u_x$}
\choice{$u_y = m^2$}
\end{multipleChoice}
\end{problem}


\begin{definition}
A function that is analytic at every point in $\C$ is called {\bf entire}.
\end{definition}


\begin{problem}(entire)
Select all of the functions below which are entire.
\begin{selectAll}
\choice[correct] {$z^3 + 2iz^2 -5z -i$}
\choice[correct] {$e^z$}
\choice[correct] {$\sin z$}
\choice[correct] {$\cos z$}
\choice {$1/z$}
\choice {$\Log z$}
\end{selectAll}
\end{problem}


%f' analytic on a disk and f' = 0 implies f is constant
%Schwartz reflection principle
%derivative of inverse
%fundamental theorem of algebra
%conformal mapping

\end{document}
 
 
 
 
 
 
 
 
