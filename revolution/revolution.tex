\documentclass{ximera}

%% You can put user macros here
%% However, you cannot make new environments



\newcommand{\ffrac}[2]{\frac{\text{\footnotesize $#1$}}{\text{\footnotesize $#2$}}}
\newcommand{\vasymptote}[2][]{
    \draw [densely dashed,#1] ({rel axis cs:0,0} -| {axis cs:#2,0}) -- ({rel axis cs:0,1} -| {axis cs:#2,0});
}


%\usepackage{tcolorbox} %%Needed for Derivative Definition supposedly and product rule, natural exp log, quotient rule, inverse trig, rates of change


% \graphicspath{{./}{firstExample/}}
% \usepackage{forest}
\usepackage{amsmath}
\usepackage{amssymb}
\usepackage{array}
\usepackage[makeroom]{cancel} %% for strike outs
\usepackage{pgffor} %% required for integral for loops
\usepackage{tikz}
\usepackage{tikz-cd}
\usepackage{tkz-euclide}
\usetikzlibrary{shapes.multipart}


% \usetkzobj{all}
\tikzstyle geometryDiagrams=[ultra thick,color=blue!50!black]


\usetikzlibrary{arrows}
\tikzset{>=stealth,commutative diagrams/.cd,
  arrow style=tikz,diagrams={>=stealth}} %% cool arrow head
\tikzset{shorten <>/.style={ shorten >=#1, shorten <=#1 } } %% allows shorter vectors

\usetikzlibrary{backgrounds} %% for boxes around graphs
\usetikzlibrary{shapes,positioning}  %% Clouds and stars
\usetikzlibrary{matrix} %% for matrix
\usepgfplotslibrary{polar} %% for polar plots
\usepgfplotslibrary{fillbetween} %% to shade area between curves in TikZ



%\usepackage[width=4.375in, height=7.0in, top=1.0in, papersize={5.5in,8.5in}]{geometry}
%\usepackage[pdftex]{graphicx}
%\usepackage{tipa}
%\usepackage{txfonts}
%\usepackage{textcomp}
%\usepackage{amsthm}
%\usepackage{xy}
%\usepackage{fancyhdr}
%\usepackage{xcolor}
%\usepackage{mathtools} %% for pretty underbrace % Breaks Ximera
%\usepackage{multicol}



\newcommand{\RR}{\mathbb R}
\newcommand{\R}{\mathbb R}
\newcommand{\C}{\mathbb C}
\newcommand{\N}{\mathbb N}
\newcommand{\Z}{\mathbb Z}
\newcommand{\dis}{\displaystyle}
%\renewcommand{\d}{\,d\!}
\renewcommand{\d}{\mathop{}\!d}
\newcommand{\dd}[2][]{\frac{\d #1}{\d #2}}
\newcommand{\pp}[2][]{\frac{\partial #1}{\partial #2}}
\renewcommand{\l}{\ell}
\newcommand{\ddx}{\frac{d}{\d x}}
\newcommand{\ppx}{\frac{\partial}{\partial x}}
\newcommand{\ppy}{\frac{\partial}{\partial y}}

\newcommand{\zeroOverZero}{\ensuremath{\boldsymbol{\tfrac{0}{0}}}}
\newcommand{\inftyOverInfty}{\ensuremath{\boldsymbol{\tfrac{\infty}{\infty}}}}
\newcommand{\zeroOverInfty}{\ensuremath{\boldsymbol{\tfrac{0}{\infty}}}}
\newcommand{\zeroTimesInfty}{\ensuremath{\small\boldsymbol{0\cdot \infty}}}
\newcommand{\inftyMinusInfty}{\ensuremath{\small\boldsymbol{\infty - \infty}}}
\newcommand{\oneToInfty}{\ensuremath{\boldsymbol{1^\infty}}}
\newcommand{\zeroToZero}{\ensuremath{\boldsymbol{0^0}}}
\newcommand{\inftyToZero}{\ensuremath{\boldsymbol{\infty^0}}}


\newcommand{\numOverZero}{\ensuremath{\boldsymbol{\tfrac{\#}{0}}}}
\newcommand{\dfn}{\textbf}
%\newcommand{\unit}{\,\mathrm}
\newcommand{\unit}{\mathop{}\!\mathrm}
%\newcommand{\eval}[1]{\bigg[ #1 \bigg]}
\newcommand{\eval}[1]{ #1 \bigg|}
\newcommand{\seq}[1]{\left( #1 \right)}
\renewcommand{\epsilon}{\varepsilon}
\renewcommand{\iff}{\Leftrightarrow}

\DeclareMathOperator{\arccot}{arccot}
\DeclareMathOperator{\arcsec}{arcsec}
\DeclareMathOperator{\arccsc}{arccsc}
\DeclareMathOperator{\si}{Si}
\DeclareMathOperator{\proj}{proj}
\DeclareMathOperator{\scal}{scal}
\DeclareMathOperator{\cis}{cis}
\DeclareMathOperator{\Arg}{Arg}
%\DeclareMathOperator{\arg}{arg}
\DeclareMathOperator{\Rep}{Re}
\DeclareMathOperator{\Imp}{Im}
\DeclareMathOperator{\sech}{sech}
\DeclareMathOperator{\csch}{csch}
\DeclareMathOperator{\Log}{Log}

\newcommand{\tightoverset}[2]{% for arrow vec
  \mathop{#2}\limits^{\vbox to -.5ex{\kern-0.75ex\hbox{$#1$}\vss}}}
\newcommand{\arrowvec}{\overrightarrow}
\renewcommand{\vec}{\mathbf}
\newcommand{\veci}{{\boldsymbol{\hat{\imath}}}}
\newcommand{\vecj}{{\boldsymbol{\hat{\jmath}}}}
\newcommand{\veck}{{\boldsymbol{\hat{k}}}}
\newcommand{\vecl}{\boldsymbol{\l}}
\newcommand{\utan}{\vec{\hat{t}}}
\newcommand{\unormal}{\vec{\hat{n}}}
\newcommand{\ubinormal}{\vec{\hat{b}}}

\newcommand{\dotp}{\bullet}
\newcommand{\cross}{\boldsymbol\times}
\newcommand{\grad}{\boldsymbol\nabla}
\newcommand{\divergence}{\grad\dotp}
\newcommand{\curl}{\grad\cross}
%% Simple horiz vectors
\renewcommand{\vector}[1]{\left\langle #1\right\rangle}


\outcome{Find the volume of a solid of revolution}

\title{Solids of Revolution}

\begin{document}

\begin{abstract}
We use disks, washers and shells to find the volume of a solid of revolution.
\end{abstract}

\maketitle

We now find the volume of a solid of revolution.  

\begin{definition}[Solid of revolution] A \textbf{solid of revolution} is obtained by revolving a region in the plane about 
a line.  The line is referred to as the \textbf{axis of revolution}.

\end{definition}

As we will soon see, some familiar solids like cylinders, cones and spheres are solids of revolution.
The main formula we will need from geometry is the area of a circle of radius $r$: $A = \pi r^2$. 

\section{Disks}

Consider a vertical line segment 2 units long that touches the $x$-axis at one end.  

(SEE FIGURE)


If we revolve this segment about the axis, it will make a circle of radius 2 whose area is $4\pi$.
As in the previous section, we will associate a thickness, $dx$, to the infinitely narrow segment.
Then after revolving the segment, we get a \textbf{disk} whose volume is 
\[
V = \pi r^2 h = \pi \cdot 2^2 \cdot dx = 4\pi dx.
\]

We now consider an example where we apply this idea to find the volume of a solid of revolution.

\begin{example}[Volume of a Cone]
In this example, we find the volume of a cone with height $h$ and base radius $r$.
To create the cone, consider the line $y = \frac{r}{h} x$ which goes through the origin and the point $(h,r)$.
Our cone is obtained by revolving the region between this line and the $x$-axis over the interval $[0,h]$

(INSERT FIGURE HERE)

We imagine our cone as being constructed by revolving each of the vertical segments in our region individually.
For any value of $x$ between $x = 0$ and $x = h$, the length of the segment is given by the corresponding $y$-coordinate
on the line, $y = \frac{r}{h} x$. Assuming that this segment has thickness $dx$, the volume of the resulting disk is 
\[
v_x = \pi r^2 dx = \pi y^2 \; dx = \pi \left(\frac{r}{h} s\right)^2 \; dx = \pi \frac{r^2}{h^2} x^2 \; dx.
\]

(INSERT FIGURE HERE)

The volume of the solid of revolution, i.e., our cone, is then obtained by adding the volumes of these infinitely thin disks using a definite integral:
\[
V = \int_0^h V_x \; dx = \int_0^h \pi r^2 \; dx = \int_0^h \pi \frac{r^2}{h^2} x^2 \; dx 
\]
\[
= \pi \frac{r^2}{h^2}\int_0^h  x^2 \; dx = \pi \frac{r^2}{h^2} \left(\frac{x^3}{3}\right)\bigg|_0^h 
\]
\[
= \pi \frac{r^2}{h^2} \left(\frac{h^3}{3} - 0\right) = \pi \frac{r^2}{h^2} \cdot \frac{h^3}{3} = \frac13 \pi r^2 h.
\]

Thus, the volume of a ccone of height $h$ and base radius $r$ is $V = \frac13 \pi r^2 h$, i.e., its volume is 1/3 of 
volume of the cylinder with the same height and radius.

(INSERT FIGURE HERE)

\end{example}





\begin{example}[Volume of a Sphere]
In this example, we find the volume of a sphere with radius $r$.
To create the sphere, consider the semi-circle or radius $r$ centered at the origin: $y = \sqrt{r^2 - x^2}$.
Our sphere is obtained by revolving the region between this semi-circle and the $x$-axis over the interval $[-r,r]$

(INSERT FIGURE HERE)

We imagine the sphere as being constructed by revolving each of the vertical segments in our region individually.
For any value of $x$ between $x = -r$ and $x = r$, the length of the segment is given by the corresponding $y$-coordinate
on the semi-circle, $y = \sqrt{r^2 - x^2}$. Assuming that this segment has thickness $dx$, the volume of the resulting disk is 
\[
v_x = \pi r^2 dx = \pi y^2 \; dx = \pi \left(\sqrt{r^2 - x^2}\right)^2 dx = \pi (r^2 - x^2)\; dx.
\]

(INSERT FIGURE HERE)

The volume of the solid of revolution, i.e., our sphere, is then obtained by adding the volumes of these infinitely thin disks using a definite integral:
\[
V = \int_{-r}^r V_x \; dx = \int_{-r}^r \pi r^2 \; dx = \int_{-r}^r \pi (r^2 - x^2) \; dx 
\]
\[
= \pi \int_{-r}^r  (r^2 - x^2) \; dx = \pi  \left(r^2 x - \frac{x^3}{3}\right)\bigg|_{-r}^r 
\]
\[
= \pi \left[ \left(r^3 - \frac{r^3}{3}\right) - \left(-r^3 + \frac{r^3}{3}\right)\right] = \pi \left(\frac{2r^3}{3} + \frac{2r^3}{3}\right) = \frac43 \pi r^3.
\]

Thus, the volume of a sphere of radius $r$ is $V = \frac43 \pi r^3$.

(INSERT FIGURE HERE)

\end{example}

\begin{example} Find the volume of the solid obtained by revolving the region bounded by the graphs of $y=x^2, y=0, x=0$, and $x = 1$
about the $x$-axis.\\
We begin with a sketch of the region.

(INSERT FIGURE HERE)

To compute the volume of the resulting solid, fix a value of $x$ between 0 and 1 and consider a vertical segment in the region at this $x$-value.

(INSERT FIGURE HERE)

Revolving this individual segment about the $x$-axis yields a disk with radius $r = x^2$ and thickness $dx$. 

(INSERT FIGURE HERE)

The volume of this disk is then
\[
V_x = \pi r^2 h = \pi (x^2)^2 \; dx = \pi x^4 \; dx.
\]
The volume of the solid can be otained by integrating $V_x$ as $x$ ranges from 0 to 1:
\[
V = \int_0^1 V_x \; dx = \int_0^1 \pi x^4 \; dx = \pi \frac{x^5}{5}\bigg|_0^1 = \frac{\pi}{5}.
\]

(INSERT GEOGEBRA APPLET HERE)

\end{example}


\begin{example} Find the volume of the solid obtained by revolving the region bounded by the graphs of $y=\sqrt, y=0, x=0$, and $x = 1$
about the $x$-axis.\\
We begin with a sketch of the region.

(INSERT FIGURE HERE)

To compute the volume of the resulting solid, fix a value of $x$ between 0 and 1 and consider a vertical segment in the region at this $x$-value.

(INSERT FIGURE HERE)

Revolving this individual segment about the $x$-axis yields a disk with radius $r = x^2$ and thickness $dx$. 

(INSERT FIGURE HERE)

The volume of this disk is then
\[
V_x = \pi r^2 h = \pi (\sqrt x)^2 \; dx = \pi x \; dx.
\]
The volume of the solid can be otained by integrating $V_x$ as $x$ ranges from 0 to 1:
\[
V = \int_0^1 V_x \; dx = \int_0^1 \pi x \; dx = \pi \frac{x^2}{2}\bigg|_0^1 = \frac{\pi}{2}.
\]


(INSERT GEOGEBRA APPLET HERE)

\end{example}



\section{Washers} In the next series of examples, the region will not border the axis of revolution. 
In this situation, the resulting cross-sections are washers rather than disks.
The area of a washer with inner radius, $r$, and outer radius, $R$, is given by 
\[
A = \pi(R^2 - r^2).
\]

(INSERT FIGURE HERE)

If we attribute an infinitesimal thickness, $dx$, to the washer, then its volume is
\[
V = \pi(R^2 - r^2) h = \pi(R^2 - r^2) \; dx.
\]
We now use this to calculuate volumes of solids of revolutions about the $x$-axis in which the region does not border the axis.

\begin{example} Find the volume of the solid obtained by revolving the region bounded by the graphs of 
$y = e^x, y = 3, x = 0$, and $x = 1$ about the $x$-axis.\\
We begin with a sketch of the region.

(INSERT FIGURE HERE)

To compute the volume of the resulting solid, fix a value of $x$ between 0 and 1 and consider a vertical segment in the region at this $x$-value.

(INSERT FIGURE HERE)

Revolving this individual segment about the $x$-axis yields a washer with inner radius, $r = e^x$, outer radius, $R = 3$, and thickness, $dx$. 

(INSERT FIGURE HERE)

The volume of this washer is then
\[
V_x = \pi (R^2 - r^2) h = \pi \left[3^2 - \left(e^x\right)^2\right] \; dx = \pi \left(9 - e^{2x}\right) \; dx.
\]
The volume of the solid can be otained by integrating $V_x$ as $x$ ranges from 0 to 1:
\[
V = \int_0^1 V_x \; dx = \int_0^1 \pi \left(9 - e^{2x}\right) \; dx = \pi \left(9x - \frac{e^{2x}}{2}\right)\bigg|_0^1 = \left(10-\frac{e^2}{2}\right)\pi.
\]


(INSERT GEOGEBRA APPLET HERE)

\end{example}


\begin{example} Find the volume of the solid obtained by revolving the region bounded by the graphs of 
$y = \sec(x), y = 2, x = 0$, and $x = \frac{\pi}{4}$ about the $x$-axis.\\
We begin with a sketch of the region.

(INSERT FIGURE HERE)

To compute the volume of the resulting solid, fix a value of $x$ between 0 and $\frac{\pi}{4}$ and consider a vertical segment in the region at this $x$-value.

(INSERT FIGURE HERE)

Revolving this individual segment about the $x$-axis yields a washer with inner radius, $r = \sec(x)$, outer radius, $R = 2$, and thickness, $dx$. 

(INSERT FIGURE HERE)

The volume of this washer is then
\[
V_x = \pi (R^2 - r^2) h = \pi \left[2^2 - \sec^2(x)\right] \; dx = \pi \left[4 - \sec^2(x)\right] \; dx.
\]
The volume of the solid can be otained by integrating $V_x$ as $x$ ranges from 0 to $\pi/4$:
\[
V = \int_0^{\pi/4} V_x \; dx = \int_0^{\pi/4} \pi \left[4-\sec^2(x)\right] \; dx = \pi \left[4x - \tan(x)\right]\bigg|_0^{\pi/4} = \pi^2 - \pi.
\]


(INSERT GEOGEBRA APPLET HERE)

\end{example}



\section{Cylindrical shells}


We now find the volume of solids of revolution obtained by revolving a region about a vertical axis.

\begin{example} Find the volume of the solid obtained by revolving the region bounded by the graphs of $y = x^2, y=0, x=0$, and $x = 1$ about the $y$-axis.\\
We begin with a sketch of the region.

(INSERT FIGURE HERE)

To compute the volume of the resulting solid, fix a value of $x$ between 0 and 1 and consider a vertical segment in the region at this $x$-value.

(INSERT FIGURE HERE)

Revolving this individual segment about the $y$-axis yields a cylindrical shell with radius, $r = x,$ height, $h = x^2$, and thickness, $dx$. 

(INSERT FIGURE HERE)

The volume of this shell is the surface area of the shell multiplied by its thickness:
\[
V_x = 2\pi rh \;dx = 2\pi x(x^2) \; dx = 2\pi x^3 \; dx.
\]
The volume of the solid can be otained by integrating $V_x$ as $x$ ranges from 0 to 1:
\[
V = \int_0^1 V_x \; dx = \int_0^1 2\pi x^3 \; dx = 2\pi \frac{x^4}{4}\bigg|_0^1 = \frac{\pi}{2}.
\]


(INSERT GEOGEBRA APPLET HERE)

\end{example}


\begin{example} Find the volume of the solid obtained by revolving the region bounded by the graphs of $y = \sqrt x, y=2, x=0$, and $x = 1$ about the $y$-axis.\\
We begin with a sketch of the region.

(INSERT FIGURE HERE)

To compute the volume of the resulting solid, fix a value of $x$ between 0 and 1 and consider a vertical segment in the region at this $x$-value.

(INSERT FIGURE HERE)

Revolving this individual segment about the $y$-axis yields a cylindrical shell with radius, $r = x,$ height, $h = 2 - \sqrt x$, and thickness, $dx$. 

(INSERT FIGURE HERE)

The volume of this shell is the surface area of the shell multiplied by its thickness:
\[
V_x = 2\pi rh \;dx = 2\pi x(2 - \sqrt x) \; dx = 2\pi \left(2x - x^{3/2}\right) \; dx.
\]
The volume of the solid can be otained by integrating $V_x$ as $x$ ranges from 0 to 1:
\[
V = \int_0^1 V_x \; dx = \int_0^1 2\pi \left(2x - x^{3/2}\right) \; dx = 2\pi \left(x^2 - \frac25 x^{5/2}\right)\bigg|_0^1 = \frac{6\pi}{5}.
\]


(INSERT GEOGEBRA APPLET HERE)

\end{example}


































\section{Video Lesson}




\begin{center}
\begin{foldable}
\unfoldable{Here is a video of Example 1}
%\youtube{} %vid of example 1
\end{foldable}
\end{center}




\end{document}




%multiple choice format; note the question environment; what is free response? what is verbatim?

%\begin{verbatim}
\begin{question}
What is your favorite color?
\begin{multipleChoice}
\choice[correct]{Rainbow}
\choice{Blue}
\choice{Green}
\choice{Red}
\end{multipleChoice}
\begin{freeResponse}
Hello
\end{freeResponse}
\end{question}
%\end{verbatim}


%note the indentation

\begin{question}
  Which one will you choose?
  \begin{multipleChoice}
    \choice[correct]{I'm correct.}
    \choice{I'm wrong.}
    \choice{I'm wrong too.}
  \end{multipleChoice}
\end{question}


%selectAll

\begin{question}
  Which one will you choose?
  \begin{selectAll}
    \choice[correct]{I'm correct.}
    \choice{I'm wrong.}
    \choice[correct]{I'm also correct.}
    \choice{I'm wrong too.}
  \end{selectAll}
\end{question}


\begin{freeResponse}
What is the chain rule used for?
\end{freeResponse}
