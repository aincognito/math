\documentclass[handout]{ximera}

%% You can put user macros here
%% However, you cannot make new environments



\newcommand{\ffrac}[2]{\frac{\text{\footnotesize $#1$}}{\text{\footnotesize $#2$}}}
\newcommand{\vasymptote}[2][]{
    \draw [densely dashed,#1] ({rel axis cs:0,0} -| {axis cs:#2,0}) -- ({rel axis cs:0,1} -| {axis cs:#2,0});
}


%\usepackage{tcolorbox} %%Needed for Derivative Definition supposedly and product rule, natural exp log, quotient rule, inverse trig, rates of change


% \graphicspath{{./}{firstExample/}}
% \usepackage{forest}
\usepackage{amsmath}
\usepackage{amssymb}
\usepackage{array}
\usepackage[makeroom]{cancel} %% for strike outs
\usepackage{pgffor} %% required for integral for loops
\usepackage{tikz}
\usepackage{tikz-cd}
\usepackage{tkz-euclide}
\usetikzlibrary{shapes.multipart}


% \usetkzobj{all}
\tikzstyle geometryDiagrams=[ultra thick,color=blue!50!black]


\usetikzlibrary{arrows}
\tikzset{>=stealth,commutative diagrams/.cd,
  arrow style=tikz,diagrams={>=stealth}} %% cool arrow head
\tikzset{shorten <>/.style={ shorten >=#1, shorten <=#1 } } %% allows shorter vectors

\usetikzlibrary{backgrounds} %% for boxes around graphs
\usetikzlibrary{shapes,positioning}  %% Clouds and stars
\usetikzlibrary{matrix} %% for matrix
\usepgfplotslibrary{polar} %% for polar plots
\usepgfplotslibrary{fillbetween} %% to shade area between curves in TikZ



%\usepackage[width=4.375in, height=7.0in, top=1.0in, papersize={5.5in,8.5in}]{geometry}
%\usepackage[pdftex]{graphicx}
%\usepackage{tipa}
%\usepackage{txfonts}
%\usepackage{textcomp}
%\usepackage{amsthm}
%\usepackage{xy}
%\usepackage{fancyhdr}
%\usepackage{xcolor}
%\usepackage{mathtools} %% for pretty underbrace % Breaks Ximera
%\usepackage{multicol}



\newcommand{\RR}{\mathbb R}
\newcommand{\R}{\mathbb R}
\newcommand{\C}{\mathbb C}
\newcommand{\N}{\mathbb N}
\newcommand{\Z}{\mathbb Z}
\newcommand{\dis}{\displaystyle}
%\renewcommand{\d}{\,d\!}
\renewcommand{\d}{\mathop{}\!d}
\newcommand{\dd}[2][]{\frac{\d #1}{\d #2}}
\newcommand{\pp}[2][]{\frac{\partial #1}{\partial #2}}
\renewcommand{\l}{\ell}
\newcommand{\ddx}{\frac{d}{\d x}}
\newcommand{\ppx}{\frac{\partial}{\partial x}}
\newcommand{\ppy}{\frac{\partial}{\partial y}}

\newcommand{\zeroOverZero}{\ensuremath{\boldsymbol{\tfrac{0}{0}}}}
\newcommand{\inftyOverInfty}{\ensuremath{\boldsymbol{\tfrac{\infty}{\infty}}}}
\newcommand{\zeroOverInfty}{\ensuremath{\boldsymbol{\tfrac{0}{\infty}}}}
\newcommand{\zeroTimesInfty}{\ensuremath{\small\boldsymbol{0\cdot \infty}}}
\newcommand{\inftyMinusInfty}{\ensuremath{\small\boldsymbol{\infty - \infty}}}
\newcommand{\oneToInfty}{\ensuremath{\boldsymbol{1^\infty}}}
\newcommand{\zeroToZero}{\ensuremath{\boldsymbol{0^0}}}
\newcommand{\inftyToZero}{\ensuremath{\boldsymbol{\infty^0}}}


\newcommand{\numOverZero}{\ensuremath{\boldsymbol{\tfrac{\#}{0}}}}
\newcommand{\dfn}{\textbf}
%\newcommand{\unit}{\,\mathrm}
\newcommand{\unit}{\mathop{}\!\mathrm}
%\newcommand{\eval}[1]{\bigg[ #1 \bigg]}
\newcommand{\eval}[1]{ #1 \bigg|}
\newcommand{\seq}[1]{\left( #1 \right)}
\renewcommand{\epsilon}{\varepsilon}
\renewcommand{\iff}{\Leftrightarrow}

\DeclareMathOperator{\arccot}{arccot}
\DeclareMathOperator{\arcsec}{arcsec}
\DeclareMathOperator{\arccsc}{arccsc}
\DeclareMathOperator{\si}{Si}
\DeclareMathOperator{\proj}{proj}
\DeclareMathOperator{\scal}{scal}
\DeclareMathOperator{\cis}{cis}
\DeclareMathOperator{\Arg}{Arg}
%\DeclareMathOperator{\arg}{arg}
\DeclareMathOperator{\Rep}{Re}
\DeclareMathOperator{\Imp}{Im}
\DeclareMathOperator{\sech}{sech}
\DeclareMathOperator{\csch}{csch}
\DeclareMathOperator{\Log}{Log}

\newcommand{\tightoverset}[2]{% for arrow vec
  \mathop{#2}\limits^{\vbox to -.5ex{\kern-0.75ex\hbox{$#1$}\vss}}}
\newcommand{\arrowvec}{\overrightarrow}
\renewcommand{\vec}{\mathbf}
\newcommand{\veci}{{\boldsymbol{\hat{\imath}}}}
\newcommand{\vecj}{{\boldsymbol{\hat{\jmath}}}}
\newcommand{\veck}{{\boldsymbol{\hat{k}}}}
\newcommand{\vecl}{\boldsymbol{\l}}
\newcommand{\utan}{\vec{\hat{t}}}
\newcommand{\unormal}{\vec{\hat{n}}}
\newcommand{\ubinormal}{\vec{\hat{b}}}

\newcommand{\dotp}{\bullet}
\newcommand{\cross}{\boldsymbol\times}
\newcommand{\grad}{\boldsymbol\nabla}
\newcommand{\divergence}{\grad\dotp}
\newcommand{\curl}{\grad\cross}
%% Simple horiz vectors
\renewcommand{\vector}[1]{\left\langle #1\right\rangle}


\pgfplotsset{compat=1.13}

\outcome{Find limits of complex sequences}

\title{4.1 Complex Sequences}

\begin{document}

\begin{abstract}
We find limits of complex sequences.
\end{abstract}

\maketitle


\begin{definition} 
A complex sequence $\{c_n\}$ is a function from the natural numbers to the complex numbers, $\C$.
\end{definition}

\begin{definition}
A complex sequence $\{c_n\}$ is said to {\bf converge} to a complex number $c$ if given $\epsilon >0$, there exists $N \in \N$
such that 
\[
|c_n - c| < \epsilon \quad \text{whenever} \quad n>N
\]
If no such complex number $c$ exists, then we say the sequence $\{c_n\}$ {\bf diverges}.
\end{definition}

 

\begin{example}[example 1]
Show that the sequence $\displaystyle \left\{\frac{i^n}{n}\right\}$ converges to $0$.\\
Let $\epsilon >0$ and let $N > 1/\epsilon$. Then for $n> N$ we have
\[
\left|\frac{i^n}{n} -0\right| = \frac{1}{n} < \frac{1}{N} < \epsilon
\]
Hence, the sequence $\displaystyle \left\{\frac{i^n}{n}\right\}$ converges to $0$.
\end{example}

\begin{problem}
Show that the sequence $\displaystyle \left\{\frac{i^n}{2^n}\right\}$ converges to $0$.
\begin{hint}
Let $\epsilon >0$
\end{hint}
\begin{hint}
Let $N > -\log_2\epsilon$
\end{hint}
\begin{hint}
If $n>N$ then $2^n > 2^{-\log_2\epsilon}$
\end{hint}
\end{problem}

\begin{problem}
Show that the sequence $\displaystyle \left\{\frac{n+i}{in}\right\}$ converges to $-i$.
\begin{hint}
Simplify $\displaystyle \left|\frac{n+i}{in} - (-i) \right|$
\end{hint}
\end{problem}

The next proposition shows that convergence of complex sequences can be determined by the convergence of its real and imaginary parts.

\begin{proposition}
A complex sequence $\{c_n\} = \{a_n + ib_n\}$ converges to $c = a+bi$ if and only if the real 
sequences $\{a_n\}$ and $\{b_n\}$ converge to $a$ and $b$ respectively.
\end{proposition}
\begin{proof}
$\left(\Rightarrow\right)$ Suppose that the real sequences $\{a_n\}$ and $\{b_n\}$ converge to $a$ and $b$ respectively, 
and consider the complex sequence $\{c_n\} = \{a_n + ib_n\}$. With $c = a+bi$, the triangle inequality gives
\[
|c_n - c| = |(a_n + ib_n) - (a+bi)| = |(a_n -a) + i(b_n -b)| \leq |a_n-a|+ |b_n -b|
\]
Now by the convergence of $\{a_n\}$ and $\{b_n\}$, for any $\epsilon >0$ there exists an integer $N$ such that $n>N$ implies
\[
|a_n-a| < \frac{\epsilon}{2} \quad \text{and} \quad  |b_n -b| < \frac{\epsilon}{2}
\]
and hence, for $n>N$ we have
\[
|c_n-c| < \frac{\epsilon}{2}+\frac{\epsilon}{2}= \epsilon 
\]
Thus $\{c_n\}$ converges to $c$.
$\left(\Leftarrow\right)$ \; Suppose the complex sequence $\{c_n\} = \{a_n + ib_n\}$ converges to $c = a+bi$.
Let $\epsilon >0$, then since $\{c_n\}$ converges to $c$, we can choose $N$ so that for $n>N$, $|c_n - c| < \epsilon$.
Using the fact that $|z| \geq |\Rep z|$ and $|z| \geq |\Imp z|$, we have 
\[
|c_n - c| = |(a_n + ib_n) - (a+bi)| = |(a_n -a) + i(b_n -b)| \geq \max\left\{|a_n-a|, |b_n -b|\right\}
\]
we can conclude that for $n>N$
\[
|a_n-a| < \epsilon \quad \text{and} \quad  |b_n -b| < \epsilon
\]
Thus $\{a_n\}$ converges to $a$ and $\{b_n\}$ converges to $b$.\\

\end{proof}

\begin{example}
Show that the complex sequence $\displaystyle \left\{\tan^{-1} n + i\sqrt[n]n\right\}$ converges to $\displaystyle \frac{\pi}{2} + i$.\\
Since the real sequences $\displaystyle \left\{\tan^{-1} n\right\}$ and $\displaystyle \left\{\sqrt[n]n\right\}$ converge to 
$\displaystyle \frac{\pi}{2}$ and $1$ respectively, the complex sequence 
$\displaystyle \left\{\tan^{-1} n + i\sqrt[n]n\right\}$ converges to $\displaystyle \frac{\pi}{2} + i$.
\end{example}

\begin{problem}
Determine if the following sequences converge or diverge.\\
a) $\dis \;\; \left\{\frac{n}{n+1}+ i\frac{\cos n}{n} \right\} \quad$ \wordChoice{\choice[correct]{converges}\choice{diverges}} \quad limit: $\answer{1}$\\
b) $\dis\;\;  \left\{\frac{n^2}{2^n}+ i\sqrt[n]{n^2 +1} \right\} \quad$ \wordChoice{\choice[correct]{converges}\choice{diverges}} \quad limit: $\answer{i}$\\
c) $\dis\;\; \left\{\left(1+\frac{1}{n}\right)^n+ i\frac{\ln n}{n} \right\} \quad$ \wordChoice{\choice[correct]{converges}\choice{diverges}} \quad limit: $\answer{e}$\\
d) $\dis\;\; \left\{\frac{n^2}{1+n^3}+i\frac{n^2}{1+n} \right\} \quad$ \wordChoice{\choice{converges}\choice[correct]{diverges}} 

\end{problem}


\end{document}




