\documentclass[handout]{ximera}

%% You can put user macros here
%% However, you cannot make new environments



\newcommand{\ffrac}[2]{\frac{\text{\footnotesize $#1$}}{\text{\footnotesize $#2$}}}
\newcommand{\vasymptote}[2][]{
    \draw [densely dashed,#1] ({rel axis cs:0,0} -| {axis cs:#2,0}) -- ({rel axis cs:0,1} -| {axis cs:#2,0});
}


\graphicspath{{./}{firstExample/}}
\usepackage{forest}
\usepackage{amsmath}
\usepackage{amssymb}
\usepackage{array}
\usepackage[makeroom]{cancel} %% for strike outs
\usepackage{pgffor} %% required for integral for loops
\usepackage{tikz}
\usepackage{tikz-cd}
\usepackage{tkz-euclide}
\usetikzlibrary{shapes.multipart}


%\usetkzobj{all}
\tikzstyle geometryDiagrams=[ultra thick,color=blue!50!black]


\usetikzlibrary{arrows}
\tikzset{>=stealth,commutative diagrams/.cd,
  arrow style=tikz,diagrams={>=stealth}} %% cool arrow head
\tikzset{shorten <>/.style={ shorten >=#1, shorten <=#1 } } %% allows shorter vectors

\usetikzlibrary{backgrounds} %% for boxes around graphs
\usetikzlibrary{shapes,positioning}  %% Clouds and stars
\usetikzlibrary{matrix} %% for matrix
\usepgfplotslibrary{polar} %% for polar plots
\usepgfplotslibrary{fillbetween} %% to shade area between curves in TikZ



%\usepackage[width=4.375in, height=7.0in, top=1.0in, papersize={5.5in,8.5in}]{geometry}
%\usepackage[pdftex]{graphicx}
%\usepackage{tipa}
%\usepackage{txfonts}
%\usepackage{textcomp}
%\usepackage{amsthm}
%\usepackage{xy}
%\usepackage{fancyhdr}
%\usepackage{xcolor}
%\usepackage{mathtools} %% for pretty underbrace % Breaks Ximera
%\usepackage{multicol}



\newcommand{\RR}{\mathbb R}
\newcommand{\R}{\mathbb R}
\newcommand{\C}{\mathbb C}
\newcommand{\N}{\mathbb N}
\newcommand{\Z}{\mathbb Z}
\newcommand{\dis}{\displaystyle}
%\renewcommand{\d}{\,d\!}
\renewcommand{\d}{\mathop{}\!d}
\newcommand{\dd}[2][]{\frac{\d #1}{\d #2}}
\newcommand{\pp}[2][]{\frac{\partial #1}{\partial #2}}
\renewcommand{\l}{\ell}
\newcommand{\ddx}{\frac{d}{\d x}}

\newcommand{\zeroOverZero}{\ensuremath{\boldsymbol{\tfrac{0}{0}}}}
\newcommand{\inftyOverInfty}{\ensuremath{\boldsymbol{\tfrac{\infty}{\infty}}}}
\newcommand{\zeroOverInfty}{\ensuremath{\boldsymbol{\tfrac{0}{\infty}}}}
\newcommand{\zeroTimesInfty}{\ensuremath{\small\boldsymbol{0\cdot \infty}}}
\newcommand{\inftyMinusInfty}{\ensuremath{\small\boldsymbol{\infty - \infty}}}
\newcommand{\oneToInfty}{\ensuremath{\boldsymbol{1^\infty}}}
\newcommand{\zeroToZero}{\ensuremath{\boldsymbol{0^0}}}
\newcommand{\inftyToZero}{\ensuremath{\boldsymbol{\infty^0}}}


\newcommand{\numOverZero}{\ensuremath{\boldsymbol{\tfrac{\#}{0}}}}
\newcommand{\dfn}{\textbf}
%\newcommand{\unit}{\,\mathrm}
\newcommand{\unit}{\mathop{}\!\mathrm}
%\newcommand{\eval}[1]{\bigg[ #1 \bigg]}
\newcommand{\eval}[1]{ #1 \bigg|}
\newcommand{\seq}[1]{\left( #1 \right)}
\renewcommand{\epsilon}{\varepsilon}
\renewcommand{\iff}{\Leftrightarrow}

\DeclareMathOperator{\arccot}{arccot}
\DeclareMathOperator{\arcsec}{arcsec}
\DeclareMathOperator{\arccsc}{arccsc}
\DeclareMathOperator{\si}{Si}
\DeclareMathOperator{\proj}{proj}
\DeclareMathOperator{\scal}{scal}
\DeclareMathOperator{\cis}{cis}
\DeclareMathOperator{\Arg}{Arg}
%\DeclareMathOperator{\arg}{arg}
\DeclareMathOperator{\Rep}{Re}
\DeclareMathOperator{\Imp}{Im}
\DeclareMathOperator{\sech}{sech}
\DeclareMathOperator{\csch}{csch}
\DeclareMathOperator{\Log}{Log}

\newcommand{\tightoverset}[2]{% for arrow vec
  \mathop{#2}\limits^{\vbox to -.5ex{\kern-0.75ex\hbox{$#1$}\vss}}}
\newcommand{\arrowvec}{\overrightarrow}
\renewcommand{\vec}{\mathbf}
\newcommand{\veci}{{\boldsymbol{\hat{\imath}}}}
\newcommand{\vecj}{{\boldsymbol{\hat{\jmath}}}}
\newcommand{\veck}{{\boldsymbol{\hat{k}}}}
\newcommand{\vecl}{\boldsymbol{\l}}
\newcommand{\utan}{\vec{\hat{t}}}
\newcommand{\unormal}{\vec{\hat{n}}}
\newcommand{\ubinormal}{\vec{\hat{b}}}

\newcommand{\dotp}{\bullet}
\newcommand{\cross}{\boldsymbol\times}
\newcommand{\grad}{\boldsymbol\nabla}
\newcommand{\divergence}{\grad\dotp}
\newcommand{\curl}{\grad\cross}
%% Simple horiz vectors
\renewcommand{\vector}[1]{\left\langle #1\right\rangle}


\pgfplotsset{compat=1.13}

\outcome{Recognize and sketch special sets}

\title{1.6 Special Sets}

\begin{document}

\begin{abstract}
We learn to recognize and sketch special sets in the complex plane.
\end{abstract}

\maketitle

\section{Circles and Disks}

\begin{example}[example 1]
The set 
\[
C(z_0,r) = \{z\in \C : |z-z_0| =r \}
\]
is a circle with center $z_0$ and radius $r$.
\end{example}



\begin{example}[example 2]
The set 
\[
D(z_0,r) = \{z\in \C : |z-z_0| <r \}
\]
is an \textbf{open} disk with center $z_0$ and radius $r$.
\end{example}


\begin{example}[example 3]
The set 
\[
\overline{D}(z_0,r) = \{z\in \C : |z-z_0| \leq r \}
\]
is a \textbf{closed} disk with center $z_0$ and radius $r$.
\end{example}

\begin{image}
\begin{tikzpicture}

\draw[blue,fill = blue!10] (1,1) circle (2cm);
\draw[<->, thick] (-3,0)--(3,0);

\draw[<->, thick] (0, 3)--(0,-3) node[ below=10pt, blue]{\large The closed disk, $\overline{D}(1+i,2)$};
\draw[white] (0,3.3) circle (0.2cm); %to add space above the figure

\draw[thin] (1,.2)--(1,-.2) node[below]{$1$};
\draw[thin] (-1,.2)--(-1,-.2) node[below]{$-1$};
\draw[thin] (2,.2)--(2,-.2) node[below]{$2$};
\draw[thin] (-2,.2)--(-2,-.2) node[below]{$-2$};

\draw[thin] (.2,1)--(-.2,1) node[left]{$i$};
\draw[thin] (.2,-1)--(-.2,-1) node[left]{$-i$};
\draw[thin] (.2,2)--(-.2,2) node[left]{$2i$};
\draw[thin] (.2,-2)--(-.2,-2) node[left]{$-2i$};

\draw[mark=*,mark size=1pt,mark options={color=blue}] plot coordinates {(1,1)} node[above right, blue]{$1+i$};
\draw[mark=*,mark size=1pt,mark options={color=blue}] plot coordinates {(-1,1)} node[above left, blue]{$-1+i$};
\draw[mark=*,mark size=1pt,mark options={color=blue}] plot coordinates {(1,-1)} node[below, blue]{$1-i$};

\end{tikzpicture}
\end{image}

\section{Lines and Half Planes}

\begin{example}[example 4]
The set of points satisfying the equation $\Imp(z) = c$ where $c \in \R$ is a horizontal line and
the set of points satisfying the equation $\Rep(z) = c$ where $c \in \R$ is a vertical line.
We can also express these lines as $x = c$ and $y = c$ with the understanding that $z = x + iy$.
Moreover, we can write $y = mx +b$ to represent a line in the complex plane with a
slope of $m$ and intersecting the imaginary axis at $bi$. 
\end{example}


\begin{example}[example 5]
The set of points satisfying the inequality $\Imp(z) < c$ where $c \in \R$ is a
half plane. If $c=0$ this set is called the lower half plane (LHP). 

\end{example}

\begin{image}
\begin{tikzpicture}
\draw[thick, dashed, blue] (1,-3) -- (1,3);
\draw [draw=none, fill=blue!10] (1,-3) rectangle (3,3);
\draw[<->, thick] (-3,0)--(3,0);
\draw[<->, thick] (0, 3)--(0,-3) node[ below=10pt, blue]{\large The set $\Rep z > 1$};
\draw[white] (0,3.3) circle (0.2cm); %to add space above the figure

\draw[thin] (1,.2)--(1,-.2) node[below]{$1$};
\draw[thin] (-1,.2)--(-1,-.2) node[below]{$-1$};
\draw[thin] (2,.2)--(2,-.2) node[below]{$2$};
\draw[thin] (-2,.2)--(-2,-.2) node[below]{$-2$};

\draw[thin] (.2,1)--(-.2,1) node[left]{$i$};
\draw[thin] (.2,-1)--(-.2,-1) node[left]{$-i$};
\draw[thin] (.2,2)--(-.2,2) node[left]{$2i$};
\draw[thin] (.2,-2)--(-.2,-2) node[left]{$-2i$};





\end{tikzpicture}
\end{image}
\section{Ellipses and Hyperbolas}

\begin{example}[example 6]
The set of points satisfying the equation
\[
|z-z_1| + |z-z_2| = d
\]
is an ellipse if $d > |z_1 -z_2|$.
\end{example}


\begin{example}[example 7]
The set of points satisfying the equation
\[
|z-z_1| - |z-z_2| = d
\]
is a hyperbola if $0 < d < |z_1 -z_2|$.
\end{example}



\section{Other Curves}


If $x(t)$ and $y(t)$ are differentiable functions of the real variable $t$ and if $x'$ and $y'$ do not both vanish simultaneously,
then
\[
\gamma(t)= x(t) + iy(t),  a \leq t \leq b
\]
is a smooth curve. Furthermore, if $\gamma(a) =\gamma(b)$, but $\gamma(t_1) \neq \gamma(t_2)$ for all $t_1,t_2 \in (a,b)$,
then $\gamma$ is called a simple closed curve.


\begin{example}[example 8]
The arc of a circle of radius $r$ centered at $z_0 = x_0 + iy_0$ that goes from angle $\theta_1$ to angle $\theta_2$ traversed in the counter-clockwise 
direction is given by
\[
\gamma(t) = x_0 + r\cos t + i(y_0 + r\sin t), \, \theta_1 \leq t \leq \theta_2
\]
\end{example}

\begin{example}[example 9]
A straight line segment from the point $z_1$ to the point $z_2$ is given by
\[
\gamma(t) = (1-t)z_1 + tz_2, \, 0 \leq t \leq 1
\]

\begin{image}
\begin{tikzpicture} %an attempt to insert arrows along a path- attempt worked!
[decoration={markings, 
	mark= at position 0.25 with {\arrow{stealth}},
	mark= at position 2cm with {\arrow{stealth}}}
] 

\draw [postaction={decorate}] (0,0) -- (2,2); 
\end{tikzpicture}
\end{image}
\end{example}

\begin{theorem}[Jordan Closed Curve Theorem]
The complement of any simple closed curve $C$ can be partitioned into two mutually exclusive domains, $I$ and $E$, in such a way that $I$ is bounded,
$E$ is unbounded, and $C$ is the boundary for both $I$ and $E$. In addition $I \cup E \cup C$ is the entire complex plane.
The domain $I$ is called the interior of $C$, and the domain $E$ is called the exterior of $C$.
\end{theorem}

\begin{remark} A domain is a connected, open set.
\end{remark}

\begin{image}
\begin{tikzpicture}

\draw[blue] (1,1) circle (2cm) ;
\node at (2,-1) {$C$};
\node at (1.2, 1.2) {$I$};
\node at (-1.8, 1)  {$E$};
\draw[<->, thick] (-3,0)--(3,0);

\draw[<->, thick] (0, 3)--(0,-3) node[ below=10pt, blue]{\large $I$, $E$ and $C$};
\draw[white] (0,3.3) circle (0.2cm); %to add space above the figure

\draw[thin] (1,.2)--(1,-.2) node[below]{$1$};
\draw[thin] (-1,.2)--(-1,-.2) node[below]{$-1$};
\draw[thin] (2,.2)--(2,-.2) node[below]{$2$};
\draw[thin] (-2,.2)--(-2,-.2) node[below]{$-2$};

\draw[thin] (.2,1)--(-.2,1) node[left]{$i$};
\draw[thin] (.2,-1)--(-.2,-1) node[left]{$-i$};
\draw[thin] (.2,2)--(-.2,2) node[left]{$2i$};
\draw[thin] (.2,-2)--(-.2,-2) node[left]{$-2i$};


\end{tikzpicture}
\end{image}

\section{Problems}
\begin{problem}(problem 1)
Sketch each the following:\\
\begin{align*}
i) & \; |z+1-i| = 2 \\
ii) &  \; |z-1| + |z+i| = 4 \\
iii) & \;  Re(z) \geq 2 \\
iv) &  \; \gamma(t) = (t-1) + it^2, 0\leq t \leq 2 \\
v) &  \; |i- z| < 3 \\
vi) &  \; |z-i| - |z+i| = 1 \\
vii) & \;  \gamma(t) = 2\cos t + 2i\sin t , \pi/4 \leq t \leq 3\pi/2 \\
viii) &  \; 1 < |z -2i| < 4 \\
ix) & \;  |z| > 1 \\
x) &  \; \gamma(t) = \cos t |\cos t| + i\sin t |\sin t|, 0 \leq t \leq 2\pi
\end{align*}
\end{problem}

Here is a video solution to problem 1viii:\\
\begin{foldable}
\youtube{9s3Iun-FWLs}
\end{foldable}

Here is a video solution to problem 1x:\\
\begin{foldable}
\youtube{NeOo_HusFhw}
\end{foldable}

\end{document}

