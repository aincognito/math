\documentclass[handout]{ximera}

%% You can put user macros here
%% However, you cannot make new environments



\newcommand{\ffrac}[2]{\frac{\text{\footnotesize $#1$}}{\text{\footnotesize $#2$}}}
\newcommand{\vasymptote}[2][]{
    \draw [densely dashed,#1] ({rel axis cs:0,0} -| {axis cs:#2,0}) -- ({rel axis cs:0,1} -| {axis cs:#2,0});
}


%\usepackage{tcolorbox} %%Needed for Derivative Definition supposedly and product rule, natural exp log, quotient rule, inverse trig, rates of change


% \graphicspath{{./}{firstExample/}}
% \usepackage{forest}
\usepackage{amsmath}
\usepackage{amssymb}
\usepackage{array}
\usepackage[makeroom]{cancel} %% for strike outs
\usepackage{pgffor} %% required for integral for loops
\usepackage{tikz}
\usepackage{tikz-cd}
\usepackage{tkz-euclide}
\usetikzlibrary{shapes.multipart}


% \usetkzobj{all}
\tikzstyle geometryDiagrams=[ultra thick,color=blue!50!black]


\usetikzlibrary{arrows}
\tikzset{>=stealth,commutative diagrams/.cd,
  arrow style=tikz,diagrams={>=stealth}} %% cool arrow head
\tikzset{shorten <>/.style={ shorten >=#1, shorten <=#1 } } %% allows shorter vectors

\usetikzlibrary{backgrounds} %% for boxes around graphs
\usetikzlibrary{shapes,positioning}  %% Clouds and stars
\usetikzlibrary{matrix} %% for matrix
\usepgfplotslibrary{polar} %% for polar plots
\usepgfplotslibrary{fillbetween} %% to shade area between curves in TikZ



%\usepackage[width=4.375in, height=7.0in, top=1.0in, papersize={5.5in,8.5in}]{geometry}
%\usepackage[pdftex]{graphicx}
%\usepackage{tipa}
%\usepackage{txfonts}
%\usepackage{textcomp}
%\usepackage{amsthm}
%\usepackage{xy}
%\usepackage{fancyhdr}
%\usepackage{xcolor}
%\usepackage{mathtools} %% for pretty underbrace % Breaks Ximera
%\usepackage{multicol}



\newcommand{\RR}{\mathbb R}
\newcommand{\R}{\mathbb R}
\newcommand{\C}{\mathbb C}
\newcommand{\N}{\mathbb N}
\newcommand{\Z}{\mathbb Z}
\newcommand{\dis}{\displaystyle}
%\renewcommand{\d}{\,d\!}
\renewcommand{\d}{\mathop{}\!d}
\newcommand{\dd}[2][]{\frac{\d #1}{\d #2}}
\newcommand{\pp}[2][]{\frac{\partial #1}{\partial #2}}
\renewcommand{\l}{\ell}
\newcommand{\ddx}{\frac{d}{\d x}}
\newcommand{\ppx}{\frac{\partial}{\partial x}}
\newcommand{\ppy}{\frac{\partial}{\partial y}}

\newcommand{\zeroOverZero}{\ensuremath{\boldsymbol{\tfrac{0}{0}}}}
\newcommand{\inftyOverInfty}{\ensuremath{\boldsymbol{\tfrac{\infty}{\infty}}}}
\newcommand{\zeroOverInfty}{\ensuremath{\boldsymbol{\tfrac{0}{\infty}}}}
\newcommand{\zeroTimesInfty}{\ensuremath{\small\boldsymbol{0\cdot \infty}}}
\newcommand{\inftyMinusInfty}{\ensuremath{\small\boldsymbol{\infty - \infty}}}
\newcommand{\oneToInfty}{\ensuremath{\boldsymbol{1^\infty}}}
\newcommand{\zeroToZero}{\ensuremath{\boldsymbol{0^0}}}
\newcommand{\inftyToZero}{\ensuremath{\boldsymbol{\infty^0}}}


\newcommand{\numOverZero}{\ensuremath{\boldsymbol{\tfrac{\#}{0}}}}
\newcommand{\dfn}{\textbf}
%\newcommand{\unit}{\,\mathrm}
\newcommand{\unit}{\mathop{}\!\mathrm}
%\newcommand{\eval}[1]{\bigg[ #1 \bigg]}
\newcommand{\eval}[1]{ #1 \bigg|}
\newcommand{\seq}[1]{\left( #1 \right)}
\renewcommand{\epsilon}{\varepsilon}
\renewcommand{\iff}{\Leftrightarrow}

\DeclareMathOperator{\arccot}{arccot}
\DeclareMathOperator{\arcsec}{arcsec}
\DeclareMathOperator{\arccsc}{arccsc}
\DeclareMathOperator{\si}{Si}
\DeclareMathOperator{\proj}{proj}
\DeclareMathOperator{\scal}{scal}
\DeclareMathOperator{\cis}{cis}
\DeclareMathOperator{\Arg}{Arg}
%\DeclareMathOperator{\arg}{arg}
\DeclareMathOperator{\Rep}{Re}
\DeclareMathOperator{\Imp}{Im}
\DeclareMathOperator{\sech}{sech}
\DeclareMathOperator{\csch}{csch}
\DeclareMathOperator{\Log}{Log}

\newcommand{\tightoverset}[2]{% for arrow vec
  \mathop{#2}\limits^{\vbox to -.5ex{\kern-0.75ex\hbox{$#1$}\vss}}}
\newcommand{\arrowvec}{\overrightarrow}
\renewcommand{\vec}{\mathbf}
\newcommand{\veci}{{\boldsymbol{\hat{\imath}}}}
\newcommand{\vecj}{{\boldsymbol{\hat{\jmath}}}}
\newcommand{\veck}{{\boldsymbol{\hat{k}}}}
\newcommand{\vecl}{\boldsymbol{\l}}
\newcommand{\utan}{\vec{\hat{t}}}
\newcommand{\unormal}{\vec{\hat{n}}}
\newcommand{\ubinormal}{\vec{\hat{b}}}

\newcommand{\dotp}{\bullet}
\newcommand{\cross}{\boldsymbol\times}
\newcommand{\grad}{\boldsymbol\nabla}
\newcommand{\divergence}{\grad\dotp}
\newcommand{\curl}{\grad\cross}
%% Simple horiz vectors
\renewcommand{\vector}[1]{\left\langle #1\right\rangle}


\outcome{Determine if a sequence converges or diverges}

\title{3.1 Sequences}

\begin{document}

\begin{abstract}
We determine if a sequence converges or diverges.
\end{abstract}

\maketitle

\begin{definition}[Sequence]
A sequence, denoted $\displaystyle \{a_n\}$ or $\displaystyle \{a_n\}_{1}^\infty$, is a function whose domain is the natural numbers: $1, 2, 3, \dots$, i.e,
\[
a_n = f(n)
\]
\end{definition}
We think of a sequence as an ordered list of numbers, $a_1, a_2, a_3, \dots$. We can also start our sequence with $a_0$ instead of $a_1$

Some examples of sequences are
\begin{align*}
i) \;\; &  \{a_n\}_{1}^\infty = \{n^2\} = 1^2, 2^2, 3^2, 4^2, \dots = 1, 4, 9, 16,  \dots\\
ii) \;\; &  \{a_n\}_{1}^\infty = \{(-1)^n\} = (-1)^1, (-1)^2, (-1)^3, \dots = -1, 1, -1, 1, \dots\\
iii) \;\; &  \{a_n\}_{1}^\infty = \left\{\frac{1}{2^n}\right\} = \frac{1}{2^1}, \frac{1}{2^2}, \frac{1}{2^3}, \frac{1}{2^4}, \dots = \frac12, \frac14, \frac18, \frac{1}{16}, \dots\\
iv) \;\; &   \{a_n\}_{1}^\infty = \left\{\frac{n}{n+1}\right\} = \frac{1}{1+1}, \frac{2}{2+1}, \frac{3}{3+1}, \frac{4}{4+1}, \dots = \frac12, \frac23, \frac34, \frac45, \dots
\end{align*}

The question we will have about a given sequence is if it has a limit.

\begin{definition}
A sequence $\{a_n\}$ has a limit $L$, written
\[
\lim_{n \to \infty} a_n \quad \mbox{or} \quad a_n \to L \; \mbox{as} \; n \to \infty
\]
if the terms of the sequence get arbitrarily close to $L$ as $n$ increases without bound.\\
If a sequence has a (finite) limit, we say that the sequence {\bf converges}.  Otherwise, we say that it {\bf diverges}.
\end{definition}


\begin{example}
Does the sequence $\displaystyle \{a_n\} = \left\{\frac{1}{2^n}\right\}$ converge or diverge?\\
We compute
\[
\lim_{n \to \infty} \frac{1}{2^n}
\]
Since $2^n \to \infty$ as $n \to \infty$, we can conclude that
\[
\lim_{n \to \infty} \frac{1}{2^n} = 0
\]
and so the sequence converges to $0$. The diagram below shows the terms of the sequence approaching $0$.
\begin{image}
\begin{tikzpicture}
\draw[thick,  ->] (0,0) -- (4,0) node[right]{$n$};
\draw[thick,  ->] (0,0) -- (0,2) node[above]{$a_n$};
\draw[mark=*,mark size=1pt,mark options={color=blue}] plot coordinates {(0.5,1.6)}; 
\draw[mark=*,mark size=1pt,mark options={color=blue}] plot coordinates {(1,.8)}; 
\draw[mark=*,mark size=1pt,mark options={color=blue}] plot coordinates {(1.5,.4)}; 
\draw[mark=*,mark size=1pt,mark options={color=blue}] plot coordinates {(2,.2)}; 
\draw[mark=*,mark size=1pt,mark options={color=blue}] plot coordinates {(2.5,.1)}; 
\draw[mark=*,mark size=1pt,mark options={color=blue}] plot coordinates {(3,.05)}; 
\draw[mark=*,mark size=1pt,mark options={color=blue}] plot coordinates {(3.5,.03)}; 
\end{tikzpicture}
\end{image}
\end{example}

\begin{remark}
The sequence in the previous problem could be written as $\displaystyle \{a_n\} = \left\{\left(\frac12\right)^n\right\}$. \\
In general, a sequence of the form $\{a_n\} = \{r^n\}$ converges if $0<r<1$ and diverges if $r>1$.
\end{remark}

\begin{example}
Does the sequence $\displaystyle \{a_n\} = \left\{(-1)^n\right\}$ converge or diverge?\\
We compute
\[
\lim_{n \to \infty} (-1)^n
\]
Since $(-1)^n$ oscillates between $-1$ and $1$ we can conclude that
\[
\lim_{n \to \infty} (-1)^n = \text{DNE}
\]
and so the sequence diverges. The diagram below shows the oscillation of the terms of the sequence.
\begin{image}
\begin{tikzpicture}
\draw[thick,  ->] (0,0) -- (4,0) node[right]{$n$};
\draw[thick,  <->] (0,-2) -- (0,2) node[above]{$a_n$};
\draw[mark=*,mark size=1pt,mark options={color=blue}] plot coordinates {(0.5,-1)}; 
\draw[mark=*,mark size=1pt,mark options={color=blue}] plot coordinates {(1,1)}; 
\draw[mark=*,mark size=1pt,mark options={color=blue}] plot coordinates {(1.5,-1)}; 
\draw[mark=*,mark size=1pt,mark options={color=blue}] plot coordinates {(2,1)}; 
\draw[mark=*,mark size=1pt,mark options={color=blue}] plot coordinates {(2.5,-1)}; 
\draw[mark=*,mark size=1pt,mark options={color=blue}] plot coordinates {(3,1)}; 
\draw[mark=*,mark size=1pt,mark options={color=blue}] plot coordinates {(3.5,-1)}; 
\draw[thin] (0.5, 0.1)--(0.5, -0.1) node[below]{$1$};
\draw[thin] (1, 0.1)--(1, -0.1) node[below]{$2$};
\draw[thin] (1.5, 0.1)--(1.5, -0.1) node[below]{$3$};
\draw[thin] (2, 0.1)--(2, -0.1) node[below]{$4$};
\draw[thin] (2.5, 0.1)--(2.5, -0.1) node[below]{$5$};
\draw[thin] (3, 0.1)--(3, -0.1) node[below]{$6$};
\draw[thin] (3.5, 0.1)--(3.5, -0.1) node[below]{$7$};
\end{tikzpicture}
\end{image}
\end{example}


\begin{example}
Does the sequence $\displaystyle \{a_n\} = \left\{n^2\right\}$ converge or diverge?\\
Since
\[
\lim_{n \to \infty} n^2 = \infty
\]
we conclude that the sequence diverges. The diagram below shows that the terms of the sequence grow without bound.
\begin{image}
\begin{tikzpicture}
\draw[thick,  ->] (0,0) -- (4,0) node[right]{$n$};
\draw[thick,  ->] (0,0) -- (0,4) node[above]{$a_n$};
\draw[mark=*,mark size=1pt,mark options={color=blue}] plot coordinates {(1,0.25)}; 
\draw[mark=*,mark size=1pt,mark options={color=blue}] plot coordinates {(2,1)}; 
\draw[mark=*,mark size=1pt,mark options={color=blue}] plot coordinates {(3,2)}; 
\draw[mark=*,mark size=1pt,mark options={color=blue}] plot coordinates {(4,4)}; 
%\draw[mark=*,mark size=1pt,mark options={color=blue}] plot coordinates {(2.5,-1)}; 
%\draw[mark=*,mark size=1pt,mark options={color=blue}] plot coordinates {(3,1)}; 
%\draw[mark=*,mark size=1pt,mark options={color=blue}] plot coordinates {(3.5,4)}; 
\end{tikzpicture}
\end{image}
\end{example}


\begin{example}
Does the sequence $\displaystyle \{a_n\} = \left\{\frac{n}{n+1}\right\}$ converge or diverge?\\
We compute
\begin{align*}
\lim_{n \to \infty} \frac{n}{n+1} &=\lim_{n \to \infty} \frac{n}{n\left(1+\frac{1}{n}\right)} \\
                                  &=\lim_{n \to \infty} \frac{1}{1+\frac{1}{n}}\\
                                  &=\frac{1}{1+0}\\
                                  &= 1
\end{align*}
Hence, the sequence converges to $1$. The diagram below shows the terms of the sequence approaching $1$.
\begin{image}
\begin{tikzpicture}
\draw[thick,  ->] (0,0) -- (4,0) node[right]{$n$};
\draw[thick,  ->] (0,0) -- (0,2) node[above]{$a_n$};
\draw[thin, dashed, red] (0,1.5) -- (4,1.5);
\draw[mark=*,mark size=1pt,mark options={color=blue}] plot coordinates {(0.5,.75)}; 
\draw[mark=*,mark size=1pt,mark options={color=blue}] plot coordinates {(1,1)}; 
\draw[mark=*,mark size=1pt,mark options={color=blue}] plot coordinates {(1.5,1.15)}; 
\draw[mark=*,mark size=1pt,mark options={color=blue}] plot coordinates {(2,1.25)}; 
\draw[mark=*,mark size=1pt,mark options={color=blue}] plot coordinates {(2.5,1.35)}; 
\draw[mark=*,mark size=1pt,mark options={color=blue}] plot coordinates {(3,1.4)}; 
\draw[mark=*,mark size=1pt,mark options={color=blue}] plot coordinates {(3.5,1.45)}; 
\end{tikzpicture}
\end{image}
\end{example}


\begin{example}
Does the sequence $\displaystyle \{a_n\} = \left\{\frac{n^2}{e^n}\right\}$ converge or diverge?\\
To compute the limit
\[
\lim_{n \to \infty} \frac{n^2}{e^n}
\]
we use the fact that if a function, $f(x)$, has a limit, then the associated sequence
$a_n = f(n)$ has the same limit. Using L'Hopital's rule twice, we have
\begin{align*}
\lim_{x \to \infty} \frac{x^2}{e^x} &= \frac{\infty}{\infty} = \lim_{x \to \infty} \frac{2x}{e^x} \;\text{(L'Hopital)}\\
                                  &=\frac{\infty}{\infty} = \lim_{x \to \infty} \frac{2}{e^x} \;\;\text{(L'Hopital again)}\\
                                  &=0
\end{align*}
Hence, 
\[
\lim_{n \to \infty} \frac{n^2}{e^n} =0
\]
as well, and the sequence converges to $0$. 
\end{example}

\begin{definition} 
A sequence $\{a_n\}$ is called {\bf increasing} if $a_{n+1} > a_n$ for all $n$.
Similarly, $\{a_n\}$ is called {\bf decreasing} if $a_{n+1} < a_n$ for all $n$.
\end{definition}

\begin{example}
Show that the sequence $\left\{\frac{1}{n}\right\}$ is decreasing.\\
We will use the fact that if $0<a<b$ then $\frac{1}{a} > \frac{1}{b}$.  Since
\[
n+1 > n \quad \text{for} \; n = 1, 2, 3, \ldots
\]
Taking reciprocals gives
\[
\frac{1}{n+1} < \frac{1}{n} \quad \text{for} \; n = 1, 2, 3, \ldots
\]
and hence the sequence $\left\{\frac{1}{n}\right\}$ is decreasing.
\end{example}


\begin{remark}
If the function $f(x)$ associated with the sequence $\{a_n\} = \{f(n)\}$ is an increasing (or decreasing) function, then the sequence 
is increasing (or decreasing).
\end{remark}

\begin{example}
Show that the sequence $\left\{\frac{n+1}{n^2 + 1}\right\}$ is decreasing.\\
We will show that the associated function $f(x) = \frac{x+1}{x^2 + 1}$ is an increasing function. To do this recall that if $f'(x)< 0$ for $x>a$,
then $f(x)$ is decreasing on the interval $(a,\infty)$. The quotient rule gives
\[
f'(x) = \frac{\left(x^2+1\right) - 2x(x+1)}{(x^2+1)^2} = \frac{-x^2-2x+1}{(x^2+1)^2}
\]
which is negative if $x>1/2$. Thus $f(x)$ is decreasing on the interval $(1/2, \infty)$ and $\left\{\frac{n+1}{n^2 + 1} \right\}$
is a decreasing sequence.
\end{example}

\section{Problems}
You can type DNE or infinity into the answer box as needed.

\begin{problem}(problem 1)
Find the limit of the sequence $\displaystyle \left\{\frac{1}{3^n}\right\}$ (if it exists), and then state whether the sequence converges or diverges.\\
$\displaystyle \lim_{n\to \infty} \frac{1}{3^n} = \;\;\answer{0}$ \quad
The sequence \wordChoice{\choice[correct]{converges}\choice{diverges}}
\end{problem}

\begin{problem}(problem 2)
Find the limit of the sequence $\displaystyle \left\{\frac{n+1}{n^3}\right\}$ (if it exists), and then state whether the sequence converges or diverges.\\
$\displaystyle \lim_{n\to \infty} \frac{n+1}{n^3} = \;\;\answer{0}$ \quad
The sequence \wordChoice{\choice[correct]{converges}\choice{diverges}}
\end{problem}

\begin{problem}(problem 3)
Find the limit of the sequence $\displaystyle \left\{\sqrt[n]{2}\right\}$ (if it exists), and then state whether the sequence converges or diverges.\\
$\displaystyle \lim_{n\to \infty} \sqrt[n]2 = \;\;\answer{1}$ \;\;
The sequence \wordChoice{\choice[correct]{converges}\choice{diverges}}
\end{problem}

\begin{problem}(problem 4)
Find the limit of the sequence $\displaystyle \left\{\frac{2^{2n}}{3^n}\right\}$ (if it exists), and then state whether the sequence converges or diverges.\\
$\displaystyle \lim_{n\to \infty} \frac{2^{2n}}{3^n} = \;\;\answer{\infty}$ \quad
The sequence \wordChoice{\choice{converges}\choice[correct]{diverges}}
\end{problem}

\begin{problem}(problem 5)
Find the limit of the sequence $\displaystyle \left\{\frac{\ln n}{n}\right\}$ (if it exists), and then state whether the sequence converges or diverges.\\
$\displaystyle \lim_{n\to \infty} \frac{\ln n}{n} = \;\;\answer{0}$ \quad
The sequence \wordChoice{\choice[correct]{converges}\choice{diverges}}
\end{problem}

\begin{problem}(problem 6)
Find the limit of the sequence $\displaystyle \left\{\frac{n^2}{5n+4}\right\}$ (if it exists), and then state whether the sequence converges or diverges.\\
$\displaystyle \lim_{n\to \infty} \frac{n^2}{5n+4} = \;\;\answer{\infty}$ \quad
The sequence \wordChoice{\choice{converges}\choice[correct]{diverges}}
\end{problem}

\begin{problem}(problem 7)
Find the limit of the sequence $\displaystyle \left\{\frac{(-1)^n}{n}\right\}$ (if it exists), and then state whether the sequence converges or diverges.\\
$\displaystyle \lim_{n\to \infty} \frac{(-1)^n}{n} = \;\;\answer{0}$ \quad
The sequence \wordChoice{\choice[correct]{converges}\choice{diverges}}
\end{problem}


\begin{problem}(problem 8)
Find the limit of the sequence $\displaystyle \left\{\cos(n\pi)\right\}$ (if it exists), and then state whether the sequence converges or diverges.\\
$\displaystyle \lim_{n\to \infty} \cos(n\pi) = \;\;\answer{DNE}$ \quad
The sequence \wordChoice{\choice{converges}\choice[correct]{diverges}}
\end{problem}


\begin{problem}(problem 9)
Find the limit of the sequence $\displaystyle \left\{\frac{n}{2^n}\right\}$ (if it exists), and then state whether the sequence converges or diverges.\\
$\displaystyle \lim_{n\to \infty} \frac{n}{2^n} = \;\;\answer{0}$ \quad
The sequence \wordChoice{\choice[correct]{converges}\choice{diverges}}
\end{problem}


\begin{problem}(problem 10)
Find the limit of the sequence $\displaystyle \left\{\frac{2n+1}{3n+2}\right\}$ (if it exists), and then state whether the sequence converges or diverges.\\
$\displaystyle \lim_{n\to \infty} \frac{2n+1}{3n+2} = \;\;\answer{2/3}$ \quad
The sequence \wordChoice{\choice[correct]{converges}\choice{diverges}}
\end{problem}


\begin{problem}(problem 11)
 Determine whether the sequence is increasing, decreasing or neither.\\
a) \;\; $\displaystyle \left\{\frac{1}{\sqrt n}\right\}$ \quad \mbox{\wordChoice{\choice{increasing} \choice[correct]{decreasing} \choice{neither}}}  \\
b) \;\; $\displaystyle \left\{\frac{n}{n^2 +1}\right\}$ \quad \mbox{\wordChoice{\choice{increasing} \choice[correct]{decreasing} \choice{neither}} } \\
c) \;\; $\displaystyle  \left\{\ln n\right\}$ \quad \mbox{\wordChoice{\choice[correct]{increasing} \choice{decreasing} \choice{neither}} } \\
d) \;\; $\displaystyle  \left\{\frac{\cos(n\pi)}{n^2}\right\}$ \quad \mbox{\wordChoice{\choice{increasing} \choice{decreasing} \choice[correct]{neither}} } \\
e) \;\; $\displaystyle \left\{\frac{n^2 + 1}{n^3 + 1}\right\}$ \quad \mbox{\wordChoice{\choice{increasing} \choice[correct]{decreasing} \choice{neither}}} 
\end{problem}


\end{document}




