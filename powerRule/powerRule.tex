\documentclass[handout]{ximera}
%\usepackage{tcolorbox}
%% You can put user macros here
%% However, you cannot make new environments



\newcommand{\ffrac}[2]{\frac{\text{\footnotesize $#1$}}{\text{\footnotesize $#2$}}}
\newcommand{\vasymptote}[2][]{
    \draw [densely dashed,#1] ({rel axis cs:0,0} -| {axis cs:#2,0}) -- ({rel axis cs:0,1} -| {axis cs:#2,0});
}


%\usepackage{tcolorbox} %%Needed for Derivative Definition supposedly and product rule, natural exp log, quotient rule, inverse trig, rates of change


% \graphicspath{{./}{firstExample/}}
% \usepackage{forest}
\usepackage{amsmath}
\usepackage{amssymb}
\usepackage{array}
\usepackage[makeroom]{cancel} %% for strike outs
\usepackage{pgffor} %% required for integral for loops
\usepackage{tikz}
\usepackage{tikz-cd}
\usepackage{tkz-euclide}
\usetikzlibrary{shapes.multipart}


% \usetkzobj{all}
\tikzstyle geometryDiagrams=[ultra thick,color=blue!50!black]


\usetikzlibrary{arrows}
\tikzset{>=stealth,commutative diagrams/.cd,
  arrow style=tikz,diagrams={>=stealth}} %% cool arrow head
\tikzset{shorten <>/.style={ shorten >=#1, shorten <=#1 } } %% allows shorter vectors

\usetikzlibrary{backgrounds} %% for boxes around graphs
\usetikzlibrary{shapes,positioning}  %% Clouds and stars
\usetikzlibrary{matrix} %% for matrix
\usepgfplotslibrary{polar} %% for polar plots
\usepgfplotslibrary{fillbetween} %% to shade area between curves in TikZ



%\usepackage[width=4.375in, height=7.0in, top=1.0in, papersize={5.5in,8.5in}]{geometry}
%\usepackage[pdftex]{graphicx}
%\usepackage{tipa}
%\usepackage{txfonts}
%\usepackage{textcomp}
%\usepackage{amsthm}
%\usepackage{xy}
%\usepackage{fancyhdr}
%\usepackage{xcolor}
%\usepackage{mathtools} %% for pretty underbrace % Breaks Ximera
%\usepackage{multicol}



\newcommand{\RR}{\mathbb R}
\newcommand{\R}{\mathbb R}
\newcommand{\C}{\mathbb C}
\newcommand{\N}{\mathbb N}
\newcommand{\Z}{\mathbb Z}
\newcommand{\dis}{\displaystyle}
%\renewcommand{\d}{\,d\!}
\renewcommand{\d}{\mathop{}\!d}
\newcommand{\dd}[2][]{\frac{\d #1}{\d #2}}
\newcommand{\pp}[2][]{\frac{\partial #1}{\partial #2}}
\renewcommand{\l}{\ell}
\newcommand{\ddx}{\frac{d}{\d x}}
\newcommand{\ppx}{\frac{\partial}{\partial x}}
\newcommand{\ppy}{\frac{\partial}{\partial y}}

\newcommand{\zeroOverZero}{\ensuremath{\boldsymbol{\tfrac{0}{0}}}}
\newcommand{\inftyOverInfty}{\ensuremath{\boldsymbol{\tfrac{\infty}{\infty}}}}
\newcommand{\zeroOverInfty}{\ensuremath{\boldsymbol{\tfrac{0}{\infty}}}}
\newcommand{\zeroTimesInfty}{\ensuremath{\small\boldsymbol{0\cdot \infty}}}
\newcommand{\inftyMinusInfty}{\ensuremath{\small\boldsymbol{\infty - \infty}}}
\newcommand{\oneToInfty}{\ensuremath{\boldsymbol{1^\infty}}}
\newcommand{\zeroToZero}{\ensuremath{\boldsymbol{0^0}}}
\newcommand{\inftyToZero}{\ensuremath{\boldsymbol{\infty^0}}}


\newcommand{\numOverZero}{\ensuremath{\boldsymbol{\tfrac{\#}{0}}}}
\newcommand{\dfn}{\textbf}
%\newcommand{\unit}{\,\mathrm}
\newcommand{\unit}{\mathop{}\!\mathrm}
%\newcommand{\eval}[1]{\bigg[ #1 \bigg]}
\newcommand{\eval}[1]{ #1 \bigg|}
\newcommand{\seq}[1]{\left( #1 \right)}
\renewcommand{\epsilon}{\varepsilon}
\renewcommand{\iff}{\Leftrightarrow}

\DeclareMathOperator{\arccot}{arccot}
\DeclareMathOperator{\arcsec}{arcsec}
\DeclareMathOperator{\arccsc}{arccsc}
\DeclareMathOperator{\si}{Si}
\DeclareMathOperator{\proj}{proj}
\DeclareMathOperator{\scal}{scal}
\DeclareMathOperator{\cis}{cis}
\DeclareMathOperator{\Arg}{Arg}
%\DeclareMathOperator{\arg}{arg}
\DeclareMathOperator{\Rep}{Re}
\DeclareMathOperator{\Imp}{Im}
\DeclareMathOperator{\sech}{sech}
\DeclareMathOperator{\csch}{csch}
\DeclareMathOperator{\Log}{Log}

\newcommand{\tightoverset}[2]{% for arrow vec
  \mathop{#2}\limits^{\vbox to -.5ex{\kern-0.75ex\hbox{$#1$}\vss}}}
\newcommand{\arrowvec}{\overrightarrow}
\renewcommand{\vec}{\mathbf}
\newcommand{\veci}{{\boldsymbol{\hat{\imath}}}}
\newcommand{\vecj}{{\boldsymbol{\hat{\jmath}}}}
\newcommand{\veck}{{\boldsymbol{\hat{k}}}}
\newcommand{\vecl}{\boldsymbol{\l}}
\newcommand{\utan}{\vec{\hat{t}}}
\newcommand{\unormal}{\vec{\hat{n}}}
\newcommand{\ubinormal}{\vec{\hat{b}}}

\newcommand{\dotp}{\bullet}
\newcommand{\cross}{\boldsymbol\times}
\newcommand{\grad}{\boldsymbol\nabla}
\newcommand{\divergence}{\grad\dotp}
\newcommand{\curl}{\grad\cross}
%% Simple horiz vectors
\renewcommand{\vector}[1]{\left\langle #1\right\rangle}


\outcome{Learn the Power Rule for differentiating power functions}

\title{2.1 Power Rule}

%\newcommand{\ffrac}[2]{\frac{\mbox{\footnotesize $#1$}}{\mbox{\footnotesize $#2$}}}
%\newcommand{\vasymptote}[2][]{
%    \draw [densely dashed,#1] ({rel axis cs:0,0} -| {axis cs:#2,0}) -- ({rel axis cs:0,1} -| {axis cs:#2,0});
%}


\begin{document}

\begin{abstract}
We learn how to find the derivative of a power function.
\end{abstract}

\maketitle

\youtube{bRZmfc1YFsQ}

\section{The Power Rule}

In this section we present the derivatives power functions. A power function is a function of the form $f(x) = x^\text{n}$.  

\begin{theorem}[Power Rule]
For any number n, the derivative of the power function $f(x) = x^\text{n}$ is given by
$f'(x) = \text{n}x^{\text{n} -1}$, i.e.,
\[\frac{d}{dx} \left(x^\text{n}\right) = \text{n}x^{\text{n}-1}.\]
\end{theorem}

We will investigate this rule in the case where n is a whole number. The key is to use 
the \link[Binomial Theorem]{https://en.wikipedia.org/wiki/Binomial_theorem}, which gives a formula for 
expanding powers of $x+h$. 
Here are some examples of the theorem:

\begin{eqnarray*}
(x+h)^{\bf 2} &=&x^2 + {\bf 2}xh + h^2\\
(x+h)^{\bf 3} &=& x^3 + {\bf 3}x^2h + 3xh^2 + h^3\\
(x+h)^{\bf 4} &=& x^4 + {\bf 4}x^3h+ 6x^2h^2 + 4xh^3 + h^4\\
(x+h)^{\bf 5} &=& x^5 + {\bf 5}x^4h + 10x^3h^2 + 10x^2h^3 + 5xh^4 + h^5
\end{eqnarray*}

The coefficients can be  determined using a device called \link[Pascal's Triangle]{https://en.wikipedia.org/wiki/Pascal's_triangle}.
Notice that the coefficient of the second term on the right hand side always matches the power on the left hand side. In general,

\[(x+h)^{\bf \text{n}} = x^\text{n} + {\bf \text{n}}x^{\text{n}-1}h + \dots + \text{n}xh^{\text{n}-1} + h^\text{n}.\]

Also observe that as the powers of $x$ decrease, the powers of $h$ increase.\\

We will now use the definition of the derivative to discover the power rule. 
Let $f(x) = x^\text{n}$, where \text{n} is a whole number. Then

\begin{eqnarray*}
f'(x) &=& \lim_{h\to 0} \frac{f(x+h)-f(x)}{h}\\[10 pt] 
&=& \lim_{h\to 0} \frac{(x+h)^\text{n}-x^\text{n}}{h}\\[10 pt]
&=& \lim_{h\to 0} \frac{\left(x^\text{n} + \text{n}x^{\text{n}-1}h + \dots + \text{n}xh^{\text{n}-1} + h^\text{n}\right)-x^\text{n}}{h}\\[10 pt]
 &=& \lim_{h\to 0} \frac{\text{n}x^{\text{n}-1}h + \dots + \text{n}xh^{\text{n}-1} + h^\text{n}}{h} \\[10 pt]
 &=& \lim_{h\to 0} \left(\text{n}x^{\text{n}-1} + \dots + \text{n}xh^{\text{n}-2} + h^{\text{n}-1}\right) \\[10 pt] 
 &=& \text{n}x^{\text{n}-1}.
 \end{eqnarray*}




\begin{example}[example 1]
Find the derivative of  $f(x) = x^4$.\\
We can use the power rule with $n = 4$ to obtain $f'(x) = 4x^3.$
\end{example}



\begin{problem}(problem 1a)
Compute
\[
\frac{d}{dx} \left(x^3\right).
\]
\begin{hint}
Use the power rule with $n = 3$
\end{hint}
\begin{prompt}
The derivative of $f(x) = x^3$ is \ $f'(x) = \answer{3x^2}.$
\end{prompt}
\end{problem}



\begin{problem}(problem 1b)
Compute
\[
%\frac{d}{dx}
\dd{x} \left(x^5\right).
\]
\begin{hint}
Use the power rule with $n = 5$
\end{hint}
The derivative of $f(x) = x^5$ is \ $f'(x) = \answer{5x^4}.$
\end{problem}



\begin{problem}(problem 1c)
Find the slope of the tangent line to the graph of 
\[
f(x) = x^6  \;\; \text{at} \;\; x = -2.
\]
\begin{hint}
The derivative gives the slope of the tangent line
\end{hint}
\begin{hint}
Use the power rule with $n = 6$
\end{hint}
The slope is  $\answer{-192}.$
\end{problem}


\begin{example}[example 2]
Find the derivative of $f(x) = x^{100}$.\\
We can use the power rule with $n = 100$ to obtain $f'(x) = 100x^{99}.$
\end{example}

\begin{problem}(problem 2)
Compute
\[
\frac{d}{dx} \left(x^{250}\right).
\]
\begin{hint}
Use the power rule with $n = 250$
\end{hint}
The derivative of $x^{250}$ is $\answer{250x^{249}}.$
\end{problem}



\begin{example}[example 3]
Find the derivative of $f(x) = 1$.\\
We can rewrite the function as a power function, since $x^0 = 1$ and then we can use the power rule with $n=0$. We get $f'(x) = 0x^{-1} = 0$. 

\begin{remark}
We can also arrive at this answer using a geometric understanding of the derivative.
The graph of the constant function $f(x) = 1$ is a horizontal line, which has slope 0. 
Since the derivative is the slope of the tangent line, the derivative is 0.
\end{remark}
\end{example}


In general, the graph of a constant function, $f(x) = c$ is a horizontal line, which has slope 0.
Hence, the derivative of any constant is 0, as stated in the following proposition.

\begin{proposition}[Constant Functions]
If $c$ is any constant, then
\[
\frac{d}{dx} \left(c\right) = 0.
\]
\end{proposition}


The graphs of $y = -2$, $y = 1/2$ and $y = 3$ are shown below.
(To see $y=3$, zoom out by clicking on the minus in the graphing window or 
by placing the cursor in the graphing window and using the scroll wheel on your mouse).
\[
\graph{y = 3, y = 1/2, y = -2}
\]



\begin{example}[example 4]
If $f(x) = 5, -7, \frac23$ or $4.15$, then $f'(x) = 0$ 
because these are all constants. Notice that none of the expressions contain the variable $x$.
\end{example}



\begin{example}[example 5]
If $f(x) = -\pi, e^3, \sqrt 2$ or $\ln(8)$, then $f'(x) = 0$ 
because these are all constants. Notice that none of the expressions contain the variable $x$.
\end{example}



\begin{problem}(problem 5)
Compute
\[
\frac{d}{dx} \left(94.1\right) = \answer{0}
\]

\[
\frac{d}{dx} \left(\pi^2\right) = \answer{0}
\]

\[
\frac{d}{dt} \left( \sqrt 5 \right) = \answer{0}
\]


\end{problem}



\begin{example}[example 6]
Find the derivative of $f(x) = x$.\\
We can use the power rule with $n=1$ to obtain $f'(x) = 1x^0 = 1$. 

\begin{remark}
This answer can also be arrived at by interpreting the derivative as slope. The graph of the function $f(x) = x$ is a straight line with slope $1$,
so the derivative is $1$.  
\end{remark}
\end{example}

\begin{problem}(problem 6)
\[
\frac{d}{dt}\left(t\right) = \answer{1}
\]
\end{problem}

The graph of a function $f(x) = mx$
is a line with slope $m$, so the derivative of $mx$ is $m$. This is stated in the proposition below.
 
\begin{proposition}[Linear Functions]

For any number $m$,
\[
\frac{d}{dx} (mx) = m.
\]
\end{proposition}


The graphs of $y = 2x$, $y = x/2$ and $y = -x$ are shown below.
Which one is which? Click on the double arrows in the upper left hand corner of the graphing window to check your answer.
\[
\graph{2x, x/2, -x}
\]



\begin{example}[example 7]
If $f(x) = -3x$ then $f'(x) = -3$.
\end{example}



\begin{problem}(problem 7a)
If $f(x) = 4x$ then $f'(x) = \answer{4}$
\end{problem}

\begin{problem}(problem 7b)
If $g(t) = -2t$ then $g'(t) = \answer{-2}$
\end{problem}



\begin{example}[example 8]
If $f(x) = \pi x$ then $f'(x) = \pi$.
\end{example}

\begin{problem}(problem 8a)
\[
\frac{d}{dx} \left(ex\right) = \answer{e}
\]
\end{problem}



\begin{problem}(problem 8b)
\[
\frac{d}{dx} \left(x\ln(5)\right) = \answer{\ln(5)}
\]
\end{problem}






\begin{example}[example 9]
If $f(x) = \frac{x}{2}$ then $f'(x) = \frac 12$.
\end{example}



\begin{problem}(problem 9a)
\[
\frac{d}{dx} \left(\frac{x}{3}\right) = \answer{1/3}
\]
\end{problem}



\begin{problem}(problem 9b)
\[
\frac{d}{dx} \left(-\frac{2x}{5}\right) = \answer{-2/5}
\]
\end{problem}






We now consider examples where the exponent $n$ is a fraction. Recall the definition of rational exponents:
\[
x^{\frac{m}{n}} = \sqrt[n]{x^m}.
\]



\begin{example}[example 10]
Find the derivative of $f(x) = \sqrt x$.\\
We can rewrite $\sqrt x$ as $x^{1/2}$ and use the power rule 
with $n = 1/2$ to obtain $f'(x) = (1/2)x^{-1/2}$.  This can then be rewritten in radical form :
\[(1/2)x^{-1/2} = \frac{1}{2}\cdot \frac{1}{x^{1/2}} = \frac{1}{2\sqrt x}.\]
This result is used frequently, so it is best to remember it as
\[\frac{d}{dx}\left(\sqrt x\right) = \frac{1}{2\sqrt x}.\]
\end{example}




\begin{example}[example 11]
Find the derivative of $f(x) = \sqrt[3] x$.\\
We can rewrite $\sqrt[3] x$ as $x^{1/3}$ and use the power rule with $n = 1/3$ to obtain
$f'(x) = (1/3)x^{-2/3}$.  This can be rewritten in radical form as 
\[(1/3)x^{-2/3} = \frac{1}{3}\cdot \frac{1}{x^{2/3}} = \frac{1}{3\sqrt[3] {x^2}}.\]
\end{example}




\begin{problem}(problem 11a)
Compute
\[
\frac{d}{dx} \left(\sqrt[4] x\right).
\]
\begin{hint}
Rewrite $\sqrt[4] x$ as $x^{1/4}$
\end{hint}
\begin{hint}
Use the power rule with $n = 1/4$
\end{hint}
The derivative of $\sqrt[4] x$ is $\answer{1/4 x^{-3/4}}.$
\end{problem}



\begin{problem}(problem 11b)
Find the slope of the tangent line to the graph of 
\[
f(x) = \sqrt[5] x  \;\; \text{at} \;\; x = -32.
\]

\begin{hint}
The derivative gives the slope of the tangent line
\end{hint}
\begin{hint}
Rewrite $\sqrt[5] x$ as $x^{1/5}$
\end{hint}
\begin{hint}
Use the power rule with $n = 1/5$
\end{hint}
The slope is  $\answer{1/80}.$
\end{problem}



\begin{example}[example 12]
Find the derivative of $f(x) = x^2\sqrt[4] {x^3}$.\\
We first rewrite $f(x)$ 
as $x^2 \cdot x^{3/4} = x^{11/4}$ (adding exponents) and then we use the power rule with $n = 11/4$
to obtain $f'(x) = (11/4)x^{7/4}$. This can be rewritten in radical form as 
\[(11/4)x^{7/4}  = \frac{11}{4} \sqrt[4] {x^7} = \frac{11}{4} x\sqrt[4] {x^3} .\]
\end{example}



\begin{problem}(problem 12)
Compute
\[
\frac{d}{dx} \left(x\sqrt[3] x\right).
\]
\begin{hint}
Rewrite $x\sqrt[3] x$ as $x^{4/3}$
\end{hint}
\begin{hint}
Use the power rule with $n = 4/3$
\end{hint}
The derivative of $x\sqrt[3] x$ is $\answer{(4/3) x^{1/3}}.$
\end{problem}



Next, we look at some examples involving negative exponents. Recall the definition of negative exponents:
\[
x^{-n} = \frac{1}{x^n}.
\]



\begin{example}[example 13]
Find the derivative of $f(x) = \dfrac{1}{x}$.\\
We can rewrite $1/x$ as $x^{-1}$ and use the power rule with $n = -1$ to obtain 
We get $f'(x) = -1x^{-2}$.
This can be rewritten without the negative exponent as
\[ 
-1x^{-2} = -\frac{1}{x^2}.
\]
%This result is used frequently, so it is best to remember it as
%\[
%\frac{d}{dx}\left(\frac{1}{x}\right) = -\frac{1}{x^2}.
%\]
\end{example}



\begin{example}[example 14]
Find the derivative of $f(x) = \dfrac{1}{x^3}$.\\
We can rewrite $1/x^3$ as $x^{-3}$ and use the power rule with $n = -3$ to obtain 
$f'(x) = -3x^{-4}$.
This can be rewritten without the negative exponent as
\[
-3x^{-4} = -\frac{3}{x^4}.
\]
\end{example}




\begin{problem}(problem 14a)
Compute
\[
\frac{d}{dx} \left(\frac{1}{x^2}\right).
\]
\begin{hint}
Rewrite $\dfrac{1}{x^2}$ as $x^{-2}$
\end{hint}
\begin{hint}
Use the power rule with $n = -2$
\end{hint}
The derivative of $f(x) = \dfrac{1}{x^2}$ is \ $f'(x) = \answer{-2x^{-3}}.$
\end{problem}




\begin{problem}(problem 14b)
Find the slope of the tangent line to the graph of 
\[
f(x) = \frac{1}{x^3} \;\; \text{at} \;\; x = 3.
\]


\begin{hint}
The derivative gives the slope of the tangent line
\end{hint}
\begin{hint}
Rewrite $\dfrac{1}{x^3}$ as $x^{-3}$
\end{hint}
\begin{hint}
Use the power rule with $n = -3$
\end{hint}
The slope is  $\answer{-1/27}.$
\end{problem}




\begin{example}[example 15]
Find the derivative of $f(x) = \dfrac{1}{\sqrt x}$.\\
We can rewrite $\frac{1}{\sqrt x}$ as $x^{-1/2}$ and use the power rule with $n = -1/2$ to obtain
\[
f'(x) = -\frac12 x^{-3/2}.
\]
This can be rewritten without the negative exponent as
\[
(-1/2)x^{-3/2}= -\frac{1}{2} \cdot \frac{1}{x^{3/2}} = -\frac{1}{2\sqrt{x^3}} = -\frac{1}{2x\sqrt x}.
\]
\end{example}




\begin{problem}(problem 15)
Compute
\[
\frac{d}{dx} \left(\frac{1}{\sqrt[5] x}\right).
\]
\begin{hint}
Rewrite $1/\sqrt[5] x$ as $x^{-1/5}$
\end{hint}
\begin{hint}
Use the power rule with $n = -1/5$
\end{hint}
The derivative of $f(x) = \frac{1}{\sqrt[5] x}$ is \ $f'(x) = \answer{(-1/5) x^{-6/5}}.$
\end{problem}




\begin{example}[example 16]
Find the derivative of $f(x) = \dfrac{\sqrt[3] {x^2}}{\sqrt[4] x}.$\\
We can rewrite $f(x)$ as $\frac{x^{2/3}}{x^{1/4}} = x^{5/12}$ 
(subtracting exponents). Now, we use the power rule with $n = \frac{5}{12}$ to obtain $f'(x) = \tfrac{5}{12}x^{-7/12}.$
This can be rewritten without the negative exponent as
\[
\tfrac{5}{12}x^{-7/12} = \tfrac{5}{12} \cdot \frac{1}{x^{7/12}} = \frac{5}{12\sqrt[12]{x^7}}.
\]
\end{example}




\begin{problem}(problem 16)
Compute
\[
\frac{d}{dx} \left(\frac{\sqrt[4] {x^3}}{\sqrt[3] x}\right).
\]
\begin{hint}
Rewrite $\frac{\sqrt[4] {x^3}}{\sqrt[3] x}$ as $\frac{x^{3/4}}{x^{1/3}} = x^{5/12}$
\end{hint}
\begin{hint}
Use the power rule with $n = 5/12$
\end{hint}
The derivative of $\frac{\sqrt[4] {x^3}}{\sqrt[3] x}$ is $\answer{(5/12) x^{-7/12}}.$
\end{problem}




\begin{center}
\begin{foldable}
\unfoldable{Below is a graph of $f(x) = x^2$ (in blue) and its derivative, $f'(x) = 2x$ (in purple).
Notice that the slope of the tangent line to $f(x)$ (in red) is the height of the corresponding point 
on $f'(x)$.}
\includeinteractive{fd01.js}
%\graph{x^2}
\end{foldable}
\end{center}



\begin{center}
\begin{foldable}
\unfoldable{Below is a graph of $f(x) = x^3$ (in blue) and its derivative, $f'(x) = 3x^2$ (in purple).
Notice that the slope of the tangent line to $f(x)$ (in red) is the height of the corresponding point on $f'(x)$.}
\includeinteractive{fd02.js}
\end{foldable}
\end{center}


%change these functions to x^2 - 3 and a cubic polynomial with 2 turning points, like  x^3 - 12x whose deriv is 3x^2 -12
%this way the horiz tan lines don't go thru the origin


\begin{center}
\begin{foldable}
\unfoldable{Below is a graph of $f(x) = \sqrt[3] x$ (in blue) and its derivative, 
$f'(x) = \frac{1}{3\sqrt[3]x^2}$ (in purple).
Notice that the slope of the tangent line to $f(x)$ (in red) is the height of the corresponding 
point on $f'(x)$.}
\includeinteractive{fd03.js}
\end{foldable}
\end{center}




\end{document}




