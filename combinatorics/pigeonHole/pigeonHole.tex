\documentclass[handout]{ximera}

%% You can put user macros here
%% However, you cannot make new environments



\newcommand{\ffrac}[2]{\frac{\text{\footnotesize $#1$}}{\text{\footnotesize $#2$}}}
\newcommand{\vasymptote}[2][]{
    \draw [densely dashed,#1] ({rel axis cs:0,0} -| {axis cs:#2,0}) -- ({rel axis cs:0,1} -| {axis cs:#2,0});
}


%\usepackage{tcolorbox} %%Needed for Derivative Definition supposedly and product rule, natural exp log, quotient rule, inverse trig, rates of change


% \graphicspath{{./}{firstExample/}}
% \usepackage{forest}
\usepackage{amsmath}
\usepackage{amssymb}
\usepackage{array}
\usepackage[makeroom]{cancel} %% for strike outs
\usepackage{pgffor} %% required for integral for loops
\usepackage{tikz}
\usepackage{tikz-cd}
\usepackage{tkz-euclide}
\usetikzlibrary{shapes.multipart}


% \usetkzobj{all}
\tikzstyle geometryDiagrams=[ultra thick,color=blue!50!black]


\usetikzlibrary{arrows}
\tikzset{>=stealth,commutative diagrams/.cd,
  arrow style=tikz,diagrams={>=stealth}} %% cool arrow head
\tikzset{shorten <>/.style={ shorten >=#1, shorten <=#1 } } %% allows shorter vectors

\usetikzlibrary{backgrounds} %% for boxes around graphs
\usetikzlibrary{shapes,positioning}  %% Clouds and stars
\usetikzlibrary{matrix} %% for matrix
\usepgfplotslibrary{polar} %% for polar plots
\usepgfplotslibrary{fillbetween} %% to shade area between curves in TikZ



%\usepackage[width=4.375in, height=7.0in, top=1.0in, papersize={5.5in,8.5in}]{geometry}
%\usepackage[pdftex]{graphicx}
%\usepackage{tipa}
%\usepackage{txfonts}
%\usepackage{textcomp}
%\usepackage{amsthm}
%\usepackage{xy}
%\usepackage{fancyhdr}
%\usepackage{xcolor}
%\usepackage{mathtools} %% for pretty underbrace % Breaks Ximera
%\usepackage{multicol}



\newcommand{\RR}{\mathbb R}
\newcommand{\R}{\mathbb R}
\newcommand{\C}{\mathbb C}
\newcommand{\N}{\mathbb N}
\newcommand{\Z}{\mathbb Z}
\newcommand{\dis}{\displaystyle}
%\renewcommand{\d}{\,d\!}
\renewcommand{\d}{\mathop{}\!d}
\newcommand{\dd}[2][]{\frac{\d #1}{\d #2}}
\newcommand{\pp}[2][]{\frac{\partial #1}{\partial #2}}
\renewcommand{\l}{\ell}
\newcommand{\ddx}{\frac{d}{\d x}}
\newcommand{\ppx}{\frac{\partial}{\partial x}}
\newcommand{\ppy}{\frac{\partial}{\partial y}}

\newcommand{\zeroOverZero}{\ensuremath{\boldsymbol{\tfrac{0}{0}}}}
\newcommand{\inftyOverInfty}{\ensuremath{\boldsymbol{\tfrac{\infty}{\infty}}}}
\newcommand{\zeroOverInfty}{\ensuremath{\boldsymbol{\tfrac{0}{\infty}}}}
\newcommand{\zeroTimesInfty}{\ensuremath{\small\boldsymbol{0\cdot \infty}}}
\newcommand{\inftyMinusInfty}{\ensuremath{\small\boldsymbol{\infty - \infty}}}
\newcommand{\oneToInfty}{\ensuremath{\boldsymbol{1^\infty}}}
\newcommand{\zeroToZero}{\ensuremath{\boldsymbol{0^0}}}
\newcommand{\inftyToZero}{\ensuremath{\boldsymbol{\infty^0}}}


\newcommand{\numOverZero}{\ensuremath{\boldsymbol{\tfrac{\#}{0}}}}
\newcommand{\dfn}{\textbf}
%\newcommand{\unit}{\,\mathrm}
\newcommand{\unit}{\mathop{}\!\mathrm}
%\newcommand{\eval}[1]{\bigg[ #1 \bigg]}
\newcommand{\eval}[1]{ #1 \bigg|}
\newcommand{\seq}[1]{\left( #1 \right)}
\renewcommand{\epsilon}{\varepsilon}
\renewcommand{\iff}{\Leftrightarrow}

\DeclareMathOperator{\arccot}{arccot}
\DeclareMathOperator{\arcsec}{arcsec}
\DeclareMathOperator{\arccsc}{arccsc}
\DeclareMathOperator{\si}{Si}
\DeclareMathOperator{\proj}{proj}
\DeclareMathOperator{\scal}{scal}
\DeclareMathOperator{\cis}{cis}
\DeclareMathOperator{\Arg}{Arg}
%\DeclareMathOperator{\arg}{arg}
\DeclareMathOperator{\Rep}{Re}
\DeclareMathOperator{\Imp}{Im}
\DeclareMathOperator{\sech}{sech}
\DeclareMathOperator{\csch}{csch}
\DeclareMathOperator{\Log}{Log}

\newcommand{\tightoverset}[2]{% for arrow vec
  \mathop{#2}\limits^{\vbox to -.5ex{\kern-0.75ex\hbox{$#1$}\vss}}}
\newcommand{\arrowvec}{\overrightarrow}
\renewcommand{\vec}{\mathbf}
\newcommand{\veci}{{\boldsymbol{\hat{\imath}}}}
\newcommand{\vecj}{{\boldsymbol{\hat{\jmath}}}}
\newcommand{\veck}{{\boldsymbol{\hat{k}}}}
\newcommand{\vecl}{\boldsymbol{\l}}
\newcommand{\utan}{\vec{\hat{t}}}
\newcommand{\unormal}{\vec{\hat{n}}}
\newcommand{\ubinormal}{\vec{\hat{b}}}

\newcommand{\dotp}{\bullet}
\newcommand{\cross}{\boldsymbol\times}
\newcommand{\grad}{\boldsymbol\nabla}
\newcommand{\divergence}{\grad\dotp}
\newcommand{\curl}{\grad\cross}
%% Simple horiz vectors
\renewcommand{\vector}[1]{\left\langle #1\right\rangle}


\pgfplotsset{compat=1.13}

\outcome{Use the Pigeonhole Principle}

\title{4.1 Pigeonhole Principle}

\begin{document}

\begin{abstract}
We use the Pigeonhole Principle
\end{abstract}

\maketitle

\section{Pigeonhole Principle}

\begin{proposition}[Pigeonhole Principle]
If $n+1$ pigeons are nesting in $n$ pigeonholes, then at least two pigeons are nesting in the same pigeonhole.
\end{proposition}

\begin{example}
A drawer contains blue socks and brown socks.  How many socks must be selected to ensure that there are two of the same color?\\
The two different colors represent the pigeonholes and the socks themselves represent the pigeons.  
According to the PHP, if 2+1 = 3 socks are selected, then at least two of them must be the same color.
\end{example}

\begin{problem}(problem 1)
A drawer contains red, green and purple socks. How many socks must be selected to ensure that there are two of the same color?\\
$\answer{4}$
\end{problem}


\begin{example}[example 2]
How many people are required to ensure that there are two with the same birthday in the group?\\
The different possible birthdays are the pigeons and the people are the pigeonholes.  
There are 366 possible birthdays (including leap day), so by the PHP, 367 people are required to be in the group to 
ensure that two of them have the same birthday.
\end{example}

\begin{problem}(problem 2)
Humans are known to have at most 200,000 hairs on their heads.  How large must the population of a city be in order to 
ensure that at least 2 people have the same number of hairs on their heads?\\
$\answer{200002}$

\begin{hint}
If $n$ is the number of hairs on a persons head, then $0 \leq n \leq 200,000$
\end{hint}

\end{problem}

\begin{example}[example 3]
Use the Pigeonhole Principle to explain why in any graph with $n$ vertices (and no loops), there are always two vertices with the same degree.\\
The degree of a vertex in a graph with $n$ vertices is a whole number from 0 to $n-1$.  Hence, it appears that there are $n$ possibilities 
for the degrees of our $n$ vertices.  However, noting that it is impossible for the graph to have both a vertex of degree 0 and a vertex of 
degree $n-1$, we see that there are really only $n-1$ possibilities for the degrees of our $n$ vertex. In other words, 
there are $n-1$ pigeonholes (the possible vertex degrees) and there are $n$ pigeons (the vertices), so by the PHP, 
there must be at least two vertices of the same degree.
\end{example}

\begin{problem}(problem 3)
A group of $n$ people convenes, and they begin shaking hands. Each pair of people can shake hands 0 or 1 times.
Use the Pigeonhole Principle to explain why there must be at least 2 people who shake the same number of hands.
\end{problem}

\begin{proposition}[Pigeonhole Principle, Strong Form]
If $kn+1$ pigeons are nesting in $n$ pigeonholes, then at least $k+1$ pigeons are nesting in the same pigeonhole.
\end{proposition}

\begin{proof}
Suppose not.  That is, suppose that each of the $n$ pigeonholes contains fewer than $k+1$ pigeons.  Then the number of pigeons in each pigeonhole is no more than $k$
and the total number of pigeons is no more than $nk$.  But this contradicts the assumption that there are $nk+1$ pigeons. Hence, there must be at least one pigeonhole containing 
$k+1$ (or more) pigeons.
\end{proof}

\begin{example}[example 4]
The edges of a graph are colored using 4 different colors.  How many edges must the graph have to ensure that there are at least 6 edges of the same color?\\
The 4 colors represent the pigeonholes and the edges represent the pigeonholes. By the strong form of the PHP, the graph must contain $5\times 4 + 1= 21$ edges to
ensure that there are $5+1=6$ of the same color.
\end{example}

\begin{problem}(problem 4)
The edges of a graph are colored using 5 different colors.  How many edges must the graph have to ensure that there are at least 4 edges of the same color?\\
$\answer{16}$
\end{problem}


\end{document}







