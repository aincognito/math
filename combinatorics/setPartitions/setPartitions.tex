\documentclass[handout]{ximera}

%% You can put user macros here
%% However, you cannot make new environments



\newcommand{\ffrac}[2]{\frac{\text{\footnotesize $#1$}}{\text{\footnotesize $#2$}}}
\newcommand{\vasymptote}[2][]{
    \draw [densely dashed,#1] ({rel axis cs:0,0} -| {axis cs:#2,0}) -- ({rel axis cs:0,1} -| {axis cs:#2,0});
}


\graphicspath{{./}{firstExample/}}
\usepackage{forest}
\usepackage{amsmath}
\usepackage{amssymb}
\usepackage{array}
\usepackage[makeroom]{cancel} %% for strike outs
\usepackage{pgffor} %% required for integral for loops
\usepackage{tikz}
\usepackage{tikz-cd}
\usepackage{tkz-euclide}
\usetikzlibrary{shapes.multipart}


%\usetkzobj{all}
\tikzstyle geometryDiagrams=[ultra thick,color=blue!50!black]


\usetikzlibrary{arrows}
\tikzset{>=stealth,commutative diagrams/.cd,
  arrow style=tikz,diagrams={>=stealth}} %% cool arrow head
\tikzset{shorten <>/.style={ shorten >=#1, shorten <=#1 } } %% allows shorter vectors

\usetikzlibrary{backgrounds} %% for boxes around graphs
\usetikzlibrary{shapes,positioning}  %% Clouds and stars
\usetikzlibrary{matrix} %% for matrix
\usepgfplotslibrary{polar} %% for polar plots
\usepgfplotslibrary{fillbetween} %% to shade area between curves in TikZ



%\usepackage[width=4.375in, height=7.0in, top=1.0in, papersize={5.5in,8.5in}]{geometry}
%\usepackage[pdftex]{graphicx}
%\usepackage{tipa}
%\usepackage{txfonts}
%\usepackage{textcomp}
%\usepackage{amsthm}
%\usepackage{xy}
%\usepackage{fancyhdr}
%\usepackage{xcolor}
%\usepackage{mathtools} %% for pretty underbrace % Breaks Ximera
%\usepackage{multicol}



\newcommand{\RR}{\mathbb R}
\newcommand{\R}{\mathbb R}
\newcommand{\C}{\mathbb C}
\newcommand{\N}{\mathbb N}
\newcommand{\Z}{\mathbb Z}
\newcommand{\dis}{\displaystyle}
%\renewcommand{\d}{\,d\!}
\renewcommand{\d}{\mathop{}\!d}
\newcommand{\dd}[2][]{\frac{\d #1}{\d #2}}
\newcommand{\pp}[2][]{\frac{\partial #1}{\partial #2}}
\renewcommand{\l}{\ell}
\newcommand{\ddx}{\frac{d}{\d x}}

\newcommand{\zeroOverZero}{\ensuremath{\boldsymbol{\tfrac{0}{0}}}}
\newcommand{\inftyOverInfty}{\ensuremath{\boldsymbol{\tfrac{\infty}{\infty}}}}
\newcommand{\zeroOverInfty}{\ensuremath{\boldsymbol{\tfrac{0}{\infty}}}}
\newcommand{\zeroTimesInfty}{\ensuremath{\small\boldsymbol{0\cdot \infty}}}
\newcommand{\inftyMinusInfty}{\ensuremath{\small\boldsymbol{\infty - \infty}}}
\newcommand{\oneToInfty}{\ensuremath{\boldsymbol{1^\infty}}}
\newcommand{\zeroToZero}{\ensuremath{\boldsymbol{0^0}}}
\newcommand{\inftyToZero}{\ensuremath{\boldsymbol{\infty^0}}}


\newcommand{\numOverZero}{\ensuremath{\boldsymbol{\tfrac{\#}{0}}}}
\newcommand{\dfn}{\textbf}
%\newcommand{\unit}{\,\mathrm}
\newcommand{\unit}{\mathop{}\!\mathrm}
%\newcommand{\eval}[1]{\bigg[ #1 \bigg]}
\newcommand{\eval}[1]{ #1 \bigg|}
\newcommand{\seq}[1]{\left( #1 \right)}
\renewcommand{\epsilon}{\varepsilon}
\renewcommand{\iff}{\Leftrightarrow}

\DeclareMathOperator{\arccot}{arccot}
\DeclareMathOperator{\arcsec}{arcsec}
\DeclareMathOperator{\arccsc}{arccsc}
\DeclareMathOperator{\si}{Si}
\DeclareMathOperator{\proj}{proj}
\DeclareMathOperator{\scal}{scal}
\DeclareMathOperator{\cis}{cis}
\DeclareMathOperator{\Arg}{Arg}
%\DeclareMathOperator{\arg}{arg}
\DeclareMathOperator{\Rep}{Re}
\DeclareMathOperator{\Imp}{Im}
\DeclareMathOperator{\sech}{sech}
\DeclareMathOperator{\csch}{csch}
\DeclareMathOperator{\Log}{Log}

\newcommand{\tightoverset}[2]{% for arrow vec
  \mathop{#2}\limits^{\vbox to -.5ex{\kern-0.75ex\hbox{$#1$}\vss}}}
\newcommand{\arrowvec}{\overrightarrow}
\renewcommand{\vec}{\mathbf}
\newcommand{\veci}{{\boldsymbol{\hat{\imath}}}}
\newcommand{\vecj}{{\boldsymbol{\hat{\jmath}}}}
\newcommand{\veck}{{\boldsymbol{\hat{k}}}}
\newcommand{\vecl}{\boldsymbol{\l}}
\newcommand{\utan}{\vec{\hat{t}}}
\newcommand{\unormal}{\vec{\hat{n}}}
\newcommand{\ubinormal}{\vec{\hat{b}}}

\newcommand{\dotp}{\bullet}
\newcommand{\cross}{\boldsymbol\times}
\newcommand{\grad}{\boldsymbol\nabla}
\newcommand{\divergence}{\grad\dotp}
\newcommand{\curl}{\grad\cross}
%% Simple horiz vectors
\renewcommand{\vector}[1]{\left\langle #1\right\rangle}


\pgfplotsset{compat=1.13}

\outcome{Use partitions to count the number of elements in a set}

\title{1.2 Set Partitions}

\begin{document}

\begin{abstract}
We use partitions to enumerate sets.
\end{abstract}

\maketitle

\section{Partitions}

When counting the number of elements in a set, it is sometimes helpful to break the set into
distinct pieces and count the number of elements in each piece. 

\begin{definition}[Partition]
A partition of a set $S$ is a collection of non-empty sets $S_1, S_2, ..., S_n$ satisfying the 
following three properties:
\begin{align*}
1) & S_i \subset S \text{ for } i = 1, 2, ..., n \text{ and} \\
2) & S_i \cap S_j = \emptyset \text{ for } i \neq j \text{ with } 1 \leq i,j \leq n\\
3) & S_1 \cup S_2 \cup \cdots \cup S_n = S
\end{align*}
\end{definition}

\begin{image}
\begin{tikzpicture}
\draw (0,0) -- (0,2) -- (3.2,2) -- (3.2, 0) -- (0,0);
\draw (1.4,0) .. controls (0.5,1) and (1,1.8) .. (1,2);
\draw (0,0.8) .. controls (1,1.4) and (2,0.8) .. (3.2,1.6);
\node at (0.6,0.4) {$S_1$};
\node at (0.5,1.5) {$S_2$};
\node at (2,1.5) {$S_3$};
\node at (2.2,0.5) {$S_4$};
\node at (3.5,1.5) {$S$};
\node at (1.6,-0.5) {A partition of $S$ into 4 subsets};
\end{tikzpicture}
\end{image}
\begin{remark}
The second property states that the subsets of $S$ are pairwise disjoint.
\end{remark}

We will denote the number of elements in a set $S$ by $|S|$. The main result
 is that the number of elements in a set is the sum of the elements in each 
subset of the partition:
\[
|S| = |S_1| + |S_2|+ \cdots + |S_n|
\]
where $S_1, S_2, ..., S_n$ is a partition of $S$.

\begin{example}[example 1]
Two 6-sided dice are thrown, one of which is red and the other green.  
In how many ways can their sum be at least 10?\\
We have seen that the total number of outcomes of the two dice when 
recorded as (red, green) is 36. How many of these 36 possible outcomes produce a 
sum of 10 or higher (10, 11 or 12)? 
If we let $S$ be the set of all outcomes whose sum is greater than 
or equal to 10, then we can partition $S$ into three 
subsets: $S_{10}, S_{11}$ and $S_{12}$
where the subscript identifies the sum. 
The set $S_{10}$ contains the outcomes $(6,4), (5,5)$ and $(4,6)$, 
so $|S_{10}| =3$. Verify that $|S_{11}| =2$ and $|S_{12}| =1$.\\
Finally, since $\{S_{10}, S_{11}, S_{12}\}$ is a partition of $S$, we have
\[
|S| = |S_{10}| +|S_{11}| +|S_{12}| = 3 + 2 + 1 = 6.
\]
Thus there are 6 ways to get a sum of at least 10 when rolling a pair of 6-sided dice.
\end{example}

\begin{problem}(problem 1a)
Two 6-sided dice are thrown, one of which is red and the other green.  
In how many ways can their sum be at most 5?\\
The number of outcomes whose sum is at most 5 is $\answer{10}$.
\end{problem}

\begin{problem}(problem 1b)
Two 6-sided dice are thrown, one of which is red and the other green.  
In how many ways can their sum be a prime number?\\
The number of outcomes whose sum is prime is $\answer{15}$.
\end{problem}

\begin{problem}(problem 1c)
Two 6-sided dice are thrown, one of which is red and the other green.  
In how many ways can their sum be an odd number?\\
The number of outcomes whose sum is odd is $\answer{18}$.
\end{problem}


\begin{example}[example 2] Steph has three different shirts, 
two different pairs of pants and 14 different hats. 
How many ways can Steph dress if the hat is optional?\\
If we denote by $S$ the set of all possible outfits Steph can wear, 
then we can partition $S$ into those outfits with a hat, $S_h$, 
and those without a hat, $S_{n}$.
The number of outfits with a hat is $|S_h| = 3 \times 2 \times 14 = 84$ and 
the number of outfits with no hat is $|S_n| = 3\times 2 = 6$. 
Thus, the total number of ways for Steph to dress is 
\[
|S| = |S_h| + |S_{n}| = 84 + 6 = 90
\]
Note that we could have counted this slightly 
differently, by including ``no hat" as a $15^{th}$ option 
for the hat: $|S| = 3 \times 2 \times 15 = 90$.
\end{example}

\begin{problem}(problem 2)
Jake is going jogging.  Jake has 4 t-shirts, 3 pairs of shorts and 2 pairs of sneakers.
How many jogging outfits can Jake make consisting of a pair of shorts, a pair of sneakers and an optional t-shirt? \; $\answer{30}$
\end{problem}


\begin{example}[example 3]
Sam wants to buy a slice of pizza.  
Sam enjoys three styles of pizza: Neapolitan, Chicago and Sicilian, 
and either eats the pizza plain or with pepperoni.  
How many different slices of pizza can Sam order?\\
If we denote by $S$ the set of all possible acceptable slices of pizza, 
then we can partition $S$ into the slices that have pepperoni, $S_p$ and 
those that do not, $S_{np}$. The number of slices of each is 3 and 
hence, the total number of possible slices Sam can order is
\[
|S| = |S_p| + |S_{np}| = 3+3 =6
\]
\end{example}

\begin{problem}(problem 3)
A menu consists of 6 appetizers, 10 entrees and 4 desserts. 
If a meal consists of an appetizer and an entree with dessert being 
optional, how many meals are possible?
$\; \answer{300}$
\end{problem}




\begin{example}[example 4]
Alex and Charlie want to have 2 or fewer children.  
How many combinations of boys and girls are possible assuming 
that birth order of each gender is irrelevant?\\

If $S$ is the set of all possible genders of their children, then we can partition $S$ according to the number of children. Let $S_0$ be the number of ways for them to have 0 children. Then $|S_0| = 1$ because the only way to have 0 children is to have 0 boys and 0 girls. Let $S_1$ be the number of ways for them to have 1 child. Then $|S_0| = 2$ because they can have 1 boy and 0 girls or 1 girl and 0 boys.
 Let $S_2$ be the number of ways for them to have 2 children. Then $|S_2| = 3$ because they
 can have 2 boys and 0 girls, 2 girls and 0 boys or 1 of each.
 Thus the total number of combinations of boys and girls that they can have is
 \[
 |S| = |S_0|+|S_1|+|S_2|= 1 + 2+3 = 6
 \]
 \end{example}
 

\begin{problem}(problem 4)
Fluffy is about to give birth to either 5 or 6 kittens.  
How many combinations of males and females are possible assuming 
that birth order of each gender is irrelevant?\\
The total number of combinations of males and females that Fluffy can have is
$\answer{13}$
\end{problem}

\end{document}


