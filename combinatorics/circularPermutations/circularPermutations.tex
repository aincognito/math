\documentclass[handout]{ximera}

%% You can put user macros here
%% However, you cannot make new environments



\newcommand{\ffrac}[2]{\frac{\text{\footnotesize $#1$}}{\text{\footnotesize $#2$}}}
\newcommand{\vasymptote}[2][]{
    \draw [densely dashed,#1] ({rel axis cs:0,0} -| {axis cs:#2,0}) -- ({rel axis cs:0,1} -| {axis cs:#2,0});
}


%\usepackage{tcolorbox} %%Needed for Derivative Definition supposedly and product rule, natural exp log, quotient rule, inverse trig, rates of change


% \graphicspath{{./}{firstExample/}}
% \usepackage{forest}
\usepackage{amsmath}
\usepackage{amssymb}
\usepackage{array}
\usepackage[makeroom]{cancel} %% for strike outs
\usepackage{pgffor} %% required for integral for loops
\usepackage{tikz}
\usepackage{tikz-cd}
\usepackage{tkz-euclide}
\usetikzlibrary{shapes.multipart}


% \usetkzobj{all}
\tikzstyle geometryDiagrams=[ultra thick,color=blue!50!black]


\usetikzlibrary{arrows}
\tikzset{>=stealth,commutative diagrams/.cd,
  arrow style=tikz,diagrams={>=stealth}} %% cool arrow head
\tikzset{shorten <>/.style={ shorten >=#1, shorten <=#1 } } %% allows shorter vectors

\usetikzlibrary{backgrounds} %% for boxes around graphs
\usetikzlibrary{shapes,positioning}  %% Clouds and stars
\usetikzlibrary{matrix} %% for matrix
\usepgfplotslibrary{polar} %% for polar plots
\usepgfplotslibrary{fillbetween} %% to shade area between curves in TikZ



%\usepackage[width=4.375in, height=7.0in, top=1.0in, papersize={5.5in,8.5in}]{geometry}
%\usepackage[pdftex]{graphicx}
%\usepackage{tipa}
%\usepackage{txfonts}
%\usepackage{textcomp}
%\usepackage{amsthm}
%\usepackage{xy}
%\usepackage{fancyhdr}
%\usepackage{xcolor}
%\usepackage{mathtools} %% for pretty underbrace % Breaks Ximera
%\usepackage{multicol}



\newcommand{\RR}{\mathbb R}
\newcommand{\R}{\mathbb R}
\newcommand{\C}{\mathbb C}
\newcommand{\N}{\mathbb N}
\newcommand{\Z}{\mathbb Z}
\newcommand{\dis}{\displaystyle}
%\renewcommand{\d}{\,d\!}
\renewcommand{\d}{\mathop{}\!d}
\newcommand{\dd}[2][]{\frac{\d #1}{\d #2}}
\newcommand{\pp}[2][]{\frac{\partial #1}{\partial #2}}
\renewcommand{\l}{\ell}
\newcommand{\ddx}{\frac{d}{\d x}}
\newcommand{\ppx}{\frac{\partial}{\partial x}}
\newcommand{\ppy}{\frac{\partial}{\partial y}}

\newcommand{\zeroOverZero}{\ensuremath{\boldsymbol{\tfrac{0}{0}}}}
\newcommand{\inftyOverInfty}{\ensuremath{\boldsymbol{\tfrac{\infty}{\infty}}}}
\newcommand{\zeroOverInfty}{\ensuremath{\boldsymbol{\tfrac{0}{\infty}}}}
\newcommand{\zeroTimesInfty}{\ensuremath{\small\boldsymbol{0\cdot \infty}}}
\newcommand{\inftyMinusInfty}{\ensuremath{\small\boldsymbol{\infty - \infty}}}
\newcommand{\oneToInfty}{\ensuremath{\boldsymbol{1^\infty}}}
\newcommand{\zeroToZero}{\ensuremath{\boldsymbol{0^0}}}
\newcommand{\inftyToZero}{\ensuremath{\boldsymbol{\infty^0}}}


\newcommand{\numOverZero}{\ensuremath{\boldsymbol{\tfrac{\#}{0}}}}
\newcommand{\dfn}{\textbf}
%\newcommand{\unit}{\,\mathrm}
\newcommand{\unit}{\mathop{}\!\mathrm}
%\newcommand{\eval}[1]{\bigg[ #1 \bigg]}
\newcommand{\eval}[1]{ #1 \bigg|}
\newcommand{\seq}[1]{\left( #1 \right)}
\renewcommand{\epsilon}{\varepsilon}
\renewcommand{\iff}{\Leftrightarrow}

\DeclareMathOperator{\arccot}{arccot}
\DeclareMathOperator{\arcsec}{arcsec}
\DeclareMathOperator{\arccsc}{arccsc}
\DeclareMathOperator{\si}{Si}
\DeclareMathOperator{\proj}{proj}
\DeclareMathOperator{\scal}{scal}
\DeclareMathOperator{\cis}{cis}
\DeclareMathOperator{\Arg}{Arg}
%\DeclareMathOperator{\arg}{arg}
\DeclareMathOperator{\Rep}{Re}
\DeclareMathOperator{\Imp}{Im}
\DeclareMathOperator{\sech}{sech}
\DeclareMathOperator{\csch}{csch}
\DeclareMathOperator{\Log}{Log}

\newcommand{\tightoverset}[2]{% for arrow vec
  \mathop{#2}\limits^{\vbox to -.5ex{\kern-0.75ex\hbox{$#1$}\vss}}}
\newcommand{\arrowvec}{\overrightarrow}
\renewcommand{\vec}{\mathbf}
\newcommand{\veci}{{\boldsymbol{\hat{\imath}}}}
\newcommand{\vecj}{{\boldsymbol{\hat{\jmath}}}}
\newcommand{\veck}{{\boldsymbol{\hat{k}}}}
\newcommand{\vecl}{\boldsymbol{\l}}
\newcommand{\utan}{\vec{\hat{t}}}
\newcommand{\unormal}{\vec{\hat{n}}}
\newcommand{\ubinormal}{\vec{\hat{b}}}

\newcommand{\dotp}{\bullet}
\newcommand{\cross}{\boldsymbol\times}
\newcommand{\grad}{\boldsymbol\nabla}
\newcommand{\divergence}{\grad\dotp}
\newcommand{\curl}{\grad\cross}
%% Simple horiz vectors
\renewcommand{\vector}[1]{\left\langle #1\right\rangle}


\pgfplotsset{compat=1.13}

\outcome{Define and enumerate circular permutations}

\title{1.6 Circular Permutations}

\begin{document}

\begin{abstract}
We define and enumerate circular permutations.
\end{abstract}

\maketitle

\section{Circular Permutations}



\begin{definition}[Circular Permutation]
A circular permutation of selection of objects placed in a circular arrangement in a particular order.
\end{definition}


In a circular permutation, all positions on the circle are considered equivalent. Thus, the position of an object
is solely determined by its position relative to the other objects. By contrast, the objects in an ordinary permutations have 
absolute positions- first, second, third etc.  An ordinary permutation can be thought of as a linear permutation.\\

\begin{proposition}
The number of circular permutations of $k$ objects taken from a group of $n$ objects ($n \geq k$) is given by
\[
C(n,k)\cdot (k-1)! = \frac{P(n,k)}{k}
\]
\end{proposition}

\begin{proof}
First, we select the $k$ objects to be placed in the circular permutation.  This can be done $C(n,k)$ ways.\\
Second, we arrange the $k$ objects in a circle and use the FPC. When the first object is placed in the circle, all of the positions are equivalent, so there is only 1 choice. Once the first object is placed, the remaining positions in the circle become distinct.  Hence, there are $k-1$ choices for the position of the second object in the circle, $k-2$ choices for the 
third object, and so on. Thus, the number of ways to arrange the $k$ objects in a circle  is $(k-1)!$.
Finally, we multiply the number of ways to select the $k$ objects by the number of ways to arrange them in a circle to obtain
\[
C(n,k)\cdot (k-1)!
\]
To see that this is equal to $\frac{P(n,k)}{k}$, we note that 
\[
\frac{(k-1)!}{k!} = \frac{1}{k}
\]
\end{proof}


\begin{corollary}
The number of circular permutations of $n$ objects (taken $n$ at a time) is $(n-1)!$
\end{corollary}
\begin{proof}
Applying the proposition with $k = n$ yields that the number of such permutations is
\[
C(n,n)\cdot (n-1)! = (n-1)!
\]
\end{proof}

\begin{example}[example 1]
In how many ways can 10 first graders march in a circle?\\
Here, we are counting the number of circular permutations of $10$ objects (taken $10$ at a time).
According to the corollary, the number of ways for the 10 children to march in a circle is $9!$.
\end{example}


\begin{problem}(problem 1)
In how many ways can the 20 members of the high-school band march in a circle during the halftime show? $\; \answer{19!}$
\end{problem}


\begin{example}[example 2]
How many ways can King Arthur, Guinevere, Merlin, Lancelot and Galahad be seated at the round table?\\
Assuming that all seats at the table are equivalent, we seek the number of circular 
permutations of 5 objects (taken 5 at a time).  According to the corollary, the number of ways for 
them to be seated around the table is $4! =24$
\end{example}


\begin{problem}(problem 2)
In how many ways can the President, Vice-President, Secretary of State, Secretary of the Treasury, 
Secretary of the Interior, Secretary of Labor and a secretary be seated at a round table? $\; \answer{720}$.

\end{problem}


\begin{example}[example 3]

\end{example}


\begin{problem}(problem 3)


\end{problem}


\begin{example}[example 4]

\end{example}


\begin{problem}(problem 4)


\end{problem}


\begin{example}[example 5]

\end{example}


\begin{problem}(problem 5)


\end{problem}


\begin{example}[example 6]

\end{example}


\begin{problem}(problem 6)


\end{problem}


\end{document}

