\documentclass[handout]{ximera}

%% You can put user macros here
%% However, you cannot make new environments



\newcommand{\ffrac}[2]{\frac{\text{\footnotesize $#1$}}{\text{\footnotesize $#2$}}}
\newcommand{\vasymptote}[2][]{
    \draw [densely dashed,#1] ({rel axis cs:0,0} -| {axis cs:#2,0}) -- ({rel axis cs:0,1} -| {axis cs:#2,0});
}


%\usepackage{tcolorbox} %%Needed for Derivative Definition supposedly and product rule, natural exp log, quotient rule, inverse trig, rates of change


% \graphicspath{{./}{firstExample/}}
% \usepackage{forest}
\usepackage{amsmath}
\usepackage{amssymb}
\usepackage{array}
\usepackage[makeroom]{cancel} %% for strike outs
\usepackage{pgffor} %% required for integral for loops
\usepackage{tikz}
\usepackage{tikz-cd}
\usepackage{tkz-euclide}
\usetikzlibrary{shapes.multipart}


% \usetkzobj{all}
\tikzstyle geometryDiagrams=[ultra thick,color=blue!50!black]


\usetikzlibrary{arrows}
\tikzset{>=stealth,commutative diagrams/.cd,
  arrow style=tikz,diagrams={>=stealth}} %% cool arrow head
\tikzset{shorten <>/.style={ shorten >=#1, shorten <=#1 } } %% allows shorter vectors

\usetikzlibrary{backgrounds} %% for boxes around graphs
\usetikzlibrary{shapes,positioning}  %% Clouds and stars
\usetikzlibrary{matrix} %% for matrix
\usepgfplotslibrary{polar} %% for polar plots
\usepgfplotslibrary{fillbetween} %% to shade area between curves in TikZ



%\usepackage[width=4.375in, height=7.0in, top=1.0in, papersize={5.5in,8.5in}]{geometry}
%\usepackage[pdftex]{graphicx}
%\usepackage{tipa}
%\usepackage{txfonts}
%\usepackage{textcomp}
%\usepackage{amsthm}
%\usepackage{xy}
%\usepackage{fancyhdr}
%\usepackage{xcolor}
%\usepackage{mathtools} %% for pretty underbrace % Breaks Ximera
%\usepackage{multicol}



\newcommand{\RR}{\mathbb R}
\newcommand{\R}{\mathbb R}
\newcommand{\C}{\mathbb C}
\newcommand{\N}{\mathbb N}
\newcommand{\Z}{\mathbb Z}
\newcommand{\dis}{\displaystyle}
%\renewcommand{\d}{\,d\!}
\renewcommand{\d}{\mathop{}\!d}
\newcommand{\dd}[2][]{\frac{\d #1}{\d #2}}
\newcommand{\pp}[2][]{\frac{\partial #1}{\partial #2}}
\renewcommand{\l}{\ell}
\newcommand{\ddx}{\frac{d}{\d x}}
\newcommand{\ppx}{\frac{\partial}{\partial x}}
\newcommand{\ppy}{\frac{\partial}{\partial y}}

\newcommand{\zeroOverZero}{\ensuremath{\boldsymbol{\tfrac{0}{0}}}}
\newcommand{\inftyOverInfty}{\ensuremath{\boldsymbol{\tfrac{\infty}{\infty}}}}
\newcommand{\zeroOverInfty}{\ensuremath{\boldsymbol{\tfrac{0}{\infty}}}}
\newcommand{\zeroTimesInfty}{\ensuremath{\small\boldsymbol{0\cdot \infty}}}
\newcommand{\inftyMinusInfty}{\ensuremath{\small\boldsymbol{\infty - \infty}}}
\newcommand{\oneToInfty}{\ensuremath{\boldsymbol{1^\infty}}}
\newcommand{\zeroToZero}{\ensuremath{\boldsymbol{0^0}}}
\newcommand{\inftyToZero}{\ensuremath{\boldsymbol{\infty^0}}}


\newcommand{\numOverZero}{\ensuremath{\boldsymbol{\tfrac{\#}{0}}}}
\newcommand{\dfn}{\textbf}
%\newcommand{\unit}{\,\mathrm}
\newcommand{\unit}{\mathop{}\!\mathrm}
%\newcommand{\eval}[1]{\bigg[ #1 \bigg]}
\newcommand{\eval}[1]{ #1 \bigg|}
\newcommand{\seq}[1]{\left( #1 \right)}
\renewcommand{\epsilon}{\varepsilon}
\renewcommand{\iff}{\Leftrightarrow}

\DeclareMathOperator{\arccot}{arccot}
\DeclareMathOperator{\arcsec}{arcsec}
\DeclareMathOperator{\arccsc}{arccsc}
\DeclareMathOperator{\si}{Si}
\DeclareMathOperator{\proj}{proj}
\DeclareMathOperator{\scal}{scal}
\DeclareMathOperator{\cis}{cis}
\DeclareMathOperator{\Arg}{Arg}
%\DeclareMathOperator{\arg}{arg}
\DeclareMathOperator{\Rep}{Re}
\DeclareMathOperator{\Imp}{Im}
\DeclareMathOperator{\sech}{sech}
\DeclareMathOperator{\csch}{csch}
\DeclareMathOperator{\Log}{Log}

\newcommand{\tightoverset}[2]{% for arrow vec
  \mathop{#2}\limits^{\vbox to -.5ex{\kern-0.75ex\hbox{$#1$}\vss}}}
\newcommand{\arrowvec}{\overrightarrow}
\renewcommand{\vec}{\mathbf}
\newcommand{\veci}{{\boldsymbol{\hat{\imath}}}}
\newcommand{\vecj}{{\boldsymbol{\hat{\jmath}}}}
\newcommand{\veck}{{\boldsymbol{\hat{k}}}}
\newcommand{\vecl}{\boldsymbol{\l}}
\newcommand{\utan}{\vec{\hat{t}}}
\newcommand{\unormal}{\vec{\hat{n}}}
\newcommand{\ubinormal}{\vec{\hat{b}}}

\newcommand{\dotp}{\bullet}
\newcommand{\cross}{\boldsymbol\times}
\newcommand{\grad}{\boldsymbol\nabla}
\newcommand{\divergence}{\grad\dotp}
\newcommand{\curl}{\grad\cross}
%% Simple horiz vectors
\renewcommand{\vector}[1]{\left\langle #1\right\rangle}


\pgfplotsset{compat=1.13}

\outcome{Use the Inclusion-Exclusion Principle to count the number of elements in a set}

\title{2.2 Inclusion-Exclusion Principle}

\begin{document}



\begin{abstract}
We use the Inclusion-Exclusion Principle to enumerate sets.
\end{abstract}

\maketitle


Recall the Inclusion-Exclusion Principle for two sets:

\[
|A \cup B| = |A| + |B| - |A \cap B|
\]

In this section, we will generalize this to $n$ sets, but first, let's extend it from two sets to three sets.

\begin{proposition}
Given sets $A, B$ and $C$ with finite cardinalities $|A|, |B|$ and $|C|$, we have
\[
|A \cup B \cup C| = |A| + |B| + |C| - |A \cap B| - |A \cap C| - |B \cap C| + |A \cap B \cap C|
\]
\end{proposition}




\begin{proof}
Consider $A \cup B$ as one set and $C$ as the second set and apply the 
Inclusion-Exclusion Principle for two sets.  We have:
\[
|A \cup B \cup C| = |A \cup B| + |C| - | (A\cup B) \cap C|
\]
Next, use the Inclusion-Exclusion Principle for two sets on the first term, and 
distribute the intersection across the union in the third term to obtain:
\[
|A \cup B \cup C| = |A| + |B| - |A \cap B| + |C| - | (A \cap C) \cup (B\cap C) |
\]
Now, use the Inclusion Exclusion Principle for two sets on the fourth term to get:
\[
|A \cup B \cup C| = |A| + |B| - |A \cap B| + |C| - \bigg(| (A \cap C)| + |B\cap C| - |(A\cap B) \cap (B\cap C)| \bigg)
\]
Finally, the set in the last term is just $A \cap B \cap C$, so we have the final form of the 
Inclusion-Exclusion Principle for three sets:
\[
|A \cup B \cup C| = |A| + |B| + |C| - |A \cap B| - | (A \cap C)| - |B\cap C| +|A\cap B\cap C|
\]
\end{proof}



For four sets, the Inclusion-Exclusion Principle is
\begin{align*}
|A \cup B \cup C \cup D | = &\;\;|A|  + |B| + |C| + |D| \\
&- |A \cap B| - | (A \cap C)| - |A \cap D| - |B\cap C| - | B \cap D| 
- |C \cap D| \\
& +|A\cap B\cap C| + |A\cap B\cap D|+|A\cap C\cap D|+| B\cap C\cap D| \\
&- |A\cap B\cap C \cap D|
\end{align*}
At this point, the number of terms is becoming quite large, so summation notation is to be preferred:
\[
|A_1 \cup A_2 \cup A_3 \cup A_4 | = \sum_{i = 1}^4 |A_i| - \sum_{1 = i < j}^4 |A_i \cap A_j| + 
\]
\[
\sum_{1 = i < j< k }^4 |A_i \cap A_j \cap A_k| - |A_1 \cap A_2 \cap A_3 \cap A_4 |
\]

We are now ready to state and prove the general case of the Inclusion-Exclusion Principle.
\begin{proposition}[Inclusion-Exclusion Principle]
Given $n$ finite sets, $A_1, A_2, \ldots, A_n$ we have
\[
|A_1 \cup A_2 \cup \cdots \cup A_n| 
\]
\[
= \sum_{i =1}^n |A_i| - \sum_{1 =i<j}^n |A_i\cap A_j| +
\sum_{1 =i<j< k}^n |A_i\cap A_j \cap A_k| - \cdots + (-1)^{n-1} |A_1 \cap A_2 \cap \cdots \cap A_n|
\]
\end{proposition}

\begin{proof}
We will prove the proposition by induction on the number of sets, $n$. The base case, $n=2$ was proved in section 2.1.
For the induction hypothesis, we assume that the result is true for some number of sets $n$. 
We then wish to show that the result is true for $n+1$ sets. We will do this in a manner similar to the way we began
this section, obtaining the Inclusion-Exclusion Principle for three sets as a consequence of the result for two sets.
Thus, we consider the union of the first $n$ sets to be a single set and we obtain:
\[
|(A_1 \cup A_2 \cup \cdots \cup A_n) \cup A_{n+1}| 
\]
\[
= |A_1 \cup A_2 \cup \cdots \cup A_n| + |A_{n+1}| - 
|(A_1 \cup A_2 \cup \cdots \cup A_n) \cap A_{n+1}|
\]
In the last term, we can distribute the intersection over the unions to obtain a union of $n$ sets:
\[
(A_1 \cup A_2 \cup \cdots \cup A_n) \cap A_{n+1} = 
(A_1 \cap A_{n+1}) \cup (A_2 \cap A_{n+1}) \cup \cdots \cup (A_n \cap A_{n+1})
\]
We can apply the induction hypothesis the number of elements in this union, and noting 
that $(A_i \cap A_{n+1}) \cap (A_j \cap A_{n+1})  = (A_i \cap A_j \cap A_{n+1})$, we obtain
\[
|(A_1 \cap A_{n+1}) \cup (A_2 \cap A_{n+1}) \cup \cdots \cup (A_n \cap A_{n+1})|
\]
\[
= \sum_{i = 1}^n |A_i \cap A_{n+1}| - \sum_{1 = i < j}^n |A_i \cap A_j \cap A_{n+1}| + \cdots + (-1)^{n-1} |A_1 \cap A_2 \cap \cdots \cap A_{n+1}|
\]
Inserting this back into the first equation, and applying the induction hypothesis to the 
first term in that equation, we get
\begin{align*}
|(A_1 \cup A_2 \cup \cdots &\cup A_n) \cup A_{n+1}|  = \\
& |A_1 \cup A_2 \cup \cdots \cup A_n| + |A_{n+1}| - |(A_1 \cup A_2 \cup \cdots \cup A_n) \cap A_{n+1}| \\
 &= \bigg( \sum_{i =1}^n |A_i| - \sum_{1 =i<j}^n |A_i\cap A_j| \\
 & +\sum_{1 =i<j< k}^n |A_i\cap A_j \cap A_k| - \cdots + (-1)^{n-1} |A_1 \cap A_2 \cap \cdots \cap A_n|\bigg) \\
  &+ |A_{n+1}| - \bigg(\sum_{i = 1}^n |A_i \cap A_{n+1}| - \sum_{1 = i < j}^n |A_i \cap A_j \cap A_{n+1}| \\
  &+ \cdots + (-1)^{n-1} |A_1 \cap A_2 \cap \cdots \cap A_{n+1}|\bigg)\\
  &= \sum_{i =1}^{n+1} |A_i| - \sum_{1 =i<j}^{n+1} |A_i\cap A_j| \\
& +\sum_{1 =i<j< k}^{n+1} |A_i\cap A_j \cap A_k| - \cdots + (-1)^{n} |A_1 \cap A_2 \cap \cdots \cap A_{n+1}| \\
\end{align*}
as required.

\end{proof}

\end{document}


\begin{example}[example 1]
How many numbers in the set $S = \{1,2,3, \ldots, 1000\}$ are multiples of either 2 or 5?\\
Let $A$ be the subset of $S$ consisting of the multiples of 2, and $B$ be the subset of $S$ consisting of the multiples of 5.
We seek $|A \cup B|$. According to the proposition, 
\[
|A \cup B| = |A| + |B| - |A \cap B|
\]
We have $|A| =500$ and $|B| = 200$ and $|A \cap B| = 100$ since the set $A \cap B$ consists of the multiples of 10.  
Using the Inclusion-Exclusion Principle (for two sets), we can conclude 
that the number of elements of $S$ that are either multiples of 2 or 5 is
\[
|A \cup B| = 500 + 200 - 100 = 600
\]
\end{example}

\begin{problem}(problem 1)
How many numbers from the given set $S= \{1,2,3, \ldots, 1000\}$ are multiples of the given numbers $a$ and $b$?\\
a)$\; a = 4, b = 5 \;\; \answer{400}$\\
b)$\;  a = 7, b = 12 \;\; \answer{214}$\\
c)$\; a = 6, b = 10 \;\; \answer{233}$\\
d)$\;  a = 18, b = 30 \;\; \answer{77}$ \\
\end{problem}

Recall set complements from section 1.2:

\begin{definition}[Complement]
Given a universal set $U$ and a subset, $A \subset U$, we define the complement of $A$, denoted, $A^c$ by
\[
A^c = \{x \in U | x \notin A \}
\]
\end{definition}


\begin{image}
\begin{tikzpicture}
\draw (0,0) -- (0,2) -- (3.2,2) -- (3.2, 0) -- (0,0);
\draw (1.6, 1) circle (0.7);
\node at (1.5,0.9) {$A$};
\node at (2.7,1.5) {$A^c$};
\node at (3.5,0.5) {$U$};
\node at (1.6,-0.5) {$A$ and $A^c$};
\end{tikzpicture}
\end{image}


De Morgan's Law gives us a formula for the complement of a union in terms of the intersection of their complements.

\begin{proposition}[De Morgan's Law]
Given sets $A$ and $B$ in a universal set $U$, De Morgan's Law states that
\[
\left(A \cup B\right)^c = A^c \cap B^c
\]
\end{proposition}

\begin{proof}
First, let $x \in (A \cup B)^c$.  Then  $x \notin A \cup B$ which implies that $x \notin A$ and $x \notin B$. 
By definition of complement, this means that $x \in A^c$ and $x \in B^c$.   
Finally, by definition of intersection, $x \in A^c \cap B^c$.
Thus $(A \cup B)^c \subseteq A^c \cap B^c$.\\
Next, let $x \in A^c \cap B^c$. By definition of intersection, $x \in A^c$ and $x \in B^c$. 
Then, $x \notin A$ and $x \notin B$ which implies that $x \notin A \cup B$. Thus $x \in (A \cup B)^c$ and 
we have shown that  $A^c \cap B^c\subseteq (A \cup B)^c $.\\
By virtue of being subsets of each other, the two sets are equal and De Morgan's Law is proved.
\end{proof}


\begin{remark}
There are actually two of De Morgan's Laws.  The second is
\[
\left(A \cap B\right)^c = A^c \cup B^c
\]
The proof of this second law follows immediately from the first law by replacing $A$ and $B$ with $A^c$ and $B^c$
and using the fact that $(S^c)^c = S$ for any set $S$.
\end{remark}


The next example uses De Morgan's Law in conjunction with the Rule for Complements and the 
Inclusion-Exclusion Principle (for two sets).

\begin{example}[example 2]
Computer passwords are eight characters long where a character can be either an upper case letter, 
lower case letter, digit 0-9, or one of 8 special symbols. 
How many passwords are possible if a password must contain at least one digit and at least one special symbol?\\

Let $A$ be the set of passwords that do not contain any digits and let $B$ be the set of passwords that do not
contain any special symbols. Since our passwords must contain a digit and a special symbol, we seek $|A^c \cap B^c|$.
By De Morgan's Law, 
\[
A^c \cap B^c = (A \cup B)^c
\]
Then according to the Rule for Complements,
\[
|(A \cup B)^c| = |S| - |A \cup B|,
\]
where $S$ is the set of all possible passwords with no restrictions. 
Combining the two equations above with the Inclusion-Exclusion Principle (for two sets), we have
\[
|A^c \cap B^c| = |S| - |A| - |B| + |A\cap B|
\]
Each of the terms on the right hand side can be computed using the Fundamental Principle of Counting. Since there are
70 different characters in total with 10 digits and 8 special symbols,
\[
|S| = 70^8, \; |A| = 60^8, \; |B| = 62^8 \; \text{ and } \; |A\cap B| = 52^8
\]
Note that the passwords in the set $A \cap B$ do not contain either digits or special symbols, so there are $70-10-8 = 52$
available characters for them. In conclusion, the number of acceptable passwords is
\[
70^8 - 60^8 - 62^8 + 52^8 \approx 243638122947000
\]

\end{example}

\begin{problem}(problem 2)
Computer passwords are eight characters long where a character can be either an upper case letter, 
lower case letter, digit 0-9, or one of 8 special symbols. 
How many passwords are possible if a password must contain:\\
a) at least one digit and at least one upper case letter? $\; \answer{70^8 - 60^8 - 44^8 + 34^8}$\\
b) at least special symbol and at least one lower case letter? $\; \answer{70^8 - 62^8 - 44^8 + 36^8}$\\
c) at least one letter? $\; \answer{70^8 - 18^8}$\\
\end{problem}

\begin{example}[example 3]
How many permutations are there of the letters in the phrase MATH ROCKS in which neither 
the word MATH nor the word ROCKS appears?\\
Such permutations include MRAOTCHKS and SKCORHTAM, but not RKMATHSOC, HROCKSMTA, or ROCKSMATH.
Let $A$ be the set of permutations which include the word MATH and let $B$ be the set of permutations which include the word
ROCKS.  We seek $|A^c \cap B^c|$. By De Morgan's Law, this is the same as $|(A \cup B)^c|$. 
Let $S$ be the set of all permutations of the letters in the phrase. Then, applying the Rule for Complements,
and the Inclusion-Exclusion Principle (for two sets), we have
\begin{align*}
|A^c \cap B^c| &= |(A \cup B)^c|\\
               &= |S| - |A\cup B|\\
               &= |S| - \Big(|A| + |B| -|A\cap B|\Big)\\
               &= |S| - |A| - |B| + |A\cap B|
\end{align*}
The number of elements in $A$ is the number of permutations of the symbols MATH, R, O, C, K, S.  
Since this is 6 distinct symbols, $|A| = 6!$. The number of elements in $B$ is the number of 
permutations of the symbols M, A, T, H, ROCKS.  
Since this is 5 distinct symbols, $|B| = 5!$. The number of elements in $A \cap B$ is 2 and the number of elements in $S$
is $9!$. Hence, the total number of ways to permute the letters in the phrase MATH ROCKS which do not 
include either of the words MATH or ROCKS is
\[
|A^c \cap B^c| = 9! - 6! - 5! + 2 = 362042
\]
\end{example}


\begin{problem}(problem 3)
How many permutations are there of the letters in the phrase DOGS RULE in which neither 
the word DOGS nor the word RULE appears? $\; \answer{40082}$
\end{problem}

\begin{example}[example 4]
How many permutations are there of the numbers 1-7 such that the 1 is not in the first 
position and the 2 is not in the second position?\\
Let $A$ be the permutations with 1 in the first position, $B$ be the permutations with 2 in the 
second position and $s$ be the set of all permutations. We seek $|A^c \cap B^c|$. 
De Morgan's Law, the Rule for Complements and the Inclusion-Exclusion Principle (for two sets) gives:
\begin{align*}
|A^c \cap B^c| &= |(A \cup B)^c|\\
               &= |S| - |A\cup B|\\
               &= |S| - \Big(|A| + |B| - |A\cap B|\Big)\\
               &= |S| - |A| - |B| + |A\cap B|\\
               &= 7! - 6! - 6! + 5!\\
               &= 3720
\end{align*}

\end{example}

\begin{problem}(problem 4)
How many permutations are there of the letters in the word ABCDEF in which the letter A is not 
first and the letter F is not last? $\; answer{504}$
\end{problem}


\begin{example}[example 5]
How many non-negative integer solutions are there to the equation
\[
x_1 + x_2 + x_3 + x_4 + x_5 = 20
\]
if $x_1 \leq 4$ and $x_2 \leq 3$?\\
With no restrictions on the variables (except that they be non-negative integers), 
the total number of solutions is $C(24, 20)$ by the stars and bars method.
Let $A$ be the set of all solutions which have $x_1 > 4$ and let $B$ be the set of all solutions with $x_2 > 3$.
We wish to find $|A^c \cap B^c|$.
De Morgan's Law, the Rule for Complements and the Inclusion-Exclusion Principle (for two sets) gives:
\begin{align*}
|A^c \cap B^c| &= |(A \cup B)^c|\\
               &= |S| - |A\cup B|\\
               &= |S| - \Big(|A| + |B| - |A\cap B|\Big)\\
               &= |S| - |A| - |B| + |A\cap B|\\
\end{align*}
where $S$ is the set of all non-negative integer solutions. By letting $y_1 = x_1 -5$, we see that 
the number of elements in the set $A$
is the number of non-negative integer solutions to the equation:
\[
y_1 + x_2 + x_3 + x_4 + x_5 = 15
\]
which is $C(19, 15)$. By letting $y_2 = x_2 -4$, we see that the number of elements in the set $B$
is the number of non-negative integer solutions to the equation:
\[
x_1 + y_2 + x_3 + x_4 + x_5 = 16
\]
which is $C(20, 16)$. Finally, by letting $y_1 = x_1 -5$ and $y_2 = x_2 -4$ we see that the number of 
elements in the set $A \cap B$
is the number of non-negative integer solutions to the equation:
\[
y_1 + y_2 + x_3 + x_4 + x_5 = 11
\]
which is $C(15, 11)$. 
Thus the total number of non-negative integer solutions to the equation
\[
x_1 + x_2 + x_3 + x_4 + x_5 = 20
\]
with $x_1 \leq 4$ and $x_2 \leq 3$ is
\[
|A^c \cap B^c| = \binom{24}{20} - \binom{19}{15} - \binom{20}{16} + \binom{15}{11} = 3270
\]

\end{example}


\begin{problem}(problem 5)
How many non-negative integer solutions are there to the equation
\[
x + y + z + w = 12
\]
if $y < 5$ and $w \leq 2 ? \; \answer{170}$

\end{problem}
\end{document}
