\documentclass[handout]{ximera}

%% You can put user macros here
%% However, you cannot make new environments



\newcommand{\ffrac}[2]{\frac{\text{\footnotesize $#1$}}{\text{\footnotesize $#2$}}}
\newcommand{\vasymptote}[2][]{
    \draw [densely dashed,#1] ({rel axis cs:0,0} -| {axis cs:#2,0}) -- ({rel axis cs:0,1} -| {axis cs:#2,0});
}


%\usepackage{tcolorbox} %%Needed for Derivative Definition supposedly and product rule, natural exp log, quotient rule, inverse trig, rates of change


% \graphicspath{{./}{firstExample/}}
% \usepackage{forest}
\usepackage{amsmath}
\usepackage{amssymb}
\usepackage{array}
\usepackage[makeroom]{cancel} %% for strike outs
\usepackage{pgffor} %% required for integral for loops
\usepackage{tikz}
\usepackage{tikz-cd}
\usepackage{tkz-euclide}
\usetikzlibrary{shapes.multipart}


% \usetkzobj{all}
\tikzstyle geometryDiagrams=[ultra thick,color=blue!50!black]


\usetikzlibrary{arrows}
\tikzset{>=stealth,commutative diagrams/.cd,
  arrow style=tikz,diagrams={>=stealth}} %% cool arrow head
\tikzset{shorten <>/.style={ shorten >=#1, shorten <=#1 } } %% allows shorter vectors

\usetikzlibrary{backgrounds} %% for boxes around graphs
\usetikzlibrary{shapes,positioning}  %% Clouds and stars
\usetikzlibrary{matrix} %% for matrix
\usepgfplotslibrary{polar} %% for polar plots
\usepgfplotslibrary{fillbetween} %% to shade area between curves in TikZ



%\usepackage[width=4.375in, height=7.0in, top=1.0in, papersize={5.5in,8.5in}]{geometry}
%\usepackage[pdftex]{graphicx}
%\usepackage{tipa}
%\usepackage{txfonts}
%\usepackage{textcomp}
%\usepackage{amsthm}
%\usepackage{xy}
%\usepackage{fancyhdr}
%\usepackage{xcolor}
%\usepackage{mathtools} %% for pretty underbrace % Breaks Ximera
%\usepackage{multicol}



\newcommand{\RR}{\mathbb R}
\newcommand{\R}{\mathbb R}
\newcommand{\C}{\mathbb C}
\newcommand{\N}{\mathbb N}
\newcommand{\Z}{\mathbb Z}
\newcommand{\dis}{\displaystyle}
%\renewcommand{\d}{\,d\!}
\renewcommand{\d}{\mathop{}\!d}
\newcommand{\dd}[2][]{\frac{\d #1}{\d #2}}
\newcommand{\pp}[2][]{\frac{\partial #1}{\partial #2}}
\renewcommand{\l}{\ell}
\newcommand{\ddx}{\frac{d}{\d x}}
\newcommand{\ppx}{\frac{\partial}{\partial x}}
\newcommand{\ppy}{\frac{\partial}{\partial y}}

\newcommand{\zeroOverZero}{\ensuremath{\boldsymbol{\tfrac{0}{0}}}}
\newcommand{\inftyOverInfty}{\ensuremath{\boldsymbol{\tfrac{\infty}{\infty}}}}
\newcommand{\zeroOverInfty}{\ensuremath{\boldsymbol{\tfrac{0}{\infty}}}}
\newcommand{\zeroTimesInfty}{\ensuremath{\small\boldsymbol{0\cdot \infty}}}
\newcommand{\inftyMinusInfty}{\ensuremath{\small\boldsymbol{\infty - \infty}}}
\newcommand{\oneToInfty}{\ensuremath{\boldsymbol{1^\infty}}}
\newcommand{\zeroToZero}{\ensuremath{\boldsymbol{0^0}}}
\newcommand{\inftyToZero}{\ensuremath{\boldsymbol{\infty^0}}}


\newcommand{\numOverZero}{\ensuremath{\boldsymbol{\tfrac{\#}{0}}}}
\newcommand{\dfn}{\textbf}
%\newcommand{\unit}{\,\mathrm}
\newcommand{\unit}{\mathop{}\!\mathrm}
%\newcommand{\eval}[1]{\bigg[ #1 \bigg]}
\newcommand{\eval}[1]{ #1 \bigg|}
\newcommand{\seq}[1]{\left( #1 \right)}
\renewcommand{\epsilon}{\varepsilon}
\renewcommand{\iff}{\Leftrightarrow}

\DeclareMathOperator{\arccot}{arccot}
\DeclareMathOperator{\arcsec}{arcsec}
\DeclareMathOperator{\arccsc}{arccsc}
\DeclareMathOperator{\si}{Si}
\DeclareMathOperator{\proj}{proj}
\DeclareMathOperator{\scal}{scal}
\DeclareMathOperator{\cis}{cis}
\DeclareMathOperator{\Arg}{Arg}
%\DeclareMathOperator{\arg}{arg}
\DeclareMathOperator{\Rep}{Re}
\DeclareMathOperator{\Imp}{Im}
\DeclareMathOperator{\sech}{sech}
\DeclareMathOperator{\csch}{csch}
\DeclareMathOperator{\Log}{Log}

\newcommand{\tightoverset}[2]{% for arrow vec
  \mathop{#2}\limits^{\vbox to -.5ex{\kern-0.75ex\hbox{$#1$}\vss}}}
\newcommand{\arrowvec}{\overrightarrow}
\renewcommand{\vec}{\mathbf}
\newcommand{\veci}{{\boldsymbol{\hat{\imath}}}}
\newcommand{\vecj}{{\boldsymbol{\hat{\jmath}}}}
\newcommand{\veck}{{\boldsymbol{\hat{k}}}}
\newcommand{\vecl}{\boldsymbol{\l}}
\newcommand{\utan}{\vec{\hat{t}}}
\newcommand{\unormal}{\vec{\hat{n}}}
\newcommand{\ubinormal}{\vec{\hat{b}}}

\newcommand{\dotp}{\bullet}
\newcommand{\cross}{\boldsymbol\times}
\newcommand{\grad}{\boldsymbol\nabla}
\newcommand{\divergence}{\grad\dotp}
\newcommand{\curl}{\grad\cross}
%% Simple horiz vectors
\renewcommand{\vector}[1]{\left\langle #1\right\rangle}


\pgfplotsset{compat=1.13}

\outcome{Define and enumerate permutations}

\title{1.4 Permutations}

\begin{document}

\begin{abstract}
We define and enumerate permutations.
\end{abstract}

\maketitle

\section{Permutations}



\begin{definition}[Permutation]
A permutation of selection of objects in a particular order.
\end{definition}

Permutations are counted using the fundamental principle of counting.

\begin{example}[example 1]
A fifth grade class consisting of 24 students wishes to elect a class president, vice-president, secretary and treasurer.
Assuming that no student can hold more than one office, in how many ways can these
positions be filled?\\
The selection of class officers can be thought of as a permutation
the order of the selection of the 4 students determines which office they will hold. 
Thus, we seek the number of permutations of 24 objects taken 4 at a time, denoted $P(24, 4)$. By the FPC, we have
\[
P(24, 4) = 24 \times 23 \times 22 \times 21 = 255,024
\]
Thus, there are just over a quarter of a million possible outcomes for the selection of the class officers.
\end{example}

\begin{problem}(problem 1)
A sports team consisting of 12 players wishes to select a captain, 
an inspirational leader and an equipment manager (from among themselves).  
In how many ways can this be done?\\
\begin{multipleChoice}
\choice[correct]{P(12, 3)}
\choice{P(15, 3)}
\choice{P(9, 6)}
\end{multipleChoice}

\end{problem}

\begin{proposition}
The number of permutations of $n$ objects taken $k$ at a time is
\[
P(n,k) = n \cdot (n-1) \cdot (n-2) \cdot \ldots \cdot (n-k+1)
\]

\end{proposition}
\begin{proof}
The number of choices for the first object selected is $n$. The number of choices for the second object selected is $n-1$. And so on, until the $k$ object, for which there are $n-k+1$ choices remaining. The result then follows directly from the FPC. 
\end{proof}
\begin{remark}
Note that the list $n, n-1, n-1, ..., n-k+1$ consists of $k$ numbers, corresponding to the $k$ objects being selected.
\end{remark}

\begin{example}[example 2]
Compute $P(7, 3)$.\\
Using the proposition with $n = 7$ and $k = 3$, so that, $n-k+1 = 7-3+1 = 5$, we have
\[
P(7,3) = 7 \cdot 6 \cdot 5 = 210
\]
\end{example}
\begin{problem}(problem 2)
Compute each of the following:
$P(5,2) = \; \answer{20}$\\
$P(8,3) = \; \answer{336}$\\
$P(10,5) = \; \answer{30240}$
\end{problem}

\section{Factorials}

\begin{definition}[Factorial]
Let $n$ be a natural number. Then n! (read n factorial) is defined as
\[
n! = n \cdot (n-1) \cdot \ldots \cdot 3 \cdot 2 \cdot 1
\]
\end{definition}

\begin{example}[example 3]
Compute the following factorials: $2!\,, 4!\,, 6!$ and $8!$\\
We have
\[
2! = 2\cdot 1 = 2,
\]
\[
4! = 4 \cdot 3 \cdot 2\cdot 1 = 24,
\]
\[
6! = 6\cdot 5\cdot 4 \cdot 3 \cdot 2\cdot 1 = 720
\]
and
\[
8! = 8 \cdot 7 \cdot 6\cdot 5\cdot 4 \cdot 3 \cdot 2\cdot 1 = 40320
\]
\end{example}

\begin{problem}(problem 3)
Compute the following factorials: \\
$3! = \; \answer{6}$\\
$5! = \; \answer{120}$\\
$7! = \; \answer{5040}$\\
$9! = \; \answer{362880}$\\
\end{problem}

We now explore the connection between factorials and permutations.

\begin{example}[example 4]
In how many ways can 5 people line up at an ATM?\\
The answer is the number of permutations of 5 objects taken 5 at a time, i.e., 
\[
P(5,5) = 5 \cdot 4 \cdot 3 \cdot 2 \cdot 1 = 5! = 120
\]
\end{example}

\begin{problem}(problem 4)
In how many ways can 6 people line up at an ATM (answer using factorials)? $\; \answer{6!}$
\end{problem}

General permutations can be expressed as a ratio of factorials.

\begin{proposition}
Let $n$ and $k$ be natural numbers with $n > k$. Then 
\[
P(n,k) = \frac{n!}{(n-k)!}
\]
\end{proposition}
\begin{proof}
The ratio of factorials lends itself to lots of cancellation. We have
\[
\frac{n!}{(n-k)!} = \frac{n \cdot (n-1) \cdot \ldots \cdot 2 \cdot 1}{(n-k) \cdot (n - k-1) \cdot \ldots \cdot  2 \cdot 1}
= n \cdot (n-1) \cdot \ldots \cdot (n-k+1) = P(n,k).
\]
Note that the entire denominator cancelled out.
\end{proof}

\begin{example}[example 5]
Express the following as a ratio of factorials: $P(7, 3), P(10, 4)$ and $P(100, 25)$.\\
For $P(7,3)$ we have $n = 7, k = 3$ and $n-k = 7-3 = 4$, so
\[
P(7, 3) = \frac{7!}{4!}
\]
Similarly,
\[
P(10,4) = \frac{10!}{6!} \;\; \mbox{and} \;\; P(100, 25) = \frac{100!}{75!}
\]
In applications, numbers such as $75!$ and $100!$ are too unwieldy to write out, so we will prefer the factorial notation
as a final form.
\end{example}

\begin{problem}(problem 5)
Express the following as a ratio of factorials:\\
$P(8,5) = \; \answer{\frac{8!}{3!}}$\\
$P(12,6) = \; \answer{\frac{12!}{6!}}$\\
$P(50,10) = \; \answer{\frac{50!}{40!}}$
\end{problem}

In the proposition, it was assumed that $n>k$.  But, what if $n = k$, for we have already seen 
the expression $P(n,n)$ arise in applications. 
In fact, $P(n,n) = n!$. To extend the proposition to cover this case we make the following definition.

\begin{definition}[0!]
We define 
\[
0! \equiv 1
\]
\end{definition}

With the above definition, we can safely revise the proposition to state that for $n$ and $k$ natural numbers with $n \geq k$,
we have
\[
P(n,k) = \frac{n!}{(n-k)!}
\]
for if $n = k$ then the denominator is $0! = 1$.


\section{Permutations of Non-distinct Objects}

In this subsection, we examine permutations of $n$ objects taken $n$ at 
a time in which the $n$ objects are not distinct.  The idea is that arranging non-distinct objects 
among themselves does not produce a new permutation.  To count such permutations, we will combine the FPC with division.

\begin{example}[example 6]
How many permutations are there of the word $AABC$?\\
We have 2 non-distinct objects among our 4 objects- namely the two $A$'s.  We begin the counting process by 
considering the 4 objects as distinct by placing subscripts on the repeated objects.  
In other words, we consider the word $A_1A_2BC$, which has $P(4,4) = 4! = 24$ permutations. Among these 24 permutations, consider the following pairs of permutations:
\[
A_1BA_2C \; \mbox{and} \; A_2BA_1C
\]

\[
BA_1CA_2 \; \mbox{and} \; BA_2CA_1
\]

\[
A_1CBA_2 \; \mbox{and} \; A_2CBA_1
\]
Each of these three pairs of permutations of the 4 distinct objects, correspond to just a single permutation of the 
non-distinct objects $AABC$.  In fact, the correspond to 
\[
ABAC, BACA \;\mbox{and} \;  ACBA
\]
respectively. In fact, the 24 permutations of $A_1A_2BC$ can be paired in the manner shown above, 
with each pair yielding a single permutation of $AABC$. 
This means that the number of permutations of $A_1A_2BC$ is twice the number of permutations of $AABC$.  
Thus, the number of permutations of $AABC$ is
$24/2 = 12$.
\end{example}

\begin{problem}(problem 6)
How many permutations are there of the letter in the word following words?
a) $\;BOO: \; \answer{3}$\\
b) $\;BOOK: \; \answer{12}$\\
c) $\;BOOKS: \; \answer{60}$
\end{problem}

\begin{example}[example 7]
How many permutations are there of the letters in the word $AAABC$?\\
Again, we begin by considering the distinct letters $A_1A_2A_3BC$, which have $P(5,5) = 5! = 120$ permutations.
Given any of these 120 permutations, there are $3! = 6$ ways to rearrange the letters $A_1, A_2$ and $A_3$ among themselves, leaving the B and C in the same position.
These 6 permutations of $A_1A_2A_3BC$ all correspond to the same permutation of $AAaBC$.  hence the number of permutations of $A_1A_2A_3$ is 6 times the number of permutations of $AAABC$.  Thus the number of permutations of $AAABC$ can be obtained by dividing 120 by 6, so there are 20 permutation of the letters in the word $AAABC$.
\end{example}

\begin{problem}(problem 7)
How many permutations are there of the letter in the word following words?
a) $\; LULL: \; \answer{4}$\\
b) $\; RARER: \; \answer{20}$\\
c) $\;ASSESS: \; \answer{30}$
\end{problem} 

\begin{example}[example 8]
How many permutations are there of the letters in the word $AAABBC$?\\
If the letters were all distinct, as in $A_1A_2A_3B_1B_2C$, then there would be $6! =720$ permutations.
However, the $A_k's$ in any such permutation can be permuted among themselves in $3! =6$ ways 
without changing the permutation
of the unsubscripted original letters.  Furthermore, the $B_k$'s can also be permuted among themselves in $2!= 2$ ways without changing the permutation of the original unsubscripted letters.  Thus the total number of permutations of $AAABBC$ can be found by dividing $6!$ first by $3!$ and then dividing that by $2!$. This is equivalent to dividing $6!$ by the product of $3!$ and $2!$.  Thus the number of permutations of the letters in the word $AAABBC$ is
\[
\frac{6!}{3!2!} = \frac{720}{12} = 60
\]
\end{example}

\begin{problem}(problem 8)
How many permutations are there of the letter in the word following words?
a) $\; HAWAII: \; \answer{180}$\\
b) $\; CALIFORNIA: \; \answer{907200}$\\
c) $\; CONNECTICUT: \; \answer{1663200}$\\
d) $\;TENNESSEE: \; \answer{3780}$\\
e) $\;MISSISSIPPI: \; \answer{34650}$
f) $\;MASSACHUSETTS: \; \answer{64864800}$\\
\end{problem}

\end{document}










