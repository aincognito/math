\documentclass[handout]{ximera}

%% You can put user macros here
%% However, you cannot make new environments



\newcommand{\ffrac}[2]{\frac{\text{\footnotesize $#1$}}{\text{\footnotesize $#2$}}}
\newcommand{\vasymptote}[2][]{
    \draw [densely dashed,#1] ({rel axis cs:0,0} -| {axis cs:#2,0}) -- ({rel axis cs:0,1} -| {axis cs:#2,0});
}


%\usepackage{tcolorbox} %%Needed for Derivative Definition supposedly and product rule, natural exp log, quotient rule, inverse trig, rates of change


% \graphicspath{{./}{firstExample/}}
% \usepackage{forest}
\usepackage{amsmath}
\usepackage{amssymb}
\usepackage{array}
\usepackage[makeroom]{cancel} %% for strike outs
\usepackage{pgffor} %% required for integral for loops
\usepackage{tikz}
\usepackage{tikz-cd}
\usepackage{tkz-euclide}
\usetikzlibrary{shapes.multipart}


% \usetkzobj{all}
\tikzstyle geometryDiagrams=[ultra thick,color=blue!50!black]


\usetikzlibrary{arrows}
\tikzset{>=stealth,commutative diagrams/.cd,
  arrow style=tikz,diagrams={>=stealth}} %% cool arrow head
\tikzset{shorten <>/.style={ shorten >=#1, shorten <=#1 } } %% allows shorter vectors

\usetikzlibrary{backgrounds} %% for boxes around graphs
\usetikzlibrary{shapes,positioning}  %% Clouds and stars
\usetikzlibrary{matrix} %% for matrix
\usepgfplotslibrary{polar} %% for polar plots
\usepgfplotslibrary{fillbetween} %% to shade area between curves in TikZ



%\usepackage[width=4.375in, height=7.0in, top=1.0in, papersize={5.5in,8.5in}]{geometry}
%\usepackage[pdftex]{graphicx}
%\usepackage{tipa}
%\usepackage{txfonts}
%\usepackage{textcomp}
%\usepackage{amsthm}
%\usepackage{xy}
%\usepackage{fancyhdr}
%\usepackage{xcolor}
%\usepackage{mathtools} %% for pretty underbrace % Breaks Ximera
%\usepackage{multicol}



\newcommand{\RR}{\mathbb R}
\newcommand{\R}{\mathbb R}
\newcommand{\C}{\mathbb C}
\newcommand{\N}{\mathbb N}
\newcommand{\Z}{\mathbb Z}
\newcommand{\dis}{\displaystyle}
%\renewcommand{\d}{\,d\!}
\renewcommand{\d}{\mathop{}\!d}
\newcommand{\dd}[2][]{\frac{\d #1}{\d #2}}
\newcommand{\pp}[2][]{\frac{\partial #1}{\partial #2}}
\renewcommand{\l}{\ell}
\newcommand{\ddx}{\frac{d}{\d x}}
\newcommand{\ppx}{\frac{\partial}{\partial x}}
\newcommand{\ppy}{\frac{\partial}{\partial y}}

\newcommand{\zeroOverZero}{\ensuremath{\boldsymbol{\tfrac{0}{0}}}}
\newcommand{\inftyOverInfty}{\ensuremath{\boldsymbol{\tfrac{\infty}{\infty}}}}
\newcommand{\zeroOverInfty}{\ensuremath{\boldsymbol{\tfrac{0}{\infty}}}}
\newcommand{\zeroTimesInfty}{\ensuremath{\small\boldsymbol{0\cdot \infty}}}
\newcommand{\inftyMinusInfty}{\ensuremath{\small\boldsymbol{\infty - \infty}}}
\newcommand{\oneToInfty}{\ensuremath{\boldsymbol{1^\infty}}}
\newcommand{\zeroToZero}{\ensuremath{\boldsymbol{0^0}}}
\newcommand{\inftyToZero}{\ensuremath{\boldsymbol{\infty^0}}}


\newcommand{\numOverZero}{\ensuremath{\boldsymbol{\tfrac{\#}{0}}}}
\newcommand{\dfn}{\textbf}
%\newcommand{\unit}{\,\mathrm}
\newcommand{\unit}{\mathop{}\!\mathrm}
%\newcommand{\eval}[1]{\bigg[ #1 \bigg]}
\newcommand{\eval}[1]{ #1 \bigg|}
\newcommand{\seq}[1]{\left( #1 \right)}
\renewcommand{\epsilon}{\varepsilon}
\renewcommand{\iff}{\Leftrightarrow}

\DeclareMathOperator{\arccot}{arccot}
\DeclareMathOperator{\arcsec}{arcsec}
\DeclareMathOperator{\arccsc}{arccsc}
\DeclareMathOperator{\si}{Si}
\DeclareMathOperator{\proj}{proj}
\DeclareMathOperator{\scal}{scal}
\DeclareMathOperator{\cis}{cis}
\DeclareMathOperator{\Arg}{Arg}
%\DeclareMathOperator{\arg}{arg}
\DeclareMathOperator{\Rep}{Re}
\DeclareMathOperator{\Imp}{Im}
\DeclareMathOperator{\sech}{sech}
\DeclareMathOperator{\csch}{csch}
\DeclareMathOperator{\Log}{Log}

\newcommand{\tightoverset}[2]{% for arrow vec
  \mathop{#2}\limits^{\vbox to -.5ex{\kern-0.75ex\hbox{$#1$}\vss}}}
\newcommand{\arrowvec}{\overrightarrow}
\renewcommand{\vec}{\mathbf}
\newcommand{\veci}{{\boldsymbol{\hat{\imath}}}}
\newcommand{\vecj}{{\boldsymbol{\hat{\jmath}}}}
\newcommand{\veck}{{\boldsymbol{\hat{k}}}}
\newcommand{\vecl}{\boldsymbol{\l}}
\newcommand{\utan}{\vec{\hat{t}}}
\newcommand{\unormal}{\vec{\hat{n}}}
\newcommand{\ubinormal}{\vec{\hat{b}}}

\newcommand{\dotp}{\bullet}
\newcommand{\cross}{\boldsymbol\times}
\newcommand{\grad}{\boldsymbol\nabla}
\newcommand{\divergence}{\grad\dotp}
\newcommand{\curl}{\grad\cross}
%% Simple horiz vectors
\renewcommand{\vector}[1]{\left\langle #1\right\rangle}


\pgfplotsset{compat=1.13}

\outcome{Learn the basics of generating functions}

\title{3.2 Generating Functions}

\begin{document}

\begin{abstract}
We will define and interpret generating functions.
\end{abstract}

\maketitle

\section{Generating Functions}

\begin{definition}
A \textbf{generating function} is a function of the form
\[
g(x) = \sum_{k=0}^\infty c_kx^k = c_0 + c_1 x + c_2 x^2 + \cdots
\]
whose coefficients $c_0, c_1, c_2, ...$ encode information related to a counting problem.
\end{definition}


\begin{example}[example 1]
The function
\[
g_n(x) = \sum_{k=0}^n \binom{n}{k}x^k  = \binom{n}{0}+ \binom{n}{1}x + \binom{n}{2}x^2 + \cdots + 
\binom{n}{k}x^k + \cdots + \binom{n}{n} x^n
\]
The coefficients $c_{n+1}, c_{n+2}, ...$ are all zero.
The non-zero coefficients are binomial coefficients- the coefficient of $x^k$ is the number of subsets of an $n$-element set.
\end{example}

\begin{problem}
Consider the generating function
\[
g_n(x) = \sum_{k=0}^n \frac{n!}{(n-k)!}x^k  = \frac{n!}{n!}+ \frac{n!}{(n-1)!}x + \frac{n!}{(n-2)!}x^2 + \cdots + 
\frac{n!}{(n-k)!}x^k + \cdots + \frac{n!}{(n-n)!} x^n
\]
Give an example of a counting problem whose solution is given by the coefficients of $g_n(x)$.
\end{problem}

\begin{example}[example 2]
Flip a coin and consider the function $g(x) = 1+x$.  The coefficient of $x^k, k = 0,1$ represents the number ways of 
obtaining $k$ ``heads". Now flip the coin twice and consider the function $g_2(x) = (1+x)^2 = 1 + 2x + x^2$.
Note that there is one way to get zero heads: TT. There are two ways to get one heads, HT and TH, 
and there is one way to get two heads, HH. Thus the function $g_2(x)$ encodes then number of ways to 
get $k$ heads in two flips of the coin, 
where $k = 0, 1, 2$.
\end{example}

\begin{problem}(problem 2a)
Based on the example above guess the generating function corresponding to counting the number of heads in 3 coin flips.
Express your guess both as a power of $1+x$ as a proper generating function with numerical coefficients.
Verify that your guess is correct. 
\end{problem}

\begin{problem}(problem 2b)
Generalize your answer to the last problem by creating the generating function corresponding to the 
number of heads in $n$ coin flips.
Express your answer as a power of $1+x$ and verify your answer using the Binomial Theorem.
\end{problem}
 

\begin{example}[example 3]
Consider a six-sided die whose faces are numbered 1 through 6. If it is rolled once, the function
\[
g_1(x) = x + x^2 + x^3 + x^4 + x^5 + x^6
\]
encodes the number of ways of obtaining the outcome corresponding to the degree of a particular term.
Now roll the die twice and consider the sum. It can be $2, 3, 4, ..., 11, 12$. Squaring $g_1(x)$ gives
\[
g_2(x) = x^2 + 2x^3 + 3x^4 + 4x^5 + 5x^6 + 6x^7 + 5x^8 + 4x^9+ 3x^{10} + 2x^{11} + x^{12} \;\; \text{verify}
\]
The coefficient of $x^k$ in $g_2(x)$ gives the number of ways of rolling a sum of $k$ (verify).
\end{example}

\begin{problem}(problem 3)
Consider a six-sided die whose faces are numbered 1 through 6. Roll the die 3 times and consider the sum.
What are the minimum and maximum possible values of the sum? Create a generating function $g_3(x)$ whose coefficients 
encode the the number of ways of rolling a sum of $k$.  Express your answer as a power of $g_1(x)$ from the example above.
Use a computer algebra system to find the coefficient of $x^7$. Verify that this coefficient is 
indeed the number of ways of obtaining a sum of $7$ by enumerating the possibilities.
\end{problem}


\begin{example}[example 4]
Consider the number of non-negative integer solutions of the equation
\[
x_1 + x_2 + \cdots + x_k = n
\]
Find the generating function $g_k(x)$ whose coefficients give the number of such solutions.\\
Each variable $x_i$ can take on any value (up to the maximum possible $n$, which is unspecified) and so it will contribute 
a factor of 
\[
1 + x + x^2 +x^3 + \cdots = \frac{1}{1-x}
\]
Hence
\[
g_k(x) = (1+x+x^2 + \cdots)^k = \frac{1}{(1-x)^k}
\]
\end{example}

\end{document}







