\documentclass[handout]{ximera}

%% You can put user macros here
%% However, you cannot make new environments



\newcommand{\ffrac}[2]{\frac{\text{\footnotesize $#1$}}{\text{\footnotesize $#2$}}}
\newcommand{\vasymptote}[2][]{
    \draw [densely dashed,#1] ({rel axis cs:0,0} -| {axis cs:#2,0}) -- ({rel axis cs:0,1} -| {axis cs:#2,0});
}


\graphicspath{{./}{firstExample/}}
\usepackage{forest}
\usepackage{amsmath}
\usepackage{amssymb}
\usepackage{array}
\usepackage[makeroom]{cancel} %% for strike outs
\usepackage{pgffor} %% required for integral for loops
\usepackage{tikz}
\usepackage{tikz-cd}
\usepackage{tkz-euclide}
\usetikzlibrary{shapes.multipart}


%\usetkzobj{all}
\tikzstyle geometryDiagrams=[ultra thick,color=blue!50!black]


\usetikzlibrary{arrows}
\tikzset{>=stealth,commutative diagrams/.cd,
  arrow style=tikz,diagrams={>=stealth}} %% cool arrow head
\tikzset{shorten <>/.style={ shorten >=#1, shorten <=#1 } } %% allows shorter vectors

\usetikzlibrary{backgrounds} %% for boxes around graphs
\usetikzlibrary{shapes,positioning}  %% Clouds and stars
\usetikzlibrary{matrix} %% for matrix
\usepgfplotslibrary{polar} %% for polar plots
\usepgfplotslibrary{fillbetween} %% to shade area between curves in TikZ



%\usepackage[width=4.375in, height=7.0in, top=1.0in, papersize={5.5in,8.5in}]{geometry}
%\usepackage[pdftex]{graphicx}
%\usepackage{tipa}
%\usepackage{txfonts}
%\usepackage{textcomp}
%\usepackage{amsthm}
%\usepackage{xy}
%\usepackage{fancyhdr}
%\usepackage{xcolor}
%\usepackage{mathtools} %% for pretty underbrace % Breaks Ximera
%\usepackage{multicol}



\newcommand{\RR}{\mathbb R}
\newcommand{\R}{\mathbb R}
\newcommand{\C}{\mathbb C}
\newcommand{\N}{\mathbb N}
\newcommand{\Z}{\mathbb Z}
\newcommand{\dis}{\displaystyle}
%\renewcommand{\d}{\,d\!}
\renewcommand{\d}{\mathop{}\!d}
\newcommand{\dd}[2][]{\frac{\d #1}{\d #2}}
\newcommand{\pp}[2][]{\frac{\partial #1}{\partial #2}}
\renewcommand{\l}{\ell}
\newcommand{\ddx}{\frac{d}{\d x}}

\newcommand{\zeroOverZero}{\ensuremath{\boldsymbol{\tfrac{0}{0}}}}
\newcommand{\inftyOverInfty}{\ensuremath{\boldsymbol{\tfrac{\infty}{\infty}}}}
\newcommand{\zeroOverInfty}{\ensuremath{\boldsymbol{\tfrac{0}{\infty}}}}
\newcommand{\zeroTimesInfty}{\ensuremath{\small\boldsymbol{0\cdot \infty}}}
\newcommand{\inftyMinusInfty}{\ensuremath{\small\boldsymbol{\infty - \infty}}}
\newcommand{\oneToInfty}{\ensuremath{\boldsymbol{1^\infty}}}
\newcommand{\zeroToZero}{\ensuremath{\boldsymbol{0^0}}}
\newcommand{\inftyToZero}{\ensuremath{\boldsymbol{\infty^0}}}


\newcommand{\numOverZero}{\ensuremath{\boldsymbol{\tfrac{\#}{0}}}}
\newcommand{\dfn}{\textbf}
%\newcommand{\unit}{\,\mathrm}
\newcommand{\unit}{\mathop{}\!\mathrm}
%\newcommand{\eval}[1]{\bigg[ #1 \bigg]}
\newcommand{\eval}[1]{ #1 \bigg|}
\newcommand{\seq}[1]{\left( #1 \right)}
\renewcommand{\epsilon}{\varepsilon}
\renewcommand{\iff}{\Leftrightarrow}

\DeclareMathOperator{\arccot}{arccot}
\DeclareMathOperator{\arcsec}{arcsec}
\DeclareMathOperator{\arccsc}{arccsc}
\DeclareMathOperator{\si}{Si}
\DeclareMathOperator{\proj}{proj}
\DeclareMathOperator{\scal}{scal}
\DeclareMathOperator{\cis}{cis}
\DeclareMathOperator{\Arg}{Arg}
%\DeclareMathOperator{\arg}{arg}
\DeclareMathOperator{\Rep}{Re}
\DeclareMathOperator{\Imp}{Im}
\DeclareMathOperator{\sech}{sech}
\DeclareMathOperator{\csch}{csch}
\DeclareMathOperator{\Log}{Log}

\newcommand{\tightoverset}[2]{% for arrow vec
  \mathop{#2}\limits^{\vbox to -.5ex{\kern-0.75ex\hbox{$#1$}\vss}}}
\newcommand{\arrowvec}{\overrightarrow}
\renewcommand{\vec}{\mathbf}
\newcommand{\veci}{{\boldsymbol{\hat{\imath}}}}
\newcommand{\vecj}{{\boldsymbol{\hat{\jmath}}}}
\newcommand{\veck}{{\boldsymbol{\hat{k}}}}
\newcommand{\vecl}{\boldsymbol{\l}}
\newcommand{\utan}{\vec{\hat{t}}}
\newcommand{\unormal}{\vec{\hat{n}}}
\newcommand{\ubinormal}{\vec{\hat{b}}}

\newcommand{\dotp}{\bullet}
\newcommand{\cross}{\boldsymbol\times}
\newcommand{\grad}{\boldsymbol\nabla}
\newcommand{\divergence}{\grad\dotp}
\newcommand{\curl}{\grad\cross}
%% Simple horiz vectors
\renewcommand{\vector}[1]{\left\langle #1\right\rangle}


\pgfplotsset{compat=1.13}

\outcome{Create and interpret generating functions}

\title{3.2 Generating Functions}

\begin{document}

\begin{abstract}
We will define, create and interpret generating functions.
\end{abstract}

\maketitle

\section{Generating Functions}

An infinite sequence $a_0, a_1, a_2, ... $ can be associated with an infinite series
\[
a_0 + a_1 x + a_2 x^2 + \cdots
\]
A finite sequence can associated with a polynomial in a similar manner.
This infinite series (or polynomial) is called the generating function of the sequence.
\begin{definition}
Given a sequence $a_0, a_1, a_2, ...$, the function 
\[
g(x) = \sum_{k=0}^\infty a_kx^k = a_0 + a_1 x + a_2 x^2 + \cdots
\]
is called the \textbf{generating function} of the sequence.
If the sequence is finite: $a_0, a_1, ..., a_n$, the its generating function is the polynomial
\[
g(x) = \sum_{k=0}^n a_kx^k = a_0 + a_1 x + a_2 x^2 + \cdots + a_n x^n
\]
\end{definition}

\begin{remark}
In this section, we prefer to start our sequence with the term denoted $a_0$ rather than $a_1$ so as to 
align the subscript in the generating function with the power of $x$.
\end{remark}

Before presenting examples of generating functions, it is important for us to recall two specific examples of power series.
The first is the geometric power series
\[
\frac{1}{1-x} = \sum_{k=0}^\infty x^k = 1 + x + x^2 + \cdots
\]
and the second is the Maclaurin series for the exponential function
\[
e^x =  \sum_{k=0}^\infty \frac{x^k}{k!} = 1 + x + \frac{x^2}{2!} + \cdots
\]
In the context of generating functions, we are not interested in the interval of convergence of these series, 
but just the relationship between
the series and the corresponding sequence of coefficients.

\begin{example}[example 1]
Find the generating function for the sequence 
\[
1, 0, 1, 0, 1, \cdots
\]
and write it in a compact form.\\
The generating function for this sequence is 
\[
g(x) = 1 + 0x + 1x^2 + 0x^3 + 1x^4 + \cdots = 1 + x^2 + x^4 + \cdots
\]
This is a geometric series with common ratio $x^2$, so it can be written more compactly as
\[
g(x) = \frac{1}{1 - x^2}
\]
\end{example}

\begin{problem}(problem 1)
Find the compact form of the generating function for each sequence:\\
a) $1, -1, 1, -1, ...$
\begin{multipleChoice}
\choice{$\dis g(x) = \frac{1}{1-x}$}\\
\choice[correct]{$\dis g(x) =  \frac{1}{1+x}$}\\
\choice{$\dis g(x) =  \frac{1}{x-1}$}
\end{multipleChoice}
b) $0, 1, 0, 1, ...$
\begin{multipleChoice}
\choice[correct]{$\dis g(x) =  \frac{x}{1-x^2}$}\\
\choice{$\dis g(x) =  \frac{1}{1-x^2}$}\\
\choice{$\dis g(x) =  \frac{x}{1-x}$}
\end{multipleChoice}
c) $1, 2, 4, 8, ...$
\begin{multipleChoice}
\choice{$\dis g(x) =  \frac{2}{1-x}$}\\
\choice{$\dis g(x) =  \frac{1}{2-x}$}\\
\choice[correct]{$\dis g(x) =  \frac{1}{1-2x}$}
\end{multipleChoice}

\end{problem}


\begin{example}[example 2]
Find the generating function for the sequence 
\[
1, -1, \frac12, -\frac16, \frac{1}{24}, \cdots
\]
and write it in a compact form.\\
Recognizing the numbers in the sequence as $\frac{(-1)^n}{n!}$, we have 
\[
g(x) = 1 - x + \frac{x^2}{2!} - \frac{x^3}{3!} + \frac{x^4}{4!} - \cdots = \sum_{n=0}^\infty (-1)^n\frac{x^n}{n!}
\]
This is the exponential function with $-x$ in the place of $x$ (since $(-1)^n x^n = (-x)^n$) and 
hence the generating function can be written more compactly as
\[
g(x) = e^{-x}
\]
\end{example}

\begin{problem}(problem 2)
Find the compact form of the generating function for each sequence:\\
a) $\dis 1, 2, \frac{4}{2!}, \frac{8}{3!}, \frac{16}{4!}, ...$
\begin{multipleChoice}
\choice{$\dis g(x) = 2e^x$}\\
\choice{$\dis g(x) =  2^x$}\\
\choice[correct]{$\dis g(x) =  e^{2x}$}
\end{multipleChoice}
b) $1, 0, -1, 0, \frac{1}{2!}, 0, -\frac{1}{3!}, ...$
\begin{multipleChoice}
\choice[correct]{$\dis g(x) =  e^{-x^2}$}\\
\choice{$\dis g(x) =  x^2e^{-x}$}\\
\choice{$\dis g(x) =  e^x + e^{-x}$}
\end{multipleChoice}
c) $0, 1, 1, \frac{1}{2!}, \frac{1}{3!} ...$
\begin{multipleChoice}
\choice{$\dis g(x) =  x+ e^x$}\\
\choice[correct]{$\dis g(x) =  xe^x$}\\
\choice{$\dis g(x) =  e^x$}
\end{multipleChoice}
d) $2, 0, \frac{2}{2!}, 0 , \frac{2}{4!}, 0,  ...$
\begin{multipleChoice}
\choice[correct]{$\dis g(x) =  e^x + e^{-x}$}\\
\choice{$\dis g(x) =  e^x - e^{-x}$}\\
\choice{$\dis g(x) =  e^{x^2}$}
\end{multipleChoice}


\end{problem}



The next example investigates the relationship between generating functions and solutions to combinatorial problems.

\begin{example}[example 3]
Flip a coin and consider the function $g(x) = 1+x$.  The coefficient of $x^k, k = 0,1$ represents the number ways of 
obtaining $k$ ``heads". Now flip the coin twice and consider the function $g_2(x) = (1+x)^2 = 1 + 2x + x^2$.
Note that there is one way to get zero heads: TT. There are two ways to get one heads, HT and TH, 
and there is one way to get two heads, HH. Thus the function $g_2(x)$ encodes then number of ways to 
get $k$ heads in two flips of the coin, 
where $k = 0, 1, 2$.
\end{example}

\begin{problem}(problem 3a)
Based on the example above guess the generating function corresponding to counting the number of heads in 3 coin flips.
Express your guess both as a power of $1+x$ as a proper generating function with numerical coefficients.
Verify that your guess is correct. 
\end{problem}

\begin{problem}(problem 3b)
Generalize your answer to the last problem by creating the generating function corresponding to the 
number of heads in $n$ coin flips.
Express your answer as a power of $1+x$ and verify your answer using the Binomial Theorem.
\end{problem}
 

\begin{example}[example 4]
Consider a six-sided die whose faces are numbered 1 through 6. If it is rolled once, the function
\[
g_1(x) = x + x^2 + x^3 + x^4 + x^5 + x^6
\]
encodes the number of ways of obtaining the outcome corresponding to the degree of a particular term.
Now roll the die twice and consider the sum. It can be $2, 3, 4, ..., 11, 12$. Squaring $g_1(x)$ gives
\[
g_2(x) = x^2 + 2x^3 + 3x^4 + 4x^5 + 5x^6 + 6x^7 + 5x^8 + 4x^9+ 3x^{10} + 2x^{11} + x^{12} \;\;\; \text{(verify)}
\]
The coefficient of $x^k$ in $g_2(x)$ gives the number of ways of rolling a sum of $k$ (verify).
\end{example}

\begin{problem}(problem 4)
Consider a six-sided die whose faces are numbered 1 through 6. Roll the die 3 times and consider the sum.
What are the minimum and maximum possible values of the sum? Create a generating function $g_3(x)$ whose coefficients 
encode the the number of ways of rolling a sum of $k$.  Express your answer as a power of $g_1(x)$ from the example above.
Find the coefficient of $x^7$. Verify that this coefficient is 
indeed the number of ways of obtaining a sum of $7$ by enumerating the possibilities.
\end{problem}


\begin{example}[example 5]
Consider the number of non-negative integer solutions of the following equation in $3$ variables:
\[
x_1 + x_2 +  x_3 = n
\]
Find the generating function $g_3(x)$ whose coefficients give the number of such solutions.\\
Each variable $x_i$ can take on any value (up to the maximum possible $n$, which is unspecified) and so it will contribute 
a factor of 
\[
1 + x + x^2 +x^3 + \cdots = \frac{1}{1-x}
\]
Hence
\[
g_3(x) = (1+x+x^2 + \cdots)^3 = \frac{1}{(1-x)^3}
\]
\end{example}

\begin{problem}(problem 5a)
Find a compact form for the generating function whose coefficients give the number of non-negative integer solutions to the 
following equation in $4$ variables:
\[
x_1 + x_2 + x_3 + x_4 = n
\]
\end{problem}


\begin{problem}(problem 5b)
Find a compact form for the generating function whose coefficients give the number of non-negative integer solutions to the 
following equation in $k$ variables:
\[
x_1 + x_2 + x_3 + \cdots + x_k = n
\]
\end{problem}

\begin{example}[example 6]
Consider the number of non-negative integer solutions of the equation 
\[
x_1 + x_2 +  + x_3  + x_4 = n
\]
if the variables are subject to the following conditions:
\[
x_1 > 2, x_2 < 5, x_3 \; \text{ is even} \;\; \text{and} \;\; x_4 \; \text{ is a multiple of 3}
\]
Each variable will contribute a factor according the the condition it must satisfy.
The variable $x_1$ will contribute the factor 
\[
x^3 + x^4 + x^5 + \cdots = x^3(1 + x + x^2 + \cdots) = \frac{x^3}{1-x}
\]
The variable $x_2$ will contribute the factor 
\[
1+x+x^2 + x^3 + x^4 = \frac{1-x^5}{1-x}
\]
The variable $x_3$ will contribute the factor 
\[
1 +  x^2 + x^4 +  \cdots = \frac{1}{1-x^2}
\]
and the variable $x_4$ will contribute the factor 
\[
1 + x^3 + x^6 + x^9 + \cdots  = \frac{1}{1-x^3}
\]
The generating function is the product of these factors:
\[
g(x) = \frac{x^3}{1-x} \cdot \frac{1-x^5}{1-x}  \cdot \frac{1}{1-x^2} \cdot  \frac{1}{1-x^3} = \frac{x^3(1-x^5)}{(1-x)^4(1+x)(1+x+x^2)}
\]
The coefficient of $x^n$ in the Maclaurin series expansion of this function gives the number of solutions to 
the given equation whose variables satisfy the given conditions.

\end{example}


\begin{problem}(problem 6)
Consider the number of non-negative integer solutions of the equation 
\[
x_1 + x_2 +  x_3  + x_4 = n
\]
if the variables are subject to the following conditions:
\[
x_1 > 3, x_2 < 4, x_3 \; \text{ is even} \;\; \text{and} \;\; x_4 \; \text{ is a multiple of 4}
\]
%x^4 / 1-x  1-x^4/ 1-x  1/1-x^2 1 / 1-x^4
\begin{multipleChoice}
\choice{$\dis \frac{x^3(1-x^3)}{(1-x)^4(1+x)}$}\\
\choice{$\dis \frac{x^4(1-x^4)}{(1-x)^4(1+x)}$}\\
\choice[correct]{$\dis \frac{x^4}{(1-x)^3(1+x)}$}
\end{multipleChoice}

\end{problem}

\end{document}

