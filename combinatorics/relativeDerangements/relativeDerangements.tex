\documentclass[handout]{ximera}

%% You can put user macros here
%% However, you cannot make new environments



\newcommand{\ffrac}[2]{\frac{\text{\footnotesize $#1$}}{\text{\footnotesize $#2$}}}
\newcommand{\vasymptote}[2][]{
    \draw [densely dashed,#1] ({rel axis cs:0,0} -| {axis cs:#2,0}) -- ({rel axis cs:0,1} -| {axis cs:#2,0});
}


\graphicspath{{./}{firstExample/}}
\usepackage{forest}
\usepackage{amsmath}
\usepackage{amssymb}
\usepackage{array}
\usepackage[makeroom]{cancel} %% for strike outs
\usepackage{pgffor} %% required for integral for loops
\usepackage{tikz}
\usepackage{tikz-cd}
\usepackage{tkz-euclide}
\usetikzlibrary{shapes.multipart}


%\usetkzobj{all}
\tikzstyle geometryDiagrams=[ultra thick,color=blue!50!black]


\usetikzlibrary{arrows}
\tikzset{>=stealth,commutative diagrams/.cd,
  arrow style=tikz,diagrams={>=stealth}} %% cool arrow head
\tikzset{shorten <>/.style={ shorten >=#1, shorten <=#1 } } %% allows shorter vectors

\usetikzlibrary{backgrounds} %% for boxes around graphs
\usetikzlibrary{shapes,positioning}  %% Clouds and stars
\usetikzlibrary{matrix} %% for matrix
\usepgfplotslibrary{polar} %% for polar plots
\usepgfplotslibrary{fillbetween} %% to shade area between curves in TikZ



%\usepackage[width=4.375in, height=7.0in, top=1.0in, papersize={5.5in,8.5in}]{geometry}
%\usepackage[pdftex]{graphicx}
%\usepackage{tipa}
%\usepackage{txfonts}
%\usepackage{textcomp}
%\usepackage{amsthm}
%\usepackage{xy}
%\usepackage{fancyhdr}
%\usepackage{xcolor}
%\usepackage{mathtools} %% for pretty underbrace % Breaks Ximera
%\usepackage{multicol}



\newcommand{\RR}{\mathbb R}
\newcommand{\R}{\mathbb R}
\newcommand{\C}{\mathbb C}
\newcommand{\N}{\mathbb N}
\newcommand{\Z}{\mathbb Z}
\newcommand{\dis}{\displaystyle}
%\renewcommand{\d}{\,d\!}
\renewcommand{\d}{\mathop{}\!d}
\newcommand{\dd}[2][]{\frac{\d #1}{\d #2}}
\newcommand{\pp}[2][]{\frac{\partial #1}{\partial #2}}
\renewcommand{\l}{\ell}
\newcommand{\ddx}{\frac{d}{\d x}}

\newcommand{\zeroOverZero}{\ensuremath{\boldsymbol{\tfrac{0}{0}}}}
\newcommand{\inftyOverInfty}{\ensuremath{\boldsymbol{\tfrac{\infty}{\infty}}}}
\newcommand{\zeroOverInfty}{\ensuremath{\boldsymbol{\tfrac{0}{\infty}}}}
\newcommand{\zeroTimesInfty}{\ensuremath{\small\boldsymbol{0\cdot \infty}}}
\newcommand{\inftyMinusInfty}{\ensuremath{\small\boldsymbol{\infty - \infty}}}
\newcommand{\oneToInfty}{\ensuremath{\boldsymbol{1^\infty}}}
\newcommand{\zeroToZero}{\ensuremath{\boldsymbol{0^0}}}
\newcommand{\inftyToZero}{\ensuremath{\boldsymbol{\infty^0}}}


\newcommand{\numOverZero}{\ensuremath{\boldsymbol{\tfrac{\#}{0}}}}
\newcommand{\dfn}{\textbf}
%\newcommand{\unit}{\,\mathrm}
\newcommand{\unit}{\mathop{}\!\mathrm}
%\newcommand{\eval}[1]{\bigg[ #1 \bigg]}
\newcommand{\eval}[1]{ #1 \bigg|}
\newcommand{\seq}[1]{\left( #1 \right)}
\renewcommand{\epsilon}{\varepsilon}
\renewcommand{\iff}{\Leftrightarrow}

\DeclareMathOperator{\arccot}{arccot}
\DeclareMathOperator{\arcsec}{arcsec}
\DeclareMathOperator{\arccsc}{arccsc}
\DeclareMathOperator{\si}{Si}
\DeclareMathOperator{\proj}{proj}
\DeclareMathOperator{\scal}{scal}
\DeclareMathOperator{\cis}{cis}
\DeclareMathOperator{\Arg}{Arg}
%\DeclareMathOperator{\arg}{arg}
\DeclareMathOperator{\Rep}{Re}
\DeclareMathOperator{\Imp}{Im}
\DeclareMathOperator{\sech}{sech}
\DeclareMathOperator{\csch}{csch}
\DeclareMathOperator{\Log}{Log}

\newcommand{\tightoverset}[2]{% for arrow vec
  \mathop{#2}\limits^{\vbox to -.5ex{\kern-0.75ex\hbox{$#1$}\vss}}}
\newcommand{\arrowvec}{\overrightarrow}
\renewcommand{\vec}{\mathbf}
\newcommand{\veci}{{\boldsymbol{\hat{\imath}}}}
\newcommand{\vecj}{{\boldsymbol{\hat{\jmath}}}}
\newcommand{\veck}{{\boldsymbol{\hat{k}}}}
\newcommand{\vecl}{\boldsymbol{\l}}
\newcommand{\utan}{\vec{\hat{t}}}
\newcommand{\unormal}{\vec{\hat{n}}}
\newcommand{\ubinormal}{\vec{\hat{b}}}

\newcommand{\dotp}{\bullet}
\newcommand{\cross}{\boldsymbol\times}
\newcommand{\grad}{\boldsymbol\nabla}
\newcommand{\divergence}{\grad\dotp}
\newcommand{\curl}{\grad\cross}
%% Simple horiz vectors
\renewcommand{\vector}[1]{\left\langle #1\right\rangle}


\pgfplotsset{compat=1.13}

\outcome{Enumerate the number of relative derangements of a set}

\title{2.4 Relative Derangements}

\begin{document}



\begin{abstract}
We use the Inclusion-Exclusion Principle to enumerate relative derangements.
\end{abstract}

\maketitle



\begin{definition}[Relative Derangements]
A relative derangement of a permutation of $n$ objects is another permutation of these $n$ objects for which each object in the original 
permutation has a different successor. The number of relative derangements of a permutation consisting 
of $n$ objects is denoted by $Q_n$.
\end{definition}

\begin{remark}
In its simplest form, a relative derangement of the permutation $123...n$ is a permutation of $S = \{1, 2, 3, ..., n\}$ which does not contain any of the strings $12, 23, 34, ..., (n-1)n$.
\end{remark}

\begin{example}
List the relative derangements of the letters $CAT$.\\
  We consider permutations of $CAT$ for which the $C$ does not proceed 
the $A$, and the $A$ does not proceed the $T$. The relative derangements of $CAT$ are
\[
CTA, ACT, \text{and} \;TCA
\]
Since there are 3 of them, we can write $Q_3 = 3$.
Note that while $TCA$ is a derangement of $CAT$, it is not a relative derangement 
since $A$ follows $C$ in both $CAT$ and $TCA$.
\end{example}

\begin{problem}(problem 1)
Show that $Q_2 = 1$ and $Q_4 = 11$ by explicitly listing all relative derangements of $AB$ and $ABCD$ respectively.
\end{problem}
%ABCD: ACBD ADCB BADC BDAC BDCA CADB CBAD CBDA DACB DBAC DCBA

Before we find a general formula for $Q_n$, we will explore $Q_5$. 
We wish to count the number of relative derangements of $12345$.\\
Let $A_1$ be the permutations of $12345$ which contain the string $12$;\\
let $A_2$ be the permutations of $12345$ which contain the string $23$;\\
let $A_3$ be the permutations of $12345$ which contain the string $34$;\\
and let $A_4$ be the permutations of $12345$ which contain the string $45$.\\
The number of permutations in $A_1$ is then number of permutations of the symbols $12, 3, 4, 5$.  
Since there are 4 distinct symbols, we have $|A_1|$ = 4!. Furthermore, $|A_2| = |A_3| = |A_4| = 4!$ as well.
Next we consider $|A_1 \cap A_2|$ and $|A_1 \cap A_3|$. Permutations in the set $A_1 \cap A_2$ include 
the strings $12$ and $23$ and hence they contain the string $123$. Thus, $|A_1 \cap A_2| = 3!$ since they 
consist of permutations of the symbols $123, 4, 5$. Now consider permutations in the set $A_1 \cap A_3$. 
These permutations contain the strings $12$ and $34$.  Unlike the previous situation, these strings do 
not overlap. We have $|A_1 \cap A_3| = 3!$ anyway, since we are considering permutations of the three 
symbols $12, 34, 5$. We see that whether the strings overlap or not, the number of permutations containing 
the same number of 2-strings is the same. In a similar manner, when we consider three way intersections
we obtain the same number of permutations even though on the surface they appear to be different.
Consider $A_1 \cap A_2 \cap A_3$ and $A_1 \cap A_2 \cap A_4$. In the first case, the set contains permutations of the symbols $1234, 5$ and in the second case $123, 45$.  In either case there are $2!$ permutations, so
$|A_1 \cap A_2 \cap A_3|=|A_1 \cap A_2 \cap A_4|$.  In this way, we see that the three way 
intersections all contain the same number of permutations. We are now able to compute $Q_5$. We have
\begin{align*}
Q_5 &= |A_1^c \cap A_2^c \cap A_3^c \cap A_4^c|\\
    &= |(A_1 \cup A_2 \cup A_3 \cup A_4)^c|\\
    &= |S| - |A_1 \cup A_2 \cup A_3 \cup A_4|\\
    &= |S| - \sum_{i =1}^4 |A_i| + \sum_{1=i<j}^4 |A_i \cap A_j|\\
    & \qquad  - \sum_{1 = i <j<k}^4 |A_i \cap A_j \cap A_k| \;\;+\;\; |A_1 \cap A_2 \cap A_3 \cap A_4|\\
    &= 5! - \binom{4}{1}4! + \binom{4}{2} 3! - \binom{4}{3}2! + 1\\
    &= 120 - 96 + 36 - 8 +1 = 53
 \end{align*}
    



We now compute the number of relative derangements, $Q_n$

\begin{theorem}[Relative Derangements]
The number of relative derangements of $123...n$, i.e., the number of permutations of the set $S = \{1, 2, 3, ..., n\}$ which do not contain any of the strings $12, 23, 34, ..., (n-1)n$, is given by
\[
Q_n = (n-1)! \sum_{k = 0}^{n-1} (-1)^k \frac{n-k}{k!}
\]
\end{theorem}



\begin{proof}
For each $j$ from $1$ to $n-1$, let $A_j$ be the set of permutations of $S$ that contain the string $j(j+1)$ 
Then $|S| = n!$ and $|A_j| = (n-1)!$ for each $j$. Furthermore, based on the pattern we observed when computing $Q_5$,  $|A_i \cap A_j| = (n-2)!$ and $|A_i \cap A_j \cap A_k| = (n-3)!$ and in general, 
the number of elements in the intersection of $m$ of these sets will be $(n-m)!$.\\
We are now ready to compute $\dis Q_n = |A_1^c \cap A_2^c \cap \cdots \cap A_{n-1}^c|$
By De Morgan's Law (for $n$ sets), the Rule for Complements and the Inclusion-Exclusion Principle, we have
\begin{align*}
Q_n &= |A_1^c \cap A_2^c \cap \cdots \cap A_{n-1}^c| = |\left(A_1 \cup A_2 \cup \cdots \cup A_{n-1}\right)^c|
   = |S| - |A_1 \cup A_2 \cup \cdots \cup A_{n-1}|\\
   &= n! - \bigg[\sum_{i =1}^{n-1} |A_i| - \sum_{1 =i<j}^{n-1} |A_i\cap A_j|+ 
   \sum_{1 =i<j< k}^{n-1} |A_i\cap A_j \cap A_k| \\
   &\hspace{2.5 in} - \cdots + (-1)^{n-2} |A_1 \cap A_2 \cap \cdots \cap A_{n-1}|\bigg]\\
   &= n! - \binom{n-1}{1}(n-1)! + \binom{n-1}{2}(n-2)! - \binom{n-1}{3}(n-3)! + \cdots + (-1)^{n-1} \binom{n-1}{n-1}(1)!\\
   &= (n-1)!\left[n -\frac{n-1}{1!} + \frac{n-2}{2!} - \frac{n-3}{3!} + \cdots + (-1)^{n-1} \frac{1}{(n-1)!}\right]\\
   &= (n-1)! \cdot \sum_{k=0}^{n-1} (-1)^k \frac{n-k}{k!}
\end{align*}


\end{proof}

\begin{problem}(problem 2) A herd of 30 elephants is walking single file on a long trek.  
At the midpoint, they stop at a watering hole. Upon resuming their single file march, they decide that 
no elephant should have the same elephant immediately in front of them.  
In how many single file formations can the elephants resume their journey?\\
\begin{multipleChoice}
\choice{$D_{30}$}\\
\choice[correct]{$Q_{30}$}\\
\choice{$30!$}\\
\choice{$30^{30}$}
\end{multipleChoice}
\end{problem}


\begin{problem}(problem 3) Use the Inclusion-Exclusion Principle to compute the number of relative derangements of ABCDE.
\end{problem}

It turns out that the relative derangement numbers, $Q_n$ are related to the derangement numbers $D_n$.

\begin{problem}(problem 4a)  Show that for $k \geq 1$
\[
(-1)^k \frac{n-k}{k!} = (-1)^k \frac{n}{k!} + (-1)^{k-1} \frac{1}{(k-1)!}
\]
and use this to show that 
\[
Q_n = D_n + D_{n-1}
\]
for $n \geq 2$.
\end{problem}


\begin{problem}(problem 4b) Use the result of problem 4a and the identity 
\[
D_n = (n-1)(D_{n-1} + D_{n-2})
\]
 to prove that for $n \geq 2$
\[
D_{n+1} = nQ_n
\]
\end{problem}

We now turn to relative derangements in a circular setting.

\begin{problem}(problem 5a)
Alice, Bob, Charles and Diane sit at a round table (in that order) for lunch every weekday. One day, Bob suggests 
that they change seats in such a way that no one is sitting to the left of the usual person.\\
i) Determine the number of possible new seating arrangements by explicitly listing them. \\
ii) Determine the number of new seating arrangements using the Inclusion-Exclusion Principle.
\end{problem}

\begin{problem}(problem 5b)
Alice, Bob, Charles, Diane and Ethel sit at a round table (in that order) for lunch every weekday. One day, Bob suggests 
that they change seats in such as way that no one is sitting to the left of the usual person.\\
i) Determine the number of possible new seating arrangements by explicitly listing them. \\
ii) Determine the number of new seating arrangements using the Inclusion-Exclusion Principle.
\end{problem}

\begin{proposition}
Let $n \geq 2$. The number of relative derangements of a circular permutation of $[n] = \{1, 2, ..., n\}$ is given by
\[
C_n = n!\sum_{k=0}^{n-1} \frac{(-1)^k}{(n-k)k!} + (-1)^n
\]
\end{proposition}

\begin{proof}
Let $S$ be the set of all circular permutations of $[n]$. Let $A_i$ be the set of circular permutations 
containing the string $i(i+1)$ for $i = 1, 2, ..., n-1$
and let $A_n$ be the set containing the string $n1$. We seek $\dis \left|\bigcap_{i=1}^n A_i^c\right|$. We have
\begin{align*}
\left|\bigcap_{i=1}^n A_i^c\right| &= |S| - \left|\bigcup_{i=1}^n A_i\right|\\
&= (n-1)! - \sum |A_i| + \sum |A_i \cap A_j| - \sum |A_i \cap A_j \cap A_k| 
    + \cdots + (-1)^n \left|\bigcap_{i=1}^n A_i\right|\\
&= (n-1)! - \binom{n}{1} (n-2)! + \binom{n}{2} (n-3)! - \binom{n}{3} (n-4)! + \cdots \\
 & \qquad \qquad\qquad\qquad\qquad\qquad\qquad\qquad\qquad+ (-1)^{n-1} \binom{n}{n-1} (n-n)! \;+ (-1)^n\\
&= \sum_{k=0}^{n-1} (-1)^k \binom{n}{k} (n-k-1)! \;+ (-1)^n\\
&= \sum_{k=0}^{n-1} (-1)^k \frac{n!}{k!(n-k)!} (n-k-1)! \;+ (-1)^n\\
&= n! \sum_{k=0}^{n-1} (-1)^k \frac{1}{(n-k) k!} \;+ (-1)^n
\end{align*}
\end{proof}


\begin{proposition}
For $n \geq 2$
\[
D_n = C_n + C_{n+1}
\]
\end{proposition}

\begin{proof}
Note that since
\[
\frac{n!}{(n-k)k!} = \binom{n}{k}(n-k-1)!
\]
$C_n$ can be written using binomial coefficients as 
\[
C_n = \sum_{k=0}^{n-1} (-1)^{k} \binom{n}{k} (n-k-1)! + (-1)^{n}
\]
Replacing $n$ by $n+1$ in the formula for $C_n$, we have
\begin{align*}
C_{n+1} &= \sum_{k=0}^{n} (-1)^k\binom{n+1}{k}(n-k)! + (-1)^{n+1}\\
        &= n! + \sum_{k=1}^{n} (-1)^k \binom{n+1}{k} (n-k)! + (-1)^{n+1}\\
        &= n! + \sum_{k=0}^{n-1} (-1)^{k+1} \binom{n+1}{k+1} (n-k-1)! + (-1)^{n+1}
\end{align*}
Now we add and use Pascal's identity to combine the summands
\begin{align*}
C_n + C_{n+1} &= \sum_{k=0}^{n-1} (-1)^{k} \binom{n}{k} (n-k-1)! + (-1)^{n}
                 + n! + \sum_{k=0}^{n-1} (-1)^{k+1} \binom{n+1}{k+1} (n-k-1)! + (-1)^{n+1}\\
              &= n! + \sum_{k=0}^{n-1} (-1)^k \left[\binom{n}{k} - \binom{n+1}{k+1}\right] (n-k-1)!\\
              &= n! + \sum_{k=0}^{n-1} (-1)^k \left[ - \binom{n}{k+1}\right] (n-k-1)!\\
              & = n! - \sum_{k=0}^{n-1} (-1)^k \binom{n}{k+1} (n-k-1)!\\
              &= n! - \sum_{k=1}^{n} (-1)^{k-1} \binom{n}{k} (n-k)!\\
              &= n! + \sum_{k=1}^{n} (-1)^k \binom{n}{k} (n-k)!\\
              &= n! + n!  \sum_{k=1}^{n} (-1)^k \frac{1}{k!}\\
              & = n!  \sum_{k=0}^{n} (-1)^k \frac{1}{k!}\\
              &= D_n
\end{align*}
\end{proof}




\end{document}


