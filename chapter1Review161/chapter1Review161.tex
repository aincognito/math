\documentclass[handout]{ximera}

%% You can put user macros here
%% However, you cannot make new environments



\newcommand{\ffrac}[2]{\frac{\text{\footnotesize $#1$}}{\text{\footnotesize $#2$}}}
\newcommand{\vasymptote}[2][]{
    \draw [densely dashed,#1] ({rel axis cs:0,0} -| {axis cs:#2,0}) -- ({rel axis cs:0,1} -| {axis cs:#2,0});
}


\graphicspath{{./}{firstExample/}}
\usepackage{forest}
\usepackage{amsmath}
\usepackage{amssymb}
\usepackage{array}
\usepackage[makeroom]{cancel} %% for strike outs
\usepackage{pgffor} %% required for integral for loops
\usepackage{tikz}
\usepackage{tikz-cd}
\usepackage{tkz-euclide}
\usetikzlibrary{shapes.multipart}


%\usetkzobj{all}
\tikzstyle geometryDiagrams=[ultra thick,color=blue!50!black]


\usetikzlibrary{arrows}
\tikzset{>=stealth,commutative diagrams/.cd,
  arrow style=tikz,diagrams={>=stealth}} %% cool arrow head
\tikzset{shorten <>/.style={ shorten >=#1, shorten <=#1 } } %% allows shorter vectors

\usetikzlibrary{backgrounds} %% for boxes around graphs
\usetikzlibrary{shapes,positioning}  %% Clouds and stars
\usetikzlibrary{matrix} %% for matrix
\usepgfplotslibrary{polar} %% for polar plots
\usepgfplotslibrary{fillbetween} %% to shade area between curves in TikZ



%\usepackage[width=4.375in, height=7.0in, top=1.0in, papersize={5.5in,8.5in}]{geometry}
%\usepackage[pdftex]{graphicx}
%\usepackage{tipa}
%\usepackage{txfonts}
%\usepackage{textcomp}
%\usepackage{amsthm}
%\usepackage{xy}
%\usepackage{fancyhdr}
%\usepackage{xcolor}
%\usepackage{mathtools} %% for pretty underbrace % Breaks Ximera
%\usepackage{multicol}



\newcommand{\RR}{\mathbb R}
\newcommand{\R}{\mathbb R}
\newcommand{\C}{\mathbb C}
\newcommand{\N}{\mathbb N}
\newcommand{\Z}{\mathbb Z}
\newcommand{\dis}{\displaystyle}
%\renewcommand{\d}{\,d\!}
\renewcommand{\d}{\mathop{}\!d}
\newcommand{\dd}[2][]{\frac{\d #1}{\d #2}}
\newcommand{\pp}[2][]{\frac{\partial #1}{\partial #2}}
\renewcommand{\l}{\ell}
\newcommand{\ddx}{\frac{d}{\d x}}

\newcommand{\zeroOverZero}{\ensuremath{\boldsymbol{\tfrac{0}{0}}}}
\newcommand{\inftyOverInfty}{\ensuremath{\boldsymbol{\tfrac{\infty}{\infty}}}}
\newcommand{\zeroOverInfty}{\ensuremath{\boldsymbol{\tfrac{0}{\infty}}}}
\newcommand{\zeroTimesInfty}{\ensuremath{\small\boldsymbol{0\cdot \infty}}}
\newcommand{\inftyMinusInfty}{\ensuremath{\small\boldsymbol{\infty - \infty}}}
\newcommand{\oneToInfty}{\ensuremath{\boldsymbol{1^\infty}}}
\newcommand{\zeroToZero}{\ensuremath{\boldsymbol{0^0}}}
\newcommand{\inftyToZero}{\ensuremath{\boldsymbol{\infty^0}}}


\newcommand{\numOverZero}{\ensuremath{\boldsymbol{\tfrac{\#}{0}}}}
\newcommand{\dfn}{\textbf}
%\newcommand{\unit}{\,\mathrm}
\newcommand{\unit}{\mathop{}\!\mathrm}
%\newcommand{\eval}[1]{\bigg[ #1 \bigg]}
\newcommand{\eval}[1]{ #1 \bigg|}
\newcommand{\seq}[1]{\left( #1 \right)}
\renewcommand{\epsilon}{\varepsilon}
\renewcommand{\iff}{\Leftrightarrow}

\DeclareMathOperator{\arccot}{arccot}
\DeclareMathOperator{\arcsec}{arcsec}
\DeclareMathOperator{\arccsc}{arccsc}
\DeclareMathOperator{\si}{Si}
\DeclareMathOperator{\proj}{proj}
\DeclareMathOperator{\scal}{scal}
\DeclareMathOperator{\cis}{cis}
\DeclareMathOperator{\Arg}{Arg}
%\DeclareMathOperator{\arg}{arg}
\DeclareMathOperator{\Rep}{Re}
\DeclareMathOperator{\Imp}{Im}
\DeclareMathOperator{\sech}{sech}
\DeclareMathOperator{\csch}{csch}
\DeclareMathOperator{\Log}{Log}

\newcommand{\tightoverset}[2]{% for arrow vec
  \mathop{#2}\limits^{\vbox to -.5ex{\kern-0.75ex\hbox{$#1$}\vss}}}
\newcommand{\arrowvec}{\overrightarrow}
\renewcommand{\vec}{\mathbf}
\newcommand{\veci}{{\boldsymbol{\hat{\imath}}}}
\newcommand{\vecj}{{\boldsymbol{\hat{\jmath}}}}
\newcommand{\veck}{{\boldsymbol{\hat{k}}}}
\newcommand{\vecl}{\boldsymbol{\l}}
\newcommand{\utan}{\vec{\hat{t}}}
\newcommand{\unormal}{\vec{\hat{n}}}
\newcommand{\ubinormal}{\vec{\hat{b}}}

\newcommand{\dotp}{\bullet}
\newcommand{\cross}{\boldsymbol\times}
\newcommand{\grad}{\boldsymbol\nabla}
\newcommand{\divergence}{\grad\dotp}
\newcommand{\curl}{\grad\cross}
%% Simple horiz vectors
\renewcommand{\vector}[1]{\left\langle #1\right\rangle}


\outcome{Sample Tests for Test 1}

\title{1.7 Chapter 1 Review}

\begin{document}

\begin{abstract}
Sample Tests for Chapter 1
\end{abstract}

\maketitle

\section{Sample Test 1a}

\begin{problem}(problem 1)
Compute $\displaystyle \int \frac{\ln^3(x)}{x} \, dx$ 

\begin{hint}
u-sub with $u = \ln(x)$
\end{hint}

\end{problem}


\begin{problem}(problem 2)
Compute $\displaystyle \int x^2(x^3 + 2)^7 \, dx$

\begin{hint}
u-sub with $u = x^3 + 2$
\end{hint}

\end{problem}

\begin{problem}(problem 3)
Compute $\displaystyle \int x^2\cos(4x) \, dx$

\begin{hint}
IBP $\int u dv = uv - \int v du$
\end{hint}

\begin{hint}
LIATE
\end{hint}

\begin{hint}
Need IBP twice- try Tabular Integration
\end{hint}

\end{problem}


\begin{problem}(problem 4)
Compute $\displaystyle \int \tan^{-1}(x) \, dx$

\begin{hint}
IBP $\int u dv = uv - \int v du$
\end{hint}

\begin{hint}
LIATE
\end{hint}

\begin{hint}
$\displaystyle \frac{d}{dx} \tan^{-1} (x) = \frac{1}{1+x^2}$
\end{hint}


\end{problem}

\begin{problem}(problem 5)
Compute $\displaystyle \int \sin^4(x) \, dx$

\begin{hint}
Use the half angle formula twice

\[
\sin^2 \theta = \frac{1-\cos(2\theta)}{2} \quad \mbox{and}  \cos^2 \theta = \frac{1+\cos(2\theta)}{2}
\]
\end{hint}

\end{problem}

\begin{problem}(problem 6)
Compute $\displaystyle \int \tan^4(x) \sec^6(x) \, dx$

\begin{hint}
set aside $sec^2(x)$ to prepare for u-sub
\end{hint}

\begin{hint}
convert remaining secants into tangents using
\[
1+ \tan^2(x) = sec^2(x)
\]

\end{hint}
\end{problem}

\begin{problem}(problem 7)
Compute $\displaystyle \int x^3 \sqrt{1-x^2} \, dx$
\begin{hint}
trig sub with $x= \cos \theta$
\end{hint}

\end{problem}

\begin{problem}(problem 8)
Compute $\displaystyle \int \frac{1}{ \sqrt{x^2 - 4}} \, dx$
\begin{hint}
trig sub with $x= \cos \theta$
\end{hint}
\end{problem}


\begin{problem}(problem 9)
Compute $\displaystyle \int \frac{3}{x^2 + x} \, dx$
\begin{hint}
Partial Fraction (distinct linear factors)
\end{hint}
\end{problem}

\begin{problem}(problem 10)
Compute $\displaystyle \int \frac{3x-4}{x^3 +8x^2 +16x} \, dx$
\begin{hint}
Partial Fraction (repeated linear factor)
\end{hint}

\begin{hint}
$x^2 + 8x + 16 = (x+4)^2$
\end{hint}

\end{problem}

\begin{problem}(problem 11)
Compute $\displaystyle \int_1^\infty 3e^{-2x} \, dx$

\begin{hint}
replace $\infty$ with $t$ and take the limit as $t$ approaches $\infty$
\end{hint}

\begin{hint}
$\displaystyle \int e^{ax} dx = \frac{1}{a} e^{ax} + C$
\end{hint}

\begin{hint}
to calculate the limit, remove the negative exponent
\end{hint}

\end{problem}


\begin{problem}(problem 12)
Compute $\displaystyle \int_{0}^2 \frac{5}{(x-2)^2} \, dx$

\begin{hint}
replace $2$ with $t$ and take the limit as $t$ approaches $2$ from the left
\end{hint}


\begin{hint}
to calculate the limit, remove the negative exponent
\end{hint}


\end{problem}




\section{Sample Test 1b}

\begin{problem}(problem 1)
Compute $\displaystyle \int \frac{x}{(x^2+1)^3} \, dx$

\end{problem}

\begin{problem}(problem 2)
Compute $\displaystyle \int \cot(x) \, dx$

\end{problem}



\begin{problem}(problem 3)
Compute $\displaystyle \int x^2\ln(x) \, dx$

\end{problem}


\begin{problem}(problem 4)
Compute $\displaystyle \int e^x \cos(x) \, dx$

\end{problem}

\begin{problem}(problem 5)
Compute $\displaystyle \int \sin^3(2x) \cos^3(2x) \, dx$

\end{problem}

\begin{problem}(problem 6)
Compute $\displaystyle \int \sec^5(x) \tan(x)  \, dx$

\end{problem}

\begin{problem}(problem 7)
Compute $\displaystyle \int \frac{x^3}{ \sqrt{1+x^2}} \, dx$

\end{problem}

\begin{problem}(problem 8)
Compute $\displaystyle \int \frac{1}{x^2 \sqrt{x^2 - 25}} \, dx$

\end{problem}

\begin{problem}(problem 9)
Compute $\displaystyle \int \frac{x^2 - 2x+3}{x^3-x} \, dx$

\end{problem}

\begin{problem}(problem 10)
Compute $\displaystyle \int \frac{x^2 + 5x + 2}{x^3 + 16x} \, dx$

\end{problem}

\begin{problem}(problem 11)
Compute $\displaystyle \int_4^\infty \frac{4}{x^4} \, dx$

\end{problem}

\begin{problem}(problem 12)
Compute $\displaystyle \int_1^5 \frac{3}{\sqrt{x-1}} \, dx$
\end{problem}
\end{document}














