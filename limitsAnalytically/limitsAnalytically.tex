\documentclass[handout]{ximera}

%% You can put user macros here
%% However, you cannot make new environments



\newcommand{\ffrac}[2]{\frac{\text{\footnotesize $#1$}}{\text{\footnotesize $#2$}}}
\newcommand{\vasymptote}[2][]{
    \draw [densely dashed,#1] ({rel axis cs:0,0} -| {axis cs:#2,0}) -- ({rel axis cs:0,1} -| {axis cs:#2,0});
}


\graphicspath{{./}{firstExample/}}
\usepackage{forest}
\usepackage{amsmath}
\usepackage{amssymb}
\usepackage{array}
\usepackage[makeroom]{cancel} %% for strike outs
\usepackage{pgffor} %% required for integral for loops
\usepackage{tikz}
\usepackage{tikz-cd}
\usepackage{tkz-euclide}
\usetikzlibrary{shapes.multipart}


%\usetkzobj{all}
\tikzstyle geometryDiagrams=[ultra thick,color=blue!50!black]


\usetikzlibrary{arrows}
\tikzset{>=stealth,commutative diagrams/.cd,
  arrow style=tikz,diagrams={>=stealth}} %% cool arrow head
\tikzset{shorten <>/.style={ shorten >=#1, shorten <=#1 } } %% allows shorter vectors

\usetikzlibrary{backgrounds} %% for boxes around graphs
\usetikzlibrary{shapes,positioning}  %% Clouds and stars
\usetikzlibrary{matrix} %% for matrix
\usepgfplotslibrary{polar} %% for polar plots
\usepgfplotslibrary{fillbetween} %% to shade area between curves in TikZ



%\usepackage[width=4.375in, height=7.0in, top=1.0in, papersize={5.5in,8.5in}]{geometry}
%\usepackage[pdftex]{graphicx}
%\usepackage{tipa}
%\usepackage{txfonts}
%\usepackage{textcomp}
%\usepackage{amsthm}
%\usepackage{xy}
%\usepackage{fancyhdr}
%\usepackage{xcolor}
%\usepackage{mathtools} %% for pretty underbrace % Breaks Ximera
%\usepackage{multicol}



\newcommand{\RR}{\mathbb R}
\newcommand{\R}{\mathbb R}
\newcommand{\C}{\mathbb C}
\newcommand{\N}{\mathbb N}
\newcommand{\Z}{\mathbb Z}
\newcommand{\dis}{\displaystyle}
%\renewcommand{\d}{\,d\!}
\renewcommand{\d}{\mathop{}\!d}
\newcommand{\dd}[2][]{\frac{\d #1}{\d #2}}
\newcommand{\pp}[2][]{\frac{\partial #1}{\partial #2}}
\renewcommand{\l}{\ell}
\newcommand{\ddx}{\frac{d}{\d x}}

\newcommand{\zeroOverZero}{\ensuremath{\boldsymbol{\tfrac{0}{0}}}}
\newcommand{\inftyOverInfty}{\ensuremath{\boldsymbol{\tfrac{\infty}{\infty}}}}
\newcommand{\zeroOverInfty}{\ensuremath{\boldsymbol{\tfrac{0}{\infty}}}}
\newcommand{\zeroTimesInfty}{\ensuremath{\small\boldsymbol{0\cdot \infty}}}
\newcommand{\inftyMinusInfty}{\ensuremath{\small\boldsymbol{\infty - \infty}}}
\newcommand{\oneToInfty}{\ensuremath{\boldsymbol{1^\infty}}}
\newcommand{\zeroToZero}{\ensuremath{\boldsymbol{0^0}}}
\newcommand{\inftyToZero}{\ensuremath{\boldsymbol{\infty^0}}}


\newcommand{\numOverZero}{\ensuremath{\boldsymbol{\tfrac{\#}{0}}}}
\newcommand{\dfn}{\textbf}
%\newcommand{\unit}{\,\mathrm}
\newcommand{\unit}{\mathop{}\!\mathrm}
%\newcommand{\eval}[1]{\bigg[ #1 \bigg]}
\newcommand{\eval}[1]{ #1 \bigg|}
\newcommand{\seq}[1]{\left( #1 \right)}
\renewcommand{\epsilon}{\varepsilon}
\renewcommand{\iff}{\Leftrightarrow}

\DeclareMathOperator{\arccot}{arccot}
\DeclareMathOperator{\arcsec}{arcsec}
\DeclareMathOperator{\arccsc}{arccsc}
\DeclareMathOperator{\si}{Si}
\DeclareMathOperator{\proj}{proj}
\DeclareMathOperator{\scal}{scal}
\DeclareMathOperator{\cis}{cis}
\DeclareMathOperator{\Arg}{Arg}
%\DeclareMathOperator{\arg}{arg}
\DeclareMathOperator{\Rep}{Re}
\DeclareMathOperator{\Imp}{Im}
\DeclareMathOperator{\sech}{sech}
\DeclareMathOperator{\csch}{csch}
\DeclareMathOperator{\Log}{Log}

\newcommand{\tightoverset}[2]{% for arrow vec
  \mathop{#2}\limits^{\vbox to -.5ex{\kern-0.75ex\hbox{$#1$}\vss}}}
\newcommand{\arrowvec}{\overrightarrow}
\renewcommand{\vec}{\mathbf}
\newcommand{\veci}{{\boldsymbol{\hat{\imath}}}}
\newcommand{\vecj}{{\boldsymbol{\hat{\jmath}}}}
\newcommand{\veck}{{\boldsymbol{\hat{k}}}}
\newcommand{\vecl}{\boldsymbol{\l}}
\newcommand{\utan}{\vec{\hat{t}}}
\newcommand{\unormal}{\vec{\hat{n}}}
\newcommand{\ubinormal}{\vec{\hat{b}}}

\newcommand{\dotp}{\bullet}
\newcommand{\cross}{\boldsymbol\times}
\newcommand{\grad}{\boldsymbol\nabla}
\newcommand{\divergence}{\grad\dotp}
\newcommand{\curl}{\grad\cross}
%% Simple horiz vectors
\renewcommand{\vector}[1]{\left\langle #1\right\rangle}


\outcome{Compute limits using algebraic techniques}

\title{1.4 Computing Limits}


\begin{document}

\begin{abstract}
Compute limits using algebraic techniques.
\end{abstract}

\maketitle


\section{Limit laws}
 
In section 1.2, Numerical Limits, we briefly discussed ``plugging in" to compute limits.
 The example we investigated was
 \[
 \lim_{x \to 3^+} \left(x^2 + 2 \right) = 3^2 + 2 = 11.
 \]

The reason plugging in worked in the previous example is a direct consequence of the limit laws presented below.
 
In each of the following laws, all of the limits are assumed to exist.

\begin{enumerate}

\item[1.]  
The limit of a constant is the constant:
 \[
 \lim_{x \to c} k = k.
 \]
 
\item[2.]
 The next law is self-evident:
 \[
 \lim_{x \to c} x = c.
 \]
 
 \item[3.]
 The limit of a multiple of a function is the multiple of the limit:
 \[
 \lim_{x \to c} kf(x) = k \cdot \lim_{x \to c} f(x). 
 \]
 
 \item[4.]
 The limit of a sum is the sum of the limits:
 \[
 \lim_{x \to c} \left[f(x) + g(x) \right] = \lim_{x \to c} f(x) + \lim_{x \to c} g(x). 
 \]
 
\item[5.]
 The limit of a difference is the difference of the limits:
 \[
 \lim_{x \to c} \left[f(x) - g(x) \right] = \lim_{x \to c} f(x) - \lim_{x \to c} g(x). 
 \]
 
\item[6.]
 The limit of a product is the product of the limits:
 \[
 \lim_{x \to c} f(x) g(x) = \lim_{x \to c} f(x) \cdot \lim_{x \to c} g(x). 
 \]


\item[7.]
 The limit of a quotient is the quotient of the limits:
 \[
 \lim_{x \to c} \frac{f(x)}{g(x)} = \frac{\lim_{x \to c} f(x)}{\lim_{x \to c} g(x)}, 
 \]
 provided that the limit in the denominator is \textbf{not equal to zero}.
 
 \item[8.]
 Limits can be moved inside of radicals:
 \[
 \lim_{x \to c} \sqrt[n]{f(x)} = \sqrt[n]{\lim_{x \to c} f(x)}. 
 \]

\end{enumerate}


Each of the above limit laws is valid if $x \to c^+, x\to c^-, x \to \infty$ or $x \to -\infty$.



\begin{example}[example 1] Given that 
\[
\lim_{x \to 3} f(x) = -5 \quad \text{and} \quad \lim_{x \to 3} g(x) = \tfrac12,
\]
find 
\[
\lim_{x \to 3} \left[x^2f(x) - 6g(x)\right].
\]
Using limit laws 2, 3, 5 and 6, we can write
\begin{align*}
\lim_{x \to 3} \left[x^2f(x) - 6g(x)\right] &= \lim_{x \to 3} x^2 f(x) - \lim_{x \to 3} 6g(x) \quad \text{(law 5)} \\
                                            &= \lim_{x \to 3} x^2 \cdot \lim_{x \to 3}f(x) - 6\cdot\lim_{x \to 3} g(x) \quad \text{(laws 6 and 3)}\\
                                            &= \lim_{x \to 3} x \cdot \lim_{x \to 3} x \cdot \lim_{x \to 3}f(x) - 6\cdot\lim_{x \to 3} g(x) \quad \text{(law 6)}\\
                                            &= 3(3) (-5) - 6(\tfrac12) \quad \text{(law 2)}\\
                                            &= -48.
\end{align*}
\end{example}


\begin{problem}(problem 1a)
Given that 
\[
\lim_{x \to -2} f(x) = 2 \quad \text{and} \quad \lim_{x \to -2} g(x) = 0,
\]
\[
\lim_{x \to -2} \left[3f(x) + xg(x)\right] = \answer{6}.
\]
\end{problem}


\begin{problem}(problem 1b)
Given that 
\[
\lim_{x \to 1^+} f(x) = \tfrac23 \quad \text{and} \quad \lim_{x \to 1+} g(x) = 4,
\]
\[
\lim_{x \to 1^+} \frac{f(x)}{\sqrt{g(x)}} = \answer{1/3}.
\]
\end{problem}


\begin{problem}(problem 1c)
Given that 
\[
\lim_{x \to \infty} f(x) = -9 \quad \text{and} \quad \lim_{x \to \infty} g(x) = 1,
\]
\[
\lim_{x \to \infty} \sqrt[3]{f(x) + g(x)} = \answer{-2}.
\]
\end{problem}


%\section{Computing Limits}
 %\graph command must contain all 4 boundaries if it contains any (xmin, xmax, ymin, ymax)

%\[
%\graph[xmin=-2, xmax=4, ymin=-2, ymax=6]{(x-2)^2 + (y-4)^2 = 0.01, x^2 \left{-1\leq x \leq 1.97\right}}
%\]
%\[\graph[xAxisLabel=x, yAxisLabel=y, xmin=-3, xmax=10, ymin=-10, ymax=10]{y = 1/x}\]

%\[
%\graph[xAxisLabel=x, yAxisLabel=y, xmin=-3, xmax=10, ymin=-10, ymax=10]{y = 1/x, y = x/2}
%\]

%\[
%\graph{ \sin(x)\left\{x<0\right\}, 2x\left\{ x>=0 \right\} }
%\]

%\[
%\graph[color=Desmos.Colors.BLUE]{y = 1/x, y = x/2}
%\]
%\[\graph{x^2 \left\{ -1 \leq x \leq 1.97 \right\}, (x-2)^2 + (y-4)^2 = 0.01 }\]

%\[
%\graph{(2, 3)}
%\]

%\geogebra{mG34YseZ}{640}{480}
%\geogebra{PMcY6Q8j}{640}{480}


%We have seen examples of limit problems where plugging in the terminal $x$-value both led to 
%reasonable answers and meaningless forms.
%In the section we will work with examples in which plugging in the terminal $x$-value initially 
%yields a meaningless expression,
%but after performing algebraic manipulations on the function in the problem, 
%plugging in the terminal value
%yields a credible solution.



\section{Factor and cancel}

\begin{example}[example 2]
Compute the limit: \[\lim_{x \to 4} \frac{x^2 - 16}{x^2 - 4x}.\]


Plugging in the terminal value, $x=4$, yields 
the indeterminate form $\frac00$.  To find this limit analytically, we will factor the numerator 
and denominator and simplify the fraction. In the numerator we have a \textbf{difference of squares}, and 
such a form always factors in the following way:
\[a^2 - b^2 = (a+b)(a-b).\]
Applying this general formula to our example, we get 
\[x^2 - 16 = (x+4)(x-4).\]  
Next, in the denominator, we can factor out a common factor
of $x$ from each of the terms:
\[x^2 - 4x = x(x-4).\]
With these factorizations, we can simplify the fraction and find the limit:

\begin{align*}
\lim_{x \to 4} \frac{x^2 - 16}{x^2 - 4x} &= \lim_{x \to 4} \frac{(x+4)(x-4)}{x(x-4)} \enspace & \text{(factoring)} \\[.4 em]
                                         &= \lim_{x \to 4} \frac{x+4}{x} & \text{(canceling)} \\[.4 em]
                                         &= \frac84  &  \\[.4 em]
																				 &= 2. &
\end{align*}																				
																				
Hence,
\[\lim_{x \to 4} \frac{x^2 - 16}{x^2 - 4x} = 2.\]
This example uses the \textbf{factor and cancel} method.
\end{example}

\begin{problem}(problem 2)
  Compute the limit:
  \[  \lim_{x \to 3} \frac{x^2 - 9}{x^2 - 3x} = \answer{2}.  \]
    \begin{hint}
      When you plug in $x = 3$, you get $\frac00$
    \end{hint}
    \begin{hint}
      Factor the numerator and the denominator
    \end{hint}
    \begin{hint}
      Difference of squares: $a^2 - b^2 = (a+b)(a-b)$
    \end{hint}
    \begin{hint}
      Common factor: $ab \pm ac = a(b \pm c)$
    \end{hint}
\end{problem}


\begin{example}[example 3]
Compute the limit: \[\lim_{x \to 2} \frac{x^2 + 3x - 10}{x^3 - 8}.\]
\\
Plugging in the terminal value $x = 2$ yields the indeterminate form $\frac00$.
The numerator factors  by finding two numbers which multiply to give $-10$ and add to give $+3$.  
The two numbers are 
$-2$ and $+5$, hence the factorization is  
\[x^2 + 3x - 10 = (x-2)(x+5).\]
In the denominator, we have a \textbf{difference of cubes}, and we use the formula:
\[a^3 - b^3 = (a-b)(a^2+ab +b^2).\]
Applying this to our example gives 
\[x^3 - 8 = (x-2)(x^2 + 2x + 4).\]
With these factorizations, we can simplify the fraction and find the limit:

\begin{align*}
\lim_{x \to 2} \frac{x^2 + 3x - 10}{x^3 - 8} &= \lim_{x \to 2}\frac{(x-2)(x+5)}{(x-2)(x^2 + 2x + 4)} 
\enspace & \text{(factoring)} \\[.4 em]
&= \lim_{x \to 2} \frac{x+5}{x^2 + 2x + 4} & \text{(canceling)} \\[.4 em]
&= \frac{2+5}{2^2 + (2\cdot 2) + 4} & \text{(plugging in)} \\[.4 em]
&= \frac{7}{12}. 
\end{align*}

Hence,
\[\lim_{x \to 2} \frac{x^2 + 3x - 10}{x^3 - 8} = \frac{7}{12}.\]
\end{example}




\begin{problem}(problem 3a)
  Compute the following limit which we investigated numerically in the previous section:
  \[
  \lim_{x \to 1} \frac{x^3 - 1}{x^2 -1}.
  \]
  
    \begin{hint}
      When you plug in $x = 1$, you get $\frac00$
    \end{hint}
    \begin{hint}
      Factor the numerator and the denominator
    \end{hint}
    \begin{hint}
      Difference of cubes: $a^3 - b^3 = (a-b)(a^2 + ab +b^2)$
    \end{hint}
		\begin{hint}
      Difference of squares: $a^2 - b^2 = (a-b)(a+b)$
    \end{hint}

		The value of the limit is
		 $\answer{\frac32}$
		
\end{problem}


\begin{problem}(problem 3b)
  Compute the limit:
  \[
  \lim_{x \to 3} \frac{x^3 - 27}{x^2 -2x - 3}.
  \]
  
    \begin{hint}
      When you plug in $x = 3$, you get $\frac00$
    \end{hint}
    \begin{hint}
      Factor the numerator and the denominator
    \end{hint}
    \begin{hint}
      Difference of cubes: $a^3 - b^3 = (a-b)(a^2 + ab +b^2)$
    \end{hint}

		The value of the limit is
		 $\answer{\frac{27}{4}}$
		
\end{problem}


\begin{example}[example 4]
Compute the limit: \[\lim_{x \to 4} \frac{4x^2 - 19x + 12}{6x^2 -19x -20}.\]
\\
Plugging in the terminal value $x = 4$ yields the indeterminate form $\frac00$.
To factor the numerator and denominator in this case, we will use the important fact that
for a polynomial $p(x)$ and a number $a$, 
\[\text{If} \quad p(a) = 0, \quad \text{then} \quad  x-a \quad \text{is a factor of} \; p(x).\]
And once we know one factor of a polynomial, it is usually much easier to find the other.
Since plugging in $x=4$ gave $0$ in both the numerator and the denominator, both polynomials 
have $x-4$ as a factor.
In the numerator, the factorization is $(x-4)(4x-3)$ and in the denominator, 
the factorization is $(x-4)(6x+5)$.

With these factorizations, we can simplify the fraction and find the limit:

\begin{align*}
\lim_{x \to 4} \frac{4x^2 - 19x + 12}{6x^2 -19x -20} &= \lim_{x \to 4}\frac{(x-4)(4x-3)}{(x-4)(6x + 5)} 
\enspace & \text{(factoring)} \\[.4 em]
&= \lim_{x \to 4} \frac{4x-3}{6x + 5} & \text{(canceling)} \\[.4 em]
&= \frac{16-3}{24+5} & \text{(plugging in)}\\[.4 em]
&= \frac{13}{29}. 
\end{align*}

Hence,
\[\lim_{x \to 4} \frac{4x^2 -19x +12}{6x^2 -19x -20} = \frac{13}{29}.\]
\end{example}



\begin{problem}(problem 4a)
  Compute the limit:
  \[
  \lim_{x \to 2} \frac{3x^2 -8x + 4}{2x^2 + x - 10}.
  \]
  
    \begin{hint}
      When you plug in $x = 2$, you get $\frac00$
    \end{hint}
    \begin{hint}
      Factor the numerator and the denominator
    \end{hint}
    \begin{hint}
      If $p(a) = 0$, then $(x-a)$ is a factor
    \end{hint}
    \begin{hint}
      $x-2$ is a factor of both the numerator and the denominator
    \end{hint}
    
		The value of the limit is
		 $\answer{4/9}$
		
\end{problem}


\begin{problem}(problem 4b)
  Compute the limit:
  \[
  \lim_{x \to -2} \frac{2x^2 + x - 6}{x^2 + 5x + 6}.
  \]
  
    \begin{hint}
      When you plug in $x = -2$, you get $\frac00$
    \end{hint}
    \begin{hint}
      Factor the numerator and the denominator
    \end{hint}
    \begin{hint}
      If $p(a) = 0$, then $(x-a)$ is a factor
    \end{hint}
    \begin{hint}
      $x+2$ is a factor of both the numerator and the denominator
    \end{hint}
    
		The value of the limit is
		 $\answer{-7}$
		
\end{problem}


\section{Conjugate radicals}


%The difference of two squares formula, , 
%is helpful for dealing with limits involving radicals. 
%When dealing with radials, one of $a$ or $b$ (or possibly even both) 
%will involve a radical symbol, like $\sqrt{x+3}$.
The expressions 
\[\sqrt u + \sqrt v \quad \text{and} \quad \sqrt u - \sqrt v\]
 are called
\textbf{conjugate radicals}. When we multiply conjugate radicals
using the difference of squares formula, $(a+b)(a-b) = a^2 - b^2$,
we get an expression that is free of radicals: 

\[(\sqrt u + \sqrt v)(\sqrt u - \sqrt v) = (\sqrt u)^2 - (\sqrt v)^2 = u - v.\]

We will now take advantage of this to find limits.

\begin{example}[example 5]
Compute the limit:
\[\lim_{x \to 4} \frac{\sqrt{x} - 2}{x-4}.\]

To solve this limit problem, we will use the conjugate radical of the
numerator, which is $\sqrt{x} + 2$.  In order to maintain the value of
the expression given in the problem, we multiply by one in the form of the conjugate radical
over itself:



\begin{align*}
\lim_{x \to 4} \frac{\sqrt{x}- 2}{x-4} &=
\lim_{x \to 4} \frac{\sqrt{x} -2}{x-4}\cdot \frac{\sqrt{x} +2}{\sqrt{x}+2} \\[.4 em]
 &= \lim_{x \to 4} \frac{x-4}{(x-4)(\sqrt{x}+2)} \\[.4 em]
&= \lim_{x \to 4}\frac{1}{\sqrt{x}+2} \\[.4 em]
&= \frac 14.
\end{align*}

In the last step, we plugged in the terminal value $x=4$ to get the final answer $\frac14$.

\end{example}



\begin{problem}(problem 5a)
  Compute the limit:
  \[
  \lim_{x \to 9} \frac{\sqrt{x}-3}{x-9}
  \]
  
    \begin{hint}
      When you plug in $x = 9$, you get $\frac00$
    \end{hint}
    \begin{hint}
      Multiply by the conjugate radical
    \end{hint}
    \begin{hint}
      $\sqrt a + \sqrt b$ and $\sqrt a - \sqrt b$ are conjugates
    \end{hint}
    \begin{hint}
      Use the difference of squares formula in the numerator: $(a-b)(a+b) = a^2 - b^2$ 
    \end{hint}
    
		The value of the limit is
		 $\answer{1/6}$
		
\end{problem}

\begin{problem}(problem 5b)
  Compute the limit:
  \[
  \lim_{x \to 2} \frac{x- \sqrt{2x}}{x-2}
  \]
  
    \begin{hint}
      When you plug in $x = 2$, you get $\frac00$
    \end{hint}
    \begin{hint}
      Multiply by the conjugate radical
    \end{hint}
    \begin{hint}
      $\sqrt a + \sqrt b$ and $\sqrt a - \sqrt b$ are conjugates
    \end{hint}
    \begin{hint}
      Use the difference of squares formula in the numerator: $(a-b)(a+b) = a^2 - b^2$ 
    \end{hint}
    
		The value of the limit is
		 $\answer{\frac12}$
		
\end{problem}

\begin{example}[example 6]
Compute the limit:
\[\lim_{h \to 0} \frac{\sqrt{9+h}-3}{h}.\]

To solve this limit problem, we will use the conjugate radical of the
numerator, which is $\sqrt{9+h}+3$.  In order to not change the value of
the expression given in the problem, we multiply by one in the form of the conjugate radical
divided by itself:



\begin{align*}
\lim_{h \to 0} \frac{\sqrt{9+h}-3}{h} &=
\lim_{h \to 0} \frac{\sqrt{9+h}-3}{h}\cdot \frac{\sqrt{9+h}+3}{\sqrt{9+h}+3} \\[.4 em]
 &= \lim_{h \to 0} \frac{(9+h) - 9}{h(\sqrt{9+h} + 3)} \\[.4 em]
&= \lim_{h \to 0} \frac{h}{h(\sqrt{9+h} + 3)} \\[.4 em]
&= \lim_{h \to 0}\frac{1}{\sqrt{9+h} + 3} \\[.4 em]
&= \frac16.
\end{align*}

In the last step, we plugged in the terminal value $h=0$ to get the final answer $\frac16$.

\end{example}

\begin{problem} (problem 6)
Compute the limit:
  \[
  \lim_{h \to 0} \frac{\sqrt{4-h}-2}{h}
  \]
  
    \begin{hint}
      When you plug in $h = 0$, you get $\frac00$
    \end{hint}
    \begin{hint}
      Multiply by the conjugate radical
    \end{hint}
    \begin{hint}
      $\sqrt a + \sqrt b$ and $\sqrt a - \sqrt b$ are conjugates
    \end{hint}
    \begin{hint}
      Use the difference of squares formula in the numerator: $(a-b)(a+b) = a^2 - b^2$ 
    \end{hint}
    
		The value of the limit is
		 $\answer{-\frac14}$
		
\end{problem}







\begin{example}[example 7]
Compute the limit: $\displaystyle{\lim_{x \to -3} \frac{x^2 - 9}{4 - \sqrt{13 -x}}}$.\\
\\
If we plug in $x = -3$, the numerator and denominator are both 0. The conjugate of the radical expression in the denominator is
\[ 4 + \sqrt{13 -x}.\]
To simplify our calculations a little bit, let's multiply the conjugates together separately:
\[(4 - \sqrt{13 -x})(4 + \sqrt{13 -x}) = 16 - (13-x) = 3+x = x+3.\]
With this in mind, we have:

\begin{align*}
\lim_{x \to -3} \frac{x^2 - 9}{4 - \sqrt{13 -x}} &= 
\lim_{x \to -3} \left(\frac{x^2 - 9}{4 - \sqrt{13 -x}}\right) \cdot \left(\frac{4 + \sqrt{13 -x}}{4 + \sqrt{13 -x}}\right) \\[.4 em]
&=\lim_{x \to -3} \frac{(x^2 - 9)\left(4 + \sqrt{13 -x}\right)}{x+3}\\[.4 em]
&=\lim_{x \to -3} \frac{(x+3)(x-3)\left(4 + \sqrt{13 -x}\right)}{x+3} \\[.4 em]
&= \lim_{x \to -3} (x-3)\left(4 + \sqrt{13 -x}\right)\\[.4 em]
&= (-3-3)\left(4 + \sqrt{13- (-3)}\right) \\[.4 em]
&= (-6)(4 + 4) \\[.4 em]
&= -48.
\end{align*}
\end{example}



		
\begin{problem}(problem 7a)
  Compute the limit:
  \[
  \lim_{x \to -4} \frac{x^2 - 16}{3 - \sqrt{5 -x}}
  \]
  
    \begin{hint}
      When you plug in $x = -4$, you get $\frac00$
    \end{hint}
    \begin{hint}
      Multiply by the conjugate radical
    \end{hint}
    \begin{hint}
      $\sqrt a + \sqrt b$ and $\sqrt a - \sqrt b$ are conjugates
    \end{hint}
    \begin{hint}
      Use the difference of squares formula in the denominator: $(a-b)(a+b) = a^2 - b^2$
    \end{hint}
    \begin{hint}
      $\sqrt a \cdot \sqrt a = a$
    \end{hint}
		The value of the limit is
		 $\answer{-48}$
		
\end{problem}



\begin{problem}(problem 7b)
  Compute the limit:
  \[
  \lim_{x \to 0} \frac{4- \sqrt{x + 16}}{x}.
  \]
  
    \begin{hint}
      When you plug in $x = 0$, you get $\frac00$
    \end{hint}
    \begin{hint}
      Multiply by the conjugate radical
    \end{hint}
    \begin{hint}
      $\sqrt a + \sqrt b$ and $\sqrt a - \sqrt b$ are conjugates
    \end{hint}
    \begin{hint}
      Use the difference of squares formula in the numerator: $(a-b)(a+b) = a^2 - b^2$
    \end{hint}
    \begin{hint}
      $\sqrt a \cdot \sqrt a = a$ 
    \end{hint}
		The value of the limit is
		 $\answer{-\frac18}$
		
\end{problem}

\section{Complex fractions}


We now consider examples involving fractions within fractions, called \textbf{complex fractions}.




\begin{example}[example 8]
Compute the limit: $\displaystyle{\lim_{x \to 4} \frac{\frac{1}{x} - \frac{1}{4}}{x-4}}.$ \\
\\
Plugging in $x=4$ yields the familiar $\frac00$ indeterminate form. The algebra skills necessary to transform the 
function in the problem to one which allows plugging in $x=4$ involve subtraction and division of fractions. 
The rules are summarized as:
\[\frac{a}{b} - \frac{c}{d} = \frac{ad-bc}{bd} \quad \text{and} \quad \frac{\frac{a}{b}}{\frac{c}{d}} = \frac{ad}{bc}.\]
Applying these to our problem, we get

\begin{align*}
\lim_{x \to 4} \frac{\frac{1}{x} - \frac{1}{4}}{x-4} &= \lim_{x \to 4} \frac{\frac{4-x}{4x}}{x-4}\\[.4 em]
&=\lim_{x \to 4} \frac{4-x}{4x(x-4)}\\[.4 em]
&= \lim_{x \to 4} -\frac{1}{4x} \\[.4 em]
&= -\frac{1}{16}.
\end{align*}

It is important to note that $a-b$ and $b-a$ are opposites which cancel, leaving $-1$: 
\[ \frac{a-b}{b-a} = -1. \]
The ratio of opposites is $-1$.
\end{example}



\begin{problem}(problem 8)
  Compute the limit:
  \[
  \lim_{x \to 5} \frac{\frac{1}{x} - \frac{1}{5}}{x-5}.
  \]
  
    \begin{hint}
      When you plug in $x = 5$, you get $\frac00$
    \end{hint}
    \begin{hint}
      Subtract the fractions in the numerator
    \end{hint}
    \begin{hint}
      $\frac{a}{b} \pm \frac{c}{d} = \frac{ad \pm bc}{bd}$.
    \end{hint}
    \begin{hint}
      To divide, multiply by the reciprocal: $\frac{a}{b} \div \frac{c}{d} = \frac{a}{b} \cdot \frac{d}{c}$ 
    \end{hint}
    \begin{hint}
      Simplify the fraction by canceling 
    \end{hint}
		The value of the limit is
		 $\answer{-\frac{1}{25}}$
		
\end{problem}




\begin{example}[example 9]
Compute the limit:
\[\lim_{x \to -2} \frac{\frac{2}{x-3} + \frac{x+4}{5}}{x^2 + 5x + 6}.\]
Carefully plugging in $x=-2$ gives $\frac00$, so we begin simplifying the complex fraction.
Since this function is bulky, let's do the addition of fractions from the numerator separately:
\begin{align*}
\frac{2}{x-3} + \frac{x+4}{5} &= \frac{10 + (x-3)(x+4)}{5(x-3)}\\[.4 em]
&= \frac{x^2 +x -2}{5(x-3)} \\[.4 em]
&= \frac{(x-1)(x+2)}{5(x-3)}.
\end{align*}
Now back to the original problem:
\begin{align*}
\lim_{x \to -2} \frac{\frac{2}{x-3} + \frac{x+4}{5}}{x^2 + 5x + 6} &= 
\lim_{x \to -2} \frac{\frac{(x-1)(x+2)}{5(x-3)}}{(x+2)(x+3)}\\[.4 em]
&= \lim_{x \to -2} \frac{(x-1)(x+2)}{5(x-3)(x+2)(x+3)} \\[.4 em]
&= \lim_{x \to -2} \frac{(x-1)}{5(x-3)(x+3)}\\[.4 em]
&= \frac{-3}{5(-5)(1)} = \frac{3}{25}.
\end{align*}
\end{example}


\begin{problem}(problem 9a)
  Compute the limit:
  \[
  \lim_{x \to -2} \frac{\frac{3}{x+1} + \frac{6}{x+4}}{x+2}.
  \]
  
    \begin{hint}
      When you plug in $x = -2$, you get $\frac00$
    \end{hint}
    \begin{hint}
      Add the fractions in the numerator
    \end{hint}
    \begin{hint}
      $\frac{a}{b} \pm \frac{c}{d} = \frac{ad \pm bc}{bd}$
    \end{hint}
    \begin{hint}
      To divide, multiply by the reciprocal: $\frac{a}{b} \div \frac{c}{d} = \frac{a}{b} \cdot \frac{d}{c}$
    \end{hint}
    \begin{hint}
      Cancel a common factor 
    \end{hint}
		The value of the limit is
		 $\answer{-\frac{9}{2}}$
		
\end{problem}



\begin{problem}(problem 9b)
  Compute the limit:
  \[
    \lim_{x \to 3} \frac{\frac{2}{x-5} + \frac{x+2}{5}}{x^2 - 5x + 6}.
  \]
  
    \begin{hint}
      When you plug in $x = 3$, you get $\frac00$
    \end{hint}
    \begin{hint}
      Add the fractions in the numerator
    \end{hint}
    \begin{hint}
      $\frac{a}{b} \pm \frac{c}{d} = \frac{ad \pm bc}{bd}$
    \end{hint}
    \begin{hint}
      To divide, multiply by the reciprocal $\frac{a}{b} \div \frac{c}{d} = \frac{a}{b} \cdot \frac{d}{c}$ 
    \end{hint}
    \begin{hint}
      Cancel a common factor
    \end{hint}
		The value of the limit is
		 $\answer{-\frac{3}{10}}$
		
\end{problem}




\section{Absolute values}


The definition of the absolute value is:

\[
\left | x \right | = 
\begin{cases}
\hfill x & \text{if $x\geq 0$}\\         -x & \text{if $x<0$}
\end{cases}
\] 

To calculate a limit involving an absolute value, we will need to remove the absolute value bars. 
To do this correctly, we can see from the definition that it is necessary to know whether the 
quantity in the absolute value bars is positive or negative.


\begin{example}[example 10]
Compute the limit: $\displaystyle{\lim_{x \to 3^{-}} \frac{|x-3|}{x-3}}$.\\
Plugging in $x = 3$ gives the indeterminate form $\frac00$. To resolve this limit, 
we will do a sign analysis on the quantity in the absolute value bars, $x-3$.

Since $x \to 3^-$, we have $x<3$ and hence $x-3 <0$. Since the quantity in the absolute value bars is negative, 
we can compute its absolute value as follows:

\[|x-3| = -(x-3).\]

Using this in the limit, we get

\begin{align*}
\lim_{x \to 3^-} \frac{|x-3|}{x-3} &= \lim_{x \to 3^-} \frac{-(x-3)}{x-3} \\[.4 em]
&= \lim_{x \to 3^-} (-1) \\
&= -1.
\end{align*}
\end{example}

\begin{problem}(problem 10a)
  Compute the limit:
  \[
  \lim_{x \to 4^-} \frac{4-x}{|x-4|}
  \]
  
    \begin{hint}
      Since $x \to 4^-$, we have $x<4$
    \end{hint}
    \begin{hint}
      Is $x-4$ positive or negative?
    \end{hint}
    \begin{hint}
      Remove the absolute value bars; include a negative sign if necessary
    \end{hint}
		\begin{hint}
      Simplify the fraction
    \end{hint}
		The value of the limit is
		 $\answer{1}$
		
\end{problem}


\begin{problem}(problem 10b)
  Compute the limit:
  \[
  \lim_{x \to -3^-} \frac{|x+3|}{2x+6}
  \]
  
    \begin{hint}
      Since $x \to -3^-$, we have $x<-3$
    \end{hint}
    \begin{hint}
      Is $x+3$ positive or negative?
    \end{hint}
    \begin{hint}
      Remove the absolute value bars; include a negative sign if necessary
    \end{hint}
		\begin{hint}
      Simplify the fraction
    \end{hint}
		The value of the limit is
		 $\answer{-1/2}$
		
\end{problem}


\begin{center}
\begin{foldable}
\unfoldable{Here are some detailed, lecture style videos on finding limits analytically:}
\youtube{JroM3rmBH50}
\youtube{zN8wuKZatxk}
\end{foldable}
\end{center}


\end{document}






