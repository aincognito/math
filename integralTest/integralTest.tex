\documentclass[handout]{ximera}

%% You can put user macros here
%% However, you cannot make new environments



\newcommand{\ffrac}[2]{\frac{\text{\footnotesize $#1$}}{\text{\footnotesize $#2$}}}
\newcommand{\vasymptote}[2][]{
    \draw [densely dashed,#1] ({rel axis cs:0,0} -| {axis cs:#2,0}) -- ({rel axis cs:0,1} -| {axis cs:#2,0});
}


\graphicspath{{./}{firstExample/}}
\usepackage{forest}
\usepackage{amsmath}
\usepackage{amssymb}
\usepackage{array}
\usepackage[makeroom]{cancel} %% for strike outs
\usepackage{pgffor} %% required for integral for loops
\usepackage{tikz}
\usepackage{tikz-cd}
\usepackage{tkz-euclide}
\usetikzlibrary{shapes.multipart}


%\usetkzobj{all}
\tikzstyle geometryDiagrams=[ultra thick,color=blue!50!black]


\usetikzlibrary{arrows}
\tikzset{>=stealth,commutative diagrams/.cd,
  arrow style=tikz,diagrams={>=stealth}} %% cool arrow head
\tikzset{shorten <>/.style={ shorten >=#1, shorten <=#1 } } %% allows shorter vectors

\usetikzlibrary{backgrounds} %% for boxes around graphs
\usetikzlibrary{shapes,positioning}  %% Clouds and stars
\usetikzlibrary{matrix} %% for matrix
\usepgfplotslibrary{polar} %% for polar plots
\usepgfplotslibrary{fillbetween} %% to shade area between curves in TikZ



%\usepackage[width=4.375in, height=7.0in, top=1.0in, papersize={5.5in,8.5in}]{geometry}
%\usepackage[pdftex]{graphicx}
%\usepackage{tipa}
%\usepackage{txfonts}
%\usepackage{textcomp}
%\usepackage{amsthm}
%\usepackage{xy}
%\usepackage{fancyhdr}
%\usepackage{xcolor}
%\usepackage{mathtools} %% for pretty underbrace % Breaks Ximera
%\usepackage{multicol}



\newcommand{\RR}{\mathbb R}
\newcommand{\R}{\mathbb R}
\newcommand{\C}{\mathbb C}
\newcommand{\N}{\mathbb N}
\newcommand{\Z}{\mathbb Z}
\newcommand{\dis}{\displaystyle}
%\renewcommand{\d}{\,d\!}
\renewcommand{\d}{\mathop{}\!d}
\newcommand{\dd}[2][]{\frac{\d #1}{\d #2}}
\newcommand{\pp}[2][]{\frac{\partial #1}{\partial #2}}
\renewcommand{\l}{\ell}
\newcommand{\ddx}{\frac{d}{\d x}}

\newcommand{\zeroOverZero}{\ensuremath{\boldsymbol{\tfrac{0}{0}}}}
\newcommand{\inftyOverInfty}{\ensuremath{\boldsymbol{\tfrac{\infty}{\infty}}}}
\newcommand{\zeroOverInfty}{\ensuremath{\boldsymbol{\tfrac{0}{\infty}}}}
\newcommand{\zeroTimesInfty}{\ensuremath{\small\boldsymbol{0\cdot \infty}}}
\newcommand{\inftyMinusInfty}{\ensuremath{\small\boldsymbol{\infty - \infty}}}
\newcommand{\oneToInfty}{\ensuremath{\boldsymbol{1^\infty}}}
\newcommand{\zeroToZero}{\ensuremath{\boldsymbol{0^0}}}
\newcommand{\inftyToZero}{\ensuremath{\boldsymbol{\infty^0}}}


\newcommand{\numOverZero}{\ensuremath{\boldsymbol{\tfrac{\#}{0}}}}
\newcommand{\dfn}{\textbf}
%\newcommand{\unit}{\,\mathrm}
\newcommand{\unit}{\mathop{}\!\mathrm}
%\newcommand{\eval}[1]{\bigg[ #1 \bigg]}
\newcommand{\eval}[1]{ #1 \bigg|}
\newcommand{\seq}[1]{\left( #1 \right)}
\renewcommand{\epsilon}{\varepsilon}
\renewcommand{\iff}{\Leftrightarrow}

\DeclareMathOperator{\arccot}{arccot}
\DeclareMathOperator{\arcsec}{arcsec}
\DeclareMathOperator{\arccsc}{arccsc}
\DeclareMathOperator{\si}{Si}
\DeclareMathOperator{\proj}{proj}
\DeclareMathOperator{\scal}{scal}
\DeclareMathOperator{\cis}{cis}
\DeclareMathOperator{\Arg}{Arg}
%\DeclareMathOperator{\arg}{arg}
\DeclareMathOperator{\Rep}{Re}
\DeclareMathOperator{\Imp}{Im}
\DeclareMathOperator{\sech}{sech}
\DeclareMathOperator{\csch}{csch}
\DeclareMathOperator{\Log}{Log}

\newcommand{\tightoverset}[2]{% for arrow vec
  \mathop{#2}\limits^{\vbox to -.5ex{\kern-0.75ex\hbox{$#1$}\vss}}}
\newcommand{\arrowvec}{\overrightarrow}
\renewcommand{\vec}{\mathbf}
\newcommand{\veci}{{\boldsymbol{\hat{\imath}}}}
\newcommand{\vecj}{{\boldsymbol{\hat{\jmath}}}}
\newcommand{\veck}{{\boldsymbol{\hat{k}}}}
\newcommand{\vecl}{\boldsymbol{\l}}
\newcommand{\utan}{\vec{\hat{t}}}
\newcommand{\unormal}{\vec{\hat{n}}}
\newcommand{\ubinormal}{\vec{\hat{b}}}

\newcommand{\dotp}{\bullet}
\newcommand{\cross}{\boldsymbol\times}
\newcommand{\grad}{\boldsymbol\nabla}
\newcommand{\divergence}{\grad\dotp}
\newcommand{\curl}{\grad\cross}
%% Simple horiz vectors
\renewcommand{\vector}[1]{\left\langle #1\right\rangle}


\outcome{Determine series behavior based on a definite integral}

\title{3.5 Integral Test}

\begin{document}

\begin{abstract}
We determine the convergence or divergence of an infinite series using a related improper integral.
\end{abstract}

\maketitle

\section{Integral Test}

\begin{theorem}[Integral Test]
Suppose that the function $f(x)$ is positive, continuous, and decreasing on the interval $(1, \infty)$, and 
suppose that the terms of an infinite series
are given by the corresponding function values, i.e., $a_n = f(n)$. \\
Then the improper integral
\[
\int_1^\infty f(x) \; dx,
\]
and the associated infinite series
\[
\sum_{n=1}^\infty a_n
\]
both converge or both diverge. In other words, the behavior of the improper integral 
determines the behavior of the associated series.
\end{theorem}

\begin{remark}
It is sufficient for the function to be decreasing eventually, i.e., on an interval of the form $(a, \infty)$
for some number $a$.
\end{remark}

The rationale for the integral test is that both the improper integral and the the sum of an infinite series can be interpreted as the area under a graph.
Moreover, these areas will be comparable.
Hence, we can use the logic of the comparison theorem of improper integrals: ``less than convergent is convergent and greater 
than divergent is divergent". 
The two figures below show that if 
the function $f(x)$ satisfies the hypotheses of the integral test and if  $a_n = f(n)$, 
then we can use the behavior of the integral to make conclusions about the behavior of the series.

\begin{center}
\begin{tikzpicture}

\begin{axis}[axis x line=middle, axis y line= middle, xlabel={$x$}, ylabel={$y$}, xtick={1, 2, 3, 4, 5, 6},
xmin=0, xmax = 6.5, ymax = 1.3,legend pos= north east,
xticklabels={1, 2, 3, 4, 5, 6}, ymin = 0, ytick={1/4, 1/3, 1/2, 1}, yticklabels={$a_4$, $a_3$, $a_2$, $a_1$}, 
title={
\begin{tabular} {c} 
If the improper integral $\displaystyle{\int_1^\infty f(x) \, dx}$ diverges, \\
then so does the associated series  $\displaystyle{\sum_{n=1}^\infty a_n}$ 
\end{tabular}  
}]
%title={\begin{tabular} {c}The area of each rectangle is $a_n$ and \\ 
%$\displaystyle{\int_1^\infty f(x) dx < \sum_{n=1}^\infty a_n}$\end{tabular}}]

\addplot+[fill][mark=none, domain=1:6, color=blue, thick, fill=blue!25]{1/x} \closedcycle;

\addplot[mark=*,cyan] coordinates {(1,1)};


\legend{$y = f(x)$, $(n\text{,}a_n)$};

\addplot[mark=*,cyan] coordinates {(2,1/2)} ;
\addplot[mark=*,cyan] coordinates {(3,1/3)}; 

\addplot[mark=*,cyan] coordinates {(4,1/4)} ;
\addplot[mark=*,cyan] coordinates {(5,1/5)} ;
\addplot[mark=*,cyan] coordinates {(6,1/6)} ;
\addplot[thick, cyan] coordinates{(1, 0) (1, 1) (2,1) (2,0)};
\addplot[thick, cyan] coordinates{ (2,1/2) (3, 1/2)  (3,0)};
\addplot[thick, cyan] coordinates{(3, 1/3) (4, 1/3)  (4,0)};
\addplot[thick, cyan] coordinates{(4,0) (4, 1/4) (5, 1/4)  (5,0)};
\addplot[thick, cyan] coordinates{(5, 1/5) (6, 1/5)  (6,0)};
\end{axis}
\node at (3.5, -1) {Note that $a_n = f(n)$};
\node at (4.5,3) {$\displaystyle{\int_1^\infty f(x) \, dx \leq \sum_{n=1}^\infty a_n}$} ;	
\end{tikzpicture}
\hspace{.51 in}
\begin{tikzpicture}

\begin{axis}
[
axis x line=middle, axis y line= middle, xlabel={$x$}, ylabel={$y$}, xtick={1, 2, 3, 4, 5, 6},
xmin=0, xmax = 6.5, ymax = 1.3, legend pos= north east,
xticklabels={1, 2, 3, 4, 5, 6}, ymin = 0, ytick={1/4, 1/3, 1/2, 1}, yticklabels={$a_4$, $a_3$, $a_2$, $a_1$}, 
title={
\begin{tabular} {c} 
If the improper integral $\displaystyle{\int_1^\infty f(x) \, dx}$ converges, \\
then so does the associated series  $\displaystyle{\sum_{n=2}^\infty a_n}$ 
\end{tabular}  
}
]


%title={\begin{tabular} {c} The area of each rectangle is $a_n$ and \\ 
%$\displaystyle{ \int_1^\infty f(x) dx > \sum_{n=2}^\infty a_n}$\end{tabular}}]

\addplot[domain=1:6, color=blue, thick,name path=f, ]{1/x};
fill between[
        of=f and axis,
    ];


\addplot[mark=*,cyan] coordinates {(2,1/2)} ;

\legend{$y = f(x)$, $(n\text{,}a_n)$};

\addplot[mark=*,cyan] coordinates {(1,1)};
\addplot[mark=*,cyan] coordinates {(3,1/3)} ;
\addplot[mark=*,cyan] coordinates {(4,1/4)} ;
\addplot[mark=*,cyan] coordinates {(5,1/5)} ;
\addplot[mark=*,cyan] coordinates {(6,1/6)} ;
\addplot[thick, cyan] coordinates{(0, 0) (0, 1) (1,1) (1,0)};
\addplot[thick, cyan, fill=cyan!25!white] coordinates{(1,0) (1,1/2) (2, 1/2)  (2,0)};
\addplot[thick, cyan, fill=cyan!25!white] coordinates{(2,0) (2, 1/3) (3, 1/3)  (3,0)};
\addplot[thick, cyan, fill=cyan!25!white] coordinates{(3,0) (3, 1/4) (4, 1/4)  (4,0)};
\addplot[thick, cyan, fill=cyan!25!white] coordinates{(4,0) (4, 1/5) (5, 1/5)  (5,0)};
\addplot[thick, cyan, fill=cyan!25!white] coordinates{(5,0) (5, 1/6) (6, 1/6)  (6,0)};
\end{axis}
\node at (4.5,3) {$\displaystyle{\int_1^\infty f(x) \, dx \geq \sum_{n=2}^\infty a_n}$} ;	
\node at (3.5, -1) {Note that $a_n = f(n)$};
\end{tikzpicture}
\end{center}


\begin{example}[example 1]
Determine whether the series
\[
\sum_{n=0}^\infty \frac{1}{1+n^2}
\]
converges or diverges.\\
Consider the function, 
\[
f(x) = \frac{1}{1+x^2}
\]
Note that $a_n = f(n)$.
This function $f(x)$ is positive and continuous on the interval $(0, \infty)$.
To check that $f(x)$ is decreasing, we need to verify that $f'(x) < 0$ on the interval $(0, \infty)$.
Recall that if the derivative is negative then the function is decreasing.
By the quotient rule, 
\[
f'(x) = -\frac{2x}{(1+x^2)^2}.
\]
which is negative on the interval $(0, \infty)$. Thus $f(x)$ is decreasing on this interval.
Having verified that the conditions of the Integral Test are met, we compute the improper integral:
\begin{align*}
\int_0^\infty \frac{1}{1+x^2} \; dx &= \lim_{b \to \infty} \int_0^b \frac{1}{1+x^2} \; dx\\
&= \lim_{b \to \infty} \tan^{-1}(x) \bigg|_0^b \\
&= \lim_{b \to \infty} \left[\tan^{-1}(b) - \tan^{-1}(0)\right]\\
&= \lim_{b \to \infty} \left[\tan^{-1}(b)\right]\\
&= \frac{\pi}{2}.
\end{align*}
The improper integral converges, so by the Integral Test, the associated infinite series
\[
\sum_{n=0}^\infty \frac{1}{1+n^2}
\]
also converges.
\end{example}



\begin{problem}(problem 1a)
Use the integral test to determine whether the infinite series
\[
\sum_{n=1}^\infty \frac{1}{n^2}
\]
converges or diverges.\\

Let $f(x) = \answer{ 1/x^2 }$\\

Is $f(x)$ positive and continuous on the interval $(1, \infty)$? \wordChoice{\choice[correct]{Yes}
\choice{No}}


The derivative, $f'(x) = \answer{-2/x^3}$\\

Is $f(x)$ decreasing on the interval $(1, \infty)$? \wordChoice{\choice[correct]{Yes}
\choice{No}}


Evaluate the improper integral: 
\[
 \int_1^\infty \frac{1}{x^2} \, dx = \answer{1}
\]

Does the infinite series
\[
\sum_{n=1}^\infty  \frac{1}{n^2}
\]
converge or diverge? \wordChoice{\choice[correct]{Converge}
\choice{Diverge}}


\end{problem}



\begin{problem}(problem 1b)
Use the integral test to determine whether the infinite series
\[
\sum_{n=0}^\infty \frac{1}{\sqrt{n+1}}
\]
converges or diverges.\\

Let $f(x) = \answer{ 1/\sqrt{x+1} }$\\

Is $f(x)$ positive and continuous on the interval $(0, \infty)$?\wordChoice{\choice[correct]{Yes}
\choice{No}}\\


The derivative, $f'(x) = \answer{-\frac12 (x+1)^{-3/2}}$\\

Is $f(x)$ decreasing on the interval $(0, \infty)$? \wordChoice{\choice[correct]{Yes}
\choice{No}}\\


Evaluate the improper integral: 
\[
 \int_0^\infty \frac{1}{\sqrt{x+1}} \, dx = \answer{\infty}
\]

Does the infinite series
\[
\sum_{n=0}^\infty \frac{1}{\sqrt{n+1}}
\]
converge or diverge?\wordChoice{\choice{Converge}
\choice[correct]{Diverge}}


\end{problem}



\begin{example}[example 2]
Determine whether the series
\[
\sum_{n=2}^\infty \frac{\ln(n)}{n}
\]
converges or diverges.\\
Consider the function
\[
f(x) = \frac{\ln(x)}{x},
\]
so that $a_n = f(n)$.
This function is positive and continuous on the interval $(2, \infty)$.
To see that the function is decreasing, compute the derivative:
\[
f'(x) = \frac{x\cdot\frac{1}{x}- \ln(x)}{x^2} = \frac{1-\ln(x)}{x^2},
\]
which is negative if $x > e$. Hence $f(x)$ is eventually decreasing and we can use the integral test:

\begin{align*}
\int_2^\infty \frac{\ln(x)}{x} \; dx &= \lim_{t \to \infty} \int_2^t \frac{\ln(x)}{x} \; dx\\
&= \lim_{t \to \infty} \int_{\ln(2)}^{\ln(t)} u \, du \;\; \left(\text{u-sub with $u = \ln(x), du = \frac{1}{x} dx$}\right)\\
&= \lim_{t \to \infty} \frac{u^2}{2} \bigg|_{\ln(2)}^{\ln(t)}\\
&= \lim_{t \to \infty} \left(\ln^2(t) - \ln^2(2) \right)\\
&= \infty.
\end{align*}
The improper integral diverges, so by the Integral Test, the associated infinite series
\[
\sum_{n=2}^\infty \frac{\ln(n)}{n}
\]
also diverges.
\end{example}



\begin{problem}(problem 2a)
Use the integral test to determine whether the infinite series
\[
\sum_{n=2}^\infty \frac{1}{n\ln(n)}
\]
converges or diverges.\\

Let $f(x) = \answer{1/(x\ln(x))}$\\

Is $f(x)$ positive and continuous on the interval $(2, \infty)$?\wordChoice{\choice[correct]{Yes}
\choice{No}}
\\ %wordChoice?

The derivative, $f'(x) = \answer{-\frac{1+\ln(x)}{x^2 \ln^2(x)}}$\\

Is $f(x)$ decreasing on the interval $(2, \infty)$? \wordChoice{\choice[correct]{Yes}
\choice{No}}\\

Evaluate the improper integral: 
\[
 \int_2^\infty \frac{1}{x\ln(x)} = \answer{\infty}
\]

Does the infinite series
\[
\sum_{n=2}^\infty \frac{1}{n\ln(n)}
\]
converge or diverge? \wordChoice{\choice{Converge}
\choice[correct]{Diverge}}


\end{problem}



\begin{problem}(problem 2b)
Use the integral test to determine whether the infinite series
\[
\sum_{n=1}^\infty \frac{n}{e^n}
\]
converges or diverges.\\

Let $f(x) = \answer{xe^{-x}}$\\

Is $f(x)$ positive and continuous on the interval $(1, \infty)$?\wordChoice{\choice[correct]{Yes}
\choice{No}}\\ %wordChoice?

The derivative, $f'(x) = \answer{e^{-x}(1-x)}$\\

Is $f(x)$ decreasing on the interval $(1, \infty)$? \wordChoice{\choice[correct]{Yes}
\choice{No}}\\


Evaluate the improper integral (hint, use integration by parts): 
\[
 \int_1^\infty \frac{x}{e^x} = \answer{2/e}
\]

Does the infinite series
\[
\sum_{n=1}^\infty \frac{n}{e^n}
\]
converge or diverge? \wordChoice{\choice[correct]{Converge}
\choice{Diverge}}


\end{problem}


\begin{definition}[p-series]
Let $p > 0$. The infinite series
\[
\sum_{n=1}^\infty \frac{1}{n^p}
\]
is called a \textbf{$p$-series}. In the special case $p = 1$, the series is known as the \textbf{harmonic series}.
\end{definition}

We have seen in the section on improper integrals that the $p$-integral

\[
\int_1^\infty \frac{1}{x^p} \; dx.
\]
converges if $p > 1$ and diverges if $p \leq 1$.
Since the function $f(x) = 1/x^p \; (p > 0)$ is continuous, positive, and decreasing on the interval $(1, \infty)$,
the integral test can be applied to conclude that the associated $p$-series has the same behavior.  This is stated in the following theorem.


\begin{theorem}[$p$-series]
The $p$-series
\[
\sum_{n=1}^\infty \frac{1}{n^p}
\]
converges if $p > 1$ and diverges if $p \leq 1$.
\end{theorem}

\begin{remark}
If the series starts at some number $n >1$, then we shall still refer to the 
series as a $p$-series, and the behavior is the same as if it started at $n=1$.
\end{remark}

\begin{remark}
A non-zero multiple of a $p$-series has the same behavior as the $p$-series.
\end{remark}

\section{Video Lesson}

\begin{center}
\begin{foldable}
\unfoldable{Here is a detailed, lecture style video on the Integral Test:}
\youtube{db7NqcbCO4w}
\end{foldable}
\end{center}



\end{document}




