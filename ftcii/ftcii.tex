\documentclass[handout]{ximera}

%% You can put user macros here
%% However, you cannot make new environments



\newcommand{\ffrac}[2]{\frac{\text{\footnotesize $#1$}}{\text{\footnotesize $#2$}}}
\newcommand{\vasymptote}[2][]{
    \draw [densely dashed,#1] ({rel axis cs:0,0} -| {axis cs:#2,0}) -- ({rel axis cs:0,1} -| {axis cs:#2,0});
}


\graphicspath{{./}{firstExample/}}
\usepackage{forest}
\usepackage{amsmath}
\usepackage{amssymb}
\usepackage{array}
\usepackage[makeroom]{cancel} %% for strike outs
\usepackage{pgffor} %% required for integral for loops
\usepackage{tikz}
\usepackage{tikz-cd}
\usepackage{tkz-euclide}
\usetikzlibrary{shapes.multipart}


%\usetkzobj{all}
\tikzstyle geometryDiagrams=[ultra thick,color=blue!50!black]


\usetikzlibrary{arrows}
\tikzset{>=stealth,commutative diagrams/.cd,
  arrow style=tikz,diagrams={>=stealth}} %% cool arrow head
\tikzset{shorten <>/.style={ shorten >=#1, shorten <=#1 } } %% allows shorter vectors

\usetikzlibrary{backgrounds} %% for boxes around graphs
\usetikzlibrary{shapes,positioning}  %% Clouds and stars
\usetikzlibrary{matrix} %% for matrix
\usepgfplotslibrary{polar} %% for polar plots
\usepgfplotslibrary{fillbetween} %% to shade area between curves in TikZ



%\usepackage[width=4.375in, height=7.0in, top=1.0in, papersize={5.5in,8.5in}]{geometry}
%\usepackage[pdftex]{graphicx}
%\usepackage{tipa}
%\usepackage{txfonts}
%\usepackage{textcomp}
%\usepackage{amsthm}
%\usepackage{xy}
%\usepackage{fancyhdr}
%\usepackage{xcolor}
%\usepackage{mathtools} %% for pretty underbrace % Breaks Ximera
%\usepackage{multicol}



\newcommand{\RR}{\mathbb R}
\newcommand{\R}{\mathbb R}
\newcommand{\C}{\mathbb C}
\newcommand{\N}{\mathbb N}
\newcommand{\Z}{\mathbb Z}
\newcommand{\dis}{\displaystyle}
%\renewcommand{\d}{\,d\!}
\renewcommand{\d}{\mathop{}\!d}
\newcommand{\dd}[2][]{\frac{\d #1}{\d #2}}
\newcommand{\pp}[2][]{\frac{\partial #1}{\partial #2}}
\renewcommand{\l}{\ell}
\newcommand{\ddx}{\frac{d}{\d x}}

\newcommand{\zeroOverZero}{\ensuremath{\boldsymbol{\tfrac{0}{0}}}}
\newcommand{\inftyOverInfty}{\ensuremath{\boldsymbol{\tfrac{\infty}{\infty}}}}
\newcommand{\zeroOverInfty}{\ensuremath{\boldsymbol{\tfrac{0}{\infty}}}}
\newcommand{\zeroTimesInfty}{\ensuremath{\small\boldsymbol{0\cdot \infty}}}
\newcommand{\inftyMinusInfty}{\ensuremath{\small\boldsymbol{\infty - \infty}}}
\newcommand{\oneToInfty}{\ensuremath{\boldsymbol{1^\infty}}}
\newcommand{\zeroToZero}{\ensuremath{\boldsymbol{0^0}}}
\newcommand{\inftyToZero}{\ensuremath{\boldsymbol{\infty^0}}}


\newcommand{\numOverZero}{\ensuremath{\boldsymbol{\tfrac{\#}{0}}}}
\newcommand{\dfn}{\textbf}
%\newcommand{\unit}{\,\mathrm}
\newcommand{\unit}{\mathop{}\!\mathrm}
%\newcommand{\eval}[1]{\bigg[ #1 \bigg]}
\newcommand{\eval}[1]{ #1 \bigg|}
\newcommand{\seq}[1]{\left( #1 \right)}
\renewcommand{\epsilon}{\varepsilon}
\renewcommand{\iff}{\Leftrightarrow}

\DeclareMathOperator{\arccot}{arccot}
\DeclareMathOperator{\arcsec}{arcsec}
\DeclareMathOperator{\arccsc}{arccsc}
\DeclareMathOperator{\si}{Si}
\DeclareMathOperator{\proj}{proj}
\DeclareMathOperator{\scal}{scal}
\DeclareMathOperator{\cis}{cis}
\DeclareMathOperator{\Arg}{Arg}
%\DeclareMathOperator{\arg}{arg}
\DeclareMathOperator{\Rep}{Re}
\DeclareMathOperator{\Imp}{Im}
\DeclareMathOperator{\sech}{sech}
\DeclareMathOperator{\csch}{csch}
\DeclareMathOperator{\Log}{Log}

\newcommand{\tightoverset}[2]{% for arrow vec
  \mathop{#2}\limits^{\vbox to -.5ex{\kern-0.75ex\hbox{$#1$}\vss}}}
\newcommand{\arrowvec}{\overrightarrow}
\renewcommand{\vec}{\mathbf}
\newcommand{\veci}{{\boldsymbol{\hat{\imath}}}}
\newcommand{\vecj}{{\boldsymbol{\hat{\jmath}}}}
\newcommand{\veck}{{\boldsymbol{\hat{k}}}}
\newcommand{\vecl}{\boldsymbol{\l}}
\newcommand{\utan}{\vec{\hat{t}}}
\newcommand{\unormal}{\vec{\hat{n}}}
\newcommand{\ubinormal}{\vec{\hat{b}}}

\newcommand{\dotp}{\bullet}
\newcommand{\cross}{\boldsymbol\times}
\newcommand{\grad}{\boldsymbol\nabla}
\newcommand{\divergence}{\grad\dotp}
\newcommand{\curl}{\grad\cross}
%% Simple horiz vectors
\renewcommand{\vector}[1]{\left\langle #1\right\rangle}


\outcome{Compute the derivative of an area function.}

\title{4.9 FTC, part II}




\begin{document}

\begin{abstract}
In this section we learn the second part of the fundamental theorem and we use it to compute the derivative of an area function.
\end{abstract}

\maketitle



\section{The Fundamental Theorem of Calculus, Part II}

\subsection{Area Functions}
\begin{definition}[Area Function]
Let $f(t)$ be a continuous function on the interval $[a,b]$ and let $x$ be a number in the interval. Then, an \textbf{area function} is a 
function of the form
\[A(x) = \int_a^x f(t) \ dt.\]
\end{definition}
An area function gives us the \textbf{net area} between the curve $y = f(t)$ and the $t$-axis over the interval $[a,x]$. Net area means the area under the curve (where the function, $f(t)$, is positive)  minus the area above the curve (where the function, $f(t)$, is negative). 
%xlabel=$t$


\begin{image}
\begin{tikzpicture}
\begin{axis}[title={A(x) = area of region I - area of region II},axis x line=  center, axis y line = left, xtick={0, 4.71},  xticklabels={$a$, $b$}, ytick={-2, -1, 0, 1, 2}]
\addplot[name path = A, domain=0:1.57, 
    samples=100, color=black]{2*cos(deg(x))};
\addplot[name path = C, domain=1.57:3.5, 
    samples=100, color=black]{2*cos(deg(x))};
\addplot[domain=3.14:4.71, 
    samples=100, color=black]{2*cos(deg(x))};

\addplot[name path =B,domain=0:3.5, samples = 100, color = black]{0};
\addplot[name path =D,domain=3.14:4.71, samples = 100, color = black]{0};
\addplot[domain=-1:0, samples = 100, color = black]{0};
\addplot[domain=4.71:6, samples = 100, color = black]{0};
\addplot[domain=6:6.15, samples = 100, color = white]{-.2};
\addplot[blue!10] fill between[of=A and B];
\addplot[red!10] fill between[of=C and B];
\node at (axis cs: 6.1,-.15){$t$};

\node at (axis cs: 2, 1){$y = f(t)$};
\node at (axis cs: .6, .8){I};
\node at (axis cs: 2.7, -.8){II};

\addplot[thin, samples = 100, color=black] coordinates {(0,0) (0,2)};
\addplot[thin, samples = 100, color=black] coordinates { (3.5,-1.87)(3.5,0)} node[above] {$x$};

\end{axis}
\end{tikzpicture}
\end{image}

\begin{problem}(problem 1a)
Use the graph of $y = f(t)$ to find the values of the area function
\[A(x)= \int_0^x f(t) \ dt.\]
%xscale=8, yscale =5,
\begin{image}
\begin{tikzpicture}
\begin{axis}[ axis x line=  center, axis y line = left, xtick={0, 1, 3, 5, 7},  xticklabels={0, 1, 3, 5, 7}, ytick={-2, -1, 0, 1, 2}]
\addplot[thick, domain=0:1, 
    samples=100, color=black]{2};
\addplot[thick, domain=1:5, 
    samples=100, color=black]{3-x};
\addplot[thick, domain=5:7, 
    samples=100, color=black]{x-7};

\addplot[domain=7:8, samples = 100, color = white]{-.2};
\node at (axis cs: 7.95,-.2){$t$};

\node at (axis cs: 2.75, 1.2){$y = f(t)$};
\addplot[-] coordinates {(0,-2.5) (0,2.5)};
%\addplot[dashed] coordinates {(1,0) (1, 2)};
%\addplot[dashed, samples = 100, color=black] coordinates {(5,0) (5,-2)};

\end{axis}
\end{tikzpicture}
\end{image}


\[A(1) = \int_0^1 f(t) \ dt = \answer{2},\]
\[A(3) = \int_0^3 f(t) \ dt = \answer{4},\]
\[A(5) = \int_0^5 f(t) \ dt = \answer{2},\]
\[A(7) = \int_0^7 f(t) \ dt = \answer{0}.\]

\end{problem}

\begin{problem}(problem 1b)
Use the graph of $y = f(t)$ to find the values of the area function
\[A(x)= \int_{-2}^x f(t) \ dt.\]

\begin{image}
\begin{tikzpicture}
\begin{axis}[ axis x line=  center, axis y line = center]
\addplot[thick, domain=-2:2, 
    samples=100, color=black]{(4 - x^2)^.5};
\addplot[thick, domain=2:4, 
    samples=100, color=black]{2-x};
\addplot[thick, domain=4:7, 
    samples=100, color=black]{x-6};
\addplot[-] coordinates {(0,-2.8) (0,2.8)};
\addplot[domain=7:7.95, samples = 100, color = white]{-.2};

\node at (axis cs: 7.9,-.3){$t$};

\node at (axis cs: 2.75, 1.2){$y = f(t)$};
%\addplot[dashed, samples = 100, color=black] coordinates {(7,1) (7,-.05)} node[below] {$7$};
%\addplot[dashed, samples = 100, color=black] coordinates {(4,0) (4, -2)};
\end{axis}

\end{tikzpicture}
\end{image}
\begin{hint}
From $-2$ to $2$ the curve is a semi-circle
\end{hint}
\[A(0) = \answer{\pi},\]
\[A(4) = \answer{2\pi - 2},\]
\[A(6) = \answer{2\pi -4},\]
\[A(7) = \answer{2\pi - 3.5}.\]

\end{problem}


\subsection{FTC, Part II}

The second part of the FTC tells us the derivative of an area function $A(x)$.  


\begin{theorem}[Fundamental Theorem of Calculus, Part II]
If $f(x)$ is continuous on the closed interval $[a, b]$ and let $A(x)$ be the area function,
\[
A(x) = \int_a^x f(t) \, dt
\]
then 
\[A'(x) = \frac{d}{dx}\int_a^x f(t) \, dt = f(x), \]
for any value of $x$ in the interval $[a, b]$.
\end{theorem}

%Its remarkable conclusion states that the integrand itself is the derivative in question.  
This conclusion establishes the existence of anti-derivatives, i.e., 
by the FTC part II, every continuous function $f(x)$ has an anti-derivative. 




\begin{example}[example 2]
 Let $f(x) = \sin(x^2)$.  This function is continuous for all $x$.  Consider the function
\[A(x) = \int_a^x \sin(t^2) \ dt.\]
By the FTC part II
we can say that for any number $a$ and any value of $x$, 
\[\frac{d}{dx}\int_a^x \sin(t^2) \ dt = \sin(x^2).\]
This example asserts that the continuous function $f(x) = \sin(x^2)$ has an anti-derivative.  
\end{example}



\begin{remark}
Finding a familiar form of $A(x)$ in the example above is an impossible task, despite the simple nature of the integrand.  
In general, the theory of anti-differentiation is much richer than the theory of differentiation. 
We can use the chain rule to find the derivative of $\sin(x^2)$ but finding an anti-derivative requires us to resort to 
using a new type of function, defined in terms of a definite integral.
\end{remark}


\begin{example}[example 3]
Find the derivative of the function
\[A(x) = \int_0^x \cos(t) \ dt.\]

By the FTC part II,
\[\frac{d}{dx}\int_0^x \cos(t) \ dt = \cos(x).\]
 
Note that this equation can be verified easily by computing the integral and then taking the derivative of the result:
\[\int_0^x \cos(t) \ dt = \sin(t) \Big|_0^x = \sin(x) - \sin(0) = \sin(x)\]
and 
\[\frac{d}{dx} \sin(x) = \cos(x).\]
%the derivative of $\sin(x)$ is $\cos(x)$.  
%Note also in this example that the function $\int_0^x \cos(t) \ dt$
%has another, familiar form, namely $\sin(x)$.
\end{example}


\begin{problem}(problem 3a)
\[\frac{d}{dx} \int_0^x \sec^3(t) \ dt = \answer{\sec^3(x)}.\]
\end{problem}

\begin{problem}(problem 3b)
\[\frac{d}{dx} \int_1^x \ln^2(t) \ dt = \answer{\ln^2(x)}.\]
\end{problem}


\begin{example}[example 4]
Find the derivative of the function 
\[A(x) = \int_1^x e^{2t}\sqrt{1+t^3} \ dt.\]

By the FTC part II,

 \[A'(x) = \frac{d}{dx}\int_1^x e^{2t}\sqrt{1+t^3} \ dt = e^{2x}\sqrt{1+x^3}.\]
\end{example} 

\begin{example}[example 5]
Find the derivative of the function 
\[A(x) = \int_x^0 f(t) \ dt.\]
First, we rewrite the integral as
\[\int_x^0 f(t) \ dt =-\int_0^x f(t) \ dt.\]
Then, by the FTC part II,
\[\frac{d}{dx}\int_x^0 f(t) \ dt = -\frac{d}{dx}\int_0^x f(t) \ dt =-f(x).\] 
\end{example}

\begin{problem}(problem 5)
\[\frac{d}{dx} \int_x^0 e^t\cos(t) \ dt = \answer{-e^x\cos(x)}.\]
\end{problem}

  
\begin{example}[example 6]
Find the following derivative:
\[\frac{d}{dx} \int_0^{x^2} e^{2t}\sqrt{1+t^3} \ dt.\]
To solve this problem, we will need to use the chain rule. Let 
\[A(x) = \int_0^x e^{2t}\sqrt{1+t^3} \ dt.\]
Then the function we wish to differentiate is $A(x^2)$.
By the chain rule, 
\[\frac{d}{dx} A(x^2) = A'(x^2) \cdot 2x.\]
By FTC, part II,
\[A'(x) = e^{2x}\sqrt{1+x^3}\]
and substituting $x^2$ for $x$, we have
\[A'(x^2) = e^{2x^2}\sqrt{1+(x^2)^3} = e^{2x^2}\sqrt{1+x^6}.\]
Now multiplying by $2x$ gives the final answer:
\[\frac{d}{dx}\int_0^{x^2} e^{2t}\sqrt{1+t^3} \ dt  = 
2xe^{2x^2}\sqrt{1+x^6}.\]
\end{example}

\begin{example}[example 7]
Find the following derivative:
\[\frac{d}{dx} \int_{2x}^0 \tan(3t) \ln(t^2) \ dt.\]
To solve this problem, we will need to use the chain rule. Let 
\[A(x) = \int_0^x \tan(3t) \ln(t^2) \ dt.\]
Then the function we wish to differentiate is $-A(2x)$.
Note that the negative sign comes from switching the endpoints of integration.
By the chain rule, 
\[\frac{d}{dx} A(2x) = A'(2x) \cdot 2.\]
By FTC, part II,
\[A'(x) = \tan(3x) \ln(x^2)\]
and substituting $2x$ for $x$, we have
\[A'(2x) = \tan(3(2x)) \ln((2x)^2) = \tan(6x) \ln(4x^2).\]
Now multiplying by $-2$ gives the final answer:
\[\frac{d}{dx} \int_{2x}^0 \tan(3t) \ln(t^2) \ dt  = -2\tan(6x)\ln(4x^2).\]
\end{example}

\begin{problem}(problem 7a)
\[\frac{d}{dx} \int_0^{x^3} t^4\cos(t) \ dt = \answer{3x^{14}\cos(x^3)}.\]
\end{problem}

\begin{problem}(problem 7b)
\[\frac{d}{dx} \int_{3x}^0 t^2\sin(t) \ dt = \answer{-27x^2\sin(3x)}.\]
\end{problem}


\begin{center}
\begin{foldable}
\unfoldable{Here are some detailed, lecture style videos on area functions:}
\youtube{EoN09fzPogM}
\youtube{Im85myPA4-E}
\end{foldable}
\end{center}






\end{document}







