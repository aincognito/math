\documentclass[handout]{ximera}
\usepackage{tcolorbox}
%% You can put user macros here
%% However, you cannot make new environments



\newcommand{\ffrac}[2]{\frac{\text{\footnotesize $#1$}}{\text{\footnotesize $#2$}}}
\newcommand{\vasymptote}[2][]{
    \draw [densely dashed,#1] ({rel axis cs:0,0} -| {axis cs:#2,0}) -- ({rel axis cs:0,1} -| {axis cs:#2,0});
}


\graphicspath{{./}{firstExample/}}
\usepackage{forest}
\usepackage{amsmath}
\usepackage{amssymb}
\usepackage{array}
\usepackage[makeroom]{cancel} %% for strike outs
\usepackage{pgffor} %% required for integral for loops
\usepackage{tikz}
\usepackage{tikz-cd}
\usepackage{tkz-euclide}
\usetikzlibrary{shapes.multipart}


%\usetkzobj{all}
\tikzstyle geometryDiagrams=[ultra thick,color=blue!50!black]


\usetikzlibrary{arrows}
\tikzset{>=stealth,commutative diagrams/.cd,
  arrow style=tikz,diagrams={>=stealth}} %% cool arrow head
\tikzset{shorten <>/.style={ shorten >=#1, shorten <=#1 } } %% allows shorter vectors

\usetikzlibrary{backgrounds} %% for boxes around graphs
\usetikzlibrary{shapes,positioning}  %% Clouds and stars
\usetikzlibrary{matrix} %% for matrix
\usepgfplotslibrary{polar} %% for polar plots
\usepgfplotslibrary{fillbetween} %% to shade area between curves in TikZ



%\usepackage[width=4.375in, height=7.0in, top=1.0in, papersize={5.5in,8.5in}]{geometry}
%\usepackage[pdftex]{graphicx}
%\usepackage{tipa}
%\usepackage{txfonts}
%\usepackage{textcomp}
%\usepackage{amsthm}
%\usepackage{xy}
%\usepackage{fancyhdr}
%\usepackage{xcolor}
%\usepackage{mathtools} %% for pretty underbrace % Breaks Ximera
%\usepackage{multicol}



\newcommand{\RR}{\mathbb R}
\newcommand{\R}{\mathbb R}
\newcommand{\C}{\mathbb C}
\newcommand{\N}{\mathbb N}
\newcommand{\Z}{\mathbb Z}
\newcommand{\dis}{\displaystyle}
%\renewcommand{\d}{\,d\!}
\renewcommand{\d}{\mathop{}\!d}
\newcommand{\dd}[2][]{\frac{\d #1}{\d #2}}
\newcommand{\pp}[2][]{\frac{\partial #1}{\partial #2}}
\renewcommand{\l}{\ell}
\newcommand{\ddx}{\frac{d}{\d x}}

\newcommand{\zeroOverZero}{\ensuremath{\boldsymbol{\tfrac{0}{0}}}}
\newcommand{\inftyOverInfty}{\ensuremath{\boldsymbol{\tfrac{\infty}{\infty}}}}
\newcommand{\zeroOverInfty}{\ensuremath{\boldsymbol{\tfrac{0}{\infty}}}}
\newcommand{\zeroTimesInfty}{\ensuremath{\small\boldsymbol{0\cdot \infty}}}
\newcommand{\inftyMinusInfty}{\ensuremath{\small\boldsymbol{\infty - \infty}}}
\newcommand{\oneToInfty}{\ensuremath{\boldsymbol{1^\infty}}}
\newcommand{\zeroToZero}{\ensuremath{\boldsymbol{0^0}}}
\newcommand{\inftyToZero}{\ensuremath{\boldsymbol{\infty^0}}}


\newcommand{\numOverZero}{\ensuremath{\boldsymbol{\tfrac{\#}{0}}}}
\newcommand{\dfn}{\textbf}
%\newcommand{\unit}{\,\mathrm}
\newcommand{\unit}{\mathop{}\!\mathrm}
%\newcommand{\eval}[1]{\bigg[ #1 \bigg]}
\newcommand{\eval}[1]{ #1 \bigg|}
\newcommand{\seq}[1]{\left( #1 \right)}
\renewcommand{\epsilon}{\varepsilon}
\renewcommand{\iff}{\Leftrightarrow}

\DeclareMathOperator{\arccot}{arccot}
\DeclareMathOperator{\arcsec}{arcsec}
\DeclareMathOperator{\arccsc}{arccsc}
\DeclareMathOperator{\si}{Si}
\DeclareMathOperator{\proj}{proj}
\DeclareMathOperator{\scal}{scal}
\DeclareMathOperator{\cis}{cis}
\DeclareMathOperator{\Arg}{Arg}
%\DeclareMathOperator{\arg}{arg}
\DeclareMathOperator{\Rep}{Re}
\DeclareMathOperator{\Imp}{Im}
\DeclareMathOperator{\sech}{sech}
\DeclareMathOperator{\csch}{csch}
\DeclareMathOperator{\Log}{Log}

\newcommand{\tightoverset}[2]{% for arrow vec
  \mathop{#2}\limits^{\vbox to -.5ex{\kern-0.75ex\hbox{$#1$}\vss}}}
\newcommand{\arrowvec}{\overrightarrow}
\renewcommand{\vec}{\mathbf}
\newcommand{\veci}{{\boldsymbol{\hat{\imath}}}}
\newcommand{\vecj}{{\boldsymbol{\hat{\jmath}}}}
\newcommand{\veck}{{\boldsymbol{\hat{k}}}}
\newcommand{\vecl}{\boldsymbol{\l}}
\newcommand{\utan}{\vec{\hat{t}}}
\newcommand{\unormal}{\vec{\hat{n}}}
\newcommand{\ubinormal}{\vec{\hat{b}}}

\newcommand{\dotp}{\bullet}
\newcommand{\cross}{\boldsymbol\times}
\newcommand{\grad}{\boldsymbol\nabla}
\newcommand{\divergence}{\grad\dotp}
\newcommand{\curl}{\grad\cross}
%% Simple horiz vectors
\renewcommand{\vector}[1]{\left\langle #1\right\rangle}

\outcome{Compute derivatives involving the basic trig functions.}


\title{2.4 Derivatives of Sine and Cosine}


\begin{document}

\begin{abstract}
In this section we compute derivatives involving $\sin(x)$ and $\cos(x)$.
\end{abstract}

\maketitle

%add in 4th derivatives of sin, cos and 51st derivatives too

We begin by computing the derivative of the trigonometric function $f(x) = \sin(x)$.  
Two key trigonometric identities will be needed:
\begin{enumerate}
\item The Pythagorean Identity:
\[
\sin^2(\theta) + \cos^2(\theta) = 1,
\]
\begin{center}
and
\end{center}
\item The Sum Identity:
\[
\sin(\alpha + \beta) = \sin(\alpha)\cos(\beta) + \cos(\alpha)\sin(\beta).
\]
\end{enumerate}

We will also need the following limit, previously discussed in the Numerical Limits section and proved in the Squeeze Theorem section:
\[
\lim_{\theta \to 0} \frac{\sin(\theta)}{\theta} = 1.
\]

We begin by using  the Pythagorean Identity and the above limit
to compute a second important limit involving the cosine function:

\begin{align*}
\lim_{\theta \to 0} \frac{1 - \cos(\theta)}{\theta} &= \lim_{\theta \to 0} \frac{1 - \cos(\theta)}{\theta} \cdot \frac{1 + \cos(\theta)}{1 + \cos(\theta)}\\
&= \lim_{\theta \to 0} \frac{1 - \cos^2(\theta)}{\theta[1 + \cos(\theta)]}\\
&= \lim_{\theta \to 0} \frac{\sin^2(\theta)}{\theta[1 + \cos(\theta)]}\\
&= \lim_{\theta \to 0} \frac{\sin(\theta)}{\theta} \cdot \frac{\sin(\theta)}{1 + \cos(\theta)}\\
&= 1 \cdot \frac{0}{2} = 0.
\end{align*}



\begin{explanation} %begin{derivation}
If $f(x) = \sin(x)$ then
\begin{center}
$\begin{aligned}
%\begin{align*}
f'(x) &= \lim_{h \to 0} \frac{f(x+h)-f(x)}{h} \\[5pt]
&= \lim_{h \to 0} \frac{\sin(x+h) - \sin(x)}{h}\\[5pt]
&=  \lim_{h \to 0} \frac{\sin(x)\cos(h) + \cos(x)\sin(h) - \sin(x)}{h}\\[5pt]
&=  \lim_{h \to 0} \frac{\sin(x)\cos(h)  - \sin(x) + \cos(x)\sin(h)}{h}\\[5pt]
&=  \lim_{h \to 0} \frac{\sin(x)[\cos(h) -1] + \cos(x)\sin(h)}{h}\\[5pt]
&=  \lim_{h \to 0} \frac{\sin(x)[\cos(h) -1]}{h} + \frac{\cos(x)\sin(h)}{h}\\[5pt]
&=  \sin(x) \cdot 0 + \cos(x) \cdot 1 \\[5pt]
&= \cos(x).
%\end{align*}
\end{aligned}$
\end{center}
Thus the derivative of sine is cosine:
\[
\frac{d}{dx}\sin(x) = \cos(x).
\]
In terms of the variables $\theta$ and $u$:
\[
\frac{d}{d\theta}\sin(\theta) = \cos(\theta) \text{  and  } \frac{d}{du}\sin(u) = \cos(u).
\]
\end{explanation}



\begin{problem}(problem 1)

Find the equation of the tangent line to the graph of $f(x) = \sin(x)$ at $x=0.$

\begin{hint}
The point of tangency is $(0, f(0))$
\end{hint}
\begin{hint}
Use the derivative to find the slope, $m$
\end{hint}
\begin{hint}
Point slope form: $y-y_0 = m(x-x_0)$
\end{hint}

The equation of the tangent line is \ $y = \answer{x}$

\end{problem}





\begin{problem}(problem 2)
  
Compute 
\[
\frac{d}{dx} \left[5\sin(x)\right]
\]
  
\begin{hint}
The derivative of $\sin(x)$ is $\cos(x)$
\end{hint}
\begin{hint}
Don't forget to multiply by 5
\end{hint}
		
The derivative of $5\sin(x)$ with respect to $x$ is
$\answer{5\cos(x)}$

\end{problem}


\begin{problem}(problem 3)

Find the equation of the tangent line to the graph of $f(x) = \sin(x)$ at $x=\pi/2.$

\begin{hint}
The point of tangency is $(\pi/2, f(\pi/2))$
\end{hint}
\begin{hint}
Use the derivative to find the slope, $m$
\end{hint}
\begin{hint}
Point slope form: $y-y_0 = m(x-x_0)$
\end{hint}

The equation of the tangent line is \ $y = \answer{1}$

\end{problem}



\begin{problem}(problem 4)

Find $x$-values in the interval $[0, 2\pi]$ for which the tangent line to the graph of $f(x) = \sin(x)$ is horizontal.


\begin{hint}
Solve the equation $f'(x) = 0$ for $x$
\end{hint}

The $x$-values are:
\begin{multipleChoice}
  \choice{$0, \pi, 2\pi$}
  \choice[correct]{$\pi/2, 3\pi/2$}
  \choice{$-\pi/2, \pi/2$}
\end{multipleChoice}

\end{problem}




%interactive graph with sin, tangent lines and its derivative
%problems with sin(x) accompanied by e^x and x^n


%derivative of cos
%interactive graph of cos, tan lines and derviv
%repeat problems for cos accompanied by sin x and x^n

We now determine the derivative of the cosine function using the definition of the derivative.

\begin{explanation}
If $f(x) = \cos(x)$ then\\
\begin{center}
$\begin{aligned}
f'(x) &= \lim_{h \to 0} \frac{f(x+h)-f(x)}{h} \\[5pt]
&= \lim_{h \to 0} \frac{\cos(x+h) - \cos(x)}{h}\\[5pt]
&=  \lim_{h \to 0} \frac{\cos(x)\cos(h) - \sin(x)\sin(h) - \cos(x)}{h}\\[5pt]
&=  \lim_{h \to 0} \frac{\cos(x)\cos(h) - \cos(x) - \sin(x)\sin(h)}{h}\\[5pt]
&=  \lim_{h \to 0} \frac{\cos(x)[\cos(h) -1] - \sin(x)\sin(h)}{h}\\[5pt]
&=  \lim_{h \to 0} \frac{\cos(x)[\cos(h) -1]}{h} - \frac{\sin(x)\sin(h)}{h}\\[5pt]
&=   \cos(x) \cdot 0 - \sin(x)\cdot 1\\[5pt]
&= -\sin(x).\\[5pt]
\end{aligned}$
\end{center}
Using other variables, this can be written as
\[
\frac{d}{dt} \cos(t) = -\sin(t) \\
\text{or}\\
\frac{d}{du} \cos(u) = -\sin(u).
\]

\end{explanation}


\begin{problem}(problem 5)
Find the equation of the tangent line to the graph of $f(x) = \cos(x)$ at $x=\pi/2.$


\begin{hint}
The point of tangency is $(\pi/2, f(\pi/2))$
\end{hint}
\begin{hint}
Use the derivative to find the slope, $m$
\end{hint}
\begin{hint}
Point slope form: $y-y_0 = m(x-x_0)$
\end{hint}

The equation of the tangent line is \ $y = \answer{-x + \pi/2}$

\end{problem}


\begin{example}[example 6]
 If $f(x) = -3\cos(x)$ then we use the constant multiple rule with $c = -3$ and we get 
\[
f'(x) = -3(-\sin(x)) = 3\sin(x).
\]
\end{example}



\begin{problem}(problem 6a)
  Compute 
  \[
  \frac{d}{dx} \left[\pi\cos(x)\right]
  \]
  
    \begin{hint}
      The derivative of $\cos(x)$ is $-\sin(x)$
    \end{hint}
		\begin{hint}
		  Don't forget to multiply by $\pi$
		\end{hint}
		
		The derivative of $\pi\cos(x)$ with respect to $x$ is
		 $\answer{-\pi\sin(x)}$
	
\end{problem}



\begin{problem}(problem 6b)

Find $x$-values in the interval $[0, 2\pi]$ for which the tangent line to the graph of $f(x) = \cos(x)$ is horizontal.


\begin{hint}
Solve the equation $f'(x) = 0$ for $x$
\end{hint}

The $x$-values are:
\begin{multipleChoice}
  \choice{$\pi/2, 3\pi/2$}
  \choice{$-\pi,0 \pi$}
  \choice[correct]{$0, \pi, 2\pi$}
\end{multipleChoice}

\end{problem}



%\begin{center}
\begin{foldable}
\unfoldable{Below is a graph of $f(x) = \cos(x)$ (in blue) and its derivative, $f'(x) = -\sin(x)$ (in purple).
Notice that the slope of the tangent line to $f(x)$ (in red) is the height of the corresponding 
point on $f'(x)$. Use it to see a graphical representation of the answers to the problem above.}
\includeinteractive{coswithderiv.js}
\end{foldable}
%\end{center}


\begin{example}[example 7]
 Find $f'(x)$ if $f(x) = 4\sin(x) + 5\cos(x).$ \\
 Using the sum and constant multiple rules, we get: 
\[
f'(x) = 4\cos(x) - 5\sin(x).
\]
\end{example}


\begin{problem}(problem 7)
  Compute 
  \[
  \frac{d}{dx} \left[3\sin(x) + 7\cos(x)\right]
  \]
  
		\begin{hint}
      Use the Constant Multiple Rule on each term
    \end{hint}
    \begin{hint}
      The Constant Multiple Rule says:
      \[
      \frac{d}{dx} cf(x) = cf'(x)
      \]
    \end{hint}    
		The derivative of $3\sin(x) + 7\cos(x)$ with respect to $x$ is
		 $\answer{3\cos(x) - 7\sin(x)}$
	
\end{problem}


\begin{example}[example 8]
 Find $f'(x)$ if  $f(x) = 2\sin(x) - 3\cos(x).$\\
 Using the Difference and Constant Multiple Rules, we get:
 \[
 f'(x) = 2\cos(x) - (-3\sin(x)) = 2\cos(x) + 3\sin(x).
 \]
\end{example}


\begin{problem}(problem 8a)
  Compute 
  \[
  \frac{d}{dx} \left[3\sin(x) - 7\cos(x)\right]
  \]
  
		\begin{hint}
      Use the difference and constant multiple rules
    \end{hint}
    \begin{hint}
      Constant Multiple Rule:
      \[
      (cu)' = cu'
      \]
    \end{hint}    
		The derivative of $3\sin(x) - 7\cos(x)$ with respect to $x$ is
		 $\answer{3\cos(x) + 7\sin(x)}$
	
\end{problem}

\begin{problem}(problem 8b)
  Compute $y'$ if
  \[
  y= \left[6\cos(x) - 8\sin(x)\right]
  \]
  
		\begin{hint}
      Use the difference and constant multiple Rules
    \end{hint}
    \begin{hint}
      Difference Rule:
      \[
      (u-v)' = u' - v'
      \]
    \end{hint}    
		
  $y' = \answer{-6\sin(x) - 8\cos(x)}$
	
\end{problem}

We close this section with an example involving rectilinear motion.

\begin{example}[example 9, Rectilinear Motion]
A point on a vibrating guitar string is moving vertically.
The position, $s$, of the point at time $t$, is given by $s = 2\cos(t)$.  Find the velocity and acceleration of the point.\\
The velocity of the point is given by 
\[
v = s' = -2\sin(t),
\]
and its acceleration is given by 
\[
a = s'' = -2\cos(t).
\]

\end{example}

\begin{problem}(problem 9)
A weight is suspended from a spring.
The height, $s$, of the weight at time $t$, is given by $s = 10-\sin(t)$.  Find the velocity and acceleration of the weight.\\
The velocity of the point is given by\\
 $v = \answer{ -\cos(t)},$ \\
and its acceleration is given by\\
 $a = \answer{\sin(t)}.$
\end{problem}



\end{document}






%\newcommand{\ffrac}[2]{\frac{\mbox{\footnotesize $#1$}}{\mbox{\footnotesize $#2$}}}
%\newcommand{\vasymptote}[2][]{
 %   \draw [densely dashed,#1] ({rel axis cs:0,0} -| {axis cs:#2,0}) -- ({rel axis cs:0,1} -| {axis cs:#2,0});
%}
