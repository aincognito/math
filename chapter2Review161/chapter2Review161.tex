\documentclass[handout]{ximera}

%% You can put user macros here
%% However, you cannot make new environments



\newcommand{\ffrac}[2]{\frac{\text{\footnotesize $#1$}}{\text{\footnotesize $#2$}}}
\newcommand{\vasymptote}[2][]{
    \draw [densely dashed,#1] ({rel axis cs:0,0} -| {axis cs:#2,0}) -- ({rel axis cs:0,1} -| {axis cs:#2,0});
}


\graphicspath{{./}{firstExample/}}
\usepackage{forest}
\usepackage{amsmath}
\usepackage{amssymb}
\usepackage{array}
\usepackage[makeroom]{cancel} %% for strike outs
\usepackage{pgffor} %% required for integral for loops
\usepackage{tikz}
\usepackage{tikz-cd}
\usepackage{tkz-euclide}
\usetikzlibrary{shapes.multipart}


%\usetkzobj{all}
\tikzstyle geometryDiagrams=[ultra thick,color=blue!50!black]


\usetikzlibrary{arrows}
\tikzset{>=stealth,commutative diagrams/.cd,
  arrow style=tikz,diagrams={>=stealth}} %% cool arrow head
\tikzset{shorten <>/.style={ shorten >=#1, shorten <=#1 } } %% allows shorter vectors

\usetikzlibrary{backgrounds} %% for boxes around graphs
\usetikzlibrary{shapes,positioning}  %% Clouds and stars
\usetikzlibrary{matrix} %% for matrix
\usepgfplotslibrary{polar} %% for polar plots
\usepgfplotslibrary{fillbetween} %% to shade area between curves in TikZ



%\usepackage[width=4.375in, height=7.0in, top=1.0in, papersize={5.5in,8.5in}]{geometry}
%\usepackage[pdftex]{graphicx}
%\usepackage{tipa}
%\usepackage{txfonts}
%\usepackage{textcomp}
%\usepackage{amsthm}
%\usepackage{xy}
%\usepackage{fancyhdr}
%\usepackage{xcolor}
%\usepackage{mathtools} %% for pretty underbrace % Breaks Ximera
%\usepackage{multicol}



\newcommand{\RR}{\mathbb R}
\newcommand{\R}{\mathbb R}
\newcommand{\C}{\mathbb C}
\newcommand{\N}{\mathbb N}
\newcommand{\Z}{\mathbb Z}
\newcommand{\dis}{\displaystyle}
%\renewcommand{\d}{\,d\!}
\renewcommand{\d}{\mathop{}\!d}
\newcommand{\dd}[2][]{\frac{\d #1}{\d #2}}
\newcommand{\pp}[2][]{\frac{\partial #1}{\partial #2}}
\renewcommand{\l}{\ell}
\newcommand{\ddx}{\frac{d}{\d x}}

\newcommand{\zeroOverZero}{\ensuremath{\boldsymbol{\tfrac{0}{0}}}}
\newcommand{\inftyOverInfty}{\ensuremath{\boldsymbol{\tfrac{\infty}{\infty}}}}
\newcommand{\zeroOverInfty}{\ensuremath{\boldsymbol{\tfrac{0}{\infty}}}}
\newcommand{\zeroTimesInfty}{\ensuremath{\small\boldsymbol{0\cdot \infty}}}
\newcommand{\inftyMinusInfty}{\ensuremath{\small\boldsymbol{\infty - \infty}}}
\newcommand{\oneToInfty}{\ensuremath{\boldsymbol{1^\infty}}}
\newcommand{\zeroToZero}{\ensuremath{\boldsymbol{0^0}}}
\newcommand{\inftyToZero}{\ensuremath{\boldsymbol{\infty^0}}}


\newcommand{\numOverZero}{\ensuremath{\boldsymbol{\tfrac{\#}{0}}}}
\newcommand{\dfn}{\textbf}
%\newcommand{\unit}{\,\mathrm}
\newcommand{\unit}{\mathop{}\!\mathrm}
%\newcommand{\eval}[1]{\bigg[ #1 \bigg]}
\newcommand{\eval}[1]{ #1 \bigg|}
\newcommand{\seq}[1]{\left( #1 \right)}
\renewcommand{\epsilon}{\varepsilon}
\renewcommand{\iff}{\Leftrightarrow}

\DeclareMathOperator{\arccot}{arccot}
\DeclareMathOperator{\arcsec}{arcsec}
\DeclareMathOperator{\arccsc}{arccsc}
\DeclareMathOperator{\si}{Si}
\DeclareMathOperator{\proj}{proj}
\DeclareMathOperator{\scal}{scal}
\DeclareMathOperator{\cis}{cis}
\DeclareMathOperator{\Arg}{Arg}
%\DeclareMathOperator{\arg}{arg}
\DeclareMathOperator{\Rep}{Re}
\DeclareMathOperator{\Imp}{Im}
\DeclareMathOperator{\sech}{sech}
\DeclareMathOperator{\csch}{csch}
\DeclareMathOperator{\Log}{Log}

\newcommand{\tightoverset}[2]{% for arrow vec
  \mathop{#2}\limits^{\vbox to -.5ex{\kern-0.75ex\hbox{$#1$}\vss}}}
\newcommand{\arrowvec}{\overrightarrow}
\renewcommand{\vec}{\mathbf}
\newcommand{\veci}{{\boldsymbol{\hat{\imath}}}}
\newcommand{\vecj}{{\boldsymbol{\hat{\jmath}}}}
\newcommand{\veck}{{\boldsymbol{\hat{k}}}}
\newcommand{\vecl}{\boldsymbol{\l}}
\newcommand{\utan}{\vec{\hat{t}}}
\newcommand{\unormal}{\vec{\hat{n}}}
\newcommand{\ubinormal}{\vec{\hat{b}}}

\newcommand{\dotp}{\bullet}
\newcommand{\cross}{\boldsymbol\times}
\newcommand{\grad}{\boldsymbol\nabla}
\newcommand{\divergence}{\grad\dotp}
\newcommand{\curl}{\grad\cross}
%% Simple horiz vectors
\renewcommand{\vector}[1]{\left\langle #1\right\rangle}


\outcome{Sample Tests for Test 2}

\title{2.6 Chapter 2 Review}

\begin{document}

\begin{abstract}
Sample Tests for Chapter 2
\end{abstract}

\maketitle

\section{Sample Test 2a}

\begin{problem}(problem 1)
Find the area between the curves $y = x^2 - 2$ and $y = 6 -x^2$.\\
\begin{hint}
Points of intersection: $x = -2, 2$
\end{hint}
$\answer{64/3}$
\end{problem}


\begin{problem}(problem 2)
Find the volume of the solid obtained by revolving the region bounded by the 
graphs of $y = x/2$ and $y = \sqrt x$ about the $x$-axis.\\
\begin{hint}
Points of intersection: $x = 0, 4$
\end{hint}\begin{hint}
Washer with $R = \sqrt x$ and $r = x/2$
\end{hint}
$\answer{8\pi/3}$
\end{problem}


\begin{problem}(problem 3)
Find the volume of the solid obtained by revolving the region bounded by the 
graphs of $y = x/2$ and $y = \sqrt x$ about the $y$-axis.\\
\begin{hint}
Shell with $r = x$ and $h = \sqrt x - x/2$
\end{hint}
$\answer{64\pi/15}$
\end{problem}


\begin{problem}(problem 4)
Find the average value of $f(x) = \cos^2(x)$ on the interval $[0, \pi/4]$.\\
\begin{hint}
Half-angle formula: $\cos(theta) = \frac12 (1+\cos(2\theta))$
\end{hint}
$\answer{1/2 + 1/\pi}$
\end{problem}


\begin{problem}(problem 5)
Find the length of the curve $\displaystyle y = \frac{x^2}{8} - \ln(x)$ over the interval $[1, 2]$.\\
\begin{hint}
$L = \int_a^b \sqrt{1 + f'(x)^2} \, dx$
\end{hint}
\begin{hint}
$\displaystyle f'(x)^2 =\left(\frac{x}{4} - \frac{1}{x}\right)^2 
= \frac{x^2}{4} - \frac{1}{2} + \frac{1}{x^2}$
\end{hint}
$\answer{3/8 + \ln(2)}$
\end{problem}


\begin{problem}(problem 6)
Solve the initial value problem: $\displaystyle y' = \frac{y^2}{x},\, y(1) = 4$.\\
\begin{hint}
Separate variables: $\frac{1}{y^2} dy = \frac{1}{x} dx$
\end{hint}
$y = \answer{\frac{4}{1- 4\ln(x)}}$
\end{problem}


\begin{problem}(problem 7)
Use the Integral Test to determine if the series $\displaystyle \sum_{n=1}^\infty \frac{3}{4n^3}$
converges or diverges. Make sure to show that the
conditions of the Integral Test are satisfied.\\
The series $\answer{converges}$
\end{problem}

\section{Sample Test 2b}

\begin{problem}(problem 1)
Find the area between the curves $x = y^2$ and $x = 2y + 3$.\\
\begin{hint}
Integrate right curve minus left curve with respect to $y$
\end{hint}
\begin{hint}
Points of intersection: $y = -1, 3$
\end{hint}
$\answer{25/3}$
\end{problem}


\begin{problem}(problem 2)
Find the volume of the solid obtained by revolving the region bounded by the 
graphs of $y = x-x^2$ and $y = 0$ about the line $ y = 0$.\\
\begin{hint}
Points of intersection: $x = 0, 1$
\end{hint}
\begin{hint}
Disk with $R = x - x^2$
\end{hint}

$\answer{\pi/30}$
\end{problem}


\begin{problem}(problem 3)
Find the volume of the solid obtained by revolving the region bounded by the 
graphs of $y = x-x^2$ and $y = 0$ about the line $x = -1$.\\
\begin{hint}
Shells $r = x+1$ and $h = x-x^2$
\end{hint}
$\answer{\pi/2}$
\end{problem}


\begin{problem}(problem 4)
Find the average value of $f(x) = xe^{3x}$ on the interval $[0, 2]$.\\
\begin{hint}
Integration by Parts: $\int u \, dv = uv - \int v \, du$
\end{hint}
$\answer{5e^6/18  + 1/18}$
\end{problem}


\begin{problem}(problem 5)
Find the length of the curve $\displaystyle y = \ln(\sec(x))$ over the interval $[0, \pi/4]$.\\
\begin{hint}
$f'(x) = \frac{1}{\sec(x)} \cdot \sec(x)\tan(x)$
\end{hint}
\begin{hint}
$\int \sec(x) \, dx = \ln|\sec(x) + \tan(x)| + C$
\end{hint}
$\answer{\ln(1 + \sqrt{2})}$
\end{problem}


\begin{problem}(problem 6)
Solve the initial value problem: $\displaystyle y' = 4x^3 y,\, y(0) = 5$.\\
\begin{hint}
Separate variables: $\frac{1}{y} dy = 4x^3 dx$
\end{hint}
$y = \answer{5e^{x^4}}$
\end{problem}


\begin{problem}(problem 7)
Use the Integral Test to determine if the series $\displaystyle \sum_{n=2}^\infty \frac{1}{n\ln(n)}$
converges or diverges. Make sure to show that the
conditions of the Integral Test are satisfied.\\
\begin{hint}
$f'(x) = -\frac{1+\ln(x)}{[x \ln(x)]^2}$
\end{hint}
\begin{hint}
Use $u = \ln(x)$ to integrate
\end{hint}
The series $\answer{diverges}$
\end{problem}

\end{document}














