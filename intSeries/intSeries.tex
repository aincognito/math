\documentclass[handout]{ximera}

%% You can put user macros here
%% However, you cannot make new environments



\newcommand{\ffrac}[2]{\frac{\text{\footnotesize $#1$}}{\text{\footnotesize $#2$}}}
\newcommand{\vasymptote}[2][]{
    \draw [densely dashed,#1] ({rel axis cs:0,0} -| {axis cs:#2,0}) -- ({rel axis cs:0,1} -| {axis cs:#2,0});
}


\graphicspath{{./}{firstExample/}}
\usepackage{forest}
\usepackage{amsmath}
\usepackage{amssymb}
\usepackage{array}
\usepackage[makeroom]{cancel} %% for strike outs
\usepackage{pgffor} %% required for integral for loops
\usepackage{tikz}
\usepackage{tikz-cd}
\usepackage{tkz-euclide}
\usetikzlibrary{shapes.multipart}


%\usetkzobj{all}
\tikzstyle geometryDiagrams=[ultra thick,color=blue!50!black]


\usetikzlibrary{arrows}
\tikzset{>=stealth,commutative diagrams/.cd,
  arrow style=tikz,diagrams={>=stealth}} %% cool arrow head
\tikzset{shorten <>/.style={ shorten >=#1, shorten <=#1 } } %% allows shorter vectors

\usetikzlibrary{backgrounds} %% for boxes around graphs
\usetikzlibrary{shapes,positioning}  %% Clouds and stars
\usetikzlibrary{matrix} %% for matrix
\usepgfplotslibrary{polar} %% for polar plots
\usepgfplotslibrary{fillbetween} %% to shade area between curves in TikZ



%\usepackage[width=4.375in, height=7.0in, top=1.0in, papersize={5.5in,8.5in}]{geometry}
%\usepackage[pdftex]{graphicx}
%\usepackage{tipa}
%\usepackage{txfonts}
%\usepackage{textcomp}
%\usepackage{amsthm}
%\usepackage{xy}
%\usepackage{fancyhdr}
%\usepackage{xcolor}
%\usepackage{mathtools} %% for pretty underbrace % Breaks Ximera
%\usepackage{multicol}



\newcommand{\RR}{\mathbb R}
\newcommand{\R}{\mathbb R}
\newcommand{\C}{\mathbb C}
\newcommand{\N}{\mathbb N}
\newcommand{\Z}{\mathbb Z}
\newcommand{\dis}{\displaystyle}
%\renewcommand{\d}{\,d\!}
\renewcommand{\d}{\mathop{}\!d}
\newcommand{\dd}[2][]{\frac{\d #1}{\d #2}}
\newcommand{\pp}[2][]{\frac{\partial #1}{\partial #2}}
\renewcommand{\l}{\ell}
\newcommand{\ddx}{\frac{d}{\d x}}

\newcommand{\zeroOverZero}{\ensuremath{\boldsymbol{\tfrac{0}{0}}}}
\newcommand{\inftyOverInfty}{\ensuremath{\boldsymbol{\tfrac{\infty}{\infty}}}}
\newcommand{\zeroOverInfty}{\ensuremath{\boldsymbol{\tfrac{0}{\infty}}}}
\newcommand{\zeroTimesInfty}{\ensuremath{\small\boldsymbol{0\cdot \infty}}}
\newcommand{\inftyMinusInfty}{\ensuremath{\small\boldsymbol{\infty - \infty}}}
\newcommand{\oneToInfty}{\ensuremath{\boldsymbol{1^\infty}}}
\newcommand{\zeroToZero}{\ensuremath{\boldsymbol{0^0}}}
\newcommand{\inftyToZero}{\ensuremath{\boldsymbol{\infty^0}}}


\newcommand{\numOverZero}{\ensuremath{\boldsymbol{\tfrac{\#}{0}}}}
\newcommand{\dfn}{\textbf}
%\newcommand{\unit}{\,\mathrm}
\newcommand{\unit}{\mathop{}\!\mathrm}
%\newcommand{\eval}[1]{\bigg[ #1 \bigg]}
\newcommand{\eval}[1]{ #1 \bigg|}
\newcommand{\seq}[1]{\left( #1 \right)}
\renewcommand{\epsilon}{\varepsilon}
\renewcommand{\iff}{\Leftrightarrow}

\DeclareMathOperator{\arccot}{arccot}
\DeclareMathOperator{\arcsec}{arcsec}
\DeclareMathOperator{\arccsc}{arccsc}
\DeclareMathOperator{\si}{Si}
\DeclareMathOperator{\proj}{proj}
\DeclareMathOperator{\scal}{scal}
\DeclareMathOperator{\cis}{cis}
\DeclareMathOperator{\Arg}{Arg}
%\DeclareMathOperator{\arg}{arg}
\DeclareMathOperator{\Rep}{Re}
\DeclareMathOperator{\Imp}{Im}
\DeclareMathOperator{\sech}{sech}
\DeclareMathOperator{\csch}{csch}
\DeclareMathOperator{\Log}{Log}

\newcommand{\tightoverset}[2]{% for arrow vec
  \mathop{#2}\limits^{\vbox to -.5ex{\kern-0.75ex\hbox{$#1$}\vss}}}
\newcommand{\arrowvec}{\overrightarrow}
\renewcommand{\vec}{\mathbf}
\newcommand{\veci}{{\boldsymbol{\hat{\imath}}}}
\newcommand{\vecj}{{\boldsymbol{\hat{\jmath}}}}
\newcommand{\veck}{{\boldsymbol{\hat{k}}}}
\newcommand{\vecl}{\boldsymbol{\l}}
\newcommand{\utan}{\vec{\hat{t}}}
\newcommand{\unormal}{\vec{\hat{n}}}
\newcommand{\ubinormal}{\vec{\hat{b}}}

\newcommand{\dotp}{\bullet}
\newcommand{\cross}{\boldsymbol\times}
\newcommand{\grad}{\boldsymbol\nabla}
\newcommand{\divergence}{\grad\dotp}
\newcommand{\curl}{\grad\cross}
%% Simple horiz vectors
\renewcommand{\vector}[1]{\left\langle #1\right\rangle}


\outcome{Integrate a Power Series}

\title{3.14 Integrating Power Series}

\begin{document}

\begin{abstract}
We integrate power series term by term.
\end{abstract}

\maketitle

\section{Integration of Power Series}


Suppose that the power series 
\[
\sum_{n=0}^\infty a_n x^n
\]
converges for all $x$ in some open interval $I$. Then, on this interval, the power series has an anti-derivative which can be obtained by integrating term-by-term:


\[
\int \left( \sum_{n=0}^\infty a_n x^n \right) \, dx  = \sum_{n=0}^\infty  \int a_n x^n \, dx   
= \sum_{n=0}^\infty   \frac{a_n}{n+1} x^{n+1} + C.
\]
In other words, the indefinite integral of a power series is computed term by term, as we would anti-differentiate a polynomial.

\begin{example}[example 1]
The power series 
\[
\sum_{n=0}^\infty x^n
\]
converges on the interval $(-1, 1)$.
On that interval, its anti-derivative is given by
\[
\int\left( \sum_{n=0}^\infty x^n \right) \, dx= \sum_{n=0}^\infty  \big(\int x^n \, dx\big) = \sum_{n=0}^\infty \frac{x^{n+1}}{n+1} + C.
\]
The interval of convergence of the anti-derivative is $[-1,1)$. In general, 
the anti-derivative might converge at an endpoint when the original series did not.
\end{example}



\begin{problem}(problem 1)
Find the anti-derivatives of the following power series. 
Also, compare the interval of convergence of the original series to its anti-derivative.
\[
a) \; \sum_{n=0}^\infty \frac{x^n}{n+1},\qquad b) \; \sum_{n=0}^\infty \frac{x^n}{2^n},
\]
\[
c) \;  \sum_{n=0}^\infty \frac{x^n}{n!},\qquad d) \; \sum_{n=0}^\infty x^{2n}
\]
\end{problem}

\begin{example}[example 2]
The power series expansion for the function $\frac{1}{1+x}$ is 
\[
\frac{1}{1+x} = \sum_{n=0}^\infty (-1)^n x^n, |x| < 1.
\]
Integrate the function and the series to obtain a power series representation for a logaritmic function.\\
First, note that
\[
\int \frac{1}{1+x} \, dx = \ln|1+x| + C.
\]
Integrating the power series:
\[
\int \left[\sum_{n=0}^\infty (-1)^n x^n \right] \, dx =  \sum_{n=0}^\infty \left[ \int(-1)^n x^n \, dx \right] = \sum_{n=0}^\infty  (-1)^n \frac{x^{n+1}}{n+1} +C.
\]
Equating these two antiderivatives, we have
\[
\ln|1+x| = \sum_{n=0}^\infty  (-1)^n \frac{x^{n+1}}{n+1} +C, \; -1 < x \leq 1,
\]
If we set $x = 0$, we can find $C$:
\[
\ln(1) = \sum_{n=0}^\infty  (-1)^n \frac{0^{n+1}}{n+1} +C = 0 + C,
\]
so $C = \ln(1) = 0$. Hence,
\[
\ln|1+x| = \sum_{n=0}^\infty  (-1)^n \frac{x^{n+1}}{n+1}, \; -1 < x < 1,
\]
and we have a power series representiation for a logarithmic function.

\begin{remark}[Remark 1]
Note that the above series converges when $x = 1$ since it is the alternating harmonic series. Substituting $x = 1$ gives its sum:
\[
\ln(2) = \sum_{n=0}^\infty  (-1)^n \frac{1^{n+1}}{n+1} = \sum_{n=0}^\infty  (-1)^n \frac{1}{n+1} = 1 - \frac12 + \frac13 - \frac14 + \cdots
\]
Hence, the sum of the alternating harmonic series is $\ln(2) \approx 0.7$.
\end{remark}

\begin{remark}[Remark 2]
Since $-1 < x\leq 1$ we have $0 < 1+x \leq 2$. Thus, the absolute value bars in the logarithm can be removed to obtain:
\[
\ln(1+x) = \sum_{n=0}^\infty  (-1)^n \frac{x^{n+1}}{n+1}, \; -1 < x \leq 1.
\]
\end{remark}

\end{example}

\begin{example}[example 3]
The power series representation for the function $\frac{1}{1+x^2}$ is 
\[
\frac{1}{1+x^2} = \sum_{n=0}^\infty (-1)^n x^{2n}, \; |x| < 1.
\]
Integrate the function and the series to obtain a power series representation for an inverse trigonometric function.\\
We have
\[
\int \frac{1}{1+x^2} \, dx = \tan^{-1}(x) + C,
\]
and integrating the power series gives:
\[
\int \left[\sum_{n=0}^\infty (-1)^n x^{2n} \right] \, dx =  \sum_{n=0}^\infty \left[ \int(-1)^n x^{2n} \, dx \right] = \sum_{n=0}^\infty  (-1)^n \frac{x^{2n+1}}{2n+1} +C.
\]
Equating these two antiderivatives, we have
\[
\tan^{-1}(x) = \sum_{n=0}^\infty  (-1)^n \frac{x^{2n+1}}{2n+1} +C, \; |x| < 1
\]
If we set $x = 0$, we can find $C$:
\[
\tan^{-1}(0) = \sum_{n=0}^\infty  (-1)^n \frac{0^{2n+1}}{2n+1} +C = 0 + C,
\]
so $C = \tan^{-1}(0) = 0$ and we have the following power series representation for the inverse tangent function:
\[
\tan^{-1}(x) = \sum_{n=0}^\infty  (-1)^n \frac{x^{2n+1}}{2n+1} +C, \; |x| < 1
\]
\begin{remark}[Remark]
If we substitute $x = 1$ into both sides of the power series representation, we obtain:
\[
\tan^{-1}(1) = \sum_{n=0}^\infty  (-1)^n \frac{1^{2n+1}}{2n+1} = \sum_{n=0}^\infty  (-1)^n \frac{1}{2n+1} 
\]
Since $\tan^{-1}(1) = \pi/4$, we can multiply by $4$ to obtain an infinite series whose sum is $\pi$:
\[
\pi = \sum_{n=0}^\infty  (-1)^n \frac{4}{2n+1} = 4 - \frac43 + \frac45 - \frac47 + \cdots
\]
\end{remark}




\end{example}

\end{document}





