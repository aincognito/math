\documentclass[handout]{ximera}

%% You can put user macros here
%% However, you cannot make new environments



\newcommand{\ffrac}[2]{\frac{\text{\footnotesize $#1$}}{\text{\footnotesize $#2$}}}
\newcommand{\vasymptote}[2][]{
    \draw [densely dashed,#1] ({rel axis cs:0,0} -| {axis cs:#2,0}) -- ({rel axis cs:0,1} -| {axis cs:#2,0});
}


\graphicspath{{./}{firstExample/}}
\usepackage{forest}
\usepackage{amsmath}
\usepackage{amssymb}
\usepackage{array}
\usepackage[makeroom]{cancel} %% for strike outs
\usepackage{pgffor} %% required for integral for loops
\usepackage{tikz}
\usepackage{tikz-cd}
\usepackage{tkz-euclide}
\usetikzlibrary{shapes.multipart}


%\usetkzobj{all}
\tikzstyle geometryDiagrams=[ultra thick,color=blue!50!black]


\usetikzlibrary{arrows}
\tikzset{>=stealth,commutative diagrams/.cd,
  arrow style=tikz,diagrams={>=stealth}} %% cool arrow head
\tikzset{shorten <>/.style={ shorten >=#1, shorten <=#1 } } %% allows shorter vectors

\usetikzlibrary{backgrounds} %% for boxes around graphs
\usetikzlibrary{shapes,positioning}  %% Clouds and stars
\usetikzlibrary{matrix} %% for matrix
\usepgfplotslibrary{polar} %% for polar plots
\usepgfplotslibrary{fillbetween} %% to shade area between curves in TikZ



%\usepackage[width=4.375in, height=7.0in, top=1.0in, papersize={5.5in,8.5in}]{geometry}
%\usepackage[pdftex]{graphicx}
%\usepackage{tipa}
%\usepackage{txfonts}
%\usepackage{textcomp}
%\usepackage{amsthm}
%\usepackage{xy}
%\usepackage{fancyhdr}
%\usepackage{xcolor}
%\usepackage{mathtools} %% for pretty underbrace % Breaks Ximera
%\usepackage{multicol}



\newcommand{\RR}{\mathbb R}
\newcommand{\R}{\mathbb R}
\newcommand{\C}{\mathbb C}
\newcommand{\N}{\mathbb N}
\newcommand{\Z}{\mathbb Z}
\newcommand{\dis}{\displaystyle}
%\renewcommand{\d}{\,d\!}
\renewcommand{\d}{\mathop{}\!d}
\newcommand{\dd}[2][]{\frac{\d #1}{\d #2}}
\newcommand{\pp}[2][]{\frac{\partial #1}{\partial #2}}
\renewcommand{\l}{\ell}
\newcommand{\ddx}{\frac{d}{\d x}}

\newcommand{\zeroOverZero}{\ensuremath{\boldsymbol{\tfrac{0}{0}}}}
\newcommand{\inftyOverInfty}{\ensuremath{\boldsymbol{\tfrac{\infty}{\infty}}}}
\newcommand{\zeroOverInfty}{\ensuremath{\boldsymbol{\tfrac{0}{\infty}}}}
\newcommand{\zeroTimesInfty}{\ensuremath{\small\boldsymbol{0\cdot \infty}}}
\newcommand{\inftyMinusInfty}{\ensuremath{\small\boldsymbol{\infty - \infty}}}
\newcommand{\oneToInfty}{\ensuremath{\boldsymbol{1^\infty}}}
\newcommand{\zeroToZero}{\ensuremath{\boldsymbol{0^0}}}
\newcommand{\inftyToZero}{\ensuremath{\boldsymbol{\infty^0}}}


\newcommand{\numOverZero}{\ensuremath{\boldsymbol{\tfrac{\#}{0}}}}
\newcommand{\dfn}{\textbf}
%\newcommand{\unit}{\,\mathrm}
\newcommand{\unit}{\mathop{}\!\mathrm}
%\newcommand{\eval}[1]{\bigg[ #1 \bigg]}
\newcommand{\eval}[1]{ #1 \bigg|}
\newcommand{\seq}[1]{\left( #1 \right)}
\renewcommand{\epsilon}{\varepsilon}
\renewcommand{\iff}{\Leftrightarrow}

\DeclareMathOperator{\arccot}{arccot}
\DeclareMathOperator{\arcsec}{arcsec}
\DeclareMathOperator{\arccsc}{arccsc}
\DeclareMathOperator{\si}{Si}
\DeclareMathOperator{\proj}{proj}
\DeclareMathOperator{\scal}{scal}
\DeclareMathOperator{\cis}{cis}
\DeclareMathOperator{\Arg}{Arg}
%\DeclareMathOperator{\arg}{arg}
\DeclareMathOperator{\Rep}{Re}
\DeclareMathOperator{\Imp}{Im}
\DeclareMathOperator{\sech}{sech}
\DeclareMathOperator{\csch}{csch}
\DeclareMathOperator{\Log}{Log}

\newcommand{\tightoverset}[2]{% for arrow vec
  \mathop{#2}\limits^{\vbox to -.5ex{\kern-0.75ex\hbox{$#1$}\vss}}}
\newcommand{\arrowvec}{\overrightarrow}
\renewcommand{\vec}{\mathbf}
\newcommand{\veci}{{\boldsymbol{\hat{\imath}}}}
\newcommand{\vecj}{{\boldsymbol{\hat{\jmath}}}}
\newcommand{\veck}{{\boldsymbol{\hat{k}}}}
\newcommand{\vecl}{\boldsymbol{\l}}
\newcommand{\utan}{\vec{\hat{t}}}
\newcommand{\unormal}{\vec{\hat{n}}}
\newcommand{\ubinormal}{\vec{\hat{b}}}

\newcommand{\dotp}{\bullet}
\newcommand{\cross}{\boldsymbol\times}
\newcommand{\grad}{\boldsymbol\nabla}
\newcommand{\divergence}{\grad\dotp}
\newcommand{\curl}{\grad\cross}
%% Simple horiz vectors
\renewcommand{\vector}[1]{\left\langle #1\right\rangle}


\outcome{Find the interval of convergence of a power series}

\title{3.11 Power Series: Interval of Convergence}

\begin{document}

\begin{abstract}
We find the interval of convergence of a power series.
\end{abstract}

\maketitle

\section{Interval of Convergence}

\section{Power Series}

We will find the interval of convergence of a power series.

\begin{definition}[Power Series]
A \textbf{power series} with center at $x = a$ is an infinite series of the form
\[
\sum_{n=0}^\infty c_n(x-a)^n
\]
\end{definition}

Some examples of power series are:
\[ 
\sum_{n=0}^\infty x^n, \; \sum_{n=0}^\infty \frac{x^n}{n!}, \;\text{and} \sum_{n=0}^\infty \frac{n^2(x+ 2)^n}{2^n}.
\]
The first two have center at $x = 0$ and the third is centered at $x = -2$. 
The question we ask when confronted with a power series is, ``for which values of $x$ does the series converge?"
The theory tells us that the power series will converge in an interval centered at the center of the power series.
To find this \textbf{interval of convergence}, we frequently use the ratio test.

\begin{example}[example 1]
Find the interval of convergence of the power series $\displaystyle{\sum_{n=0}^\infty x^n}$.\\
Noting that this series happens to be a geometric series (with common ratio $x$), we can use the fact that this series will converge 
if and only in $|x| < 1$.  This is equivalent to the interval $(-1, 1)$ and this is the interval of convergence of the power series.
In other words, for any value of $x$ in this interval, 
the resulting series will converge and for any value of $x$ outside of this interval, 
the series will diverge. Notice that the interval of convergence is centered around $x = 0$, which is the center of the power series.
\end{example}

\begin{problem}(problem 1)
Find the interval of convergence of the power series
\[
\sum_{n=0}^\infty \frac{x^n}{2^n}
\]
The interval of convergence is
\begin{multipleChoice}
\choice{$(-1/2, 1/2)$}
\choice{$(-1, 1)$}
\choice[correct]{$(-2, 2)$}
\choice{$(-\infty, \infty)$}
\end{multipleChoice}
\end{problem}


\begin{example}[example 2]
Find the interval of convergence of the power series $\displaystyle{\sum_{n=1}^\infty \frac{x^n}{n^2}}$.\\
We will use the ratio test:
\[
L = \lim_{n \to \infty} \left|\frac{x^{n+1}}{(n+1)^2} \cdot \frac{n^2}{x^n} \right| = \lim_{n \to \infty} \frac{n^2}{(n+1)^2}\cdot \left|  \frac{x^{n+1}}{x^n} \right| 
\]
\[
= \lim_{n \to \infty} \frac{n^2}{(n+1)^2}\cdot \left|  x \right| = |x|.
\]
By the rules for the ratio test, the series converges when $|x| < 1$ and diverges when $|x| > 1$.
Unfortunately, the ratio test gives no conclusion when $|x| = 1$, which corresponds to $x = \pm 1$.
To determine the behavior of the series at these values, we plug them into the power series.
If $x = 1$, the power series becomes
\[
\sum_{n=1}^\infty \frac{x^n}{n^2} = \sum_{n=1}^\infty \frac{1}{n^2},
\]
which is a $p$-series with $p = 2$. Since $p >1$, this series converges and therefore our power series converges when $x = 1$.
Next, if $x = -1$, the power series becomes:
\[
\sum_{n=1}^\infty \frac{x^n}{n^2} = \sum_{n=1}^\infty \frac{(-1)^n}{n^2},
\]
which is an alternating $p$-series and so it converges. Therefore our power series converges when $x = -1$.
The interval of convergence of the power series is thus $[-1, 1]$, and we again note that this is an interval centered about the 
center of the power series, $x = 0$.
\end{example}


\begin{problem}(problem 2)
Find the interval of convergence of the power series
\[
\sum_{n=1}^\infty \frac{x^n}{n}
\]
The interval of convergence is
\begin{multipleChoice}
\choice{$(-1, 1)$}
\choice[correct]{$[-1, 1)$}
\choice{$(-1, 1]$}
\choice{$[-1, 1]$}
\end{multipleChoice}
\end{problem}


\begin{example}[example 3]
Find the interval of convergence of the power series $\displaystyle{\sum_{n=1}^\infty \frac{x^n}{n!}}$.\\
We will use the ratio test:
\[
L = \lim_{n \to \infty} \left|\frac{x^{n+1}}{(n+1)!} \cdot \frac{n!}{x^n} \right| = \lim_{n \to \infty} \frac{n!}{(n+1)!}\cdot \left|  \frac{x^{n+1}}{x^n} \right| 
\]
\[
= \lim_{n \to \infty} \frac{1}{n+1}\cdot |x| = 0.
\]
By the rules for the ratio test, the series converges regardless of the value of $x$.
Hence the interval of convergence is $(-\infty, \infty)$.
\end{example}




\begin{problem}(problem 3)
Find the interval of convergence of the power series
\[
\sum_{n=0}^\infty \frac{(n+2)}{n!}x^n
\]
The interval of convergence is
\begin{multipleChoice}
\choice{$0$}
\choice{$(-1, 1)$}
\choice{$[-1, 1]$}
\choice[correct]{$(-\infty, \infty)$}
\end{multipleChoice}
\end{problem}


\begin{example}[example 4]
Find the interval of convergence of the power series $\displaystyle{\sum_{n=1}^\infty \frac{(x-3)^n}{n}}$.\\
We will use the ratio test:
\[
L = \lim_{n \to \infty} \left|\frac{(x-3)^{n+1}}{n+1} \cdot \frac{n}{(x-3)^n} \right| = \lim_{n \to \infty} \frac{n}{n+1} \cdot \left|  \frac{(x-3)^{n+1}}{(x-3)^n} \right| 
\]
\[
= \lim_{n \to \infty} \frac{n}{n+1}\cdot |x-3| = |x-3|.
\]
By the rules for the ratio test, the series converges when $|x-3| < 1$ and diverges when $|x-3| > 1$.
Unfortunately, the ratio test gives no conclusion when $|x-3| = 1$, which corresponds to $x = 2$ and $x = 4$.
To determine the behavior of the series at these values, we plug them into the power series.
If $x = 4$, the power series becomes
\[
\sum_{n=1}^\infty \frac{(x-3)^n}{n} = \sum_{n=1}^\infty \frac{1}{n},
\]
which is the divergent harmonic series.
Next, if $x = 2$, the power series becomes:
\[
\sum_{n=1}^\infty \frac{(x-3)^n}{n} = \sum_{n=1}^\infty \frac{(-1)^n}{n},
\]
which is the convergent alternating harmonic series.
The interval of convergence of the power series is thus $[2, 4)$, and we again note that this is an interval centered about the 
center of the power series, $x = 3$.
\end{example}



\begin{problem}(problem 4)
Find the interval of convergence of the power series
\[
\sum_{n=1}^\infty \frac{(x+2)^n}{n^2}
\]
The interval of convergence is
\begin{multipleChoice}
\choice{$(-3, -1)$}
\choice{$[-3, -1)$}
\choice{$(-3, -1]$}
\choice[correct]{$[-3, -1]$}
\end{multipleChoice}
\end{problem}

\begin{example}[example 5]
Find the interval of convergence of the power series $\displaystyle{\sum_{n=1}^\infty \frac{(x+2)^n}{(2n+1)3^n}}$.\\
We will use the ratio test:
\[
L = \lim_{n \to \infty} \left|\frac{(x+2)^{n+1}}{(2n+3)3^{n+1}} \cdot \frac{(2n+1) 3^n}{(x+2)^n} \right| = 
\lim_{n \to \infty} \frac{2n+1}{2n+3} \cdot \frac{3^n}{3^{n+1}} \cdot \left|  \frac{(x+2)^{n+1}}{(x+2)^n} \right| 
\]
\[
= \lim_{n \to \infty} \frac{2n+1}{2n+3} \cdot \frac13 \cdot \left|  x+2 \right| = \frac{|x+2|}{3}.
\]
By the rules for the ratio test, the series converges when $\dfrac{|x+2|}{3} < 1$ and diverges when $\dfrac{|x+2|}{3} > 1$.
Unfortunately, the ratio test gives no conclusion when $\dfrac{|x+2|}{3} = 1$, which corresponds to $x = -5$ and $x = 1$.
To determine the behavior of the series at these values, we plug them into the power series.
If $x = 1$, the power series becomes
\[
\sum_{n=1}^\infty \frac{(x+2)^n}{(2n+1)3^n} = \sum_{n=1}^\infty \frac{3^n}{(2n+1)3^n} = \sum_{n=1}^\infty \frac{1}{2n+1}
\]
which diverges by the limit comparison test with the harmonic series.
Next, if $x = -5$, the power series becomes:
\[
\sum_{n=1}^\infty \frac{(x+2)^n}{(2n+1)3^n} = \sum_{n=1}^\infty \frac{(-3)^n}{(2n+1)3^n} = \sum_{n=1}^\infty \frac{(-1)^n}{2n+1}
\]
which converges by the alternating series test.
The interval of convergence of the power series is thus $[-5, 1)$, and we again note that this is an interval centered about the 
center of the power series, $x = -2$.
\end{example}




\begin{problem}(problem 5)
Find the interval of convergence of the power series
\[
\sum_{n=1}^\infty \frac{(x-4)^n}{n2^n}
\]
The interval of convergence is
\begin{multipleChoice}
\choice{$(2, 6)$}
\choice[correct]{$[2, 6)$}
\choice{$(2, 6]$}
\choice{$[2, 6]$}
\end{multipleChoice}
\end{problem}



\begin{center}
\begin{foldable}
\unfoldable{Here is a detailed, lecture style video on the interval of convergence of a power series:}
\youtube{DdrevgO7_eY}
\end{foldable}
\end{center}

\begin{center}
\begin{foldable}
Here is a detailed, lecture style video on the interval of convergence of a power series:
\youtube{DdrevgO7_eY}
\end{foldable}
\end{center}





\end{document}






