\documentclass[handout]{ximera}

%% You can put user macros here
%% However, you cannot make new environments



\newcommand{\ffrac}[2]{\frac{\text{\footnotesize $#1$}}{\text{\footnotesize $#2$}}}
\newcommand{\vasymptote}[2][]{
    \draw [densely dashed,#1] ({rel axis cs:0,0} -| {axis cs:#2,0}) -- ({rel axis cs:0,1} -| {axis cs:#2,0});
}


%\usepackage{tcolorbox} %%Needed for Derivative Definition supposedly and product rule, natural exp log, quotient rule, inverse trig, rates of change


% \graphicspath{{./}{firstExample/}}
% \usepackage{forest}
\usepackage{amsmath}
\usepackage{amssymb}
\usepackage{array}
\usepackage[makeroom]{cancel} %% for strike outs
\usepackage{pgffor} %% required for integral for loops
\usepackage{tikz}
\usepackage{tikz-cd}
\usepackage{tkz-euclide}
\usetikzlibrary{shapes.multipart}


% \usetkzobj{all}
\tikzstyle geometryDiagrams=[ultra thick,color=blue!50!black]


\usetikzlibrary{arrows}
\tikzset{>=stealth,commutative diagrams/.cd,
  arrow style=tikz,diagrams={>=stealth}} %% cool arrow head
\tikzset{shorten <>/.style={ shorten >=#1, shorten <=#1 } } %% allows shorter vectors

\usetikzlibrary{backgrounds} %% for boxes around graphs
\usetikzlibrary{shapes,positioning}  %% Clouds and stars
\usetikzlibrary{matrix} %% for matrix
\usepgfplotslibrary{polar} %% for polar plots
\usepgfplotslibrary{fillbetween} %% to shade area between curves in TikZ



%\usepackage[width=4.375in, height=7.0in, top=1.0in, papersize={5.5in,8.5in}]{geometry}
%\usepackage[pdftex]{graphicx}
%\usepackage{tipa}
%\usepackage{txfonts}
%\usepackage{textcomp}
%\usepackage{amsthm}
%\usepackage{xy}
%\usepackage{fancyhdr}
%\usepackage{xcolor}
%\usepackage{mathtools} %% for pretty underbrace % Breaks Ximera
%\usepackage{multicol}



\newcommand{\RR}{\mathbb R}
\newcommand{\R}{\mathbb R}
\newcommand{\C}{\mathbb C}
\newcommand{\N}{\mathbb N}
\newcommand{\Z}{\mathbb Z}
\newcommand{\dis}{\displaystyle}
%\renewcommand{\d}{\,d\!}
\renewcommand{\d}{\mathop{}\!d}
\newcommand{\dd}[2][]{\frac{\d #1}{\d #2}}
\newcommand{\pp}[2][]{\frac{\partial #1}{\partial #2}}
\renewcommand{\l}{\ell}
\newcommand{\ddx}{\frac{d}{\d x}}
\newcommand{\ppx}{\frac{\partial}{\partial x}}
\newcommand{\ppy}{\frac{\partial}{\partial y}}

\newcommand{\zeroOverZero}{\ensuremath{\boldsymbol{\tfrac{0}{0}}}}
\newcommand{\inftyOverInfty}{\ensuremath{\boldsymbol{\tfrac{\infty}{\infty}}}}
\newcommand{\zeroOverInfty}{\ensuremath{\boldsymbol{\tfrac{0}{\infty}}}}
\newcommand{\zeroTimesInfty}{\ensuremath{\small\boldsymbol{0\cdot \infty}}}
\newcommand{\inftyMinusInfty}{\ensuremath{\small\boldsymbol{\infty - \infty}}}
\newcommand{\oneToInfty}{\ensuremath{\boldsymbol{1^\infty}}}
\newcommand{\zeroToZero}{\ensuremath{\boldsymbol{0^0}}}
\newcommand{\inftyToZero}{\ensuremath{\boldsymbol{\infty^0}}}


\newcommand{\numOverZero}{\ensuremath{\boldsymbol{\tfrac{\#}{0}}}}
\newcommand{\dfn}{\textbf}
%\newcommand{\unit}{\,\mathrm}
\newcommand{\unit}{\mathop{}\!\mathrm}
%\newcommand{\eval}[1]{\bigg[ #1 \bigg]}
\newcommand{\eval}[1]{ #1 \bigg|}
\newcommand{\seq}[1]{\left( #1 \right)}
\renewcommand{\epsilon}{\varepsilon}
\renewcommand{\iff}{\Leftrightarrow}

\DeclareMathOperator{\arccot}{arccot}
\DeclareMathOperator{\arcsec}{arcsec}
\DeclareMathOperator{\arccsc}{arccsc}
\DeclareMathOperator{\si}{Si}
\DeclareMathOperator{\proj}{proj}
\DeclareMathOperator{\scal}{scal}
\DeclareMathOperator{\cis}{cis}
\DeclareMathOperator{\Arg}{Arg}
%\DeclareMathOperator{\arg}{arg}
\DeclareMathOperator{\Rep}{Re}
\DeclareMathOperator{\Imp}{Im}
\DeclareMathOperator{\sech}{sech}
\DeclareMathOperator{\csch}{csch}
\DeclareMathOperator{\Log}{Log}

\newcommand{\tightoverset}[2]{% for arrow vec
  \mathop{#2}\limits^{\vbox to -.5ex{\kern-0.75ex\hbox{$#1$}\vss}}}
\newcommand{\arrowvec}{\overrightarrow}
\renewcommand{\vec}{\mathbf}
\newcommand{\veci}{{\boldsymbol{\hat{\imath}}}}
\newcommand{\vecj}{{\boldsymbol{\hat{\jmath}}}}
\newcommand{\veck}{{\boldsymbol{\hat{k}}}}
\newcommand{\vecl}{\boldsymbol{\l}}
\newcommand{\utan}{\vec{\hat{t}}}
\newcommand{\unormal}{\vec{\hat{n}}}
\newcommand{\ubinormal}{\vec{\hat{b}}}

\newcommand{\dotp}{\bullet}
\newcommand{\cross}{\boldsymbol\times}
\newcommand{\grad}{\boldsymbol\nabla}
\newcommand{\divergence}{\grad\dotp}
\newcommand{\curl}{\grad\cross}
%% Simple horiz vectors
\renewcommand{\vector}[1]{\left\langle #1\right\rangle}


\pgfplotsset{compat=1.13}

\outcome{Find limits of a complex function}

\title{3.1 Complex Limits}

\begin{document}

\begin{abstract}
We find limits of complex functions.
\end{abstract}

\maketitle


\begin{definition} If $f$ is defined on the punctured disk $D_\circ(z_0, r)$ for some $r>0$ we say that
\[
\lim_{z \to z_0} f(z) = w_0
\]
if given $\epsilon > 0$ there exists $\delta >0$ such that 
\[
0< |z-z_0| < \delta \Rightarrow |f(z) - w_0| < \epsilon
\]
\end{definition}
 

\begin{example}[example 1]
Show that $\displaystyle \lim_{z \to 1+i} z^2 = 2i$.\\[6pt]
Suppose $|z-(1+i)| < \delta$ \; for some $0< \delta < 1$.
Note that
\[
z ^2 - 2i=\left[z-(1+i)\right] \cdot \left[z + (1+i)\right]
\]
By the Triangle Inequality
\[
|z+(1+i)| \leq |z| + |1+i| = |z| + \sqrt 2
\]
Now, since $|z-(1+i)| < \delta$, \; the Triangle Inequality also gives
\[
|z| = |z-(1+i) +(1+i)| \leq |z-(1+i)| + |1+i| < \delta + \sqrt 2 
\]
Thus,
\[
|z+(1+i)| \leq \delta + 2\sqrt 2 <  4 \; \mbox{since} \; \delta < 1
\]
Returning to $z^2 - 2i$, we have
\[
|z ^2 - 2i| = |z-(1+i)| \cdot | z+(1+i)| < \delta \cdot 4
\]

Now, let $\epsilon > 0$ and choose $\delta> 0$ such that $\delta < \min\left\{\frac{\epsilon}{4}, 1\right\}$. 
From the computations above, if $|z-(1+i)| < \delta$ then
\[
|z^2 - 2i| < 4\delta < \epsilon
\]
Hence,
\[
\lim_{z \to 1+i} z^2 = 2i
\]


\end{example}

\begin{problem}(problem 1a) Use the definition of limit to show each of the following:
\begin{align*}
i) & \displaystyle \lim_{z \to i} z^2 = -1\\
ii) & \displaystyle \lim_{z \to z_0} (az+b) = az_0 + b \;\;(a \neq 0)\\
iii) & \displaystyle \lim_{z \to z_0} \overline{z} = \overline{z_0}
\end{align*}
\end{problem}

Here is a video solution of problem 1a, part iii:\\
\begin{foldable}
\youtube{6gnoHj34QSA}
\end{foldable}

Complex limits have the same familiar properties as real limits.

\begin{theorem}
Suppose $f$ and $g$ are defined on the punctured disk $D_\circ(z_0, r)$ for some $r>0$, and suppose
\[
\lim_{z \to z_0} f(z) = w_f \quad \mbox{and} \quad \lim_{z \to z_0} g(z) = w_g
\]
then the following limits hold:
\begin{align*}
a) \; & \lim_{z \to z_0} \left[f(z)+ g(z)\right] = w_f +w_g\\
b) \; & \lim_{z \to z_0} \left[\alpha f(z)\right] = \alpha w_f, \;\; \mbox{for any} \;\;\alpha \in \C\\
c) \; & \lim_{z \to z_0} \left[f(z)\cdot g(z)\right] = w_f \cdot w_g\\
d) \; & \lim_{z \to z_0} \frac{f(z)}{g(z)} = \frac{w_f}{w_g}, \;\; \mbox{if}\;\; w_g \neq 0
\end{align*}
\end{theorem}

\begin{proof}
Part a)  Since $\lim_{z \to z_0} f(z) = w_f$, for a given $\epsilon >0$, there exists $\delta_f >0$ such that
\[
0<|z-z_0| < \delta_f \Rightarrow |f(z) - w_f| < \frac{\epsilon}{2}
\]
Similarly, there exists $\delta_g >0$ such that
\[
0<|z-z_0| < \delta_f \Rightarrow |g(z) - w_g| < \frac{\epsilon}{2}
\]
Let $\delta = \min\{\delta_f, \delta_g \}$ then for $0< |z-z_0| < \delta$, the Triangle Inequality gives
\[
|\left(f(z) + g(z)\right) - (w_f + w_g)| \leq |f(z) - w_f|+|g(z) - w_g| < \frac{\epsilon}{2}+\frac{\epsilon}{2} = \epsilon
\]
as required.\\[6pt]

Part d) It is sufficient to prove that 
\[
\lim_{z \to z_0} \frac{1}{g(z)} = \frac{1}{w_g}, \;\; \mbox{if}\;\; w_g \neq 0
\]
for then the result follows by observing that $\displaystyle \frac{f}{g} = f \cdot \frac{1}{g}$ and applying part c.\\
Let $\epsilon > 0$ and let $\epsilon' = \min\left\{\frac{|w_g|}{2}, \frac{\epsilon |w_g|^2}{2}\right\}$.

Then there exists $\delta >0$ such that
\[
 0< |z-z_0| < \delta \Rightarrow |g(z) - w_g| < \epsilon'
 \]
By the Reverse Triangle Inequality and the definition of $\epsilon'$, if $0 < |z-z_0| < \delta $ then
\[
|g(z)| > |w_g| - \epsilon' \geq |w_g| - \frac{|w_g|}{2} = \frac{|w_g|}{2}
\]
and so 
\[
\frac{1}{|g(z)|} < \frac{2}{|w_g|}
\]
Now, for $z$ such that $0 < |z-z_0| < \delta $ we have 
\begin{align*}
\left| \frac{1}{g(z)} - \frac{1}{w_g} \right| & = \frac{|w_g - g(z)|}{|g(z)|\cdot |w_g|}\\
& < \frac{2\epsilon'}{|w_g|^2}\\
& \leq \frac{2}{|w_g|^2} \cdot \frac{\epsilon |w_g|^2}{2}\\
& = \epsilon
\end{align*}
as required.

\end{proof}


\begin{problem}(problem 1b) 
Use the Reverse Triangle Inequality $\left(|z_2 - z_1| \geq |z_2| -|z_1|\right)$ to prove 
that if $|g(z) - w_g| < \epsilon'$, then $|g(z)| > |w_g| - \epsilon'$.
\end{problem}

Here is a video solution of problem 1b:\\
\begin{foldable}
\youtube{0iA2lzjdJo0}
\end{foldable}

\begin{problem}(problem 1c) 
Prove parts b and c of the theorem above.
\end{problem}

Here is a video solution of problem 1c (multiplication only):\\
\begin{foldable}
\youtube{v2YBmYYDKTo}
\end{foldable}

\begin{problem}
Use parts a and b of the theorem above to show that the limit of a difference is the difference
of the limits.
\end{problem}


\begin{problem}
Use parts a, b and c of the theorem above to prove that for a polynomial $p(z)$
\[
\lim_{z\to z_0} p(z) = p(z_0)
\]
for any $z_0 \in \C$.
\end{problem}

The limit of a complex function can be determined from the limits of its real and imaginary parts.
\begin{proposition}
Let $z_0 = x_0 + iy_0 \in \C$ and suppose $f(z)=f(x+iy) = u(x,y) + iv(x,y)$ is defined 
on $D_\circ(z_0,r)$ for some $r>0$. Let $w_0 = u_0 + iv_0$, then
\[
\lim_{z \to z_0} f(z) = w_0 
\]
if and only if
\[
\lim_{(x,y) \to (x_0, y_0)} u(x,y) =  u_0 
\]
and
\[
\lim_{(x,y) \to (x_0, y_0)} v(x,y) =  v_0 
\]
%where $z_0 = x_0 + iy_0$.
\end{proposition}

To prove the proposition, we need to recall that 
\[
|\Rep z|, |\Imp z| \leq |z| \leq |\Rep z| + |\Imp z|
\]
for all $z \in \C$.
Also, we will use the notation 
\[
|(x,y) - (x_0,y_0)|
\]
 for the distance between the points
$(x,y)$ and $(x_0,y_0)$ in $\R^2$.  Thus
\[
|(x,y) - (x_0,y_0)| = |(x+iy) - (x_0 + iy_0)|
\]

%For ease of readability we will also abbreviate $u(x,y)$ and $v(x,y)$ with $u$  and $v$ respectively. 

\begin{proof} %We now prove the proposition.\\[6pt]
Suppose $\lim_{z \to z_0} f(z) = w_0 = u_0 + iv_0$.
Then for any $\epsilon > 0$ there exists $\delta > 0$ such that
\[
0< |z-z_0| < \delta \Rightarrow |f(z) - w_0| < \epsilon
\]

%In particular, this statement holds for all $|z - z_0|< \epsilon$ with the additional constraint that $y = y_0$

This statement can be rewritten as
\[
|(x+iy)-(x_0+i y_0)| < \delta \Rightarrow |u(x,y) + iv(x,y) - (u_0 + iv_0)|< \epsilon
\]
Since $|u(x,y) - u_0| \leq |u(x,y) + iv(x,y) - (u_0 + iv_0)|$
we have
\[
|(x,y)-(x_0, y_0)| < \delta \Rightarrow |u(x,y)  - u_0 |< \epsilon
\]
Thus
\[
\lim_{(x,y) \to (x_0, y_0)} u(x,y) =  u_0 
\]
 

%since $|v(x,y) - v_0|<  |f(z) -w_0|$
%we have
%\[
%|(x,y)-(x_0, y_0)| < \delta \Rightarrow |v(x,y)  - v_0 |< \epsilon
%\]
%Thus

Similarly,
\[
\lim_{(x,y) \to (x_0, y_0)} v(x,y) =  v_0 
\]
Next, suppose
\[
\lim_{(x,y) \to (x_0, y_0)} u(x,y) =  u_0  \quad \mbox{and} \quad \lim_{(x,y) \to (x_0, y_0)} v(x,y) =  v_0 
\]
Then for $\epsilon > 0$, there exists $\delta_u > 0$ and $\delta_v > 0$ such that
\[
0<|(x,y)-(x_0, y_0)| < \delta_u \Rightarrow |u(x,y)  - u_0 |< \frac{\epsilon}{2}
\]
and
\[
0<|(x,y)-(x_0, y_0)| < \delta_v \Rightarrow |v(x,y)  - v_0 |< \frac{\epsilon}{2}
\]
Let $\delta = \min\{\delta_u, \delta_v\}$ and suppose $|z-z_0| = |(x+iy) - (x_0 + iy_0)| <  \delta$. Then 

%by the Triangle Inequality

\[
|f(z) - w_0| \leq  |u(x,y)  - u_0 | + |v(x,y)  - v_0 | < \frac{\epsilon}{2}+ \frac{\epsilon}{2} = \epsilon
\]
which shows that
\[
\lim_{z \to z_0} f(z) = w_0
\]
\end{proof}

Since complex limits are equivalent to two real limits of two variables, we now examine some limits of a real functions of two variables.

\begin{example}[example 2]
Find
\[
\lim_{(x,y) \to (0,0)} \frac{x^3}{x^2 + y^2}
\]
if it exists.\\
We will use the Squeeze Theorem.  
Note that 
\[
0 \leq \frac{x^2}{x^2 + y^2} \leq 1
\]
for any $(x,y) \neq (0,0)$. Furthermore
\[
-|x| \leq x \leq |x|
\]
for any $x \in \R$. Hence,
\[
-|x| \leq - \frac{|x|x^2}{x^2 + y^2}\leq \frac{x^3}{x^2 + y^2} \leq  \frac{|x|x^2}{x^2 + y^2} \leq |x|
\]
Finally, since
\[
\lim_{(x,y) \to (0,0)} -|x| = 0 \;\; \mbox{and} \;\; \lim_{(x,y) \to (0,0)} |x| = 0
\]
we can use the Squeeze Theorem to can conclude that
\[
\lim_{(x,y) \to (0,0)} \frac{x^3}{x^2 + y^2} = 0
\]
as well.
\end{example}

\begin{problem}(problem 2a)
Find
\[
\lim_{(x,y) \to (0,0)} \frac{x^2y}{x^2 + y^2} =\answer{0}
\]
if it exists.
\end{problem}


\begin{problem}(problem 2b)
Find
\[
\lim_{(x,y) \to (0,0)} \frac{2x^2y^3}{x^4 + y^4} =\answer{0}
\]
if it exists.
\begin{hint}
$a^2 + b^2 \geq 2ab$ for any $a, b \in \R$
\end{hint}

\end{problem}

Here is a video solution of problem 2b:\\
\begin{foldable}
\youtube{Avtmml-7rlk}
\end{foldable}


In the next example we show that a limit does not exist because different paths lead to different limits. This is akin to 
a two-sided limit not existing in the single variable case when the one-sided are different.

\begin{example}[example 3]
Find
\[
\lim_{(x,y) \to (0,0)} \frac{xy}{x^2 + y^2}
\]
if it exists.\\
We will let $(x,y)$ approach $(0,0)$ along different lines. If we let $y = mx, m \in \R$, then the limit becomes
\begin{align*}
\lim_{(x,y) \to (0,0)} \frac{xy}{x^2 + y^2} &= \lim_{(x,y) \to (0,0)} \frac{mx^2}{x^2 + (mx)^2} \\
                                            &= \lim_{(x,y) \to (0,0)} \frac{mx^2}{(1+m^2)x^2}\\
                                            &  = \frac{m}{1+m^2}
\end{align*}
The limit depends on the path, and therefore the limit does not exist.
\end{example}

\begin{problem}(problem 3)
Find the limit, if it exists.
\begin{align*}
i) \; & \lim_{(x,y) \to (0,0)} \frac{y^2}{x^2 + y^2} \\
ii) \; & \lim_{(x,y) \to (0,0)} \frac{x}{x^2 + y^2}\\
iii) \; & \lim_{(x,y) \to (0,0)} \frac{x^2y}{x^4 + y^2} \;\; \mbox{(hint: use parabolas instead of lines)}
\end{align*}
\end{problem}

Here is a video solution of problem 3, part iii:\\
\begin{foldable}
\youtube{rw5q7qIq_5E}
\end{foldable}

\section{Continuity}
Continuity is a significant application of limits. Recall that we define a function of a single real 
variable to be continuous at $x = x_0$
if 
\[
\lim_{x \to x_0} f(x) = f(x_0)
\]
We define continuity for a complex function analogously.

\begin{definition}
Let $f$ be a complex function defined on the disk $D(z_0, r)$ for some $r>0$. We say that $f$ is continuous at $z_0$ if
\[
\lim_{z \to z_0} f(z) = f(z_0)
\]
\end{definition}

\begin{example}[example 4]
If $p(z)$ is a polynomial (with complex coefficients), then $p$ is continuous in $\C$ (see the related problem above).
\end{example}

\begin{proposition}
A complex function $f(z) = u(x,y) + iv(x,y)$ is continuous at $z_0$ if and only if
$u(x,y)$ and $v(x,y)$ are continuous at $(x_0,y_0)$.
\end{proposition}

The proof of this proposition is a direct application of the earlier proposition relating limits 
of a complex function to the limits of its real and imaginary parts.\\
Recalling that the real exponential and trigonometric functions are continuous on their domains makes it easy to see that their complex analogues are also continuous.

\begin{example}[example 5]
The complex exponential function $e^z$ is continuous on $\C$.\\
Since
\[
e^z = e^x \cis y = e^x \cos y + ie^x \sin y
\]
and the real functions $e^x \cos y$ and $e^x \sin y$ are continuous on $\R^2$,
the complex exponential function, $e^z$, is continuous on $\C$.
\end{example}

\begin{problem}(problem 5)
Show that the functions $\sin z$ and $\cos z$ are continuous on $\C$.
\end{problem}

\begin{example}[example 6]
The principal branch of the complex logarithm, $\Log z$ is not continuous on the non-positive real axis.\\
To see this, first note that $\Log 0$ is undefined, so $\Log z$ is not continuous at $0$. Now let $c<0$ be any negative real number.
Then $\Log c = \ln |c| + \pi i$, but if we compute the limit as $z \to c$ along a path in the third quadrant, we get
\[
\lim_{z \to c \atop y < 0} \Log z = \ln|c| - \pi i \neq \Log c
\]
\end{example}

\begin{problem}(problem 6)
Show that the principal argument function, $\Arg z$, is not continuous on the negative real axis.
\end{problem}

Here is a video solution of problem 6:\\
\begin{foldable}
\youtube{p3KdyhYc2hQ}
\end{foldable}

\end{document}









