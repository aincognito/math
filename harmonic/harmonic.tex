\documentclass[handout]{ximera}

%% You can put user macros here
%% However, you cannot make new environments



\newcommand{\ffrac}[2]{\frac{\text{\footnotesize $#1$}}{\text{\footnotesize $#2$}}}
\newcommand{\vasymptote}[2][]{
    \draw [densely dashed,#1] ({rel axis cs:0,0} -| {axis cs:#2,0}) -- ({rel axis cs:0,1} -| {axis cs:#2,0});
}


%\usepackage{tcolorbox} %%Needed for Derivative Definition supposedly and product rule, natural exp log, quotient rule, inverse trig, rates of change


% \graphicspath{{./}{firstExample/}}
% \usepackage{forest}
\usepackage{amsmath}
\usepackage{amssymb}
\usepackage{array}
\usepackage[makeroom]{cancel} %% for strike outs
\usepackage{pgffor} %% required for integral for loops
\usepackage{tikz}
\usepackage{tikz-cd}
\usepackage{tkz-euclide}
\usetikzlibrary{shapes.multipart}


% \usetkzobj{all}
\tikzstyle geometryDiagrams=[ultra thick,color=blue!50!black]


\usetikzlibrary{arrows}
\tikzset{>=stealth,commutative diagrams/.cd,
  arrow style=tikz,diagrams={>=stealth}} %% cool arrow head
\tikzset{shorten <>/.style={ shorten >=#1, shorten <=#1 } } %% allows shorter vectors

\usetikzlibrary{backgrounds} %% for boxes around graphs
\usetikzlibrary{shapes,positioning}  %% Clouds and stars
\usetikzlibrary{matrix} %% for matrix
\usepgfplotslibrary{polar} %% for polar plots
\usepgfplotslibrary{fillbetween} %% to shade area between curves in TikZ



%\usepackage[width=4.375in, height=7.0in, top=1.0in, papersize={5.5in,8.5in}]{geometry}
%\usepackage[pdftex]{graphicx}
%\usepackage{tipa}
%\usepackage{txfonts}
%\usepackage{textcomp}
%\usepackage{amsthm}
%\usepackage{xy}
%\usepackage{fancyhdr}
%\usepackage{xcolor}
%\usepackage{mathtools} %% for pretty underbrace % Breaks Ximera
%\usepackage{multicol}



\newcommand{\RR}{\mathbb R}
\newcommand{\R}{\mathbb R}
\newcommand{\C}{\mathbb C}
\newcommand{\N}{\mathbb N}
\newcommand{\Z}{\mathbb Z}
\newcommand{\dis}{\displaystyle}
%\renewcommand{\d}{\,d\!}
\renewcommand{\d}{\mathop{}\!d}
\newcommand{\dd}[2][]{\frac{\d #1}{\d #2}}
\newcommand{\pp}[2][]{\frac{\partial #1}{\partial #2}}
\renewcommand{\l}{\ell}
\newcommand{\ddx}{\frac{d}{\d x}}
\newcommand{\ppx}{\frac{\partial}{\partial x}}
\newcommand{\ppy}{\frac{\partial}{\partial y}}

\newcommand{\zeroOverZero}{\ensuremath{\boldsymbol{\tfrac{0}{0}}}}
\newcommand{\inftyOverInfty}{\ensuremath{\boldsymbol{\tfrac{\infty}{\infty}}}}
\newcommand{\zeroOverInfty}{\ensuremath{\boldsymbol{\tfrac{0}{\infty}}}}
\newcommand{\zeroTimesInfty}{\ensuremath{\small\boldsymbol{0\cdot \infty}}}
\newcommand{\inftyMinusInfty}{\ensuremath{\small\boldsymbol{\infty - \infty}}}
\newcommand{\oneToInfty}{\ensuremath{\boldsymbol{1^\infty}}}
\newcommand{\zeroToZero}{\ensuremath{\boldsymbol{0^0}}}
\newcommand{\inftyToZero}{\ensuremath{\boldsymbol{\infty^0}}}


\newcommand{\numOverZero}{\ensuremath{\boldsymbol{\tfrac{\#}{0}}}}
\newcommand{\dfn}{\textbf}
%\newcommand{\unit}{\,\mathrm}
\newcommand{\unit}{\mathop{}\!\mathrm}
%\newcommand{\eval}[1]{\bigg[ #1 \bigg]}
\newcommand{\eval}[1]{ #1 \bigg|}
\newcommand{\seq}[1]{\left( #1 \right)}
\renewcommand{\epsilon}{\varepsilon}
\renewcommand{\iff}{\Leftrightarrow}

\DeclareMathOperator{\arccot}{arccot}
\DeclareMathOperator{\arcsec}{arcsec}
\DeclareMathOperator{\arccsc}{arccsc}
\DeclareMathOperator{\si}{Si}
\DeclareMathOperator{\proj}{proj}
\DeclareMathOperator{\scal}{scal}
\DeclareMathOperator{\cis}{cis}
\DeclareMathOperator{\Arg}{Arg}
%\DeclareMathOperator{\arg}{arg}
\DeclareMathOperator{\Rep}{Re}
\DeclareMathOperator{\Imp}{Im}
\DeclareMathOperator{\sech}{sech}
\DeclareMathOperator{\csch}{csch}
\DeclareMathOperator{\Log}{Log}

\newcommand{\tightoverset}[2]{% for arrow vec
  \mathop{#2}\limits^{\vbox to -.5ex{\kern-0.75ex\hbox{$#1$}\vss}}}
\newcommand{\arrowvec}{\overrightarrow}
\renewcommand{\vec}{\mathbf}
\newcommand{\veci}{{\boldsymbol{\hat{\imath}}}}
\newcommand{\vecj}{{\boldsymbol{\hat{\jmath}}}}
\newcommand{\veck}{{\boldsymbol{\hat{k}}}}
\newcommand{\vecl}{\boldsymbol{\l}}
\newcommand{\utan}{\vec{\hat{t}}}
\newcommand{\unormal}{\vec{\hat{n}}}
\newcommand{\ubinormal}{\vec{\hat{b}}}

\newcommand{\dotp}{\bullet}
\newcommand{\cross}{\boldsymbol\times}
\newcommand{\grad}{\boldsymbol\nabla}
\newcommand{\divergence}{\grad\dotp}
\newcommand{\curl}{\grad\cross}
%% Simple horiz vectors
\renewcommand{\vector}[1]{\left\langle #1\right\rangle}


\pgfplotsset{compat=1.13}

\outcome{Create harmonic conjugates}

\title{3.5 Harmonic Functions}

\begin{document}

\begin{abstract}
We determine and create harmonic functions.
\end{abstract}

\maketitle


\begin{definition}
A function $u(x,y)$ is called {\bf harmonic} on a disk $D \subset \R^2$ if it the second order partial derivatives
$u_{xx}$ and $u_{yy}$ exist on $D$ and satisfy the equation
\[
\Delta u = u_{xx} + u_{yy} = 0
\]
on $D$.
\end{definition}

\begin{example}
The function $u(x,y) = 3x + 5y +1$ is harmonic in the entire plane, $\R^2$ since
\[
u_{xx} \quad \text{and} \quad u_{yy} = 0
\]
everywhere. Hence $\Delta u = 0$ on $\R^2$ and $u$ is harmonic everywhere.
\end{example}

\begin{problem}
Select all of the functions that are harmonic on $\R^2$?\\
\begin{selectAll}
\choice[correct]{$u(x,y) = 2x -3y$}
\choice[correct]{$u(x,y) = xy$}
\choice{$u(x,y) = x^2 + y^2$}
\choice[correct]{$u(x,y) = x^2 - y^2$}
\end{selectAll}
\end{problem}


\begin{theorem}
If $f(z) = u(x,y) + iv(x,y)$ is analytic in a disk $D$, and if the second order 
partial derivatives of $u$ and $v$ exist and are continuous on $D$ then $u$ and $v$ are both harmonic on $D$.
\end{theorem}
\begin{proof}
We prove that $u$ is harmonic on $D$ and leave the proof of $v$ as an exercise.\\
By the Cauchy-Riemann equations, $u_y = -v_x$ on $D$.  Therefore, $u_{yy} = -v_{xy} = -v_{yx}$ where the 
second equality follows from the symmetry of mixed partial derivatives. Applying the Cauchy-Riemann 
equations (specifically $v_y = u_x$)  gives
\[
u_{yy} = -v_{yx} = -u_{xx}
\]
on $D$. Hence
\[
\Delta u = u_{xx} + u_{yy} = u_{xx} - u_{xx} = 0
\]
and $u(x,y)$ is harmonic on $D$.
\end{proof}

\begin{problem}
Suppose $f(z) = u(x,y) + iv(x,y)$ satisfies the hypotheses of the above theorem on a disk $D$. 
Prove $v$ is harmonic on $D$.\\
Note that $v$ is the real part of which function?
\begin{multipleChoice}
\choice{$-f(z)$}
\choice{$i f(z)$}
\choice[correct]{$-i f(z)$}
\end{multipleChoice}
Is this function analytic on $D$? \wordChoice{\choice[correct]{yes}\choice{no}}\\
How do we know $v$ is harmonic on $D$?
\end{problem}
  
  
\begin{theorem}
Suppose $u(x,y)$ is harmonic on a disk $D$.  Then there exists a complex function $f(z)$, analytic on $D$, 
such that $u(x,y)$ is the real part of $f$. The imaginary part of $f$ is called the {\bf harmonic conjugate} of $u$.
\end{theorem}


\begin{example}
Find the harmonic conjugate of $u(x,y) = x^3 - 3xy^2$.\\
First, we should verify that $u$ is harmonic:
\[
\Delta u = u_{xx} + u_{yy} = (6x) + (6x) = 0
\]
so $u$ is harmonic on $\C$. According to the theorem we can find $v(x,y)$ such that
$f(z) = u(x,y) + iv(x,y)$ is analytic (entire in this case). To construct $v$, we begin by noting that
$v_y = u_x = 3x^2 - 3y^2$ since $u$ and $v$ must satisfy the Cauchy-Riemann equations.
Now, we integrate with respect to $y$ treating $x$ as a constant:
\[
v(x,y) = \int \left(3x^2 -3y^2\right) \; dy
\]
which yields
\[
v(x,y) = 3x^2y - y^3 + C(x)
\]
Note that the constant of integration is written as $C(x)$ since we were assuming in 
the integral that $x$ was constant. To determine $C(x)$, we use the other Cauchy-Riemann equation. 
On one hand, $v_x = -u_y = 6xy$ and on the other hand, since we have established a form for $v$, we can use it
to obtain $v_x = 6xy + C'(x)$. Equating these two expressions for $v_x$ we see 
that $C'(x) =0$ and hence $C(x)$ is a constant. Thus the harmonic conjugate of $u(x,y) = x^3 - 3xy^2$ is
$v(x,y) = 3x^2y - y^3 + C$ where $C$ is any complex constant.
\end{example}

\begin{remark}
In general, a harmonic function $u$ has infinitely many harmonic conjugates, each pair differing by a constant.
\end{remark}

\begin{problem}
Verify that each function is harmonic on $\C$ and then find its harmonic conjugate.
\begin{align*}
a) \;\;u(x,y) &= x^4 -6x^2y^2 + y^4\\
v_y &= \answer{4x^3 - 12xy^2}\\
v(x,y) &= \answer{4x^3y - 4xy^3}+C\\
b) \;\;u(x,y) &= e^x\sin y\\
v_y &= \answer{4x^3 - 12xy^2}\\
v(x,y) &= \answer{4x^3y - 4xy^3}+C\\
c) \;\;u(x,y) &= \cos x \cosh y\\
v_y &= \answer{4x^3 - 12xy^2}\\
v(x,y) &= \answer{4x^3y - 4xy^3}+C\\
d) \;\; u(x,y) &= \sinh x \sin y\\
v_y &= \answer{4x^3 - 12xy^2}\\
v(x,y) &= \answer{4x^3y - 4xy^3}+C
\end{align*}
\end{problem}



\end{document}

