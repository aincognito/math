\documentclass[handout]{ximera}

%% You can put user macros here
%% However, you cannot make new environments



\newcommand{\ffrac}[2]{\frac{\text{\footnotesize $#1$}}{\text{\footnotesize $#2$}}}
\newcommand{\vasymptote}[2][]{
    \draw [densely dashed,#1] ({rel axis cs:0,0} -| {axis cs:#2,0}) -- ({rel axis cs:0,1} -| {axis cs:#2,0});
}


\graphicspath{{./}{firstExample/}}
\usepackage{forest}
\usepackage{amsmath}
\usepackage{amssymb}
\usepackage{array}
\usepackage[makeroom]{cancel} %% for strike outs
\usepackage{pgffor} %% required for integral for loops
\usepackage{tikz}
\usepackage{tikz-cd}
\usepackage{tkz-euclide}
\usetikzlibrary{shapes.multipart}


%\usetkzobj{all}
\tikzstyle geometryDiagrams=[ultra thick,color=blue!50!black]


\usetikzlibrary{arrows}
\tikzset{>=stealth,commutative diagrams/.cd,
  arrow style=tikz,diagrams={>=stealth}} %% cool arrow head
\tikzset{shorten <>/.style={ shorten >=#1, shorten <=#1 } } %% allows shorter vectors

\usetikzlibrary{backgrounds} %% for boxes around graphs
\usetikzlibrary{shapes,positioning}  %% Clouds and stars
\usetikzlibrary{matrix} %% for matrix
\usepgfplotslibrary{polar} %% for polar plots
\usepgfplotslibrary{fillbetween} %% to shade area between curves in TikZ



%\usepackage[width=4.375in, height=7.0in, top=1.0in, papersize={5.5in,8.5in}]{geometry}
%\usepackage[pdftex]{graphicx}
%\usepackage{tipa}
%\usepackage{txfonts}
%\usepackage{textcomp}
%\usepackage{amsthm}
%\usepackage{xy}
%\usepackage{fancyhdr}
%\usepackage{xcolor}
%\usepackage{mathtools} %% for pretty underbrace % Breaks Ximera
%\usepackage{multicol}



\newcommand{\RR}{\mathbb R}
\newcommand{\R}{\mathbb R}
\newcommand{\C}{\mathbb C}
\newcommand{\N}{\mathbb N}
\newcommand{\Z}{\mathbb Z}
\newcommand{\dis}{\displaystyle}
%\renewcommand{\d}{\,d\!}
\renewcommand{\d}{\mathop{}\!d}
\newcommand{\dd}[2][]{\frac{\d #1}{\d #2}}
\newcommand{\pp}[2][]{\frac{\partial #1}{\partial #2}}
\renewcommand{\l}{\ell}
\newcommand{\ddx}{\frac{d}{\d x}}

\newcommand{\zeroOverZero}{\ensuremath{\boldsymbol{\tfrac{0}{0}}}}
\newcommand{\inftyOverInfty}{\ensuremath{\boldsymbol{\tfrac{\infty}{\infty}}}}
\newcommand{\zeroOverInfty}{\ensuremath{\boldsymbol{\tfrac{0}{\infty}}}}
\newcommand{\zeroTimesInfty}{\ensuremath{\small\boldsymbol{0\cdot \infty}}}
\newcommand{\inftyMinusInfty}{\ensuremath{\small\boldsymbol{\infty - \infty}}}
\newcommand{\oneToInfty}{\ensuremath{\boldsymbol{1^\infty}}}
\newcommand{\zeroToZero}{\ensuremath{\boldsymbol{0^0}}}
\newcommand{\inftyToZero}{\ensuremath{\boldsymbol{\infty^0}}}


\newcommand{\numOverZero}{\ensuremath{\boldsymbol{\tfrac{\#}{0}}}}
\newcommand{\dfn}{\textbf}
%\newcommand{\unit}{\,\mathrm}
\newcommand{\unit}{\mathop{}\!\mathrm}
%\newcommand{\eval}[1]{\bigg[ #1 \bigg]}
\newcommand{\eval}[1]{ #1 \bigg|}
\newcommand{\seq}[1]{\left( #1 \right)}
\renewcommand{\epsilon}{\varepsilon}
\renewcommand{\iff}{\Leftrightarrow}

\DeclareMathOperator{\arccot}{arccot}
\DeclareMathOperator{\arcsec}{arcsec}
\DeclareMathOperator{\arccsc}{arccsc}
\DeclareMathOperator{\si}{Si}
\DeclareMathOperator{\proj}{proj}
\DeclareMathOperator{\scal}{scal}
\DeclareMathOperator{\cis}{cis}
\DeclareMathOperator{\Arg}{Arg}
%\DeclareMathOperator{\arg}{arg}
\DeclareMathOperator{\Rep}{Re}
\DeclareMathOperator{\Imp}{Im}
\DeclareMathOperator{\sech}{sech}
\DeclareMathOperator{\csch}{csch}
\DeclareMathOperator{\Log}{Log}

\newcommand{\tightoverset}[2]{% for arrow vec
  \mathop{#2}\limits^{\vbox to -.5ex{\kern-0.75ex\hbox{$#1$}\vss}}}
\newcommand{\arrowvec}{\overrightarrow}
\renewcommand{\vec}{\mathbf}
\newcommand{\veci}{{\boldsymbol{\hat{\imath}}}}
\newcommand{\vecj}{{\boldsymbol{\hat{\jmath}}}}
\newcommand{\veck}{{\boldsymbol{\hat{k}}}}
\newcommand{\vecl}{\boldsymbol{\l}}
\newcommand{\utan}{\vec{\hat{t}}}
\newcommand{\unormal}{\vec{\hat{n}}}
\newcommand{\ubinormal}{\vec{\hat{b}}}

\newcommand{\dotp}{\bullet}
\newcommand{\cross}{\boldsymbol\times}
\newcommand{\grad}{\boldsymbol\nabla}
\newcommand{\divergence}{\grad\dotp}
\newcommand{\curl}{\grad\cross}
%% Simple horiz vectors
\renewcommand{\vector}[1]{\left\langle #1\right\rangle}


\outcome{Determine the behavior of a Telescoping Series}

\title{3.3 Telescoping Series}

\begin{document}

\begin{abstract}
We determine the convergence or divergence of a Telescoping Series.
\end{abstract}

\maketitle

\section{Telescoping Series}

A telescoping series is a special type of series whose terms cancel each out in such a way that it is relatively easy to determine the
exact value of its partial sums. In general, such a series is formed as a sum of differences:
\begin{align*}
\sum_{n=1}^\infty \left(a_n -a_{n+1}\right) &= \lim_{N \to \infty} \sum_{n=1}^N \left(a_n -a_{n+1}\right)\\
&=\lim_{N \to \infty}\left[\left(a_1 - a_2\right) + \left(a_2 - a_3\right) + \cdots + \left(a_N - a_{N+1}\right) \right]\\
&= \lim_{N \to \infty}\left[a_1 - a_{N+1}\right]
\end{align*}

In practice, creating the telescoping effect frequently involves a partial fraction decomposition.

\begin{example}[example 1]
Consider the series
\[
\sum_{n=1}^\infty \frac{1}{n^2 +n}
\]

The $N^{th}$ partial sum is given by

\[
S_N = \sum_{n=1}^N \frac{1}{n^2 +n} = \frac{1}{2} + \frac{1}{6} + \frac{1}{12} + \cdots + \frac{1}{N^2 - N} + \frac{1}{N^2 + N}
\]
Note that the second to last term above comes from letting $n = N-1$, so that in the denominator we have 
\[
n^2 + n = (N-1)^2 + (N-1) = N^2 - 2N + 1 + N - 1 = N^2 -N.
\]
There does not appear to be an easier way to express this partial sum.
However, if we rewrite $\frac{1}{n^2 +n}$ using a partial fraction decomposition, we will see a nice formula for the partial sum.
The decomposition is:
\[
\frac{1}{n^2 + n} = \frac{1}{n(n+1)} = \frac{A}{n} + \frac{B}{n+1}, 
\]
with $A = 1$ and $B=-1$ (verify).
Now rewrite the $N^{th}$ partial sum using this decomposition:
\begin{align*}
S_N = \sum_{n=1}^N \frac{1}{n^2 +n} =&\sum_{n=1}^N \left(\frac{1}{n}-\frac{1}{n+1}\right)\\
=& \left(1  - \frac12 \right) + \left(\frac12 - \frac13 \right) + \left(\frac13 - \frac14 \right) + \cdots \\
&+ \left(\frac{1}{N-1} - \frac{1}{N} \right) +\left(\frac{1}{N} - \frac{1}{N+1} \right) 
\end{align*}



Looking carefully at the terms in this sum, we see a lot of cancellation. The terms $-\frac12$ and $+\frac12$ cancel as do the pair
 $-\frac13$ and $+\frac13$ and the pair $-\frac{1}{N}$ and $+\frac{1}{N}$. The only fractions that don't cancel are the first, $1$, and the last $-1/(N+1)$.
 The series collapses like an old-time telescope and the $N^{th}$ partial sum is:
\[
S_N = \sum_{n=1}^N \frac{1}{n^2 +n} = 1 - \frac{1}{N+1}.
\]
We can now find the sum of the series by taking a limit:
\[
S = \sum_{n=1}^\infty  \frac{1}{n^2 +n} = \lim_{N\to \infty} S_N = \lim_{N\to \infty} \left(1 - \frac{1}{N+1}\right) = 1.
\]
Thus, the series converges and its sum is 1.
\end{example}




\begin{example}[example 2]
Consider the series
\[
\sum_{n=1}^\infty \frac{1}{n^2 +2n}
\]

We begin with a partial fraction decomposition:
\[
\frac{1}{n^2 + 2n} = \frac{1}{n(n+2)} = \frac{A}{n} + \frac{B}{n+2}
\]
with $A = 1/2$ and $B=-1/2$ (verify). Thus,
\[
\sum_{n=1}^\infty \frac{1}{n^2 + 2n}  = \frac12 \sum_{n=1}^\infty \left( \frac{1}{n} - \frac{1}{n+2} \right)
\]
Using the telescoping form, we can find $S_N$:
\begin{align*}
 \frac12 \sum_{n=1}^N  \left(\frac{1}{n}-\frac{1}{n+2}\right) &= \frac12 \Bigg[\left(1 - \frac13 \right) + \left(\frac12 - \frac14 \right) 
 + \left(\frac13 - \frac15 \right) + \cdots \\
 & + \left(\frac{1}{N-2} - \frac{1}{N} \right) +\left(\frac{1}{N-1} - \frac{1}{N+1} \right) +\left(\frac{1}{N} - \frac{1}{N+2} \right) \Bigg]
\end{align*}

Observe that the first negative term is $-1/3$ and the last positive term is $1/N$. This means that the first two positive terms, 
$1$ and $1/2$ and the last two negative terms, $-1/(N+1)$ 
and $-1/(N+2)$  will survive the cancellation:
\[
S_N = \frac12 \left[ 1 + \frac12 - \frac{1}{N+1} - \frac{1}{N+2}\right]
\]
We can now find the sum of the series as a limit of its partial sums:
\[
S  = \lim_{N\to \infty} S_N = \frac12 \left[1 + \frac12 - 0- 0\right] =  \frac34
\]
Thus, the series converges and its sum is 3/4.
\end{example}



\begin{problem}(problem 1)
Find a formula for $S_N$, and find the sum of the series (if it converges) or state that it diverges:
\[
\sum_{n=0}^\infty \frac{1}{n^2 + 3n + 2}
\]
The partial fraction decomposition has the form
\[
\frac{A}{n+1} + \frac{B}{n+2}
\]
with 
\[
A = \answer{1} \quad \mbox{and} \quad B = \answer{-1}
\]
The $N^{th}$ partial sum is  $S_N = \sum_{n=0}^N \frac{1}{n^2 + 3n + 2} = \answer{1 - 1/(N+2)}$\\
The limit of the partial sums is $S = \lim_{N\to \infty}S_N = \answer{1}$\\
The series \wordChoice{\choice[correct]{converges}\choice{diverges}} to $\answer{1}$

\end{problem}

\begin{problem}(problem 2)
Find a formula for $S_N$, and find the sum of the series (if it converges) or state that it diverges:
\[
\sum_{n=2}^\infty \frac{1}{n^2 - 1}    
\]

The partial fraction decomposition has the form
\[
\frac{A}{n-1} + \frac{B}{n+1}
\]
with 
\[
A = \answer{1/2} \quad \mbox{and} \quad B = \answer{-1/2}
\]
The $N^{th}$ partial sum is  $S_N = \sum_{n=2}^N \frac{1}{n^2 - 1} =  \answer{3 - 1/(2N) - 1/(2N+2)}$\\
The limit of the partial sums is $S = \lim_{N\to \infty}S_N = \answer{3}$\\
The series \wordChoice{\choice[correct]{converges}\choice{diverges}} to $\answer{3}$

\end{problem}

\begin{problem}(problem 3)
Find a formula for $S_N$, and find the sum of the series (if it converges) or state that it diverges:
\[
\sum_{n=0}^\infty \frac{1}{n^2 + 4n + 3}    
\]

The partial fraction decomposition has the form
\[
\frac{A}{n+1} + \frac{B}{n+3}
\]
with 
\[
A = \answer{1/2} \quad \mbox{and} \quad B = \answer{-1/2}
\]
The $N^{th}$ partial sum is  $S_N = \sum_{n=0}^N \frac{1}{n^2 + 4n + 3} =  \answer{3 -  1/(2N+4)-  1/(2N+6)}$\\
The limit of the partial sums is $S = \lim_{N\to \infty}S_N = \answer{3}$\\
The series \wordChoice{\choice[correct]{converges}\choice{diverges}} to $\answer{3}$

\end{problem}


\begin{problem}(problem 4)
Find a formula for $S_N$, and find the sum of the series (if it converges) or state that it diverges:
\[
\sum_{n=1}^\infty \left( 2^{\frac{1}{n}} - 2^{\frac{1}{n+1}}\right) 
\]
The $N^{th}$ partial sum is  $S_N = \sum_{n=1}^N \left( 2^{\frac{1}{n}} - 2^{\frac{1}{n+1}}\right) =  \answer{2 - 2^{1/(N+1)}}$\\
The limit of the partial sums is $S = \lim_{N\to \infty}S_N = \answer{1}$\\
The series \wordChoice{\choice[correct]{converges}\choice{diverges}} to $\answer{1}$

\end{problem}


\begin{problem}(problem 5)
Find a formula for $S_N$, and find the sum of the series (if it converges) or state that it diverges:
\[
\sum_{n=2}^\infty  \ln\left(1 + \frac{1}{n}\right) 
\]

Writing $1 + \frac{1}{n}$ as a fraction gives 
\begin{multipleChoice}
\choice{$\frac{2}{n}$}\\
\choice{$\frac{2}{n+1}$}\\
\choice[correct]{$\frac{n+1}{n}$}
\end{multipleChoice}

Next, we use the log property 
\[
\ln\frac{u}{v} = \ln u - \ln v
\]
to write 
\[
\ln\left(1 + \frac{1}{n}\right)
\]
as 
\begin{multipleChoice}
\choice{$\ln n - \ln(n+1)$}\\
\choice[correct]{$\ln(n+1) - \ln n$}\\
\choice{$1 - \ln n$}
\end{multipleChoice}

The $N^{th}$ partial sum is  $S_N = \sum_{n=2}^N \ln\left(1 + \frac{1}{n}\right) 
 =  \answer{\ln(N+1)}$\\

The limit of the partial sums is $S = \lim_{N\to \infty}S_N = \answer{\infty}$\\

The series \wordChoice{\choice{converges}\choice[correct]{diverges}}

\end{problem}


\begin{problem}(problem 6)
Find a formula for $S_N$, and find the sum of the series (if it converges) or state that it diverges:
\[
\sum_{n=1}^\infty  \frac{1}{\sqrt n + \sqrt{n+1}} 
\]
Multiplying by the conjugate radical, $\sqrt{n+1} - \sqrt{n}$, over itself yields
\[
\frac{1}{\sqrt n + \sqrt{n+1}} \cdot \frac{\sqrt{n+1} - \sqrt{n}}{\sqrt{n+1} - \sqrt{n}} = \answer{\sqrt{n+1} -\sqrt{n}}
\]

The $N^{th}$ partial sum is  $S_N = \sum_{n=1}^N \frac{1}{\sqrt n + \sqrt{n+1}} 
 =  \answer{\sqrt{N+1} - 1}$\\

The limit of the partial sums is $S = \lim_{N\to \infty}S_N = \answer{\infty}$\\

The series \wordChoice{\choice{converges}\choice[correct]{diverges}}

\end{problem}

\begin{problem}(problem 6)
Find a formula for $S_N$, and find the sum of the series (if it converges) or state that it diverges:
\[
\sum_{n=0}^\infty  (-1)^n 
\]

If $N$ is even, then the $N^{th}$ partial sum is  $S_N = \sum_{n=0}^N  (-1)^n
 =  \answer{1}$\\

If $N$ is odd, then the $N^{th}$ partial sum is  $\displaystyle S_N = \sum_{n=0}^N  (-1)^n
 =  \answer{0}$\\


The limit of the partial sums is $\displaystyle S = \lim_{N\to \infty}S_N = \answer{DNE}$\\

The series \wordChoice{\choice{converges}\choice[correct]{diverges}}

\end{problem}




\end{document}







