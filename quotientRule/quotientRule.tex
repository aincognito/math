\documentclass[handout]{ximera}
% \usepackage{tcolorbox}
%% You can put user macros here
%% However, you cannot make new environments



\newcommand{\ffrac}[2]{\frac{\text{\footnotesize $#1$}}{\text{\footnotesize $#2$}}}
\newcommand{\vasymptote}[2][]{
    \draw [densely dashed,#1] ({rel axis cs:0,0} -| {axis cs:#2,0}) -- ({rel axis cs:0,1} -| {axis cs:#2,0});
}


\graphicspath{{./}{firstExample/}}
\usepackage{forest}
\usepackage{amsmath}
\usepackage{amssymb}
\usepackage{array}
\usepackage[makeroom]{cancel} %% for strike outs
\usepackage{pgffor} %% required for integral for loops
\usepackage{tikz}
\usepackage{tikz-cd}
\usepackage{tkz-euclide}
\usetikzlibrary{shapes.multipart}


%\usetkzobj{all}
\tikzstyle geometryDiagrams=[ultra thick,color=blue!50!black]


\usetikzlibrary{arrows}
\tikzset{>=stealth,commutative diagrams/.cd,
  arrow style=tikz,diagrams={>=stealth}} %% cool arrow head
\tikzset{shorten <>/.style={ shorten >=#1, shorten <=#1 } } %% allows shorter vectors

\usetikzlibrary{backgrounds} %% for boxes around graphs
\usetikzlibrary{shapes,positioning}  %% Clouds and stars
\usetikzlibrary{matrix} %% for matrix
\usepgfplotslibrary{polar} %% for polar plots
\usepgfplotslibrary{fillbetween} %% to shade area between curves in TikZ



%\usepackage[width=4.375in, height=7.0in, top=1.0in, papersize={5.5in,8.5in}]{geometry}
%\usepackage[pdftex]{graphicx}
%\usepackage{tipa}
%\usepackage{txfonts}
%\usepackage{textcomp}
%\usepackage{amsthm}
%\usepackage{xy}
%\usepackage{fancyhdr}
%\usepackage{xcolor}
%\usepackage{mathtools} %% for pretty underbrace % Breaks Ximera
%\usepackage{multicol}



\newcommand{\RR}{\mathbb R}
\newcommand{\R}{\mathbb R}
\newcommand{\C}{\mathbb C}
\newcommand{\N}{\mathbb N}
\newcommand{\Z}{\mathbb Z}
\newcommand{\dis}{\displaystyle}
%\renewcommand{\d}{\,d\!}
\renewcommand{\d}{\mathop{}\!d}
\newcommand{\dd}[2][]{\frac{\d #1}{\d #2}}
\newcommand{\pp}[2][]{\frac{\partial #1}{\partial #2}}
\renewcommand{\l}{\ell}
\newcommand{\ddx}{\frac{d}{\d x}}

\newcommand{\zeroOverZero}{\ensuremath{\boldsymbol{\tfrac{0}{0}}}}
\newcommand{\inftyOverInfty}{\ensuremath{\boldsymbol{\tfrac{\infty}{\infty}}}}
\newcommand{\zeroOverInfty}{\ensuremath{\boldsymbol{\tfrac{0}{\infty}}}}
\newcommand{\zeroTimesInfty}{\ensuremath{\small\boldsymbol{0\cdot \infty}}}
\newcommand{\inftyMinusInfty}{\ensuremath{\small\boldsymbol{\infty - \infty}}}
\newcommand{\oneToInfty}{\ensuremath{\boldsymbol{1^\infty}}}
\newcommand{\zeroToZero}{\ensuremath{\boldsymbol{0^0}}}
\newcommand{\inftyToZero}{\ensuremath{\boldsymbol{\infty^0}}}


\newcommand{\numOverZero}{\ensuremath{\boldsymbol{\tfrac{\#}{0}}}}
\newcommand{\dfn}{\textbf}
%\newcommand{\unit}{\,\mathrm}
\newcommand{\unit}{\mathop{}\!\mathrm}
%\newcommand{\eval}[1]{\bigg[ #1 \bigg]}
\newcommand{\eval}[1]{ #1 \bigg|}
\newcommand{\seq}[1]{\left( #1 \right)}
\renewcommand{\epsilon}{\varepsilon}
\renewcommand{\iff}{\Leftrightarrow}

\DeclareMathOperator{\arccot}{arccot}
\DeclareMathOperator{\arcsec}{arcsec}
\DeclareMathOperator{\arccsc}{arccsc}
\DeclareMathOperator{\si}{Si}
\DeclareMathOperator{\proj}{proj}
\DeclareMathOperator{\scal}{scal}
\DeclareMathOperator{\cis}{cis}
\DeclareMathOperator{\Arg}{Arg}
%\DeclareMathOperator{\arg}{arg}
\DeclareMathOperator{\Rep}{Re}
\DeclareMathOperator{\Imp}{Im}
\DeclareMathOperator{\sech}{sech}
\DeclareMathOperator{\csch}{csch}
\DeclareMathOperator{\Log}{Log}

\newcommand{\tightoverset}[2]{% for arrow vec
  \mathop{#2}\limits^{\vbox to -.5ex{\kern-0.75ex\hbox{$#1$}\vss}}}
\newcommand{\arrowvec}{\overrightarrow}
\renewcommand{\vec}{\mathbf}
\newcommand{\veci}{{\boldsymbol{\hat{\imath}}}}
\newcommand{\vecj}{{\boldsymbol{\hat{\jmath}}}}
\newcommand{\veck}{{\boldsymbol{\hat{k}}}}
\newcommand{\vecl}{\boldsymbol{\l}}
\newcommand{\utan}{\vec{\hat{t}}}
\newcommand{\unormal}{\vec{\hat{n}}}
\newcommand{\ubinormal}{\vec{\hat{b}}}

\newcommand{\dotp}{\bullet}
\newcommand{\cross}{\boldsymbol\times}
\newcommand{\grad}{\boldsymbol\nabla}
\newcommand{\divergence}{\grad\dotp}
\newcommand{\curl}{\grad\cross}
%% Simple horiz vectors
\renewcommand{\vector}[1]{\left\langle #1\right\rangle}


\outcome{Compute the derivative of a quotient.}

\title{2.6 Quotient Rule}

%\newcommand{\ffrac}[2]{\frac{\mbox{\footnotesize $#1$}}{\mbox{\footnotesize $#2$}}}
%\newcommand{\vasymptote}[2][]{\draw [densely dashed,#1] ({rel axis cs:0,0} -| {axis cs:#2,0}) -- ({rel axis cs:0,1} -| {axis cs:#2,0});}


\begin{document}

\begin{abstract}
We compute the derivative of a quotient.
\end{abstract}

\maketitle


\begin{center}

\bf{The Quotient Rule}

\end{center}

\begin{theorem} If $f(x)$ and $g(x)$ are differentiable then
\[\left(\frac{f(x)}{g(x)}\right)' = \frac{g(x)f'(x) - f(x)g'(x)}{g^2(x)}\]
at all points where $g(x) \neq 0$.
\end{theorem}

In words, the derivative of a quotient is the bottom times the derivative of the top minus 
the top times the derivative of the bottom, all over the bottom squared.\\


\begin{example}[example 1]
Find $h'(x)$ if 
\[
h(x) = \frac{x+1}{x-1}.
\]
We have $\displaystyle{h(x) = \frac{f(x)}{g(x)}}$ 
with:
\[f(x) =x+1 \ \text{and} \  g(x)= x-1.\]
To use the quotient rule we need the derivatives:
\[f'(x) = 1 \ \text{and} \  g'(x) = 1.\]
We can now write: 
\begin{align*}
h'(x) &= \frac{g(x)f'(x) - f(x)g'(x)}{g^2(x)}\\
&= \frac{(x-1)(1)- (x+1)(1)}{(x-1)^2}\\
&= \frac{(x-1)-(x+1)}{(x-1)^2} \\
&= -\frac{2}{(x-1)^2}.
\end{align*}
\end{example}



\begin{center}
\begin{foldable}
\unfoldable{Here is a video of Example 1}
\youtube{mQLa_jvXMKM} %vid of example 1
\end{foldable}
\end{center}



\begin{problem}(problem 1a)
  Compute
  \[
  \frac{d}{dx} \left(\frac{x-6}{x+3}\right)
  \]
  
    \begin{hint}
      \[\left(\frac{f}{g}\right)' = \frac{f'g-g'f}{g^2}\]
    \end{hint}
    \begin{hint}
      The derivative of $x-6$ is $1$
    \end{hint}
    \begin{hint}
      The derivative of $x+3$ is $1$
    \end{hint}
    \begin{hint}
      Collect like terms in the numerator
    \end{hint}
		The derivative of $\frac{x-6}{x+3}$ with respect to $x$ is
		 $\answer{\frac{9}{(x+3)^2}}$
		
\end{problem}


\begin{problem}(problem 1b)
  Compute
  \[
  \frac{d}{dx} \left(\frac{2x+1}{3x-5}\right)
  \]
  
    \begin{hint}
      \[\left(\frac{f}{g}\right)' = \frac{f'g-g'f}{g^2}\]
    \end{hint}
    \begin{hint}
      The derivative of $2x+1$ is $2$
    \end{hint}
    \begin{hint}
      The derivative of $3x-5$ is $3$
    \end{hint}
    \begin{hint}
      Collect like terms in the numerator
    \end{hint}
		The derivative of $\frac{2x+1}{3x-5}$ with respect to $x$ is
		 $\answer{-\frac{13}{(3x-5)^2}}$
		
\end{problem}


\begin{problem}(problem 1c)
  Compute
  \[
  \frac{d}{dx} \left(\frac{8x-5}{12x - 7}\right)
  \]
  
    \begin{hint}
      \[\left(\frac{f}{g}\right)' = \frac{f'g-g'f}{g^2}\]
    \end{hint}
    \begin{hint}
      The derivative of $8x-5$ is $8$
    \end{hint}
    \begin{hint}
      The derivative of $12x-7$ is $12$
    \end{hint}
    \begin{hint}
      Collect like terms in the numerator
    \end{hint}
		The derivative of $\frac{8x-5}{12x-7}$ with respect to $x$ is
		 $\answer{\frac{4}{(12x-7)^2}}$
		
\end{problem}





\begin{example}[example 2]
Find $h'(x)$ if 
\[
h(x) = \frac{x^2 - 3x}{x^3 +5}.
\]
We write $\displaystyle{h(x) = \frac{f(x)}{g(x)}}$ 
with:
\[f(x) =x^2 - 3x \ \text{and} \  g(x)= x^3 + 5.\]
 To use the quotient rule we need the derivatives:
\[f'(x) = 2x-3 \ \text{and} \  g'(x) = 3x^2.\]
We can now write: 
\begin{align*}
h'(x) &= \frac{g(x)f'(x) - f(x)g'(x)}{g^2(x)}\\
&= \frac{(x^3 + 5)(2x-3) - (x^2-3x)(3x^2)}{(x^3 + 5)^2}\\
&= \frac{(2x^4 -3x^3 + 10x - 15) -(3x^4-9x^3)}{(x^3 + 5)^2}\\
&= \frac{2x^4 -3x^3 + 10x - 15 -3x^4+9x^3}{(x^3 + 5)^2}\\
&= \frac{-x^4 +6x^3 + 10x - 15}{(x^3 + 5)^2}
\end{align*}
\end{example}


\begin{center}
\begin{foldable}
\unfoldable{Here is a video of Example 2}
\youtube{a0xcaLdlqZY} %vid of example 2
\end{foldable}
\end{center}


\begin{problem}(problem 2a)
  Compute
  \[
  \frac{d}{dx} \left(\frac{x^2 - 2}{x^2 + 1}\right)
  \]
  
    \begin{hint}
      \[\left(\frac{f}{g}\right)' = \frac{f'g-g'f}{g^2}\]
    \end{hint}
    \begin{hint}
      The derivative of $x^2 - 2$ is $2x$
    \end{hint}
    \begin{hint}
      The derivative of $x^2 + 1$ is $2x$
    \end{hint}
    \begin{hint}
      Collect like terms in the numerator
    \end{hint}
		The derivative of $\frac{x^2 - 2}{x^2 + 1}$ with respect to $x$ is
		 $\answer{\frac{6x}{(x^2 + 1)^2}}$
		
\end{problem}


\begin{problem}(problem 2b)
  Compute
  \[
  \frac{d}{dx} \left(\frac{x^2 + x + 4}{x^2 -3x + 1}\right)
  \]
  
    \begin{hint}
      \[\left(\frac{f}{g}\right)' = \frac{f'g-g'f}{g^2}\]
    \end{hint}
    \begin{hint}
      The derivative of $x^2 + x + 4$ is $2x+1$
    \end{hint}
    \begin{hint}
      The derivative of $x^2 -3x + 1$ is $2x-3$
    \end{hint}
    \begin{hint}
      Collect like terms in the numerator
    \end{hint}
		\begin{hint}
      The numerator simplifies to $-4x^2 -6x +13$
    \end{hint}
		The derivative of $\frac{x^2 + x + 4}{x^2 -3x + 1}$ with respect to $x$ is
		 $\answer{\frac{-4x^2 -6x +13}{(x^2 -3x + 1)^2}}$
		
\end{problem}



\begin{example}[example 3]
Find $h'(x)$ if 
\[
h(x) = \frac{\cos(x)}{2x+1}.
\]
We write $\displaystyle{h(x) = \frac{f(x)}{g(x)}}$ 
with: 
\[f(x) = \cos(x) \ \text{and} \  g(x)= 2x+1.\]
To use the quotient rule we need the derivatives:
\[f'(x) = -\sin(x) \ \text{and} \  g'(x) = 2.\]
We can now write: 
\begin{align*}
h'(x) &= \frac{g(x)f'(x) - f(x)g'(x)}{g^2(x)}\\
&= \frac{(2x+1)(-\sin(x))- \cos(x)\cdot 2}{(2x+1)^2}\\
&= -\frac{(2x+1)\sin(x) + 2\cos(x)}{(2x+1)^2}
\end{align*}
\end{example}



\begin{center}
\begin{foldable}
\unfoldable{Here is a video of Example 3}
\youtube{IPYC30SSAro} %vid of example 3
\end{foldable}
\end{center}


\begin{problem}(problem 3)
  Compute
  \[
  \frac{d}{dx} \left(\frac{\sin(x)}{x^2 + 1}\right)
  \]
  
		The derivative of $\frac{\sin(x)}{x^2 + 1}$ with respect to $x$ is
		 $\answer{\frac{(x^2 + 1)\cos(x) - 2x\sin(x)}{(x^2 + 1)^2}}$
		
\end{problem}


\begin{example}[example 4]
Find $h'(x)$ if
\[
h(x) = \frac{e^x}{x^2 + 1}.
\]
We write $\displaystyle{h(x) = \frac{f(x)}{g(x)}}$ 
with:
\[f(x) =e^x \ \text{and} \  g(x)= x^2 + 1.\]
To use the quotient rule we need the derivatives:
\[f'(x) = e^x \ \text{and} \  g'(x) =2x.\] 
We can now write: 
\begin{align*}
h'(x) &= \frac{g(x)f'(x) - f(x)g'(x)}{g^2(x)}\\
&= \frac{(x^2+1)e^x - e^x(2x)}{(x^2 + 1)^2} \\
&= \frac{(x^2 - 2x +1)e^x}{(x^2 + 1)^2} \\
&= \frac{e^x(x-1)^2}{(x^2 + 1)^2} 
\end{align*}
\end{example}



\begin{center}
\begin{foldable}
\unfoldable{Here is a video of Example 4}
\youtube{mbUekZP2IsY} %vid of example 4
\end{foldable}
\end{center}



\begin{problem}(problem 4)
  $\displaystyle{\frac{d}{dx} \left(\frac{3e^x}{x + 1}\right)=}$\\
  
  \begin{multipleChoice}
  \choice{$\displaystyle{\frac{3xe^x+6e^x}{(x+1)^2}}$}
  \choice{$\displaystyle{\frac{3xe^x}{(x+1)}}$}
  \choice[correct]{$\displaystyle{\frac{3xe^x}{(x+1)^2}}$}
  \end{multipleChoice}
\end{problem}



\begin{example}[example 5]
Find $h'(x)$ if
\[
h(x) = \frac{\ln(x)}{x+1}.
\]
We write $\displaystyle{h(x) = \frac{f(x)}{g(x)}}$ 
with:
\[f(x) =\ln(x) \  \text{and} \  g(x)= x+1.\]
To use the quotient rule we need the derivatives:
\[f'(x) =1/x  \ \text{and} \  g'(x) = 1.\]
We can now write: 
\begin{align*}
h'(x) &= \frac{g(x)f'(x) - f(x)g'(x)}{g^2(x)}\\
&= \frac{(x+1)(1/x) - \ln(x)}{(x+1)^2}\\
&= \frac{(x+1)(1/x) - \ln(x)}{(x+1)^2}\cdot \frac{x}{x}\\
&= \frac{(x+1) - x\ln(x)}{x(x+1)^2}.
\end{align*}
\end{example}



\begin{center}
\begin{foldable}
\unfoldable{Here is a video of Example 5}
\youtube{jg4hL-hPI48} %vid of example 5
\end{foldable}
\end{center}



\begin{problem}(problem 5)
  $\displaystyle{\frac{d}{dx} \left(\frac{2\ln(x)}{3x + 2}\right)=}$\\
  
  \begin{multipleChoice}
  \choice{$\displaystyle{\frac{2 - 6\ln(x)}{x(3x+2)}}$}
  \choice{$\displaystyle{\frac{6x+4 - 6\ln(x)}{x(3x+2)^2}}$}
  \choice[correct]{$\displaystyle{\frac{6x+4 - 6x\ln(x)}{x(3x+2)^2}}$}
  \end{multipleChoice}
\end{problem}




\begin{example}[example 6]
Find $h'(x)$ if
\[
h(x) = \tan(x).
\]
First rewrite $h(x)$ as 
$\displaystyle{\frac{\sin(x)}{\cos(x)}}.$ \\
Then $\displaystyle{h(x) = \frac{f(x)}{g(x)}}$ with:
\[f(x) = \sin(x) \ \mbox{and} \  g(x)= \cos(x).\] 
To use the quotient rule we need the derivatives:
\[f'(x) = \cos(x) \ \mbox{and} \  g'(x) = -\sin(x).\]
We can now write: 
\begin{align*}
h'(x) &= \frac{g(x)f'(x) - f(x)g'(x)}{g^2(x)}\\
&= \frac{\cos(x)\cos(x) - \sin(x)(-\sin(x))}{\cos^2(x)}\\
&= \frac{\cos^2(x)+ \sin^2(x)}{\cos^2(x)}\\
&= \frac{1}{\cos^2(x)} = \sec^2(x)
\end{align*}
This important formula is worth remembering: if $f(x) = \tan(x)$ then $f'(x) =\sec^2(x)$.
\end{example}



\begin{center}
\begin{foldable}
\unfoldable{Here is a video of Example 6}
\youtube{YW68eyqZJ8g} %vid of example 6
\end{foldable}
\end{center}

\begin{problem}(problem 6a)
  $\displaystyle{\frac{d}{dx} 5\tan(x)= \answer{5\sec^2(x)}}$\\
\end{problem}

\begin{problem}(problem 6b)
  $\displaystyle{\frac{d}{dx} x\tan(x)= \answer{\tan(x) + x\sec^2(x)}}$\\
\end{problem}

\begin{problem}(problem 6c)
  Find the equation of the tangent line to $y = \tan(x)$ at $x = 0$.\\
  The equation is $y = \answer{x}$
\end{problem}

%Make this next one a problem
\begin{example}[example 7]
Find $h'(x)$ if $h(x) = \cot(x)$.\\
First rewrite $h(x)$ as $\displaystyle{\frac{\cos(x)}{\sin(x)}}$ and then 
$\displaystyle{h(x) = \frac{f(x)}{g(x)}}$ with:
\[f(x) = \cos(x) \ \text{and} \  g(x)= \sin(x).\]
To use the quotient rule we need the derivatives:
\[f'(x) = -\sin(x) \ \text{and} \  g'(x) =\cos(x).\]
We can now write: 
\begin{align*}
h'(x) &= \frac{g(x)f'(x) - f(x)g'(x)}{g^2(x)}\\
&= \frac{\sin(x)(-\sin(x)) -\cos(x)\cos(x) }{\sin^2(x)}\\
&= -\frac{\sin^2(x)+ \cos^2(x)}{\sin^2(x)}\\
&= -\frac{1}{\sin^2(x)} = -\csc^2(x)
\end{align*}

This important formula is worth remembering: if $f(x) = \cot(x)$ then $f'(x) =-\csc^2(x)$.
\end{example}



\begin{center}
\begin{foldable}
\unfoldable{Here is a video of Example 7}
\youtube{nLwYWSf_XJA} %vid of example 7
\end{foldable}
\end{center}


\begin{problem}(problem 7a)
  $\displaystyle{\frac{d}{dx} 3\cot(x)= \answer{-3\csc^2(x)}}$\\
\end{problem}

\begin{problem}(problem 7b)
  $\displaystyle{\frac{d}{dx} x^3\cot(x)= \answer{3x^2\cot(x) - x^3\csc^2(x)}}$\\
\end{problem}



\begin{example}[example 8]
Find $h'(x)$ if $h(x) = \sec(x).$\\
First we rewrite $h(x)$ as $\displaystyle{\frac{1}{\cos(x)}}$ 
and then $\displaystyle{h(x) = \frac{f(x)}{g(x)}}$ with:
\[f(x) = 1 \  \mbox{and} \  g(x)= \cos(x).\]
To use the quotient rule we need the derivatives:
\[f'(x) = 0 \  \mbox{and} \   g'(x) = -\sin(x).\]
We can now write: 
\begin{align*}
h'(x) &= \frac{g(x)f'(x) - f(x)g'(x)}{g^2(x)}\\
&= \frac{\cos(x)\cdot 0 - 1\cdot (-\sin(x))}{\cos^2(x)}\\
&= \frac{ \sin(x)}{\cos^2(x)} \\
&= \frac{ 1}{\cos(x)} \cdot \frac{\sin(x)}{\cos(x)}\\
&=  \sec(x) \tan(x)
\end{align*}

This important formula is worth remembering: if $f(x) = \sec(x)$ then $f'(x) =\sec(x) \tan(x)$.
\end{example}



\begin{center}
\begin{foldable}
\unfoldable{Here is a video of Example 8}
\youtube{Qslesf590W4} %vid of example 8
\end{foldable}
\end{center}


\begin{problem}(problem 8a)
  $\displaystyle{\frac{d}{dx} 4\sec(x)= \answer{4\sec(x)\tan(x)}}$\\
\end{problem}

\begin{problem}(problem 8b)
  $\displaystyle{\frac{d}{dx} 2e^x \sec(x)= \answer{2e^x\sec(x)(1+\tan(x))}}$\\
\end{problem}

\begin{problem}(problem 8c)
  Find the equation of the tangent line to $y = \sec(x)$ at $x = 0$.\\
  The equation is $y = \answer{1}$
\end{problem}



\begin{example}[example 9]
Find $h'(x)$ if $h(x) = \csc(x).$\\
First we rewrite $h(x)$ as $\displaystyle{\frac{1}{\sin(x)}}$ 
and then $\displaystyle{h(x) = \frac{f(x)}{g(x)}}$ with:
\[f(x) = 1 \  \mbox{and} \  g(x)= \sin(x).\]
To use the quotient rule we need the derivatives:
\[f'(x) = 0 \  \mbox{and} \   g'(x) = \cos(x).\]
We can now write: 
\begin{align*}
h'(x) &= \frac{g(x)f'(x) - f(x)g'(x)}{g^2(x)}\\
&= \frac{\sin(x)\cdot 0 - 1\cdot (\cos(x))}{\sin^2(x)}\\
&= \frac{ -\cos(x)}{\sin^2(x)} \\
&= -\frac{ 1}{\sin(x)} \cdot \frac{\cos(x)}{\sin(x)}\\
&=  -\csc(x) \cot(x)
\end{align*}

This important formula is worth remembering: if $f(x) = \csc(x)$ then $f'(x) =-\csc(x) \cot(x)$.
\end{example}



\begin{problem}(problem 9a)
  $\displaystyle{\frac{d}{dx} \pi\csc(x)= \answer{-\pi\csc(x)\cot(x)}}$\\
\end{problem}

\begin{problem}(problem 9b)
  $\displaystyle{\frac{d}{dx} \ln(x) \csc(x)= \answer{\csc(x)/x - \ln(x)\csc(x)\cot(x)}}$\\
\end{problem}

\begin{problem}(problem 9c)
  Find the equation of the tangent line to $y = \csc(x)$ at $x = \pi/2$.\\
  The equation is $y = \answer{1}$
\end{problem}






\begin{example}[example 10]
Find $h'(x)$ if 
\[
h(x) = \frac{x^2 + 3x + 5}{x}.
\]
We could use the quotient rule, 
but it is easier to rewrite $h(x)$ as $h(x) = x + 3 + \frac{5}{x} = x + 3 + 5x^{-1}$ and 
find $h'(x)$ using the power and constant rules.  We have:
\[h'(x) = 1 + (-5)x^{-2} = 1 - \frac{5}{x^2}.\] 
\end{example}

\begin{problem}(problem 10a)
Find $h'(x)$ if 
\[
h(x) = \frac{x^3 - 4x + 3}{x^2}.
\]
Rewrite $h(x)$ as
\begin{multipleChoice}
\choice{$x - 4x+ 3$}
\choice[correct]{$x - 4x^{-1} + 3x^{-2}$}
\choice{$x^3 - 4x + \frac{3}{x^2}$}
\end{multipleChoice}

$h'(x) = \answer{1 + 4x^{-2} -6x^{-3}}$

\end{problem}



\begin{problem}(problem 10b)
Find $h'(x)$ if 
\[
h(x) = \frac{x - 2\sqrt{x} + 6}{2x}.
\]
Rewrite $h(x)$ as
\begin{multipleChoice}
\choice[correct]{$\frac12 - x^{-1/2} + 3 x^{-1}$}
\choice{$\frac12 - 2\sqrt{x} + 6$}
\choice{$\frac12 - x^{-3/2} + 3 x^{-1}$}
\end{multipleChoice}

$h'(x) = \answer{\frac12 x^{-3/2} -3x^{-2}}$

\end{problem}


%$\displaystyle{h(x) = \frac{f(x)}{g(x)}}$ with $f(x) =$ and $g(x)= $. To use the quotient rule 
%we need $f'(x) = $ and $g'(x) = $. We get 
%$$h'(x) = \frac{g(x)f'(x) - f(x)g'(x)}{g^2(x)}$$
%$$= $$


\begin{center}
\begin{foldable}
\unfoldable{Here is a detailed, lecture style video on the Quotient Rule:}
\youtube{0Y_Kap9ZuAo}
\end{foldable}
\end{center}



\end{document}
