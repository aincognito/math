\documentclass[handout]{ximera}
%\usepackage{tcolorbox}
%% You can put user macros here
%% However, you cannot make new environments



\newcommand{\ffrac}[2]{\frac{\text{\footnotesize $#1$}}{\text{\footnotesize $#2$}}}
\newcommand{\vasymptote}[2][]{
    \draw [densely dashed,#1] ({rel axis cs:0,0} -| {axis cs:#2,0}) -- ({rel axis cs:0,1} -| {axis cs:#2,0});
}


\graphicspath{{./}{firstExample/}}
\usepackage{forest}
\usepackage{amsmath}
\usepackage{amssymb}
\usepackage{array}
\usepackage[makeroom]{cancel} %% for strike outs
\usepackage{pgffor} %% required for integral for loops
\usepackage{tikz}
\usepackage{tikz-cd}
\usepackage{tkz-euclide}
\usetikzlibrary{shapes.multipart}


%\usetkzobj{all}
\tikzstyle geometryDiagrams=[ultra thick,color=blue!50!black]


\usetikzlibrary{arrows}
\tikzset{>=stealth,commutative diagrams/.cd,
  arrow style=tikz,diagrams={>=stealth}} %% cool arrow head
\tikzset{shorten <>/.style={ shorten >=#1, shorten <=#1 } } %% allows shorter vectors

\usetikzlibrary{backgrounds} %% for boxes around graphs
\usetikzlibrary{shapes,positioning}  %% Clouds and stars
\usetikzlibrary{matrix} %% for matrix
\usepgfplotslibrary{polar} %% for polar plots
\usepgfplotslibrary{fillbetween} %% to shade area between curves in TikZ



%\usepackage[width=4.375in, height=7.0in, top=1.0in, papersize={5.5in,8.5in}]{geometry}
%\usepackage[pdftex]{graphicx}
%\usepackage{tipa}
%\usepackage{txfonts}
%\usepackage{textcomp}
%\usepackage{amsthm}
%\usepackage{xy}
%\usepackage{fancyhdr}
%\usepackage{xcolor}
%\usepackage{mathtools} %% for pretty underbrace % Breaks Ximera
%\usepackage{multicol}



\newcommand{\RR}{\mathbb R}
\newcommand{\R}{\mathbb R}
\newcommand{\C}{\mathbb C}
\newcommand{\N}{\mathbb N}
\newcommand{\Z}{\mathbb Z}
\newcommand{\dis}{\displaystyle}
%\renewcommand{\d}{\,d\!}
\renewcommand{\d}{\mathop{}\!d}
\newcommand{\dd}[2][]{\frac{\d #1}{\d #2}}
\newcommand{\pp}[2][]{\frac{\partial #1}{\partial #2}}
\renewcommand{\l}{\ell}
\newcommand{\ddx}{\frac{d}{\d x}}

\newcommand{\zeroOverZero}{\ensuremath{\boldsymbol{\tfrac{0}{0}}}}
\newcommand{\inftyOverInfty}{\ensuremath{\boldsymbol{\tfrac{\infty}{\infty}}}}
\newcommand{\zeroOverInfty}{\ensuremath{\boldsymbol{\tfrac{0}{\infty}}}}
\newcommand{\zeroTimesInfty}{\ensuremath{\small\boldsymbol{0\cdot \infty}}}
\newcommand{\inftyMinusInfty}{\ensuremath{\small\boldsymbol{\infty - \infty}}}
\newcommand{\oneToInfty}{\ensuremath{\boldsymbol{1^\infty}}}
\newcommand{\zeroToZero}{\ensuremath{\boldsymbol{0^0}}}
\newcommand{\inftyToZero}{\ensuremath{\boldsymbol{\infty^0}}}


\newcommand{\numOverZero}{\ensuremath{\boldsymbol{\tfrac{\#}{0}}}}
\newcommand{\dfn}{\textbf}
%\newcommand{\unit}{\,\mathrm}
\newcommand{\unit}{\mathop{}\!\mathrm}
%\newcommand{\eval}[1]{\bigg[ #1 \bigg]}
\newcommand{\eval}[1]{ #1 \bigg|}
\newcommand{\seq}[1]{\left( #1 \right)}
\renewcommand{\epsilon}{\varepsilon}
\renewcommand{\iff}{\Leftrightarrow}

\DeclareMathOperator{\arccot}{arccot}
\DeclareMathOperator{\arcsec}{arcsec}
\DeclareMathOperator{\arccsc}{arccsc}
\DeclareMathOperator{\si}{Si}
\DeclareMathOperator{\proj}{proj}
\DeclareMathOperator{\scal}{scal}
\DeclareMathOperator{\cis}{cis}
\DeclareMathOperator{\Arg}{Arg}
%\DeclareMathOperator{\arg}{arg}
\DeclareMathOperator{\Rep}{Re}
\DeclareMathOperator{\Imp}{Im}
\DeclareMathOperator{\sech}{sech}
\DeclareMathOperator{\csch}{csch}
\DeclareMathOperator{\Log}{Log}

\newcommand{\tightoverset}[2]{% for arrow vec
  \mathop{#2}\limits^{\vbox to -.5ex{\kern-0.75ex\hbox{$#1$}\vss}}}
\newcommand{\arrowvec}{\overrightarrow}
\renewcommand{\vec}{\mathbf}
\newcommand{\veci}{{\boldsymbol{\hat{\imath}}}}
\newcommand{\vecj}{{\boldsymbol{\hat{\jmath}}}}
\newcommand{\veck}{{\boldsymbol{\hat{k}}}}
\newcommand{\vecl}{\boldsymbol{\l}}
\newcommand{\utan}{\vec{\hat{t}}}
\newcommand{\unormal}{\vec{\hat{n}}}
\newcommand{\ubinormal}{\vec{\hat{b}}}

\newcommand{\dotp}{\bullet}
\newcommand{\cross}{\boldsymbol\times}
\newcommand{\grad}{\boldsymbol\nabla}
\newcommand{\divergence}{\grad\dotp}
\newcommand{\curl}{\grad\cross}
%% Simple horiz vectors
\renewcommand{\vector}[1]{\left\langle #1\right\rangle}


\outcome{Discuss the continuity of a function.}

\title{1.8 Continuity}

%\newcommand{\ffrac}[2]{\frac{\mbox{\footnotesize $#1$}}{\mbox{\footnotesize $#2$}}}
%\newcommand{\vasymptote}[2][]{
 %   \draw [densely dashed,#1] ({rel axis cs:0,0} -| {axis cs:#2,0}) -- ({rel axis cs:0,1} -| {axis cs:#2,0});
%}


\begin{document}

\begin{abstract}
In this section we learn the definition of continuity and we study the types of discontinuities.
\end{abstract}

\maketitle






\section{Continuity}

Limits can be used to give precise meaning to the concept of continuity. Intuitively, a continuous function is one whose graph can be drawn
without lifting the pencil off of the paper. In the figure below, the function can be drawn without lifting the pencil 
off of the paper at $x = 2$, so we say that the function is continuous at $x = 2$.

\begin{image}
\begin{tikzpicture}
\begin{axis}[axis x line=  none, axis y line = none,
title={A Continuous Function}] 


\addplot[domain=-1:5, samples = 100, color=blue, thick]{1.5 + cos(deg(x))};
\addplot[<->] coordinates {(-0.8,-0.5) (-0.8, 2.8)} node[above] {$y$}; %y-axis
\addplot[<->] coordinates {(-1.3,0) (4.3,0)} node[right] {$x$}; %x-axis
\addplot[thin] coordinates {(1.57,0.1) (1.57, -0.1)} node[below] {$2$}; 
\node at (axis cs: 2,2.1){$y = f(x)$};
\node at (axis cs: 2,-.6){$f(x)$ is continuous at $x = 2$};
\addplot[smooth,mark=*,blue] plot coordinates {(1.57,1.5)};

\end{axis}
\end{tikzpicture}
\end{image}
Mathematically, what is happening in the graph above that allows us to claim continuity of $f(x)$ at $x = 2$ 
is the left hand limit, right hand limit and function value are all equal at $x = 2$.
In symbols,
\[
\lim_{x\to 2^-} f(x) = \lim_{x\to 2^+} f(x) = f(2).
\]
Since the one-sided limits are equal, we can condense this into a single equation using a two sided limit:
\[
\lim_{x\to 2} f(x) = f(2).
\]
This leads us to the mathematical definition of continuity at a point:

\begin{definition}[Continuity]
The function $f(x)$ is \textbf{continuous} at $x = a$ if:
\[\lim_{x \to a} f(x) = f(a).\]
\end{definition}

\begin{example}[example 1]
Since
\[\lim_{x \to 5} x^2 = 5^2 = 25,\]
we can say that the function $f(x) = x^2$ is continuous at $x = 5$.
\end{example}

\begin{problem}(problem 1)
Since
\[\lim_{x \to 4} \sqrt{x} = \sqrt{4} = 2,\]
we can say that the function
$f(x) = \answer{\sqrt x}$
is continuous at
$x = \answer{4}.$
\end{problem}


\begin{example}[example 2]
Since
\[\lim_{x \to \pi} \cos(x) = \cos(\pi) = -1,\]
we can say that the function $f(x) = \cos(x)$ is continuous at $x = \pi$.
\end{example}

\begin{problem}(problem 2)
Since
\[\lim_{x \to 3} 2^x = 2^3 = 8,\]
we can say that the function
$f(x) = \answer{2^x}$
is continuous at
$x = \answer{3}.$
\end{problem}

\begin{example}[example 3]
Since
\[\lim_{x \to 1} \ln(x) = \ln(1) = 0,\]
we can say that the function $f(x) = \ln(x)$ is continuous at $x = 1$.
\end{example}

\begin{problem}(problem 3)
Since
\[\lim_{x \to \sqrt 3} \tan^{-1}(x) = \tan^{-1}(\sqrt 3) = \pi/3,\]
we can say that the function
$f(x) = \answer{\tan^{-1}(x)}$
is continuous at
$x = \answer{\sqrt 3}.$
\end{problem}

If a function is continuous at each point in an interval, $I$, then we say that $f(x)$ is \textbf{continuous on $I$}.

A function $f(x)$ is called \textbf{continuous from the left at $x=a$} if 
\[\lim_{x \to a^-} f(x) = f(a),\]
and it is called \textbf{continuous from the right at $x = a$} if
\[\lim_{x \to a^+} f(x) = f(a).\]
Note that if $f(x)$ is continuous at $x=a$ then it is both continuous from the left and 
continuous from the right at $x = a$.

If for some reason, a limit cannot be computed by plugging in, then we say that the function is discontinuous.
In other words, if
\[\lim_{x \to a} f(x) \neq f(a),\]
then we say that $f(x)$ has a \textbf{discontinuity} at $x = a$.
Next, we explore the types of discontinuities.

%As an example the function $f(x) = \tan(x)$ is not continuous at $x = \frac{\pi}{2}$ because $\tan(\frac{\pi}{2})$
%is undefined.

% and hence
%\[\lim_{x \to \frac{\pi}{2}} \tan(x) \neq \tan(\frac{\pi}{2}).\]



%We will study examples of continuity from the left and right in the section on piecewise functions below.


\section{Types of Discontinuities}


A function $f(x)$ has a discontinuity at $x = a$ if
\[\lim_{x \to a} f(x) \neq f(a).\]
There are four main types of discontinuities: removable, jump, infinite and essential. 
%They are called removable, jump, infinite and essential. 
%Suppose $f(x)$ has a  discontinuity at $x=a$, then

%Depending on the reason for these two quantities not being equal, we will get a different type of discontinuity.
First, a discontinuity is called a \textbf{removable discontinuity} if 
\[\lim_{x \to a} f(x)  \  \  \text{exists and is finite }.\]
%In other words, the limit is equal to a number but that number is not $f(a)$.  
%This might be because $f(a)$ is undefined
%or it could be that even though $f(a)$ is defined, it just doesn't equal the limit. Such a discontinuity is 
%called removable because if we were to define the function appropriately at $x=a$ it would become 
%continuous, i.e., the discontinuity would have been removed.

Here are two examples of graphs of functions that have removable discontinuities:
\begin{center}
\begin{tikzpicture}
\begin{axis}[axis lines = center, title={A Removable Discontinuity at $x=1$}]
\addplot[domain=0:0.98, color=blue]{x+1};
\addplot[smooth,mark=*,blue] plot coordinates {(1,1.5)};
\addplot[domain=1.02:2, 
    samples=100, color=blue]{x+1};
\addplot[smooth,mark=o,blue] plot coordinates {(1,2)};
\end{axis}
\end{tikzpicture}
\hspace{1.5 in}
\begin{tikzpicture}
\begin{axis}[axis lines = center, title={A Removable Discontinuity at $x=2$}]
\addplot[domain=1:1.985, color=blue]{x^2 - 3};
\addplot[smooth,mark=o,blue] plot coordinates {(2,1)};
\addplot[domain=2.015:3, 
    samples=100, color=blue]{x^2 - 3};
\end{axis}
\end{tikzpicture}
\end{center}


The second type of discontinuity is called a \textbf{jump discontinuity}. 
This happens when the one-sided
limits are different numbers, so that
%\[\lim_{x \to a} f(x) \ \ \text{does not exist}\]
%for the following reason:
%\[\lim_{x \to a^-} f(x) \ \ \text{is finite, and}\]
%\[\lim_{x \to a^+} f(x) 

%\ \ \text{is finite, but}\]
\[\lim_{x \to a^-} f(x) \neq \lim_{x \to a^+} f(x).\]
and hence
\[
\lim_{x \to a} f(x) \ \text{DNE}.
\]

It is important in this case that the one-sided limits are both finite.

% finite numbers but they are not equal to each other, 
%and therefore the two sided limit does not exist.



\begin{example}[example 4]
The function 
\[f(x) = \frac{|x|}{x}\]
 has a jump discontinuity at $x = 0$ because
\[\lim_{x \to 0^-} \frac{|x|}{x} = -1 \text{  but} \]
\[\lim_{x \to 0^+} \frac{|x|}{x} = 1. \]

A graph of this function is shown below.

\includeinteractive{cf01.js}
\end{example}

Here are two examples of graphs of functions that have jump discontinuities:

\begin{center}
\begin{tikzpicture}
\begin{axis}[axis lines = center, title={A Jump Discontinuity at $x=1$}]
\addplot[domain=-1:1, color=blue]{x^2};
\addplot[smooth,mark=*,blue] plot coordinates {(1,1)};
\addplot[domain=1.02:2, 
    samples=100, color=blue]{x+1};
\addplot[smooth,mark=o,blue] plot coordinates {(1,2)};
\end{axis}
\end{tikzpicture}
\hspace{1.5 in}
\begin{tikzpicture}
\begin{axis}[axis lines = center, title={A Jump Discontinuity at $x=2$}]
\addplot[domain=0:1.99, color=blue]{x^2 - 1};
\addplot[smooth,mark=o,blue] plot coordinates {(2,3)};
\addplot[domain=2.02:4, 
    color=blue]{3-x};
\addplot[smooth,mark=o,blue] plot coordinates {(2,1)};
\end{axis}
\end{tikzpicture}
\end{center}


The third type of discontinuity is called an \textbf{infinite discontinuity}. 
This occurs when either of the one-sided limits is either $\pm \infty$, i.e., 
\[\lim_{x \to a^-} f(x) = \pm\infty \ \text{or}  \]
\[\lim_{x \to a^+} f(x) = \pm\infty. \]
The graph of the function $f(x)$ has a vertical asymptote at an infinite discontinuity.


\begin{example}[example 5]
The function 
\[
f(x) = \frac{1}{x}
\]
  has an infinite discontinuity at $x=0$ since
\[\lim_{x \to 0^-} \frac{1}{x} = -\infty \ \text {  and}\]
\[\lim_{x \to 0^+} \frac{1}{x} = \infty.\]
The graph of $f(x) = 1/x$ is shown below.
\[
\graph{1/x}
\]
\end{example}

%\includeinteractive{cf02.js}

\begin{example}[example 6]
The function 
\[
f(x) = \tan(x)
\]
 has an infinite discontinuity at $x = \frac{\pi}{2}$
since
\[\lim_{x \to \frac{\pi}{2}^-} \tan(x) = \infty \ \text {  and}\]
\[\lim_{x \to \frac{\pi}{2}^+} \tan(x) = -\infty.\]
In fact, $f(x) = \tan(x)$ has infinite discontinuities at all odd multiples of $\pi/2$. 
The graph of $f(x) = \tan(x)$ is below.

\[
\graph{tan(x)}
\]
%\includeinteractive{cf03.js}

\end{example}


Finally, if a discontinuity is not one of the first three types, it is called an 
\textbf{essential discontinuity}.


\begin{example}[example 7] The function 
\[
f(x) = \sin \left(\frac{1}{x}\right),
\]
 shown below, has an essential discontinuity at $x = 0$.
Neither of one-sided limits at $x=0$ exist due to oscillation of the function, 
and the function does not have a vertical asymptote since 
\[-1 \leq \sin\left(\frac{1}{x}\right) \leq 1 \]
for all values of $x$ except $x = 0$, where the function is undefined.

\[
\graph{sin(1/x)}
\]
\end{example}
%Here is an example of the graph of a function that has an essential discontinuity:


%\begin{center}
%\begin{tikzpicture}
%\begin{axis}[axis lines = center, title={An Essential Discontinuity at $x=0$}]
%\addplot[domain=-1:-0.01, color=blue]{sin(deg(1/x))};
%\addplot[domain=0.01:1, color=blue]{sin(deg(1/x))};

%\end{axis}
%\end{tikzpicture}
%\end{center}

\begin{problem}(problem 7a)
Which type of discontinuity does $f(x)$ have at $x=2$ if
\[
\lim_{x \to 2^+} f(x) = -\infty?
\]

\begin{multipleChoice}
  \choice{removable}
  \choice{jump}
  \choice[correct]{infinite}
  \choice{essential}
\end{multipleChoice}
\end{problem}

\begin{problem}(problem 7b)
Which type of discontinuity does $f(x)$ have at $x=2$ if
\[
\lim_{x \to 2^-} f(x) = \lim_{x \to 2^+} f(x) = 5,
\]
but $f(2)$ is undefined?
\begin{multipleChoice}
  \choice[correct]{removable}
  \choice{jump}
  \choice{infinite}
  \choice{essential}
\end{multipleChoice}
\end{problem}

\begin{problem}(problem 7c)
Which type of discontinuity does $f(x)$ have at $x=2$ if
\[
\lim_{x \to 2^+} f(x) \ \text{DNE}?
\]

\begin{multipleChoice}
  \choice{removable}
  \choice{jump}
  \choice{infinite}
  \choice[correct]{essential}
\end{multipleChoice}
\end{problem}

\begin{problem}(problem 7d)
Which type of discontinuity does $f(x)$ have at $x=2$ if
\[
\lim_{x \to 2^-} f(x) = 3
\]
and
\[
\lim_{x \to 2^+} f(x) = -1
\]
\begin{multipleChoice}
  \choice{removable}
  \choice[correct]{jump}
  \choice{infinite}
  \choice{essential}
\end{multipleChoice}
\end{problem}



\section{Piecewise Functions}


In this section we will examine the continuity of piecewise defined functions.

\begin{example}[example 8]
Is the function defined below continuous at $x = 2$?
\[
f(x) = \left\{
     \begin{array}{lr}
       x-1 & \ \text{for} \  x \leq 2 \\
       3 & \ \text{for} \ x > 2
     \end{array}
   \right.
\]


We need to  compute $f(2)$ and the one-sided limits as $x$ approaches 2.
If all three of these are equal, then $f(x)$ is continuous at $x=2$.
Otherwise, $f(x)$ has a discontinuity at $x=2$.\\
First, 
\[
f(2) = 2-1 = 1.
\]
Next,  
\[\lim_{x \to 2^-} f(x) = \lim_{x \to 2^-} x-1 = 2-1 = 1.\]
Finally,
\[\lim_{x \to 2^+} f(x) = \lim_{x \to 2^+}3 = 3.\]
Since the one-sided limits are different, the two-sided limit
\[
\lim_{x \to 2} f(x) \ \text{DNE}
\]
and hence $f(x)$ is not continuous at $x = 2$.
In this case, $f(x)$ has a jump discontinuity at $x=2$ since the one-sided limits are finite but different.
\end{example}

\begin{problem}(problem 8)
Determine whether the function $f(x)$ given below is continuous at $x = 1$.
\[
f(x) = \left\{
     \begin{array}{lr}
       2x & \ \text{for} \  x < 1 \\
       x^2 + 2 & \ \text{for} \ x \geq 1
     \end{array}
   \right.
\]
\[
f(1) = \answer{3}
\]
\[
\lim_{x \to 1^-} f(x) = \answer{2}
\]
\[
\lim_{x \to 1^+} f(x) = \answer{3}
\]
Is $f(x)$ continuous at $x = 1$?
\begin{multipleChoice}
  \choice{Yes}
  \choice[correct]{No}
\end{multipleChoice}
\end{problem}





\begin{example}[example 9]
Is the function define below continuous at $x = 2$?
\[f(x) = \left\{
     \begin{array}{lr}
       x-1 & \ \text{for} \  x \leq 2 \\
       x^2 - 3 & \ \text{for} \ x > 2
     \end{array}
   \right.
\]

We need to  compute $f(2)$ and the one-sided limits as $x$ approaches 2.
If all three of these are equal, then $f(x)$ is continuous at $x=2$.
Otherwise, $f(x)$ has a discontinuity at $x=2$.\\
First, 
\[
f(2) = 2-1 = 1.
\]
Next, 
\[
\lim_{x \to 2^-} f(x) = \lim_{x \to 2^-} x-1 = 2-1 = 1,
\]
and finally,
\[
\lim_{x \to 2^+} f(x) = \lim_{x \to 2^+} x^2 - 3 = 2^2 - 3 = 1.
\]
Since all three of these values are equal, we can write
\[
\lim_{x \to 2} f(x) = f(2)
\]
and conclude that $f(x)$ is continuous at $x = 2$.
\end{example}


\begin{problem}(problem 9)
Determine whether the function $f(x)$ given below is continuous at $x = 1$.
\[
f(x) = \left\{
     \begin{array}{lr}
       2x & \ \text{for} \  x < 1 \\
       x^2 + 1 & \ \text{for} \ x \geq 1
     \end{array}
   \right.
\]
\[
f(1) = \answer{2}
\]
\[
\lim_{x \to 1^-} f(x) = \answer{2}
\]
\[
\lim_{x \to 1^+} f(x) = \answer{2}
\]
Is $f(x)$ continuous at $x = 1$?
\begin{multipleChoice}
  \choice[correct]{Yes}
  \choice{No}
\end{multipleChoice}
\end{problem}




\begin{example}[example 10]
Is the function $f(x)$ defined below continuous at $x = 2$?
\[
f(x) = \left\{
     \begin{array}{lr}
       x-1 & \ \text{for} \  x < 2 \\
			 3 & \ \text{for} \  x = 2 \\
       x^2 - 3 & \ \text{for} \ x > 2
     \end{array}
   \right.
\]


We need to  compute $f(2)$ and the one-sided limits as $x$ approaches 2.
If all three of these are equal, then $f(x)$ is continuous at $x=2$.
Otherwise, $f(x)$ has a discontinuity at $x=2$.\\
First, 
\[
f(2) = 3 \ \text{(given)}.
\]
Next, 
\[
\lim_{x \to 2^-} f(x) = \lim_{x \to 2^-} x-1 = 2-1 = 1
\]
and
\[
\lim_{x \to 2^+} f(x) = \lim_{x \to 2^+} x^2 - 3 = 2^2 - 3 = 1.
\]
Since the one-sided limits are equal, the two-sided limit exists:
\[
\lim_{x \to 2} f(x) = 1.
\]
However,
\[
\lim_{x \to 2} f(x) \neq f(2) = 3.
\]
Hence $f(x)$ is not continuous at $x = 2$.
This is an example of a removable discontinuity, since 
\[\lim_{x \to 2} f(x) \]
is a finite number.
\end{example}


\begin{problem}(problem 10)
Determine whether the function $f(x)$ given below is continuous at $x = 1$.
\[
f(x) = \left\{
     \begin{array}{lr}
       2x & \ \text{for} \  x < 1 \\
			 5 & \ \text{for} \  x = 1 \\
       x^2 + 1 & \ \text{for} \ x > 1
     \end{array}
   \right.
\]
\[
f(1) = \answer{5}
\]
\[
\lim_{x \to 1^-} f(x) = \answer{2}
\]
\[
\lim_{x \to 1^+} f(x) = \answer{2}
\]
Is $f(x)$ continuous at $x = 1$?
\begin{multipleChoice}
  \choice{Yes}
  \choice[correct]{No}
\end{multipleChoice}
\end{problem}







\begin{example}[example 11]
Determine the value of $k$ that will make the function $f(x)$ given below 
continuous at $x = 2$.
 
\[f(x) = \left\{
     \begin{array}{lr}
       kx-5 & \ \text{for} \  x \leq 2 \\
			 x^2 - k & \ \text{for} \ x > 2
     \end{array}
   \right.
\]

We need to  compute $f(2)$ and the one-sided limits as $x$ approaches 2.
We then set these expressions equal to each other and solve for $k$.\\
First,
\[
f(2) = 2k-5.
\]
Next,
\[
\lim_{x \to 2^-} f(x) = \lim_{x \to 2^-} kx-5 = 2k-5,
\]
and
\[
\lim_{x \to 2^+} f(x) = \lim_{x \to 2^+} x^2 - k = 2^2 - k = 4 - k.
\]
Setting these expressions equal to one another gives the equation:
\[2k - 5 = 4 - k.\]
Solving for $k$ gives
\[3k = 9\]
and hence,
\[k = 3.\]
This value of $k$ will make the function $f(x)$ continuous at $x=2$.
\end{example}

\begin{problem}(problem 11)
Determine the value of $k$ that will make the function $f(x)$ given below 
continuous at $x = 1$.
 
\[f(x) = \left\{
     \begin{array}{lr}
        x^2 - k & \ \text{for} \  x < 1 \\
			 kx-3 & \ \text{for} \ x \geq 1
     \end{array}
   \right.
\]

\[
f(1) = \answer{k-3}
\]
\[
\lim_{x \to 1^-} f(x) = \answer{1-k}
\]
\[
\lim_{x \to 1^+} f(x) = \answer{k-3}
\]
The value of $k$ that makes $f(x)$ continuous at $x = 1$ is
$k = \answer{2}$
\end{problem}


\section{Continuity of Familiar Functions}


Consider the function $f(x) = x^2$ and any number $a$. We can compute the limit 
\[\lim_{x \to a} x^2 = a^2\]
by plugging in. Therefore, we can say that $f(x) = x^2$ is continuous for all real numbers.
We can also say that $f(x) = x^2$ is continuous on the interval $(-\infty, \infty)$.
Actually, this is true for all polynomials, including constant functions.
If $p(x)$ is a polynomial, then $p(x)$ is continuous on the interval $(-\infty, \infty)$.\\
There are some other familiar functions which are also continuous on the interval $(-\infty, \infty)$.
These are: 
\[f(x) = \sin(x), \cos(x), e^x, \tan^{-1}(x), \sqrt[3] x \ \text{and} \ |x|.\]
The function $f(x) = \ln(x)$ is only defined for $x$ in the interval $(0, \infty)$ and it is 
continuous on this interval.\\
The function $f(x) = \tan(x)$ has vertical asymptotes at odd multiples of $\frac{\pi}{2}$. 
It is continuous between these vertical asymptotes, so, for example, $f(x) = \tan(x)$ is continuous on the 
interval $(-\frac{\pi}{2},\frac{\pi}{2})$.
The function $f(x) = \sec(x)$ is similar to $\tan(x)$ in that it has vertical asymptotes at odd 
multiples of $\frac{\pi}{2}$, and it is continuous on the intervals between them.
The functions $\cot(x)$ and $\csc(x)$ have vertical asymptotes at multiples of $\pi$ and like 
$\tan(x)$ and $\sec(x)$, they are continuous between their asymptotes. For example, the function
$f(x) = \cot(x)$ is continuous on the interval $(0, \pi)$.\\



The function $f(x) = \sqrt x$ is only defined for $x \geq 0$ and it is continuous for all of these values of $x$.
In other words, $f(x) = \sqrt x$ is continuous on the interval $[0, \infty)$. 
To say that the function is continuous at the left endpoint of this interval ($x = 0$) it is sufficient 
that the function is right continuous at this point. And it is indeed true that
\[\lim_{x \to 0^+} \sqrt x = \sqrt 0 = 0.\]
In general, a root function, $f(x) = \sqrt[n] x$ is continuous on the interval $[0, \infty)$ if $n$ is even and 
it is continuous on the interval $(-\infty, \infty)$ if $n$ is odd.\\

The last type of familiar function that we will discuss here is the rational function.  
A rational function is a ratio of polynomials, 
\[f(x) = \frac{p_1(x)}{p_2(x)}, \]
where $p_1(x)$ and $p_2(x)$ are both polynomials and the degree of $p_2(x)$ is at least 1.
Such a function is continuous for all values of $x$ such that $p_2(x) \neq 0$.
For example, the function
\[f(x) = \frac{x}{x^2 + 1}\]
is continuous on the interval $(-\infty, \infty)$ since $x^2 + 1 \neq 0$ for any $x$.
On the other hand, the function
\[f(x) = \frac{x}{x^2 - 1}\]
is continuous on the intervals $(-\infty, -1), (-1, 1)$ and $(1, \infty)$ since $ x^2 - 1 = 0$
when $x = \pm 1$.



\section{Properties of Continuity}



Continuous functions combine nicely with respect to the operations addition, subtraction, multiplication, 
division and composition. Specifically, if $f(x)$ and $g(x)$ are both continuous at $x = a$, then so are
\[f(x) + g(x),\]
\[f(x) - g(x) \ \text{and}\]
\[f(x) \cdot g(x).\]
Furthermore, if $g(a) \neq 0$ then 
\[\frac{f(x)}{g(x)}\]
is also continuous at $x = a$. 
In words, we say that the sum, difference, product and quotient of continuous functions is continuous 
(with the understanding that $g(a) \neq 0$ in the case of the quotient.)
Things are slightly more complicated for the composition. If $g(x)$ is continuous at $x = a$ and $f(x)$ is 
continuous at $x = g(a)$ then then composition $f(g(x))$ is continuous at $x = a$. This situation is 
different from the four basic operations because in composition, when plugging in $x = a$, we plug $a$ 
into $g(x)$ and then plug $g(a)$ into $f(x)$,
whereas for the first four basic operations we plug $x = a$ into both $f(x)$ and $g(x)$.



 
\begin{center}
\begin{foldable}
\unfoldable{Here is a detailed, lecture style video on discontinuities:}
\youtube{3T5jQUiW5FY}
\end{foldable}
\end{center}






\end{document}








