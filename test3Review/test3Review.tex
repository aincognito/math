\documentclass[handout]{ximera}

%% You can put user macros here
%% However, you cannot make new environments



\newcommand{\ffrac}[2]{\frac{\text{\footnotesize $#1$}}{\text{\footnotesize $#2$}}}
\newcommand{\vasymptote}[2][]{
    \draw [densely dashed,#1] ({rel axis cs:0,0} -| {axis cs:#2,0}) -- ({rel axis cs:0,1} -| {axis cs:#2,0});
}


%\usepackage{tcolorbox} %%Needed for Derivative Definition supposedly and product rule, natural exp log, quotient rule, inverse trig, rates of change


% \graphicspath{{./}{firstExample/}}
% \usepackage{forest}
\usepackage{amsmath}
\usepackage{amssymb}
\usepackage{array}
\usepackage[makeroom]{cancel} %% for strike outs
\usepackage{pgffor} %% required for integral for loops
\usepackage{tikz}
\usepackage{tikz-cd}
\usepackage{tkz-euclide}
\usetikzlibrary{shapes.multipart}


% \usetkzobj{all}
\tikzstyle geometryDiagrams=[ultra thick,color=blue!50!black]


\usetikzlibrary{arrows}
\tikzset{>=stealth,commutative diagrams/.cd,
  arrow style=tikz,diagrams={>=stealth}} %% cool arrow head
\tikzset{shorten <>/.style={ shorten >=#1, shorten <=#1 } } %% allows shorter vectors

\usetikzlibrary{backgrounds} %% for boxes around graphs
\usetikzlibrary{shapes,positioning}  %% Clouds and stars
\usetikzlibrary{matrix} %% for matrix
\usepgfplotslibrary{polar} %% for polar plots
\usepgfplotslibrary{fillbetween} %% to shade area between curves in TikZ



%\usepackage[width=4.375in, height=7.0in, top=1.0in, papersize={5.5in,8.5in}]{geometry}
%\usepackage[pdftex]{graphicx}
%\usepackage{tipa}
%\usepackage{txfonts}
%\usepackage{textcomp}
%\usepackage{amsthm}
%\usepackage{xy}
%\usepackage{fancyhdr}
%\usepackage{xcolor}
%\usepackage{mathtools} %% for pretty underbrace % Breaks Ximera
%\usepackage{multicol}



\newcommand{\RR}{\mathbb R}
\newcommand{\R}{\mathbb R}
\newcommand{\C}{\mathbb C}
\newcommand{\N}{\mathbb N}
\newcommand{\Z}{\mathbb Z}
\newcommand{\dis}{\displaystyle}
%\renewcommand{\d}{\,d\!}
\renewcommand{\d}{\mathop{}\!d}
\newcommand{\dd}[2][]{\frac{\d #1}{\d #2}}
\newcommand{\pp}[2][]{\frac{\partial #1}{\partial #2}}
\renewcommand{\l}{\ell}
\newcommand{\ddx}{\frac{d}{\d x}}
\newcommand{\ppx}{\frac{\partial}{\partial x}}
\newcommand{\ppy}{\frac{\partial}{\partial y}}

\newcommand{\zeroOverZero}{\ensuremath{\boldsymbol{\tfrac{0}{0}}}}
\newcommand{\inftyOverInfty}{\ensuremath{\boldsymbol{\tfrac{\infty}{\infty}}}}
\newcommand{\zeroOverInfty}{\ensuremath{\boldsymbol{\tfrac{0}{\infty}}}}
\newcommand{\zeroTimesInfty}{\ensuremath{\small\boldsymbol{0\cdot \infty}}}
\newcommand{\inftyMinusInfty}{\ensuremath{\small\boldsymbol{\infty - \infty}}}
\newcommand{\oneToInfty}{\ensuremath{\boldsymbol{1^\infty}}}
\newcommand{\zeroToZero}{\ensuremath{\boldsymbol{0^0}}}
\newcommand{\inftyToZero}{\ensuremath{\boldsymbol{\infty^0}}}


\newcommand{\numOverZero}{\ensuremath{\boldsymbol{\tfrac{\#}{0}}}}
\newcommand{\dfn}{\textbf}
%\newcommand{\unit}{\,\mathrm}
\newcommand{\unit}{\mathop{}\!\mathrm}
%\newcommand{\eval}[1]{\bigg[ #1 \bigg]}
\newcommand{\eval}[1]{ #1 \bigg|}
\newcommand{\seq}[1]{\left( #1 \right)}
\renewcommand{\epsilon}{\varepsilon}
\renewcommand{\iff}{\Leftrightarrow}

\DeclareMathOperator{\arccot}{arccot}
\DeclareMathOperator{\arcsec}{arcsec}
\DeclareMathOperator{\arccsc}{arccsc}
\DeclareMathOperator{\si}{Si}
\DeclareMathOperator{\proj}{proj}
\DeclareMathOperator{\scal}{scal}
\DeclareMathOperator{\cis}{cis}
\DeclareMathOperator{\Arg}{Arg}
%\DeclareMathOperator{\arg}{arg}
\DeclareMathOperator{\Rep}{Re}
\DeclareMathOperator{\Imp}{Im}
\DeclareMathOperator{\sech}{sech}
\DeclareMathOperator{\csch}{csch}
\DeclareMathOperator{\Log}{Log}

\newcommand{\tightoverset}[2]{% for arrow vec
  \mathop{#2}\limits^{\vbox to -.5ex{\kern-0.75ex\hbox{$#1$}\vss}}}
\newcommand{\arrowvec}{\overrightarrow}
\renewcommand{\vec}{\mathbf}
\newcommand{\veci}{{\boldsymbol{\hat{\imath}}}}
\newcommand{\vecj}{{\boldsymbol{\hat{\jmath}}}}
\newcommand{\veck}{{\boldsymbol{\hat{k}}}}
\newcommand{\vecl}{\boldsymbol{\l}}
\newcommand{\utan}{\vec{\hat{t}}}
\newcommand{\unormal}{\vec{\hat{n}}}
\newcommand{\ubinormal}{\vec{\hat{b}}}

\newcommand{\dotp}{\bullet}
\newcommand{\cross}{\boldsymbol\times}
\newcommand{\grad}{\boldsymbol\nabla}
\newcommand{\divergence}{\grad\dotp}
\newcommand{\curl}{\grad\cross}
%% Simple horiz vectors
\renewcommand{\vector}[1]{\left\langle #1\right\rangle}


\outcome{Review problems}

\title{Review}

\begin{document}

\begin{abstract}
Test 3 Review
\end{abstract}

\maketitle

\section{Sample Test 3A}





%\begin{problem}(problem 1)
%Find the sum of the infinite series, if it converges: $\displaystyle\sum_{n=1}^\infty \frac{1}{n^2 +n}$

%\begin{hint}
%Make a partial fraction decomposition
%\end{hint}

%\begin{hint}
%Compute the $N^{th}$ partial sum, $S_N$
%\end{hint}

%\begin{hint}
%Take the limit as $N \to \infty$ to find the sum of the series
%\end{hint}

%\end{problem}


%\begin{problem}(problem 2)
%Find the sum of the infinite series, if it converges: $\displaystyle \sum_{n=0}^\infty \frac{2^{2n}}{3^n}$
%\begin{hint}
%This is a geometric series with $r = 4/3$
%\end{hint}

%\begin{hint}
%The series diverges since $|r| = 4/3 \geq 1$
%\end{hint}

%\end{problem}



\begin{problem}(problem 1)
Determine whether the infinite series converges or diverges: $\displaystyle \sum_{n=1}^\infty \frac{n+1}{2n+3}$

\begin{hint}
Use the Test for Divergence
\end{hint}

\begin{hint}
Ratio of lead coeff: $\displaystyle \lim_{n \to \infty} a_n = \frac12 \neq 0$
\end{hint}

\end{problem}

\begin{problem}(problem 2)
Use the Integral Test to determine whether the infinite series converges or diverges (if it applies):
$\displaystyle \sum_{n=2}^\infty \frac{1}{n\ln(n)}$.

\begin{hint}
Show the derivative is negative for $x$ sufficiently large
\end{hint}
\begin{hint}
Use u-sub to get the anti-derivative
\end{hint}
\begin{hint}
Write the improper integral using a limit
\end{hint}\begin{hint}
The integral diverges
\end{hint}
\begin{hint}
See problem 2a in section 3.5
\end{hint}
\end{problem}


\begin{problem}(problem 3)
Determine whether the infinite series converges or diverges: 
$\displaystyle \sum_{n=1}^\infty \frac{\sin^2(n)}{n^3}$

\begin{hint}
Use the Direct Comparison Test
\end{hint}

\begin{hint}
$\displaystyle 0 \leq \sin^2(n) \leq 1$ for all $n$
\end{hint}

\begin{hint}
Compare to $p$-series with $p = 3 > 1 \implies$ convergence
\end{hint}

\end{problem}

\begin{problem}(problem 4)
Determine whether the infinite series converges or diverges: 
$\displaystyle \sum_{n=1}^\infty \frac{n^2 + 5n + 12}{n^3 + 6n^2 + 2n}$
\begin{hint}
Use the Limit Comparison Test 
\end{hint}

\begin{hint}
Compare with the divergent Harmonic Series 
\end{hint}

\begin{hint}
$\displaystyle \lim_{n \to \infty} \frac{a_n}{b_n} = 1$ and $ 0 < 1 < \infty$
\end{hint}

\end{problem}

\begin{problem}(problem 5)
Determine whether the infinite series converges absolutely, converges conditionally or diverges: $\displaystyle \sum_{n=0}^\infty (-1)^n \frac{n^2 + 2}{n^3 + 3}$

\begin{hint}
Use the Alternating Series Test to see that the series converges
\end{hint}

\begin{hint}
Use the Limit Comparison Test (with the Harmonic Series) to see that the 
non-alternating counterpart diverges
\end{hint}

\begin{hint}
The alternating series converges conditionally
\end{hint}

\end{problem}

\begin{problem}(problem 6)
Determine whether the infinite series converges or diverges: 
$\displaystyle \sum_{n=1}^\infty (-1)^n\frac{n^2}{3^n}$

\begin{hint}
Use the Ratio Test
\end{hint}

\begin{hint}
$L = 1/3 < 1$ so the series converges absolutely (this is an alternating series)
\end{hint}

\begin{hint}
$\frac{3^n}{3^{n+1}} = \frac13$
\end{hint}

\end{problem}


\begin{problem}(problem 7)
Determine whether the infinite series converges or diverges: 
$\displaystyle \sum_{n=1}^\infty \left(\frac{n+ 2}{3n+1}\right)^n$

\begin{hint}
Use the Root Test
\end{hint}

\begin{hint}
Ratio of lead coeff: $L = 1/3 < 1$ so the series converges
\end{hint}

\end{problem}

\begin{problem} (problem 8)
Find the interval of convergence of the power series (be sure to check the endpoints):
$\displaystyle \sum_{n=1}^\infty \frac{n(x-2)^n}{5^n}$

\begin{hint}
Use the Ratio Test
\end{hint}
\begin{hint}
The endpoints are $-3$ and $7$
\end{hint}
\begin{hint}
The series diverges at both endpoints by the test for divergence
\end{hint}

\end{problem}


\begin{problem} (problem 9)
Find a power series representation for the function and include the interval of convergence in your answer:
$\displaystyle f(x) = \frac{x^2}{3 + x}$

\begin{hint}
Factor $x^2$ from the numerator and $3$ from the denominator
\end{hint}
\begin{hint}
Use the formula: $\displaystyle \frac{1}{1-x} = \sum_{n = 0}^\infty x^n, \; |x| <1$\\
with $-x/3$ in place of $x$
\end{hint}
\begin{hint}
Answer: 
$\displaystyle \frac{x^2}{3 + x} = \sum_{n = 0}^\infty (-1)^n \frac{x^{n+2}}{3^{n+1}}, \; |x| <3$
\end{hint}
\end{problem}


\section{Sample Test 3B}



%\begin{problem}(problem 1)
%Find the sum of the infinite series, if it converges: $\displaystyle \sum_{n=0}^\infty \frac{3}{2^n}$

%\begin{hint}
%This is a geometric series with $r = 1/2$
%\end{hint}

%\begin{hint}
%The series converges since $|r| = 1/2 < 1$
%\end{hint}

%\begin{hint}
%The sum of the series is 6
%\end{hint}

%\end{problem}


%\begin{problem}(problem 2)
%Find the sum of the infinite series, if it converges: $\displaystyle \sum_{n=1}^\infty \frac{1}{n^2 + 2n}$
%\begin{hint}
%Make a partial fraction decomposition
%\end{hint}

%\begin{hint}
%Compute the $N^{th}$ partial sum, $S_N$
%\end{hint}

%\begin{hint}
%Take the limit as $N \to \infty$ to find the sum of the series
%\end{hint}

%\end{problem}


\begin{problem}(problem 1)
Determine whether the infinite series converges or diverges: 
$\displaystyle \sum_{n=1}^\infty \sqrt[n] 2$

\begin{hint}
Use the Test for Divergence
\end{hint}

\begin{hint}
$\displaystyle \lim_{n \to \infty} a_n = 2^0 = 1 \neq 0$
\end{hint}

\end{problem}

\begin{problem}(problem 2)
Use the Integral Test to determine whether the infinite series converges or diverges (if it applies):
$\displaystyle \sum_{n=1}^\infty \frac{1}{1 + n^2}$.
\begin{hint}
Show the derivative is negative for $x$ sufficiently large
\end{hint}
\begin{hint}
The anti-derivative is the inverse tangent function
\end{hint}
\begin{hint}
Write the improper integral using a limit
\end{hint}
\begin{hint}
The integral converges
\end{hint}
\begin{hint}
See example 1 in section 3.5
\end{hint}

\end{problem}


\begin{problem}(problem 3)
Determine whether the infinite series converges or diverges: 
$\displaystyle \sum_{n=1}^\infty \frac{1}{\sqrt{n^2 + 1}}$

\begin{hint}
Use the Limit Comparison Test with the Harmonic Series
\end{hint}


\end{problem}

\begin{problem}(problem 4)
Determine whether the infinite series converges or diverges: $\displaystyle \sum_{n=1}^\infty \frac{\cos^4(n)}{n\sqrt{n}}$

\begin{hint}
Use the Direct Comparison Test
\end{hint}

\begin{hint}
$\displaystyle 0 \leq \cos^4(n) \leq 1$ for all $n$
\end{hint}

\begin{hint}
Since $n\sqrt n = n^{3/2}$, compare to $p$-series with $p = 3/2 > 1 \implies$ convergence
\end{hint}

\end{problem}

\begin{problem}(problem 5)
Determine whether the infinite series converges absolutely, converges conditionally or diverges: 
$\displaystyle \sum_{n=1}^\infty (-1)^{n+1} \frac{n^3}{n^4 + 2}$

\begin{hint}
Use the Alternating Series Test to see that the series converges
\end{hint}
\begin{hint}
Use the derivative to show that the terms are decreasing
\end{hint}
\begin{hint}
Use the Limit Comparison Test (with the Harmonic Series) to see that the 
non-alternating counterpart diverges
\end{hint}


\begin{hint}
The alternating series converges conditionally
\end{hint}

\end{problem}

\begin{problem}(problem 6)
Determine whether the infinite series converges or diverges: 
$\displaystyle \sum_{n=1}^\infty \frac{n^5 10^{n+1}}{(2n)!}$

\begin{hint}
Use the Ratio Test
\end{hint}

\begin{hint}
$L = 0 < 1$ so the series converges
\end{hint}

\end{problem}


\begin{problem}(problem 7)
Determine whether the infinite series converges or diverges: 
$\displaystyle \sum_{n=0}^\infty (-1)^n\left(\frac{n^2 + 2n+ 4}{4n^2 + 5}\right)^n$

\begin{hint}
Use the Root Test
\end{hint}

\begin{hint}
Ratio of lead coeff: $L = 1/4 < 1$ so the series converges absolutely (this is an alternating series)
\end{hint}

\end{problem}

\begin{problem} (problem 8)
Find the interval of convergence of the power series (be sure to check the endpoints):
$\displaystyle \sum_{n=1}^\infty \frac{(x + 1)^n}{n 3^n}$

\begin{hint}
Use the Ratio Test
\end{hint}
\begin{hint}
The endpoints are $-4$ and $2$
\end{hint}
\begin{hint}
The series converges at $x = -4$ (alternating Harmonic) and diverges at $x = 2$ (Harmonic)
\end{hint}

\end{problem}


\begin{problem} (problem 9)
Find a power series representation for the function and include the interval of convergence in your answer:
$\displaystyle f(x) = \frac{4x}{1 - 4x^2}$

\begin{hint}
Factor $4x$ from the numerator
\end{hint}
\begin{hint}
Use the formula: $\displaystyle \frac{1}{1-x} = \sum_{n = 0}^\infty x^n, \; |x| <1$\\
with $-4x^2$ in place of $x$
\end{hint}

\begin{hint}
Answer: 
$\displaystyle \frac{4x}{1-4x^2} = \sum_{n = 0}^\infty 4^{n+1}x^{2n+1} , \; |x| <\frac12$
\end{hint}

\end{problem}

\end{document}


\begin{problem}(problem 1)

Determine if the sequence $\{a_n\}$ converges.  If it does, find the limit.

\begin{enumerate}

\item $a_n = \frac{5n^2 + 2n + 20}{3n^5 - 3n^2 - 5}$
The sequence 
\begin{multipleChoice}
\choice[correct]{converges}\\
\choice{diverges}
\end{multipleChoice}

If the sequence converges, its limit is $\answer{0}$

\item $a_n = 5^\frac{2n}{5n^2+1}$

The sequence 
\begin{multipleChoice}
\choice[correct]{converges}\\
\choice{diverges}
\end{multipleChoice}

If the sequence converges, its limit is $\answer{0}$

\item $a_n = \left(\frac{3}{2}\right)^n$

The sequence 
\begin{multipleChoice}
\choice{converges}\\
\choice[correct]{diverges}
\end{multipleChoice}

If the sequence converges, its limit is $\answer{}$

\end{enumerate}

\end{problem}

\begin{problem}(problem 1)

Determine if the sequence $\{a_n\}$ converges.  If it does, find the limit.

\begin{enumerate}


\item $a_n = \frac{2n^3 + 1}{1500n^2 + 100n + 3000}$
The sequence 
\begin{multipleChoice}
\choice{converges}\\
\choice[correct]{diverges}
\end{multipleChoice}

If the sequence converges, its limit is $\answer{}$

\item $a_n = \left(\frac{1}{4}\right)^n$

The sequence 
\begin{multipleChoice}
\choice[correct]{converges}\\
\choice{diverges}
\end{multipleChoice}

If the sequence converges, its limit is $\answer{0}$




\item $a_n = (-1)^n\frac{ n - 1}{n}$

The sequence 
\begin{multipleChoice}
\choice{converges}\\
\choice[correct]{diverges}
\end{multipleChoice}

If the sequence converges, its limit is $\answer{}$

\end{enumerate}

\end{problem}


\begin{problem}(problem 4)
Use the Integral Test to determine whether the infinite series converges or diverges:  $\displaystyle \sum_{n=2}^\infty \frac{\ln(n)}{n}$

\end{problem}



\begin{problem}(problem 4)
Use the Integral Test to determine whether the infinite series converges or diverges:  $\displaystyle \sum_{n=0}^\infty \frac{1}{1+n^2}$

\end{problem}





