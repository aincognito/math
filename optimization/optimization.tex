\documentclass[handout]{ximera}

%% You can put user macros here
%% However, you cannot make new environments



\newcommand{\ffrac}[2]{\frac{\text{\footnotesize $#1$}}{\text{\footnotesize $#2$}}}
\newcommand{\vasymptote}[2][]{
    \draw [densely dashed,#1] ({rel axis cs:0,0} -| {axis cs:#2,0}) -- ({rel axis cs:0,1} -| {axis cs:#2,0});
}


%\usepackage{tcolorbox} %%Needed for Derivative Definition supposedly and product rule, natural exp log, quotient rule, inverse trig, rates of change


% \graphicspath{{./}{firstExample/}}
% \usepackage{forest}
\usepackage{amsmath}
\usepackage{amssymb}
\usepackage{array}
\usepackage[makeroom]{cancel} %% for strike outs
\usepackage{pgffor} %% required for integral for loops
\usepackage{tikz}
\usepackage{tikz-cd}
\usepackage{tkz-euclide}
\usetikzlibrary{shapes.multipart}


% \usetkzobj{all}
\tikzstyle geometryDiagrams=[ultra thick,color=blue!50!black]


\usetikzlibrary{arrows}
\tikzset{>=stealth,commutative diagrams/.cd,
  arrow style=tikz,diagrams={>=stealth}} %% cool arrow head
\tikzset{shorten <>/.style={ shorten >=#1, shorten <=#1 } } %% allows shorter vectors

\usetikzlibrary{backgrounds} %% for boxes around graphs
\usetikzlibrary{shapes,positioning}  %% Clouds and stars
\usetikzlibrary{matrix} %% for matrix
\usepgfplotslibrary{polar} %% for polar plots
\usepgfplotslibrary{fillbetween} %% to shade area between curves in TikZ



%\usepackage[width=4.375in, height=7.0in, top=1.0in, papersize={5.5in,8.5in}]{geometry}
%\usepackage[pdftex]{graphicx}
%\usepackage{tipa}
%\usepackage{txfonts}
%\usepackage{textcomp}
%\usepackage{amsthm}
%\usepackage{xy}
%\usepackage{fancyhdr}
%\usepackage{xcolor}
%\usepackage{mathtools} %% for pretty underbrace % Breaks Ximera
%\usepackage{multicol}



\newcommand{\RR}{\mathbb R}
\newcommand{\R}{\mathbb R}
\newcommand{\C}{\mathbb C}
\newcommand{\N}{\mathbb N}
\newcommand{\Z}{\mathbb Z}
\newcommand{\dis}{\displaystyle}
%\renewcommand{\d}{\,d\!}
\renewcommand{\d}{\mathop{}\!d}
\newcommand{\dd}[2][]{\frac{\d #1}{\d #2}}
\newcommand{\pp}[2][]{\frac{\partial #1}{\partial #2}}
\renewcommand{\l}{\ell}
\newcommand{\ddx}{\frac{d}{\d x}}
\newcommand{\ppx}{\frac{\partial}{\partial x}}
\newcommand{\ppy}{\frac{\partial}{\partial y}}

\newcommand{\zeroOverZero}{\ensuremath{\boldsymbol{\tfrac{0}{0}}}}
\newcommand{\inftyOverInfty}{\ensuremath{\boldsymbol{\tfrac{\infty}{\infty}}}}
\newcommand{\zeroOverInfty}{\ensuremath{\boldsymbol{\tfrac{0}{\infty}}}}
\newcommand{\zeroTimesInfty}{\ensuremath{\small\boldsymbol{0\cdot \infty}}}
\newcommand{\inftyMinusInfty}{\ensuremath{\small\boldsymbol{\infty - \infty}}}
\newcommand{\oneToInfty}{\ensuremath{\boldsymbol{1^\infty}}}
\newcommand{\zeroToZero}{\ensuremath{\boldsymbol{0^0}}}
\newcommand{\inftyToZero}{\ensuremath{\boldsymbol{\infty^0}}}


\newcommand{\numOverZero}{\ensuremath{\boldsymbol{\tfrac{\#}{0}}}}
\newcommand{\dfn}{\textbf}
%\newcommand{\unit}{\,\mathrm}
\newcommand{\unit}{\mathop{}\!\mathrm}
%\newcommand{\eval}[1]{\bigg[ #1 \bigg]}
\newcommand{\eval}[1]{ #1 \bigg|}
\newcommand{\seq}[1]{\left( #1 \right)}
\renewcommand{\epsilon}{\varepsilon}
\renewcommand{\iff}{\Leftrightarrow}

\DeclareMathOperator{\arccot}{arccot}
\DeclareMathOperator{\arcsec}{arcsec}
\DeclareMathOperator{\arccsc}{arccsc}
\DeclareMathOperator{\si}{Si}
\DeclareMathOperator{\proj}{proj}
\DeclareMathOperator{\scal}{scal}
\DeclareMathOperator{\cis}{cis}
\DeclareMathOperator{\Arg}{Arg}
%\DeclareMathOperator{\arg}{arg}
\DeclareMathOperator{\Rep}{Re}
\DeclareMathOperator{\Imp}{Im}
\DeclareMathOperator{\sech}{sech}
\DeclareMathOperator{\csch}{csch}
\DeclareMathOperator{\Log}{Log}

\newcommand{\tightoverset}[2]{% for arrow vec
  \mathop{#2}\limits^{\vbox to -.5ex{\kern-0.75ex\hbox{$#1$}\vss}}}
\newcommand{\arrowvec}{\overrightarrow}
\renewcommand{\vec}{\mathbf}
\newcommand{\veci}{{\boldsymbol{\hat{\imath}}}}
\newcommand{\vecj}{{\boldsymbol{\hat{\jmath}}}}
\newcommand{\veck}{{\boldsymbol{\hat{k}}}}
\newcommand{\vecl}{\boldsymbol{\l}}
\newcommand{\utan}{\vec{\hat{t}}}
\newcommand{\unormal}{\vec{\hat{n}}}
\newcommand{\ubinormal}{\vec{\hat{b}}}

\newcommand{\dotp}{\bullet}
\newcommand{\cross}{\boldsymbol\times}
\newcommand{\grad}{\boldsymbol\nabla}
\newcommand{\divergence}{\grad\dotp}
\newcommand{\curl}{\grad\cross}
%% Simple horiz vectors
\renewcommand{\vector}[1]{\left\langle #1\right\rangle}


\outcome{Optimize a function which models a real world situation}

\title{3.7 Optimization}

\begin{document}

\begin{abstract}
We find extremes of functions which model real world situations.
\end{abstract}

\maketitle

\begin{center}
\textbf{Optimization}
\end{center}
 

\begin{example}[example 1]
Farmer Bob has 2400 linear feet of fence. He wishes to build a rectangular enclosure with a partition fence parallel to one of the sides.  
What are the dimensions, $(x,y)$, that maximize the total area of the enclosure? What is the maximum area?

% Farmer Bob can enclose if he has 2400 linear feet of fence at his disposal?

%dimensions will maximize the total area covered by the enclosure and what is the maximum area?
\begin{image}
\begin{tikzpicture}
\node at (5, 6.5) {\fontsize{16 pt}{18 pt}\selectfont Area = length $\times$ width};
\draw[brown, thick] (0,0) --(10, 0) -- (10, 6) -- (0, 6) -- (0, 0);
\draw[<->] (10.5, 0) --(10.5,6) node[right, midway]{\fontsize{16 pt}{18 pt}\selectfont y};
\draw[brown, thick] (6, 0) -- (6, 6);
\draw[<->] (0, -.5) --(10,-.5) node[below, midway]{\fontsize{16 pt}{18 pt}\selectfont x};
\end{tikzpicture}
\end{image}
Let $x$ be the length of the enclosure and $y$ the width.  The total area of the enclosure is then
\[A= xy.\]
This is called the {\it objective} equation.
To maximize the area, Farmer Bob should use all 2400 linear feet of fence.  Since there are two horizontal segments of length $x$ and three vertical segments of length $y$ (because of the partition), we get the {\it constraint} equation:
\[2x+3y=2400.\]
In the standard language of optimization, we can now state the problem as follows.

Maximize the objective
\[A = xy,\]
subject to the constraint 
\[2x+3y = 2400.\]
To find the maximum area, we need to eliminate one of the variables $x, y$ from the objective function.
We use the constraint to do this.  Solve the constraint for either $x$ or $y$.
In this case, solving the constraint for $y$ gives
\[y = 800 - \frac23 x. \;\; \text{(verify)}\]
We {\it substitute} this expression of $y$ into the objective function to get 
\[A(x) = x\left(800 - \frac23 x\right) = 800x - \frac23 x^2.\]
Since $x$ represents a length of fence, we have $x \geq 0$. Furthermore, since we have 2400 feet of fence $2x \leq 2400$ which implies $x\leq 1200$.
Thus we would like find the absolute maximum of $A(x)$ on the interval $[0, 1200]$. Since $A(x)$ is a continuous function, the EVT says that it has an abs max on the closed interval $[0, 1200]$.  We use the closed interval method of section 3.1 to find the abs max.

The critical numbers are:
\[A'(x) = 800 - \frac43 x = 0 \]
which yields
\[x= 600 \,\mbox{ft.}\]
which is in the interval. Comparing values, we have $A(0) = 0, A(600) = 240000$ and $A(1200) = 0$.
Thus the abs max occurs when $x = 600$. 
%We now use the second derivative test to verify that this is a maximum:
%\[
%A'' = -\frac43.
%\]
%Since, $A'' < 0$ at the critical number $x = 600$, $A$ has a local maximum there.

We can also note that $x = 600$ gives a maximum since the graph of $A(x)$ is a parabola that opens down.


\begin{image}
\begin{tikzpicture}
\begin{axis}[axis lines = center,  axis y line=center,  xmin=-0.5, xmax = 4.5, xtick={2}, ytick={0}, yticklabels={,,},
 xticklabel={$600$ ft.}, xlabel=$x$, ylabel=$A$, ymin = -0.5, ymax = 4.5, title={Graph of objective function}]
\addplot[smooth,blue, domain=0:4, thick] {4*x - x^2};
\addplot[blue, mark = *] coordinates{(2,4)} node[above,blue] {maximum area};
\node[blue] at (axis cs:2, 1.5){$A = 800x - \frac23 x^2$};
%\node[blue] at (axis cs:-8, 250){\begin{tabular}{c} concave\\down \end{tabular}};
%\node[blue] at (axis cs:0.2, 50){\begin{tabular}{c} concave\\up \end{tabular}};
\end{axis}
\end{tikzpicture}
\end{image}


We next find $y$ by plugging $x = 600$ into the constraint:
\[y = 800 - \frac23 x = 800 - \frac23(600) = 400 \,\mbox{ft.}\]
Finally the maximum area of the enclosure is
\[A = xy = (600)(400) = 240{,}000 \,\mbox{ft$^2$}.\]
\end{example}


\begin{center}
\begin{foldable}
\unfoldable{Here is a video of the example above}
\youtube{jNTcXGVWbbY} 
\end{foldable}
\end{center}




\begin{problem}(problem 1a)
Farmer Bob has 1000 linear feet of fence with which to build a rectangular enclosure.  
What dimensions will maximize the total area covered by the enclosure and what is the maximum area?

\begin{hint}
Let $x$ be the length and $y$ the width
\end{hint}
\begin{hint}
The material constraint is $2x+2y = 1000$
\end{hint}
\begin{hint}
Write the area as a function of $x$
\end{hint}
\begin{hint}
Set the derivative equal to zero
\end{hint}

The optimal length is $x= \answer{250}$ ft.\\
The optimal width is $y = \answer{250}$ ft.\\
The maximum area is $\answer{62500}$ sq. ft.


\end{problem}


\begin{problem}(problem 1b)
Farmer Bob has 3200 linear feet of fence with which to build a rectangular enclosure with two partitions, as shown below.
What dimensions will maximize the total area covered by the enclosure and what is the maximum area?

\begin{center}
\begin{tikzpicture}
\draw (0,0) --(5, 0) -- (5, 3) -- (0, 3) -- (0, 0);
\draw (1.5, 0) -- (1.5, 3);
\draw (3, 0) -- (3, 3);
\end{tikzpicture}
\end{center}
\begin{hint}
Let $x$ be the length and $y$ the width
\end{hint}
\begin{hint}
The material constraint is $2x+4y = 3200$
\end{hint}
\begin{hint}
Write the area as a function of $x$
\end{hint}
\begin{hint}
Set the derivative equal to zero
\end{hint}

The optimal length is $x= \answer{800}$ ft.\\
The optimal width is $y = \answer{400}$ ft.\\
The maximum area is $\answer{320000}$ sq. ft.
 
\end{problem}

\begin{problem}(problem 1c)
Farmer Bob has 400 linear feet of fence with which to build a 
rectangular enclosure along the bank of a straight river, as shown below.  
If no fence is required along the river bank, what dimensions will maximize the total area 
covered by the enclosure and what is the maximum area?
\begin{center}
\text{river}\\
\begin{tikzpicture}
\draw (0,0) --(5, 0) -- (5, 3) -- (0, 3) -- (0, 0);
%\node[above]{river};
\draw (-1, 3) -- (6, 3);
\end{tikzpicture}
\end{center}
\begin{hint}
Let $x$ be the length and $y$ the width
\end{hint}
\begin{hint}
The material constraint is $x+2y = 400$
\end{hint}
\begin{hint}
Write the area as a function of $x$
\end{hint}
\begin{hint}
Set the derivative equal to zero
\end{hint}

The optimal length is $x= \answer{200}$ ft.\\
The optimal width is $y = \answer{100}$ ft.\\
The maximum area is $\answer{20000}$ sq. ft.


\end{problem}


\begin{problem}(problem 1d)
The sum of two numbers is 100.  Find the maximum value of their product.\\
Let the two numbers be $x$ and $y$ and let their product be $P$.\\
The objective equation is $\answer{P=xy}$.\\
The constraint equation is $\answer{x + y = 100}$.\\
The maximum value of the product is $P= \answer{2500}$.
\end{problem}



\begin{example}[example 2]
A box with a square base and an open top is to be constructed using 4800 square inches of cardboard.  
Find the dimensions of the box that will maximize its volume.  What is the maximum volume?\\

If we let $x$ represent the length and width of the box, and $y$ its height, then our objective is to maximize 
the volume,


\begin{image}
\begin{tikzpicture}
\draw (0,0) --node[below] {$x$} (4, 0) -- (4, 3) -- (0, 3)--(0,0);
\draw (2,1) --(6, 1) -- node[right] {$y$}(6, 4) -- (2, 4) -- (2, 1);
\draw (0, 0)--(2,1);
\draw (4, 0)--node[right] {\; $x$}(6,1);
\draw (4,3 )--(6,4);
\draw (0,3)--(2,4);
\end{tikzpicture}
\end{image}

\[V = \text{length $\times$ width $\times$ height} = x^2y.\]

We have a material constraint which says that the surface area of the box should be 4800 sq. in:
\[\text{area of base + area of 4 sides} = x^2 + 4xy = 4800.\]
Solving the constraint for $y$ gives
\[y = \frac{4800 - x^2}{4x}.\]
Substituting this into the objective gives:
\[V(x) = x^2 \left(\frac{4800 - x^2}{4x}\right),\]
where $0\leq x \leq \sqrt{4800}$.
Simplifying $V(x)$ gives:
\[V(x) = 1200x - \frac14 x^3.\]
The volume will be at a maximum when its derivative is zero:
\[V'(x) = 1200 - \frac34 x^2 = 0.\]
Solving for $x$, we get:
\[ \frac34 x^2 = 1200, \]
\[x^2 = 1600,\]
\[x = \pm 40.\]
Since $x=-40$ is not in the interval $[0, \sqrt{4800}]$, we eliminate it from consideration. 
Furthermore, since $V(0) = 0$ and $V(\sqrt{4800})= 0$ (since $y=0$ in this case), the maximum volume occurs when $x = 40$.
Plugging this into the constraint equation, we can find $y$:
\[y = \frac{4800 - x^2}{4x} = \frac{4800 - 1600}{160} = 20 \mbox{ in.}\]
Thus the volume is maximized when $x = 40$ and $y = 20$ and the maximum volume of the box is:
\[V = x^2y = (40)^2 (20) = 32{,}000 \mbox{ cubic inches}.\]

\end{example}


\begin{center}
\begin{foldable}
\unfoldable{Here is a video of the example above}
\youtube{eFutZtccyck} 
\end{foldable}
\end{center}




\begin{problem}(problem 2)
A box with a square base and an open top are to be constructed using 7500 sq. in. of cardboard.  
Find the dimensions of the box that will maximize its volume.  What is the maximum volume?
\begin{hint}
Let $x$ be the length and width of the square base
\end{hint}
\begin{hint}
Let $y$ be the height of the box
\end{hint}
\begin{hint}
The material constraint is $x^2 + 4xy = 7500$
\end{hint}
\begin{hint}
The volume is $x^2 y$; replace $y$ using the constraint
\end{hint}
\begin{hint}
Set the derivative equal to zero
\end{hint}

The optimal length and width are $x = \answer{50}$ inches.\\
The optimal height is $y = \answer{25}$ inches.\\
The maximum volume is $ \answer{62500}$ cubic inches.
\end{problem}

In some problems, the objective function involves a composition, $f(g(x))$.  
If the outsode function $f$ is strictly increasing, then the max or min of the composition occurs at the same $x$-value as the max/min of $g(x)$. 
We will see this in the next example.
\begin{example}[example 3]
Scientist Sam wants to know how close a comet moving in a parabolic trajectory will get to the sun. 
We will assume that the sun is located at the origin and that the path of the comet follows the parabola $y = x^2 - 5$. The units 
are in millions of miles.

\begin{center}
\begin{tikzpicture}
\draw[blue, thick] plot[smooth, domain=-2:2] (\x, {(\x)^2 - 5});
\filldraw[orange] (0,-3) circle (3pt) node[below]{sun};
\filldraw[black] (-1.5, -2.75) circle (1pt) node[left]{comet};
\draw[dashed, gray, thin] (-1.5, -2.75)-- (0,-3) node[below, midway]{d};
\end{tikzpicture}
\end{center}
To solve this problem, we need the formula for the distance between two points in a plane.
If the points are $(x_1, y_1)$ and $(x_2, y_2)$, then the Pythagorean Theorem gives the distance between them:
\[
d = \sqrt{(x_2 - x_1)^2 + (y_2 - y_1)^2}.
\]
\begin{image}
\begin{tikzpicture}
\draw (0,0) --node[below] {$x_2-x_1$} (4, 0) -- node[right]{$y_2-y_1$} (4, 3) --  (0,0);
\node at (-.3, -.3) {$(x_1, y_1)$};
\node at (4.3, 3.3) {$(x_2, y_2)$};
\node at (2, 1.8){$d$};
\filldraw[black] (0, 0) circle (1pt);
\filldraw[black] (4, 3) circle (1pt);
\draw[thin] (3.7, 0) -- (3.7, 0.3) -- (4, 0.3);
\end{tikzpicture}
\end{image}


We apply the distance formula to the comet at the point $(x, y)$ and the sun at the origin to get the objective equation:
\[d = \sqrt{(x-0)^2 + (y-0)^2} = \sqrt{x^2 + y^2}.\]
Since the comet must stay on the parabola we get the constraint equation $y = x^2 - 5$.  Substituting this into the objective equation gives:
\[d = \sqrt{x^2 + (x^2 - 5)^2}.\]
Now, we would like to find the minimum value of $d$. Since the square root function is increasing, 
we can locate the minimum by finding the min of the radicand (the quantity under the square root).
So, we want to minimize 
\[
f(x) = x^2 + (x^2 - 5)^2.
\]
The minimum occurs when the derivative is zero, so we solve
$f'(x) = 0$ for $x$ yielding
\[f'(x) = 2x + 2(x^2 - 5)(2x) = 2x[1+ 2(x^2 - 5)] = 2x(2x^2 - 9) = 0\]
which gives either 
\[2x = 0 \;\; \text{or} \;\; 2x^2 - 9 = 0.\]
The three solutions are $x = 0, \pm \dfrac{3}{\sqrt2}$.
The corresponding $y$-values can be found by plugging these $x$-values into the parabola $y = x^2 - 5$
giving the three points
\[(0, -5), \left(-\dfrac{3}{\sqrt 2}, -\dfrac12 \right) \mbox{ and } \left(\dfrac{3}{\sqrt 2}, -\dfrac12 \right).\]
If the comet is at $(0, -5)$ then its distance to the sun is 5 million miles.  If the comet is at 
either $\left(\pm\dfrac{3}{\sqrt 2}, -\dfrac12\right)$ then the distance is 
\[d = \sqrt{x^2 + y^2} = \sqrt{{\frac{9}{2} + \frac14}} = \frac{\sqrt {19}}{2} \mbox{ million miles}.\]
Hence the minimum distance is $\dfrac{\sqrt{19}}{2}$ million miles when the comet is at either of the points 
$\left(\pm\dfrac{3}{\sqrt 2}, -\dfrac12 \right)$.
\end{example}

\begin{problem}(problem 3)
Scientist Sam wants to know how close a comet moving in a parabolic trajectory will get to the sun. 
We will assume that the sun is located at the origin, 
the path of the comet follows the parabola $y = x^2 - 1$ and that the units 
on the axes are in millions of miles.



\begin{hint}
The distance formula is $d = \sqrt{(x_2 - x_1)^2 + (y_2 - y_1)^2}$
\end{hint}
\begin{hint}
The sun is at $(0,0)$
\end{hint}
\begin{hint}
The comet is on the parabola, so it is at $(x, x^2 - 1)$
\end{hint}
\begin{hint}
To minimize a square root function, minimize the radicand
\end{hint}
\begin{hint}
Set the derivative equal to zero
\end{hint}

The comet is closest to the sun at two points.
 The x-coordinates of these points are (in ascending order)\\
$x =  \answer{-1/\sqrt2}$ and $x = \answer{1/\sqrt2}$.\\
The minimum distance is $d = \answer{\sqrt{3/4}}$ million miles.
\end{problem}



\begin{example}[example 4]
Gardner Harold wants to construct a 1600 square foot  rectangular enclosure that has both a 
horizontal and a vertical partition. What dimensions will require the minimum amount of fencing?  
How much fence will he need?

\begin{center}
\begin{tikzpicture}
\draw (0,0) --(5, 0) -- (5, 3) -- (0, 3) -- (0, 0);
\draw (3, 0) -- (3, 3);
\draw (0, 1.2) -- (5, 1.2);
\end{tikzpicture}
\end{center}

Let $x$ be the length of the enclosure and $y$ the width. Because of the partitions, there are 3 horizontal and 3 vertical sections of fence. Hence the total amount of fence needed can be expressed as 
\[F = 3x + 3y.\]
 This is our objective function. Since the area of the enclosure needs to be 1600 sq. ft., we have the constraint 
\[xy = 1600.\]
Solving the constraint for $y$ gives 
\[
y = 1600/x.
\]
  Substituting this into the objective yields
\[F(x) = 3x + 3\left(\frac{1600}{x}\right) = 3x + \frac{4800}{x}.\]
To find the minimum value of $F$ , we solve $F'(x) = 0$. This gives
\[F'(x) = 3 - \frac{4800}{x^2} = 0\]
which means $3x^2 = 4800$ and so $x = \pm 40$ ft.  Since $x>0$ from context, we have $x = 40$ ft. as our solution.
Plugging this value of $x$ into the constraint equation gives
\[y = \frac{1600}{x} = \frac{1600}{40} = 40 \mbox{ ft.}\]
So the solution is for the gardener to build a square, 40 feet on a side, using a total of $F = 3x+ 3y = 3(40) + 3(40)
= 240$ feet of fence.
\end{example}

\begin{problem}(problem 4)
Gardner Harold wants to construct a 600 square foot  rectangular enclosure that has a 
vertical partition. What dimensions will require the minimum amount of fencing?  
How much fence will he need?

\begin{hint}
Let $x$ be the length and $y$ the width
\end{hint}
\begin{hint}
The size constraint is $xy = 600$
\end{hint}
\begin{hint}
Write the amount of fence used as a function of $x$
\end{hint}
\begin{hint}
Set the derivative equal to zero
\end{hint}

The optimal length is $x= \answer{30}$ ft.\\
The optimal width is $y = \answer{20}$ ft.\\
The minimum amount of fence needed is $\answer{120}$ ft.
\end{problem}


\begin{example}[example 5]
A cheetah is on the bank of a 50 meter wide, straight river.   The cheetah spots prey on the opposite bank of the river, 100 meters upstream. 
The cheetah decides to walk along the rivers edge for part of the way before swimming directly at the prey. If the cheetah walks at 2 m/s and swims at 1 m/s,
how far should the cheetah walk before swimming so as to minimize the total time to get to the prey?\\

Let $x$ meters be the distance that the cheetah walks and $y$ meters the distance that the cheetah swims, as shown in the figure below. 


\begin{image}
\begin{tikzpicture}
%\node at (5, 6.5) {\fontsize{16 pt}{18 pt}\selectfont Area = length $\times$ width};
\draw[blue, thick] (0,0) --(0, 4);
\draw[blue, thick] (2,0) --(2, 4);
\draw[thin] (0, 2.5) --(2,2.5) node[below, midway]{$50$};
\draw[thin, dashed] (0, 2.5) --(2,4);
\node at (2.7, 3.25) {$100-x$};
\node at (-0.3, 1.25) {$x$};
\filldraw[brown] (0, 0) circle (1pt) node[left]{Cheetah};
\filldraw[brown] (0, 2.5) circle (1pt) node[left]{start swim};
\filldraw[brown] (2, 4) circle (1pt) node[right]{Prey};
\node at (1, 0.5) {river};
\draw[thin] (1.7, 2.5) -- (1.7, 2.8) -- (2, 2.8);
\node at (0.9, 3.4) {$y$};
\end{tikzpicture}
\end{image}

The total time for the trip is the time walking plus the time swimming. Since 
\[
\text{time} = \frac{\text{distance}}{\text{rate}},
\]
the total time is 
\[
T = \frac{x}{2} + \frac{y}{1}.
\]
This is the objective equation.
The constraint comes from the Pythagorean Theorem, since the swim portion of the trip is along the hypotenuse of a right triangle 
with legs parallel and perpendicular to the river bank. One leg of the triangle is $50$ and the other is $100-x$.  So 
\[
y = \sqrt{50^2 + (100-x)^2}.
\]
Substituting this into the objective gives:
\[
T(x) = \frac{x}{2} + \sqrt{50^2 + (100-x)^2},\]
where $x$ is between $0$ and $100$.
To find the minimum value of $T$, we find the critical numbers (using the chain rule):
\[
T' = \frac12 +  \frac{2(100-x)(-1)}{2\sqrt{50^2 + (100-x)^2}} = \frac12 - \frac{100-x}{\sqrt{50^2 + (100-x)^2}} = 0
\]

Solving for $x$ gives
\[
\frac{100 - x}{\sqrt{50^2 + (100-x)^2}} = \frac12
\]
\[
2(100 - x) = \sqrt{50^2 + (100-x)^2}.
\]
Square both sides:
\[
4(100-x)^2 = 50^2 + (100-x)^2
\]
\[
3(100-x)^2 = 50^2
\]
\[
100-x = \frac{50}{\sqrt{3}} = \frac{50\sqrt{3}}{3}.
\]
Hence
\[
x = 100 - \frac{50\sqrt{3}}{3} \approx 71 \text{meters}.
\]
To verify that this is indeed the minimum time, we can use the closed interval method from section 3.2 (Extreme Value Theorem).
This is because our independent variable $x$ must be between $0$ and $100$. Thus we compare $T(0), T(100)$ and $T(50/\sqrt 3)$.
We have 
\[
T(0) = \frac02 +  \sqrt{50^2 + 100^2} = \sqrt{12500} \approx 111,
\]
\[
T(100) = \frac{100}{2} +  \sqrt{50^2 + 0} = 50 + 50 = 100, \text{and}
\]
\[
T(50/\sqrt 3) = \frac12 \cdot \left(100- \frac{50}{\sqrt 3}\right) +  \sqrt{50^2 + (50/\sqrt 3)^2} \approx 93.3
\]
Hence the minimum time to reach the prey is approximately $T=93.3$ seconds and this is achieved by walking approximately $x =71$ meters, 
then swimming the rest of the way.
\end{example}


\begin{problem}(problem 5)
A woman launches her boat from a point on the bank of a straight river, 3 km wide. 
She wants to reach a point 8 km downstream on the other side of the river via a combination of rowing and running.
What is the minimum amount of time it will take her to reach her destination if she runs at 8km/hr and rows at 6km/hr?


\begin{hint}
Let $x$ be the distance running and $y$ the distance rowing.
\end{hint}
\begin{hint}
time = distance/rate
\end{hint}
\begin{hint}
total time = time rowing + time running
\end{hint}
\begin{hint}
Set the derivative equal to zero
\end{hint}

The minimum amount of time to reach her destination (to 2 decimal places) is $\answer{1.33}$ hours.
\end{problem}



\end{document}






